\documentclass[red]{textolivre}
\usepackage[utf8]{inputenc}
\usepackage[portuguese,english]{babel}

\usepackage{amsmath}
\usepackage{amsfonts}
\usepackage{amssymb}
\usepackage{amsthm}

\usepackage{csquotes}

\usepackage{graphicx} 
\usepackage{booktabs}

% used to create dummy text for the template file
\usepackage{lipsum} 

\usepackage{setspace} % for \onehalfspacing and \singlespacing macros
\onehalfspacing 

\def \tituloportugues {Modelo de artigo para a revista Texto Livre}
\def \tituloingles {Article template for Free Text Journal}
\title{%
    \vspace{-15mm}%
    \selectfont\textbf{\MakeUppercase{\tituloportugues}} \\
    \vspace{2.1mm}
    \selectfont\textbf{\textit{{\MakeUppercase{\tituloingles}}}}%
} 

\date{\today}

\tlauthor{Leonardo Carneiro de Araujo}{Universidade Federal de São João del Rei, Brasil}{leolca@ufsj.edu.br}
\tlauthor{Daniervelin Renata Marques Pereira}{Universidade Federal de Minas Gerais, Brasil}{revista@textolivre.org}


\begin{otherlanguage}{english}
\begin{abstract}
% write your abstract here
This article is a template article which aims to guide authors who will submit their papers to the Texto Livre journal. This template should facilitate their endeavor, viewing and configuring the text in the format already configured accordingly. Just write or paste your content in the desired places, replacing texts and figures, checking if they keep up the format. The abstract should have between 150 and 200 words.
\keywords{word1; word2; many-words; word3}
\end{abstract}
\end{otherlanguage}

\begin{resumo}
% escreva o resumo aqui
Este artigo é um modelo que visa orientar os autores que submeterão seus textos à revista Texto Livre a fazê-lo no padrão determinado pela revista a fim de facilitar sua empreitada, visualizando e configurando o texto no formato já configurado nesse modelo. Para fazê-lo, basta escrever ou colar seus textos no lugar aqui designado, substituindo os textos e figuras aqui existentes, verificando se eles mantêm a formatação aqui descrita. O resumo deverá ter entre 150 e 200 palavras.
\palavraschave{palavra1; palavra2; muitas-palavras; palavra3}
\end{resumo}

\usepackage[backend=biber,style=abnt, ittitles]{biblatex}
\DeclareLanguageMapping{brazil}{brazil-apa}
\addbibresource{tl-article-template.bib}     
% use biber instead of bibtex
% $ biber tl-article-template
% $ pdflatex tl-article-template.tex


\begin{document}
\selectlanguage{portuguese}
\maketitle

\section{Introdução}
Este artigo é o modelo de documento que visa orientar os autores que submeterão seus textos à Revista Texto Livre. Este documento fornece o padrão determinado pela revista, a fim de facilitar sua empreitada, visualizando e configurando o texto no formato já configurado no modelo para a submissão. Para redigir seu texto neste modelo, basta escrever ou colar seus textos no lugar aqui designado, substituindo os textos e figuras aqui existentes, verificando se eles mantêm a formatação aqui descrita.

O corpo do texto deverá ter um recuo de 1,25 cm, utilizar a fonte “Liberation Sans” em tamanho 12, justificado, com entrelinhado simples formatar. O espaço abaixo de cada paragrafo deverá ser de 0,1cm.

\section{Título do artigo}
O título do artigo deverá ser em caixa alta, negrito e utilizando a fonte “Liberation Sans” com tamanho 12. Abaixo do título em português, virá o título em inglês, com espaço de 0,4cm entre eles. O título em inglês utilizará a mesma formatação, porém deverá estar em itálico. Entre o título em inglês e o resumo do artigo, deverá haver um espaçamento de 1,10cm.

\section{Título de seções}
O título de cada seção deverá utilizar a fonte “Liberation Sans” com tamanho 12, em negrito, numerados com algarismos arábicos. Após o título de cada seção, será deixado um espaço em branco de 0,5cm, separando o título do texto da referida seção. O texto será justificado, com entrelinhamento simples e espaço de 0,10cm abaixo de cada parágrafo.

\subsection{Organização em seções, subseções e subsubseções}
O artigo poderá ser organizado de forma a conter seções, subseções e subsubseções. Em todos os casos, a formação dos títulos e parágrafos destas seguirão as mesmas regras dispostas anteriormente. Ou seja, deverá ser utilizada da fonte “Liberation Sans” em tamanho 12. Os títulos de seções serão numerados com algarismos arábicos e destacados em negrito. Haverá um espaço em branco de 1,00cm acima e 0,5cm abaixo do título de cada seção.

\section{Litas numeradas e não numeradas}
Os artigos poderão conter listas numeradas ou não. Quando não houver numeração, deverão ser marcadas com ponto, conforme o exemplo abaixo: 
\begin{enumerate}
\item exemplo 1
\item exemplo 2
\item exemplo 3
\end{enumerate}

\begin{quoting}
\lipsum[11] 
\end{quoting}

\lipsum[1-10]
\lipsum[11][1-2]\cite{donaldknuth1984,leslielamport1994}\lipsum[11][3-5]

\begin{figure}[htbp]
 \centering
 \includegraphics[width=0.5\textwidth]{example-image-a}
 \caption{Imagem de teste.}
 \label{fig-img-a}
\end{figure}

\lipsum[12]

\section{Metodologia}
\lipsum[13-14]
\begin{equation}\label{eq-bin}
\binom{n}{k} = \frac{n!}{k!(n-k)!}
\end{equation}
\lipsum[15] (ver Equação \ref{eq-bin} e Tabela \ref{tab-exemplo}).\footnote{\lipsum[30]}

\begin{table}[htbp]
\centering
\caption{Tabela de exemplo para o Texto Livre}\label{tab-exemplo}
\begin{tabular}{cccc}
\toprule
  Dec  & Bin       & Octal & Hexa \\
\midrule  
  33   & 100001    &  41   & 21   \\
\midrule
  117  & 1110101   & 165   & 75   \\
\midrule
  451  & 111000011 & 703   & 1C3  \\
\midrule
  431  & 110101111 & 657   & 1AF  \\
\bottomrule
\end{tabular}
\end{table}

\lipsum[16]

\begin{equation}\label{eq-plancherel}
\int_{-\infty}^\infty \left| f(x) \right|^2\,dx = \int_{-\infty}^\infty \left| \hat{f}(\xi) \right|^2\,d\xi.
\end{equation}

\lipsum[20-21]

\section{Conclusão}
\lipsum[17-19]
Equação \ref{eq-plancherel} e Tabela \ref{tab-exemplo}.
\lipsum[18]


% Referências
\printbibliography[title=Refer\^{e}ncias]

\end{document}

