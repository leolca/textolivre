% !TEX TS-program = XeLaTeX
% use the following command:
% all document files must be coded in UTF-8
\documentclass[english]{textolivre}
% build HTML with: make4ht -e build.lua -c textolivre.cfg -x -u article "fn-in,svg,pic-align"

\journalname{Texto Livre}
\thevolume{15}
%\thenumber{1} % old template
\theyear{2022}
\receiveddate{\DTMdisplaydate{2021}{12}{8}{-1}} % YYYY MM DD
\accepteddate{\DTMdisplaydate{2022}{4}{25}{-1}}
\publisheddate{\DTMdisplaydate{2022}{5}{17}{-1}}
\corrauthor{Paulo Barroso}
\articledoi{10.35699/1983-3652.2022.37426}
%\articleid{NNNN} % if the article ID is not the last 5 numbers of its DOI, provide it using \articleid{} commmand 
% list of available sesscions in the journal: articles, dossier, reports, essays, reviews, interviews, editorial
\articlesessionname{articles}
\runningauthor{Barroso} 
%\editorname{Leonardo Araújo} % old template
\sectioneditorname{Daniervelin Pereira}
\layouteditorname{Carolina Garcia}

\title{From reality to the hyperreality of simulation}
\othertitle{Da realidade à hiper-realidade da simulação}
% if there is a third language title, add here:
%\othertitle{Artikelvorlage zur Einreichung beim Texto Livre Journal}

\author[1]{Paulo Barroso \orcid{0000-0001-7638-5064} \thanks{Email: \url{pbarroso1062@gmail.com}}}
\affil[1]{Instituto Politécnico de Viseu, Escola Superior de Educação,
Departamento de Comunicação e Arte, Viseu, Portugal.}


\addbibresource{article.bib}
% use biber instead of bibtex
% $ biber article

% used to create dummy text for the template file
\definecolor{dark-gray}{gray}{0.35} % color used to display dummy texts
\usepackage{lipsum}
\SetLipsumParListSurrounders{\colorlet{oldcolor}{.}\color{dark-gray}}{\color{oldcolor}}

% used here only to provide the XeLaTeX and BibTeX logos
\usepackage{hologo}

% if you use multirows in a table, include the multirow package
\usepackage{multirow}

% provides sidewaysfigure environment
\usepackage{rotating}

% CUSTOM EPIGRAPH - BEGIN 
%%% https://tex.stackexchange.com/questions/193178/specific-epigraph-style
\usepackage{epigraph}
\renewcommand\textflush{flushright}
\makeatletter
\newlength\epitextskip
\pretocmd{\@epitext}{\em}{}{}
\apptocmd{\@epitext}{\em}{}{}
\patchcmd{\epigraph}{\@epitext{#1}\\}{\@epitext{#1}\\[\epitextskip]}{}{}
\makeatother
\setlength\epigraphrule{0pt}
\setlength\epitextskip{0.5ex}
\setlength\epigraphwidth{.7\textwidth}
% CUSTOM EPIGRAPH - END

% LANGUAGE - BEGIN
% ARABIC
% for languages that use special fonts, you must provide the typeface that will be used
% \setotherlanguage{arabic}
% \newfontfamily\arabicfont[Script=Arabic]{Amiri}
% \newfontfamily\arabicfontsf[Script=Arabic]{Amiri}
% \newfontfamily\arabicfonttt[Script=Arabic]{Amiri}
%
% in the article, to add arabic text use: \textlang{arabic}{ ... }
%
% RUSSIAN
% for russian text we also need to define fonts with support for Cyrillic script
% \usepackage{fontspec}
% \setotherlanguage{russian}
% \newfontfamily\cyrillicfont{Times New Roman}
% \newfontfamily\cyrillicfontsf{Times New Roman}[Script=Cyrillic]
% \newfontfamily\cyrillicfonttt{Times New Roman}[Script=Cyrillic]
%
% in the text use \begin{russian} ... \end{russian}
% LANGUAGE - END

% EMOJIS - BEGIN
% to use emoticons in your manuscript
% https://stackoverflow.com/questions/190145/how-to-insert-emoticons-in-latex/57076064
% using font Symbola, which has full support
% the font may be downloaded at:
% https://dn-works.com/ufas/
% add to preamble:
% \newfontfamily\Symbola{Symbola}
% in the text use:
% {\Symbola }
% EMOJIS - END

% LABEL REFERENCE TO DESCRIPTIVE LIST - BEGIN
% reference itens in a descriptive list using their labels instead of numbers
% insert the code below in the preambule:
%\makeatletter
%\let\orgdescriptionlabel\descriptionlabel
%\renewcommand*{\descriptionlabel}[1]{%
%  \let\orglabel\label
%  \let\label\@gobble
%  \phantomsection
%  \edef\@currentlabel{#1\unskip}%
%  \let\label\orglabel
%  \orgdescriptionlabel{#1}%
%}
%\makeatother
%
% in your document, use as illustraded here:
%\begin{description}
%  \item[first\label{itm1}] this is only an example;
%  % ...  add more items
%\end{description}
% LABEL REFERENCE TO DESCRIPTIVE LIST - END


% add line numbers for submission
%\usepackage{lineno}
%\linenumbers

\begin{document}
\maketitle

\begin{polyabstract}
\begin{abstract}
We live today in a new virtual and global space. Computers and electronic devices (smartphones) make us stay online, immersed in the cyberspace, in a network connected in an all-to-all system. An increasingly hyperreal world implies how our perception depends on simulations. The whole system is swamped by indeterminacy and reality is absorbed by the hyperreality of the simulation, says Baudrillard. Hyperreality and simulation replace and seem more real than reality itself. We must reflect on what the virtual is and what are its effects or consequences, since each new electronic medium or digital device brings new procedures, behaviors, and ways of being. Following a theoretical and conceptual approach, the aims of this study are: a) to understand the implications of the virtual and its effects, and b) to problematize the ordinary experiences of hyperreality that reshape and restructure patterns of culture and social interaction. The virtual is not just what Baudrillard defines as illusion. The virtual thinks for us. In the recent past, it was the opposite. We conclude that technology has accustomed us to virtual mediatization and now we perceive it as real without distinction, preferring the unlimited power of the illusory with its effects to the limitations of the real.

\keywords{Cyberspace \sep Cyberculture \sep Hyperreality \sep Simulation \sep Virtual}
\end{abstract}

\begin{portuguese}
\begin{abstract}
Vivemos hoje numa nova esfera virtual e global. Computadores e aparelhos eletrônicos (\textit{smartphones}) nos fazem ficar \textit{on-line}, imersos na ciberesfera, numa rede conectada num sistema todos-para-todos. Um mundo cada vez mais hiper-real tem implicações na dependência da nossa percepção em simulações. Todo o sistema é inundado pela indeterminação e a realidade é absorvida pela hiper-realidade da simulação, diz Baudrillard. A hiper-realidade e a simulação substituem e parecem mais reais do que a própria realidade. Devemos refletir sobre o que é o virtual e quais são os seus efeitos ou consequências, uma vez que cada novo meio eletrônico ou dispositivo digital traz novos procedimentos, comportamentos e modos de ser. Seguindo uma abordagem teórica e conceitual, pretende-se: a) compreender as implicações do virtual e os seus efeitos e b) problematizar as experiências cotidianas de hiper-realidade que remodelam e reestruturam padrões de cultura e interação social. O virtual não é apenas o que Baudrillard define como ilusão. O virtual pensa por nós. No passado recente nós é que pensávamos no virtual. Conclui-se que a tecnologia nos habituou à mediatização virtual e agora a percebemos como real, sem distingui-la do real, preferindo a potência ilimitada do ilusório com seus efeitos às limitações do real.

\keywords{Ciberespaço \sep Cibercultura \sep Hiper-realidade \sep Simulação \sep Virtual}
\end{abstract}
\end{portuguese}
% if there is another abstract, insert it here using the same scheme
\end{polyabstract}

\section{Introduction}

\textcite{eco_faith_1998}, in \textit{Faith in fakes: Travels in hyperreality}, and \textcite{baudrillard_america_1989} in \textit{America} use the concept of “hyperreality” to describe how our perception of the world increasingly depends on simulations of reality. As we become a more communicational and technological society, we also become an increasingly hyperreal society, a world perceived in an increasingly different way and far removed from what it really is.

The problem is due to the signs we receive and interpret as stimuli. The signs of hyperreality escape from referentiality. \textcite{eco_faith_1998,baudrillard_america_1989} argue “a tendency for signs to break loose from their referential moorings, to fly free of cognitive meaning and take on a hyper-life of their own that is more real than reality and hence hyperreal”, points out \textcite[p.~41]{tiffin_hyperreality_2005}.

As \textcite[p.~2]{baudrillard_symbolic_2000} refers in \textit{Symbolic exchange and death}, “today the whole system is swamped by indeterminacy, and every reality is absorbed by the hyperreality of the code and simulation”. According to \textcite[p.~2]{baudrillard_symbolic_2000}, the principle of simulation governs us now, rather than the outdated reality principle, and “we feed on those forms whose finalities have disappeared”. There are only simulacra, concludes \textcite{baudrillard_symbolic_2000}.

Such indeterminacy is reshaped by deep social transformations caused by hyperreality and simulation. For Baudrillard, what regulates social life today is a principle of simulation, not reality. The core of Baudrillard’s concerns is the symbolic exchange \cite[p.~36-37]{smith_baudrillard_2010}. Symbolic exchange is the authentic form of both simulation and reality. Communication and social interaction are made through symbolic exchange in any cultural context.

Hyperreality and simulation are connected. They produce and represent the reproduction, copy or sign (image) without referent, i.e. without objective correspondence with reality. Hyperreality refers to a simulation of reality, which is more real than reality itself. This is paradoxical, on one hand, and the main concern about hyperreality, on the other hand, since the concept of “hyperreality” implies the effects of media and mass culture, especially the virtual reproduction of objects, events, or everyday experiences that replace the authenticity of real. This is the classical idea that the copy replaces and seems more real than the original.

Hyperreality is the new technological and communicational paradigm, which is associated with the development of technology. Hyperreality implies profound transformations, but also several concerns and effects. The concept of “phygital” (a fusion of the words “physical” and “digital”) represents the evolution of the modern day-to-day experience based and influenced by technology, which adapts to our changes in social behavior. One example is the hyperMirror, “a video conversation system that is not meant to simulate face-to-face communication but rather allow various users from different locations to feel as though they are all in the same room” \cite[p.~68]{kipper_augmented_2013}. The integration between the physical world and the digital world provided by the “phygital” reveals, on one hand, the fusion of these two worlds and, on the other hand, the inability to distinguish them, such as the augmented reality as it will be shown further on.

Therefore, hyperreality invites us to reflect on what it is, what its nature is and what its characteristics are. Furthermore, it is also relevant to conceptualize and problematize how the future will be. Technique always brings a new way of thinking about it. In “The question concerning technology”, \textcite[p.~13]{heidegger_question_1977} points out that “\textit{techné} belongs to bringing-forth, to \textit{poiesis}; it is something poietic”. \textit{Techné} is part of the “production”, the \textit{poiesis}. Each new medium or device is a poietic or “productive” communication technique that brings new procedures and new ways of being in the world; it invites reflection.

Therefore, following a theoretical and conceptual approach, based on an exploratory bibliographic survey, this article aims to understand the associations and implications between our everyday application of the virtual with reality itself, which is perceptibly rootless and indistinct from hyperreality. The purpose is to problematize the ordinary experiences of hyperreality that reshape and restructure our patterns of social interaction and interdependence, on one hand, and, at the same time, the ways of seeing, thinking, feeling, acting, and meaning / interpreting ourselves and reality around us.

\section{Beyond the virtual public space}

In recent decades, technological changes are faster and more profound. Since the emergence of the Internet, scientific advances and technical and electronic developments have allowed global changes in everyday ways of life. These changes have become so rapid and profound that we hardly notice them anymore, nor do we reflect on their effects. Societies are on their way to merge into one and become an e-Sphere, i.e. what \textcite[p.~1]{pelton_e-sphere:_2000} defines as a virtual and contemporary public space.

An e-Sphere has multiple characteristics; it is complex and multiform. However, an e-Sphere may be understood by the following nine aspects: 
\begin{enumerate*}[label=\alph*)]
\item network, the web of electronic relations;
\item electronic communication (through technological and electronic devices);
\item connectivity and interactivity, the transition to an online state;
\item global ways of thinking, feeling, acting, seeing, and understanding;
\item online collective or community, sense of integration, belonging, and interaction in an electronic culture (sharing a virtual space);
\item virtuality, a digital dimension outside time and space (hyperreality and possible worlds where one wishes to be and participate);
\item simulation, a set of simulacra (perceptions of appearances), which gives the perception of authenticity or does not even allow this perception (because it is distracting) or the distinction between reality and unreality (because it is an \textit{analogon});
\item virtual images that evoke imagination, fantasy, spectacle, a distracting and appealing component of the images that populate the virtual and contemporary public space and that absorb attention and interest (the \textit{synopticon});
\item contemporaneity, a time of immanence, immediacy, ephemerality, and superficiality, i.e. an epiphenomenon, a superficial appearance of something or situation and its underlying reality, as explained by \textcite[p.~89]{bruce_sage_2006}.
\end{enumerate*}
%a) network, the web of electronic relations; 
%b) electronic communication (through technological and electronic devices); 
%c) connectivity and interactivity, the transition to an online state; 
%d) global ways of thinking, feeling, acting, seeing, and understanding; 
%e) online collective or community, sense of integration, belonging, and interaction in an electronic culture (sharing a virtual space); 
%f) virtuality, a digital dimension outside time and space (hyperreality and possible worlds where one wishes to be and participate); 
%g) simulation, a set of simulacra (perceptions of appearances), which gives the perception of authenticity or does not even allow this perception (because it is distracting) or the distinction between reality and unreality (because it is an \textit{analogon}); 
%h) virtual images that evoke imagination, fantasy, spectacle, a distracting and appealing component of the images that populate the virtual and contemporary public space and that absorb attention and interest (the \textit{synopticon}); 
%i) contemporaneity, a time of immanence, immediacy, ephemerality, and superficiality (i.e. an epiphenomenon, a superficial appearance of something or situation and its underlying reality, as explained by \textcite[p.~89]{bruce_sage_2006}.

Hyperreality is a modern, visual, and attractive manifestation and need of simulacra in the virtual world. This world is contemporary; that world is not and cannot be a reference. So, how is it that hyperreality and the spectacle, the simulation, and the appearance underlying the spectacle emerge from reality and present themselves in societies and in contemporaneity?

The question of hyperreality poses the problem, among others, of authenticity. Other equally important problems are the effects of de-realization; not adding new knowledge on the real world; the indistinction between the real and the fictitious. What is authentic or real is raised using images and technological devices. The images are popular and amplify the effects of distraction and social alienation. The image is immediately absorbed, spectacular, attractive; it is an ephemeral and instant “ready-to-think” image that eliminates or dilutes concepts and produces a liquid culture that is equally ephemeral and instant. The experience of hyperreality is appealing. It is a “new world” of possibilities that opens a world of possible, of fantasy and the impossible that is reshaping and restructuring not only cultural patterns, social life, and social interdependence, but also the ways in which we see, think, feel, act, or just want to communicate and interact with others and interpret reality.

Until recently, circa 1990, we only had the physical public space and the surrounding reality as it objectively presents itself, that is, the natural, social, and cultural environment as it is subject to our perception, interpretation, and interaction. After the birth of the Internet and the cyberspace (a computerized virtual world designed to increase all possibilities of online interaction, namely social, informational, and communicational) and after the web is made available to everyone, a new and exponential world became available. We stopped interacting only offline and started to accomplish many things online and spend more time immerged and connected. The new technologies change human behavior, as well as our attitude, and our ways of seeing, thinking, feeling, acting, and meaning/interpreting ourselves and the reality around us.

With the various scientific and technological developments, we have acquired new technological tools, resources and means to apply in daily life and interact both with reality and the surrounding environment. We even started to be able to interact with non-existent realities and “people”, that is, virtual interlocutors. Advances and developments in artificial intelligence have registered such significant progress that it is now possible to talk to deceased loved ones through a computer application. New artificial intelligence tools (the chatbot, a computer programme) allow such innovative conversations and interactions.

It is in this sense that \textcite[p.~31]{baudrillard_intelligence_2005} considers immersion, immanence, and immediacy as the characteristics of the virtual world. The interactive world abolishes the demarcation line between the subject and the object \cite[p.~78]{baudrillard_intelligence_2005}. Do we live in the hyperreality of simulations? Is everything an image/sign? Do images replace the meaning and authenticity of the human experience?

We currently live in a public hybrid (virtual and real) space, which is divided between moments and interactions whether online or offline. It is a public space increasingly riddled with signs, in particular images; a resemantized space of meanings, as it always was, but now with more different meanings, because there are more signs, more visibility and signs are mainly images, regardless of whether they convey more information than meaning.

\section{From virtual to hyperreality}

In \textit{Différence et répetition}, \textcite{deleuze_difference_1968} explains what he understands to be the virtual: “Le virtuel ne s’oppose pas au réel, mais seulement à l’actuel. Le virtuel possède une pleine réalité, en tant que virtuel. […] Le virtuel doit même être défini comme une stricte partie de l’objet réel.” \cite[p.~269]{deleuze_difference_1968}. The concept “virtual” in French, as it is used by \textcite{deleuze_difference_1968} (“\textit{virtuel}”) means “potential”, “what is possible”, “what does not happen”. \textcite[p.~3]{zizek_organs_2004} calls Deleuze the philosopher of the virtual. The focus of \textcite{deleuze_difference_1968} is not virtual reality, but the reality of the virtual. Virtual reality implies the idea of imitating reality and reproducing experiences through an artificial medium. The reality of the virtual “stands for the reality of the virtual as such, for its real effects and consequences”, according to \textcite{zizek_organs_2004} \textit{Organs without bodies: On Deleuze and consequences}.

The concept of “virtual” is polysemic, ambiguous and equivocal. Following the etymology of this concept, it reveals the derivation from the medieval Latin term \textit{virtualis}, meaning energy, strength, power (in producing an effect). However, the word also derives from the Latin \textit{virtus}, \textit{virtutis}, which means the human quality of courage, value, merit, like one having certain virtues, i.e. moral excellence \cite[p.~135]{barroso_reality_2019}.

According to the “virtual” entry of the Merriam-Webster English Dictionary, the virtual is “very close to being something without actually being it” and “existing or occurring on computers or on the Internet”, i.e. something simulated on a computer and existing within a virtual reality. Today, the use of social media and digital devices is done anywhere, anytime and by anyone. The ubiquitous and very frequent use of digital devices makes the virtual world look and feel like the real world, on one hand, and the distinction between fiction and reality is less and less noticeable. Such slippage of reality characterizes the hyperreal and leads to the indistinction between these two dimensions. The distinction between what is real and what is virtual is not precise. We live on a hybrid world, where one cannot easily distinguish whether what one sees, hears, smells, and touches results from a physical world, or a world mediated by information technology.

Taking the concept of “hyperreality”, the prefix “hyper” highlights the main ingredient of combined reality and the imaginary. It is a mixture of reality and signs of reality. Signs are just representations of reality. It’s their function to represent whatever exists or, in this case of hyperreality, what doesn’t exist. In a hyperreality dimension, there is no clear indication as to how far reality goes and the signs that represent reality begin. Hyperreality refers to something that does not really exist. However, experiencing hyperreality can be so intense and realistic that it can cause confusion, even for brief moments, on what is real and what is not real.

Therefore, “hyper” means “more in excess”, something disproportionate, an over-reality, an extra world that goes beyond what is reasonable or is “excessive in extension or quality”, something that is “located above”. In turn, the word “reality” means an idea of common sense: “the quality or state of being real”, “the real nature or constitution of something”, “what has objective existence, what is not a mere idea, which is not imaginary, fictitious or pretended”, “what necessarily exists” \cite[p.~379]{barroso_contributions_2020}.

What is a state of hyperreal? If reality is the quality of being real or having a real and objective existence, hyperreality is a simulated reality above reality itself. Regarding this subject, \textcite[p.~48]{deleuze_plato_1983} states that “the simulacrum is an image without resemblance”.

The concept of “simulacrum” comes from the Latin \textit{simulare}, “to make like (likeness), imitate, copy, represent”, from the stem of \textit{similis} “like, resembling, of the same kind”, that is, “to give an appearance of”. A simulacrum is an image, a form, a representation of something; shadowy likeness, deceptive substitute, pretence, dissimilation. A simulacrum means an appearance without substance, a resemblance, image, representation. There is no simulacrum without signs and the abundance of simulacra originates two situations: a) the rise of hyperreality and the possibility of virtual and simulated worlds; b) the crisis of representation, i.e. what \textcite[p.~112]{virilio_lost_1991} calls a crisis caused by modern media technology, diluting differences or not allowing us to distinguish what is real and true from what is fictional.

Consequently, hyperreality is an artificialism in which reality and fiction seem indistinct. The concept of hyperreality is also used to mean the technological communication infrastructure that supports continuous and unified interaction between:
\begin{enumerate*}[label=\roman*)]
\item virtual people and virtual objects; 
\item real people and real objects;
\item human intelligence and artificial intelligence \cite[p.~380]{barroso_contributions_2020}.
\end{enumerate*}
%i) virtual people and virtual objects; ii) real people and real objects; iii) human intelligence and artificial intelligence \cite[p.~380]{barroso_contributions_2020}. 
Hyperreality is a new configuration of the human world and brings a different way of perceiving, communicating, and living.

Hyperreality effectiveness as a technological infrastructure is the present and points the way to the future. “The technical challenge of hyperreality is to make physical and virtual reality appear to the full human sensory apparatus to intermix seamlessly”, argues \textcite[p.~7]{terashima_definition_2005}. Hyperreality provides a point or place for the unified interaction between human intelligence and artificial intelligence. It is the framing or environment of people, objects, and situations in physical and virtual reality, with human intelligence and artificial intelligence between facts and fiction, which results in processes of interaction and communication, as if everything were part of the same plane or world.

\textcite{terashima_definition_2005} points out that the term “hyper” taken from the concept of “hyperreality” emphasizes that hyperreality is more than the sum of physical and virtual reality. Hyperreality is “predicated on systematic interaction between the two component realities”, in a new form or reality that “has attributes above and beyond its component realities” \cite[p.~12]{terashima_definition_2005}. According to \textcite[p.~41]{tiffin_hyperreality_2005}, “‘hyper’ means an extra dimension beyond the normal” and hyperreality “means a reality in which there is the extra dimension of virtual reality within normal physical reality”. For the human species it will be a fundamental reformulation of their perception of reality and of the world they live in, concludes \textcite{tiffin_hyperreality_2005}.

Hyperreality is a technological meta-concept. The emphasis on the prefix “hyper” underlines an extra dimension. Hyperreality has a communicational scope; it means a reality in which there is an additional dimension of virtual reality within a normal physical reality.

The hyper-world is a consistent and coherent mixture of a real (physical) world and a virtual world. The real-world consists of real and natural things and objects, i.e. what is present atomically in a set, being describable as such, as it is. The virtual world is presented as bits of information generated by a computer. A virtual world consists of images of reality captured by a photographic camera, which are visually recognized by the computer and, later, reproduced by the computer and transmitted by technological devices in virtual reality \cite[p.~8]{terashima_definition_2005}, being recognized as such, i.e. as something distinct from reality itself.

A field of coaction (joint action) provides a common place for objects and inhabitants derived from physical reality and virtual reality and serves as a workplace or area of activity in which they interact. A coaction field is within the context of a hyper-world; it provides a common site for objects and inhabitants derived from physical reality and virtual reality \cite[p.~9]{terashima_definition_2005}. It serves as a workplace or an activity area within which they interact, providing the means of communication for its inhabitants to interact in such joint activities. The field of action provides the means of communication (including words, gestures, body orientation and movement, sounds, and touches) for its inhabitants to interact in joint activities (e.g. games). The human behavior and the aspects of objects involved in the field are in accordance with shared natural (from the physics, chemistry, biology) and human laws, which govern the same elements of reality. This produces and reveals realism. In this perspective, coaction is defined by a field or place of interaction with virtual inhabitants, means of communication, knowledge, laws and controls \cite[p.~9]{terashima_definition_2005}. A field is the place of interaction and serves as a goal of cooperation; it is a system with defined limits and known rules.

Hyperreality and virtual reality are distinct. Hyperreality includes virtual reality. Hyperreality and virtual reality are increasingly difficult to distinguish. The perceptive phenomenon of the virtual and hyperreality happens like the perception of signs in any process of semiosis (i.e. the perception and recognition of signs and their meanings and our consequent responses to them), when we become aware of a sign (in its signifier, formal dimension) and we are led to create a mental image of its meaning. As objects of perception, signs are mentally processed as if they want to represent something and invite us to make a semantic transition between their form (signifier) and their content (meaning). In hyperreality, the signs that compose it are equally objects of perception, regardless of being virtual. Signs are substitutes of something “re-presented” and absent; they are “in the place of”, according to the classic definition \textit{aliquid stat pro aliquo} \cite[p.~14]{eco_semiotics_1986}. However, this function is not always evident and perceptible. This definition of the sign, which is corroborated by Peirce, fits the semiotic characterization of hyperreality. According to \textcite[§ 2228]{peirce_collected_1978}, a means is “something which stands to somebody for something”. “We already live in a mixture of the real and the virtual”, argues \textcite[p.~32]{tiffin_hyperreality_2005}, but “the virtual realities generated outside ourselves are normally separated from our physical surroundings by some kind of frame”. Therefore, the long-term goal of hyperreality research is that the frames will disappear, and we will cease to be conscious of any seams between the virtual and the real \cite[p.~32]{tiffin_hyperreality_2005}.

Hyperreality is a technological paradigm. Hyperreality is associated with the development of technology and implies profound transformations. Hyperreality is also the era of new social relationships in a virtual world. Hyperreality is everywhere. According to Baudrillard, the world becomes hyperreal, riddled with simulacra, in which images replace the concepts of “production” and “class conflict” as key components of contemporary societies. A hologram is an example of hyperreality. The etymology of this term explains it, since it is derived from the Greek \textit{holos}, “all” (in the sense of the three dimensions), and \textit{grafia}, “message”, meaning an intermediate photograph that contains information to reproduce a three-dimensional image by holography. For Baudrillard, there is no need for imaginary mediation to reproduce what it represents. A holographic reproduction, says Baudrillard in \textit{Simulacra and simulation}, is no longer real, it is already hyperreal. “Nothing resembles itself, and holographic reproduction, like all fantasies of the exact synthesis or resurrection of the real (this also goes for scientific experimentation), is already no longer real, is already hyperreal” \cite[p.~109]{baudrillard_simulacra_1997}. According to \textcite[p.~109]{baudrillard_simulacra_1997}, “it thus never has reproductive (truth) value, but always simulation value”. It is not an exact, but a transgressive truth, i.e. already on the other side of the truth.

In this perspective, Baudrillard emphasizes the question “why is there nothing instead of something?”, since the hyperreal replaces the real, i.e. the transcendent replaces the immanent and the contingent. This is what corresponds to the strength of the virtual, in which everything (events or activities) can only come from the immanence.

\section{Constructing the cyberspace and cyberculture}

There are several changes taking place in current social interaction, which are now virtual. Cultures are no longer traditional, i.e. based on institutional and formal identities, belongings, and participation. Today cultures have more online patterns and network interactivity. In this perspective, the expansion of the Internet and its incorporation to everyday procedures led to an emphasis on network interactions. With the development of technology and telecommunications, the renewal of social bonds on a global scale has been generated, in a new cultural ecosystem: an invisible and mobile (de-territorialized) space, without borders.

We live in an ever-changing cultural ecosystem. In recent years, changes have been faster and deeper. The ecosystem became one in the world, through globalization and the virtual. Today, hyperreality is a stable feature of modern world-life. We are facing the appearance of a new electronic configuration that does not yet allow us to determine its next or final stage. Understanding the ecosystem and its changes is more urgent than in previous years. Therefore, any comprehensible study of cultural transformations and the phenomena and factors that originate these changes, namely the development of new information and communication technologies and devices, as well as the consequences in the structure of social relations, are welcome. Information and communication technologies are now more versatile and effective, imposing a new reorganization of society, which appropriately acquires the designation of “information society”.

Information and its flows have always characterized societies, but never as in contemporary years. The fast rise and predominance of the Internet, social media, digital devices, social networks, and mobile communications demonstrate the large flow of information and its virtual dimension on a global scale. As \textcite[p.~426]{toffler_third_1981} early warned long before what is happening now, “the permutations offered by the new communications technologies are endless and extraordinary”. Regarding the technological development of networks, communication, and societies, Castells characterizes this recent network society in \textit{The network society: A cross-cultural perspective:} “a network society is a society whose social structure is made of networks powered by microelectronics-based information and communication technologies” \cite[p.~3]{castells_network_2004}. The social structure is organized on a global network, a set of interconnected nodes.

In \textit{The information age: Economy, society, and culture - Volume III: End of millennium}, \textcite{castells_information_2010} analyzes the contemporary societies organized in a global information network. For \textcite[p.~2]{castells_information_2010}, the current era is that of information, marked by “informationalization, globalization, networking, identity-building, the crisis of patriarchalism, and of the nation-state”. \textcite{castells_information_2010} explores some of these macro transformations considering the interaction between processes of the information age. The trends do constitute a new historical landscape, whose dynamics are likely to have lasting effects on our lives \cite[p.~2]{castells_information_2010}. It is a new society, a new dominant social structure called “network society”, which emerged in the second half of the 20th century with the revolution of information and communication technologies, with informational and global capitalism and with “real virtuality” immersed in an environment of virtual images.

According to \textcite[p.~386]{castells_information_2010}, the “real virtuality” is a system in which reality itself (i.e. people’s material and symbolic existence) is fully immersed in a virtual image setting, in the world of make believe, in which symbols comprise the actual experience. With the gradual predominance of attractive and spectacular signs, the profusion of images develops the visuality in contemporary cultures. The image and visuality become massified with innovative techniques to reproduce images (e.g. photography and video). In 1936, \textcite[p.~104]{benjamin_work_2002} designates this period as the “age of the technological reproducibility”. Technical reproduction, by industrializing the artifice of reproduction and by calling into question the aesthetic value and the authenticity value, is a visible part of post-modernity. This period extends to today with the phenomenon of globalization, which intensifies not only innovation and visual reproduction, but also digital and technological industrialization and the profusion of virtual images in the daily lives of cultures and societies that, in this way, are increasingly visual and global. Globalization has brought transformations, including the transition from analogue to digital. With the invention and use of photography and the cinematographic camera, societies and cultures became visual and visible, objects of registration, interpretation, and meaning.

In the thought of \textcite[p.~17]{levy_cibercultura_1999b}, cyberculture is the set of techniques (material and intellectual), practices, attitudes, ways of thinking, and values that develop together with the growth of cyberspace. Cyberspace is a construct, a vast and virtual space for action. According to the meaning attributed by \textcite{gibson_neuromancer_2003}, who coined this term in 1984 in the book \textit{Neuromancer}, cyberspace is “a virtual reality representation of a vast city which is perhaps best described as a totally immersive version of the Internet”, where individuals can exist solely in this space, and even continue to exist after their death \cite[p.~39]{bell_cyberculture:_2005}. Cyberspace designates the space created through the confluence of electronic communications networks (e.g. the Internet) which enables computer-mediated communication between any number of people who may be geographically dispersed around the globe. It is a public space where individuals can meet, exchange ideas, share information, provide social support, conduct business, create artistic media, play simulation games, or engage in political discussion \cite[p.~41]{bell_cyberculture:_2005}. “Such human interaction does not require a shared physical or bodily co-presence, but is rather characterized by the interconnection of millions of people throughout the world communicating by email, usenet newsgroups, bulletin board systems, and chat rooms” \cite[p.~41]{bell_cyberculture:_2005}.

Gibson’s \textit{Neuromancer} underlines the cyberspace as a “consensual hallucination experienced daily by billions of legitimate operators, in every nation”, a “graphic representation of data abstracted from the banks of every computer in the human system”, a “unthinkable complexity” and “lines of light ranged in the non-space of the mind, clusters and constellations of data” \cite[p.~51]{gibson_neuromancer_2003}.

Cyberspace is the world-culture network, an electronic, virtual, and unlimited space for interactive communication, where technology and information converge and where people (Internet users or “netizens”) interact when sharing or looking for the same interests (and not the same ideals, beliefs, values, and principles, as in the past). Cyberspace and cyberculture are the result of the recent technological revolution. Cyberspace is the communication space opened by the worldwide interconnection of computers and computer memories \cite[p.~92]{levy_cibercultura_1999b}. It is a vast and open communication system from all to all, that is, of all those who are interconnected. According to \textcite{levy_cibercultura_1999b}, communication takes place through shared virtual worlds. Virtual realities increasingly serve as means of communication \cite[p.~105]{levy_cibercultura_1999b}. In the same view, Pelton argues that the exploding pattern of global change pervades our planet, and it is coming to us from every direction via cell phones, fibre-optic cables, high performance and personal computers, satellites, and the all-pervasive Internet. “All these complex electronic and communications networks and the advanced software and processing power that support their operation is what is meant by the word ‘cyberspace’” \cite[p.~3]{pelton_e-sphere:_2000}.

Cyberspace technologies transform our culture. Information systems assume, at the same time, an omnipresent, omniscient, and omnipotent role. Cyberspace’s power, reach, and immediacy creates an overarching presence that transcends the Global Village paradigm defined by Marshall \textcite{mcluhan_understanding_1994}. “Now the Internet and modern telecommunications and computer networks can let us think interactively”, since now we are a “world-wide mind” that thinks and interacts together \cite[p.~4]{pelton_e-sphere:_2000}.

Cyberculture is associated with both cyberspace and virtual dimension. Immersion characterizes cyberculture and the virtual participation or presence in it. Cyberculture, cyberspace, and the virtual characterize the contemporary culture and current technological modes of instant communication, digital mediation, and easy access to the world of network information.

Virtualization spans all areas of human life. Today, a general virtualization movement affects not only information and communication, but also bodies, economic functioning, collective structures of sensitivity or the exercise of intelligence. According to \textcite{levy_que_1999a}, virtualization even reaches the ways of being together, the formation of “we”: virtual communities, virtual companies, virtual democracy, etc. \cite[p.~7]{levy_que_1999a}. The digitization of messages and the extension of cyberspace play an important role in the ongoing transformation. It is a background wave that largely overflows computerization \cite[p.~7]{levy_que_1999a}.

Communities are now interacting virtually, although physical presence and interaction continues. But virtual communities are organized into telematic communication systems that seem to not only satisfy the needs for interaction and belonging/presence in the public space. They are more comfortable, mobile, immediate and portray a modern lifestyle. The public space is no longer necessarily physical now. As \textcite[p.~19]{levy_que_1999a} underlines, we can be here and there at the same time due to the communication and telepresence techniques.

Technologies and communication devices are experiencing massive and radical developments. A new communication device appears within the very large de-territorialized communities as one of the main effects of the ongoing transformation \cite[p.~90]{levy_que_1999a}. “This can be experienced on the Internet, on bulletin boards, in electronic conferences or forums, in cooperative work or learning systems, in groupware or collective programs, in virtual worlds and in knowledge trees.” \cite[p.~90]{levy_que_1999a}. Cyberspace facilitates large-scale non-mediatic communication that constitutes a decisive advance for new, more evolved forms of collective life \cite[p.~90]{levy_que_1999a}.

In a period of rapid and profound technological development, when people are present most of the time in a digital environment and with virtual experiences and interactions, the perception of reality (not to mention “unreality” or “de-realization”) is influenced (modified) and fragmented (parcelled) by devices and means of communication. Therefore, we participate in a virtual reality, an augmented reality, or alternative reality. None of these “realities” is properly the reality, the physical and concrete, that is, the commonly perceived as that which is external to the subject who perceives it; they are virtual environments produced by technological devices and in which we immerse.

Augmented reality is defined by the addition of virtual elements, data, or information to the environment (called “reality”) where we immerse and with which we interact (e.g. Pokemon Go). This is the reason for augmented reality: it adds something. In \textit{Virtual reality and augmented reality: Myths and realities}, \textcite[p.~xxvi]{arnaldi_virtual_2018} argue that the goal of augmented reality is to enrich the perception and knowledge of a real environment by adding digital (and most often visual) information related to this environment.

As for virtual reality, it is defined by what is not and cannot be tangible, so, in common sense, it is the illusory, the unreal, or what has no concrete material existence, because the virtual is what is “de-territorialized”. Virtual reality is characterized by immersion, which allows us to interact with an environment composed of images produced by a computer. Through immersion, we enter (taking the initiative) or are transported (only mentally, when guided by images) to a virtual dimension or “world”, as when playing a computer game or watching a three-dimensional film with appropriate glasses (e.g. Google Glass). Regarding computer games, they “provide an extraordinary access to immersive possibilities, to the extent that players are liberated from physical and even from moral boundaries which are characteristic of the real world” \cite[p.~16]{ariso_is_2017}.

Virtual reality technologies immerse the user in a synthetic environment. According to \textcite[p.~1]{kipper_augmented_2013}, “while immersed, the user cannot see the real world around him” and, in contrast, augmented reality takes digital or computer-generated information (images, audio, video, and touch or haptic sensations) overlaying them in a real-time environment. “Augmented Reality technically can be used to enhance all five senses, but its most common present-day use is visual” \cite[p.~1]{kipper_augmented_2013}. Augmented reality supplements reality rather than completely replaces it, since it allows the user to see the real world, unlike virtual reality.

The human being is the product of the environment. Despite his limitations and needs, the human being can adapt to the environment, even when this environment is technological and produced by him, as the virtual environment. The human is immersed in the virtual and \textcite{mcluhan_understanding_1994} warned about the effects of technology some 60 years ago. For \textcite{mcluhan_understanding_1994}, the medium is reshaping and restructuring patterns of social interdependence and all aspects of our personal lives (as the use of smartphones, social networks, and the consumption of digital information through these technological means) making us numb, deaf, blind and dumb \cite[p.~17]{mcluhan_understanding_1994}. In turn, Ariso states that augmented reality environments are barely modified when a mixed reality is created, and provides a privileged look and hypersensitivity, which has a feedback effect on intellectual capacity \cite[p.~18]{ariso_is_2017}.

In \textit{The VR book: Human-centered design for virtual reality}, \textcite{jerald_vr_2016} says that virtual reality can provide our minds with direct access to digital media in a way that seemingly has no limits and “the results are brilliant and pleasurable experiences that go beyond what we can do in the real world” \cite[p.~1]{jerald_vr_2016}. Virtual reality allows its user to perform virtual tasks. The user believes that he is acting in the real world, as virtual reality generates and evokes that sensation. The technology “tricks” the brain, providing information identical to the information that the brain would perceive in the real environment. Virtual reality allows the users to virtually execute tasks while believing that they are executing them in the real world. “To generate this sensation, the technology must ‘deceive the brain’ by providing it with information identical to the information the brain would perceive in the real environment” \cite[p.~xxii]{arnaldi_virtual_2018}.

In short, the concept of “virtual reality” is paradoxical, because what is reality (i.e. the state or quality of something being real) cannot be virtual. Virtual reality corresponds to an artificial environment which is experienced through sensory stimuli provided by computer with which and through which we act and react without noticing the difference between the dimensions of reality and the virtual. Virtual reality is the transmission of data (signs structured in a certain code), information and communication. Traditionally, communication is an interaction between two or more people. Now, with new technologies, communication becomes more abstract and is also the interaction between human and technology or simply the transfer of energy or information between two entities: human being versus machine; real people versus virtual people; human intelligence versus artificial intelligence; physical reality versus virtual reality.

\section{Conclusions}

We live today with a digital mediatization. The digital and virtual dimensions of communicative interaction and the digital devices produce signs (specially images) that generate hyperreality with effects on our perception and sensation. The culture is converted into a cyberculture. In fact, the Internet created another world, a virtual and global world without borders called the cyberspace. The revolution in information technologies and the impact of computerization and virtualization of social actions lead to structural transformations in societies and cultures, which are now digital.

In a world increasingly riddled with signs and representations (images), the virtuality of these signs/images is the potency, force, and violence of the signs/images, which increase their effects and impel the human senses. It is like the sign-feeling, a sign in excess and force that punctures. With the virtual and in the hyperreal dimension, the virtual leads to an excess of meaning, going beyond the limits of representation itself and even entering the domains of hyperreality.

Therefore, the concepts of “virtual” and “virtuality” mean what exists potentially and not in action, what is likely to be realized or exercised, what is possible or potential. In the theory of causes the effect is contained in the cause; in the virtual dimension the possible is contained in the actual (in what exists). The presence of the effect in the cause is virtual, just as the presence of the possible in the actual is also virtual; both are neither effective nor formal.

The problem concerning what is true or the question about what the truth is, is the problem of the virtual, because both involve discerning what the case is (truth, real, factual, or actual) from what is not the case (appearance, false, unreal or illusion). If we see the virtual as an illusion, it is present, for example, in the processes of perception, representation and meaning provided by the different means of communication. The illusion caused by the medium and all social media or digital devices resides in the capacity for virtualization that this medium produces. The medium’s ability to create an illusion is a process of virtualization. But the virtual is not, as \textcite[p.~83]{baudrillard_intelligence_2005} clarifies in \textit{The intelligence of evil or the lucidity pact}, as “the lost world”; it is just the virtual illusion, the illusion of the virtual. The virtual is now what asserts itself as real. A virtual that thinks us.

With this conceptual approach, an understanding of both the nature of the virtual and the complex and aporetic effects arising from virtuality and hyperreality is expected. They are effects produced using signs/images in daily experiences (e.g. take digital photographs and edit them, play video games, use avatars, design architectural projects or simply browse the internet) whose associated representation conceptually overflows these signs/images. This contribution is also extended to the understanding of the virtualization of reality through signs/images, the virtualization of forms of representation/signification and perception/recognition as power, passage from the actual/real to the virtual. This virtualization is an irreducible social and cultural phenomenon that is increasingly current in modern societies.

\printbibliography\label{sec-bib}
% if the text is not in Portuguese, it might be necessary to use the code below instead to print the correct ABNT abbreviations [s.n.], [s.l.]
%\begin{portuguese}
%\printbibliography[title={Bibliography}]
%\end{portuguese}





\end{document}

