% !TEX TS-program = XeLaTeX
% use the following command:
% all document files must be coded in UTF-8
\documentclass[english]{textolivre}
% build HTML with: make4ht -e build.lua -c textolivre.cfg -x -u article "fn-in,svg,pic-align"

\journalname{Texto Livre}
\thevolume{15}
%\thenumber{1} % old template
\theyear{2022}
\receiveddate{\DTMdisplaydate{2021}{12}{29}{-1}} % YYYY MM DD
\accepteddate{\DTMdisplaydate{2022}{1}{11}{-1}}
\publisheddate{\DTMdisplaydate{2022}{2}{23}{-1}}
\corrauthor{Mansour Amini}
\articledoi{10.35699/1983-3652.2022.37677}
%\articleid{NNNN} % if the article ID is not the last 5 numbers of its DOI, provide it using \articleid{} commmand
% list of available sesscions in the journal: articles, dossier, reports, essays, reviews, interviews
\articlesessionname{articles}
\runningauthor{Tee et al.} 
%\editorname{Leonardo Araújo} % old template
\sectioneditorname{Daniervelin Pereira}
\layouteditorname{Leonado Araújo}

\title{English to Chinese fansub translation of humour in \textit{The Marvellous Mrs. Maisel}}
\othertitle{Tradução de inglês para chinês do fansub de humor em \textit{The Marvelous Mrs. Maisel}}
% if there is a third language title, add here:
%\othertitle{Artikelvorlage zur Einreichung beim Texto Livre Journal}

\author[1]{Yee Han Tee \orcid{0000-0001-9779-0915} \thanks{Email: \url{tyhan2397@live.com}}}
\author[2]{Mansour Amini \orcid{0000-0003-2149-4604} \thanks{Email: \url{mansour@usm.my}}}
\author[3]{Ching Sin Siau \orcid{0000-0001-7612-6839} \thanks{Email: \url{chingsin.siau@gmail.com}}}
\author[2]{Amin Amirdabbaghian \orcid{0000-0001-6503-8446} \thanks{Email: \url{amirdabbaghian@yahoo.com}}}
\affil[1]{UCSI University Malaysia, Faculty of Social Sciences and Liberal Arts, Kuala Lumpur, Malaysia.}
\affil[2]{Universiti Sains Malaysia, School of Languages, Literacies and Translation, Penang, Malaysia.}
\affil[3]{Universiti Kebangsaan Malaysia, Faculty of Health Sciences, Kuala Lumpur, Malaysia.}



\addbibresource{article.bib}
% use biber instead of bibtex
% $ biber article

% used to create dummy text for the template file
\definecolor{dark-gray}{gray}{0.35} % color used to display dummy texts
\usepackage{lipsum}
\SetLipsumParListSurrounders{\colorlet{oldcolor}{.}\color{dark-gray}}{\color{oldcolor}}

% used here only to provide the XeLaTeX and BibTeX logos
\usepackage{hologo}

% if you use multirows in a table, include the multirow package
\usepackage{multirow}

% provides sidewaysfigure environment
\usepackage{rotating}

% CUSTOM EPIGRAPH - BEGIN 
%%% https://tex.stackexchange.com/questions/193178/specific-epigraph-style
\usepackage{epigraph}
\renewcommand\textflush{flushright}
\makeatletter
\newlength\epitextskip
\pretocmd{\@epitext}{\em}{}{}
\apptocmd{\@epitext}{\em}{}{}
\patchcmd{\epigraph}{\@epitext{#1}\\}{\@epitext{#1}\\[\epitextskip]}{}{}
\makeatother
\setlength\epigraphrule{0pt}
\setlength\epitextskip{0.5ex}
\setlength\epigraphwidth{.7\textwidth}
% CUSTOM EPIGRAPH - END

\usepackage{xeCJK}
%LANGUAGE - BEGIN
%CHINESE
% for languages that use special fonts, you must provide the typeface that will be used
%\setCJKmainfont{BabelStone Han}
%\newfontfamily\CJKmainfont[Script=chinese]
%\newfontfamily\CJKmainfontsf[Script=chinese]
%\newfontfamily\CJKmainfontt[Script=chinese]
%
% in the article, to add chinese text use: \textlang{chinese}{ ... }
%
% RUSSIAN
% for russian text we also need to define fonts with support for Cyrillic script
% \usepackage{fontspec}
% \setotherlanguage{russian}
% \newfontfamily\cyrillicfont{Times New Roman}
% \newfontfamily\cyrillicfontsf{Times New Roman}[Script=Cyrillic]
% \newfontfamily\cyrillicfonttt{Times New Roman}[Script=Cyrillic]
%
% in the text use \begin{russian} ... \end{russian}
% LANGUAGE - END

% EMOJIS - BEGIN
% to use emoticons in your manuscript
% https://stackoverflow.com/questions/190145/how-to-insert-emoticons-in-latex/57076064
% using font Symbola, which has full support
% the font may be downloaded at:
% https://dn-works.com/ufas/
% add to preamble:
% \newfontfamily\Symbola{Symbola}
% in the text use:
% {\Symbola }
% EMOJIS - END

% LABEL REFERENCE TO DESCRIPTIVE LIST - BEGIN
% reference itens in a descriptive list using their labels instead of numbers
% insert the code below in the preambule:
%\makeatletter
%\let\orgdescriptionlabel\descriptionlabel
%\renewcommand*{\descriptionlabel}[1]{%
%  \let\orglabel\label
%  \let\label\@gobble
%  \phantomsection
%  \edef\@currentlabel{#1\unskip}%
%  \let\label\orglabel
%  \orgdescriptionlabel{#1}%
%}
%\makeatother
%
% in your document, use as illustraded here:
%\begin{description}
%  \item[first\label{itm1}] this is only an example;
%  % ...  add more items
%\end{description}
% LABEL REFERENCE TO DESCRIPTIVE LIST - END


% add line numbers for submission
%\usepackage{lineno}
%\linenumbers

\begin{document}
\maketitle

\begin{polyabstract}
\begin{abstract}
Humour is bounded by culture or language, and it involves shared knowledge and history between the sender and the receiver, which could make humour subtitling even more complicated. This study aimed at exploring the subtitling strategies used in fansub in the television series \textit{The Marvelous Mrs. Maisel}, as well as the humour elements of source and target text in each humour, and the translation errors made in English to Chinese fansub. Humour instances were identified and analysed, and the subtitling strategies used by fans were explained. The findings showed the inconsistent quality of fansub could be due to linguistic and technical constraints, insufficient cultural knowledge of the source and target cultures, and deviations from translation norms, resulting in linguistic, pragmatic, cultural and text-specific translation errors. The study could have some theoretical and practical implications for translators, subtitlers, and trainers.

\keywords{Chinese \sep English \sep Fansub \sep Humor \sep Subtitling}
\end{abstract}

\begin{portuguese}
\begin{abstract}
O humor é limitado pela cultura ou linguagem, e envolve conhecimento e história compartilhados entre o emissor e o receptor, o que poderia tornar a legendagem de humor ainda mais complicada. Este estudo teve como objetivo explorar as estratégias de legendagem utilizadas no fansub da série televisiva \textit{The Marvelous Mrs. Maisel}, bem como os elementos humorísticos do texto fonte e alvo em cada humor, e os erros de tradução cometidos do inglês para o fansub chinês. Foram identificadas e analisadas instâncias de humor e explicadas as estratégias de legendagem utilizadas pelos fãs. Os resultados mostraram que a qualidade inconsistente do fansub pode ser devido a restrições linguísticas e técnicas, conhecimento cultural insuficiente das culturas de origem e destino e desvios das normas de tradução, resultando em erros de tradução linguísticos, pragmáticos, culturais e específicos do texto. O estudo pode ter algumas implicações teóricas e práticas para tradutores, legendadores e treinadores. 

\keywords{Chinês \sep Inglês \sep Fansub \sep Humor \sep Legendagem}
\end{abstract}
\end{portuguese}
% if there is another abstract, insert it here using the same scheme
\end{polyabstract}

\section{Introduction}\label{sec-intro}
Humour is anything that makes people laugh or is amusing, or the capacity to recognise what is funny about a situation or a person. However, humour in Chinese constitutes an event that triggers a thoughtful smile or smile resulting from the meeting of hearts \cite{lin_impressions_1936}. Humour appears anywhere in people's daily life and it plays an essential role in media productions. Owing to the proliferation of both traditional (TV, cinema) and on-demand streaming platforms, humour is becoming increasingly internationalized within the multifaceted entertainment mediascape \cite{dore_editorial:_2019}. Although everyone shares the ability to appreciate and enjoy humour, each culture perceives what constitutes humour differently, that is what is entertaining in one culture can be tiresome or even offensive in other cultures. One must be aware of the cultural differences, especially the way humour is assumed and used, in order to communicate with those who come from various backgrounds appropriately \cite{amini_quality_2013}. 

Subtitling is a type of translation to which theorists in the field are increasingly regarding as important in translation research \cite{pardo_translation_2013}. Subtitling strategy is a ``potentially conscious procedure for the solution of a problem'', which an individual “is faced with when translating a text segment from one language into another, the conscious procedure for the solution of a translation problem” in the process of translating subtitles \cite[p.~91]{lorscher_translation_1991}.

Humour subtitling is one of the difficult challenges\footnote{The challenges are not subjective, that is, they are not seen from the author’s point of view, but they are research themes/topics that have been investigated by researchers.} in Translation Studies. Yet, it is a stimulating challenge because it involves the core of principles of the translation theory, especially in terms of translatability and equivalence \cite{chiaro_humor_2005,yap_problems_2018}. Humour study in China appeared relatively late compared to the Western world, partly owing to the insufficient attention it has received. Traditionally, the predominant Chinese philosophy, Confucianism, regarded humour negatively as a self-denigrating activity and prescribed self-restraint and seriousness in one's attitude \cite{jiang_cultural_2019}. In 1924, Yutang Lin first introduced the word `humour' into China and transliterated it as youmo, but it was not until the 1980s that the systematic study of humour as an important discipline began \cite{chen_personal_1993}. 

The cross-cultural use of humour is evident in many areas in life, such as advertising, entertainment and language acquisition. Different languages are likely to constitute different sets of symbolic references; therefore, an advertisement has to tap into such references to avoid being a blunder proposition. \cite{amini_quality_2013} Humour is used in persuasion and cognition of advertisements for any product because the use of humour in advertisements affects the sale directly and promotes the product easily \cite{venkatesh_effectiveness_2015}. In addition, humour may play an important role in language learning. For example, \textcite[p. 122]{garcia-escribano_effects_2017} suggested, “learners of English as a foreign language prefer learning through audiovisual” consumption and those humorous audiovisual products seemed to draw the attention of learners. While humour “plays an important role in increasingly globalized contexts, the expression of humour in various cultures and languages” is different because the humorous message is interpreted differently in different cultures \cite[p. 183]{yap_secondary_2020}. Hence, it is important to study the translation of humour spanning two different languages and cultures.  

From the definition of humour, the Chinese (Mandarin) understanding of humour is narrower than that of western people. For example, western people consider even funny elements like humor, while this kind of funny, translated as huaji in Chinese, is not regarded as humour for the Chinese people. The untranslatability of humour is possibly caused by the different language families these two languages belong to, and the linguistic differences are almost insurmountable \cite{han_untranslatability_2011}. However, in recent years, humour has gradually evolved to become a broad umbrella term in China for many laughter-related phenomena, whereby scholars now advocate the use of youmo in a broad sense \cite{chen_sense_2013}. The study of humour involves speakers and receivers' understanding of the cultural elements. It is a challenge to get humour across when people use different languages, come from different cultural backgrounds and have different sets of norms \cite{lutviana_failure_2012}. In humour translation, both rendering the meaning and maintaining the humorous effect in the target language (TL) are equally important \cite{nufus_acceptability_2014}. In other words, the translator should be capable to transfer and reconstruct the similar effect of the source culture (SC) in the target culture (TC), instead of merely translating the meaning. In terms of the English-to-Chinese translation of American movies, humour is understood in different ways in these two cultures.

Fansubs are homemade subtitles for cartoons, films and television series that have not yet been out in the target language country. Fansubbing refers to the process of subtitling made by fans while fansubbers refer to those fans who are involved in the process of making the subtitles \cite{diaz-cintas_subtitling:_2013}. The first fansub group in China “was established for the translation of Japanese animation, followed by American television series” \cite[p. 19]{gao_role_2018}. Fansub has only recently started to be recognised in the field of Translation Studies mostly by scholars working in the field of audiovisual translation. Fansub groups could be the only way for the Chinese audience to access foreign TV programmes. Even though they play an important role, fansubbers “have created subtitles that would probably not be considered acceptable in a professional context”, and the quality of their work is “lower than the professionally-produced subtitles of the same films” \cite[p. 101]{wilcock_comparative_2013}. The quality of Chinese fan-made subtitles still has room for improvement. 

Humour has different functions, such as bridging cultural gaps, strengthening the impact of advertisements, increasing the effectiveness of business communication, enhancing the teaching-learning process, and it is thus an important element to study. Therefore, the following research objectives were proposed: 

\begin{enumerate}
    \item “To identify the subtitling strategies employed in English to Chinese fansub translation of humour in \textit{The Marvelous Mrs. Maisel}”
    \item “To explore the humour elements in English to Chinese fansub translation of humour in \textit{The Marvelous Mrs. Maisel}”
    \item “To analyse errors in English to Chinese fansub translation of humour in \textit{The Marvelous Mrs. Maisel}”
\end{enumerate}

In the present study, humour elements refer to the knowledge resources as a list of different parameters that contribute to the humorous effect, the elements that construct a humour instance including script opposition, logical mechanism, situation, target, narrative strategy and language \cite{attardo_script_1991}. A translation error refers to a “particular expression or utterance that does not in itself have the quality of being incorrect” but is ``assigned that quality by the recipient in the right of a particular norm or standard'' \cite[p. 73]{nord_translation_2001}. 

\section{Literature review}\label{sec-normas}
Throughout this section, the history and modes of AVT, generally and specifically in China, modes of subtitling, particularly fansubbing, and translations errors related to the AVT will be described. It is rational to show some basic information on the topic to provide readers with a better understanding of the data analysis and the discussion that will come after this section.

\subsection{Audiovisual Translation}\label{sec-conduta}
Research related to audiovisual translation began in 1932. Nevertheless, it started to be studied and was included as part of the discipline of translation studies around the 1980s. The 1990s was the golden age of audiovisual translation, with significant scholarly contributions in this field and the true scholarly emergence of the field \cite{diaz-cintas_introduction-audiovisual_2009}. The audiovisual translation had not been considered as a part of the discipline of translation studies until that time. Nowadays, audiovisual technology leads the broadening of the communication landscape. 


\subsubsection{Intralingual audiovisual translation}\label{sec-fmt-manuscrito}
Intralingual audiovisual translation (AVT) is an interpretation of verbal signs by other signs of the same language. It means that the source text (ST) and the target text (TT) share the same language whereby the translation is meant to meet the needs of the hearing impaired and involves rendering the dialogues into written subtitles. The four main types of intralingual AVT are “subtitling for the deaf and hard of hearing, audio description for the blind, live subtitling, and subtitling for the opera and theatre” \cite[p. 401]{denton_new_2012}.  


\subsubsection{Interlingual audiovisual translation }\label{sec-formato}
Interlingual audiovisual translation is an interpretation of verbal signs using other languages as the purpose of translation is to make audiovisual products understandable for audiences who cannot comprehend the source language \cite{altahri_issues_2013}.
The popular audiovisual language transfer methods are “subtitling, dubbing, narration, and free commentary” \cite[p.~25]{amini_quality_2013}. 
Subtitling and dubbing are the common types of audiovisual translation. Interlingual AVT from English to other languages has been getting popular because many “viewers perceive series not only as a form of entertainment but also as a tool for learning English and universities have begun to include it as a subject for study in their curricula” \cite[p.~33]{amir_dabbaghian_cultural-discourse_2014}.

\subsubsection{Audiovisual Translation in China }\label{sec-modelo}
Audiovisual translation plays an essential role in cross-cultural communication, industrial development and social integration in China \cite{yves_audiovisual_2018}. A great quantity of Western TV products was imported, even before China developed its film industry at the beginning of the 20th century \cite{haikuo_film_2015}. 

The Shanghai Peacock Film Company first started to make foreign films with subtitles in Chinese in 1922. The audience had to read the subtitles shown in slides to comprehend the content of the foreign movies \cite{haikuo_film_2015}. In 1939, the Grand Guangming Theatre introduced `Earphone' from America. The audiences without “knowledge of the source language (SL) could put on the earphones to listen to the Chinese explanation of the foreign film story and the Chinese version of the dialogues while watching the foreign films” \cite[p. 28]{jin_comparative_2018}. 

The golden era of dubbed films in China was in the 1980s, when the Chinese authorities allowed the Chinese audience to gain insights from the world outside and encouraged them to connect with foreigners. Most imported foreign movies were dubbed because there was only a limited number of people who could understand other languages and most of the Chinese people were illiterate. Subtitled movies became a popular form of audiovisual text among young people since the beginning of the 1990s, especially among college students and educated people \cite{haikuo_film_2015}. 

Audiovisual translation in China has three major directions. First and foremost, translation of foreign audiovisual products into Chinese, to be shown in cinemas, television and online channels, has a long history. It appeared sometime after foreign films were imported to China by the end of the 19th century. Translated foreign movies and TV shows have been “exerting a great impact on Chinese society and Chinese culture since then” \cite[p. 27]{jin_comparative_2018}. 

The second direction refers to the translation of Chinese audiovisual products into foreign languages to attract international audiences. In recent years, with the sponsoring of the “Sino-Africa Film and Television Cooperation Project” in 2012, the “Contemporary Works Translation Project” in 2013 and the “Silk Road Film and Television Bridge Project” in 2014, a great number of Chinese movies, TV dramas, documentaries and cartoons have been translated and dubbed into over 20 languages \cite[p. 27]{jin_comparative_2018}. It is including “Swahili, Hausa, and Russian”, as a way of “demonstrating Chinese culture and values to international audiences and reshaping China's image” universally \cite[p.~27]{jin_comparative_2018}.  

The third direction is the translation of Chinese movies and TV shows into ethnic minority languages inside China. It is estimated that there are 129 languages currently being spoken in China \cite{sun__2007}. Since the 1950s, audiovisual products in China have been translated into ethnic minority languages such as “Mongolian and Tibetan to achieve ideological, educational, and social purposes and meet the entertainment needs of ethnic minorities” \cite[p. 28]{jin_comparative_2018}.   

Audiovisual translations in the current Chinese market share some common errors \cite{jin_comparative_2018}, such as problems related to format and punctuation \cite{jing_problems_2019}, unstandardised translations of the film title, lengthy sentences, literal translation without referring to the visual image, rigid literal translation, and free translation. These errors may cause failure to deliver humorous effects in the target text. 

\subsection{Subtitling}\label{sec-organizacao}
Subtitling is incorporating a written text (subtitles) on the screen in the target language where an original version of the audiovisual product is shown, given that the subtitles coincide approximately with the screen actors' dialogues \cite{gambier_position_2013}.

\subsubsection{Intralingual Subtitling}\label{sec-organizacao-latex}
Intralingual subtitling is “a shift from the spoken mode of the verbal exchange in a film or television programme to the written mode of the subtitles” \cite[p. 273]{gambier_position_2013}. The two main purposes of using intralingual subtitles are language learning and accessibility. Intralingual subtitles are used by young people or migrants to learn a particular language. It is often a teletext option on television, hence it is also known as closed captions, which means the subtitles cannot be turned off. The two types of intralingual subtitling are different in the way they are processed. The first type, language learning, does not mention signal noises, doors slamming, angry voices, telephones ringing, shouting and so on. The second type, which is made for the deaf or hearing impaired, usually renders verbal and non-verbal audio material into the text.

\subsubsection{Interlingual Subtitling}\label{sec-autores}
Interlingual subtitling involves “moving from oral dialogue in one or several languages to one or two written lines” \cite[p. 274]{gambier_position_2013}. The principle is that there is a need for a transfer or translation between two different verbal sign systems, yet it denotes subtitling instead of traditional written translation \cite{liu_foreign_2015}. It is noteworthy that in this type, not only are the two languages involved but also two dimensions, speeches, and writings. 

\subsubsection{Subtitling Strategies}\label{sec-idioma}
Vinay and Darbelnet proposed the very first general taxonomy of translation strategies in the late 1950s, and it can also be applied in subtitling. \textcite[p.~1]{pedersen_how_2017} identified seven subtitling strategies i.e. “Retention, Specification, Direct Translation, Generalization, Substitution, Omission, Official Equivalence”. 

One of the most popular subtitling typologies is \textcite[p. 163]{gottlieb_subtitles_1997} typology, which includes “ten subtitling strategies of expansion, paraphrase, transfer, imitation, transcription, dislocation, condensation, decimation, deletion, and resignation”. Gottlieb's typology provides translators with a comprehensive model that includes the available subtitling strategies in the process of translating subtitles. It is widely used by scholars and researchers in studying the translation of subtitles (e.g. \cite{ghaemi_strategies_2011,bak_analysis_2016,jin_comparative_2018,usroh_subtitling_2017}. \textcite{cheng_chinese_2014} study on 35 English films imported to Taiwan suggested that the subtitling in Taiwan is source text (ST)-oriented. Adapting Gottlieb's classifications of subtitling strategies, \textcite{kianbakht_audiovisual_2015} found that transfer was the most frequent strategy used. \textcite{bosch_translation_2016} concluded that substitution is the dominant strategy in Spanish dubbing, while Spanish translation employs a more official equivalence strategy. \cite{jin_comparative_2018} conducted a study and analysed the subtitling of animation films from English to Chinese using Gottlieb's typology and concluded that ten subtitling strategies of transfer, condensation, paraphrase, transcription, globalization and localization were more effective.

\subsubsection{Subtitling Humour}\label{sec-resumo}
As humour is bounded with culture or language, the `common knowledge' involving a `shared history' between `the sender' and `the receiver' is essential in understanding the joke \cite{cheng_chinese_2014}. The more `culture or language-specific' the humor, the more `complicated' it is to `translate' \cite{martinez-sierra_translating_2006}. There are two main humorous stimuli; incongruity and superiority. \textcite{vandaele_humor_2010} notes that incongruity represents the cognitive aspect of humor, whereas superiority highlights the social functioning of humor. Incongruity refers to a conflict between what is predicted and what actually happens in a joke. Humour is the outcome of something incompatible, something that contradicts the mental patterns and anticipations in the Incongruity Theory \cite{ghodsi_relevance-theoretic_2016}. 

The `General Theory of Verbal Humour (GTVH)' introduced by Attardo and Raskin in 1991 is an adaptation of Raskin's `Semantic Script Theory of Humour' \cite{shipley_towards_2007}. Jokes can be broken down into six parameters, referred to as knowledge sources; Language (LA), Narrative Strategy (NS), Target (TA), Situation (SI), Logical Mechanism (LM), and Script Opposition (SO). The GTVH is one of the dominant linguistic humour theories within the last two decades along with the Semantic-Script Theory of Humour \cite{attardo_routledge_2017}. As \textcite{shipley_towards_2007} notes, this theory is an incongruity theory with a significant impact on humour translation. 

\textcite{diaz-cintas_audiovisual_2007} stress that humour does not function on its own, but different types of communicative situations and sociocultural, linguistic and personal contexts affect the perception of humor. \textcite{zabalbeascoa_translating_1996,martinez-sierra_translating_2006,diaz-cintas_audiovisual_2007} outlined jokes as binational (or international), cultural-reference, community-sense-of-humor, pure visual and/or audio.  

According to \textcite{chiaro_verbally_2008}, different types of strategies for translating language-dependent humour are summarised. The strategies are as followed:

\begin{enumerate}[label=\roman*.]
    \item Leaving the pun unchanged, whereby the ST pun is directly duplicated into the TT
    \item Replacing the ST pun with a TL pun
    \item Replacing the ST pun with a rhetorical device
    \item The ST pun rendered as non-pun which causes the loss of effect of the pun
    \item Omission, whereby the pun is eliminated in the TT
    \item Compensation, whereby adding a TL pun in the adjacent text and no corresponding ST pun is found
    \item Inserting notes, to further explain the pun.
\end{enumerate}

Translation of language-dependent humour usually involves compromise since it is rare to find the same pun word in two different languages; hence, the tendency of rendering the text into a natural text with less literal accuracy is more common.

\subsubsection{Fansubbing}\label{sec-secoes}
Fansubs, fansubbing or subbing, are homemade subtitles for television series, films and cartoons that have not yet been broadcasted in the TL country \cite{diaz-cintas_subtitling:_2013}. Fansubbing is a popular phenomenon in audiovisual translation, “both because of the growing communities of people who enjoy foreign audiovisual products, and because the computer software for home subtitling is increasingly available and easy to use” \cite[p. 117]{chaume_turn_2013}. Fansubbing is more creative and individualistic and less dogmatic compared to traditional translation. Creative humour translation “helps the audiences to overcome the cultural barriers, experience cultural otherness and have great potential to be adopted by other translators in dealing with translating cultural difficulties” \cite[p. 438]{cai_fansubbing_2015}.  

As noted by \textcite{diaz-cintas_fansubs:_2006}, the participants involved in the process of fansubbing include the `raw providers', `translators', `timers', `typesetters', `editors', `proofreaders' and `encoders'. Raw providers provide the source material, which is the original, untranslated video capture of the shows. Then translators translate the subtitles from a particular language into another. Timers usually set the in-and-out-times of each subtitle; this process is also known as cueing and spotting. Later, typesetters define the fonts of the subtitles and editors and proofreaders are responsible for revising the translation to produce coherent and natural subtitles in the TL along with the correction of any possible typos. Encoders are those who use the raw material provided and the final script to produce the subtitled version of a given episode by using a software. The fansubbing process is described as:

\begin{quote}
    First, the episode raw is obtained and sent to the encoder to decide the usability of the material based on its quality of sound and image, then the material is sent to the translator. The translated script is timed with the audio and the typesetter chooses the fonts used. Last, the karaokes for opening and ending songs are usually done when translating the first episode and used every time the same song is included in subsequent episodes \cite[p. 39]{diaz-cintas_fansubs:_2006}. 
\end{quote}

Several features of the fansub include but are not limited to the use of different fonts throughout the same program, use of colors to identify different actors, use of subtitles of more than two lines, use of notes at the top of the screen, use of glosses in the body of the subtitles, the varying position of subtitles on the screen, karaoke subtitling for opening and ending songs, addition of information regarding the fansubbers, and the translation of opening and closing credits \cite{diaz-cintas_fansubs:_2006,chaume_turn_2013}. Mistakes are common in fansubbing due to the amateur nature of the translational practice. The mistakes arose from the confusion between the substantive in the original text and the adverb in the translation, as well as from the interference caused by an English cultural reference \cite{diaz-cintas_fansubs:_2006}.

Fansubbing includes the use of colors, position of subtitles on the screen, use of more characters than conventional subtitles, use of various fonts, and higher reading speeds. \textcite{cai_fansubbing_2015} pointed out that the fansub tends to rewrite and manipulate the translation in a more idiosyncratic and creative way. 

\subsection{Translation errors}\label{sec-format-simple}
Translation errors are non-equivalences between the source and target texts, or non-adequacy of the TT \cite{hansen_translation_2010}. They are the significant mismatches of denotational meaning between source and target texts or breaches of the target-language system. Translation errors are manifestation of a defect in any factors entering into the skills in translation. They are categorised into binary and non-binary errors, whereby binary errors happen with only one right solution, while non-binary errors involve at least two right answers and then the wrong ones \cite{pym_1992}, or the four categories of pragmatic translation errors, cultural translation errors, linguistic translation errors and text-specific translation errors \cite{nord_translation_2001}.

\section{Methods}\label{sec-links}
This qualitative study employs the translation approaches fansubbers used to preserve humour in the target culture and the problems that occurred when decisions were made on selecting related strategies. In this study, the subtitling strategies were analysed based on \textcite{gottlieb_1992} typology, where all ten strategies were used in the analysis to reach the research objective 1. The humour elements were analysed based on the General Theory of Verbal Humour \cite{attardo_script_1991}, whereby the six parameters were involved in the analysis to make it to the research objective 2. The similarity of humour elements was determined through the general theory of verbal humour proposed by \cite{attardo_script_1991} and the translation errors were analysed based on \textcite{nord_translation_2001} typology, where four types of translation errors were incorporated in the analysis to determine the preservation of function of the ST in TT and follow-up with the research objective 3.  The respective outputs were listed aside, and the data collected was analysed through the same sequence.

\subsection{Corpus}\label{sec-outras-estr}
The genre of \textit{The Marvelous Mrs. Maisel} is a sitcom. A sitcom, or situation comedy, “is a popular genre that presents fictional humorous social and cultural situations in the lives of the people” \cite[p.~53]{alharthi_challenges_2016}. \textit{The Marvelous Mrs. Maisel} was chosen as the corpus based on the following crucial characteristics of the series:

\begin{itemize}
    \item References to Jewish and American culture;
    \item The use of taboo topics;
    \item The utilization of stand-up comedy.
\end{itemize}

First, there are references to both Jewish and American cultures in \textit{The Marvelous Mrs. Maisel}, whereby they construct some of the humour in the show. These cultural references are used in the show to express humour. Second, the humour in \textit{The Marvelous Mrs. Maisel} is expressed in the use of taboo expressions, which are related to sex, swearing, dietary restrictions, etc. \textit{The Marvelous Mrs. Maisel} reflects the unique humorous features of the American language which both American and Jewish cultures could relate to. Translating humour into another culture is most certainly a complicated task, especially for the fansubbers who did not go through professional training \cite{alharthi_challenges_2016}. For the present study, 18 episodes were downloaded and eight episodes from season 1 were selected randomly. The subtitles were produced by the Lafeng fansub group.


\section{Findings}\label{sec-listas}

\subsection{Analysis of the subtitling strategies}

Among the eight episodes in season 1 of \textit{The Marvelous Mrs. Maisel}, the translation strategies used by fans in translating the humour from English to Chinese were identified. A total of eighteen samples with back translations in English are provided in this section.

\subsubsection{Expansion}\label{sec-figuras-tabelas}
Expansion is the strategy used to explain when the cultural nuances are not retrievable in the TL \cite{gottlieb_1992}. The analysis of data revealed that expansion was employed to explain some unknown concepts, to adapt to the Chinese language structure, and to strengthen the statements. In the example of \Cref{tbl01}, the fans tried to explain Modigliani, a famous artist. Therefore, they insert “paintings” after “Modigliani's”, whereby it introduces his identity of being an artist without disrupting the tempo of watching the show. 

\begin{table}[htpb]
\caption{Example NO. 1.}
\label{tbl01}
\begin{tabularx}{\linewidth}{XXX}
\toprule 
ST  & TT & Back Translation \\ 
\midrule
The people who buy the dish soap and the dog food, who pay for the Modigliani's. & 买香皂盒和狗粮的人会买莫迪里阿尼绘画作品的人  & People who buy soap dish, people who buy Modigliani's paintings. \\ 
\bottomrule
\end{tabularx}
\source{Own elaboration.}
\end{table}

Expansion is adopted to strengthen the statements to increase the “funniness”. The conversation in \Cref{tbl02} shows Abe's anger and impatience when listening to the 13 Jews' story from someone he hates so much. Hence, the fansubbers choose to use a stronger statement in the phrase “have no choice but to say (不得不说 [bu de bu shuo])” to increase the hostility of Abe against Moishe, which then rationalises Abe's behaviour criticizing Moishe later. 

\begin{table}[htpb]
\caption{Example NO. 2.}
\label{tbl02}
\begin{tabularx}{\linewidth}{XXX}
\toprule 
ST  & TT & Back Translation \\ 
\midrule
And I have to say, if he talks about getting those 13 Jews out of Germany one more time… & 我不得不说 如果他再谈起把那 13个犹太人弄出德国一次  & I have no choice but to say, if he talks about getting out those 13 Jews out of Germany one more time… \\ 
\bottomrule
\end{tabularx}
\source{Own elaboration.}
\end{table}

In the Chinese language, it sounds abnormal when the adjective is used after the noun without the adverb “很” [hen]. It is important to note that it does not have any actual meaning related to the degree of the expression in the sentence.  

\subsubsection{Paraphrase}
Paraphrasing is the strategy employed when the phraseology of the SL is different from TL \cite{gottlieb_1992}. This strategy is frequently used because of the different phraseology between the English and Chinese languages. 

The English language tends to formulate longer sentences, while the Chinese language tends to put the sentences into smaller segments. As in “What do you do if you see one?”, which is translated into “如果你看到一只蜘蛛 你会怎么办?” [ru guo ni kan dao yi zhi zhi zhu ni hui zen me ban] (If you see a spider, what will you do?), the sentence is broken into smaller segments to ease the understanding. Hence, paraphrasing is an essential strategy that is mostly used when translating English into the Chinese language. 

\subsubsection{Transfer}
Transferring is a translation strategy where the ST is transferred completely and accurately \cite{gottlieb_1992}. It is used when the sentence structures from different languages are similar. However, it is one of the strategies that leads to translation errors when there are differences between the languages in terms of syntactic structures.

\begin{table}[htpb]
\caption{Example NO. 3.}
\label{tbl03}
\begin{tabularx}{\linewidth}{XXX}
\toprule 
ST  & TT & Back Translation \\ 
\midrule
If my kids got kidnapped and I had to describe them, I'd have to say, “They look like kids.” “I-I don't know. The whosit's got a head. The other one's got a… head.” & 如果我孩子被绑架了而我必须描述他们 我只能说 他们看起来像小孩我 我不知道 有一个有头另一个也有头  & If my kids got kidnapped, and I have to describe them, I could only say, they look like kids.
I-I don't know, one of them has head, the other one has head. \\ 
\bottomrule
\end{tabularx}
\source{Own elaboration.}
\end{table}

The example in \Cref{tbl03} happens when Midge is performing a stand-up comedy. She did not have photos of her children, so when her children were kidnapped, she could only describe them both with uncertainty. Nonetheless, transfer was used as the strategy to translate this situation; the structure of the English language did not fit in with the Chinese language, but the structures were transferred literally. The unnatural narrative style showed the clumsiness of the TT that disturbs the audiences' watching experience, and it may reduce the conveyance of the humorous effect. The inappropriate use of Transfer may be due to the deficiencies of language proficiency of fansubbers that causes the failure in observing the structural differences between the languages. 

\subsubsection{Imitation}
Imitation is the strategy that allows the forms of ST to remain the same, typically used to translate the names of people and places \cite{gottlieb_1992}. The STs remain the same in the TL with a slight modification, which they are naturalised to adapt the phonological form of the TL. For instance, the name of a character, Petra, was translated into 佩特拉 [pei te la], which sounds similar to the ST. The characters' names were translated with imitation of the phonological elements. Imitation was used to highlight the unknown concept, such as “rabbi”, translated into 拉比, which is pronounced as `la bi' in Chinese. It means a teacher of the Torah in Judaism, whereas there is no such concept in Chinese culture. Through imitation, the concept of rabbi is introduced to the target audiences (TA). 

However, imitation may not be appropriate for translating celebrities' names or famous places, because the TA's background knowledge should be taken into account. Without introducing the proper nouns to the audience, it may not help in delivering the humorous effect. Translators are suggested to use imitation along with expansion. While transliterating the names of people and places, the insertion of a brief introduction may provide adequate information for the audiences to understand the humour and the source culture (SC). 

\subsubsection{Condensation}
Condensation is employed to “reduce the SL message without reducing its meaningful content” \cite[p. 168]{gottlieb_1992}. Condensation is observed when the ST contains irrelevant information to achieve the particular function of the text, such as (\Cref{tbl04}): 

\begin{table}[htpb]
\caption{Example NO. 4.}
\label{tbl04}
\begin{tabularx}{\linewidth}{XXX}
\toprule 
ST  & TT & Back Translation \\ 
\midrule
so you walk around on the verge of passing out  & 因此你就在昏厥的边缘  & so you are at the edge of passing out \\ 
\bottomrule
\end{tabularx}
\source{Own elaboration.}
\end{table}

The verb `walk around' is removed to make the utterance concise, because the focus is on the edge of passing out. Hence, the reduction is made in a reasonable way to enable the audiences to finish reading the subtitles in a limited time. 

Condensation is used when stammer occurs among the conversations. As the example shown below (\Cref{tbl05}), the `wh…' is removed to make the subtitles clear without simply putting the repetition into the subtitles.

\begin{table}[htpb]
\caption{Example NO. 5.}
\label{tbl05}
\begin{tabularx}{\linewidth}{XXX}
\toprule 
ST  & TT & Back Translation \\ 
\midrule
Wh…Where the hell are we?  & 我们到底在哪里  & Where exactly are we? \\ 
\bottomrule
\end{tabularx}
\source{Own elaboration.}
\end{table}

\subsubsection{Decimation}

Decimation is an extreme type of condensation to adjust to the discourse speed, which may be accompanied by the omission of potentially important elements \cite{gottlieb_1992}. As in the example below (\Cref{tbl06}), `the other way' (implying breastfeeding) is substituted by `what', which is simplified and shortened. The use of this strategy makes the translation vague, so that it may lead to confusion in the process of watching the show. 

\begin{table}[htpb]
\caption{Example NO. 6.}
\label{tbl06}
\begin{tabularx}{\linewidth}{XXX}
\toprule 
ST  & TT & Back Translation \\ 
\midrule
“How you like to be fed, by the bottle or by the other way?”  & “他们怎么喂你的?用瓶子还是什么呢?”  & “How they feed you? Using 
bottle or what?” \\
\bottomrule
\end{tabularx}
\source{Own elaboration.}
\end{table}

Decimation may confuse audiences when the key message is omitted and hence leads to pragmatic translation errors.  

\subsubsection{Deletion}
Deletion eliminates parts of a text \cite{gottlieb_1992}. It was used to remove the interjections and taboo words within the conversations. The example below (\Cref{tbl07}) shows the deletion applied to illustrate the hesitation of the characters to adapt to the discourse speed and to make it concise to provide smooth (to follow) subtitles.  

\begin{table}[htpb]
\caption{Example NO. 7.}
\label{tbl07}
\begin{tabularx}{\linewidth}{XXX}
\toprule 
ST  & TT & Back Translation \\ 
\midrule
“So um why are we at war?” 
“Uh because it's 1812.”  & 我们为什么处在战争中啊?因为是 1812 年  & Why are we in the war? Because it's 1812. \\
\bottomrule
\end{tabularx}
\source{Own elaboration.}
\end{table}

The appropriate use of deletion can effectively make the TT concise without losing the meaning, but it may lead to the loss of the sarcastic nuance in the conversation. 

\subsubsection{Resignation}

Resignation is employed when the meaning is inevitably lost and there is no translation solution \cite{gottlieb_1992}. Only three examples were found in applying resignation (\Cref{tbl08}). The loss of meaning and humour was caused because of the misunderstanding of the meaning and ST context.  

\begin{table}[htpb]
\caption{Example NO. 8.}
\label{tbl08}
\begin{tabularx}{\linewidth}{XXX}
\toprule 
ST  & TT & Back Translation \\ 
\midrule
Do you know I've seen her twice with her shirt on inside out? 
Penny. Twice. 
Once, fine. You were rushed in the morning. 
Twice, you can only be trusted to butter people's cornet the county fair.  & 你知道我看过她两次衣服穿反的么潘妮 两次 
一次 还可以 你早上着急 
两次 你就是天天相信这些无聊人  & Do you know I've seen her twice with her shirt on inside out? Penny. Twice. 
Once, it's okay.  
You were rushed in the morning. Twice, you just believe these boring people every day. \\ 
\bottomrule
\end{tabularx}
\source{Own elaboration.}
\end{table}

Here, without understanding the message intended to be delivered, the TT is translated with resignation, while the sentence “you can only be trusted to butter people's cornet in the county fair” is translated into a different sentence “你就是天天相信这些无聊人” [ni jiu shi tian tian xiang xin zhe xie wu liao ren] (you just believe these boring people every day). The meaning of the ST is twisted, and humour is lost in the translation. 

\subsection{Subtitling strategies}
According to \Cref{tbl09}, paraphrasing was the most used translation strategy among the others (28\%), followed by transfer (23\%), and imitation (16\%). Decimation and resignation were the least used strategies (3\%). Transcription and dislocation were employed in the translation. 

\begin{table}[htpb]
\caption{Frequency and percentage of subtitling strategies}
\label{tbl09}
\begin{tabularx}{\linewidth}{XXX}
\toprule 
Subtitling strategies  & Frequency & Percentage \\ 
\midrule
Expansion & 19 & 12\% \\
Paraphrase & 44 & 28\% \\
Transfer & 37 & 23\% \\
Imitation & 25 & 16\% \\
Transcription & 0  & 0\% \\
Dislocation & 0  & 0\% \\
Condensation & 15  & 9\% \\
Decimation & 3  & 2\% \\
Deletion & 12  & 8\% \\
Resignation & 3  & 2\% \\
\bottomrule
\end{tabularx}
\source{Own elaboration.}
\end{table}

\subsection{Analysis of humour elements}

With the application of GTVH, the humour instances were analysed in terms of humour elements and knowledge resources. It includes six parameters in each joke; script opposition (SO), logical mechanism (LM), situation (SI), target (TA), narrative strategy (NS) and language (LA). The first five parameters are shown in the \Cref{tbl10} to indicate the similarity between the ST and TT. 

\begin{table}[htpb]
\caption{Example NO. 10.}
\label{tbl10}
\begin{tabularx}{\linewidth}{XXX}
\toprule 
ST  & TT & Back Translation \\ 
\midrule
That roll of pink tulle you're dragging across the ground… 
It's French  
Do you know what else is French? 
The guillotine!  & 你铺在地上的那卷粉红的薄纱 
是法国货  
你知道法国还有什么吗断头台  &You laid on the floor that roll of pink tulle is French. 
You know France has what else? 
Guillotine. \\ 
\bottomrule
\end{tabularx}
\source{Own elaboration.}
\end{table}

The script opposition is expected and unexpected, as the question is “Do you know what else is French?” and, given that the previous conversation is related to clothes, it is expected that the answer is associated with clothes. However, it is unexpected when Moishe answers “guillotine is French”, which shows his anger towards the employee who dragged the tulle across the ground. The logical mechanism of the text is Juxtaposition to show the contrast between them - in this case, the clothes and guillotine. The situation refers to the circumstance in which that humour instance is produced - Moishe is scorning his employee, who drags the tulle across the ground. The target of the humour instance is the employee. The narrative strategy is dialogue, whereas the humour instance is presented in the way of conversation. 

Every linguistic element was transferred into the TT. It involved paraphrasing, such as “Do you know what else is French?” is translated into “你知道法国还有什么吗” [ni zhi dao fa guo hai you shen me ma] (You know France has what else), whereby it sounds more natural in the Chinese language. Most of the parameters are similar, and the humour is delivered in the TC (\Cref{tbl11}).  

\begin{table}[htpb]
\caption{Example NO. 11.}
\label{tbl11}
\begin{tabularx}{\linewidth}{XXX}
\toprule 
ST  & TT & Back Translation \\ 
\midrule
I used to live in the large house on the hill. 
And now I live nowhere. As of an hour ago, my address changed to “huh?”  & 我曾经住在山上的一所大房子里 
但现在我哪儿都没住 
就在一个小时前我的住址变成了“嗯?”  & I used to live on the hill in the large house. 
But now I live nowhere. 
Just at one hour ago, my address changed to “Hmm?” \\ 
\bottomrule
\end{tabularx}
\source{Own elaboration.}
\end{table}

However, in some of the excerpts, although GTVH enables the translators to produce the translations that are highly similar in terms of the parameters, the function of the text, humour, is not preserved in the TT (\Cref{tbl12}). 

\begin{table}[htpb]
\caption{Example NO. 12.}
\label{tbl12}
\begin{tabularx}{\linewidth}{XXX}
\toprule 
ST  & TT & Back Translation \\ 
\midrule
First, you stand me up. 
Then you Bataan Death March me through Buckingham 
Palace. 
You make me hold your kid's 
filthy hand,  
Which means I probably got 14 different kinds of cholera right now.  & 首先 你放了我鸽子然后带来了次巴丹死亡行军 
在你的白金汉宫内 
你让我牵着你孩子脏兮兮的小手 
这意味着我可能会得十四种不同的流行病   & First, you release my pigeon. 
Then bring one time of Bataan Death March, in your 
Buckingham Palace. 
You make me hold your kid's dirty little hand.  
This means that I may get 14 different types of cholera. \\ 
\bottomrule
\end{tabularx}
\source{Own elaboration.}
\end{table}

In this example, the logical mechanism is exaggeration. Susie exaggerates that only by holding the kid's hand, she will probably get cholera immediately. The context for the humour instance is Susie nagging about Midge “standing her up” early that morning. The target of the humour instance is Midge and her son, whereas Susie is attacking the dirtiness of the kid and Midge's failure to keep the appointment. It is narrated in the way of a dialogue. 

All linguistic elements were transferred to the TT, whereby the phrase “stand me up” was translated into “放了我鸽子” [fang le wo ge zi] (release my pigeon), and both mean failing to keep an appointment. However, the humorous effect was not delivered to the TT receivers. It may be due to the unfamiliarity with the historical and cultural background of the conversation, such as Bataan Death March and Buckingham Palace - without knowing those proper nouns, the audiences may not understand the humour (\Cref{tbl13}).

\begin{table}[htpb]
\caption{Example NO. 13.}
\label{tbl13}
\begin{tabularx}{\linewidth}{XXX}
\toprule 
ST  & TT & Back Translation \\ 
\midrule
And I have to say, if he talks about getting those 13 Jews out 
of Germany one more time… 
“Abe…” 
He brought them here and stuck them in his factory. 
They're all working there. 
Is he paying these poor people? Are there toilets for them? I've seen their faces. 
I can't be sure of this, but one of them has a look like, “I should have taken my chances back in 
Germany.”  & 我不得不说 如果他再谈起把那 13个犹太人弄出德国一次 
“亚伯” 
他把他们带到这儿 把他们捆在这个工厂他们都在那工作他付钱给这些穷人吗? 
他们有卫生间吗 我见过他们的脸 
我不确定但其中的一个有种表情 像在说 
“有机会真想回德国” & I couldn't stand not to say, if he talks about getting out those 13 Jews out of Germany one more time…
“Abe.”
He brought them here, stuck them in this factory. They're working there. Does he pay these poor people? Do they have washroom? I've seen their faces.
I'm not sure but one of them has the expression, like “If there is chance, I would want to be back to Germany.” \\ 
\bottomrule
\end{tabularx}
\source{Own elaboration.}
\end{table}

In this example, the script opposition is “actual” and “non-actual”. The logical mechanism used is role exchanges while Abe plays the role of the worker for a short time to show Moishe how he imagined his reaction. The context is Abe complaining about Moishe repeating the same story. The target of the humour instance is Moishe, who is being attacked by Abe due to his vicious behaviors towards his employees. The humour instance is narrated in the way of a dialogue. 

The linguistic elements were transferred to TT. “Poor people” was literally translated into “穷人” [qiong ren], i.e., the person who does not have enough money. Nonetheless, “poor people” here means the people who are pitiful and wretched. Therefore, there is little consistency between the referential and pragmatic meaning, because the word is transmitted to the TT but does not deliver the intended message. This text intends to show the disgust of Abe towards Moishe, since it describes how he tortures his workers and shows off his courage to rescue them from Germany. Moreover, the sentence “I should have taken my chances back in Germany” further reinforces the vicious Moishe, who makes the employee want to go back to Germany, a place where Jews were persecuted during World War II (\Cref{tbl14}).  

\begin{table}[htpb]
\caption{Example NO. 14.}
\label{tbl14}
\begin{tabularx}{\linewidth}{XXX}
\toprule 
ST  & TT & Back Translation \\ 
\midrule
Our little girl is looking more and 	more like Winston Churchill every day, you know, with that big Yalta head. 
But that's not a reason to leave, right?  & 我们的小姑娘一天比一天长得像丘吉尔了每天 那个大雅尔塔脑袋但那不是离开的原因不是么 & Our little girl day by day growing like Churchill every day, that big Yalta head, but that's not a reason to leave, isn't it? \\ 
\bottomrule
\end{tabularx}
\source{Own elaboration.}
\end{table}

The script opposition is “expected” and “unexpected”. It is expected that the little girl's head is growing bigger day by day as it is previously mentioned in the show, but, unexpectedly, Midge relates her daughter's bigger head to Joel's leaving. The logical mechanism is a false analogy, which means two objects that seem to be similar are put together but the comparison does not hold up - in the case, Midge's daughter is compared with Winston Churchill due to the big head, which makes the humour instance even more ridiculous. Midge's daughter is attacked due to her big head and Joel is attacked due to the affair. It is narrated in the way of a dialogue. 

The language parameter in this joke is completely transferred, while imitation is used to transfer the words such as Winston Churchill and Yalta. They are transliterated into Chinese and the pronunciation sounds similar. Nonetheless, the target receivers may not find the joke funny if they do not understand who Winston Churchill is and/or what Yalta is. Despite all six parameters remaining the same in the two languages, the funniness of the joke was not maintained in the ST.  

Although GTVH can be used to ensure each of the knowledge resources remains in the TT, it does not ensure the conveyance of the humorous effect. Hence, it leads to the emergence of translation errors especially when the translation is related to source culture. 

\subsection{Analysis of translation errors}

If the purpose of translation is to achieve a specific function for the target audience, anything that obstructs this purpose is a translation error \cite[p. 74]{nord_translation_2001}. In T\textit{he Marvelous Mrs. Maisel,} the translation aims to achieve a humorous effect among the Chinese audience - hence, the failure of maintaining the funniness of the ST in TT is considered a translation error. \textcite{nord_text_1997} categorized them into four categories namely `pragmatic', `cultural', `linguistics', and `text-specific' translation errors, and they are discussed accordingly in the next section.

\subsubsection{Pragmatic translation errors}

Pragmatic translation errors happen due to inadequate solutions to pragmatic translation problems such as the lack of receiver orientations \cite{rahmatillah_translation_2013, yap_problems_2018}. The inconsistency between the referential meaning of the word or phrase and the pragmatic meaning may cause pragmatic translation errors \cite{zheng_approach_2012}. 

It is further categorised into two situations. The first one happens when insignificant, redundant, or irrelevant information in the ST is removed. Here, the term “condensed” means the shortening of the text, and “removed” means the deletion of the text. Hence, the particular function of the text was not achieved and pragmatic translation errors were made (\Cref{tbl15}).

\begin{table}[htpb]
\caption{Example NO. 15.}
\label{tbl15}
\begin{tabularx}{\linewidth}{XXX}
\toprule 
ST  & TT & Back Translation \\ 
\midrule
Who is this? &  你是谁? & Who are you?  \\

Well, 	\underline{I…}I 	told 	you, ma'am, I'm a, I'm a friend of Midge's.  &
\underline{嗯 我}告诉你我是麦琪的一个朋友  & 
Hmm, I tell you 
I am Midge's friend. \\ 
\bottomrule
\end{tabularx}
\source{Own elaboration.}
\end{table}

In the example above, the interjections in the conversation are not completely transferred into the TT. However, those interjections do not help in delivering funniness to the receivers. Therefore, removing interjections such as “uh” is suggested if they do not necessarily play a crucial role in achieving the function of the text, which is to make the audience laugh. 

The second situation happens when significant, relevant, or potentially important implied information in the ST is condensed or removed. Here, the term “condensed” refers to the shortening of the text, while “removed” refers to the deletion of the text. The translators remove the necessary information that is related to the achievement of particular functions of the text. With the removal, the pragmatic translation error is made due to the omission of important information. 

\subsubsection{Cultural translation errors}
Cultural translation errors are caused by “an inadequate decision concerning the reproduction or adaptation of culture-specific conventions” \cite[p. 177]{nord_text_1997}. It relates to the translation of cultural-specific expressions or terms. The use of proper nouns of celebrities may be considered as related to the cultural background (\Cref{tbl16}). 

\begin{table}[htpb]
\caption{Example NO. 16.}
\label{tbl16}
\begin{tabularx}{\linewidth}{XXX}
\toprule 
ST  & TT & Back Translation \\ 
\midrule
Who is this? & 你是谁?& Who are you? \\

Well, \underline{I…I told you}, ma'am, I'm a, I'm a friend of Midge's. & 嗯 \underline{我告诉你}我是麦琪的一个朋友 & Hmm, I tell you I am Midge's friend. \\

Your name. What's your name? & 你的名字 你叫什么 & Your name. What's your name? \\

Uh… Carol.  & 嗯 考尔  & Uh… Carol.  \\

\underline{You had to think about it?} Janet? & \underline{你想过么}简妮特么稍 & You thought about it? Janet?  \\

Wait. Do you not know who you are? & 等 你不知道你是谁么 & Wait. You don't know who you are? \\

\underline{Not at the moment}.  &  \underline{不是一会儿}  & Not for a while. \\ 
\bottomrule
\end{tabularx}
\source{Own elaboration.}
\end{table}

As for cultural translation errors, the insertion of additional explanations or substitution of the cultural-specific terms with similar terms in the TC is suggested to help the receivers understand the culture-specific jokes.

\subsubsection{Linguistic translation errors}
Linguistic translation errors are often caused by deficiencies of the translator's language competence \cite{rahmatillah_translation_2013} (\Cref{tbl17}).

\begin{table}[htpb]
\caption{Example NO. 17.}
\label{tbl17}
\begin{tabularx}{\linewidth}{XXX}
\toprule 
ST  & TT & Back Translation \\ 
\midrule
-Women aren't funny. & -女人没有兴趣 &  Women have no interest.  \\

-Your wife must have a sense of humour.  & -你妻子一定有幽默感她看 & Your wife must have a sense of humour.  \\

She's seen you naked.  & 
见你裸着 & She sees you naked. \\ 
\bottomrule
\end{tabularx}
\source{Own elaboration.}
\end{table}

When the choice of the wrong vocabulary causes failure to deliver the intended humorous effect, it is considered a linguistic translation error. There are two inadequate uses of Chinese vocabulary in the \Cref{tbl18}.

\begin{table}[htpb]
\caption{Example NO. 18.}
\label{tbl18}
\begin{tabularx}{\linewidth}{XXX}
\toprule 
ST  & TT & Back Translation \\ 
\midrule
Is this your baby? & 这是你的宝宝吗? & Is this your baby? \\

“He \underline{thinks} it is.” \underline{Shut up}, man.  & 他\underline{觉得}是\underline{别说话} &  He \underline{feels} it is. \underline{Don't talk}. \\ 
\bottomrule
\end{tabularx}
\source{Own elaboration.}
\end{table}

The underlying causes of linguistic translation errors may be the translators' deficiencies regarding their language competence in both languages and the insufficient interpretation of the implied meaning. Linguistic translation errors can be avoided through the application of expansion and substitution of relevant or similar culture-specific terms. 

\subsubsection{Text-specific translation errors}
Text-specific translation errors are “associated with a text-specific translation problem and, in the case of the corresponding translation problems, can usually be evaluated from a functional or pragmatic” point of view \cite[p. 127]{nord_text_1997}. The intended function of the particular text types should be given priority over the other functions in the translation. Hence, in humour translation, the achievement of the humorous effect should be put at the forefront of all elements. It may be associated with the translation of interjections, whereby the meaning of the interjections is not produced as they are in the ST (\Cref{tbl19}). 

\begin{table}[htpb]
\caption{Example NO. 19.}
\label{tbl19}
\begin{tabularx}{\linewidth}{XXX}
\toprule 
ST  & TT & Back Translation \\ 
\midrule
And \underline{yes}, there is shrimp in the egg rolls. \underline{Shut up}, man.  & 还有\underline{对了} 鸡蛋卷里有虾 & And \underline{right}, egg rolls inside have shrimp. \\ 
\bottomrule
\end{tabularx}
\source{Own elaboration.}
\end{table}

The text-specific translation errors were found when the vulgar word used to express the strength of the statement was deleted.

\section{Conclusions}

Based on \textcite{gottlieb_1992} typology, the subtitling strategies employed by fans were analysed. The most often used strategy was paraphrasing, while the least used strategies were decimation and resignation. Transcription and dislocation were not found in the fansub translation of \textit{The Marvelous Mrs. Maisel}. The dominating subtitling strategies were paraphrasing, transfer, and imitation. The fansub tended to transfer the ST to the TT literally, although, with the use of paraphrasing, the unnaturalness of the TT could be observed because the TT was not adapted to the TL norms. 

The cultural proper nouns were not translated with further explanations but transliterated with the use of Imitation. The low frequency of expansion among the subtitling strategies can be explained by the technical constraints of subtitling, specifically the limitation of time and space. There were only limited characters that can be displayed in the subtitles, which restrained the information the translators could provide. Besides, there is only limited time for fansubbers to translate the subtitles because they need to publish their work on the internet as fast as possible to gain more views. Hence, the technical constraints of subtitling may be one of the possible reasons for the low frequency of expansion, which might lead to the loss of humor. While transfer and imitation were frequently used in fansub translation, the subtitles were ST-oriented because the two subtitling strategies involved some literal translation. 

The analysis of the present study showed that fansub provided a more conservative translation of humor, including the avoidance of taboo and vulgar words and the tendency to transfer the ST literally into TT. Such provision of conservative translations indicates the failure of employing appropriate subtitling strategies and adapting to the Chinese culture and language. The inappropriate use of subtitling strategies does not only lead to translation errors but may lead to loss of meaning. For instance, decimation and deletion may involve the pragmatic translation errors caused by the failure to identify relevant information and further eliminating the information. As supported by \textcite[p. 77]{kianbakht_audiovisual_2015}, “the lack of scientific methods and mastery over subtitling strategies would not allow translators to follow straightforward guidelines in their course of translating humour”. Furthermore, English-Chinese translation tends to be ST-oriented in terms of the use of subtitling strategies \cite{cheng_chinese_2014}. The familiarity of the TA with the SC determines how the ST is translated, whereas the assumption that the TA has the knowledge related to the SC may cause translators not to further explain the context in the translation. The China government controls the import of foreign cultural products. The audiences in China has limited access to the foreign cultures so that the subtitling strategies that are more TT-oriented are more suitable in the Chinese context \cite{hsieh_translation_2010}. 

The main purpose of humour translation is to preserve the funniness rather than create a similar text in the TT \cite{zabalbeascoa_humor_2005}. Although GTVH helps to measure the similarity between the ST and TT, it does not entail the conveyance of humour in the translations, which was supported by the finding of the present study. Humour translation does not only require high proficiency in both languages but requires the translators to research both SC and TC, to achieve the humorous effect in TT.  

In the present study, linguistic translation errors were the most common type of errors in the fansub \textit{The Marvelous Mrs. Maisel}. This could be due to the deficiencies in the language competence of the translators. Pragmatic translation errors were found when the relevant information was removed or when the irrelevant information was not removed, such as when the unnecessary interjections among the sentence were not deleted causing redundancy in subtitles. Pragmatic translation errors can be avoided by interpreting the implied meaning instead of only understanding the referential meaning and identifying the function of the text. The inadequate choice of vocabulary and syntactic structure was another identified problem, which could be solved by the use of paraphrasing. The text-specific translation errors were observed when the function of the text was not prioritized. Text-specific translation errors could be solved by observing the intended function of the text and giving priority to the function(s) in the TT. To avoid text-specific translation errors, the translators are encouraged to acknowledge the main function of the text to further decide the use of subtitling strategies before starting translation.  

On the other hand, cultural translation errors were identified when inadequate solutions were used to translate cultural-specific conventions, such as when the names of celebrities in the joke were not further explained. Cultural translation errors could be resolved by providing useful information to decode the humour in the ST (expansion) and substitution of similar concepts in the TC, which helps the TA to understand the joke more easily. 

It could be concluded that the translation errors made in the fansubbing of \textit{The Marvelous Mrs. Maisel} were mainly due to the lack of language proficiency and professional training. Lutviana and Subiyanto (2012) state that the failure of translating humour may be due to errors and the inappropriate application of translation strategies, which implies the lack of professional training of the fansubbers. The other explanation for the translation errors could be the inadequate cultural knowledge of the source and target culture (TC) as also supported by \textcite{kianbakht_audiovisual_2015}. The restriction caused by technical constraints is the other possible reason for translation errors. The translators have to take different aspects of cultural references, extra-textuality, intertextuality, and the priority of humor. The finding related to GTVH suggests that the theory has to be expanded to preserve the conveyance of humor. 

The analysis of translation errors could help the translator trainers to acknowledge the possible translation errors made by the trainees and design the syllabus to strengthen such weaknesses.  Moreover, film producers could use the findings as a sample and/or framework for future productions. For example, film producers may use the findings to examine the work of subtitlers and evaluate the quality of translation. 

Finally, the identification of translation errors may enable translation trainers to use audiovisual materials that involve humour as teaching material to attract their students' attention. Future researchers are suggested to put the effort in further specifying translation errors - linguistic translation errors can be examined from semantic and syntactic aspects. By specifying the translation errors, future researchers can provide solutions that are specific to a particular kind of error. 


\printbibliography\label{sec-bib}
% if the text is not in Portuguese, it might be necessary to use the code below instead to print the correct ABNT abbreviations [s.n.], [s.l.]
%\begin{portuguese}
%\printbibliography[title={Bibliography}]
%\end{portuguese}

\begin{contributors}[sec-contributors]
\authorcontribution{Yee Han Tee}[conceptualization,datacuration,formalanalysis,methodology,projadm,resources,visualization,writing]
\authorcontribution{Mansour Amini}[conceptualization,datacuration,formalanalysis,methodology,projadm,supervision,resources,visualization,writing]
\authorcontribution{Ching Sin Siau}[investigation,resources,validation,review]
\authorcontribution{Amin Amirdabbaghian}[investigation,projadm,resources,supervision,validation,visualization,writing,review]
\end{contributors}

\end{document}

