% !TEX TS-program = XeLaTeX
% use the following command:
% all document files must be coded in UTF-8
\documentclass[english]{textolivre}
% build HTML with: make4ht -e build.lua -c textolivre.cfg -x -u article "fn-in,svg,pic-align"

\journalname{Texto Livre}
\thevolume{15}
%\thenumber{1} % old template
\theyear{2022}
\receiveddate{\DTMdisplaydate{2022}{7}{17}{-1}} % YYYY MM DD
\accepteddate{\DTMdisplaydate{2022}{8}{5}{-1}}
\publisheddate{\DTMdisplaydate{2022}{9}{28}{-1}}
\corrauthor{Inmaculada Ávalos Ruiz}
\articledoi{10.35699/1983-3652.2022.40508}
%\articleid{NNNN} % if the article ID is not the last 5 numbers of its DOI, provide it using \articleid{} commmand 
% list of available sesscions in the journal: articles, dossier, reports, essays, reviews, interviews, editorial
\articlesessionname{dossier}
\runningauthor{Ávalos Ruiz et al.} 
%\editorname{Leonardo Araújo} % old template
\sectioneditorname{Daniervelin Pereira}
\layouteditorname{Leonado Araújo}

\title{Use of ICT and social networks in the neurodevelopment of minors. Implications for the prevention of the risk of social exclusion}
\othertitle{Utilização das TIC e redes sociais no neurodesenvolvimento de menores. Implicações para a prevenção do risco de exclusão social}
% if there is a third language title, add here:
%\othertitle{Artikelvorlage zur Einreichung beim Texto Livre Journal}

\author[1]{Inmaculada Ávalos Ruiz \orcid{0000-0003-0809-7727} \thanks{Email: \href{mailto:iavalos@ujaen.es}{iavalos@ujaen.es}}}
\author[2]{Mercedes Cuevas López \orcid{0000-0002-5814-2285} \thanks{Email: \href{mailto:mmcuevas@ugr.es}{mmcuevas@ugr.es}}}
\author[3]{Emilio Jesús Lizarte Simón \orcid{0000-0002-9964-7271} \thanks{Email: \href{mailto:elizarte@ugr.es}{elizarte@ugr.es}}}
\author[3]{Slava López Rodríguez \orcid{0000-0002-9360-3518} \thanks{Email: \href{mailto:slavalr@ugr.es}{slavalr@ugr.es}}}
\affil[1]{Universidad de Jaén, Facultad de Humanidades y Ciencias de la Educación, Departamento de Pedagogía, Jaén, España.}
\affil[2]{Universidad de Granada, Facultad de Educación, Economía y Tecnología de Ceuta, Departamento de Didáctica y Organización Escolar, Ceuta, España.}
\affil[3]{Universidad de Granada, Facultad de Ciencias de la Educación, Departamento de Didáctica de la Lengua y la Literatura, Granada, España.}


\addbibresource{article.bib}
% use biber instead of bibtex
% $ biber article

% used to create dummy text for the template file
\definecolor{dark-gray}{gray}{0.35} % color used to display dummy texts
\usepackage{lipsum}
\SetLipsumParListSurrounders{\colorlet{oldcolor}{.}\color{dark-gray}}{\color{oldcolor}}

% used here only to provide the XeLaTeX and BibTeX logos
\usepackage{hologo}

% if you use multirows in a table, include the multirow package
\usepackage{multirow}

% provides sidewaysfigure environment
\usepackage{rotating}

% CUSTOM EPIGRAPH - BEGIN 
%%% https://tex.stackexchange.com/questions/193178/specific-epigraph-style
\usepackage{epigraph}
\renewcommand\textflush{flushright}
\makeatletter
\newlength\epitextskip
\pretocmd{\@epitext}{\em}{}{}
\apptocmd{\@epitext}{\em}{}{}
\patchcmd{\epigraph}{\@epitext{#1}\\}{\@epitext{#1}\\[\epitextskip]}{}{}
\makeatother
\setlength\epigraphrule{0pt}
\setlength\epitextskip{0.5ex}
\setlength\epigraphwidth{.7\textwidth}
% CUSTOM EPIGRAPH - END

% LANGUAGE - BEGIN
% ARABIC
% for languages that use special fonts, you must provide the typeface that will be used
% \setotherlanguage{arabic}
% \newfontfamily\arabicfont[Script=Arabic]{Amiri}
% \newfontfamily\arabicfontsf[Script=Arabic]{Amiri}
% \newfontfamily\arabicfonttt[Script=Arabic]{Amiri}
%
% in the article, to add arabic text use: \textlang{arabic}{ ... }
%
% RUSSIAN
% for russian text we also need to define fonts with support for Cyrillic script
% \usepackage{fontspec}
% \setotherlanguage{russian}
% \newfontfamily\cyrillicfont{Times New Roman}
% \newfontfamily\cyrillicfontsf{Times New Roman}[Script=Cyrillic]
% \newfontfamily\cyrillicfonttt{Times New Roman}[Script=Cyrillic]
%
% in the text use \begin{russian} ... \end{russian}
% LANGUAGE - END

% EMOJIS - BEGIN
% to use emoticons in your manuscript
% https://stackoverflow.com/questions/190145/how-to-insert-emoticons-in-latex/57076064
% using font Symbola, which has full support
% the font may be downloaded at:
% https://dn-works.com/ufas/
% add to preamble:
% \newfontfamily\Symbola{Symbola}
% in the text use:
% {\Symbola }
% EMOJIS - END

% LABEL REFERENCE TO DESCRIPTIVE LIST - BEGIN
% reference itens in a descriptive list using their labels instead of numbers
% insert the code below in the preambule:
%\makeatletter
%\let\orgdescriptionlabel\descriptionlabel
%\renewcommand*{\descriptionlabel}[1]{%
%  \let\orglabel\label
%  \let\label\@gobble
%  \phantomsection
%  \edef\@currentlabel{#1\unskip}%
%  \let\label\orglabel
%  \orgdescriptionlabel{#1}%
%}
%\makeatother
%
% in your document, use as illustraded here:
%\begin{description}
%  \item[first\label{itm1}] this is only an example;
%  % ...  add more items
%\end{description}
% LABEL REFERENCE TO DESCRIPTIVE LIST - END


% add line numbers for submission
%\usepackage{lineno}
%\linenumbers

\begin{document}
\maketitle

\begin{polyabstract}
\begin{abstract}
The presence of technologies in all areas and stages of life increases the risk of addiction to devices, networks, and the use of various programmes. Excessive use of devices and technologies during childhood and adolescence has negative effects on neurodevelopment and causes emotional disorders related to anxiety, stress, and depression. These disorders can be associated with the development of other addictions (alcohol, psychoactive substances) or other situations of risk of social exclusion such as bullying, among others. In our study of the risk of social exclusion in adolescents \cite{avalos2022} we have applied a complete diagnostic battery to 25 subjects of both sexes between 12 and 17 years of age focusing on six dimensions of risk: anxiety and depression, alcohol and other substance abuse, bullying, addiction to technologies, gender violence and risk behaviour in love and sexual relationships, and difficult family situation. The aim of this paper is to present and analyse the correlations found between each of the first three dimensions, and the technology addiction dimension as a predictor of other risk factors for social exclusion. The data show that addiction to video games, in particular, is related to high levels of depression, while addiction to any of the technologies studied (mobile, internet and video games) is significantly related to being a victim of bullying.

\keywords{Influence of technology \sep Neuropsychology \sep Mental health \sep Substance Abuse \sep Bullying}
\end{abstract}

\begin{portuguese}
\begin{abstract}
A presença de tecnologias em todas as áreas e fases da vida aumenta o risco de dependência de dispositivos, redes e uso de diversos programas. O uso excessivo de dispositivos e tecnologias durante a infância e adolescência tem efeitos negativos no neurodesenvolvimento e provoca transtornos emocionais relacionados à ansiedade, estresse e depressão. Esses transtornos podem estar associados ao desenvolvimento de outros vícios (álcool, substâncias psicoativas) ou outras situações de risco de exclusão social como \textit{bullying}, entre outras. Em nosso estudo sobre o risco de exclusão social em adolescentes \cite{avalos2022} aplicamos uma bateria diagnóstica completa a 25 sujeitos de ambos os sexos entre 12 e 17 anos de idade enfocando seis dimensões de risco: ansiedade e depressão, álcool e outras abuso de substâncias, bullying, dependência de tecnologias, violência de gênero e comportamento de risco nas relações amorosas e sexuais e situação familiar difícil. O objetivo deste artigo é apresentar e analisar as correlações encontradas entre cada uma das três primeiras dimensões, e a dimensão dependência de tecnologia como preditora de outros fatores de risco para exclusão social. Os dados mostram que o vício em videogames, em particular, está relacionado a altos níveis de depressão, enquanto o vício em qualquer uma das tecnologias estudadas (celular, internet e videogames) está significativamente relacionado a ser vítima de \textit{bullying}.

\keywords{Influência da tecnologia \sep Neuropsicologia \sep Saúde mental \sep Abuso de substâncias \sep Assédio moral}
\end{abstract}
\end{portuguese}
% if there is another abstract, insert it here using the same scheme
\end{polyabstract}

\section{Introduction}\label{sec-intro}
Technology has evolved and shaped our lives in many ways. We currently live surrounded by technology and screens, especially the youngest and from an increasingly early age. This brings us numerous benefits in different aspects, such as education or in bringing us closer to our loved ones who live elsewhere, but it is necessary to reflect on the effects that the excessive use or misuse of technology can have on the neurodevelopment of children and young people. According to \textcite{matali2021}, young people are the age group that lives most connected to the Internet and this fact means that we should pay attention to their behaviour towards technology, since the changes that occur during adolescence are not only physical or behavioural, but the brain reorganises itself structurally, creating new connections and eliminating others that are no longer useful. These changes cause the brain to be unstable during this period, resulting in a higher level of vulnerability to certain environmental or sociocultural stimuli that should be paid attention to \cite{russi2021}, as they can lead to chronic conditions, as indicated by \textcite{galan2017abordaje}. %Galán-López, Lascarez-Martínez, Gómez-Tello and Galicia-Alvarado (2017).

Following the \textcite{american2014dsm}, % DSM-5 (AMERICAN PSYCHIATRIC ASSOCIATION, 2014),
neurodevelopmental disorders include intellectual disabilities, communication disorders, autism spectrum disorder, attention deficit hyperactivity disorder, specific learning disorder, motor disorders, tic disorders and other disorders. The most frequently occurring problems related to neurodevelopment, according to \textcite{galan2017abordaje}, %Galán-López, Lascarez-Martínez, Gómez-Tello and Galicia-Alvarado (2017), 
are depression, anxiety and behavioural disorders, which in turn can lead to substance use and abuse, delinquency, suicidal acts and unwanted pregnancies. \textcite{dealvarado2019trastornos} %De Alvarado (2019)
points out that some of these problems may be caused by the omnipresence of technologies.

Not all disorders present the same difficulties, as pointed out in the study by 
\textcite{perez2021theory}. %Pérez-Vigil, Ilzarbe, Garcia-Delgar, Moler, Pomares, Puig, Lera-Miguel, Rosa, Romero, Calvo-Escalona and Lázaro (2021). 
The authors analyse the perception, interpretation and attribution of other people's mental states (theory of mind) in adolescents with autism spectrum disorder, obsessive-compulsive disorder and Tourette's syndrome. The sample consisted of 62 children aged 11-17 years, from the outpatient clinics of the Child and Adolescent Psychiatry Service of the Hospital Clínic de Barcelona. All the participants were men since, as they indicate, the prevalence of autism spectrum disorder in women is very low. Their results show that children with Tourette's syndrome and autistic spectrum disorder have similar difficulties in this respect, while children with obsessive-compulsive disorder have very similar results to children with no disorder.

Suffering from anxiety poses a risk to well-being and development in children and adolescents, as \textcite{lagosansiedad2014} %Lagos-San Martín, García-Fernández and Inglés (2014) 
state, while pointing out that a high level of anxiety reduces learning efficiency. This is because it results in less attention, concentration and retention of information. Regarding issues such as depression, authors such as \textcite{kassis2017understanding} %Kassis, Artz and White (2017) 
indicate that it is a complex problem that represents a public health challenge, affecting quality of life if it is undetected and untreated. Among the consequences for adolescents of alcohol consumption, \textcite{rubiogonzalez2016consumo}, %Rubio (2016),
with his study conducted in Andalusia, notes the continuity of consumption, both of alcohol and other substances, causes academic problems, risky sexual practices, criminal behaviour, brain injuries, traffic accidents, emotional disorders and neurodegeneration.

It should be noted that, among the many guidelines indicated by \textcite{russi2021} %Russi (2021) 
for proper brain stimulation during adolescence, is that of controlling the time adolescents spend in front of the computer and with technology, teaching them to make good use of them. In this sense, \textcite{rocabelijar2019adiccion} %Roca (2019) 
makes a series of proposals among which we find training aimed at parents, the creation of meeting places for adolescents, training aimed at adolescents on the appropriate use of technologies and mediation in the event that family conflicts arise due to addiction to the devices.

It is worth considering the situations that entail a risk of addiction to technologies, which is why the research carried out by \textcite{diaz2019uso} %Díaz-Vicario, Mercader and Gairín (2019) 
is relevant. In it, they reflect, through the analysis of different situations that can lead to inappropriate use of technologies, that the use of these technologies by adolescents presents a high risk, indicating the need for training in this regard from ages even before the age of 12. The analysis they carry out contemplates not only the vision of adolescents, but also the vision of teachers, experts, parents, and counsellors.

It is important to point out that excessive internet use, which in adolescence can occur at night, can affect memory or synthesis capacity, with poor school performance, concentration difficulties or drowsiness being warning signs \cite{cerisola2017impacto}. %(CERISOLA, 2017). 
Similarly, their use before going to sleep can have an impact on concentration problems or lead to poor sleep, which can affect cognitive performance \cite{russi2021,cerisola2017impacto,pin2019sueno}. %(RUSSI, 2021; CERISOLA, 2017; PIN, 2019). 
In addition, exposure to scenes with high levels of violence, both in television programmes and video games, can lead to an acceptance of violence as a means of conflict resolution, while it can also generate aggressive thoughts, fear or anxiety, as \textcite{russi2021} %Russi (2021) 
points out. In relation to social networks, \textcite{cerisola2017impacto} %Cerisola (2017) 
points out that overuse in adolescence can lead to depressive symptoms, sleep deficit, social isolation or overweight, which makes it necessary for parents to be involved in their children's digital lives in the same way as they regulate the age for going out at night, driving or drinking alcohol. If we are faced with a situation of addiction to technologies, such as video games, we must bear in mind that this addiction can cause the brain to adapt to high levels of dopamine or serotonin, with the brain demanding an increasingly higher level and causing an imbalance at brain level that can lead to mental illness in the future \cite{dealvarado2019trastornos}. %(DE ALVARADO, 2019).
\textcite{vicente2019adiccion} %Vicente-Escudero, Saura-Garre, López-Soler and Martínez (2019) 
indicate as protective factors against mobile phone and internet abuse, in relation to suffering from some psychopathology during adolescence, the fact of belonging to an organisation, having an adequate relationship with parents and good academic performance, and being autonomous. As \textcite{fungfallas2020impacto} %Fung, Rojas and Delgado (2020) 
point out, these problems may be due to a longer exposure time to screens than it should be. In addition, the authors emphasise the need to train parents and caregivers.

But the use of technology is not always harmful or negative and can have numerous advantages. \textcite{cerisola2017impacto} %Cerisola (2017) 
distinguishes between passive technologies and interactive technologies, and indicates that the latter can have numerous benefits during adolescence, as they can serve as a source of information and contribute to the development of critical thinking and intelligence, among other things. Also relevant is the work of \textcite{guzman2017nuevas}, %Guzmán, Putrino, Martínez and Quiroz (2017), 
who analyse the beneficial effects of the use of technologies for people with autism spectrum disorders, especially as an aid for communication. As \textcite{alonso2021tecnologia} %Alonso-García et al. (2021) 
indicate, the use of technology is essential in the teaching-learning processes of students in the 21st century.

There are numerous factors that should be taken into account to avoid neurodevelopmental problems derived from the excessive use of technology, especially in the classroom. Detecting and caring for students with neurodevelopmental problems is a necessary task for teachers, since, as \textcite{huertareyes2021trastornos} %Huertas and Castineyra (2021) 
point out, it is a fundamental issue in their professional performance. Along the same lines, \textcite{valencia2021problematica} %Valencia-Ortiz, Cabero-Almenara, Garay and Fernández (2021) 
point out that the training teachers should receive should focus on the acquisition of strategies to help their students reflect on the negative effects of the abusive use of social networks, on the guidance they should provide regarding their use and on the organisation of training actions in this regard, among others.

After considering all the facts previously mentioned, in this study we decided to analyse the correlations that may exist between some of these variables.


\section{Objectives}\label{sec-obj}
The aim of this study is to check whether there is a relationship between addiction to technology and factors such as depression, anxiety, substance or alcohol consumption, and bullying, which can cause neurodevelopmental problems during adolescence. To this end, the specific objectives are as follows:
\begin{enumerate}
    \item To detect situations in which there is addiction to technologies (internet, mobile and video games) or risk of suffering from it.
    \item To diagnose those cases in which there are symptoms compatible with anxiety or depression.
    \item Identify those participants who are addicted to alcohol or psychoactive substances, or at risk of such addiction.
    \item To point out participants who are victims of bullying.
    \item To establish correlations between addiction to technologies and the rest of the variables.
\end{enumerate}

\section{Methodology}\label{sec-meth}
The methodology followed during the development of the study consists on the application of a questionnaire for the diagnosis of addiction to technologies together with the application of 2 more questionnaires. One of them is a personality diagnosis questionnaire for adolescents, which has subscales focused on anxiety, depression, alcohol consumption and the consumption of psychoactive substances, among others. The last instrument used is a questionnaire for the detection of cases in which participants are victims of bullying. It is important to note that the study was approved by the School Council and has the approval of the Ethics Committee of the University of Granada.

\subsection{Participants}
The study was carried out in a school located in the city of Granada (Spain), with students from different socio-economic and cultural backgrounds. The sample consisted of twenty-five adolescents (twelve boys and thirteen girls), aged between twelve and seventeen, all of whom were in compulsory secondary education (6 participants in the first year, 7 in the second year, 5 in the third year and 7 in the fourth year), as shown in \Cref{tbl01}.

\begin{table}[htpb]
\centering
\begin{threeparttable}
\caption{Participants in the study}\label{tbl01}
\begin{tabular}{@{}llllll@{}}
\toprule
Course & 1º & 2º & 3º & 4º & Total \\
Participants & 6 & 7 & 5 & 7 & 25 \\
\bottomrule
\end{tabular}
\source{own elaboration.}
\end{threeparttable}
\end{table}

The spaces used to apply the instruments for data collection were the classrooms provided by the centre, where the different questionnaires were applied. The application of the different tools was carried out in several sessions, and each session lasted a maximum of 1 hour, to avoid overloading the participants. It is important to highlight that the participants were selected by the school based on pre-established criteria that included ensuring the presence of ethnic minorities, ensuring gender parity, including internal and external participants, students from different socio-economic and cultural backgrounds, ensuring the participation of adolescents from both urban and rural areas, foreign students and students suspected of being at risk in some aspect.


\subsection{Instruments}\label{sec-instr}
The questionnaire for the detection of technology addiction we have used in this study is the ADITEC \cite{choliz2016aditec}. %(CHÓLIZ, MARCO and CHÓLIZ, 2016). 
This instrument is composed of three scales oriented to different technologies: mobile, internet and video games. It allows us to determine whether there is addiction to any of these technologies or risk of addiction. Each of the questionnaires is divided into two parts, indicating in the first part how often the behaviours described occur or do not occur, and in the second part the degree of agreement or disagreement with a series of statements. The questionnaire aimed at detecting the presence, risk, or absence of addiction to mobile phones consists of a total of 22 items, the one for the Internet of 23 items and, finally, the one for video games consists of 24 items. Therefore, the application of the three questionnaires makes a total of 69 items.

The tool for diagnosing whether we are dealing with adolescents who suffer from anxiety or depression, or consume alcohol and substances, is the Spanish adaptation of the Personality Assessment Inventory for Adolescents (PAI-A) \cite{cardenal2018}, %(CARDENAL, ORTIZ-TALLO, SANTAMARÍA \& CAMPOS, 2018), 
whose subscales referring to anxiety, depression, alcohol consumption and substance use will be analysed in this study. The instrument is composed of 22 scales, which means that it is made up of 264 items in total.

Finally let us explain that, for the detection of cases in which participants suffer bullying, we will use the instrument Acoso y Violencia Escolar (AVE) \cite{pinuel2006ave}, %(PIÑUEL and OÑATE, 2007), 
which offers us different levels of severity in which the individual may be a victim of bullying. The questionnaire has a total of 94 items, 50 of which focus on those behaviours that the adolescents may suffer, and which place them in a position of being bullied, and 44 statements whose aim is to detect whether there is psychological damage in the person filling in the questionnaire.

The three instruments belong to the TEA publishing house and are answered in self-report format, as well as allowing them to be corrected through a specific platform provided by the publishing house itself.

Once the instruments had been applied and evaluated, the corresponding analyses were carried out to check whether there was any kind of correlation between the different variables under study. The SPSS statistical programme was used to analyse the data. It should be noted that the application of these questionnaires was approved by the Ethics Committee of the University of Granada, in addition to the study having been presented and approved by the School Council of the centre in which the data were collected, with a prior commitment to confidentiality on the part of the researchers.


\section{Results}\label{sec-results}
\subsection{Diagnosis carried out}\label{diagnosis}
After the application of the set of instruments, 8 participants were diagnosed with addiction or risk of addiction in terms of internet use, 6 participants with addiction or risk of addiction in terms of mobile phone use, and 8 participants with addiction or risk of addiction in relation to video games. In addition, 11 of the participants presented anxiety problems, another 11 presented symptoms associated with depression, 3 presented problems with alcohol consumption and 7 with substance use, while 12 of the participants were victims of bullying. These data are summarised in \Cref{tbl02}.

\begin{table}[htpb]
\centering
\begin{threeparttable}
\caption{Diagnostics performed}\label{tbl02}
\begin{tabular}{@{}*{8}{p{0.1\textwidth}}@{}}
\toprule
Internet addiction & Mobile phone addiction & Video game addiction & Anxiety & Depression & Alcohol consumption & Substance use & Victim of bullying \\
8 & 6 & 8 & 11 & 11 & 3 & 7 & 12 \\
\bottomrule
\end{tabular}
\source{Ávalos (2022).}
\end{threeparttable}
\end{table}


\subsection{Correlations between variables}\label{sec-corr}
With these results, the correlations between addiction to any of the different technologies studied and the rest of the variables were analysed. Furthermore, the correlation between the risk of addiction or addiction to technologies in general was analysed, considering this when risk or addiction was present in any of the three questionnaires applied, and the rest of the variables. All the data can be seen in \Cref{tbl03}.

According to \textcite{morales2008estadistica}, %Morales (2008), 
to determine the magnitude of the correlations, the following assessments are considered:
\begin{center}
\begin{tabular}{l@{}c@{}l} 
a value of r between: & \hspace{10em} & indicates a relation: \\
0 and .20 & \dotfill & very low \\
.20 and .40 & \dotfill & low \\
.40 and .60 & \dotfill & moderate \\
.60 and .80 & \dotfill &  significant, rather high \\
.80 and 1 & \dotfill & high or very high 
\end{tabular}
\end{center}

% TESTE
% Conversão de tabela com Pandoc.
% pandoc -f odt -t latex -o table3.tex table3.odt
\begin{table}[htpb]
\begin{threeparttable}
\caption{Correlations between addiction or risk of addiction to
technologies and the rest of the variables.}\label{tbl03}
\begin{tabular}{@{}lllllll@{}}
%\begin{longtable}[]{@{}lllllll@{}}
\toprule
& & Anxiety & Depression & Bullying & Alcohol & Drugs \tabularnewline
\midrule
\arrayrulecolor[gray]{.7}
\multirow{3}{*}{Internet} & Pearson correlation & ,083 & ,083 & ,199 & ,011 &
,145\tabularnewline
& Sig. (bilateral) & ,694 & ,694 & ,340 & ,960 & ,489 \tabularnewline
& N & 25 & 25 & 25 & 25 & 25 \tabularnewline
\midrule
\multirow{3}{*}{Mobile} & Pearson correlation & ,068 & -,121 & ,210 & ,369 &
,067\tabularnewline
& Sig. (bilateral) & ,747 & ,565 & ,314 & ,070 & ,751 \tabularnewline
& N & 25 & 25 & 25 & 25 & 25 \tabularnewline
\midrule
\multirow{3}{*}{Videogames} & Pearson correlation & ,083 & ,428\textsuperscript{*} & ,027
& -,253 & -,046\tabularnewline
& Sig. (bilateral) & ,694 & ,033 & ,896 & ,222 & ,828 \tabularnewline
& N & 25 & 25 & 25 & 25 & 25 \tabularnewline
\midrule
\multirow{3}{*}{ADITEC} & Pearson correlation & ,230 & ,395 & ,458\textsuperscript{*} & ,050 & -,036\tabularnewline
& Sig. (bilateral) & ,268 & ,051 & ,021 & ,811 & ,863 \tabularnewline
& N & 25 & 25 & 25 & 25 & 25 \tabularnewline
\arrayrulecolor{black}
\bottomrule
%\end{longtable}
\end{tabular}
\source{Ávalos (2022)}
\end{threeparttable}
\end{table}



There is a moderate and positive correlation between addiction or risk of addiction to video games and depression. This would suggest that the greater the addiction to this technology, the greater the chances of suffering from depression or manifesting depressive symptoms.

The same is applicable to addiction or risk of addiction to technologies in general and bullying. As we have already mentioned, the risk of addiction to technologies was considered positive when a positive result was obtained in any of the three technologies studied: Internet, mobile phone or video games. In this case, there is a moderate and positive correlation between addiction or risk of addiction to technologies and being a victim of bullying. The higher the level of addiction to technologies, the more likely the individual is to be a victim of bullying.

In order to consider a correlation as real and not as the result of chance, it is important to observe at what level the correlation is significant. In this case, the first correlation, related to depression, is significant at the 0.033 level, so the probability that the correlation has been established randomly is low. The same is the case for the correlation established with bullying, with a level of 0.021.

Given that the correlations obtained are moderate, cross-tabulations were made with those variables among which correlations are found, in order to look at these data in more detail. The data relating to addiction or risk of addiction to video games and depression are in \Cref{tbl04}, while those relating to addiction or risk of addiction to technology and bullying are in \Cref{tbl05}.

\begin{table}[htpb]
\centering
\begin{threeparttable}
\caption{Cross-referenced data on video game addiction and depression.}\label{tbl04}
\begin{tabular}{@{}*{5}{l}@{}}
\toprule
 & & & Depression & \\
 & & No risk & Risk & Total \\
\midrule
\multirow{2}{*}{Videogames} & No risk & 12 & 5 & 17 \\
 & Risk or addiction & 2 & 6 & 8 \\
Total & & 14 & 11 & 25 \\
\bottomrule
\end{tabular}
\source{Data obtained after application of the set of instruments.}
\end{threeparttable}
\end{table}

There were a total of 8 participants with a risk of video game addiction and 11 with symptoms of depression. When crossing the data, we saw that those 12 subjects, practically half of the sample, did not present a risk of addiction to video games or signs of depression, while there were 2 subjects with addiction or risk of addiction to video games and who did not present depressive symptoms. On the other hand, we found a total of 5 participants with a risk of suffering depression and no risk of addiction to video games, while there were 6 participants who obtained a score indicating a risk of addiction or addiction to video games and, in addition, present depressive symptoms.

\begin{table}[htpb]
\centering
\begin{threeparttable}
\caption{Cross-data on technology addiction and bullying.}\label{tbl05}
\begin{tabular}{@{}*{5}{l}@{}}
\toprule
 & & & Depression & \\
 & & No risk & Risk & Total \\
\midrule
\multirow{2}{*}{ADITEC} & No risk & 8 & 2 & 10 \\
 & Risk or addiction & 5 & 10 & 15 \\
Total & & 13 & 12 & 25 \\
\bottomrule
\end{tabular}
\source{Data obtained after application of the set of instruments.}
\end{threeparttable}
\end{table}

In this case, 15 subjects were found to be addicted to one or more of the technologies studied, while 12 were victims of bullying. If we cross the data, we find that there were 8 subjects who were not addicted or at risk of being addicted to any technology and who were not victims of bullying. There were 5 subjects with a technology addiction who were not victims of bullying and 2 who were victims of bullying, but there were no signs of technology addiction. Finally, a total of 10 participants obtained results that point to an addiction to technologies and who were also victims of bullying.



\section{Discussion y conclusions}\label{sec-disc}
After all the work carried out we can conclude that the overuse of technologies can lead to numerous problems, some of them related to the neurodevelopment of children and adolescents. Moreover, early detection of certain situations in educational centres and their intervention, for example, during tutoring hours, can prevent future problems \cite{avalos2022}, %(Ávalos and Fernández Cruz, 2022), 
both related to neurodevelopment and to a situation of social exclusion. For this reason, we decided to diagnose cases of addiction to technologies and analyse the correlations that exist with some of the variables that can be derived from such use and be problematic in terms of neurodevelopment.

In terms of addiction to technologies, a total of 15 subjects were detected with addiction or risk of addiction to one or more technologies (mobile phone, internet and/or video games). This represents 60\% of the sample, a fairly high percentage. A study carried out by \textcite{cuestacambra2020smartphone} %Cuesta, Cuesta and Martínez (2020) 
on the use of mobile phones indicates that the participants in this study stated that they were constantly connected to the Internet via their mobile phones, so the percentage obtained in this study is in line with this statement.

Anxiety is highly prevalent in childhood and adolescence, as indicated by \textcite{sanchez2020}, %Sánchez and Cohen (2020), 
even more so than depression. In the case of our study, 44\% of participants present anxiety, a percentage that is matched by those suffering from depression.

By focusing on the consumption of alcohol and psychoactive substances, we can compare the results with those of the Health Behaviour in School-age Children (HBSC) study in Spain in 2018 \cite{moreno2018adolescencia}. %(MORENO, RAMOS, RIVERA, SÁNCHEZ-QUEIJA, JIMÉNEZ-IGLESIAS, GARCÍA-MOYA, MORENO-MALDONADO, PANIAGUA, VILLAFUERTE-DÍAZ, CIRIA-BARREIRO, MORGAN and LEAL-LÓPEZ, 2020). 
This study indicates that 8\% of adolescents consume alcohol every week, while our percentage is slightly higher, reaching 12\% of participants with alcohol consumption problems. The same is the case with the consumption of other substances, with 28\% of the participants in our study having problems in this regard, while the aforementioned study indicates that 18\% of the adolescent participants smoke bongs, while 13\% claim to consume cannabis.

With regard to diagnosed bullying, 48\% of our sample obtained values that indicate they suffer from bullying or that they perceive it as such. The HSBC reports that 12\% of adolescents between 15 and 18 years of age are bullied. Our result is much higher, as is the age range of the participants, which could explain this difference. In this sense, it is worth noting that 28.2\% of the participants in the study carried out by \textcite{plaza2021evolucion} %Plaza (2021) 
have received threats on some occasion through technological channels.

When analysing the correlations found between addiction to technology and the other variables, we cannot speak of a significant correlation with anxiety or with alcohol or substance use. However, video game addiction is correlated with depression. This data clashes with other studies that use video games as a form of therapy against depression, such as that of Moscardi (2020), or that consider them to be a useful tool in terms of raising awareness of depression or other mental illnesses, such as the work carried out by \textcite{paredesotero2020enemigo}. %Paredes (2020).
This only goes to show the need to focus on how video games are used, given that depending on how they are used, they can be beneficial or can contribute to the development of serious problems.

Having an addiction to technology is generally correlated with being a victim of bullying. This could be due to the fact that, by being connected to the network for many hours a day, adolescents are not only exposed to bullying situations in real life, but that this risk goes beyond the screens, and cases of cyberbullying can also be found, as indicated by \textcite{perez2021theory}. %Pérez (2019). 
In this sense, \textcite{cortesalfaro2020acoso} %Cortés (2020) 
points out that bullying, even if it is via the Internet, should not be seen as a problem of the use of technology, but rather as a problem of abuse of power, since treating it as a result of technology means that it loses its character as a social problem.

Despite not having obtained strong correlations, it has been found that there are numerous authors who link addiction to technologies during adolescence with neurodevelopmental problems. In this sense, \textcite{alvarez2010neurociencias} %Álvarez (2010)
states that there is insufficient awareness of the need to assess neurodevelopment in the child population, which could bring benefits to society as a whole, especially in terms of controlling the effective investment of resources in community development projects.

\printbibliography\label{sec-bib}
% if the text is not in Portuguese, it might be necessary to use the code below instead to print the correct ABNT abbreviations [s.n.], [s.l.]
%\begin{portuguese}
%\printbibliography[title={Bibliography}]
%\end{portuguese}


%full list: conceptualization,datacuration,formalanalysis,funding,investigation,methodology,projadm,resources,software,supervision,validation,visualization,writing,review
\begin{contributors}[sec-contributors]
\authorcontribution{Inmaculada Ávalos Ruiz}[conceptualization,datacuration,investigation,methodology,writing]
\authorcontribution{Mercedes Cuevas López}[investigation,methodology]
\authorcontribution{Emilio Lizarte Simón}[datacuration,formalanalysis]
\authorcontribution{Slava López Rodríguez}[conceptualization,formalanalysis,writing]
\end{contributors}



\end{document}

