% !TEX TS-program = XeLaTeX
% use the following command:
% all document files must be coded in UTF-8
\documentclass[portuguese]{textolivre}
% build HTML with: make4ht -e build.lua -c textolivre.cfg -x -u article "fn-in,svg,pic-align"

\journalname{Texto Livre}
\thevolume{15}
%\thenumber{1} % old template
\theyear{2022}
\receiveddate{\DTMdisplaydate{2022}{7}{13}{-1}} % YYYY MM DD
\accepteddate{\DTMdisplaydate{2022}{8}{30}{-1}}
\publisheddate{\DTMdisplaydate{2022}{10}{24}{-1}}
\corrauthor{Maria Selma Lima do Nascimento}
\articledoi{10.35699/1983-3652.2022.40459}
%\articleid{NNNN} % if the article ID is not the last 5 numbers of its DOI, provide it using \articleid{} commmand 
% list of available sesscions in the journal: articles, dossier, reports, essays, reviews, interviews, editorial
\articlesessionname{dossier}
\runningauthor{Nascimento et al.} 
%\editorname{Leonardo Araújo} % old template
\sectioneditorname{Daniervelin Pereira}
\layouteditorname{Leonado Araújo}

\title{Neuroeducação e tecnologia: parceiras emergentes no processo ensino-aprendizagem no contexto educacional do século XXI}
\othertitle{Neuroeducation and technology: emerging partners in the teaching-learning process in the educational context of the XXI century}
% if there is a third language title, add here:
%\othertitle{Artikelvorlage zur Einreichung beim Texto Livre Journal}

\author[1]{Maria Selma Lima do Nascimento~\orcid{0000-0003-1000-3822}\thanks{Email: \href{mailto:selmalima21@yahoo.com.br}{selmalima21@yahoo.com.br}}}
\author[2]{Leonice da Silva Santos~\orcid{0000-0002-6248-339X}\thanks{Email: \href{mailto:leosinha.leosinha@gmail.com}{leosinha.leosinha@gmail.com}}}
\author[1]{Maria da Penha Cardoso~\orcid{0000-0001-8417-3141}\thanks{Email: \href{mailto:professorapenhacardoso@hotmail.com}{professorapenhacardoso@hotmail.com}}}
\author[1]{Márcia Morais de Melo~\orcid{0000-0002-3948-2612}\thanks{Email: \href{mailto:marciamoraismelo@hotmail.com}{marciamoraismelo@hotmail.com}}}
\affil[1]{World University Ecumenical-WUE. Florida Department of Education, Miami-FL, USA.}
\affil[2]{Universidad Autónoma de Asunción, Departamento de Pedagogía, Asunción, Paraguay.}

\addbibresource{article.bib}
% use biber instead of bibtex
% $ biber article

% used to create dummy text for the template file
\definecolor{dark-gray}{gray}{0.35} % color used to display dummy texts
\usepackage{lipsum}
\SetLipsumParListSurrounders{\colorlet{oldcolor}{.}\color{dark-gray}}{\color{oldcolor}}

% used here only to provide the XeLaTeX and BibTeX logos
\usepackage{hologo}

% if you use multirows in a table, include the multirow package
\usepackage{multirow}

% provides sidewaysfigure environment
\usepackage{rotating}

% CUSTOM EPIGRAPH - BEGIN 
%%% https://tex.stackexchange.com/questions/193178/specific-epigraph-style
\usepackage{epigraph}
\renewcommand\textflush{flushright}
\makeatletter
\newlength\epitextskip
\pretocmd{\@epitext}{\em}{}{}
\apptocmd{\@epitext}{\em}{}{}
\patchcmd{\epigraph}{\@epitext{#1}\\}{\@epitext{#1}\\[\epitextskip]}{}{}
\makeatother
\setlength\epigraphrule{0pt}
\setlength\epitextskip{0.5ex}
\setlength\epigraphwidth{.7\textwidth}
% CUSTOM EPIGRAPH - END

% LANGUAGE - BEGIN
% ARABIC
% for languages that use special fonts, you must provide the typeface that will be used
% \setotherlanguage{arabic}
% \newfontfamily\arabicfont[Script=Arabic]{Amiri}
% \newfontfamily\arabicfontsf[Script=Arabic]{Amiri}
% \newfontfamily\arabicfonttt[Script=Arabic]{Amiri}
%
% in the article, to add arabic text use: \textlang{arabic}{ ... }
%
% RUSSIAN
% for russian text we also need to define fonts with support for Cyrillic script
% \usepackage{fontspec}
% \setotherlanguage{russian}
% \newfontfamily\cyrillicfont{Times New Roman}
% \newfontfamily\cyrillicfontsf{Times New Roman}[Script=Cyrillic]
% \newfontfamily\cyrillicfonttt{Times New Roman}[Script=Cyrillic]
%
% in the text use \begin{russian} ... \end{russian}
% LANGUAGE - END

% EMOJIS - BEGIN
% to use emoticons in your manuscript
% https://stackoverflow.com/questions/190145/how-to-insert-emoticons-in-latex/57076064
% using font Symbola, which has full support
% the font may be downloaded at:
% https://dn-works.com/ufas/
% add to preamble:
% \newfontfamily\Symbola{Symbola}
% in the text use:
% {\Symbola }
% EMOJIS - END

% LABEL REFERENCE TO DESCRIPTIVE LIST - BEGIN
% reference itens in a descriptive list using their labels instead of numbers
% insert the code below in the preambule:
%\makeatletter
%\let\orgdescriptionlabel\descriptionlabel
%\renewcommand*{\descriptionlabel}[1]{%
%  \let\orglabel\label
%  \let\label\@gobble
%  \phantomsection
%  \edef\@currentlabel{#1\unskip}%
%  \let\label\orglabel
%  \orgdescriptionlabel{#1}%
%}
%\makeatother
%
% in your document, use as illustraded here:
%\begin{description}
%  \item[first\label{itm1}] this is only an example;
%  % ...  add more items
%\end{description}
% LABEL REFERENCE TO DESCRIPTIVE LIST - END


% add line numbers for submission
%\usepackage{lineno}
%\linenumbers

\usepackage{siunitx}
\sisetup{output-decimal-marker = {,}}

\begin{document}
\maketitle

\begin{polyabstract}
\begin{abstract}
A tecnologia facilita a vida de muitas pessoas. Inserida no meio educacional, ela oportuniza ao professor diferentes formas de dinamizar as aulas, tornando-as interessantes para os alunos. Por isso, repensar as práticas e o processo pedagógico é importante para se adquirir uma educação de qualidade. Da mesma forma surgem a neuroeducação e a neurodidática que, como disciplinas, fortalecem cada vez mais o ensino e a aprendizagem através do entendimento de como o cérebro funciona. Compreender essa dinâmica é garantir aos alunos uma aula mais bem elaborada, baseada em muita reflexão. Portanto, essa pesquisa surgiu com o objetivo geral de analisar a neuroeducação e a tecnologia como parceiras emergentes no processo de ensino-aprendizagem no contexto do século XXI. A neuroeducação e a tecnologia podem ser parceiras no processo de ensino-aprendizagem? É o problema que move esta pesquisa. Por fim, neste trabalho encontraremos a oportunidade de gerar reflexões a respeito da tecnologia associada à neurociência e sua contribuição no processo educacional. Cabe à escola, portanto, a adaptação a esses novos momentos. Assim, não resta dúvida de que a tecnologia é um grande recurso para as atividades educativas e sociais.

\keywords{Tecnologia \sep Neuroeducação \sep Ensino-aprendizagem \sep Educação}
\end{abstract}

\begin{english}
\begin{abstract}
Technology facilitates the lives of many people, and inserted in the educational  environment, it provides the teacher with different ways to streamline classes and make them interesting for students. Therefore, rethinking practices and the pedagogical process are routine ways of acquiring quality education. In the same way, neuroeducation and neurodidactics arise, which, as, as disciplines, increasingly strengthen teaching and learning through the understanding of how the brain works. To understand this dynamics is to offer students a more elaborate class, rethought based on a lot of reflection. Therefore, this research emerged with the general objective of analyzing neuroeducation and technology as emerging partners in the teaching-learning process in the context of the 21st century. Can neuroeducation and technology be partners in the teaching-learning process? It is the problem that drives this research. Finally, in this work, we will find the opportunity to generate reflections on the technology associated with neuroscience and its contribution to the educational process. It is up to the school, therefore, to adapt to these new moments, so there is no doubt that technology is a great resource for educational and social activities.

\keywords{Technology \sep Neuroeducation \sep Teaching-learning \sep Education}
\end{abstract}
\end{english}
% if there is another abstract, insert it here using the same scheme
\end{polyabstract}

\section{Introdução}\label{sec-intro}
Os conhecimentos neurocientíficos baseados no processo de ensino-aprendizagem surgem de forma a revolucionar o ensino e o ambiente educacional. Compreender como o cérebro aprende, como ele funciona no momento da aprendizagem, como são armazenadas as informações, como os estímulos acontecem durante esse processo são assuntos extremamente atuais e relevantes e podem ser compreendidos mediante os saberes neurocientíficos. 

Com o avanço desses novos conhecimentos neurocientíficos, é possível repensar a aprendizagem e reorganizar novas metodologias no intuito de avançar a aprendizagem dos alunos, como também tornar o ensino mais agradável e atraente aos alunos. Na educação é viável aproveitar a evolução dessa ciência e seus benefícios no campo educacional. Segundo \textcite{fernandez_martinez_propuesta_2021}, a neuroeducação através da neurodidática fortalece o processo de ensino-aprendizagem, pois com esse conhecimento é possível dinamizar as aulas e inovar metodologias em favor do processo educacional. A utilização da neurodidática no campo educacional proporciona aos docentes a criação de métodos para a dinâmica de aulas focando sempre as necessidades individuais de cada aluno \cite{molina_formacion_2020}. Essa ciência possibilita uma combinação excelente em relação aos sentimentos e emoções dos educandos, observando que o foco está centrado em manter o aluno curioso, atento, focando a leitura, escrita e conhecimentos matemáticos. Com esse conhecimento, o professor apresenta um excelente papel no processo educacional tornando-se capaz de desenvolverem um bom trabalho através de um planejamento alicerçado nos conhecimentos neurocientíficos. A atualização desses conhecimentos assegura um desenvolvimento promissor, observado na realidade e individualidade de cada um dos alunos.

\section{Neuroeducação}\label{sec-normas}
De acordo com \textcite{hernandez_fernandez_inclusion_2021}, a neuroeducação é dos grandes desafios neste século XXI, juntamente com a atenção à diversidade, tendo em conta a necessidade e os mais diversos aspectos. Segundo \textcite{nascimento_o_2011}, apesar de possuírem propostas diferentes, a educação e a Neurociência criam condições vantajosas de aprendizagem. Assim, a autora defende que as descobertas neurocientíficas constituem uma verdadeira revolução na educação, pois os estudos relacionados à Neurociência Cognitiva se detêm a pesquisar as funções cognitivas do cérebro explicitando como funcionam as aprendizagens do ser humano.

A neuroeducação tem por finalidade integrar as três áreas do conhecimento: Psicologia, Educação e Neurociências, objetivando compreender os comportamentos de aprendizagem. Assim, com a neuroeducação é possível explicar a importância das emoções no processo de ensino-aprendizagem. Estão envolvidos nesse processo a mente, a educação e o cérebro, no intuito de compreender as aprendizagens escolares mediante uma reação bioquímica formando memórias e conceitos. \cite{brandao_contribuicoes_2019}.

\subsection{Perspectivas neurodidáticas na educação do século XXI}\label{sec-conduta}
Para \textcite{fernandez_martinez_propuesta_2021}, a neurodidática é uma disciplina que deriva das Neurociências, pois objetivam fortalecer o ensino-aprendizagem através do conhecimento e desenvolvimento do cérebro, potencializando o aprendizado e a inovação.

Dessa forma, a Neurodidática é uma disciplina nova, que surgiu para contrapor e reflexionar a respeito do ensino tradicional. Alguns autores defendem que a Neurodidática será uma disciplina do futuro, pois permite ao professor compreender o processo mental na aprendizagem. Isso permite que o professor possa melhorar seu planejamento com práticas e metodologias adequadas a cada aluno. \cite{fernandez_martinez_propuesta_2021}.

\subsection{Tecnologia}\label{sec-fmt-manuscrito}
Com a chegada das Tecnologias da Informação e da Comunicação (TIC), podemos vislumbrar inúmeras mudanças na sociedade. Com a expansão das TIC, a informação consegue expandir velozmente através do acesso, da organização e difusão da informação. Nessa perspectiva, os avanços seguem ou devem seguir em todas as etapas do sistema educativo \cite[p. 15]{navas-parejo_entornos_2022}.

Segundo a \textcite[p. 06]{unesco_enfoques_2013}, alguns aspectos são extremamente relevantes para a renovação das práticas educativas, permitindo revisão das estratégias no âmbito da aprendizagem e das políticas de formação docente, nas quais as tecnologias educativas são um forte apoio na implementação de novas mudanças. Dessa forma, observamos que as tecnologias são relevantes nas mudanças educacionais do século XXI.

Dessa maneira, as políticas públicas em tecnologias e voltadas à formação docente incorporam as TIC mediando o processo de aprendizagem, além de valorizar, harmonizar e reflexionar sobre a eficácia no processo de ensino-aprendizagem. \cite[p. 19]{lima_do_nascimento_formacion_2020}. 


\subsection{Neurociência e tecnologia no sistema educativo do século XXI}\label{sec-formato}
De acordo com \textcite[p. 499]{fernandez_batanero_capacitacion_2019}, os estudos sobre as TIC  são um campo em evolução e de interesse contemporâneo, pois permite a promoção da inclusão, da equidade e igualdade o âmbito educacional. Os autores apontam que, com o avanço do acesso à tecnologia, sua presença no setor educacional é uma realidade necessária e atual, conhecida inicialmente por Novas Tecnologias (NT), seguidas de TIC (Tecnologias da informação e Comunicação), passando por TAC (Tecnologia para o Aprendizado e Conhecimento) e TEP (Tecnologias para Empreendimentos e Participação). 

Com essas afirmações, podemos verificar a importância da neurociência no processo educacional, sabendo da capacidade infinita do cérebro ao proporcionar a inclusão e a humanização mediante o uso das novas tecnologias.

Sendo assim, \textcite{castells_galaxia_2003} enfatiza que o uso da Internet e da tecnologia no processo educacional é muito vantajoso, porém ressalta que é preciso levar em consideração os professores e sua formação para lidar com hardware e tecnologia, treinamento de professores e pessoal da escola para a tecnologia. Desse modo, com a tecnologia disponível, os professores devem ser treinados para usá-las com muita eficiência, além de proporcionar uma nova reestruturação nas organizações pedagógicas e nas institucionais, cujo objetivo é o de estimular novas habilidades de aprendizagem.

\subsection{Ensino-aprendizagem}\label{sec-modelo}

O ensino-aprendizagem permeia práticas de ensino que venham a mudar uma realidade existente no percurso da educação. Assim, discutir uma prática de ensino-aprendizagem mediante os recursos tecnológicos podem transformar a aprendizagem em uma prática prazerosa e eficaz.

Uma prática voltada para um ensino responsável e eficaz prima por diversos fatores. Aqui mencionamos um ensino-aprendizagem voltado para a verdadeira necessidade do aluno. Assim, observar certas particularidades é importante no transcurso do processo, um espaço que apresente a inclusão, a equidade, o planejamento e programas voltados para uma educação de qualidade, capazes de repensar estratégias pedagógicas, enfrentar os obstáculos \cite{sousa_educacao_2020}.

Para \textcite{amorim_as_2019}, na educação, a tecnologia é uma ferramenta muito valiosa no cenário da prática pedagógica, é facilitadora do conhecimento, é capaz de gerar uma reflexão e mudar a concepção de ensino-aprendizagem no cenário educacional. Muitas transformações são possíveis tendo em vista que a escola, a sociedade e o professor estão cercados de mudanças e elas chegam ao ambiente escolar.

\textcite{levy_cibercultura_1999} enaltece as tecnologias e seus impactos na vida humana. Impossível separar o homem de seu ambiente material, dos signos, das imagens, tendo em vista que tudo faz sentido para sua vida. Assim, não é possível separar o mundo e sua parte mais artificial, nem os objetos técnicos criados e as pessoas que os inventam, produzem e usam. Enfim, é dada ao homem e suas instituições a oportunidade de interagir com a sociedade, através da comunicação e das memórias artificiais, além da tecnologia inserida no ambiente social.

\section{Método}\label{sec-organizacao}
O objetivo geral desta pesquisa foi analisar a neuroeducação e a tecnologia como parceiras emergentes no processo de ensino-aprendizagem no contexto do século XXI. A problemática da pesquisa é saber se a neuroeducação e a tecnologia podem ser parceiras no processo de ensino-aprendizagem.

Dessa forma, os objetivos específicos abordados nesta pesquisa são: 1-Identificar os conhecimentos dos professores sobre a neuroeducação; 2- Especificar a função do conhecimento neurodidático dos professores para uma educação emergente; 3- Detalhar a importância da tecnologia no processo de ensino-aprendizagem. 4- Descrever o processo de ensino-aprendizagem e sua relação com o procedimento neuroeducacionais. 5- Analisar se o contexto educacional tem contribuído para uma educação emergente.

A investigação aqui desenvolvida é do tipo descritivo e explicativo, baseado no paradigma interpretativo e não experimental. Por isso, a metodologia é quantitativa, conforme acorda \textcite{hurtado_leon_paradigmas_1998}, quando defende que a investigação quantitativa objetiva esclarecer como os elementos conduzem o problema e onde ele se inicia. Trata-se de um processo rigoroso, detalhado e minucioso que prima por uma sistematização que conduzirá à solução do problema. Nesta investigação foi executada a utilização de uma escala Likert.

\subsection{População e amostra}\label{sec-organizacao-latex}
Esta pesquisa contou com uma população formada por 23 professores brasileiros. Esses professores pertencem à região Nordeste do Brasil, local escolhido por fazer parte de um contexto difícil em relação às diversas temáticas sociais. Aqui priorizamos a temática da educação. Contamos com 6 professores do sexo masculino e 18 professores do sexo feminino. A faixa etária média dos participantes é de 30 a 45 anos. O anonimato e a participação voluntária dos entrevistados foram primordiais para uma investigação de qualidade.

\subsection{Instrumento, dimensiones, variáveis e hipóteses}\label{sec-idioma}
O instrumento escolhido foi um questionário baseado numa escala Likert. Esse instrumento foi construído através de uma matriz de operacionalização de variáveis sempre observando os objetivos específicos. 

\begin{enumerate}[label=\Alph*.]
    \item Neuroeducação
    \item Neurodidática
    \item Tecnologia
    \item Ensino-aprendizagem
    \item Contextos educacionais emergentes
\end{enumerate}

A escala Likert ficou composta por 36 itens, amparada em cinco dimensões: A (neuroeducação), B (neurodidática), C (tecnologia), D (ensino-aprendizagem) e E (contextos educacionais emergentes). Foram ofertadas cinco opções de respostas, ficando dessa forma assim: 1 discordo totalmente, 2 discordo, 3 indiferente, 4 concordo e 5 concordo totalmente. Algumas informações sociodemográficas foram solicitadas dos participantes, como dados de sexo e idade. Em relação à análise dos dados, foram processados através do Google Forms e da mesma forma enviado online aos participantes. As variáveis independentes são: neuroeducação e tecnologias, entretanto as variáveis dependentes são ensino-aprendizagem e contexto educacional. 
A respeito da hipótese, temos a H0. – Não há relação entre a neuroeducação e a tecnologia como parceira no processo de ensino aprendizagem.

\section{Resultados}\label{sec-resumo}
Nesta fase da pesquisa, seguiremos observando as análises e os resultados mediante a validez de conteúdo, a confiabilidade e, por fim, a validez de construção. 

\subsection{Validez de conteúdo}\label{sec-secoes}
A fase validez de conteúdo foi realizada por oito doutores especialistas nessa temática. Foi observado o conteúdo abordado, contudo as sugestões apresentadas foram acatadas objetivando as possíveis melhorias, a mencionar a clareza, a coesão e a coerência do tema em investigação. O processo seguinte foi uma prova piloto através de um subgrupo da amostra cujo objetivo foi de verificar pormenores importantes para o bom andamento do processo investigativo, como: dúvidas e compreensão.

\subsection{Confiabilidade }\label{sec-format-simple}
A confiabilidade é muito importante e consistente no que diz respeito à autenticidade dos resultados. Ela consiste em verificar os resultados de uma escala observando se eles estão de acordo com os resultados da mesma escala, porém em outro ambiente. Assim, nessa investigação o coeficiente de confiabilidade foi calculado através de alguns procedimentos. Por exemplo, a aplicação da prova piloto, em que as respostas foram codificadas através do programa estatístico SPSS, sendo os resultados obtidos e transcrito em matriz de entrada dupla. Dessa forma, os valores foram interpretados mediante sugestões de \textcite{ruiz_instrumentos_1998}. 

\subsection{Validez de construção}\label{sec-links}
A validez de construção foi realizada mediante uma análise fatorial exploratória, pois não conhecemos a princípio os números de fatores. O objetivo principal é apresentar uma escala cientificamente comprovada e foi utilizada a medida Kaiser-Meyer-Olkin de adequação amostral. Dessa forma, seguiu-se uma extração das comunalidades onde os com valores apresentados foram muito bons, não precisando descartar nenhuma. 

Analisando a \Cref{tab01}, podemos visualizar a distribuição da frequência do objetivo 1 desta pesquisa mediante observação da dimensão A (neuroeducação).

\begin{table}[h!]
\centering
\small
\begin{threeparttable}
\caption{Distribuição dos resultados do objetivo 1e da dimensão A (neuroeducação).}
\label{tab01}
\begin{tabular}{@{}p{0.42\textwidth}@{} *{5}{S}}
\toprule
Dimensão (A) neuroeducação & \multicolumn{1}{p{1.2cm}}{Discordo muito} & \multicolumn{1}{p{1.2cm}}{Discordo} & \multicolumn{1}{p{1.2cm}}{Indife\-ren\-te} & \multicolumn{1}{p{1.2cm}}{Concordo} & \multicolumn{1}{p{1.2cm}}{Concordo muito} \\
\midrule
A1- A neuroeducação enriquece a educação do século XXI. & & & & 39,1 \% & 60,9 \% \\
A2- Os conhecimentos em neuroeducação facilitam o aprendizado dos alunos de forma mais eficiente. & & & 4,4 \% & 56,5 \% & 39,1 \% \\
A3- Os conhecimentos sobre neuroeducação ampliam as possibilidades de um ensino-aprendizado mais significativo. & & & & 65,2 \% & 34,8 \% \\
A4- Conhecer o funcionamento das atividades cerebrais dos alunos faz parte dos estudos da neurociência. & & & & 52,2 \% & 47,8 \% \\
A5- Os conhecimentos neuroeducacionais facilitam o desempenho metodológico do professor. & & & & 65,2 \% & 34,8 \% \\
A6- A formação adquirida em neuroeducação contribui para melhor desempenho do professor durante o processo de ensino-aprendizagem. & & & & 52,2 \% & 47,8 \% \\
\bottomrule
\end{tabular}
\source{Elaborado pelas autoras.}
\end{threeparttable}
\end{table}

Ao identificar os conhecimentos dos professores a respeito da dimensão neuroeducação (\Cref{tab01}), percebe-se que os participantes em sua maioria estão seguros de que concordam e/ou concordam muito sobre os pontos apresentados. Importante destacar que na \Cref{tab01} os itens A3 e A5 são destaques, pois ampliam e confirmam que os participantes consideram significativos os conhecimentos neuroeducacionais para um melhor desempenho metodológico do professor. Podemos observar que na opinião dos participantes os conhecimentos neurocientíficos pelos educadores facilitam o aprendizado dos alunos de forma eficiente.

\begin{table}[h!]
\centering
\small
\begin{threeparttable}
\caption{Distribuição dos resultados do objetivo 2 e da dimensão B (neurodidática).}
\label{tab02}
\begin{tabular}{@{}p{0.42\textwidth}@{} *{5}{S}}
\toprule
Dimensão (B) neurodidática & \multicolumn{1}{p{1.2cm}}{Discordo muito} & \multicolumn{1}{p{1.2cm}}{Discordo} & \multicolumn{1}{p{1.2cm}}{Indife\-ren\-te} & \multicolumn{1}{p{1.2cm}}{Concordo} & \multicolumn{1}{p{1.2cm}}{Concordo muito} \\
\midrule
B1- A neurodidática é relevante para as mudanças significativas no atual contexto educativo. & & & & 56,5 \% & 43,5 \% \\
B2- Os conhecimentos neurodidáticos são essenciais para enriquecer os conhecimentos do professor atualizado. & & & & 56,5 \% & 43,5 \% \\
B3-Ampliar a compreensão da funcionalidade cerebral no processo de ensino-aprendizagem com o auxílio da neurodidática é pertinente ao professor atualizado. & & & & 56,5 \% & 43,5 \% \\
B4- A neurodidática visa aos profissionais da educação compreender todos os aspectos do processo de ensino-aprendizagem. & & 4,3 \% & 4,3 \% & 60,9 \% & 30,4 \% \\
B5- Os conhecimentos neurodidáticos permitem ao professor uma melhor organização em sua prática educativa no intuito de melhorar a atividade cerebral dos alunos durante o processo de ensino-aprendizagem. & & & & 60,9 \% & 39,1 \% \\
B6-Neurodidática no processo educacional visa melhorar o nível de aprendizagem dos alunos. & & & & 56,5 \% & 43,5 \% \\
\bottomrule
\end{tabular}
\source{Elaborado pelas autoras.}
\end{threeparttable}
\end{table}

Observou-se a respeito da função dos conhecimentos neurodidáticos dos professores para uma educação emergente, além da dimensão B, neurodidática. Visualizamos na \Cref{tab02} o destaque para os itens B4 e B5, em que o B4 vislumbra que a neurodidática visa aos profissionais da educação compreenderem todos os aspectos do processo de ensino-aprendizagem. Para essa opção percebemos que 60,9\% dos participantes concordam com essa afirmativa, enquanto que 30,4\% concordam muito com o que esse item descreve. A respeito do item B5, em que se descreve que os conhecimentos neurodidáticos permitem ao professor uma melhor organização em sua prática educativa no intuito de melhorar a atividade cerebral dos alunos durante o processo de ensino-aprendizagem, o destaque é que 60,9\% dos entrevistados concordam, enquanto que 30,4\% concordam muito com o que esse item descreve. Assim, avaliam a dimensão neurodidática como relevante na mudança do contexto atual.


\begin{table}[h!]
\centering
\small
\begin{threeparttable}
\caption{Distribuição dos resultados do objetivo 3 e da dimensão C (tecnologia).}
\label{tab03}
\begin{tabular}{@{}p{0.42\textwidth}@{} *{5}{S}}
\toprule
Dimensão (C) tecnologia & \multicolumn{1}{p{1.2cm}}{Discordo muito} & \multicolumn{1}{p{1.2cm}}{Discordo} & \multicolumn{1}{p{1.2cm}}{Indife\-ren\-te} & \multicolumn{1}{p{1.2cm}}{Concordo} & \multicolumn{1}{p{1.2cm}}{Concordo muito} \\
\midrule
C1- O recurso tecnológico utilizado durante as aulas facilita a prática pedagógica. & & & & 34,8 \% & 65,2 \% \\
C2- Os alunos demonstraram um comportamento mais participativo nas aulas quando utilizo um recurso tecnológico. & & & & 26,1 \% & 73,8 \% \\
C3- O recurso tecnológico escolhido para ministrar as aulas influi nos resultados esperados. & & & & 43,5 \% & 56,5 \% \\
C4- A tecnologia aliada a uma boa metodologia propicia resultados satisfatórios no processo de ensino aprendizagem. & &  &  & 39,1 \% & 60,9 \% \\
C5- A metodologia docente foi renovada após o uso dos recursos tecnológicos. & & 4,3 \% & 8,7 \% & 47,8 \% & 39,1 \% \\
C6- Sinto-me seguro ao ministrar os recursos tecnológicos durante as aulas. & & & 8,7 \% & 52,2 \% & 39,1 \% \\
\bottomrule
\end{tabular}
\source{Elaborado pelas autoras.}
\end{threeparttable}
\end{table}

Na \Cref{tab03}, vislumbramos observar os resultados do objetivo 3 desta investigação, em que almejamos detalhar a importância da tecnologia no processo de ensino-aprendizagem sempre em observância à dimensão C, tecnologia. Como resultado, no item C2 73,8\% dos entrevistados escolheram concordar muito, fazendo referência aos alunos e a seu comportamento mais participativo nas aulas quando utilizado um recurso tecnológico. O item C4 busca entender se os entrevistados consideram que a tecnologia aliada a uma boa metodologia propicia resultados satisfatórios no processo de ensino-aprendizagem, chegando ao resultado de 60,9\% dos entrevistados concordando muito com essa afirmativa.


\begin{table}[h!]
\centering
\small
\begin{threeparttable}
\caption{Distribuição dos resultados do objetivo 4 e da dimensão D (ensino-aprendizagem).}
\label{tab04}
\begin{tabular}{@{}p{0.42\textwidth}@{} *{5}{S}}
\toprule
Dimensão (D) ensino-aprendizagem & \multicolumn{1}{p{1.2cm}}{Discordo muito} & \multicolumn{1}{p{1.2cm}}{Discordo} & \multicolumn{1}{p{1.2cm}}{Indife\-ren\-te} & \multicolumn{1}{p{1.2cm}}{Concordo} & \multicolumn{1}{p{1.2cm}}{Concordo muito} \\
\midrule
D1- É necessária a formação docente em neurociência para melhorar o desempenho do professor e do aluno no processo ensino-aprendizagem. & & 4,3 \% & 8,7 \% & 65,2 \% & 21,7 \% \\
D2- A formação em neurociência recebida corresponde as minhas necessidades durante a formulação das estratégias didáticas. & & 4,3 \% & 13,0 \% & 65,2 \% & 17,4 \% \\
D3- A formação em neurociência adquirida contribui para o meu desempenho durante as aulas. & & 4,3 \% & 13,0 \% & 65,2 \% & 17,4 \% \\
D4- A formação em neurociência recebida me deixa seguro(a) na hora de ministrar as aulas. & & 4,3 \% & 13,0 \% & 65,2 \% & 17,4 \% \\
D5- Percebo uma grande necessidade de atualização da minha formação em neurociência. & & 4,3 \% & & 69,6 \% & 26,1 \% \\
D6- A formação em neurociência recebida favorece o desempenho da minha aula. & & 4,3 \% & 8,7 \% & 69,6 \% & 17,4 \% \\
\bottomrule
\end{tabular}
\source{Elaborado pelas autoras.}
\end{threeparttable}
\end{table}

Na \Cref{tab04}, observamos o resultado do objetivo 4, em que é necessário descrever o processo de ensino-aprendizagem e sua relação com os procedimentos neuroeducacionais. A observação à dimensão D, ensino-aprendizagem, foi importante. Observamos na \Cref{tab04} que as escolhas dos participantes foram bem diversificadas. Apareceram escolhas que apontam “discordo” e “indiferente”, e as escolhas mais expressivas pertencem à escolha “concordo” e “concordo muito”. 69,6\% dos participantes escolheram o item D5, que reforça a grande necessidade de atualização da formação dos profissionais em neurociência e o item D6: A formação em neurociência recebida favorece o desempenho da minha aula, que teve concordância de 69,6\% dos entrevistados. Os professores entendem e concordam que a atualização dos conhecimentos na neurociência é muito importante para os conhecimentos referente à neuroeducação e para o processo de ensino-aprendizagem.


\begin{table}[h!]
\centering
\small
\begin{threeparttable}
\caption{Distribuição dos resultados do objetivo 5 e da dimensão E (contextos educacionais emergentes).}
\label{tab05}
\begin{tabular}{@{}p{0.42\textwidth}@{} *{5}{S}}
\toprule
Dimensão (E) contextos educacionais emergentes & \multicolumn{1}{p{1.2cm}}{Discordo muito} & \multicolumn{1}{p{1.2cm}}{Discordo} & \multicolumn{1}{p{1.2cm}}{Indife\-ren\-te} & \multicolumn{1}{p{1.2cm}}{Concordo} & \multicolumn{1}{p{1.2cm}}{Concordo muito} \\
\midrule
E1- Os contextos educacionais devem ser colaborativos e cooperativos. & & & & 56,5 \% & 43,5 \% \\
E2- Uma educação emergente suscita o uso da tecnologia em seu contexto educacional. & & & 17,4 \% & 56,5 \% & 26,1 \% \\
E3- O contexto escolar em que estou inserido é favorável a uma educação emergente. & 4,3 \% & 8,7 \% & 21,7 \% & 56,5 \% & 8,7 \% \\
E4- O contexto escolar emergente suscita dos professores o conhecimento em neurociência. & & & 13,0 \% & 73,9 \% & 13,0 \% \\
E5- O contexto educacional em que estou inserido (a) contempla as diversidades dos alunos. & 4,3 \% & 8,7 \% & 8,7 & 65,2 \% & 13,0 \% \\
E6-O contexto educacional em que estou inserido contempla a pluralidade cultural dos alunos. & 4,3 \% & 8,7 \% & 13,0 \% & 60,9 \% & 13,0 \% \\
\bottomrule
\end{tabular}
\source{Elaborado pelas autoras.}
\end{threeparttable}
\end{table}

Foram apresentados aos participantes itens em referência à dimensão E, contextos educacionais emergentes. O objetivo 5 buscou analisar se o contexto educacional tem contribuído para uma educação emergente. Ao observar a \Cref{tab05}, percebemos que as escolhas dos itens foram muito diversificadas, porém o item E4 foi muito destacado com uma porcentagem de 73,9\%, em que os entrevistados concordam que o contexto escolar emergente suscita dos professores o conhecimento em neurociência. 65,25\% concordam que o contexto educacional em que estão inseridos contempla as diversidades dos alunos.  60,9\% dos entrevistados concordam com o item E6: O contexto educacional em que estou inserido contempla a pluralidade cultural dos alunos.

\section{Discussão}\label{sec-outras-estr}
Para início de nossa discussão, é importante relembrar o objetivo geral desta investigação, que é analisar a neuroeducação e a tecnologia como parceiras emergentes no processo de ensino-aprendizagem no contexto do século XXI. Ficou constatado que os professores valorizam essa temática, sabem e reconhecem que a atualização profissional é de extrema importância. Acreditam que a união da neuroeducação com a tecnologia produzirá efeitos positivos na educação, pois trará uma maior diversidade de metodologia e de recursos necessários ao bom desempenho dos alunos e professores. 

Portanto, ao analisarmos o primeiro objetivo, “1-Identificar os conhecimentos dos professores sobre a neuroeducação”, os professores opinam que a neuroeducação e a tecnologia no processo de aprendizagem são parceiras no contexto educacional do século XXI. Defendem a ideia de que os conhecimentos neuroeducacionais facilitam o aprendizado dos alunos e produz um aprendizado diferenciado e significativo. O conhecimento do funcionamento das atividades cerebrais faz parte dos estudos neurocientíficos; portanto, conhecer as temáticas da neuroeducação é uma forma de melhorar o desempenho do professor em vários aspectos, bem como a melhoria do aluno em seu processo de ensino-aprendizagem. Assim, a maioria dos entrevistados concordou e concordou muito com todos esses aspectos mencionados, como pode ser visto na \Cref{tab01}.

Seguindo nosso estudo, analisamos o objetivo específico 2, sobre especificar a função do conhecimento neurodidático dos professores para uma educação emergente. Podemos observar na \Cref{tab02} que a maioria dos entrevistados concordam e concordam muito que a neurodidática é relevante para o atual contexto educativo, que os conhecimentos neurodidáticos são essenciais para manter os professores atualizados. Portanto, ampliar os conhecimentos em face da compreensão cerebral no processo de ensino-aprendizagem permite ao professor melhorar e organizar suas práticas de modo que produzam atividades que desenvolvam e despertem nos alunos as práticas significativas. Nessa perspectiva, os participantes “concordam” e “concordam muito” com essas atitudes.

Na \Cref{tab03}, observamos o objetivo 3, que é detalhar a importância da tecnologia no processo de ensino-aprendizagem. Sendo assim, na dimensão C, tecnologia, observamos que, enquanto recurso tecnológico, ela facilita a prática pedagógica, é atrativa aos alunos, passando esses a participar melhor na aula, pois a aula flui de forma mais dinâmica. A tecnologia permite aos docentes melhorar suas metodologias, podendo renovar suas práticas de forma segura e dinâmica. Assim, de acordo com na \Cref{tab03}, prevaleceu a opção “concordo muito”.

Na \Cref{tab04}, analisamos o objetivo 4, que é descrever o processo de ensino-aprendizagem e sua relação com o procedimento neuroeducacionais. Tendo como base a dimensão ensino-aprendizagem, a maioria dos entrevistados concorda que é necessária a formação docente em neurociência para melhorar o processo de ensino-aprendizagem, que a formação docente em neurociência favorece o desempenho da aula, torna-os mais seguros, mas que, mesmo assim, demanda atualização em neurociência. 

Na \Cref{tab05}, o objetivo é analisar se o contexto educacional tem contribuído para uma educação emergente. Sob a ótica da dimensão E, contextos educacionais emergentes, a maioria dos entrevistados concorda que os contextos educacionais devem ser colaborativos e cooperativos, com a presença da tecnologia no contexto, com professores capacitados em neurociência e a contemplação das diversidades dos alunos.

Assim, os professores valorizam a importância dessa temática e acreditam que a atualização dos professores passa por uma formação significativa. A necessidade de uma mudança curricular converge para o que defendem \textcite{molina_formacion_2020} ao defenderem o aprimoramento das práticas inclusivas observando princípios, mudanças e evitando a exclusão. Portanto, na opinião dos participantes dessa pesquisa a neuroeducação contribui para aprimorar o ensino e a aprendizagem. Nesse intuito, a tecnologia faz uma parceria muito significativa nesse processo educacional.

\section{Conclusão}\label{sec-listas}
Ao discutir a temática em referência à neuroeducação e a tecnologia no cenário educacional atual, observamos que é um desafio iminente. Sabemos das dificuldades apresentadas mediante o uso da tecnologia na educação, uma realidade que permeia caminhos que apresentam falta de recursos, de equipamentos e de suportes técnicos para o bom desempenho de todos visando ao alcance dos objetivos esperados.
	
Pelo resultado do objetivo 1, que visa identificar os conhecimentos dos professores sobre a neuroeducação, ficou evidente que os professores valorizam os conhecimentos neuroeducacionais no aprendizado dos alunos, pois propiciam um aprendizado significativo, evidenciando o funcionamento das atividades cerebrais durante o processo de ensino-aprendizagem.

Em relação ao objetivo 2, ao especificar a função do conhecimento neurodidático dos professores para uma educação emergente, os participantes concordam que é de grande importância que os professores tenham um conhecimento neurodidático na atual contexto educativo, através da ampliação dos conhecimentos de compreensão cerebral. 

Para o objetivo 3, detalhar a importância da tecnologia no processo de ensino-aprendizagem, observou-se que, com os recursos tecnológicos, os alunos precisam ser mais participativos e exigem dos professores uma maior dinâmica no preparo das aulas.

Seguindo como objetivo 4, descrever o processo de ensino-aprendizagem e sua relação com os procedimentos neuroeducacionais, observou-se que no processo de ensino-aprendizagem é necessário o conhecimento em neurociência para superar dificuldades de ensino-aprendizagem.

No objetivo 5, analisar se o contexto educacional tem contribuído para uma educação emergente, obteve-se resultado positivo na relação entre tecnologia e neurociência contemplando a diversidade no processo de ensino-aprendizagem.

Dessa forma, concluímos que a hipótese é confirmada: há relação entre a neuroeducação e a tecnologia como parceira no processo de ensino aprendizagem. E a resposta da problemática é obvio: a neuroeducação e a tecnologia podem ser parceiras no processo de ensino-aprendizagem.

Dessa forma, notamos que o objetivo geral que move esta pesquisa foi alcançado, pois foram analisadas a neuroeducação e a tecnologia como parceiras emergentes no processo de ensino-aprendizagem no contexto do século XXI. Foram revisadas temáticas no âmbito da neurodidática. São necessárias transformações seguras e atuais preparando os profissionais para atualizá-los dentro da temática abordada. Portanto, concluímos que são necessárias mudanças profundas baseadas nos anseios contemporâneos e no preparo dos profissionais na área da educação e no conhecimento neuocientífico.


\printbibliography\label{sec-bib}
% if the text is not in Portuguese, it might be necessary to use the code below instead to print the correct ABNT abbreviations [s.n.], [s.l.]
%\begin{portuguese}
%\printbibliography[title={Bibliography}]
%\end{portuguese}


%full list: conceptualization,datacuration,formalanalysis,funding,investigation,methodology,projadm,resources,software,supervision,validation,visualization,writing,review
\begin{contributors}[sec-contributors]
\authorcontribution{Maria Selma Lima do Nascimento}[conceptualization,methodology,software]
\authorcontribution{Leonice da Silva Santos}[datacuration,writing]
\authorcontribution{Maria da Penha Cardoso}[visualization,investigation]
\authorcontribution{Márcia Morais de Melo}[supervision,review]
\end{contributors}


\end{document}

