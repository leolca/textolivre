% !TEX TS-program = XeLaTeX
% use the following command:
% all document files must be coded in UTF-8
\documentclass[french]{textolivre}
% build HTML with: make4ht -e build.lua -c textolivre.cfg -x -u article "fn-in,svg,pic-align"

\journalname{Texto Livre}
\thevolume{15}
%\thenumber{1} % old template
\theyear{2022}
\receiveddate{\DTMdisplaydate{2022}{7}{13}{-1}} % YYYY MM DD
\accepteddate{\DTMdisplaydate{2022}{8}{8}{-1}}
\publisheddate{\DTMdisplaydate{2022}{9}{29}{-1}}
\corrauthor{Cristina Pinto Díaz}
\articledoi{10.35699/1983-3652.2022.40458}
%\articleid{NNNN} % if the article ID is not the last 5 numbers of its DOI, provide it using \articleid{} commmand 
% list of available sesscions in the journal: articles, dossier, reports, essays, reviews, interviews, editorial
\articlesessionname{dossier}
\runningauthor{Pinto Díaz et Flores Melero} 
%\editorname{Leonardo Araújo} % old template
\sectioneditorname{Daniervelin Pereira}
\layouteditorname{Leonado Araújo}

\title{La neuroéducation et la technologie comme outil d'inclusion des groupes vulnérables, notamment les Roms}
\othertitle{A neuroeducação e a tecnologia como ferramenta de inclusão de grupos vulneráveis, especificamente a etnia cigana}
\othertitle{Neuroeducation and technology as a tool for the inclusion of vulnerable groups, specifically the gypsy ethnicity}
% if there is a third language title, add here:
%\othertitle{Artikelvorlage zur Einreichung beim Texto Livre Journal}

\author[1]{Cristina Pinto Díaz \orcid{0000-0003-0842-4003} \thanks{Email: \href{mailto:cpd00008@red.ujaen.es}{cpd00008@red.ujaen.es}}}
\author[1]{Carmen Flores Melero \orcid{0000-0002-7916-992X} \thanks{Email: \href{mailto:carmenfloresmelero17@gmail.com}{carmenfloresmelero17@gmail.com}}}
\affil[1]{Universidad de Jaén, Facultad de Humanidades y Ciencias de la Educación, Departamento Pedagogía, Jaén, España.}

\addbibresource{article.bib}
% use biber instead of bibtex
% $ biber article

% used to create dummy text for the template file
\definecolor{dark-gray}{gray}{0.35} % color used to display dummy texts
\usepackage{lipsum}
\SetLipsumParListSurrounders{\colorlet{oldcolor}{.}\color{dark-gray}}{\color{oldcolor}}

% used here only to provide the XeLaTeX and BibTeX logos
\usepackage{hologo}

% if you use multirows in a table, include the multirow package
\usepackage{multirow}

% provides sidewaysfigure environment
\usepackage{rotating}

% CUSTOM EPIGRAPH - BEGIN 
%%% https://tex.stackexchange.com/questions/193178/specific-epigraph-style
\usepackage{epigraph}
\renewcommand\textflush{flushright}
\makeatletter
\newlength\epitextskip
\pretocmd{\@epitext}{\em}{}{}
\apptocmd{\@epitext}{\em}{}{}
\patchcmd{\epigraph}{\@epitext{#1}\\}{\@epitext{#1}\\[\epitextskip]}{}{}
\makeatother
\setlength\epigraphrule{0pt}
\setlength\epitextskip{0.5ex}
\setlength\epigraphwidth{.7\textwidth}
% CUSTOM EPIGRAPH - END

% LANGUAGE - BEGIN
% ARABIC
% for languages that use special fonts, you must provide the typeface that will be used
% \setotherlanguage{arabic}
% \newfontfamily\arabicfont[Script=Arabic]{Amiri}
% \newfontfamily\arabicfontsf[Script=Arabic]{Amiri}
% \newfontfamily\arabicfonttt[Script=Arabic]{Amiri}
%
% in the article, to add arabic text use: \textlang{arabic}{ ... }
%
% RUSSIAN
% for russian text we also need to define fonts with support for Cyrillic script
% \usepackage{fontspec}
% \setotherlanguage{russian}
% \newfontfamily\cyrillicfont{Times New Roman}
% \newfontfamily\cyrillicfontsf{Times New Roman}[Script=Cyrillic]
% \newfontfamily\cyrillicfonttt{Times New Roman}[Script=Cyrillic]
%
% in the text use \begin{russian} ... \end{russian}
% LANGUAGE - END

% EMOJIS - BEGIN
% to use emoticons in your manuscript
% https://stackoverflow.com/questions/190145/how-to-insert-emoticons-in-latex/57076064
% using font Symbola, which has full support
% the font may be downloaded at:
% https://dn-works.com/ufas/
% add to preamble:
% \newfontfamily\Symbola{Symbola}
% in the text use:
% {\Symbola }
% EMOJIS - END

% LABEL REFERENCE TO DESCRIPTIVE LIST - BEGIN
% reference itens in a descriptive list using their labels instead of numbers
% insert the code below in the preambule:
%\makeatletter
%\let\orgdescriptionlabel\descriptionlabel
%\renewcommand*{\descriptionlabel}[1]{%
%  \let\orglabel\label
%  \let\label\@gobble
%  \phantomsection
%  \edef\@currentlabel{#1\unskip}%
%  \let\label\orglabel
%  \orgdescriptionlabel{#1}%
%}
%\makeatother
%
% in your document, use as illustraded here:
%\begin{description}
%  \item[first\label{itm1}] this is only an example;
%  % ...  add more items
%\end{description}
% LABEL REFERENCE TO DESCRIPTIVE LIST - END


% add line numbers for submission
%\usepackage{lineno}
%\linenumbers

\begin{document}
\maketitle

\begin{polyabstract}
\begin{abstract}
Cet article est un modèle qui vise à guider le personnel enseignant et non enseignant sur l'influence des neurosciences et de la technologie sur l'inclusion des groupes vulnérables, plus particulièrement les Roms. L'inclusion de ce groupe doit être abordée d'un point de vue éducatif et social, c'est-à-dire que le professionnel doit éduquer aux valeurs morales et éthiques, à l'empathie, à la solidarité et au respect. Et les enfants devraient être conscients que tout le monde devrait avoir accès à l'éducation. Les enseignants doivent donc se spécialiser dans l'inclusion et aborder leurs classes avec des méthodologies inclusives qui garantissent le bien-être de tous les élèves. Cela permettra d'éliminer la discrimination et les obstacles à l'apprentissage \cite{echeita_sarrionandia_pandemia_2020}. Il est essentiel de comprendre que la neuroéducation devient un outil méthodologique qui parvient à créer des environnements sensibles aux populations les plus vulnérables, indépendamment de leur race, de leur ethnie ou de leur diversité fonctionnelle.

\keywords{Inclusion \sep Technologie \sep Neuroscience \sep Groupes vulnérables \sep Gitan ethnique}
\end{abstract}

\begin{portuguese}
\begin{abstract}
Este artigo é um modelo que busca orientar os docentes e não docentes sobre a influência da neurociência e da tecnologia na inclusão dos grupos vulneráveis, mais especificamente da etnia cigana. A inclusão desse coletivo deve ser tratada do ponto de vista educativo e social, isto é, o profissional deve educar para os valores morais e éticos, para a empatia, a solidariedade e o respeito. E as crianças devem saber que todos devem ter acesso à educação. Portanto, o professor deve especializar-se em inclusão e abordar em suas aulas as metodologias inclusivas que garantam o bem-estar de todos os alunos. Isto eliminará a discriminação e os obstáculos à aprendizagem \cite{echeita_sarrionandia_pandemia_2020}. Torna-se imprescindível compreender que a Neuroeducação se converte em uma ferramenta metodológica que consegue criar ambientes sensíveis às populações mais vulneráveis, sem importar sua raça, etnia ou diversidade funcional.

\keywords{Inclusão \sep Tecnologia \sep Neurociência \sep Grupos vulneráveis \sep Etnia cigana}
\end{abstract}
\end{portuguese}

\begin{english}
\begin{abstract}
This article is a model that seeks to guide professors and non-professors about the influence of neuroscience and technology in the inclusion of vulnerable groups, more specifically the Roma. The inclusion of this collective must be treated from an educational and social point of view, that is, the professional must educate for moral and ethical values, for empathy, solidarity and respect. And children should know that everyone should have access to education. Therefore, the teacher must specialize in inclusion and address in their classes inclusive methodologies that guarantee the well-being of all students. This will eliminate discrimination and obstacles to learning \cite{echeita_sarrionandia_pandemia_2020}. It is essential to understand that Neuroeducation becomes a methodological tool that manages to create sensitive environments for the most vulnerable populations, regardless of their race, ethnicity or functional diversity.

\keywords{Inclusion \sep Technology \sep Neuroscience \sep Vulnerable groups \sep Gypsy ethnicity}
\end{abstract}
\end{english}
% if there is another abstract, insert it here using the same scheme
\end{polyabstract}

\section{Introduction}\label{sec-intro}
Études et recherches sur les neurosciences et leur contribution bénéfique à l'éducation \cite{ruiz_diaz_aportes_2020}.

Les neurosciences visent à fournir à l'enseignement de nouveaux outils pédagogiques afin de garantir la qualité et de donner aux étudiants la possibilité de développer leur esprit critique \cite{rosell_aiquel_neurociencia_2020}.

Les neurosciences sont étroitement liées à l'éducation, c'est pourquoi il est nécessaire d'établir une discipline intermédiaire qui facilite les connexions entre les deux sciences \cite{carnine_professional_1995}. Par conséquent, les neurosciences sont très applicables à l'éducation avec la théorie de l'apprentissage basé sur le cerveau \cite{rosell_aiquel_neurociencia_2020}.

La neuroéducation est une science récente qui relie les neurosciences à l'éducation \cite{hernández_inclusion_2021}.

En d'autres termes, la neuroéducation se base sur la compréhension du fonctionnement du cerveau et de ses fonctions dans le but d'autonomiser les qualités de chaque personne et d'adapter davantage le processus d'enseignement-apprentissage \cite{orbe_aplicacion_2019}.

Par conséquent, la neuroéducation intègre les connaissances de différentes sciences telles que les neurosciences, la psychologie et la pédagogie \cite{hernandez_martinez_neuroeducacion_2020}.

Selon \textcite{romero_rodriguez_tecnologias_2020}, les technologies d’apprentissage sont essentielles pour l’enseignement et l’apprentissage des étudiants au XXI siècle.  L'Espagne dispose d'une institution spécialisée dans ce domaine, l'Institut national des technologies éducatives et de la formation des enseignants (INTEF), un organisme dépendant du ministère de l'Éducation et de la formation professionnelle (MEPF-Espagne), qui est responsable de l'intégration des technologies de l’information et la communication (TIC) dans l'éducation et de la formation des enseignants travaillant dans les écoles.

En termes d'accès aux TIC, selon \textcite{acevedo_ciencia_1998}, dans les écoles d'Espagne, les données indiquent que 96,7 \% des écoles disposent d'une connexion Internet et 60,1 \% d'appareils numériques pour soutenir le processus d'enseignement et d'apprentissage, tels que des projecteurs, des panneaux interactifs, des tables tactiles, des télévisions interactives, entre autres \cite{mepf-espana_datos_2021}. Parallèlement, le taux d'utilisateurs par ordinateur1 dans les écoles publiques est de 2,8 \cite{mepf-espana_datos_2021}, un résultat qui, selon le rapport PISA 2018, indique que l'Espagne se situe dans la moyenne des pays de l'OCDE et de l'Union européenne \cite{mepf-espana_pisa_2020}. Parallèlement, le taux d'ordinateurs par élève est plus élevé dans les écoles favorisées (1,01) que dans les écoles défavorisées (0,74), bien que la différence (0,27) ne soit pas statistiquement significative \cite{mepf-espana_pisa_2020}.

Tout au long de l'histoire, les enseignants qui travaillent avec des élèves nécessitant des mesures compensatoires et supplémentaires ont reçu une formation de plus en plus poussée, les méthodologies sont actualisées et améliorées au fur et à mesure que les besoins existants sont connus. \textcite{carmona_santiago_colaboracion_2021}, expliquent l'importance de lier les concepts de vulnérabilité, de soins et de droits de l'homme en termes de l'ethnie Rom, pour finalement montrer comment des études récentes démontrent l'importance de l'éducation dès le plus jeune âge pour l'inclusion correcte et le respect des droits de ce groupe par le reste.

L'intervention auprès des Roms peut être réalisée à l'aide de différentes méthodologies, cependant, afin de développer une intervention compensatoire à partir de la neuroéducation, il est nécessaire d'approfondir ce concept. Il est indispensable d'intervenir auprès de ce groupe, car la société stigmatise souvent les Roms. Des auteurs tels que \textcite{blasco_education_2006}, affirment que le fait d'appartenir à ce groupe ethnique entraîne la discrimination des jeunes, en particulier des filles roms. Ainsi, grâce aux TIC, il est possible de promouvoir l’apprentissage et l’inclusion des élèves appartenant à des groupes défavorisés \cite{ramos-navas-parejo_uso_2020}.

En Espagne, il existe un groupe ethnique rom qui représente une minorité. Ce groupe ethnique présente des caractéristiques économiques, sociales, culturelles, éducatives et sanitaires qui le différencient du reste de la population et exposent ses membres à un risque réel en termes sociaux, sanitaires et éducatifs. De même, les familles immigrées qui arrivent dans notre pays présentent une situation similaire, qu'elles soient régulières ou irrégulières, et constituent également un risque socio-sanitaire et nécessitent des mesures de protection sociale plus fréquemment que le reste de la population \cite{olivan-gonzalvo_comparacion_2004}.

Dans le contexte scolaire, la présence d'élèves immigrés ou issus d'autres groupes ethniques n'est pas récente, cependant, peu d'études étudient ce groupe en profondeur. Heureusement, notre société actuelle n'est pas homogène. Au sein des sociétés, de nombreuses personnes vivent ensemble et sont représentées dans différentes cultures, groupes ethniques, rangs, etc. Les enfants roms ainsi que les fils et les filles d'immigrants qui sont arrivés dans notre pays en fuyant quelque chose de pire coexistent avec les autochtones du lieu, devenant eux-mêmes des résidents, des voisins, des membres de la famille et des amis \cite{benavidez_importancia_2019}.

L'inclusion doit être abordée d'un point de vue éducatif et social, c'est-à-dire que l'enseignant doit éduquer aux valeurs morales et éthiques, à l'empathie, à la solidarité et au respect. Et les enfants doivent être conscients que tous les élèves doivent avoir accès à l'éducation. Les enseignants doivent donc se spécialiser dans l'inclusion et aborder leurs classes avec des méthodologies inclusives qui garantissent le bien-être de tous les élèves. Cela permettra d'éliminer la discrimination et les obstacles à l'apprentissage \cite{echeita_sarrionandia_pandemia_2020}.

Il convient de tenir compte de contributions telles que celles de \textcite{lacruz-perez_teachers_2021}, qui soulignent la nécessité d'une plus grande diversité méthodologique afin d'obtenir une analyse plus complète des attitudes des enseignants à l'égard de l'inclusion et de la formation de leurs élèves, cette dernière étant l'un des outils les plus puissants pour la création d'attitudes positives envers la diversité. Il est possible de créer une méthodologie générale qui s'adapte aux besoins spécifiques de chaque individu. Cette porte n'est pas seulement ouverte aux enseignants, mais aussi aux chercheurs, neuroscientifiques, thérapeutes, psychologues, informaticiens, etc. Ils peuvent partager ces connaissances et les inclure dans leurs propres thérapies, en les adaptant à leurs propres utilisateurs afin d'atteindre l'objectif commun à tous : l'amélioration de la qualité de vie des personnes issues d'autres cultures.

Dans le contexte éducatif, il est essentiel que le personnel enseignant et non enseignant projette un développement émotionnel adéquat, ainsi qu'une réponse sociale, morale et éthique correspondant aux valeurs de la société de notre siècle, jouant un rôle important dans l'inclusion adéquate des adolescents immigrés et des minorités ethniques \cite{crockett_acculturative_2007}.

Les adolescents de notre pays vivent dans des environnements où la différence culturelle est une réalité quotidienne. Cependant, même s'ils ont vécu pendant des années avec des races et des cultures différentes, cette coexistence multiculturelle peut être conflictuelle à l'intérieur et à l'extérieur de la classe. Le conflit désigne toute situation scolaire dans laquelle il existe une divergence entre deux ou plusieurs membres de groupes culturels différents vivant ensemble dans une école. Ces conflits peuvent être dus à une mauvaise communication entre les familles, à une mauvaise information de la part de l'école, à des malentendus dus à l'utilisation de codes de communication différents, à des différences de pensée sur les plans affectif, politique, éthique, religieux, etc. \cite{leiva_olivencia_conflictos_2007}.

Dans une étude réalisée à l'université de Malaga par \textcite{hombrados-mendieta_apoyo_2013}, un échantillon de 512 élèves adolescents a été constitué, dans lequel coexistaient des élèves adolescents autochtones, immigrés et roms. Les résultats montrent des différences notables entre eux en fonction de leur origine. Il existe des difficultés de coexistence dans la classe en raison des différences ethniques et culturelles, bien que ces relations ne soient pas significatives pour le groupe ethnique rom. La nécessité d'analyser la base de l'enseignement et du soutien à partir du contexte scolaire, comme l'indiquent les modèles écologiques et systémiques est confirmée.

Il est indiqué que les élèves adolescents, en plus de lutter contre les changements physiques et émotionnels de l'adolescence, sont confrontés à des événements stressants qui accompagnent leurs caractéristiques culturelles ou ethniques. Se démarque l'importance d'une intervention éducative appropriée pour un processus correct d'inclusion des personnes non indigènes, ainsi que l'importance de savoir tolérer les autres et accepter les différences \cite{graham_peer_2006}. Enfin, \textcite{hombrados-mendieta_apoyo_2013} concluent que le soutien social dont peuvent bénéficier les adolescents roms et immigrés améliore le climat scolaire et réduit les conflits. Sachant cela, disent-ils, des stratégies d'intervention peuvent être développées pour améliorer les relations sociales multiculturelles. Ils soulignent également l'importance de former et d'informer tant les étudiants que l'ensemble de la communauté éducative (personnel enseignant et non enseignant, familles et élèves) sur la réalité qui existe dans les classes, afin de pouvoir travailler dans une même ligne pédagogique sur la diversité de notre société.

La situation éducative des élèves roms en termes de fréquentation scolaire est la suivante: environ 80 \% des élèves roms fréquentent la première année de l'ESO, mais ils abandonnent avant la fin de la quatrième année. D'autre part, bien que les filles roms soient moins scolarisées dans l'ESO, elles ont des taux d'abandon plus faibles \cite{andres_imagenes_2006}.

D'autre part, certains résultats de recherche concluent que les élèves roms ont un statut social faible parmi leurs pairs du groupe majoritaire (le groupe indigène). Cependant, il est indiqué que l'interaction sociale des enseignants est positive, bien qu'il y ait un petit pourcentage d'enseignants qui ont des attentes académiques plus faibles par rapport aux élèves non roms \cite{leon_atencion_2008}.

D'autre part, une étude de \textcite{battro_neuroeducacion:_2012} affirme que leurs résultats nous permettent d'assurer qu'il existe deux types de personnes avec des tendances comportementales différentes qui, en combinant les stéréotypes de moralité et les réactions émotionnelles, peuvent être exposés comme ceux qui font une évaluation négative des Roms (hommes et femmes) et les autres qui n'ont pas tendance à juger. Ce sont les premiers qui devraient être ciblés afin de dissiper la discrimination à l'égard des Roms.

À l'heure actuelle, l'inclusion n'a pas encore été atteinte, car la société se trouve dans une période d'évolution du modèle d'intégration au modèle d'inclusion \cite{higueras_active_2020}. Cependant, il est possible d'affirmer qu'il existe une préoccupation sociale pour aller vers l'inclusion, puisque les principes éducatifs exigent que l'éducation soit de qualité, inclusive et universelle, c'est-à-dire avec une culture inclusive, égalitaire et respectueuse \cite{florian_enhancing_2020}. En plus d'une telle culture inclusive, des politiques et des pratiques inclusives seront nécessaires \cite{tzuc_salinas_inclusion_2022}.

L'inclusion doit être abordée d'un point de vue éducatif et social, c'est-à-dire que l'enseignant doit éduquer aux valeurs morales et éthiques, à l'empathie, à la solidarité et au respect. Et les enfants doivent être conscients que tous les élèves doivent avoir accès à l'éducation. Les enseignants doivent donc se spécialiser dans l'inclusion et aborder leurs classes avec des méthodologies inclusives qui garantissent le bien-être de tous les élèves. Cela permettra d'éliminer la discrimination et les obstacles à l'apprentissage \cite{echeita_sarrionandia_pandemia_2020}.

Il convient de tenir compte de contributions telles que celles de \textcite{lacruz-perez_teachers_2021}, qui soulignent la nécessité d'une plus grande diversité méthodologique afin d'obtenir une analyse plus complète des attitudes des enseignants à l'égard de l'inclusion et de la formation de leurs élèves, cette dernière étant l'un des outils les plus puissants pour la création d'attitudes positives envers la diversité. Il est possible de créer une méthodologie générale qui s'adapte aux besoins spécifiques de chaque individu. Cette porte n'est pas seulement ouverte aux enseignants, mais aussi aux chercheurs, neuroscientifiques, thérapeutes, psychologues, informaticiens, etc. Ils peuvent partager ces connaissances et les inclure dans leurs propres thérapies, en les adaptant à leurs propres utilisateurs afin d'atteindre l'objectif commun à tous: l'amélioration de la qualité de vie des personnes handicapées.

\section{Méthode}\label{sec-normas}
Le problème auquel nous sommes confrontés dans cette recherche est le suivant: comment mettre en œuvre l'inclusion des groupes vulnérables, notamment les Roms, par le biais de la neuroéducation et de la technologie?

L'objectif principal de cette recherche est d'analyser l'influence de la neuroéducation sur les groupes vulnérables, plus spécifiquement sur les Roms en vue de leur inclusion par la technologie.

L'hypothèse nulle suivante est définie : H1.-Il est possible d'inclure le groupe ethnique rom par le biais de la neuroéducation et de la technologie.

La population a été considérée comme étant les étudiants de quatrième année de la licence en éducation primaire de l'Universidad Autónoma de Madrid (315). Pour calculer la taille de l'échantillon composé d'étudiants du degré d'enseignement primaire, la formule proposée par \textcite{murray_estadistica_2005} est appliquée, le résultat est de 306 sujets du degré d'enseignement primaire, sélectionnés par commodité.

Cette recherche se caractérise par un plan non expérimental, exploratoire, descriptif et corrélationnel, et sera menée à l'aide d'une méthodologie quantitative. Une échelle de Likert a été utilisée pour la collecte des données. Le logiciel utilisé était le progiciel statistique SPSS v.25.

Pour la collecte des données, une matrice d'opérationnalisation a été préalablement élaborée dans laquelle les différentes variables, items et unités de mesure ont été exprimés \cite{mejia_metodologiinvestigacion_2005}. Celle-ci a été utilisée pour construire une échelle de Likert pour la collecte des données, comprenant 50 items, regroupés en cinq dimensions (extraites des objectifs spécifiques) : A (Neuroéducation), B (Technologie), C (Inclusion), D (Groupes vulnérables) et E (Ethnicité rom).

\section{Résultats}\label{sec-conduta}
Afin de déterminer la validité du contenu, une étude de validité du contenu a d'abord été réalisée avec la participation de spécialistes titulaires d'un doctorat de différentes universités \cite{malla_prevision_1978} autorisés pour cette évaluation, le coefficient de connaissance ou d'information (Kc) et le coefficient d'argumentation (Ka) ont été calculés pour ces médecins, puis la valeur du coefficient de compétence (K) a été calculée, ce qui a déterminé quels experts ont été pris en considération pour cette recherche, le résultat a été quinze spécialistes avec un K moyen de 0,9, ce qui montre un haut niveau de compétence \cite{mengual_importancia_2011}. Après analyse des questionnaires de validation, certaines questions ont été réajustées, sans que le fond de la question ne soit affecté. En outre, un test pilote a été réalisé sur un sous-groupe de l'échantillon afin de vérifier les difficultés de compréhension, d'identifier les questions qui généraient des doutes, etc., en utilisant la liste de contrôle correspondante \cite{iraossi_poder_2006}. Les résultats du test pilote ont été satisfaisants et l'instrument a été validé au niveau de son contenu.
 
\subsection{Analyse factorielle exploratoire}
 
La procédure d'analyse factorielle que nous avons appliquée dans notre recherche est conforme aux règles marquées par les étapes suivantes \cite{diaz-de-rada_tecnicas_2002}:

La matrice de corrélation doit être étudiée pour voir si nos données se prêtent à une analyse factorielle. Pour ce faire, cette matrice doit avoir une certaine structure. Pour confirmer cela, la mesure de l'adéquation de l'échantillonnage de Kaiser-Meyer-Olkin (coefficient KMO) a été utilisée, dans notre cas la valeur est de .745, donc nous continuons avec l'analyse.

Le tableau des communalités qui en résulte montre que les facteurs ont une valeur supérieure à .646, alors il ne faut pas éliminer des items de l'analyse factorielle. Les articles les mieux représentés sont : B13 Les technologies permettent l'accès à l'information (.937) et C22 L'inclusion enrichit tous les apprenants (.922).

Pour effectuer les rotations, il existe plusieurs procédures selon le critère d'optimalité. L'une d'entre elles est la rotation Varimax qui augmente les charges factorielles de manière à ce qu'elles soient aussi extrêmes que possible dans les facteurs (élevées et faibles). Il existe des règles pour connaître le nombre le plus approprié de facteurs à conserver, par exemple, celle connue sous le nom de critère de \textcite{kaiser_index_1974}, dans notre cas, il s'agit des 6 premiers facteurs, qui expliquent 81,783\% de la variance accumulée.

Le coefficient Alpha de Cronbach présente une cohérence interne élevée des 50 variables en raison du fait qu'elles présentent une valeur de α = .989, ce qui est une excellente valeur.

Pour effectuer la corrélation, nous soumettons l'échelle de Likert au test de Kruskal-Wallis, qui nous donne comme résultat de retenir l'hypothèse nulle que les données suivent une distribution anormale, utilisant ainsi la corrélation Rho de Spearman.

Des corrélations significatives ont été trouvées :

A (neuroéducation) <> B (technologie) : selon les données obtenues, la variable A de la neuroéducation présente une corrélation significative avec la variable B de la technologie, réciproquement à .639.

C (inclusion) > B (technologie) : selon les données obtenues, la variable C sur l'inclusion est corrélée avec la variable B sur la technologie à .630.

D (groupes vulnérables) <> E (ethnicité rom) : selon les données obtenues, la variable D sur les groupes vulnérables présente une corrélation significative avec la variable E sur l'ethnicité rom, réciproquement à .744.
 
\subsection{Analyse descriptive}
 
Dimension A (neuroéducation) : les participants sont d'accord (=3,57) pour dire que les neurosciences s'attachent à comprendre le fonctionnement biologique du cerveau à travers les réponses aux circonstances vécues par les individus, et que la neuroéducation favorise la motivation pour l'apprentissage des élèves.

Dimension B (Technologie) : les répondants sont d'accord (=4.16) pour dire que les technologies augmentent le sens de l'initiative et l'esprit d'entreprise. Cependant, ils ne sont pas d'accord (=2,80) pour dire que les technologies augmentent la motivation des étudiants.

Dimension C (inclusion) : les sujets sont tout à fait d'accord pour dire que les environnements inclusifs doivent être adaptés aux apprenants (=4.54).

Dimension D (groupes vulnérables) : les répondants sont indifférents - d'accord (=3,93) pour dire que les groupes vulnérables participent activement aux activités sociales.

Dimension E (ethnicité rom) : les répondants sont indifféremment d'accord (=3,76) pour dire que les Roms bénéficieraient d'un accès à des informations utiles sur les habitudes de vie saines dans les médias et les informations.


\section{Discussion et conclusion}\label{sec-fmt-manuscrito}
La recherche présentée a été conçue pour une population d'étudiants de l'enseignement primaire. Dans une recherche future, il serait acceptable de réaliser cette recherche avec des étudiants en Master, afin de corréler les données à différents niveaux universitaires. Il serait également utile d’apporter  cette recherche auprès des professionnels de l'éducation dans les écoles pour corréler les données entre les étudiants et les professionnels.

L’objectif de cette recherche est d'analyser l'influence de la neuroéducation sur les groupes vulnérables, plus spécifiquement sur le groupe ethnique des Roms pour l'inclusion par la technologie. Afin d'atteindre cet objectif, une échelle de Likert a été conçue à l'aide d'un tableau d'opérationnalisation. Une analyse de fiabilité a été réalisée, donnant un excellent résultat, selon l'alpha de Cronbach (.989). La validation de l'échelle est effectuée par une analyse factorielle exploratoire (KMO (.745), qui confirme les dimensions de notre étude, sans réduire l'échelle.

Ensuite, après avoir vérifié avec le test de Kruskal-Wallis que la distribution des données est anormale, la corrélation Rho de Spearman est calculée. En termes généraux, nous pouvons souligner que les corrélations les plus significatives se situent principalement entre les dimensions D (groupes vulnérables) et E (ethnicité rom).

L'analyse descriptive nous permet d'apprécier que les sujets enquêtés sont tout à fait d'accord pour que les contextes d'inclusion soient adaptés aux étudiants. En revanche, ils sont peu d'accord - indifférents (=2,80) - pour dire que les technologies augmentent la motivation des élèves.


\printbibliography\label{sec-bib}
% if the text is not in Portuguese, it might be necessary to use the code below instead to print the correct ABNT abbreviations [s.n.], [s.l.]
%\begin{portuguese}
%\printbibliography[title={Bibliography}]
%\end{portuguese}


%full list: conceptualization,datacuration,formalanalysis,funding,investigation,methodology,projadm,resources,software,supervision,validation,visualization,writing,review
\begin{contributors}[sec-contributors]
\authorcontribution{Cristina Pinto Díaz}[conceptualization,datacuration,formalanalysis,investigation,methodology,software,validation,visualization,writing,review]
\authorcontribution{Carmen Flores Melero}[methodology,projadm,resources,validation,writing,review]
\end{contributors}



\end{document}

