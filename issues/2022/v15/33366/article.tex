% !TEX TS-program = XeLaTeX
% use the following command:
% all document files must be coded in UTF-8
\documentclass[portuguese]{textolivre}
% build HTML with: make4ht -e build.lua -c textolivre.cfg -x -u article "fn-in,svg,pic-align"

\journalname{Texto Livre}
\thevolume{15}
%\thenumber{1} % old template
\theyear{2022}
\receiveddate{\DTMdisplaydate{2021}{4}{23}{-1}} % YYYY MM DD
\accepteddate{\DTMdisplaydate{2021}{5}{18}{-1}}
\publisheddate{\DTMdisplaydate{2021}{10}{27}{-1}}
\corrauthor{Sandra Miranda}
\articledoi{10.35699/1983-3652.2022.33366}
%\articleid{NNNN} % if the article ID is not the last 5 numbers of its DOI, provide it using \articleid{} commmand
\runningauthor{Miranda et al.} 
%\editorname{Leonardo Araújo} % old template
\sectioneditorname{Daniervelin Pereira}
\layouteditorname{Leonado Araújo}

\title{Navegar por redes nunca antes navegadas: motivações e engagement dos seniores ibéricos}
\othertitle{Browse networks never browsed before: motivations and engagement of the Iberian seniors}
% if there is a third language title, add here:
%\othertitle{Artikelvorlage zur Einreichung beim Texto Livre Journal}

\author[1]{Sandra Miranda~\orcid{0000-0002-5544-5942} \thanks{Email: \url{smiranda@escs.ipl.pt}}}
\author[2]{Ana Cristina Antunes~\orcid{0000-0001-8983-2062} \thanks{Email: \url{aantunes@escs.ipl.pt}}}
\author[3]{Ana Gama~\orcid{0000-0002-0647-9820} \thanks{Email: \url{agama@eselx.ipl.pt}}}
\author[4]{Branco Di Fatima~\orcid{0000-0001-6981-7228} \thanks{Email: \url{brancodifatima@gmail.com}}}
\author[5]{Ana Teresa Machado~\orcid{0000-0002-0415-2058} \thanks{Email: \url{amachado@escs.ipl.pt}}}

\affil[1]{Centro de Investigação e Estudos de Sociologia (CIES- Iscte); Instituto Politécnico de Lisboa, Escola Superior de Comunicação Social, Departamento de Ciências Sociais, Lisboa, Portugal.}

\affil[2]{Instituto Politécnico de Lisboa, Escola Superior de Comunicação Social, Departamento de Ciências Humanas, Lisboa, Portugal.}

\affil[3]{Instituto Politécnico de Lisboa, Escola Superior de Educação, Lisboa, Portugal.}

\affil[4]{Instituto Universitário de Lisboa, Centro de Investigação e Estudos de Sociologia, Lisboa, Portugal.}

\affil[5]{Centro de investigação, desenvolvimento e inovação no Turismo (CiTUR); Instituto Universitário de Lisboa, Escola Superior de Comunicação Social, Departamento de Estudos em Publicidade e Marketing, Lisboa, Portugal.}

\addbibresource{article.bib}
% use biber instead of bibtex
% $ biber article

% used to create dummy text for the template file
\definecolor{dark-gray}{gray}{0.35} % color used to display dummy texts
\usepackage{lipsum}
\SetLipsumParListSurrounders{\colorlet{oldcolor}{.}\color{dark-gray}}{\color{oldcolor}}

% used here only to provide the XeLaTeX and BibTeX logos
\usepackage{hologo}

% if you use multirows in a table, include the multirow package
\usepackage{multirow}

% provides sidewaysfigure environment
\usepackage{rotating}

% CUSTOM EPIGRAPH - BEGIN 
%%% https://tex.stackexchange.com/questions/193178/specific-epigraph-style
\usepackage{epigraph}
\renewcommand\textflush{flushright}
\makeatletter
\newlength\epitextskip
\pretocmd{\@epitext}{\em}{}{}
\apptocmd{\@epitext}{\em}{}{}
\patchcmd{\epigraph}{\@epitext{#1}\\}{\@epitext{#1}\\[\epitextskip]}{}{}
\makeatother
\setlength\epigraphrule{0pt}
\setlength\epitextskip{0.5ex}
\setlength\epigraphwidth{.7\textwidth}
% CUSTOM EPIGRAPH - END

% LANGUAGE - BEGIN
% ARABIC
% for languages that use special fonts, you must provide the typeface that will be used
% \setotherlanguage{arabic}
% \newfontfamily\arabicfont[Script=Arabic]{Amiri}
% \newfontfamily\arabicfontsf[Script=Arabic]{Amiri}
% \newfontfamily\arabicfonttt[Script=Arabic]{Amiri}
%
% in the article, to add arabic text use: \textlang{arabic}{ ... }
%
% RUSSIAN
% for russian text we also need to define fonts with support for Cyrillic script
% \usepackage{fontspec}
% \setotherlanguage{russian}
% \newfontfamily\cyrillicfont{Times New Roman}
% \newfontfamily\cyrillicfontsf{Times New Roman}[Script=Cyrillic]
% \newfontfamily\cyrillicfonttt{Times New Roman}[Script=Cyrillic]
%
% in the text use \begin{russian} ... \end{russian}
% LANGUAGE - END

% EMOJIS - BEGIN
% to use emoticons in your manuscript
% https://stackoverflow.com/questions/190145/how-to-insert-emoticons-in-latex/57076064
% using font Symbola, which has full support
% the font may be downloaded at:
% https://dn-works.com/ufas/
% add to preamble:
% \newfontfamily\Symbola{Symbola}
% in the text use:
% {\Symbola }
% EMOJIS - END

% LABEL REFERENCE TO DESCRIPTIVE LIST - BEGIN
% reference itens in a descriptive list using their labels instead of numbers
% insert the code below in the preambule:
%\makeatletter
%\let\orgdescriptionlabel\descriptionlabel
%\renewcommand*{\descriptionlabel}[1]{%
%  \let\orglabel\label
%  \let\label\@gobble
%  \phantomsection
%  \edef\@currentlabel{#1\unskip}%
%  \let\label\orglabel
%  \orgdescriptionlabel{#1}%
%}
%\makeatother
%
% in your document, use as illustraded here:
%\begin{description}
%  \item[first\label{itm1}] this is only an example;
%  % ...  add more items
%\end{description}
% LABEL REFERENCE TO DESCRIPTIVE LIST - END


% add line numbers for submission
%\usepackage{lineno}
%\linenumbers

\begin{document}
\maketitle

\begin{polyabstract}
\begin{abstract}
A sociedade ocidental contemporânea é caracterizada pelo envelhecimento demográfico, pelo empoderamento do consumidor sénior (65+) e pela popularização das tecnologias digitais, sobretudo da internet e das mídias sociais. Porém, o conhecimento científico sobre a relação dos seniores com as plataformas digitais ainda apresenta numerosas lacunas, principalmente na Península Ibérica – uma das regiões mais envelhecidas do planeta. Este artigo problematiza essa relação, em Portugal e na Espanha, escrutinando os motivos de adesão e permanência dos mais velhos as mídias sociais, aferindo quais as páginas das marcas (na rede social Facebook) os seniores mais visitam e com as quais mais interagem e descortinando o papel das \textit{fanpages} das marcas no processo de tomada de decisão de consumo. A investigação, de caráter exploratório, seguiu uma metodologia mista. Os dados qualitativos foram obtidos a partir de cinco grupos focais (\textit{Focus Group} – FG) realizados nos dois países em estudo. Já os dados quantitativos foram extraídos do Facebook Audience Insights, a partir de uma amostra oficialmente representativa dos seniores na região. Os resultados mostram evidentes sinais de “fratura digital” com os seniores a configurarem apenas 17,8\% (10\% PT e 7,8\% ES) dos perfis ativos na Península Ibérica, destacando-se uma maior presença das mulheres. Os portugueses tendem a gostar, comentar e partilhar conteúdos com mais frequência do que os espanhóis. No que tange aos os motivos de adesão e permanência na rede social, verifica-se unanimidade ibérica na indicação do incentivo dos familiares próximos, tal como na constatação de que se trata de um espaço potenciador do reforço e da multiplicação dos laços sociais. A procura e o envolvimento com as páginas de marcas comerciais serve, sobretudo, como ferramenta de apoio e redução da incerteza no processo de tomada de decisão de compra \textit{on} e \textit{offline}.

\keywords{Seniores \sep Tecnologias de informação e comunicação (TIC) \sep Social media \sep Facebook \sep Páginas das marcas \sep Península Ibérica}
\end{abstract}

\begin{english}
\begin{abstract}
Contemporary western societies are characterized by demographic aging, by the empowerment of senior consumers (aged 65+) and by the popularization of digital technologies, especially the internet and social media. However, scientific knowledge about the relationship between seniors and digital platforms still presents numerous gaps, mainly in the Iberian Peninsula - one of the most aged regions on the planet. This article problematizes this relationship in Portugal and Spain, examining the motives for adherence and use of social media by older people, assessing which brand pages (on Facebook) are most visited and interacted with by seniors, and unveils the role of Facebook fan pages brands in the consumer decision-making process. The research uses a mixed method approach. Qualitative data were obtained from five Focus Group (FG) carried out in the two countries. The quantitative data were extracted from Facebook Audience Insights, from an officially representative sample of seniors in the region. The results show evident signs of a “digital fracture” with the seniors representing only 17.8\% (10\% PT and 7.8\% ES) of the active profiles in the Iberian Peninsula, with a higher presence of women. Portuguese seniors reveal a tendency to like, comment and share content more often than Spaniards. About the reasons for joining and staying in social networking sites, there is a unanimous Iberian stance indicating the encouragement of close family members, as well as in the finding that it is a space that enhances the strengthening and multiplication of social ties. Search and involvement with Facebook brand pages count for support and uncertainty reduction tool in the consumer journey.


\keywords{Seniors \sep Information and communication technologies (ICT) \sep Social media \sep Facebook \sep Brand pages \sep Iberian Peninsula}
\end{abstract}
\end{english}
% if there is another abstract, insert it here using the same scheme
\end{polyabstract}

\section{Introdução}\label{sec-intro}
Uma das tendências marcantes das sociedades contemporâneas é o envelhecimento demográfico, que regista uma evolução sem precedentes na história da humanidade \cite{eurostat2019, onu2021}. Atualmente 1 em cada 9 pessoas têm mais de 60 anos e estima-se que esse rácio passe para 1 em cada 5 indivíduos em 2050 \cite{onu2021}. Na Península Ibérica, este processo tem sido particularmente rápido e generalizado, verificando-se que esta apresenta um dos maiores índices de envelhecimento no contexto europeu e mundial \cite{european_commission2021}. A análise da evolução demográfica, em Portugal e Espanha, revela alterações significativas na sua estrutura e composição \cite{bandeira2014, ine2017, ine2020}, registando-se uma dinâmica progressiva de redução da taxa de fecundidade e uma tendência constante para o aumento da população envelhecida.

À medida que o processo planetário de envelhecimento da população se acentua, observa-se um contínuo e acelerado desenvolvimento tecnológico, visível, entre outros aspectos, na evolução das Tecnologias de Informação e Comunicação (TIC) e no surgimento dos chamados media sociais. Apesar do uso das TIC e da internet ser pervasivo e estar imbricado no tecido das nossas próprias vidas, bem como dos seus múltiplos papéis e benefícios para as diversas faixas etárias \cite{castells2009, barroso2015, gil2016}, o conhecimento relativo à relação das gerações mais velhas com as TIC e com os media sociais ainda apresenta numerosas lacunas \cite{givskov2018}.

A literatura da especialidade recai em particular na chamada fractura digital, destacando não somente as desigualdades de acesso e uso das TIC e da internet por parte dos seniores, mas apresentando-os como iliteratos digitais, na medida em que denotam um menor domínio dos conhecimentos e competências que as TIC e a internet requerem \cite{friemel2016, francis2019}. Sendo ainda uma realidade que são os cidadãos seniores que apresentam as menores taxas de acesso e de uso das tecnologias digitais \cite{eurostat2019, eurostat2020, statista2020}. A análise das tendências mais recentes revela um movimento evolutivo positivo e a progressiva domesticação e adesão das gerações mais velhas às TIC, sugerindo que esta fractura digital não se trata de um fenómeno imutável nem incontornável.

O panorama atual torna evidente que os níveis de acesso, a compreensão e o uso, bem como a literacia e as competências digitais de muitos cidadãos seniores são ainda limitadas e comportam uma ampla margem de progressão. Todavia, a palavra de ordem parece ser a heterogeneidade: estudos recentes sugerem uma enorme variabilidade em termos das suas atitudes e uso de computadores, tecnologia móvel ou mesmo redes sociais, bem como níveis de proficiência e de literacia digital muito distintos no seio das gerações mais velhas \cite{chopik2016, coelho2019, miranda2020, van_boekel2017}. Uma das áreas na qual verifica uma relação cada vez mais estreita dos seniores com as TIC diz respeito aos media sociais \cite{hutto2015, yang2016}, registando-se não apenas um crescimento na quantidade de utilizadores, como um aumento do número de horas diárias despendidas nestas plataformas digitais. Tendo presente o papel fundamental das motivações enquanto preditoras e impulsionadoras do comportamento, um dos objetivos deste artigo é examinar quais os motivos subjacentes à adesão e permanência das gerações mais velhas às redes sociais e, em particular, ao Facebook, que constitui atualmente, segundo Pew Research Centre (PRC), a rede social mais popular do mundo \cite{pew2019}.

Acresce que, enquanto consumidores, os seniores afiguram-se como um segmento de mercado cada vez mais atrativo para as marcas \cite{kantar2018}, pelo seu poder económico, pelo maior tempo disponível e pelo aumento de longevidade da população \cite{friemel2016}. Conforme se verifica esta paulatina adesão às TIC e ao reino digital por parte deste segmento, multiplicam-se também os canais de comunicação e as oportunidades para as marcas junto deste público-alvo. Contudo, verifica-se ainda uma clara ausência de estudos que examinem o impacto e a relevância da internet e, em particular, do social media, no comportamento de consumo dos seniores ibéricos \cite{gama2020, miranda2020}. Tendo presente esta lacuna identificada na literatura, este estudo tem como segundo objetivo examinar o papel das páginas das marcas presentes nas redes sociais digitais, mais concretamente do Facebook, nos processos de tomada de decisão de consumo das gerações mais velhas.

\section{Enquadramento teórico}
\subsection{Os seniores no reino digital: Motivações para o uso das TIC, da \emph{Internet} e dos \emph{social media}}
Um factor fulcral para compreender o processo de adopção e uso continuado da tecnologia radica no conjunto multiforme de motivações que origina, impele e direciona os comportamentos dos utilizadores seniores\footnote{De acordo com a Organização Mundial de Saúde (OMS), referimo-nos a pessoas mais velhas com 65 anos ou mais.}.

A adopção das TIC é explicada, de acordo com \textcite{lee2015}, com base em dez motivos, entre os quais se contam: 1) independência, 2) experiência, 3) apoio técnico, 4) confiança, 5) suporte social, 6) emoções, 7) acessibilidade, 8) usabilidade, 9) acessibilidade e 10) valor. Alguns destes motivos assumem uma relevância particular neste domínio. De acordo com esta pesquisa, os seniores mostram-se mais interessados em adotar novas tecnologias, tais como computadores ou smartphones, se a usabilidade e a utilidade da tecnologia for elevada \cite{lee2015}.

Além do interesse pelos conteúdos, o estudo transcultural de \textcite{ferreira2019} identifica a comunicação como o principal motivo de uso das TIC pelas gerações mais velhas. Diversos estudos têm, igualmente, elencado motivos sociais como determinantes primários para o uso das TIC e da internet pelos seniores, nomeadamente para adaptação social ou como forma de evitar o isolamento e manter o contacto com família e amigos \cite{juznic2006}. As motivações sociais parecem aliás ser mais pregnantes que outro tipo de motivos para explicar a adesão às TIC. Estudos desenvolvidos em Portugal sugerem que a necessidade de comunicação e interação com a família, com colegas e amigos e com pessoas com interesses comuns, bem como desenvolver novos laços relacionais e buscar companhia, são alguns dos principais motivos usados para explicar o uso da internet \cite{dias2012} e dos social media \cite{gama2020, miranda2020} por parte dos seniores portugueses. Estes surgem enquadrados num elenco mais vasto de motivos, entre os quais figuram necessidades de pesquisa e de obtenção de informação, além de motivos ligados ao entretenimento e lazer.

Relativamente às motivações dos seniores para a presença e uso das redes sociais e das comunidades digitais voltam a figurar em primeiro plano motivos sociais e comunicacionais. É de ressalvar que a literatura da especialidade tem-se focado nos motivos de uso de uma rede social em particular, o Facebook, entre outras razões por que atualmente é nesta rede que se regista globalmente uma maior presença dos seniores \cite{statista2020}.

té à data, os estudos realizados sugerem que as redes sociais favorecem a sociabilidade e a obtenção de suporte social e diminuem o isolamento social, permitindo ainda o desenvolvimento das competências e de oportunidades dos seniores para comunicar e para a criação e manutenção de (novos) relacionamentos \cite{hutto2015, coto2017}. Como \textcite[p. 179]{cardoso2005} sublinham, as redes sociais “não só têm um efeito multiplicador dos contactos estabelecidos com a família e os amigos, independentemente do local do mundo onde estejam, como também é entre os utilizadores que se verificam menores ocorrências da sensação de estar isolado do mundo ou deprimido”. Nomeadamente, o Facebook é visto pelos mais velhos como um elo privilegiado de ligação com familiares e amigos, fortalecendo (ou até mesmo recuperando) o vínculo afetivo e social que os une \cite{erickson2011, jung2001, miranda2020}. São estas novas práticas de comunicação digital intergeracional, que se firmam no seio da família, que levam \textcite{taipale2019} a desenvolver a noção de “família digital”. Segundo este autor, as famílias servem como estruturas sociais em que as TIC e os social media servem não somente objetivos individuais mas também coletivos, na medida em que fortalecem a solidariedade familiar e permitem a manutenção dos laços familiares e afetivos.

Diversos outros motivos impulsionam a adesão e o uso do Facebook pelos seniores, como a curiosidade ou os pedidos de familiares, que criam as contas de Facebook dos seniores e incentivam a sua adesão \cite{jung2001}, bem como a partilha de conhecimentos e experiências acumuladas \cite{bell2013} e o contacto e integração na sociedade \cite{lin2013}. Tal é fortalecido pela capacidade do Facebook em favorecer o estabelecimento de “pontes” sociais intergeracionais, permitindo que indivíduos de diferentes gerações possam partilhar ideias, ligar-se e socializar \cite{jung2001}. Adicionalmente, o Facebook permite aos seniores recuperar memórias e reconectarem-se com o passado, possibilitando reencontros e o reatar de relações pessoais e contactos perdidos \cite{rebelo2015, miranda2020}.

Motivos utilitários, nomeadamente ligados à poupança de tempo e facilidade de uso, são também elencados por \textcite{chopik2016}, a par de motivos hedónicos, como o entretenimento e o divertimento \cite{leist2013, ramirez-correa2019}.

Num estudo exploratório realizado em Portugal, \textcite{rebelo2015} apurou que fatores como a solidão, a necessidade de ocupar tempo e a necessidade de integração na comunidade são os motivos mais apontados pelos seniores para manter uma presença on-line. Num outro estudo exploratório, realizado com participantes latinoamericanos com mais de 60 anos, \textcite{morales2016} constataram a existência de três motivos específicos que os impelem a usar o Facebook: sociais, familiares e motivos relacionados com a sua realização e/ou produtividade. Nesse sentido, o Facebook constitui, por um lado, um espaço importante para a manutenção do contacto com amigos, familiares e com o mundo em geral. Por outro lado, constitui-se como uma ferramenta chave para a produtividade, facilitando a concretização de objetivos, projetos e sonhos dos seniores.

\subsection{Os seniores, as redes sociais e as marcas}
Os mais recentes indicadores registam o aumento progressivo de cidadãos seniores que um pouco por todo o mundo aderem e usam redes sociais digitais, embora se mantenha a sua menor expressividade face a faixas etárias mais jovens, quer relativamente à percentagem de utilizadores de redes sociais, quer quanto ao número de plataformas usadas e ao tipo de utilização \cite{eurostat2019, pew2019, statista2020}. À medida que os segmentos etários mais velhos, como uma maré crescente, se vão instalando na “rede” e acedendo às redes sociais, um conjunto de estudos e indicadores de mercado trazem a lume o seu enorme potencial de consumo. Para as marcas, a importância de conhecer cada corte geracional assenta, em certa medida, no seu poder económico e o poder económico dos seniores tem vindo a mudar.

Em Portugal e segundo dados de Nielsen Homescan \cite{pinto2017} os shoppers seniores representaram em 2016 mais de um quarto das vendas de bens de grande consumo. É de notar que também as suas aspirações, valores, motivações, necessidades e comportamentos têm vindo a conhecer alterações ao longo do tempo \cite{noble2003, nunan2019}, tal como a sua relação com as TIC e os social media. A este propósito, o estudo Meaningful Brands, realizado pela \textcite{havas2018}, registou não só um crescimento significativo da presença on-line dos consumidores com mais de 55 anos como também um aumento do comércio eletrónico por parte destes, sugerindo que a tecnologia digital tem um papel não negligenciável nos seus processos de consumo. Neste âmbito, a pesquisa desenvolvida por \textcite{miranda2020} sugere que as redes sociais digitais constituem fontes de informação para alguns seniores, sendo que a pesquisa de informação relacionada com produtos e marcas surge como um dos motivos para os seniores portugueses usarem o Facebook. Aliás, \textcite[p. 469]{nunan2019} alertam que atendendo ao seu “poder económico e às crescentes taxas de adesão digital entre os consumidores mais velhos, compreender como eles adotam a tecnologia e usam os canais digitais é cada vez mais importante para os marketers”.

Os seniores surgem assim como um segmento de mercado atractivo devido ao seu poder económico, ao tempo disponível e ao aumento de longevidade \cite{friemel2016}. Um estudo realizado pela \textcite{kantar2018} sublinha algumas destas mudanças ligadas ao consumo, destacando tanto o potencial como o lado sexy destes \emph{shoppers}. No estudo \emph{Esqueça os Millennials! Quem vai ditar as tendências são os seniores}, a \textcite{kantar2018} carateriza os seniores como consumidores com um apurado poder de compra, além de serem os que maior peso têm nas ocasiões de compra. Acresce que são descritos como valorizando a qualidade, são mais fiéis às marcas e recorrem cada vez mais a tecnologias digitais. Para além disso, valorizam a perspetiva simbólica, hedónica e identitária do consumo e são sensíveis ao modo como as marcas se comunicam com eles. A maior lealdade às marcas, frequentemente apontada como uma característica dos consumidores seniores, decorre do maior vínculo emocional que os une às marcas \cite{lambert-pandraud2010}. Outro factor que os carateriza enquanto consumidores é a propensão para uma menor pesquisa de informação antes da tomada de decisão de compra, que é menos intensa e menos exaustiva por comparação com os jovens adultos \cite{mata2010, lockenhoff2018}.

As oportunidades que os consumidores seniores representam para as marcas, pelas tendências demográficas e económicas que se desenham a médio ou longo prazo são certamente muitas, sobretudo se foram conhecidas suas motivações e seus comportamentos e se as marcas ajustarem suas estratégias de comunicação a esta realidade. É de ressalvar que a comunicação com este \emph{target} deve dar-lhe visibilidade, atender à sua heterogeneidade e às capacidades que este detém e não tanto alicerçar-se em aspetos negativos, realçando fragilidades ou apresentando uma imagem estereotipada que não corresponde à realidade \cite{niemelanyrhinen2007}.

Esta comunicação marca-consumidor sénior torna-se tanto mais eficaz quanto melhor se conhecer este coorte geracional e os canais a que este recorre no âmbito dos seus processos de tomada de decisão de consumo. A adoção progressiva dos \emph{social media} pelo consumidor sénior fornece um conjunto de plataformas digitais que podem ser usadas pelas marcas para fomentar o relacionamento com os seus clientes destas faixas etárias, constituindo estas igualmente um meio através do qual indivíduos das gerações mais velhas podem pesquisar sobre produtos e serviços ou compartilhar as suas experiências de compra, entre outros aspetos. Nesse sentido, é relevante examinar o papel e a relevância das TIC e das redes sociais nestes processos de decisão de consumo, até porque uma análise atenta da literatura da especialidade é reveladora das suas fragilidades e limitações relativamente a esta problemática, mantendo-se genericamente omissa acerca do modo como os seniores e as marcas comunicam e se relacionam através das plataformas digitais.

\section{Método}
Na literatura, a relação entre os seniores, as TIC e os \emph{social media} tem sido examinada de forma parcial, pouco integrada e com recurso a uma diversidade de métodos e técnicas. No âmbito da presente pesquisa, tendo em conta as especificidades da população em estudo e dos objetivos visados, optou-se por uma investigação exploratória, de carácter misto, qualitativo e quantitativo, com recurso ao FG e à plataforma \emph{Facebook Audience Insights} para a recolha de dados, respetivamente.

Pretende-se que, através da triangulação articulada de métodos e técnicas, seja reunido um acervo de informação que permita a caracterização de uma realidade complexa e ainda não estudada no contexto ibérico.

\subsection{Participantes}
No que diz respeito ao método qualitativo, foram realizados cinco FG, 3 em Portugal e 2 em Espanha (o que é manifesto no esquema de codificação adotado, junto de seniores que frequentavam universidades seniores, em conformidade com a recomendação de 4 a 6 grupos por estudo \cite{morgan1996}. No total participaram 45 pessoas, 29 mulheres (64,4\%) e 16 homens (35,6\%), com idades compreendidas entre os 50 e os 88 anos, sendo a média de idades 69,5 anos. Em termos de habilitações académicas predomina o ensino secundário. A codificação utilizada teve em conta o país onde foram efetuados os FG (FGP realizados em Portugal e FGE realizados em Espanha), seguida da ordem cronológica de realização dos mesmos. O esquema de codificação aduz o número atribuído a cada participante (P1,...) bem como o género (H/M).

\subsection{Instrumentos de recolha de dados}
Para a recolha de dados com o FG foi construído um guião semiestruturado de entrevista em grupo para averiguar o modo como os seniores se relacionam com as TIC e com os media sociais no geral e, em particular, como se envolvem e qual a importância que atribuem às páginas das marcas no Facebook. Uma versão preliminar do guião foi aplicada a uma amostra de 6 pessoas, sendo este pré-teste usado para aferir a validade facial e a validade de conteúdo deste instrumento de recolha de dados.

O guião inicia-se com questões relacionadas com a frequência dos entrevistados nas universidades seniores, seguindo-se depois dois blocos de perguntas relacionadas com a utilização das TIC e dos media sociais (quais, formas de acesso, atividades, razões, motivações e opiniões), tal como sobre a relação dos seniores com as \emph{fanpages} das marcas no Facebook (visitas de páginas, interação e ligação com as páginas, motivações, preferência de posts e influencia no processo de intenção de compra).

Para a aferição dos dados quantitativos, recolheu-se informação da plataforma do \emph{Facebook Audience Insights} que permite o acesso a dados referentes à demografia, geografia, estilos de vida e interesses, \emph{engagement} com as páginas e intenções de comportamento de compra.

\subsection{Procedimentos}
As sessões de FG tiveram lugar em universidades da terceira idade da região da Grande Lisboa (Portugal) e Badajoz (Espanha), entre os meses de Novembro de 2019 e Janeiro de 2020. No início de cada \emph{Focus Group}, foi solicitada aos participantes autorização para a gravação áudio da sessão, para permitir a posteriori uma transcrição fiel dos conteúdos, sendo salientada a total confidencialidade e anonimato das respostas. Somente após o consentimento informado por parte dos participantes se deu início à sessão.

As entrevistas tiveram uma duração média de 1h30min. Para a análise dos protocolos do FG foi utilizada a análise de conteúdo. Iniciámos este processo com uma leitura flutuante sobre os dados que nos permitiu a apropriação “dos discursos recolhidos e pelos sentidos gerais neles contidos” \cite[p. 113]{morgan1996}. A partir daqui se efetuou uma análise categorial \cite{bardin2004} com recurso à identificação, bem como à contagem das categorias e subcategorias. A palavra foi eleita como unidade de registo.

Os dados quantitativos foram extraídos, no dia 23 de julho de 2020, do \emph{Facebook Audience Insights}. A extração focou-se nas subcategorias de \emph{fanpages} (notícia, política, comunidade, turismo etc), com recorte das cinco páginas mais seguidas pelos seniores em cada país.

\section{Análise dos dados}
\subsection{Resultados quantitativos}
Os dados do \emph{Audience Insights} mostram que de 500 a 600 mil contas do Facebook Portugal estão registadas por utilizadores seniores, enquanto em Espanha os perfis ativos dessa coorte ficam entre 1,5 e 2 milhões. Embora, no geral, os espanhóis tenham até três vezes mais registos, são os seniores portugueses que têm mais probabilidade de criar uma conta na rede social. Com base em todos os utilizadores com perfil ativo em cada país, os seniores portugueses representam cerca de 10,0\% do total, contra 7,8\% dos espanhóis.

Em termos demográficos, em ambos os países estudados, as mulheres representam a maior parte de utilizadores seniores, com uma ligeira vantagem para as espanholas (ES: 55,0\% / PT: 52,0\%). A categoria de género é a que mais se aproxima do utilizador padrão do Facebook na Península Ibérica. São as mulheres de todas as faixas etárias que lideram no número de contas ativas. Já a diferença entre homens e mulheres seniores, em Portugal, é de só 4,0\%, enquanto, em Espanha, pode chegar aos 10,0\%.

Parece existir uma relação entre níveis de escolaridade e utilização da rede social, uma tendência notada em toda a amostra. Em Portugal, 57,0\% dos seniores declaram, no Facebook, ter ensino superior completo, enquanto, em Espanha, o valor passa dos 64,0\%. Entretanto, o ensino secundário ainda representa uma fatia expressiva desses utilizadores de Facebook (PT: 40,8\% / ES: 35,2\%).

Em ambos os países, as grandes cidades e regiões metropolitanas são o cluster dos seniores com perfil ativo na rede social. O Quadro \ref{tab1} revela as cinco localidades de maior concentração de utilizadores, com 65 anos ou mais, em Portugal e Espanha. Os destaques ficam nos distritos de Lisboa (26,2\%) e do Porto (10,0) e nas comunidades autónomas da Andaluzia (12,8\%) e da Catalunha (10,2\%). São justamente as capitais dos distritos e das comunidades, com estruturas informacionais mais robustas, que agregam o maior número de utilizadores, como Lisboa (15,0\%) ou Barcelona (4,0\%). Já regiões caracterizadas por processos históricos de exclusão e zonas essencialmente rurais apresentam dados abaixo da média, como o distrito de Évora (0,6\%) e a comunidade da Estremadura (1,1\%).

\begin{table}[htpb]
\caption{Regiões com a maior concentração de utilizadores.}
\label{tab1}
\centering
\begin{tabular}{ll|ll}
\toprule 
\multicolumn{2}{c}{Portugal} & \multicolumn{2}{c}{Espanha}
\\
Distritos & \% & Comunidades & \%
\\
\midrule
Lisboa & 26,2 & 
Andaluzia & 12,8
\\
Porto & 10,0 &
Catalunha & 10,2
\\
Setúbal & 5,6 &
Madrid & 8,0
\\
Faro & 4,0 & 
Valencia & 6,5
\\
Braga & 3,0 & 
Galiza & 3,6
\\ 
\bottomrule
\end{tabular}
\source{Elaboração própria a partir de dados do Facebook Audience Insights (2020).}
%\notes{Esta é uma nota exemplo que poderá, opcionalmente, ser adicionada a uma tabela ou figura.}
\end{table}

Quando o assunto são os níveis de interação, verificamos que em média, os portugueses tendem a gostar (PT: 9 / ES: 5), comentar (PT: 6 / ES: 3) e partilhar (PT: 2 / ES: 1) conteúdos na rede com mais frequência que os espanhóis. Também são os lusitanos que mais dedicam tempo aos anúncios publicitários. Para o mesmo período, os portugueses consumiram em média 14 publicidades veiculadas no Facebook, enquanto os espanhóis ficaram pelas 12. É importante notar que, quando comparados com a média de todos os utilizadores da rede social (maiores de 18 anos), os seniores tendem a consumir cerca de 25,0\% menos anúncios nos dois países.

A diversidade de interesse do público sénior na maior rede social do mundo foi   analisada pelas \emph{fanpages} que têm atraído seguidores dessa coorte. O Quadro \ref{tab2} apresenta as cinco páginas mais relevantes para o grupo, com base no número total dos seguidores e na percentagem de seniores ativos. Nesse caso, são consideradas as próprias subcategorias fornecidos pelo Facebook para caracterizar os tipos de página na rede social.

\begin{table}[htpb]
\caption{\emph{Fanpages} mais seguidas pelos seniores ibéricos.}
\label{tab2}
\centering
\begin{tabular}{llllll}
\toprule 
\multirow{6}{*}{Portugal} & \emph{Fanpages}
& Lugar & Público & Seniores & Categoria
\\
\cmidrule{2-6}
& Cantando o Alentejo & 1º & 99 500 & 24,2\% & Entretenimento
\\
& Braga Cidade Vídeo-jornal & 2º & 53 000 & 23,8\% & Notícia
\\
& Jornalq & 3º & 73 700 & 23,2\% & Notícia
\\
& Nós, Cidadãos & 4º & 58 400 & 22,8\% & Política
\\
& Apoio aos Bombeiros de Portugal & 5º & 121 100 & 22,7\% & Comunidade
\\
\midrule
\multirow{6}{*}{Espanha} & \emph{Fanpages}
& Lugar & Público & Seniores & Categoria 
\\
\cmidrule{2-6}
& Queridos Recuerdos & 1º & 96 300 & 37,3\% & Revista
\\
& Super Abuelos.es & 2º & 192 000 & 32,7\% & Comunidade
\\
& Un millón de "me gusta" en... & 3º & 261 900 & 26,3\% & Comunidade
\\
& Catalunya m'agrada & 4º & 166 200 & 21,6\% & Turismo
\\
& Calzados Pitillos & 5º & 291 900 & 19,6\% & Loja
\\ 
\bottomrule
\end{tabular}
\source{Elaboração própria a partir de dados do Facebook Audience Insights (2020).}
%\notes{Esta é uma nota exemplo que poderá, opcionalmente, ser adicionada a uma tabela ou figura.}
\end{table}

Os dados do \emph{Audience Insights} sugerem uma dieta mediática variada em Portugal e Espanha, que conduz pelo menos três argumentos centrais. Primeiro, os veículos de imprensa ocupam um lugar mais importante para os seniores portugueses que espanhóis, embora os chamados grandes media não estejam no TOP 5. Páginas como Braga Cidade Vídeo-Jornal e Jornalq também ofertam informações com algum tratamento jornalístico. Segundo, \emph{fanpages} de comunidades, geralmente associadas com temas ou com causas específicas, tendem a atrair atenção de uma parcela considerável da coorte. Esse é o caso de Super Abuelos.es, que se denomina como “maior comunidade espanhola dedicada aos avós”. Terceiro, grandes marcas e franquias que podem ter os seniores como público-alvo não estão representadas na rede social. Com exceção da Calzados Pitillos, nenhuma outra empresa é relevante ao ponto de integrar o TOP 5.

\subsection{Resultados qualitativos}
Da análise realizada aos dados recolhidos através do FG emergiram um conjunto de categorias. Aqui apresentamos as três categorias principais: os motivos evidenciados pelos entrevistados em relação à adesão e à permanência nas redes sociais; a relação com as páginas das marcas no Facebook; o papel das plataformas no processo de tomada de decisão de consumo.

\subsubsection{Motivos de adesão e permanência nas redes sociais}
No quadro \ref{tab3}, apresentamos as subcategorias e as frequências das unidades de registo.

\begin{table}[htpb]
\caption{Subcategorias e frequência das unidades de registo/Portugal, Espanha: motivos de adesão e permanência nas redes.}
\label{tab3}
\centering
\begin{tabular}{p{0.14\textwidth}p{0.38\textwidth}p{0.1\textwidth}p{0.1\textwidth}p{0.14\textwidth}}
\toprule 
Categoria & Subcategorias & Frequência Portugal & Frequência Espanha & Total por subcategoria
\\
\midrule
\multirow{8}{=}{Motivos de adesão e permanência nas redes} & Comunicar & 19 & 8 & 27
\\
& Para saber as novidades & 12 & 1 & 13
\\
& Partilhar/publicar & 9 & 3 & 12
\\
& Utilidade & 5 & 5 & 10
\\
& Fazer amizades/procurar companhia & 8 & 0 & 8
\\
& Para seguir pessoas & 5 & 0 & 5
\\
& Por obrigação & 2 & 0 & 2
\\
& Motivos de não presença/uso de rede & 1 & 1 & 2
\\ 
\bottomrule
\end{tabular}
\source{Elaboração própria.}
%\notes{Esta é uma nota exemplo que poderá, opcionalmente, ser adicionada a uma tabela ou figura.}
\end{table}

Como podemos ver, os motivos mais referidos para aderir e permanecer nas redes está relacionado com aspetos relacionais e comunicacionais, tendo sido mais valorizados pelos inquiridos portugueses. A família, os colegas da universidade, os amigos e as pessoas com interesses semelhantes são aqueles com que os inquiridos mais comunicam:

\begin{quote}
A minha filha não pode estar muito tempo ao telefone quando está a trabalhar. E quando telefono é... "espera aí um bocadinho". E eu estou a ver o que ela faz. E agrada-me, pronto (FGP1P6M).


Tenho um grupo que é sobre Trás-os-Montes. É um grupo lá... falando de tarefas, eventos, que se fazem. Nós estamos distantes e que não vemos (FGP1P3M).
\end{quote}

Relativamente aos motivos que estiveram na génese de adesão às redes sociais, o fator com maior expressividade diz respeito ao incentivo que é dado por outros, sendo a família e, especialmente, os filhos, que assumem um papel crucial, como mencionado em: “Mas interessa-me por tudo. Depois tenho um filho que também me incentiva a essas coisas, que não me deixa adormecer. Pronto, é isso. É um misto destas duas coisas. Por essa razão, tenho Facebook, estou no Facebook” (FGP1P9M).

Outro motivo que foi referido por alguns inquiridos diz respeito à sua necessidade de aplicar os conhecimentos adquiridos nas aulas de TIC. Nessas aulas são abordados conteúdos sobre o uso das redes sociais, sendo “obrigatório” a adesão a estas redes para a aprendizagem. De igual modo, mas com menor expressividade, foram referidos motivos relacionados com a atividade profissional (quando se encontravam ainda em situação de vida ativa).

Outras razões que foram aludidas apenas pelos seniores portugueses estão relacionadas com a preocupação de “estar onde estão os outros”, mas também pela curiosidade que têm para perceber como funcionam as redes sociais, demonstrando que estes seniores não querem ficar alheados das dinâmicas existentes nas mesmas: “Curiosidade também. A pessoa tem curiosidade para saber. (...) Toda a gente fala na coisa (...) Tanto falar que nós queríamos saber como é” (FGP3P2M).

Também foram evidenciadas motivações que emergem dos interesses de cada um dos inquiridos, tendo sido mais referidas pelos seniores portugueses do que pelos espanhóis. A vontade de saber novidades é a motivação que tem uma maior expressividade nos discursos dos entrevistados: “Gosto mais de coscuvilhar e ver” (FGP3P6M); “para saber o que se passa” (FGE2P2M).

Ainda neste campo, partilha e publicação de fotografias, comentários, também assumem destaque para estarem nas redes sociais. Para além disso, existem motivações que estão relacionadas com o facto de valorizarem a utilidade que as redes sociais têm na sua vida, embora seja diferente entre os seniores portugueses e espanhóis. Neste caso, enquanto os portugueses usam as redes sociais para entretenimento e lazer (ocupar os tempos livres), os seniores espanhóis utilizam mais as redes sociais para ouvir música, ver notícias/vídeos e enviar mensagens.

\subsubsection{Relação com as páginas das marcas no Facebook}
O quadro \ref{tab4} dá-nos uma visão da relação que os entrevistados estabelecem com as páginas das marcas existentes no Facebook. Os dados evidenciam que, de uma maneira geral, não existe uma boa relação com as páginas das marcas no Facebook, havendo uma percepção negativa por parte da maioria dos inquiridos que visitam estas páginas. São apontadas várias razões para este sentimento.

A primeira razão diz respeito ao sentimento de pressão que alguns entrevistados sentem com o excesso de publicidade que é veiculada nas páginas do Facebook, causando um sentimento de “invasão”:

\begin{quote}
Nem é preciso fazer, porque a gente está a trabalhar e a publicidade está sempre a cair em cima da gente (FGP2P5M).


Pessoalmente, especificamente no Facebook, de uma maneira geral no Facebook, ou outra coisa qualquer pela internet, o meu comportamento é igual ao da televisão. Como eu não vejo televisão em direto, em princípio, não vejo televisão em direto para passar por cima da publicidade, não quero ver publicidade, portanto, no Facebook, faço a mesma coisa. Passo à frente, apago e acabou-se (FGP1P2H).
\end{quote}

Este sentimento de excesso origina a que as pessoas nem vejam a publicidade e tenham aversão à mesma, mesmo que possa ter marcas que lhes interesse.

\begin{table}[htpb]
\caption{Subcategorias e frequência das unidades de registo/Portugal, Espanha: relação com as páginas das marcas no Facebook.}
\label{tab4}
\centering
\begin{tabular}{p{0.2\textwidth}p{0.3\textwidth}p{0.1\textwidth}p{0.1\textwidth}p{0.14\textwidth}}
\toprule 
Categoria & Subcategorias & Frequência Portugal & Frequência Espanha & Total por subcategoria
\\
\midrule
\multirow{4}{=}{Relação com as páginas das marcas no Facebook} & Frequência de pesquisa & 14 & 7 & 21
\\
& Opinião sobre a informação disponibilizada & 4 & 4 & 8
\\
& Publicidade das marcas no facebook & 3 & 0 & 3
\\
& Perceção negativa sobre as páginas das marcas no facebook & 18 & 7 & 12
\\ 
\bottomrule
\end{tabular}
\source{Elaboração própria.}
%\notes{Esta é uma nota exemplo que poderá, opcionalmente, ser adicionada a uma tabela ou figura.}
\end{table}

A segunda razão está relacionada com o facto dos comentários e das informações postadas nas páginas das marcas no Facebook serem, muitas vezes, enganadoras. Este sentimento advém de algumas experiências em que se sentiram enganados:

\begin{quote}
Às vezes dou uma volta por algumas. Ainda há pouco tempo, numa página da Worten, vinha lá um smartphone, pá, aquilo era um charme. Um Asus não sei quê, com 6 GB de RAM e mais 128 de memória e mais não sei quê. Estava um preço realmente bom para aquilo. Só depois fui à pesquisa, fora, portanto da página, na página da marca... Primeiro à marca a ver e aquelas referências que lá colocavam, com aquelas potencialidades todas não existiam, pura e simplesmente (FGP1P4H).
\end{quote}

No que diz respeito à frequência de pesquisa nas páginas das marcas no Facebook, tanto os seniores portugueses como os espanhóis ainda pesquisam pouco. Dos que pesquisam, são os seniores espanhóis que procuram mais frequentemente. Em relação à informação disponibilizada nas páginas de Facebook das marcas, tanto os portugueses como os espanhóis consideram que a informação é escassa ou está pouco atualizada: “Eu não sigo porque é escassa a informação que aparece no Facebook. Muito escassa. Aparece em termos fotográficos, e pouco mais. Não tem dados técnicos. Por isso eu não sigo” (FGP1P4H).

\subsubsection{O papel das plataformas no processo de tomada de decisão de consumo}
A maioria dos entrevistados acede às plataformas para tomar decisões sobre produtos/serviços que pretendem adquirir. Neste processo de tomada de decisão foram identificadas quatro etapas: pesquisa de informação, avaliação de alternativas, decisão de compra e comportamentos pós compra (quadro \ref{tab5}). A etapa mais valorizada pelos inquiridos foi a primeira em que a informação pesquisada foi essencialmente sobre as características dos produtos, tal como evidenciado nos seguintes testemunhos:

\begin{quote}
Qualquer coisa que eu quero ver e a marca que é. Agora estou por exemplo à procura de um telemóvel barato. Estou a ver determinadas marcas, a ver muito bem as características, a tentar entender, porque o telemóvel para mim não é entendível. Aquilo é escusado. É tão complicado. Mas, o mínimo, pelo menos, que eu consiga. Olha essas características: "eu quero isso ou aquilo", e isso faço. Pronto, em determinadas situações, quando é necessário, qualquer coisa, isso faço (...) Vou ao site das marcas (FGP2P4M).
\end{quote}

\begin{table}[htpb]
\caption{Subcategorias e frequência das unidades de registo/Portugal, Espanha: as plataformas no processo de tomada de decisão de consumo.}
\label{tab5}
\centering
\begin{tabular}{p{0.2\textwidth}p{0.3\textwidth}p{0.1\textwidth}p{0.1\textwidth}p{0.14\textwidth}}
\toprule 
Categoria & Subcategorias & Frequência Portugal & Frequência Espanha & Total por subcategoria
\\
\midrule
\multirow{4}{=}{O papel das plataformas no processo de tomada de decisão de consumo} & Pesquisa de informação
& 20 & 11 & 31
\\
& Decisão de compra & 14 & 7 & 21
\\
& Comportamento pós compra & 14 & 0 & 14
\\
& Avaliação de alternativas & 5 & 0 & 5
\\ 
\bottomrule
\end{tabular}
\source{Elaboração própria.}
%\notes{Esta é uma nota exemplo que poderá, opcionalmente, ser adicionada a uma tabela ou figura.}
\end{table}

Depois da etapa de pesquisa, segue-se a etapa de avaliação das alternativas. Neste caso, foi pouco evidenciada pelos inquiridos, tendo sido apenas referidas pelos portugueses. Esta avaliação passou por verificar \emph{in loco} as informações (características técnicas, preço, qualidade) obtidas na internet sobre os produtos: "Eu se quiser comprar um smartphone em que não conheça e que estão a dar todas as características técnicas, inclusivamente a fotografia dele, eu se calhar vou a uma loja de marca que eu conheço, verificar" (FGP1P4H).

As plataformas assumem um papel importante na decisão de compra pela internet, sendo mais valorizadas pelos inquiridos portugueses. Nestas compras são adquiridos uma diversidade de produtos – tecnologia, alimentação, vestuário/calçado e cultura/lazer:

\begin{quote}
"Compro informática lá da Coreia. Já fiz algumas, mas pronto, normalmente pela Amazon" (FGP1P8H).


"Sim. Alguma coisa, sim (...) telemóveis já comprei. Já comprei gel de banho. Veio, sei lá, num site espanhol" (FGP1P9M).
\end{quote}

No que diz respeito à última etapa, apenas foram referenciados comportamentos pós compra pelos inquiridos portugueses. Estes comportamentos incidem sobre o processo de devolução dos produtos, a publicação de comentários sobre os produtos e o processo e o grau de satisfação em relação ao site, ao serviço e aos produtos. Postar comentários sobre os produtos/processo nas redes sociais é um dos comportamentos mais evidentes nesta etapa, sendo realizado principalmente no Facebook:

\begin{quote}
"Eu deixo o que aconteceu. Ou bem, ou mal, se correu tudo bem, se veio dentro dos prazos previstos ou não, se o produto correspondeu às expectativas ou a qualidade de preço. Normalmente eles pedem e eu ponho lá [refere-se aos comentários]" (FGP1P4H).
\end{quote}

A satisfação em relação à compra de produtos pela internet foi evidenciada por alguns entrevistados no que diz respeito ao serviço de entrega, ao produto em si e ao site da marca.

\section{Discussão}
Tendo como enquadramento a sociedade em rede na Península Ibérica, partindo de uma abordagem metodológica triangulada, a presente investigação pretendeu, por um lado, captar as diferentes dimensões da relação dos seniores com as TIC e com os \emph{social media} e, por outro lado, perceber a relação e o envolvimento que estabelecem com as páginas das marcas, nomeadamente durante o processo de tomada de decisão de compra onl-ine. Para o efeito, adotamos perspetivas analíticas plurais e abordagens com diferentes graus de extensividade e intensividade recolhendo informação qualitativa junto de grupos focais compostos por seniores, e informação de cariz quantitativo extraída a partir dos dados do \emph{Facebook Audience Insights}.

Sistematizamos as suas principais conclusões e contributos em três partes: uma primeira, que incide nos aspetos mais estruturais de caracterização e conhecimento da população em estudo; uma segunda, que aborda de uma forma mais fina a relação dos seniores com as TIC e os media sociais, como é o caso das motivações para o uso; e uma terceira, que foca a relação que estabelecem com as \emph{funpages} das marcas e o processo de decisão de compra on-line.

Uma ideia central e ponto de partida de análise é a de que as gerações mais velhas, embora estejam a aderir crescentemente às TIC e aos social media, apresentam ainda evidentes sinais de fratura digital \cite{gil2016} e elevados índices de infoexclusão, representando apenas 17,8\% dos utilizadores da península ibérica com perfil ativo na rede social Facebook. Esta asserção surge na esteira dos dados aferidos pela \textcite{eurostat2019} e INE (em Portugal e Espanha) evidenciando que a idade é inversamente proporcional ao uso das TIC entre os 65 e 74 anos, nos dois países. Retoma-se a narrativa de uma entrevistada: “apareceram aqui aulas de informática, entrei, porque eu nem sequer sabia escrever a máquina” (FGP3P3M).

Para além disso, fatores como a reduzida exposição às TIC durante o trajeto de vida, o facto de se encontrarem fora dos percursos formais de ensino, e as mudanças sensoriais e cognitivas relacionadas com o processo de envelhecimento podem, igualmente, constituir-se como obstáculos \cite{van_deursen2014}. Frise-se, porém, a emergência e o investimento, nas duas últimas décadas, de políticas públicas, ações e dinâmicas de inclusão digital entre as gerações mais velhas, na Península Ibérica e um pouco por toda a Europa, que agregadamente têm contribuído para se alcançarem “impactos macro e colocarem a ênfase da participação num mercado digital único” \cite[p. 311]{coelho2019} e outros efeitos mais direcionados para o envelhecimento ativo e bemestar \cite{paul2005}. O desafio é fazer com que o envelhecimento se desenvolva com qualidade em todos os domínios e que esta etapa da vida seja de mais valia para a sociedade.  

No que diz respeito à caracterização sociodemográfica da população em estudo, ressalta o facto de as mulheres representarem a maior fatia dos utilizadores com conta ativa do Facebook na Península Ibérica. Esta é, aliás, uma tendência que tem sido retratada na literatura da especialidade sendo apontada como um interessante elemento que rompe com a tradição clássica de dominação masculina no mundo tecnológico e digital. Em \textcite{roxo2016}, encontramos uma pletora de estudos, como é o caso do de Muller, que mostram que o território feminino impera já que as mulheres tendem a sentir-se mais atraídas pelos \emph{social media} do que os homens, que preferem jogos de computadores ou sites de apostas, revelando, as primeiras que, caso a internet findasse, aquilo de que mais falta sentiriam seria da possibilidade de interagirem nas redes. Procurando aventar uma explicação para tal fenómeno, na esteira de \textcite{braga2011}, \textcite{roxo2016} avança que cada período histórico carrega consigo uma configuração muito particular do uso que fazemos da tecnologia e da distribuição social do poder sobre ela. No caso particular das mulheres, o empoderamento e a emancipação do movimento feminista e o proliferar da internet possibilitaram um reposicionamento da mulher na sociedade e uma aproximação ao universo tecnológico e ambiente virtual suportados num novo horizonte de conexão, relacionamento afeto e emoção – aspetos cruciais e mandatórios do universo feminino e do seu posicionamento sociocultural.

Um outro aspeto que sobressai prende-se com a asserção de que, no caso em estudo, a esmagadora maioria dos seniores com perfil ativo no Facebook têm ensino superior ou ensino secundário completo situando-se maioritariamente nas grandes cidades e regiões metropolitanas - sede de estruturas informacionais tendencialmente mais complexas e robustas contrariando, de certa forma, a asserção estereotipada de que se trata de um cluster homogéneo, pouco instruído e escolarizado. Com efeito, na esteira das investigações de \textcite{van_boekel2017} e \textcite{coelho2019}, à semelhança de outras coortes, os seniores são um grupo diversificado e heterogéneo podendo espelhar modos diversos de relação e apropriação das TIC.

Ao escrutinarmos os motivos para adesão dos seniores aos social media, verificamos o papel central e basilar que o incentivo de terceiros, com ênfase para a rede familiar (nomeadamente os descendentes, filhos e netos) desempenham neste processo - principalmente entre os seniores com maior nível de iliteracia digital, no estímulo, vontade e utilidade para usar: “tenho um filho que me incentiva a essas coisas, não me deixa adormecer” (FGP1P9M); “Fui influenciada pelas minhas filhas” (FGP3P2M) e no auxílio à aprendizagem e no apoio técnico na resolução de problemas: “Elas é que me fizeram a página, depois eu continuei” (FGP3P2M). Estes dados encontram retaguarda na literatura da especialidade, nomeadamente nas investigações de \textcite{neves2012}, \textcite{jung2001} e \textcite{coelho2019}, todas destacando o papel da sociabilidade próxima e da coeducação intergeracional na relação com as TIC.

Realce-se que, sobretudo na ausência de familiares próximos, a curiosidade dos seniores para explorar o mundo digital e navegar nas redes sociais faz com que, por iniciativa pessoal se interessem e adiram: “mas a iniciativa foi minha, pressupondo que toda a gente está lá. Nós estamos lá também” (FGP1P4H); “Sentia que estava tudo ali ao meu lado. Então tive que entrar” (FGP3P2M), recorrendo, não raras vezes, a mecanismos e programas formais e informais de educação para adultos, como é o caso das universidades seniores, para obter as competências e o apoio técnico necessário para dirimir e ultrapassar as dificuldades que vão encontrando: “Aderi pelas aulas, pelo curso que tivemos” (FGE1P1M); “Eu estava aqui inscrita, apareceram as aulas de informática (...). E então aprendi aqui” (FGP3P3M).

Para além da escalpelização e compreensão dos motivos de adesão, foi nossa intenção perceber as razões de permanência nos media sociais. Neste caso, coligimos que os motivos sociais e relacionais revestem-se aqui de particular destaque com os seniores a assumirem que se trata de um espaço privilegiado para interagir, estabelecer e desenvolver laços (e, por vezes, o reencontro) com a família, os amigos ou fazer novas amizades, como também é entre estes utilizadores que se verificam menores ocorrências de sensação de estar isolado do mundo ou de estar só. De acordo com as narrativas dos entrevistados: “utilizo especialmente para falar com a família que está fora” (FGP1P3M); “já tenho muitos amigos pessoais que os conheci no Facebook” (FGP1P8H); “Nas redes sociais,(...), eu também procuro companhia” (FGP2P3M). Na literatura sobre o tema, também as investigações de \textcite{nimrod2012}, \textcite{nguyen2013} e \textcite{balcerzak2017} destacam os aspetos supra-citados contrariando, de certa forma, a ideia de que as TIC e os social media vieram desamarrar laços sociais e contribuir para o isolamento dos indivíduos. Retoma-se a asserção de \textcite{taipale2019} quando desenvolve a noção de “família digital” ao gizar que os social media servem propósitos coletivos e fortalecem laços familiares e afetivos. Já \textcite{cardoso2005, cardoso2015} defendem que as redes sociais podem potenciar o reforço e a multiplicação dos laços sociais, sendo que a combinação das formas de relacionamento presencial e virtual devem ser vistas como acumuláveis e não como substitutas umas das outras.

Às motivações sociais e relacionais acrescem as comunicacionais materializando-se no facto de os seniores considerarem os social media como um espaço de expressão de identidade pessoal, local onde se pode partilhar, debater, confrontar ideias e ideais, tal como envolverem-se ativamente nas suas comunidades: “Há um grupo, com um tema determinado, e eu entro no tema e opino. Não só a nível regional, mas sim a nível mundial” (FGE2P1H). Uma janela aberta para o mundo onde, ao mesmo tempo que possibilita o fácil acesso a notícias e novidades, permite também que se recuperem memórias da história pessoal de cada um, de sítios, tradições e vivências: “tenho um grupo que é sobre Trás-os-Montes. É um grupo que se fala de tarefas, eventos, que lá se fazem” (FGP1P3M). Conforme \textcite{ferreira2019, ferreira2016} gizaram a comunicação é um dos elementos basilares do uso das TIC e social media pelas gerações mais velhas.

O aumento progressivo dos seniores que aderem às TIC e às redes sociais, aliado à literatura que, por um lado, realça o seu eminente lado sexy no sentido em que apresentam um enorme potencial de consumo mediado por canais digitais \cite{friemel2016, nunan2019} e, por outro lado, a clara ausência de investigações que procuram perceber como os consumidores mais velhos se envolvem neste processo, nomeadamente como se comportam ao longo do processo de tomada de decisão de compra online e como se envolvem com as marcas nos media sociais, conduziu a última parte do presente estudo.

Um primeiro aspeto a salientar tem a ver com o facto de que, quando comparados com a média (maiores de 18 anos) do universo dos utilizadores da rede social Facebook, são os seniores que tendem a consumir menos anúncios na Península Ibérica (com ligeiro destaque para os portugueses que se apresentam mais ativos, seja na visualização de anúncios publicitários, seja na interação com as marcas, materializando-se em mais gostos, comentários e partilhas). Esta tendência surge na esteira da literatura da especialidade, donde destacamos o trabalho de \textcite{antunes2014}, assim como o de \textcite{reams2016}, tendo coligido que este grupo é, historicamente e em todos os meios, o que menos se identifica com a publicidade e o menos atreito aos seus estímulos, que consideram, grande parte das vezes, estereotipados negativamente.

Para além disso e com exceção da página da marca Calzado Pitillos, as grandes marcas ou insígnias comerciais que têm os seniores como público e consumidor alvo não estão representadas no Top 5 dos seus interesses ou preferências. Pelo contrário, aferimos que a generalidade da amostra tende a importar-se, a envolver-se e a interagir primordialmente com temas e páginas não comerciais, sobremaneira relacionados com o entretenimento. Este interesse é bem patente pela análise das \emph{funpages} que têm atraído mais seguidores deste coorte geracional, em que as páginas relacionadas com o entretenimento e com as comunidades, associadas a temas ou causas especificas, tendem a suscitar maior atenção. Esse é o caso de Super Abuelos.es (a maior comunidade espanhola dedicada aos avós), Cantando o Alentejo (página dedicada a prestigiar o Alentejo e a sua cultura musical), Apoio aos bombeiros de Portugal (página dedicada para apoiar e prestigiar os bombeiros de Portugal) ou um millón de “me gusta” en favor de regresso de Garzón (página dedicada para apoiar o regresso do juiz Baltazar Garzón à carreira de magistratura em Espanha). Salienta-se que, de acordo com os relatos dos entrevistados, a procura e a consulta das páginas das marcas comerciais serve, sobremaneira, para os apoiar e reduzir a incerteza no processo de tomada de decisão na compra de um produto ou serviço que pretendem adquirir, nomeadamente no processo de procura de informação, como é o caso das características, funcionalidades ou preços do produto/serviço. Tal como evidenciado por alguns testemunhos: “Estou a ver determinadas marcas, a ver muito bem as características, a tentar entender, porque o telemóvel para mim não é entendível” (FGP2P4M); “Eles dão as características e todo tipo de informação que possam dar” (FGE1P3H).  

No que diz respeito à efetivação da compra em si, embora, na generalidade, se tenha coligido resistência por parte dos seniores da península Ibérica (que preferem concretizá-la em loja física), são os seniores portugueses que arriscam, mais vezes, na compra on-line, sobretudo de produtos e serviços de baixa implicação. Para além disso, revelam-se mais ativos na publicação de feedback, comentários e grau de satisfação em relação ao desempenho dos produtos, tal como de todo o processo envolvido na entrega: “Eu deixo o que aconteceu. Ou bem, ou mal, se correu tudo bem, se veio dentro dos prazos previstos ou não, se o produto correspondeu às expectativas ou a qualidade de preço. Normalmente eles pedem e eu ponho lá. Sim” (FGP1P4H). Sublinhe-se que a relutância dos mais velhos para comprar online tem sido aferida e retratada pela literatura da especialidade, nomeadamente por autores como \textcite{nunan2019}, elencando uma série preditores que explicam as razões de tal comportamento.   

\section{Considerações finais}
Em jeito de síntese, quando examinado globalmente, este conjunto de resultados sugere que os seniores ibéricos que marcam presença nos \emph{social media}, nomeadamente no Facebook, o fazem por motivos comunicacionais e sociais, por descortinarem utilidade nesta rede social e para se manterem atualizados. A relação dos seniores com as páginas das marcas através desta rede social não se afigura (ainda) como marcante para a sua jornada do consumidor. O seu envolvimento nesta plataforma digital não é tanto com marcas comerciais, mas sim com páginas que apelam ao entretenimento ou que estão associadas a temas ou causas específicas que os mobilizam. Ainda assim, concluímos que a procura e consulta das páginas das marcas comerciais no Facebook intervém, em certa medida, na decisão de compra, ao reduzir a incerteza e ao apoiar a avaliação de alternativas neste processo, mas também para expressar a satisfação/insatisfação com o produto adquirido, à semelhança do que se verifica com segmentos etários mais jovens.

Este estudo procurou contribuir para aprofundar a compreensão da relação entre os seniores e as marcas nas redes sociais, uma temática que ainda se encontra na periferia do interesse dos investigadores. A riqueza de informação dada pela triangulação metodológica e de dados adotada traz valor acrescentado a esta investigação, uma vez que permite validar e também desenvolver resultados de pesquisas anteriores \cite{denzin1978}. Aduz-se o caráter transcultural da pesquisa, que permitiu examinar, contrastar e comparar as realidades portuguesa e espanhola, uma mais-valia numa área onde são escassas pesquisas transculturais. O target sénior tem suscitado um interesse crescente junto das marcas, pelas tendências demográficas e económicas que se desenham a médio/longo prazo. Compreender as suas forças motrizes, o que os motiva e os leva a aderir e permanecer nestas plataformas digitais é também um dado relevante que deve ser tido em atenção pelas marcas. Em termos práticos, este estudo sugere que o reino digital e, em particular, os social media já são parte integrante e relevante da jornada de alguns consumidores seniores, fator que deve ser capitalizado nas estratégias comerciais das marcas.

Importa referir que esta pesquisa padece de algumas limitações, que estudos futuros devem procurar colmatar. As amostras são de conveniência e de pequena dimensão, envolvendo seniores que estudam em universidades seniores em regiões específicas de Portugal e Espanha. Atendendo a que este se trata do primeiro estudo realizado a nível ibérico que se debruça sobre o papel das redes sociais digitais na relação entre o sénior e o processo de tomada de decisão de compra, seria pertinente a realização de investigações futuras com uma amostra representativa da população sénior ibérica, que permita a generalização de resultados. Numa outra perspetiva, os dados foram recolhidos em duas culturas que manifestam alguma proximidade cultural. Futuros estudos transculturais podem examinar se a teia de resultados e relações identificadas são replicadas em diferentes contextos culturais. Além disso, trata-se de um estudo sincrónico, observando-se as limitações inerentes a este tipo de estudos. Tendo em atenção a dinâmica evolutiva do comportamento do consumidor (inclusive do consumidor sénior), pesquisas posteriores de carácter longitudinal podem captar alterações que se venham a manifestar neste âmbito.

\printbibliography\label{sec-bib}
% if the text is not in Portuguese, it might be necessary to use the code below instead to print the correct ABNT abbreviations [s.n.], [s.l.]
%\begin{portuguese}
%\printbibliography[title={Bibliography}]
%\end{portuguese}


%full list: conceptualization,datacuration,formalanalysis,funding,investigation,methodology,projadm,resources,software,supervision,validation,visualization,writing,review
\begin{contributors}[sec-contributors]
\authorcontribution{Sandra Miranda}[conceptualization,funding,investigation,methodology,projadm,supervision,validation,writing,review]
\authorcontribution{Ana Cristina Antunes}[formalanalysis,investigation,methodology,writing]
\authorcontribution{Ana Gama}[datacuration,formalanalysis,investigation,methodology,resources,writing]
\authorcontribution{Branco Di Fatima}[datacuration,formalanalysis,investigation,methodology,resources,software,visualization,writing]
\authorcontribution{Ana Teresa Machado}[methodology,visualization,writing]
\end{contributors}


\end{document}