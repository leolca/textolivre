% !TEX TS-program = XeLaTeX
% use the following command:
% all document files must be coded in UTF-8
\documentclass[portuguese]{textolivre}
% build HTML with: make4ht -e build.lua -c textolivre.cfg -x -u article "fn-in,svg,pic-align"

\journalname{Texto Livre}
\thevolume{15}
%\thenumber{1} % old template
\theyear{2022}
\receiveddate{\DTMdisplaydate{2022}{8}{6}{-1}} % YYYY MM DD
\accepteddate{\DTMdisplaydate{2022}{9}{7}{-1}}
\publisheddate{\DTMdisplaydate{2022}{9}{20}{-1}}
\corrauthor{Cláudia de Barros}
\articledoi{10.35699/1983-3652.2022.41235}
%\articleid{NNNN} % if the article ID is not the last 5 numbers of its DOI, provide it using \articleid{} commmand 
% list of available sesscions in the journal: articles, dossier, reports, essays, reviews, interviews, editorial
\articlesessionname{editorial}
\runningauthor{De Barros and Hernández Fernández} 
%\editorname{Leonardo Araújo} % old template
\sectioneditorname{Daniervelin Pereira}
\layouteditorname{Leonado Araújo}

\title{Neuroscience, neuroeducation, neurodidactics and technology}
\othertitle{Neurociência, neuroeducação, neurodidáctica e tecnologia}
% if there is a third language title, add here:
%\othertitle{Artikelvorlage zur Einreichung beim Texto Livre Journal}

\author[1]{Claudia De Barros \orcid{0000-0002-2286-8674} \thanks{Email: \href{mailto:claudia.barros@uam.es}{claudia.barros@uam.es}}}
\author[2]{Antonio Hernández Fernández \orcid{0000-0002-7807-4363} \thanks{Email: \href{mailto:antonio.hernandez@ujaen.es}{antonio.hernandez@ujaen.es}}}
\affil[1]{Universidad Autónoma de Madrid, Facultad de Formación de Profesorado y de la Educación, Departamento Pedagogía, Madrid, España.}
\affil[2]{Universidad de Jaén, Facultad de Humanidades y Ciencias de la Educación, Departamento Pedagogía, Jaén, España.}

\addbibresource{article.bib}
% use biber instead of bibtex
% $ biber article

% used to create dummy text for the template file
\definecolor{dark-gray}{gray}{0.35} % color used to display dummy texts
\usepackage{lipsum}
\SetLipsumParListSurrounders{\colorlet{oldcolor}{.}\color{dark-gray}}{\color{oldcolor}}

% used here only to provide the XeLaTeX and BibTeX logos
\usepackage{hologo}

% if you use multirows in a table, include the multirow package
\usepackage{multirow}

% provides sidewaysfigure environment
\usepackage{rotating}

% CUSTOM EPIGRAPH - BEGIN 
%%% https://tex.stackexchange.com/questions/193178/specific-epigraph-style
\usepackage{epigraph}
\renewcommand\textflush{flushright}
\makeatletter
\newlength\epitextskip
\pretocmd{\@epitext}{\em}{}{}
\apptocmd{\@epitext}{\em}{}{}
\patchcmd{\epigraph}{\@epitext{#1}\\}{\@epitext{#1}\\[\epitextskip]}{}{}
\makeatother
\setlength\epigraphrule{0pt}
\setlength\epitextskip{0.5ex}
\setlength\epigraphwidth{.7\textwidth}
% CUSTOM EPIGRAPH - END

% LANGUAGE - BEGIN
% ARABIC
% for languages that use special fonts, you must provide the typeface that will be used
% \setotherlanguage{arabic}
% \newfontfamily\arabicfont[Script=Arabic]{Amiri}
% \newfontfamily\arabicfontsf[Script=Arabic]{Amiri}
% \newfontfamily\arabicfonttt[Script=Arabic]{Amiri}
%
% in the article, to add arabic text use: \textlang{arabic}{ ... }
%
% RUSSIAN
% for russian text we also need to define fonts with support for Cyrillic script
% \usepackage{fontspec}
% \setotherlanguage{russian}
% \newfontfamily\cyrillicfont{Times New Roman}
% \newfontfamily\cyrillicfontsf{Times New Roman}[Script=Cyrillic]
% \newfontfamily\cyrillicfonttt{Times New Roman}[Script=Cyrillic]
%
% in the text use \begin{russian} ... \end{russian}
% LANGUAGE - END

% EMOJIS - BEGIN
% to use emoticons in your manuscript
% https://stackoverflow.com/questions/190145/how-to-insert-emoticons-in-latex/57076064
% using font Symbola, which has full support
% the font may be downloaded at:
% https://dn-works.com/ufas/
% add to preamble:
% \newfontfamily\Symbola{Symbola}
% in the text use:
% {\Symbola }
% EMOJIS - END

% LABEL REFERENCE TO DESCRIPTIVE LIST - BEGIN
% reference itens in a descriptive list using their labels instead of numbers
% insert the code below in the preambule:
%\makeatletter
%\let\orgdescriptionlabel\descriptionlabel
%\renewcommand*{\descriptionlabel}[1]{%
%  \let\orglabel\label
%  \let\label\@gobble
%  \phantomsection
%  \edef\@currentlabel{#1\unskip}%
%  \let\label\orglabel
%  \orgdescriptionlabel{#1}%
%}
%\makeatother
%
% in your document, use as illustraded here:
%\begin{description}
%  \item[first\label{itm1}] this is only an example;
%  % ...  add more items
%\end{description}
% LABEL REFERENCE TO DESCRIPTIVE LIST - END


% add line numbers for submission
%\usepackage{lineno}
%\linenumbers
\begin{document}
\maketitle

\section{Editorial}

This monograph entitled "Neuroscience, neuroeducation, neurodidactics and technology" is inserted not only in the current scientific situation, but also responds to the need to establish the research bases that allow the advancement of education with a neuro and technological base. 

Charles Scott Sherrington is considered the pioneer of Neuroscience \cite{mora_teruel_como_2019}. Neurosciences constitute a set of knowledge that focuses on studying the structure and functioning of the nervous system, as well as the interaction of the elements of the brain that give rise to the behavior of human beings \cite{blakemore_como_2007,manes_usar_2014}, in order to understand how thought, consciousness, social interaction, creativity, perception, free will, emotion, among other facts, originate, which leads to the multidisciplinary nature of this new science. It must bring together neurologists, psychologists, psychiatrists, philosophers, linguists, biologists, engineers, physicists and mathematicians \cite{manes_usar_2014}, as well as physicians, sociologists, theologians and a long list of others, since understanding brain functioning is everyone's responsibility \cite{cumpa_valencia_usos_2019}.

The application of neuroscience in the educational field is called neuroeducation. This discipline that aims to develop new teaching and learning methods by combining pedagogy, neurobiology and cognitive sciences \cite{manes_usar_2014}. Neuroeducation is a branch of education, which links knowledge based on neuroimaging with the way the brain interacts with its environment. Specifically, it focuses on the learning/teaching process. A new look at the school process that links Neuroscience and Education \cite{bejar_neuroeducacion_2014}.

The practical application of neuroeducation is neurodidactics, proposed by Gerhard Preiss, in 1988 at the University of Freiburg. It is the application of knowledge about the functioning of the brain and the intervention of neurobiological processes in learning, in order to make it optimal and efficient \cite{fores_miravalles_descubrir_2009}.

The need to provide this scientific space is given by the emergence of Neuroscience in all sectors of knowledge, and lately with great impact in the educational context, where Neuroeducation and especially Neurodidactics have become emerging disciplines, necessary and inevitable in the work of teachers.

This monograph, wants to provide the current state of neuroeducation and technology, for which, we begin with the innovative studies of neuroimaging in neuropedagogy and neuromethodology, we continue relating neuroeducation and technology in vulnerable groups, as well as the relationship with the teaching-learning process. On the other hand, it includes a study on neuromarketing and technologies, neurodidactic factors in the prediction of university dropout, teacher professionalization, cognitive neuroscience, IOT and neurodiversity, the influence of ICT and social networks in social exclusion and the monograph ends with the concept map cmaptools as a neurodidactic tool. 

We hope that the reader will enjoy this monograph and that it will mark the beginning of future research in neuroscience, neuroeducation, neurodidactics and technology.

\section{Work arising from the projects}

"Formación del Profesorado Universitario en TIC Como Apoyo al Alumnado con Discapacidad". Type of Project/Grant: State Plan 2017-2023 Challenges - R+D+i Projects. Referencia: PID2019-108230RB-I00. 

Included in the Ibero-American Network for the Development of the Professional Identity of Teachers (RED RIDIPD) (University of Jaén, Spain).

Project (2021-1-HR01-KA220-HED-000027562). Autonomous University of Madrid (Spain). Amount of the approved project: 371,219.00 EUR. Duration: 1.2.2022.-31.1.2025. EMIPE Group (Team for interdisciplinary improvement of educational practice). Autonomous University of Madrid.


\printbibliography\label{sec-bib}

\end{document}

