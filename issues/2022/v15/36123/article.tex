% !TEX TS-program = XeLaTeX
% use the following command:
% all document files must be coded in UTF-8
\documentclass[portuguese]{textolivre}
% build HTML with: make4ht -e build.lua -c textolivre.cfg -x -u article "fn-in,svg,pic-align"

\journalname{Texto Livre}
\thevolume{15}
%\thenumber{1} % old template
\theyear{2022}
\receiveddate{\DTMdisplaydate{2021}{9}{18}{-1}} % YYYY MM DD
\accepteddate{\DTMdisplaydate{2021}{10}{14}{-1}}
\publisheddate{\DTMdisplaydate{2021}{11}{26}{-1}}
\corrauthor{José Teófilo de Carvalho}
\articledoi{10.35699/1983-3652.2022.36123}
%\articleid{36123} % if the article ID is not the last 5 numbers of its DOI, provide it using \articleid{} commmand
\runningauthor{Carvalho} 
%\editorname{Leonardo Araújo} % old template
\sectioneditorname{Daniervelin Pereira}
\layouteditorname{Anna Izabella Pereira}

\title{Resenha: Um mundo sem livros e sem livrarias?}
\othertitle{Review: A world without books and libraries?}
% if there is a third language title, add here:
%\othertitle{Artikelvorlage zur Einreichung beim Texto Livre Journal}

\author[1]{José Teófilo de Carvalho \orcid{0000-0003-4201-6214} \thanks{Email: \url{jteofilo.carvalho@gmail.com}}}

\addbibresource{article.bib}
% use biber instead of bibtex
% $ biber article

% used to create dummy text for the template file
\definecolor{dark-gray}{gray}{0.35} % color used to display dummy texts
\usepackage{lipsum}
\SetLipsumParListSurrounders{\colorlet{oldcolor}{.}\color{dark-gray}}{\color{oldcolor}}

% used here only to provide the XeLaTeX and BibTeX logos
\usepackage{hologo}

% if you use multirows in a table, include the multirow package
\usepackage{multirow}

% provides sidewaysfigure environment
\usepackage{rotating}

% CUSTOM EPIGRAPH - BEGIN 
%%% https://tex.stackexchange.com/questions/193178/specific-epigraph-style
\usepackage{epigraph}
\renewcommand\textflush{flushright}
\makeatletter
\newlength\epitextskip
\pretocmd{\@epitext}{\em}{}{}
\apptocmd{\@epitext}{\em}{}{}
\patchcmd{\epigraph}{\@epitext{#1}\\}{\@epitext{#1}\\[\epitextskip]}{}{}
\makeatother
\setlength\epigraphrule{0pt}
\setlength\epitextskip{0.5ex}
\setlength\epigraphwidth{.7\textwidth}
% CUSTOM EPIGRAPH - END

% LANGUAGE - BEGIN
% ARABIC
% for languages that use special fonts, you must provide the typeface that will be used
% \setotherlanguage{arabic}
% \newfontfamily\arabicfont[Script=Arabic]{Amiri}
% \newfontfamily\arabicfontsf[Script=Arabic]{Amiri}
% \newfontfamily\arabicfonttt[Script=Arabic]{Amiri}
%
% in the article, to add arabic text use: \textlang{arabic}{ ... }
%
% RUSSIAN
% for russian text we also need to define fonts with support for Cyrillic script
% \usepackage{fontspec}
% \setotherlanguage{russian}
% \newfontfamily\cyrillicfont{Times New Roman}
% \newfontfamily\cyrillicfontsf{Times New Roman}[Script=Cyrillic]
% \newfontfamily\cyrillicfonttt{Times New Roman}[Script=Cyrillic]
%
% in the text use \begin{russian} ... \end{russian}
% LANGUAGE - END

% EMOJIS - BEGIN
% to use emoticons in your manuscript
% https://stackoverflow.com/questions/190145/how-to-insert-emoticons-in-latex/57076064
% using font Symbola, which has full support
% the font may be downloaded at:
% https://dn-works.com/ufas/
% add to preamble:
% \newfontfamily\Symbola{Symbola}
% in the text use:
% {\Symbola }
% EMOJIS - END

% LABEL REFERENCE TO DESCRIPTIVE LIST - BEGIN
% reference itens in a descriptive list using their labels instead of numbers
% insert the code below in the preambule:
%\makeatletter
%\let\orgdescriptionlabel\descriptionlabel
%\renewcommand*{\descriptionlabel}[1]{%
%  \let\orglabel\label
%  \let\label\@gobble
%  \phantomsection
%  \edef\@currentlabel{#1\unskip}%
%  \let\label\orglabel
%  \orgdescriptionlabel{#1}%
%}
%\makeatother
%
% in your document, use as illustraded here:
%\begin{description}
%  \item[first\label{itm1}] this is only an example;
%  % ...  add more items
%\end{description}
% LABEL REFERENCE TO DESCRIPTIVE LIST - END


% add line numbers for submission
%\usepackage{lineno}
%\linenumbers

\begin{document}
\maketitle

\begin{quote}
    \fullcite{chartier2020}
\end{quote}


\emph{Um mundo sem livros e sem livrarias?} é o título do livro mais recente de Roger Chartier, lançado no Brasil pela Editora Letraviva. Coordenada por Guiomar de Grammont, que escreve também o prefácio, a obra reúne artigos de conferências e de pesquisas desse conhecido autor francês, realizadas em universidades e em instituições ligadas ao mercado editorial, nas duas últimas décadas. O livro contém prefácio, seis capítulos — a maioria escrita em português — e epílogo com análises de pesquisas publicadas recentemente nos mercados editoriais francês, norte-americano e brasileiro.

O título traz um questionamento sobre o destino do livro impresso, nos aspectos relacionados às suas diversas manifestações e processos, hoje, em transição para meio digital. Nessa obra, o autor discute três transformações relacionadas ao livro na sociedade atual: mudança na forma de escrita e de leitura — do meio impresso para o digital; a redução gradual do número de leitores e de livros impressos; como consequência, a repercussão desses fatores na produção editorial e na forma de distribuição. Desse olhar atento de historiador, surgem novas teorias e hipóteses sobre o futuro da escrita, da leitura e das livrarias físicas.

Além disso, Chartier retoma também as transformações do texto, em seu sentido amplo, relacionando-as à evolução do suporte de leitura — do antecedente rolo ao codex, desse à prensa de papel e, atualmente, do papel para a tela. Para esse autor, a primeira e a terceira transformações são revolucionárias, porque mudaram e mudam a encenação da leitura — gestos, postura corporal e concentração no texto —, a segunda, por sua vez, é apenas uma evolução da técnica de produção, introduzida pela prensa de Gutenberg no século XV.

Nesse contexto, o prefácio já adianta alguns pontos de discussão do livro, relacionando-os às obras anteriores do autor, cujos temas giram em torno das inquietações que dão nome a esse compêndio.

O capítulo inicial — \emph{A morte do livro?} — é o tema de uma conferência, pronunciada em espanhol, no Fórum das Letras da UFOP em 2006. O autor faz um diagnóstico e aponta as diferenças entre o livro impresso e o digital. Porém, afirma que o futuro do digital se tornou, hoje, incerto e aponta algumas razões para isso, como, por exemplo, as práticas de leitura e de escrita produzidas pela digitalização — leitura superficial, rápida e descontínua.

Antes, porém, de responder à pergunta do título desse capítulo, a autor faz outra: o que é um livro? Como resposta, recorre a vários conceitos, entre os quais aponta o da \emph{Metafísica dos Costumes}, de Kant, que distingue duas dimensões de um livro: como objeto material, pertencente a quem comprou, e como discurso dirigido ao leitor, cujo proprietário é o autor. Chartier cita Borges que, em 1952, dizia: “Um livro é mais que uma estrutura verbal, ou que uma série de estruturas verbais; é o diálogo que trava com seu leitor e a entonação que impõe à voz dele e as imagens mutantes e duráveis que deixa em sua memória. [...]” \apud{borges2012}[p. 57]{chartier2020}. Insensível à materialidade da obra, o que importa para \textcite{borges2012} é a leitura do texto, cujo sentido é apropriado pelo leitor e não o objeto de papel.

Chartier pensa diferente. Para ele, em linguagem metafórica, é como se o livro tivesse corpo e alma, por isso justifica a pergunta sobre a morte do livro. Recorrendo a várias teorias — Filosofia, Direito, Literatura e Linguística —, infere que o livro não morrerá como discurso; mas, sim, como objeto de leitura. Tal percurso, no entanto, vai depender das novas tecnologias da tela, dos hábitos e das expectativas dos leitores que dialogam com as obras por meio do pensamento e dos sonhos.

O segundo capítulo, \emph{Edições científicas}, foi apresentado em um evento sobre o livro digital na Embaixada do Brasil na França em 2014. O autor apresenta a estagnação do mercado de livros digitais nos últimos seis anos e aponta vários motivos para esse fenômeno. Um deles é o conflito entre a lógica intelectual, que exige livre acesso ao conhecimento e compartilhamento do saber, e a lógica comercial, baseada nos conceitos de propriedade intelectual e nas regras de mercado. Ressalta ainda que o leitor passou a controlar as interpretações e a ter acesso às “provas” científicas e aos dados que são utilizados na argumentação do pesquisador.

Esse autor recorre às definições do capítulo anterior e discute a tensão entre a imaterialidade das obras e a materialidade dos textos, a mesma que caracteriza a relação entre os leitores e seus livros, dos quais não são críticos nem editores. Todavia,  reconhece que, ao se apropriarem de um texto, os leitores se tornam coautores e dão-lhe novo sentido. Talvez, seja essa relação uma das motivações da pesquisa acadêmica.

Em seguida, o autor expõe a multiplicação das editoras universitárias no século XX, as transformações culturais nos hábitos de leitura e de compra de livro dos franceses entre 1973 e 2008. A essas se somam as mudanças nas práticas de leitura — falta de hábito de ler na faixa de 15 a 19 anos, mais recursos digitais, repasse de livros, cultura da fotocópia —, levando à diminuição na compra de livros na faixa etária entre 19 e 25 anos. Nesse cenário, o leitor de texto científico ainda prefere o texto impresso, porém multiplicam-se diariamente os textos digitais.

O capítulo seguinte — \emph{Livrarias} — foi exposto em 28 de agosto de 2019 na conferência de abertura na 29ª Convenção Nacional de Livrarias, durante a Bienal do Livro do Rio de Janeiro. Como não podia deixar ser, o tema em discussão desse evento era o movimento de fechamento de livrarias em todo o mundo, devido ao crescimento exponencial de vendas de livros pela Internet.

Nesse evento, o autor apresenta estatísticas dos mercados editoriais francês, americano e brasileiro e realça o fenômeno das vendas on-line pelas editoras e distribuidoras, mercado monopolizado pela Amazon. Para o autor, as livrarias sobreviventes, com espaço de leitura ao fundo, são utopias. A força das utopias e dos sonhos, entretanto, pode inspirar decisões institucionais, ações coletivas e condutas individuais, as quais evitarão um mundo sem livrarias.

No quarto capítulo, denominado de \emph{Autoedição}, Chartier utiliza um texto de uma palestra, apresentado na Universidade Sta. Úrsula em 2019. Resgata a história das práticas autorais, da antiguidade aos dias atuais, ressaltando que as edições autorais eram mais difíceis no passado. Por isso, ele entregava as tarefas, da produção à distribuição, ao editor. No entanto, o mundo digital muda a relação entre autores e seus leitores, graças à Web, e facilita a autoedição; até porque a produção, na atualidade, se dá sob demanda. A autopublicação é, portanto, uma maneira de inventar novas formas discursivas e pode ser a ressurreição do autor, cuja morte se prenuncia desde o século XX.

\emph{Ler sem livros} é o título do quinto capítulo, no qual o autor traz à memória lembranças de suas leituras na adolescência. Num texto produzido em espanhol, a pedido da revista eletrônica Alabe, afirma ainda que escrever lembranças pessoais é sempre produzir (conscientemente, ou não) uma representação do passado imaginado, desejado, e não o que aconteceu realmente. Logo, escrever sobre a memória dos leitores se tornou um verdadeiro gênero literário na atualidade.

Chartier distingue, também, o leitor nascido em um mundo saturado de livros daquele nascido em um mundo sem livros e reconhece o papel que a escola desempenha na formação desse último, local onde o estudante aprende a ler. Nesse caso, a leitura é uma conquista. No entanto, o autor admite novas modalidades de leituras sem livro, prometidas e impostas pelas telas do mundo digital. Recapitula também algumas obras do século XVI e XVII, as quais já indicavam a possibilidade de outras formas de narrativas, como a icônica, a partir da teoria renascentista da equivalência entre discurso e imagem; escritor e pintor.

O pesquisador lembra ainda que a palavra “leitura” está associada à ideia segundo a qual a prática de leitura vai além da apropriação da escrita. Como justificativa para sua posição, utiliza, como exemplo: \emph{A importância do ato de ler}, de \textcite{freire1989}, para quem ler supunha, primeiro, alfabetização e os sentidos de “ler” poderiam significar ler letras, palavras ou livros, mas também “ler” o mundo, a natureza, a memória, os gestos, os sentimentos. Designava esse sentido aberto da leitura com o neologismo ‘palavramundo’.

Por fim, Chartier discute, como primeiro desafio ao leitor, o significado ampliado de texto e se sua compreensão deve ser pensada como uma leitura. Nesse debate, admite a possibilidade de ler sem livros; porém, citando \textcite{marin1993}, nos adverte contra a tentação de pensarmos que podemos ler imagens na tela, por exemplo, como se lêssemos textos (nesse caso, texto verbal). O segundo desafio é a leitura fragmentada, descontextualizada e superficial na tela: “A digitalização do mundo é uma magnífica promessa e, ao mesmo tempo, uma perda se ignora ou se apaga as heranças que permitiram e ainda permitem múltiplas experiências do ler e do escrever” \cite[p. 166-167]{chartier2020}. Uma dessas heranças é a existência de uma forte relação com a totalidade textual e com cada um de seus elementos constituintes.

O último capítulo, \emph{Experiências brasileiras}, apresentado na Jornada Intercultural França-Brasil, em julho de 2019, mostra uma pesquisa sobre três autores franceses que trabalharam no Brasil — Fernand Braudel — ou que visitaram o país várias vezes — Michel de Certeau e Michel Foucault —, acrescido de Pierre Bourdieu. Esse último nunca viajou ao Brasil, mas manteve fortes relações com a realidade e com sociólogos brasileiros, nos campos da educação, da sociologia dos esportes e da história dos intelectuais. Os autores escolhidos, constituem apenas um recorte de vários pesquisadores franceses que estiveram ou trabalharam no Brasil. Esse é o capítulo menos alinhado às discussões da obra, cujo tema central é o livro, acrescido da leitura e a escrita em outras formas de textos, em transição hoje do impresso para o digital.

Para finalizar, Chartier retoma o tema do livro com um epílogo: \emph{Um mundo sem livros e sem livrarias?} Texto escrito, especificamente, para a obra publicada, mostra que, no mundo digital, se lê cada vez mais sem livros. O autor atualiza ainda dados de pesquisas francesas anteriores, de 1973 a 2008, com três observações: 1) o número de leitores de livros vem caindo sistematicamente nas últimas gerações; 2) a porcentagem de leitores constantes e bons compradores de livros acompanha a tendência, no geral, mas revela uma inversão do público masculino para o feminino; 3) o percentual de indivíduos que se dizem em um sistema totalmente digital vem aumentando, principalmente entre os mais jovens. As pesquisas no Brasil apontam no mesmo sentido e a venda de livros vem caindo na última década. A questão é: será que esse universo permanecerá sempre juvenil? Vale lembrar que a comercialização de e-books representa menos de 10\% de vendas de livros na Europa, nível semelhante ao do Brasil, e um pouco mais alto no mercado norte-americano.

Pela atualidade das discussões e pertinência das reflexões, esta obra merece uma leitura atenta dos profissionais interessados na produção de textos, na leitura e no mercado editorial. Os capítulos, muitas vezes, atualizam conceitos já discutidos por Chartier em publicações anteriores e podem ser lidos em qualquer ordem, por terem princípio, meio e fim. É preciso pontuar que este livro não esgota o tema. Além disso, a obra traz, também, dicas de leitura para pesquisadores, autores, editores, professores, estudantes de pós-graduação e de graduação que têm o livro como objeto de pesquisa ou de trabalho, seja como suporte de leitura, discurso ou texto. Afinal, são esses profissionais que respondem pelas tecnologias do livro e pelo percurso dele do autor ao leitor, no meio impresso ou no digital.

\printbibliography\label{sec-bib}
% if the text is not in Portuguese, it might be necessary to use the code below instead to print the correct ABNT abbreviations [s.n.], [s.l.]
%\begin{portuguese}
%\printbibliography[title={Bibliography}]
%\end{portuguese}


\end{document}