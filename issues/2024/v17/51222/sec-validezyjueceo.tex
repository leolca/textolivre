\section{Validez y jueceo de expertos}\label{sec-Validezyjueceo}

\textbf{Validez y jueceo de expertos}

La integración de estas nuevas tecnologías nos obliga a revisar los
procesos ya planteados y utilizados por los investigadores, como lo es
la validez y el juicio de expertos. Según los Estándares para pruebas
psicológicas y educativas de la \textcite{AERA2014}, la validez es
un indicador clave de la calidad de cualquier instrumento de medición
debido a que forma parte del fundamento al momento de interpretar y
tomar decisiones, nos provee de confianza en los resultados, haciendo
que forme parte de la base de las mejoras educativas y políticas,
impactando en la responsabilidad y rendición de cuentas. Esto se
complementa con la noción de confiabilidad, que se refiere a la
consistencia de un instrumento de medición, es decir, a la coherencia de
puntajes entre replicaciones de un procedimiento de evaluación,
independientemente de cómo se estime o reporte esta coherencia \cite{AERA2014}.

Es importante señalar que, incluso en la actualidad en investigaciones
como la de \textcite{Galicia2017}, se sigue utilizando el nombre de
validez de contenido al juicio de expertos, no obstante, comprendiendo
las ideas de \textcite{Kane2006,Kane2013} y \textcite{Chapelle2021}, el
Enfoque Basado en Argumentos (EBA), propuesto en gran medida por \textcite{Kane2006,Kane2013}, implica que la validez no se limita al contenido del
test, sino que se expande para incluir la interpretación y el uso de los
resultados del test. Según este enfoque, así como la \textcite{AERA2014}, la validación se convierte en un proceso integral y
argumentativo, donde cada interpretación de los resultados del test se
justifica a través de un argumento de validez estructurado y basado en
evidencia, por lo que se debe hablar de evidencias de validez y no de
validez de contenido.

En este contexto, el juicio de expertos es definido como una opinión
informada proporcionada por personas reconocidas como expertas
cualificadas en un tema específico, capaces de ofrecer información,
evidencia, juicios, y valoraciones; este tipo de método forma parte de
las evidencias de validez \cite{AERA2014}; que antes solían
llamarse validez de contenido. Se lleva a cabo mediante la evaluación de
ítems de un instrumento por estos expertos, quienes juzgan su claridad,
coherencia, relevancia y suficiencia. La selección cuidadosa de los
jueces, basada en su conocimiento y experiencia, es fundamental para
obtener evaluaciones precisas y útiles, que normalmente va acompañado de
una rúbrica para apegarse a ciertos criterios y evitar sesgos \cite{Galicia2017}.

Debido a lo anterior, se entiende que las preguntas generadas por IAGen
deben someterse a los mismos procesos que un ser humano; este tema
podría traer de nuevo al debate la profundidad del concepto de validez
ahondado por \textcite{Messick1989}, \textcite{Kane2006,Kane2013} y \textcite{Chapelle2021}.
La integración de la IAGen en este proceso abre nuevas posibilidades
para analizar la interacción entre los ítems y las tecnologías
emergentes, mejorando así la metodología de validación. Este avance
permite una evaluación más profunda de cómo se interpretan y utilizan
los ítems en variados contextos, enriqueciendo la comprensión del juicio
de expertos.

Además, los Estándares también ofrecen guías claras para el juicio de
expertos \cite{AERA2014}, incluyendo la documentación de las
cualificaciones de los expertos, el manejo de la interacción entre
participantes y la importancia de mantener juicios objetivos y bien
razonados. Estas prácticas aseguran que los procesos de evaluación sean
justos, confiables y capaces de soportar un escrutinio riguroso.

La introducción de la inteligencia artificial generativa, como ChatGPT,
en el contexto del juicio de expertos y el EBA, propone una
transformación en la metodología de validación. La IA puede jugar un
papel crucial en la mejora de los procesos de evaluación, ofreciendo
nuevas perspectivas y capacidades analíticas. En el enfoque basado en
argumentos, como propone \textcite{Chapelle2021}, la IA generativa puede
contribuir significativamente a la construcción de argumentos de
validez, proporcionando análisis avanzados y simulaciones que apoyen o
desafíen las interpretaciones de los datos. La IA puede también
identificar patrones y correlaciones que podrían pasar desapercibidos en
las revisiones tradicionales.