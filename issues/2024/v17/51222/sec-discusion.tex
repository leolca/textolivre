\section{Discusión y Conclusiones}\label{sec-DiscusiónyConclusiones}

El uso e investigación del uso de la IAGen se encuentra en un momento
crucial de la historia educativa. Este estudio no solo se alinea con la
evolución tecnológica en la educación, sino que también aborda la
intersección de la IA y las metodologías de evaluación, un tema que ya
ha capturado la atención de académicos y educadores por igual \cite{Sadiku2021,Hosseini2023}. Esto a pesar de la discusión pública
sobre el papel de ChatGPT en la educación, marcada por voces críticas
como \textcite{Chomsky2023}, que contrasta con perspectivas
más optimistas como la de\textcite{Yell2023}, quien subraya el potencial de la
IAGen para enriquecer el aprendizaje.

Uno de estos ejemplos es la capacidad de ChatGPT para generar ítems de
examen válidos y relevantes, como se demostró en la investigación de
\textcite{Nasution2023}, refleja un avance hacia la automatización en la
creación de contenido educativo, respaldando los argumentos de eficacia
y eficiencia en la utilización de tecnologías avanzadas en la educación
\cite{Feuerriegel2024,Dimitriadou2023,Tlili2023}. Este estudio trata de abonar a esta visión optimista de \textcite{Yell2023} y \textcite{Nasution2023} sobre el uso e inclusión de la IAGen en procesos educativos.

Los resultados obtenidos sugieren que, bajo un juicio de expertos
cuidadosamente diseñado, los ítems generados por ChatGPT alcanzan un
nivel de aceptación comparable a los creados por humanos. Este hallazgo
es consistente con las observaciones de \textcite{Rauber2024}, quienes
también destacaron la utilidad de la tecnología de aprendizaje
automático en la evaluación educativa. Sin embargo, la variabilidad en
la aceptación de ítems entre los diseñadores humanos y ChatGPT resalta
la importancia de la supervisión humana y la necesidad de ajustes
específicos para alinear los ítems generados por IA con los estándares
educativos \cite{Nasution2023, Ruiz2023}.

Las contribuciones de este estudio se centran en demostrar el potencial
de la IAGen para asistir en la creación de contenido educativo validado,
al tiempo que se subraya la necesidad de un marco de juicio de expertos
robusto para evaluar la calidad de este contenido. A pesar de los
resultados prometedores sobre la no diferencia de comportamiento entre
jueces humanos y ChatGPT, existen limitaciones inherentes al estudio,
como la dependencia de la especificidad de los \emph{prompts} y la
variabilidad en la capacidad de juicio de los expertos, lo que sugiere
la necesidad de una investigación futura para optimizar los procesos de
generación y evaluación de ítems con IAGen, sobre todo debido a los
resultados de la confiabilidad inter-jueces, que a pesar de ser
positivos para el ChatGPT, hubo discordancia entre los jueces humanos
\cite{Galicia2017}.

Finalmente, este estudio no solo refuerza la viabilidad de utilizar
ChatGPT y otras tecnologías de IAGen en la educación, sino que también
destaca la importancia crítica del juicio humano experto en la
validación de contenido generado por IA. Al vincular estrechamente los
hallazgos con los objetivos planteados, este trabajo contribuye
significativamente a la discusión sobre el equilibrio entre la
innovación tecnológica y la necesidad de mantener altos estándares de
calidad y relevancia en la educación. La investigación futura debería
centrarse en perfeccionar la sinergia entre la inteligencia artificial y
el juicio humano para maximizar los beneficios de ambas en el desarrollo
educativo.

Algunos estudios posteriores:

\begin{itemize}
\item Se dará seguimiento a la evaluación de los ítems, comparando los desarrollados por humanos y por ChatGPT; se puede adelantar que los ítems se muestran consistentes, pero será publicado posteriormente (\Cref{appdx1}).
\item Se dará seguimiento y se seguirán desarrollando este tipo de ítems con IAGen para observar sus limitaciones y posibles aportes importantes; recordando que hay constantes actualizaciones de ChatGPT y similares.
\end{itemize}