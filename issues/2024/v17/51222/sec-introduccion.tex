\section{Introducción}\label{sec-Introducción}

La evolución de la inteligencia artificial (IA) ha marcado un hito
significativo en la actualidad sofocada por infodemia e infoxicación.
Esta transformación tecnológica, inicialmente concebida por John
McCarthy en 1956, ha avanzado significativamente, incorporando
aplicaciones de aprendizaje automático y procesamiento del lenguaje
natural (NLP) en herramientas educativas \cite{Sadiku2021}. No obstante, el gran salto se dio con la presentación de
ChatGPT en 2020, y con mejor respuesta a principios del 2023, siendo la
IA Generativa (IAGen) otra forma de entender a la IA; la IAGen crea
contenido original a partir de datos existentes mediante algoritmos y
redes neuronales avanzadas \cite{Feuerriegel2024}.

En el campo de la educación, los modelos de lenguaje grandes (LLM,
\emph{Large Language Model}), como ChatGPT \cite{OpenAI2023}, ha generado
diversos debates públicos y digitales como el de la opinión emitida por
el lingüista {\textcite{Chomsky2023}}, donde califica a ChatGPT como una forma de plagio de alta
tecnología, pudiendo socavar la educación al motivar a los estudiantes
en la búsqueda de atajos para la entrega de trabajos, como los ya
clásicos ensayos o resolución a preguntas cerradas, como en un
cuestionario de reforzamiento, por ejemplo. Ante este tipo de
reflexiones, han surgido otras como la de \textcite{Yell2023}, un
profesor retirado de la Universidad de Wisconsin, argumentan sobre que,
si se utiliza de forma adecuada, ChatGPT puede ser un recurso valioso
para fomentar el aprendizaje basado en la búsqueda e investigación,
permitiendo promover el pensamiento crítico. Aunque es indiscutible que
este tipo de tecnología es capaz de crear contenido nuevo en formato de
texto, imágenes o audio, permitiendo hasta asistir en tareas de
conocimiento y necesidades cotidianas \cite{Feuerriegel2024}.

La aplicación de ChatGPT en la educación se ve reflejada en el análisis
de \textcite{Dimitriadou2023}, quienes abordan la integración de la
IA en las aulas inteligentes y los desafíos éticos asociados, así como
en el estudio de \textcite{Tlili2023} abordando el uso de
\emph{chatbots}, que examinan la aplicación de ChatGPT en la elaboración
de ejercicios en forma de cuestionarios. Estos enfoques resaltan el
equilibrio necesario entre las capacidades de la IA y la intervención
humana para garantizar la relevancia, la exactitud y la equidad en la
educación.

Recientes investigaciones, como las de \textcite{Nasution2023} y \textcite{Ruiz2023},
han explorado el uso de ChatGPT 4.0 en la generación de ítems de examen,
destacando no solo su capacidad para crear preguntas de elección
múltiple relevantes y coherentes, sino también abordando desafíos como
irregularidades y redundancias en interacciones más prolongadas. Estos
estudios subrayan la importancia de la especificidad y sistematización
en los prompts para generar exámenes eficientes y precisos,
capitalizando las fortalezas de la IA para la educación \cite{Nasution2023, Ruiz2023}.

La investigación de \textcite{Nasution2023} se enfocó en la validez y
confiabilidad de las preguntas generadas por IA, un tema que ha
suscitado tanto interés como preocupación en la comunidad educativa. Con
una muestra de 272 estudiantes, Nasution emprendió la tarea de evaluar
una serie de preguntas creadas por ChatGPT, obteniendo resultados que
son tanto prometedores como reveladores. De las 21 preguntas generadas
por la IA, 20 resultaron ser válidas, lo que indica una alta tasa de
éxito. Este hallazgo es significativo, ya que subraya la capacidad de la
IA para producir contenido educativo que no solo es relevante, sino
también de calidad.

No obstante, también hay investigaciones en torno al uso de Machine
Learning (ML) como el de \textcite{Rauber2024} quienes desarrollaron un
modelo automatizado para medir el aprendizaje de conceptos y prácticas
de clasificación de imágenes mediante redes neuronales. Se basó en datos
de 240 estudiantes de secundaria y bachillerato, concluyendo que la
evaluación es confiable y válida. Además, destacaron la efectividad del
modelo resaltando la importancia de incluir ML en la educación escolar y
la capacidad del modelo para asistir en el proceso de evaluación,
facilitando la carga de trabajo de los docentes.

A medida que la tecnología de IA continúa evolucionando, con avances
significativos en las versiones más recientes de ChatGPT, se presenta
una oportunidad única para mejorar y sistematizar el proceso de creación
de exámenes. Las investigaciones de \textcite{Nasution2023,Ruiz2023} se
alinean con esta visión, proponiendo un enfoque metodológico que combina
la exploración y descripción detallada de las capacidades de ChatGPT 4.0
en la generación de ítems de examen, proporcionando así una perspectiva
integral de su aplicabilidad y eficacia en el ámbito educativo. No
obstante, todavía hacen falta estudios que comparen el comportamiento de
la IAGen y si los seres humanos somos capaces de detectar esas
diferencias, o bien, podrían ayudar a reducir la carga de los docentes e
instituciones al momento de crear exámenes de alto impacto; como los del
ingreso a la universidad.

Ante todos estos acontecimientos, y retomando estos modelos de lenguaje
que podrían ayudarnos a evaluar nuestra propia forma de comunicar, el
objetivo principal de esta investigación fue explorar y comparar la
eficacia de la IAGen, representada por ChatGPT 4.0, y los diseñadores
humanos en el desarrollo de ítems para el Examen de Ingreso a la
Educación Superior (ExIES), en el área de Lengua Escrita, a través del
método de juicio de expertos. Lo anterior, con el fin determinar la
calidad, relevancia y alineación de los ítems generados por ambas
fuentes (IA y humanos) con los estándares establecidos para la
evaluación educativa, centrándose en aspectos como claridad,
neutralidad, concisión, alineación curricular y adecuación de formato y
contenido.