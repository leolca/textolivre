\documentclass[portuguese]{textolivre}

% metadata
\journalname{Texto Livre}
\thevolume{17}
%\thenumber{1} % old template
\theyear{2024}
\receiveddate{\DTMdisplaydate{2024}{5}{9}{-1}}
\accepteddate{\DTMdisplaydate{2024}{8}{1}{-1}}
\publisheddate{\today}
\corrauthor{Ana Claudia Loureiro}
\articledoi{10.1590/1983-3652.2024.52564}
%\articleid{NNNN} % if the article ID is not the last 5 numbers of its DOI, provide it using \articleid{} commmand 
% list of available sesscions in the journal: articles, dossier, reports, essays, reviews, interviews, editorial
\articlesessionname{articles}
\runningauthor{Loureiro; Ibáñez-Cubillas e Miranda-Pinto}
%\editorname{Leonardo Araújo} % old template
\sectioneditorname{Daniervelin Pereira}
\layouteditorname{João Mesquita}

\title{Validade de conteúdo por avaliação de especialistas para medir as competências digitais de estudantes de mestrado em Educação Especial}
\othertitle{Content validity by expert evaluation to measure the digital competences of master's degree students in Special Education}

\author[1]{Ana Claudia Loureiro~\orcid{0000-0001-7919-6891 }\thanks{Email: \href{mailto:analoureiro@esev.ipv.pt}{analoureiro@esev.ipv.pt}}}
\author[2]{Pilar Ibáñez-Cubillas~\orcid{0000-0001-7117-5746}\thanks{Email: \href{mailto:pcubillas@uma.es}{pcubillas@uma.es}}}
\author[3]{Maribel dos Santos Miranda-Pinto~\orcid{0000-0003-0813-1497 }\thanks{Email: \href{mailto:maribel.miranda@uab.pt}{maribel.miranda@uab.pt}}}
\affil[1]{Instituto Politécnico de Viseu, Escola Superior de Educação, Departamento de Comunicação e Arte, Viseu, Portugal.}
\affil[2]{Universidade de Málaga, Departamento de Didáctica y Organización Escolar, Málaga, Espanha.}
\affil[3]{Universidade Aberta, Departamento de Educação e Ensino a Distância, Lisboa, Portugal.}

\addbibresource{article.bib}

\usepackage{array}
\usepackage{xfrac}

\begin{document}
\maketitle
\begin{polyabstract}
\begin{abstract}
O presente estudo aborda a utilização da avaliação de especialistas como estratégia na validação de um instrumento de investigação na área educativa. O nosso objetivo foi o de validar um questionário sobre a utilização da robótica e da programação educativa em contextos de inclusão. A relevância do estudo centra-se na necessidade de ter um instrumento validado que avalie o conhecimento e o desenvolvimento das competências digitais nos alunos dos cursos de mestrado em Educação Especial, em Portugal. A metodologia adotada foi um estudo psicométrico de validade de conteúdo através da avaliação de especialistas. Os especialistas foram selecionados através da técnica do “Coeficiente de competência do perito” (Coeficiente K). A recolha e análise de dados decorreu no segundo semestre de 2023. Foram contabilizados um total de doze especialistas de Portugal na área da Educação Inclusiva, robótica e programação educativa. Os resultados apontam que o painel de especialistas que validou o questionário apresenta um Coeficiente K de elevada influência (0,84) na área de conhecimento. A avaliação global do questionário apresentou um valor de V de Aiken de 0,79. Com este estudo foi possível aferir que o questionário tem evidências suficientes de validade de conteúdo em termos de coerência, relevância e pertinência, sendo um instrumento fiável para conhecer, como a oferta formativa na área da Educação Inclusiva promove o desenvolvimento das competências digitais nos futuros docentes desta área, em Portugal.

\keywords{Validação de instrumento \sep Robótica educativa \sep Programação \sep Educação Inclusiva}
\end{abstract}

\begin{english}
\begin{abstract}
This study looks at the use of expert judgement as a strategy for validating a research tool in the field of education. Our aim was to validate a questionnaire on the use of robotics and programming in inclusive contexts. The relevance of the study is based on the need to have a validated instrument that measures the knowledge and development of digital competences in students on master's programmes in Special Education in Portugal. The methodology adopted was a psychometric study of content validity through expert judgement. The experts were selected using the ‘Expert Competence Coefficient’ (K Coefficient) technique. Data collection and analysis took place in the second half of 2023. A total of twelve experts from Portugal in the field of inclusive education, robotics and educational programming were counted. The results show that the panel of experts that validated the questionnaire has a highly influential K Coefficient (0.84) in expertise. The overall evaluation of the questionnaire showed an Aiken's V value of 0.79. With this study, it was possible to ascertain that the questionnaire has sufficient evidence of content validity in terms of coherence, relevance and pertinence, making it a reliable instrument for finding out how training provision in Inclusive Education promotes the development of digital competences in future teachers in this area in Portugal.

\keywords{Instrument validation \sep Educational robotics \sep Programming \sep Inclusive education}
\end{abstract}
\end{english}
\end{polyabstract}

\section{Introdução}
Aprender a programar um robô tornou-se importante nos programas curriculares em Portugal e a nível mundial e ganhou destaque nas atividades dos contextos educativos. Tal relevância deu-se com a introdução do pensamento computacional nos currículos obrigatórios e não obrigatórios como no pré-escolar, como competência fundamental a ser desenvolvida pelas crianças e jovens, durante a sua escolaridade \cite{bers_coding_2017,kafai_computational_2017,loureiro_educational_2023,loureiro_robotica_2022,ramos_introducao_2022,resnick_coding_2020,valente_pensamento_2019}, o que tornou o desenvolvimento do pensamento computacional um tema de investigação em grande expansão na última década. Dentre as muitas perspectivas conceituais encontradas na literatura, verifica-se uma convergência para o relacionar com o desenvolvimento do raciocínio lógico aplicando competências computacionais de decomposição, abstração, reconhecimento de padrões e pensamento algorítmico. O termo \textit{computational thinking} foi utilizado pela primeira vez por \textcite{papert_mindstorms:_1993} no seu livro intitulado \textit{Mindstorms: Children, Computers and Powerful Ideas}. No entanto, quem apresentou o pensamento computacional definindo-o como uma competência fundamental para todos na sociedade digital, foi \textcite{wing_computational_2006}. J. Wing sugere, desde então, a integração do pensamento computacional no contexto educativo fundamentada na conceção de que este é transdisciplinar, universal e de utilidade para todos, independentemente da área de formação das futuras crianças e jovens. Diferentes investigadores e educadores têm trabalhado nesse conceito gerando um número substancial de publicações, associando frequentemente, o pensamento computacional à programação \cite{bers_blocks_2008,bers_coding_2018,bers_coding_2019,kafai_computational_2016} e à robótica educativa.

Os conceitos de pensamento computacional e programação são pilares fundamentais para o trabalho com a robótica educativa, apoiando as atividades que podem ser desenvolvidas com a ajuda de robôs. Integrar a robótica educativa e a programação nas atividades escolares permite uma aprendizagem baseada na resolução de problemas reais, bem como a inclusão de crianças com necessidades educativas, uma vez que estes recursos têm um potencial relevante para a aprendizagem, promovendo o desenvolvimento de competências socioemocionais e pessoais, fatores-chave no processo de aprendizagem e inclusão.

Diversos estudos argumentam que a robótica facilita a inclusão de todos os alunos, especialmente daqueles com necessidades educativas \cite{casaca_crescer_2018,conchinha_playful_2015,loureiro_educational_2023,lourenco_utilizacao_2019,marcao_robots_2017}. Esses estudos revelam que a integração de atividades de robótica tem se mostrado eficaz no currículo que é desenvolvido com crianças com necessidades educativas, pois os robôs incentivam a participação e a interação dessas crianças com seus pares sem necessidades educativas, durante as atividades curriculares. A atividade com robôs envolve a aprendizagem da programação e a forma como são utilizados incluindo a ludicidade, incorporando os conceitos da programação e consequentemente da robótica (sensores, luzes coloridas ou sequências e/ou percursos), para que possam estimular e interagir tanto com crianças com necessidades educativas quanto com aquelas sem essas necessidades. Na área educativa, essa prática é vista como um recurso valioso para promover a Educação Inclusiva, atuando não apenas no desenvolvimento social e lúdico, mas também no apoio ao progresso cognitivo e motor das crianças a quem se destina.

As potencialidades da programação e da robótica em contextos inclusivos, também podem ser verificadas nos estudos de \textcite{adams_k.;_alvarez_l.;_rios_a.;_encarnacao_p.;_cook_a._big_2012,ferm_participation_2015,gonzalez_gonzalez_estrategias_2019,sheehy_beaming_2011}, cujos estudos destacam a importância de encontrar formas de ajudar as crianças e jovens com necessidades educativas a vivenciar as experiências motoras, o desenvolvimento das capacidades de comunicação, manipulação, exploração e utilização de instrumentos para agir sobre os objetos e as pessoas, promovendo o desenvolvimento cognitivo e a percepção dessas crianças e jovens. É de realçar que a robótica e a programação podem contribuir para o desenvolvimento dessas capacidades, que lhes são limitadas.

No entanto, sabemos que para que esta experiência de aprendizagem ocorra em contexto educativo, os profissionais da educação necessitam de formação para se sentirem capazes de integrar eficazmente as atividades de programação e robótica em contextos educativos e de forma inclusiva para todas as crianças e jovens. Esta formação pode surgir quer na formação inicial, como pós-graduada, mas sempre acompanhada de especialistas que promovam o conhecimento e a utilização de recursos adequados para o trabalho que se pode realizar na Educação Inclusiva.

Este estudo parte de uma investigação exploratória anterior \cite{loureiro_robotica_2022}, realizada em uma instituição de ensino superior politécnico, na região Centro-Interior de Portugal, que nos permitiu realizar um estudo sobre as necessidades de formação dos futuros profissionais da Educação Inclusiva. Pretendemos agora, ampliar a nossa investigação para verificar a oferta formativa na área de competências digitais nos contextos de formação pós-graduada de Educação Inclusiva, nas demais regiões de Portugal. Iniciamos esta nova fase investigativa com a validação de um questionário elaborado pelas autoras. Para isso, utilizamos a abordagem de seleção de especialistas que ocorreu em duas etapas: a primeira etapa consistiu na seleção dos especialistas e foi utilizada a determinação do Coeficiente K \cite{cabero_almenara_utilizacion_2013,garcia_procedimiento_2008,lopez_gutierrez_establecimiento_2011,luis_desenvolvimento_2023}, que é calculado de acordo com a autoavaliação do candidato sobre o seu nível de competência especializada no tema da investigação e as fontes que lhe permitem argumentar os seus critérios; na segunda etapa foi feita a análise quantitativa de cada item do questionário, através do cálculo do coeficiente V de Aiken \cite{aiken_content_1980}. O presente trabalho descreve essas etapas e a análise dos resultados obtidos.

Realçamos que no ano de 2018, foi criado o Decreto-Lei n.º 54/2018 \cite{portugal_decreto-lei_2018} que trouxe mudanças significativas no campo da Educação Especial em Portugal, promovendo uma abordagem mais inclusiva e equitativa para alunos com necessidades educativas especiais (NEE). Este diploma substituiu o anterior Decreto-Lei n.º 3/2008 e introduziu novas diretrizes para a Educação Inclusiva, dentre elas, a substituição do conceito de “Educação Especial” pelo de “Educação Inclusiva”, que visa promover uma educação mais abrangente, adaptada às necessidades de cada aluno e comprometida com o desenvolvimento de todos, independentemente das suas dificuldades. Também, a sigla NEE (Necessidades Educativas Especiais) deixou de existir para haver apenas “necessidades educativas”. Fundamentadas neste decreto, utilizamos para esta investigação o conceito “Educação Inclusiva” e “necessidades educativas”, no que diz respeito à formação dos futuros profissionais, mantendo o termo “Educação Especial” na designação nominal dos cursos de mestrado que, em Portugal, ainda continuam a ser assim designados pelas instituições de ensino superior e aprovados pela agência de acreditação (A3Es), sendo que já se encontram alguns cursos com as duas designações “Educação Especial e Inclusiva”.

\subsection{Enquadramento teórico}\label{sec-normas}
A metodologia de avaliação de um instrumento de pesquisa que utiliza a seleção de especialistas com base no “Coeficiente de competência especializada” – Coeficiente K, é uma abordagem valiosa para garantir a validade e a fiabilidade de um instrumento e deve ser considerada pelo investigador como uma abordagem metodológica essencial em qualquer método experimental \cite{betancurth_loaiza_validation_2015,cabero_almenara_utilizacion_2013,ventura-silva_escala_2023}. Os resultados do Coeficiente K ajudam o investigador a avaliar a validade e a qualidade do instrumento elaborado para a recolha de dados. Um coeficiente alto sugere que os especialistas estão de acordo sobre a qualidade do instrumento, enquanto um coeficiente baixo pode indicar a necessidade de revisão e aprimoramento do instrumento. Com base no \textit{feedback} dos especialistas e nos resultados do Coeficiente K, o investigador pode fazer ajustes e melhorias no instrumento de pesquisa, se necessário. Essa abordagem ajuda a garantir que o instrumento de pesquisa seja confiável e válido, uma etapa crucial no processo de pesquisa, pois afeta diretamente a qualidade dos resultados obtidos. A validade de conteúdo estabelece a relação entre o conceito teórico e o indicador empírico referenciado, indicando o grau em que os indicadores cobrem a variedade de significados do conceito \cite{nora_validacao_2018}.


\section{Métodos}\label{sec-conduta}
\subsection{Amostra 1}
Neste estudo foi efetuado um desenho descritivo e psicométrico para apresentar o procedimento de validação de conteúdo por julgamento de especialistas de um questionário para saber de que forma a oferta formativa na área da Educação Inclusiva promove o desenvolvimento de competências digitais em contextos inclusivos.

A seleção dos especialistas foi determinada pelo Coeficiente K, que consiste numa técnica centrada na autoavaliação realizada pela pessoa para determinar a sua competência especializada no objeto da investigação \cite{cabero_almenara_utilizacion_2013}. Para o cálculo do Coeficiente K, foi tida em conta a opinião do especialista sobre o seu nível de conhecimento a respeito da Educação Inclusiva, robótica e programação educativa e as fontes que lhe permitem argumentar, utilizando as questões sobre o "coeficiente de conhecimento" (Kc) e o "coeficiente de argumentação" (Ka), aplicando a fórmula $K= \sfrac{1}{2} (Kc+Ka)$ indicada por \textcite[p. 29]{cabero_almenara_utilizacion_2013}, que referem a:

\begin{itemize}
    \item Kc, corresponde ao "coeficiente de conhecimento" ou à informação que o especialista possui sobre o assunto ou problema colocado. É calculado com base na avaliação do próprio especialista, numa escala de 0 a 10, multiplicada por 0,1;
    \item Ka, relativo ao "coeficiente de argumentação" ou fundamentação dos critérios do especialista. É calculado através da atribuição de uma série de pontuações (0,1, 0,2, 0,3, 0,4, 0,5, 0,05) às diferentes fontes de argumentação que o especialista foi capaz de tratar.
\end{itemize}

Uma vez obtidas as pontuações de cada especialista, o nível de competência é determinado pelo valor obtido no seu Coeficiente K, de modo que:

\begin{itemize}
    \item Se K for maior que 0,8 há uma alta influência;
    \item Se K for menor ou igual a 0,8, e maior que 0,7 há uma influência média;
    \item Se K for menor ou igual a 0,7 e maior que 0,5, a influência é baixa.
\end{itemize}

A identificação das pessoas que farão parte da avaliação dos especialistas é uma parte fundamental deste processo. Para o efeito, contamos com um painel inicial de 21 especialistas em Educação Inclusiva, robótica e programação educativas de Portugal. Os especialistas receberam por e-mail as informações sobre o estudo, seus objetivos e a solicitação para participação no processo de validação de conteúdo do questionário. Um total de 12 especialistas aceitaram participar voluntariamente no estudo, assinalaram a caixa de Consentimento Informado e Esclarecido e preencheram o questionário \textit{Expert Competence} \cite{cabero_almenara_utilizacion_2013} elaborado na ferramenta Google Forms® (Formulários do Google).

Como resultado, a pontuação média obtida pelos 12 especialistas é de 0,84, o que implica que o painel de especialistas que validou o questionário "Robótica e Programação Educativas em contextos de inclusão" tem um Coeficiente K de elevada influência na área de conhecimento. A nível individual, o painel é constituído por 7 especialistas de elevada influência (K superior a 0,8), 3 de média influência (K inferior ou igual a 0,8 e superior a 0,7) e 2 de baixa influência (K inferior ou igual a 0,7 e superior a 0,5). A \Cref{tab01} apresenta o nível de competência dos especialistas na área de conhecimento, a posição acadêmica e a cidade portuguesa em que cada especialista trabalha.

\begin{table}[h!]
\centering
\begin{threeparttable}
\caption{Caracterização do painel de especialistas.}
\label{tab01}
\begin{tabular}{l l >{\raggedright\arraybackslash}p{4cm} l}
\toprule
Experto & \multicolumn{1}{>{\raggedright\arraybackslash}p{2cm}}{Coeficiente de competência especializada (K)} & Cargo acadêmico &
Cidade \\
 \midrule
1 & 0,95 & Professor Adjunto & Viseu \\
2 & 0,75 & Professor Coordenador & Viseu \\
3 & 0,7 & Professora Adjunta & Leiria \\
4 & 0,95 & Formadora Externa & Coimbra \\
5 & 0,75 & Professora Auxiliar & Ponta Delgada \\
6 & 0,95 & Professor Coordenador & Bragança \\
7 & 0,65 & Professora Adjunta & Bragança \\
8 & 0.8 & Professor Adjunto & Porto \\
9 & 0.85 & Professor (CCTIC – Instituto de Educação da Universidade do Minho) & Amarante \\
10 & 0.9 & Professora Coordenadora Reformada & Setúbal \\
11 & 0.95 & Professora do 1º Ciclo & Barcelos \\
12 & 0.95 & Professora de TIC (1º e 2º Ciclos) & Braga \\
\bottomrule
\end{tabular}
\source{Criada pelas autoras.}
\end{threeparttable}
\end{table}

\subsection{Instrumentos de recolha de dados}\label{sec-fmt-manuscrito}
O processo de validação por julgamento de especialistas foi aplicado a um questionário de 20 itens construído sobre o conhecimento de robótica e programação educativas, que é entendido como instrumento de recolha de dados. O instrumento foi elaborado pelas autoras desta pesquisa e dividido em quatro dimensões: Bloco 1 – Consentimento informado e esclarecido (1 item); Bloco 2 – Dados pessoais (7 itens); Bloco 3 – Percepção da robótica e da codificação em contextos inclusivos (10 itens); Bloco 4 – Relevância da robótica e da codificação em contextos inclusivos (2 itens). O objetivo é verificar se a formação oferecida nos cursos de Mestrado em Educação Especial promove o desenvolvimento de competências digitais em contextos de inclusão. O questionário foi avaliado item a item, utilizando a grelha de avaliação de especialistas. Os especialistas avaliaram o instrumento com base em 3 categorias: (i) Coerência, o item é suscetível de ser entendido ou interpretado de uma só forma; (ii) Pertinência, o item é suscetível de pertencer ao grupo-alvo e à faixa etária; (iii) Relevância, adequação e relação do item com o objetivo da investigação. 

Foram estabelecidas quatro alternativas de opinião do item para cada categoria, com os valores entre: 0 – Nulo | 1 – Baixo | 2 – Boa | 3 – Muito boa, e uma questão aberta para os especialistas poderem fazerem comentários qualitativos sobre cada item.


\subsection{Procedimentos}\label{sec-formato}
Em primeiro lugar, os especialistas foram contactados por correio eletrônico, convidando-os a participar como membros do painel de especialistas (garantindo a sua confidencialidade) para validar o questionário "Robótica e Programação Educativas em contextos inclusivos". A aceitação do convite implicou no preenchimento online do “Questionário de Especialistas” (K-\textit{Quotient}) e a seleção da caixa de consentimento informado. Uma vez preenchido o “K-\textit{Quotient}”, os especialistas tiveram acesso à segunda fase, que consistiu na avaliação do questionário, instrumento de recolha de dados deste estudo. Foi pedido aos especialistas que classificassem várias características do instrumento, tais como a clareza das perguntas, a relevância das perguntas e a estrutura geral. Os dados foram recolhidos e analisados nos meses de julho a dezembro de 2023.


\subsection{Análise estatística}\label{sec-modelo}
Os dados obtidos foram armazenados e tratados estatisticamente no Excel da Microsoft 365. Primeiro, determinou-se o nível de \textit{Expert Competence} dos especialistas \cite{cabero_almenara_utilizacion_2013}. De seguida, determinou-se a concordância dos especialistas relativamente à relevância de cada item através da técnica do V de Aiken \cite{aiken_three_1985,merino-soto_coeficientes_2023} e calculou-se o intervalo de confiança do V de Aiken através do método de \textcite{wilson_probable_1927}, conhecido como método da pontuação, de forma a conhecer o grau de imprecisão associado ao item.

O coeficiente V de Aiken permite estimar quantitativamente a validade de conteúdo dos itens que compõem o instrumento, a partir das pontuações obtidas pelos especialistas. Esse coeficiente tem valores entre 0,00 e 1,00, sendo o valor 1,00 a magnitude máxima, indicando concordância máxima entre os especialistas. Quanto mais elevado for o valor calculado, mais elevada é a validade de conteúdo do item \cite{mayaute_cuantificacion_1988,pereira_escala_2023}. Consequentemente, se $V=0$, significa que há discordância total com os itens; se $V=1$, significa que há concordância total com todos os itens.

A fim de controlar o erro de amostragem, é importante especificar a gama de valores possíveis que o coeficiente assumiria utilizando intervalos de confiança. Assim, foram aplicadas as equações de \textcite{penfield_applying_2004} para calcular o coeficiente V de Aiken e os respetivos intervalos de confiança (IC). Como critério de decisão para considerar um item como válido, o valor do V de Aiken deve ser $\geq 0,78$ e o valor do menor intervalo de confiança deve ser $\geq 0,50$ \cite{cicchetti_guidelines_1994}. A significância estatística do coeficiente V foi obtida a partir da tabela de probabilidades binomiais de \textcite[p. 133]{aiken_three_1985}, que fornece "probabilidades próximas, mas não excedendo os níveis de 0,05 e 0,01". Os itens com um valor inferior ao indicado seriam objeto de revisão com base nas recomendações de melhoria dos especialistas, a menos que estes indicassem que deveriam ser eliminados.

\section{Resultados}\label{sec-organizacao}
A análise da avaliação quantitativa através do cálculo do coeficiente V de Aiken produziu coeficientes ótimos com uma avaliação global de 0,79. A \Cref{tab02} apresenta os resultados dos itens quantificados pelo V de Aiken e os intervalos de confiança para cada um dos critérios (i) coerência, (ii) pertinência e (iii) relevância.

\begin{table}[h!]
\centering
\begin{small}
\begin{threeparttable}
\caption{Média do V de Aiken e Intervalo de Confiança (IC) da relevância de cada item por critério.}
\label{tab02}
\begin{tabular}{*{10}{l}}%{p{1,5cm} p{0,4cm} p{1cm} p{1,5cm} p{1cm} p{1cm} p{1,5cm} p{1cm} p{1cm} p{1,5cm}}
\toprule
\multicolumn{4}{l}{Coerência} & \multicolumn{3}{l}{Pertinência} & \multicolumn{3}{l}{Relevância} \\
 \midrule
Item & Média & \multicolumn{1}{>{\raggedright\arraybackslash}p{1cm}}{V de Aiken} & IC 95\% & Média & \multicolumn{1}{>{\raggedright\arraybackslash}p{1cm}}{V de Aiken} & IC 95\% & Média &
\multicolumn{1}{>{\raggedright\arraybackslash}p{1cm}}{V de Aiken} & IC 95\% \\
Item 1 & 3.42 & 0.81 & 0.65-0.90 & 3.50 & 0.83 & 0.68-0.92 & 3.50 & 0.83 & 0.68-0.92 \\
Item 2 & 3.17 & 0.72* & 0.56-0.84 & 3.42 & 0.81 & 0.65-0.90 & 3.33 & 0.78 & 0.62-0.88 \\
Item 3 & 3.42 & 0.81 & 0.65-0.90 & 3.25 & 0.75* & 0.59-0.86 & 3.42 & 0.81 & 0.65-0.90 \\
Item 4 & 3.25 & 0.75* & 0.59-0.86 & 3.25 & 0.75* & 0.69-0.86 & 3.42 & 0.81 & 0.65-0.90 \\
Item 5 & 3.67 & 0.89 & 0.75-0.96 & 3.58 & 0.86 & 0.71-0.94 & 3.58 & 0.86 &
0.71-0.94 \\
Item 6 & 3.50 & 0.83 & 0.68-0.92 & 3.50 & 0.83 & 0.68-0.92 & 3.50 & 0.83 & 
0.68-0.92 \\
Item 7 & 3.67 & 0.89 & 0.75-0.96 & 3.67 & 0.89 & 0.75-0.96 & 3.75 & 0.92 & 0.78-0.97 \\
Item 8 & 3.33 & 0.78 & 0.62-0.88 & 3.42 & 0.81 & 0.65-0.90 & 3.42 & 0.81 & 0.65-0.90 \\
Item 9 & 3.25 & 0.75* & 0.59-0.86 & 3.33 & 0.78 & 0.62-0.88 & 3.33 & 0.78 &
0.62-0.88 \\
Item 10 & 3.25 & 0.75* & 0.59-0.86 & 3.33 & 0.78 & 0.62-0.88 & 3.33 & 0.78 & 0.62-0.88 \\
Item 12 & 3.17 & 0.72* & 0.56-0.84 & 3.17 & 0.72* & 0.56-0.84 & 3.08 & 0.69* & 0.53-0.82 \\
Item 13 & 3.33 & 0.78 & 0.62-0.88 & 3.33 & 0.78 & 0.62-0.88 & 3.33 & 0.78 & 0.62-0.88 \\
Item 14 & 2.92 & 0.64* & 0.48-0.78 & 3.08 & 0.69* & 0.53-0.82 & 3.08 & 0.69* & 0.53-0.82 \\
Item 15 & 3.08 & 0.69* & 0.53-0.82 & 3.33 & 0.78 & 0.62-0.88 & 3.33 & 0.78 & 0.62-0.88 \\
Item 16 & 3.50 & 0.83 & 0.68-0.92 & 3.50 & 0.83 & 0.68-0.92 & 3.50 & 0.83 & 0.68-0.92 \\
Item 17 & 3.25 & 0.75* & 0.59-0.86 & 3.25 & 0.75* & 0.59-0.86 & 3.33 & 0.78 & 0.62-0.88 \\
Item 18 & 3.33 & 0.78 & 0.62-0.88 & 3.42 & 0.81 & 0.65-0.90 & 3.42 & 0.81 & 0.65-0.90 \\
Item 19 & 3.08 & 0.69* & 0.53-0.82 & 3.50 & 0.83 & 0.68-0.92 & 3.50 & 0.83 & 0.68-0.92 \\
Item 20 & 3.58 & 0.86 & 0.71-0.94 & 3.58 & 0.86 & 0.71-0.94 & 3.58 & 0.86 & 0.71-0.94 \\
\multicolumn{2}{>{\raggedright\arraybackslash}p{2cm}}{V de Aiken por Critério} & 0.78 & & & 0.80 & & & 0.80 & \\
\bottomrule
\end{tabular}
\begin{tablenotes}
\small{
\item{* “Valores inferiores a 0,78 para um p>0,05”.}
 }
\end{tablenotes}
\source{Criada pelas autoras.}
\end{threeparttable}
\end{small}
\end{table}

Para fazer a análise qualitativa dos itens, decorrente das observações escritas pelos especialistas nas questões abertas, relativamente à coerência, pertinência e relevância dos itens, agrupou-se os comentários segundo cada item avaliado, o que nos permitiu analisar a pertinência de inclusão ou exclusão dos mesmos e a pertinência da adoção ou não das alterações propostas pelos especialistas. Para essa análise, foram utilizados os resultados estatísticos obtidos na Média do V de Aiken e Intervalo de Confiança (IC) da relevância de cada item por critério. A análise permitiu especificar a fundamentação para a correção e melhoria dos itens com valores $>0,78$ no V de Aiken, bem como os itens que poderiam ser ajustados e melhorados sem pôr em causa a validade de conteúdo, cumprindo assim o processo exigido para o estudo. Assim, dos 20 itens submetidos aos especialistas, 10 tinham sugestões de adaptação e 4 apenas observações sobre erros de digitação. Algumas adaptações foram necessárias para tornar as questões claras e de fácil compreensão, outras eram apenas sugestões de inclusão ou reelaboração da escrita. A versão final ficou com 24 itens.

\section{Discussão}\label{sec-organizacao-latex}
O número de especialistas (12) é suficiente e estatisticamente significativo, uma vez que, de acordo com \textcite{landeta_metodo_1999}, uma maior participação não teria reduzido o erro cometido. A pontuação média de 0,84 obtida pelos especialistas participantes deste estudo, significa que este painel tem um Coeficiente K de elevada influência na área de conhecimento, sendo válida as contribuições sugeridas para uniformizar os termos do questionário, tornando os itens mais coerentes, claros e de fácil percepção. Segundo \textcite{souza_propriedades_2017}, a validade de conteúdo refere-se ao grau em que o conteúdo de um instrumento reflete adequadamente o que se quer medir, ou seja, é a avaliação do quanto uma amostra de itens é representativa de um universo definido. Os resultados da validação de conteúdo pelos especialistas indicaram a necessidade de algumas adaptações quanto à clareza e escrita. Nesse contexto, é preciso analisar a coerência da sugestão com os objetivos do estudo e a pertinência da adoção ou não das alterações propostas pelos especialistas. A partir dessa análise, prosseguimos para a validação de cada um dos itens do questionário.

No total, 10 itens do questionário foram apontados com necessidades de alteração. Essas alterações incluíam: (i) a reescrita de algumas questões, a procura de maior clareza para os inquiridos; (ii) a substituição do termo “codificação” para “programação”; (iii) a inclusão da opção de mais alguns modelos de robôs educativos e; (iv) a divisão do Bloco 4 “Pertinência da Robótica e Programação em contextos de inclusão”, que contava com 5 afirmativas, para 10 afirmativas separadas em 5 afirmativas sobre a robótica educativa e 5 sobre a programação educativa. A \Cref{tab03} apresenta os itens alterados de acordo com as sugestões dos especialistas.

\begin{table}[h!]
\centering
\begin{threeparttable}
\caption{Validação do instrumento, segundo os especialistas.}
\begin{small}
\label{tab03}
\begin{tabular}{p{0.5cm} >{\raggedright\arraybackslash}p{4.5cm} >{\raggedright\arraybackslash}p{6cm}}
\toprule
Item & Bloco/ questão – Original & Sugestão - Alteração \\
 \midrule
1 & Bloco 2 – Dados pessoais Q.2 Idade & Incluir a categoria “acima de 60 anos”. \\
2 & Bloco 2 – Percepção sobre Robótica e Codificação em contextos de inclusão. Q.1 “Sabe o que é robótica? & Especificar o tipo de robótica (robótica educativa); reformular a pergunta: “Considera possuir conhecimentos sobre robótica educativa?” \\
3 & Q.2 "Em caso afirmativo, o quanto considera que sabe?" & Reformular a pergunta: "Em caso afirmativo, como classificaria o seu grau de conhecimento?” \\
4 & Q.3 “Conhece o trabalho que pode ser desenvolvido na área da robótica, em contextos de inclusão?” & Reformular a pergunta: “Considera possuir conhecimento sobre o trabalho que pode ser desenvolvido na área da robótica educativa, em contextos de inclusão?” \\
5 & Q.4 “Em caso afirmativo, o quanto considera que sabe?" & Reformular a pergunta: "Em caso afirmativo, como classificaria o seu grau de conhecimento?” \\
6 & Q.5 “Identifique o/os Robô/s Educativo/s que conhece que sejam adequados para o contexto de inclusão” & Incluir robôs da MATATALAB, que não constavam na listagem de robôs apresentadas nas alternativas originais. \\
7 & Q.7 “Sabe o que é codificação?” & Alterar a nomenclatura “codificação” para “programação” e reelaborar a forma da pergunta: “Considera possuir conhecimentos sobre programação educativa?” \\
8 & Q.8 "Em caso afirmativo, o quanto considera que sabe?" & Reformular a pergunta: "Em caso afirmativo, como classificaria o seu grau de  conhecimento?” \\
9 & Q.9 “Conhece o trabalho que pode ser desenvolvido na área da Codificação, em contextos de inclusão?” & Alterar a nomenclatura “codificação para “programação” e reelaborar a pergunta: “Considera possuir conhecimento sobre o trabalho que pode ser desenvolvido na área da programação educativa em contextos de inclusão?” \\
10 & Bloco 3 – Pertinência da Robótica e Programação em contextos de inclusão. “Responda, de acordo com a pertinência de cada uma das seguintes afirmações sobre robótica e programação educativas”. & Reformular a questão separando os temas: “Responda, de acordo com a pertinência de cada uma das seguintes afirmações sobre robótica educativa”. 

“Responda, de acordo com a pertinência de cada uma das seguintes afirmações sobre programação educativa”. \\
\bottomrule
\end{tabular}
\source{Criada pelas autoras.}
\end{small}
\end{threeparttable}
\end{table}

Com a avaliação dos especialistas, foi possível aferir que o Bloco 2 “Percepção da robótica e da codificação em contextos inclusivos”, com 10 itens, foi o que obteve maior número de sugestões de alteração (8 questões), todas elas muito pertinentes e de alto valor contributivo para uniformizar os termos do questionário, tornando os itens mais coerentes, claros e de fácil percepção, para além de adequar o que se pretende medir com os objetivos do estudo, possibilitando uma recolha de dados mais representativa para o tema investigado.

Seis dos itens do questionário não tiveram nenhum comentário ou sugestão dos especialistas e permaneceram da mesma forma como foram elaborados e os demais quatro itens tiveram observações sobre erros de digitação, que foram corrigidos. O termo de Consentimento Informado e Esclarecido encontrava-se na última seção do questionário na versão de avaliação dos especialistas, sendo movido para a primeira seção do questionário final, destinado ao público-alvo da investigação, de modo que somente será possível avançar aos demais blocos de perguntas, se o termo de consentimento for assinalado “concordo” pelo participante.

\section{Conclusão}\label{sec-autores}
É muito comum a utilização de avaliação de especialistas como estratégia de avaliação de instrumentos de recolha de dados para uma investigação científica. Para este estudo foi efetuado um desenho descritivo e psicométrico, para apresentar o procedimento de validação de conteúdo por avaliação de especialistas de um questionário, com o objetivo de verificar a oferta formativa nos cursos de mestrado em Educação Especial, em Portugal, quanto ao desenvolvimento de competências digitais em contextos inclusivos. A seleção dos especialistas foi determinada pelo Coeficiente K, numa técnica centrada na autoavaliação realizada pela pessoa para determinar a sua competência especializada no objeto da investigação \cite{cabero_almenara_utilizacion_2013}. Para o cálculo do Coeficiente K, foi tida em conta a opinião do perito sobre o seu nível de conhecimento sobre Educação Inclusiva, robótica e programação educativas e as fontes que lhe permitem argumentar, utilizando as questões sobre o "coeficiente de conhecimento" (Kc) e o "coeficiente de argumentação" (Ka), aplicando a fórmula $K= \sfrac{1}{2} (Kc+Ka)$ indicada por \textcite{cabero_almenara_utilizacion_2013}. O questionário foi avaliado item a item, utilizando a grelha de avaliação de especialistas.

O método V de \textcite{aiken_content_1980} foi utilizado para determinar o valor estatístico do teste. As pontuações foram calculadas para as 3 categorias por indicador: (i) coerência do item, (ii) relevância e (iii) pertinência, com quatro alternativas de opinião do item para cada categoria, com os valores entre: 0 – Nulo | 1 – Baixo | 2 – Boa | 3 – Muito boa, e uma questão aberta para os especialistas poderem fazer comentários qualitativos sobre cada item. A validação do conteúdo do questionário como instrumento de recolha e análise de dados por um comitê de especialistas é essencial para os investigadores na área da robótica e programação educativas em contextos inclusivos, bem como para os planos de estudo dos cursos de mestrado em Educação Especial, uma vez que a validação aumenta a sua fiabilidade. É importante recorrer a especialistas para confirmar que o questionário constitui um universo de itens que delimita claramente o tema em estudo, bem como ajuda a homogeneizar conceitos, tornando as questões mais claras, fáceis de compreender e pertinentes aos objetivos do estudo. Com base nos valores obtidos, nos comentários e sugestões dos especialistas, o questionário apresentou evidências suficientes de validade de conteúdo em termos de coerência, relevância e pertinência, sendo um instrumento fiável para conhecer como a oferta formativa na área da Educação Especial, em Portugal, promove o desenvolvimento das competências digitais nos futuros docentes em Educação Especial e Inclusiva. O presente estudo poderá servir de orientação para as futuras ofertas formativas nesta área.

Esta investigação tem como proposta futura o segundo momento deste estudo, previsto para decorrer no primeiro semestre de 2024/2025, mais especificamente entre os meses de maio e junho, período em que iremos aplicar o questionário validado aos estudantes de mestrado em Educação Especial de 12 instituições de ensino superior de Portugal. Essa etapa está aprovada pelo Conselho Ético da Universidade Aberta, uma das instituições pertencentes ao estudo. Como contribuição futura, pretendemos apresentar uma proposta formativa na área da robótica e programação educativas em contextos inclusivos.



\printbibliography\label{sec-bib}
%conceptualization,datacuration,formalanalysis,funding,investigation,methodology,projadm,resources,software,supervision,validation,visualization,writing,review
\begin{contributors}[sec-contributors]
\authorcontribution{Ana Claudia Loureiro}[conceptualization,investigation,projadm,visualization,writing,review]
\authorcontribution{Pilar Ibáñez-Cubillas}[conceptualization,formalanalysis,methodology,validation,writing]
\authorcontribution{Maribel dos Santos Miranda-Pinto}[conceptualization,writing,review]
\end{contributors}
\end{document}
