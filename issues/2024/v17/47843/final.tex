\section{Para finalizar}\label{sec-parafinalizar}

As duas últimas décadas trouxeram muitas mudanças no mundo, aceleradamente, principalmente no que diz respeito ao uso de tecnologias digitais. As gerações que hoje estão na escola encontram dentro e fora dela um mundo muito diferente do que era o de seus professores ou pais. Trata-se de um mundo mais complexo no sentido de que as relações entre as coisas e as pessoas se colocam em formatos de redes, com direcionamentos múltiplos e com uma maior diversidade, que veio com a pluralidade das informações que agora se encontram disponíveis. As formas de vida tecnológicas \cite{lash2012} modificaram essas relações de maneiras que ainda estão sendo percebidas. Além disso, os tipos de habilidades que as crianças e jovens precisarão ter no futuro são difíceis de serem imaginados ou previstos hoje, razão pela qual é necessário todo um cuidado na escolha do que ensinar e que, por isso mesmo, demanda uma ênfase nos modos de aprender, nas habilidades, no aprofundamento reflexivo. 

A tecnologia não pode ser classificada simplesmente como boa ou ruim, mas vista como um modo de ordenamento das informações no mundo com o qual aprendemos a conviver a partir dessa virada tecnológica digital tão significativa e provocadora de mudanças de paradigmas. Embora não possa ser colocada em nenhuma das extremidades do eixo de valoração positiva ou negativa, a tecnologia certamente é política e deve ser vista como produto de visões e de escolhas operadas por determinados grupos sociais e que atendem a determinados interesses. É importante a atenção às diferenças que se apresentam, do que se conhecia para o que se tem hoje, e a desconfiança de que certas habilidades como profundidade de pensamento, raciocínio, memória, continuam e continuarão sendo úteis e necessárias.

Consideramos oportuna trazer aqui, quase ao final de nossas considerações, as reflexões de \textcite{arendt2014} sobre a educação, feita em meados do século XX, mas ainda bastante pertinentes porque, de certo modo, atemporais. \textcite{arendt2014} situa a educação em um eixo do tempo no qual há um jogo de equilíbrio entre o passado e o futuro, no qual se situam a apresentação à criança de um mundo existente e seu preparo para um mundo que será construído por ela a partir da educação que recebeu. Para \textcite{arendt2014}, esse equilíbrio solicita, ao mesmo tempo, a responsabilidade pelo desenvolvimento da criança e a responsabilidade pela conservação do mundo. Isso pode ser conflitante porque a responsabilidade pela criança volta-se contra o mundo, já que é necessário cuidar dela e protegê-la para que nada de destrutivo, por parte do mundo, lhe aconteça, e, ao mesmo tempo, o mundo necessita de proteção para que não seja destruído pela irrupção do novo que chega com cada geração. Aos que fazem a educação, cabe promover o equilíbrio entre o novo e o velho, entre o que se apresenta e o que se conhece, este simbolizado pelo mundo e aquele, pela criança, que acaba de chegar \cite{noronha2020}. 

Essas reflexões foram feitas por \textcite{arendt2014} em um momento de grande ruptura, da reconstrução da ordem do mundo após a segunda grande guerra. Embora pensar na reorganização do mundo seja algo da ordem da continuidade, pois em todo momento da história há embates entre o novo e o velho, entre as novas gerações e o mundo existente, pensamos que o momento atual de avanço tecnológico se apresente como especialmente disruptivo pelas mudanças que apontamos neste texto e outras tantas referentes ao uso cada vez mais intenso e extenso da tecnologia digital pelos sujeitos contemporâneos. Assim, o papel da escola atualmente é o de, ao mesmo tempo, garantir a alunos espaço para se desenvolverem, protegidos para poderem fazer aflorar seu potencial de criatividade e de aprendizado e também o de apresentar esse mundo a esses alunos, preservando o que dele se conhece e que já se mostrou importante. Há um embate entre o manejo da atenção para dar conta da infinidade de informações que se apresenta e a parada necessária para a leitura profunda.  Um embate como o visto por \textcite{arendt2014} entre o passado e o futuro, que pode ser figurativizado hoje na oposição entre o papel e a tela digital, ou entre o material e o virtual\footnote{ Embora colocadas como oposições, nosso ponto de vista é de que se trata de gradações entre os polos e do tratamento deles também, muitas vezes, como abarcando os dois (virtual e material ao mesmo tempo).}. Para \textcite{sibilia2012}, trata-se de encontrar meios de que a dispersão provocada pelas tecnologias digitais e as novas formas tecnológicas de vida deem lugar a experiências que tenham densidade. Para a antropóloga, o caminho para encontrá-las passa pela criação de experiências sensíveis e conjuntas, tais quais as descritas por \textcite{carr2010} e por \textcite{wolf2019} como sendo possíveis apenas a partir de um cérebro de funcionamento mais lento e verticalizado. 

A desaceleração, quer das leituras, quer das experiências propostas, em vez de ser vista com uma excentricidade em meio ao mundo da conexão incessante e excessiva, nos parece ser um modo de vida que permite tanto o adensamento quanto a produção de significação apontados como essenciais. Trata-se de uma maneira de se (re)estabelecimento a conexão gerada pelas experiências conjuntas não apenas entre a escola e os jovens, mas entre os diferentes sujeitos que os modos de estar no mundo colocam em cena.