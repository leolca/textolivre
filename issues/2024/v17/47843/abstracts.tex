\begin{polyabstract}
\begin{abstract}
Este ensaio tem por objetivo trazer algumas reflexões sobre os efeitos do uso exacerbado da tecnologia nos dias de hoje e sobre o modo como isso afeta os sujeitos contemporâneos e sua capacidade de raciocínio complexo e de aprendizagem dentro e fora da escola. Essas reflexões se assentam sobre questões situadas na intersecção entre tecnologia e ciências humanas, como a superestimulação e a continuidade trazidas pelo uso das tecnologias digitais que se chocam com a necessidade de quietude e desaceleração para a aprendizagem e o raciocínio complexo. A intensificação do uso das tecnologias digitais que vivemos afeta o funcionamento do cérebro e dispersa a atenção, apontada como o principal ativo contemporâneo – donde o termo “economia da atenção” surge para substituir o de \enquote{economia da informação}. Reflete-se também sobre qual o papel da escola no direcionamento de seus alunos quanto a essa mudança de paradigma.
	
\keywords{Tecnologia e Educação \sep Atenção \sep Economia da Informação \sep Hiperconectividade}
\end{abstract}

\begin{abstract}
The aim of this essay is to provide some reflections on the effects of the excessive use of technology today and how this affects contemporary subjects and their capacity for complex thinking and learning in and out of school. These reflections are based on issues situated at the intersection of technology and the human sciences, such as the over-stimulation and continuity brought about by the use of digital technologies, which clash with the need for stillness and deceleration for learning and complex reasoning. The intensified use of digital technologies that we are experiencing affects the functioning of the brain and disperses attention, which has been identified as the main contemporary asset - hence the term \enquote{attention economy} has emerged to replace \enquote{information economy}. It also reflects on the role of schools in guiding their students through this paradigm shift.
	
\keywords{Technology and Education \sep Attention \sep Information economy \sep Hyperconnectivity}
\end{abstract}
\end{polyabstract}
