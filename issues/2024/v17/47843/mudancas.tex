\section{Mudanças no cérebro e no funcionamento da memória}\label{sec-mudançasnocerebroenofuncionamentodamemoria}

Pesquisadores têm atestado, por meio de experimentos, as mudanças em áreas específicas do cérebro por causa de exposição prolongada aos estímulos \textit{online} \cite{bbc2021}.  Se, por um lado, isso vai fazendo com que o cérebro humano se adapte aos estímulos e seja capaz de processar mais rapidamente um número maior de informações, por outro se perde a capacidade de leitura mais concentrada e de elaboração de pensamentos ou raciocínios mais profundos. Trata-se de uma mudança muito significativa e que, como toda mudança, trará efeitos que ainda não podem ser previstos e que só serão percebidos com o tempo.

\textcite[p. 123]{carr2010} explica que o cérebro humano funciona com dois tipos de memória: uma memória curta e uma memória longa. A curta se forma a partir de impressões imediatas, sensações e pensamentos. Sua tendência é existir apenas por alguns segundos. As coisas que aprendemos são armazenadas em memórias de longo prazo, que permanecem no cérebro por dias, meses, anos ou mesmo por toda a vida. Há um tipo específico de memória curta que se chama memória de trabalho \textit{working memory}. É ela que transfere as informações para a memória longa e, portanto, é a responsável por criar nosso armazém de conhecimentos pessoais. É ela também que busca nas memórias armazenadas aquelas que são necessárias para o momento. Está comprovado nessas pesquisas científicas que nosso progresso intelectual deriva dos esquemas que foram adquiridos ao longo de extensos períodos de tempo. Somos capazes de compreender conceitos em nossas áreas de especialidade porque temos esquemas mentais associados a esses conceitos. 

No entanto, a velocidade da internet dificulta a transferência de informação da memória de trabalho para a memória longa. Por causa disso, diminui a habilidade de aprendizagem e torna a compreensão mais superficial. Há experimentos mostrando que quando se atinge o limite da memória de trabalho (o que acontece facilmente quando se está online por causa do número excessivo de informações), essa memória fica sobrecarregada e isso torna difícil ou mesmo impossível distinguir o que é relevante do que não é. O sujeito passa, então, a ser um consumidor de dados acrítico. Os efeitos dessa diminuição da transformação da memória de curso prazo na memória de longo prazo não se fazem conhecer de imediato. O uso de máquinas como ferramentas de extensão do corpo humano facilita muito alguns aspectos da vida, como, por exemplo, não precisarmos mais memorizar números de telefones das pessoas, coisa que se fazia quando eles eram fixos, pois agora temos agendas de contatos armazenadas nos \textit{smartphones}. Por outro lado, os números que se tinham que memorizar eram em quantidade muito menor, um por família, e hoje são um por pessoa. Ou seja, libera-se memória para outras coisas, mas tem-se também um volume de informação muito maior do que antes, o que leva à exaustão apontada por \textcite{carr2010}. 

Em médio prazo, há uma mudança no cérebro, como já tem sido provado em estudos mais recentes. A professora e pesquisadora \textcite{wolf2019} chama nossa atenção para o fato de que o cérebro já foi modificado pelo advento da escrita e da leitura, e que a leitura em telas e recursos digitais o tem modificado de outras maneiras e tem diminuído a capacidade de concentração das pessoas. Ela explica que a leitura em nível superficial serve apenas para se obter informação, enquanto a leitura profunda utiliza mais do córtex cerebral e nos leva a analogias e inferências, o que permite mais criticidade, capacidade de análise e empatia \cite[Wolf em entrevista à BBC]{bbc2021}.

O uso contínuo da internet tem consequências neurológicas que já foram percebidas por aqueles que não são nativos a ela, na redução da capacidade de concentração e de produção em atividades que demandam maior atenção, ou mesmo na incapacidade de manter-se longe do \textit{smartphone} por muito tempo. As gerações que já nasceram nessa roda viva digital, por sua vez, não têm a percepção dessa falta ou dos benefícios que o estágio de concentração podem trazer.

Na era da “economia da atenção” em que nos encontramos, as gigantes da tecnologia nos superestimulam para respondermos o tempo todo às suas demandas, com incessantes notificações sonoras e ou visuais ou tramas narrativas em séries que terminam um capítulo no auge, de forma a obrigar o espectador a continuar a assistir para saber o que vai acontecer. Ou seja, se deixarmos, nosso tempo e atenção ficam devotados a essas demandas infinitas que não apenas distraem e dispersam, mas que, como consequência disso, impedem a desaceleração dos pensamentos. Ao mesmo tempo, o mundo que habitamos torna-se mais complexo e compreendê-lo e habitá-lo requer mais não apenas da atenção, mas também do raciocínio e da reflexão. A sobrevivência em um mundo cada vez mais automatizado demanda que o sujeito se diferencie por aquilo que é próprio do ser humano: raciocínio complexo, reflexão, empatia, criatividade. Mas, a interrupção constante impossibilita a obtenção desses estados de atenção máxima \cite{wolf2019}. Coloca-se, desse modo, outro paradoxo entre a superestimulação dos sentidos, que leva ao entorpecimento e faz o indivíduo permanecer na superfície, e a necessidade de quietude para que se possa atingir raciocínios mais profundos e complexos, imprescindíveis para que se seja humano. 

