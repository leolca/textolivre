\documentclass[spanish]{textolivre}

% metadata
\journalname{Texto Livre}
\thevolume{17}
%\thenumber{1} % old template
\theyear{2024}
\receiveddate{\DTMdisplaydate{2024}{3}{15}{-1}}
\accepteddate{\DTMdisplaydate{2024}{5}{20}{-1}}
\publisheddate{\DTMdisplaydate{2024}{9}{17}{-1}}
\corrauthor{Sebastián Miras}
\articledoi{10.1590/1983-3652.2024.51641}
%\articleid{NNNN} % if the article ID is not the last 5 numbers of its DOI, provide it using \articleid{} commmand 
% list of available sesscions in the journal: articles, dossier, reports, essays, reviews, interviews, editorial
\articlesessionname{articles}
\runningauthor{Rovira Collado, Martínez Carratalá y Miras}
%\editorname{Leonardo Araújo} % old template
\sectioneditorname{Hugo Heredia Ponce}
\layouteditorname{João Mesquita}

\title{Booktok: análisis de las estrategias discursivas multimodales para la promoción de la lectura en TikTok}
\othertitle{Booktok: análise das estratégias discursivas multimodais para a promoção da leitura no TikTok}
\othertitle{Booktok: analysis of multimodal discursive strategies for the promotion of reading in TikTok}

\author[1]{José Rovira Collado~\orcid{0000-0002-3491-8747}\thanks{Email: \href{mailto:jrovira.collado@ua.es}{jrovira.collado@ua.es}}}
\author[1]{Francisco Antonio Martínez Carratalá~\orcid{0000-0002-0587-5063}\thanks{Email: \href{mailto:famc@gcloud.ua.es}{famc@gcloud.ua.es}}}
\author[1]{Sebastián Miras~\orcid{0000-0002-4259-3890}\thanks{Email: \href{mailto:sebastian.miras@ua.es}{sebastian.miras@ua.es}}}
\affil[1]{Universidad de Alicante, Facultad de Educación, Alicante, España.}


\addbibresource{article.bib}
\usepackage{seqsplit}
\begin{document}
\maketitle
\begin{polyabstract}
\begin{abstract}
Los booktoks son vídeos breves que exponen contenidos sobre libros y literatura en la plataforma TikTok, como evolución de otras prácticas literarias en línea como los blogs o booktubers. Estos contenidos, con un formato único adaptado al móvil, han incrementado progresivamente desde 2020, convirtiéndose en una dinámica que ha alcanzado cifras millonarias, especialmente entre los lectores más jóvenes. El objetivo de esta investigación es identificar y diferenciar las estrategias multimodales que los booktoks utilizan para la promoción de la lectura. A partir de un análisis de contenido de una muestra previa de 500 vídeos, se identifican 10 dinámicas distintas de booktoks, algunas novedosas frente a modelos precedentes. Posteriormente, se realiza un análisis multimodal de diez booktoks concretos para mostrar las nuevas formas de hablar de libros a través de breves vídeos. Se describen los tipos de interacción entre el vídeo, el sonido y los contenidos que proponen los booktoks. Se identifican nuevas dinámicas, junto a otras heredadas de los booktuber, pero se confirma que el móvil, como plataforma de difusión de contenidos, impone una enorme velocidad en el consumo de vídeos sobre lectura.

\keywords{Educación \sep Internet \sep Lectura \sep Redes Sociales \sep Vídeos Interactivos}
\end{abstract}

\begin{abstract}
Os \textit{booktoks} são vídeos breves que apresentam conteúdos sobre livros e literatura na plataforma TikTok, como uma evolução de outras práticas literárias \textit{online}, como \textit{blogs} ou \textit{booktubers}. Esses conteúdos, com um formato único adaptado para dispositivos móveis, têm aumentado progressivamente desde 2020, tornando-se uma dinâmica que atingiu números milionários, especialmente entre os leitores mais jovens. O objetivo desta pesquisa é identificar e diferenciar as estratégias multimodais que os \textit{booktoks} utilizam para a promoção da leitura. A partir de uma análise de conteúdo de uma amostra anterior de 500 vídeos, foram identificadas dez dinâmicas diferentes de \textit{booktoks}, algumas inovadoras em relação a modelos anteriores. Posteriormente, foi realizada uma análise multimodal de dez \textit{booktoks} específicos para mostrar as novas formas de falar sobre livros por meio de vídeos breves. São descritos os tipos de interação entre o vídeo, o som e os conteúdos propostos pelos \textit{booktoks}. São identificadas novas dinâmicas, juntamente com outras herdadas dos \textit{booktubers}, mas confirma-se que o celular, como plataforma de difusão de conteúdos, impõe uma enorme velocidade no consumo de vídeos sobre leitura.

\keywords{Educação \sep Internet \sep Leitura \sep Redes Sociais \sep Vídeos Interativos}
\end{abstract}

\begin{abstract}
 Booktoks are short videos that showcase content about books and literature on the TikTok platform, evolving from other online literary practices such as blogs or booktubers. These contents, with a unique mobile-adapted format, have progressively increased since 2020, becoming a phenomenon that has reached millions, especially among younger readers. The aim of this research is to identify and differentiate the multimodal strategies that booktoks use for promoting reading. Through a content analysis of a previous sample of 500 videos, ten different booktok dynamics are identified, some of which are innovative compared to previous models. Subsequently, a multimodal analysis of ten specific booktoks is conducted to demonstrate the new ways of discussing books through short videos. The types of interaction between video, sound, and content proposed by booktoks are described. New dynamics are identified, along with others inherited from booktubers, but it is confirmed that the mobile platform imposes a significant speed in the consumption of reading-related videos.
 
\keywords{Education \sep Internet \sep Interactive Video \sep Reading \sep Social Networks}
\end{abstract}

\end{polyabstract}



\section{Introducción}\label{sec-introducción}

En la evolución de los espacios de promoción y mediación lectora en
Internet se encuentran los epitextos virtuales de lectura \cite{lluch2015} que aportan información sobre los
libros y la literatura, animando a leer uno u otro texto literario. El
último espacio de desarrollo de este ámbito son los \emph{booktoks},
breves vídeos sobre lectura y literatura en TikTok, una plataforma o red
social de vídeos para ser disfrutados y creados principalmente a través
del móvil. Estos epitextos son una evolución de modelos anteriores, como
los basados en el texto escrito, como los blogs o los foros literarios o
los basados en vídeos, como los \emph{booktubers} \cite{sorensen2014}
o fotografías y vídeos, como los \emph{bookstagramers} \cite{sánchezgarcía2020}.


Todos estos espacios permiten la \emph{lectura social} \cite{cordon2013}, donde cualquier persona puede compartir a través
de Internet y las redes sociales su opinión sobre sus lecturas e
interactuar con otras personas para hablar de literatura. El fenómeno
global de los \emph{booktubers} como espacio antecedente directo para la
animación lectora, y sus características en lengua española, ha
despertado gran interés en los últimos años \cite{vizcaino2019,tomasena2021}, incluso con
distintos análisis de sus posibilidades educativas \cite{roviracollado2016}. Ya no se escribe sobre libros, si no que se habla, se comenta o
se discute oralmente en vídeos de distinta extensión.

Una de las diferencias principales de estos vídeos con los
\emph{booktubers} es el móvil como espacio natural para su desarrollo,
pudiendo relacionar su uso con el \emph{Mobile} \emph{Learning} \cite{unesco2013}. Las redes sociales, desde su aparición, son espacio para el
desarrollo del aprendizaje informal o invisible, ya que están
constantemente ofreciendo información que se puede convertir en
conocimiento \cite{piscitelli2010}. Después de la
transformación digital provocada por la pandemia del covid-19 se deben
incorporar muchos de estos contenidos y estrategias de aprendizaje
informal a la hora de hablar de aprendizaje digital. También se deben
entender las estrategias discursivas como dinámicas de comunicación para
realizar el mensaje más comprensible y atractivo para los oyentes \cite{coll2021}. En estos espacios, estas estrategias deben aprovechar
todas las posibilidades de la multimodalidad (Martín Menéndez, 2020).
Posteriormente, se plantearán los elementos del análisis multimodal en
el presente estudio.

Instagram también ofrece espacios para los libros, y los
\emph{bookstragramer} publican tanto fotografías como vídeos
(\emph{reels} o \emph{stories}) para hablar de lecturas. Pero son los
\emph{booktoks} los que están adquiriendo mayor atención en este
momento. Actualmente, se encuentran distintas noticias e informes que
destacan el creciente uso que están adquiriendo estos vídeos en el mundo
de la literatura y la producción editorial como elementos de publicidad
que influyen en la elección de lecturas de sus espectadores \cite{nielsen2021}. A principios de esta década, la \textcite{penguin2020}
recomendaba los perfiles de \emph{booktokers} que había que seguir.
Solamente falta acercarse a los principales informes de usos de redes
sociales \cite{IABSpain2022} para comprobar que ambas redes tienen el mayor
crecimiento y audiencia en estos momentos.

Estos espacios generan nuevas figuras como los \emph{influencers}, las
personas con mucho éxito en redes sociales que se convierten en
celebridades \cite{abidin2020}. También ha aparecido el concepto de
``creadores de contenido'', para referirse a las personas que publican
vídeos elaborados, de cualquier temática, de forma periódica, ya que la
grabación y edición de estos vídeos supone en muchos casos una
especialización. Su visibilidad \cite{cordon2019} y
número de visionados permite que sus perfiles reciban una retribución
económica atendiendo al número de reacciones y visitas que tienen sus
vídeos por parte de la plataforma. En la sección de análisis de este
estudio se incluyen algunos de estos vídeos. En el ámbito de la
promoción literaria con esta plataforma se centran distintas
investigaciones sobre este fenómeno, principalmente, en el ámbito
anglosajón \cite{merga2021,martens2022} que detectan
el interés por esta nueva vía de comunicación literaria. \textcite{sanztejeda} han realizado una completa revisión sistemática sobre los
métodos y resultados de investigación sobre lectura en ambas plataformas
donde señalan las diferencias y novedades en su uso, y también los
elementos comunes con anteriores dinámicas.

Aunque TikTok e Instagram puedan parecer herramientas solamente para el
ocio, estas ofrecen múltiples aplicaciones educativas para promover
tanto aprendizajes formales como informales. Como herramientas
audiovisuales, estas ofrecen mensajes multimodales \cite{newlondongroup1996} que se analizarán detenidamente en el cuerpo de la presente
investigación. Muchos vídeos se replican en ambas plataformas, pero la
investigación se concreta exclusivamente en TikTok.

El objetivo principal de este estudio se centra en describir los
principales usos de \emph{booktok} en lengua española a través de un
análisis de contenido del discurso multimodal \cite{gomez2016}
empleado por las personas creadoras de los diferentes contenidos para
hacer los vídeos más atractivos.

A partir de este objetivo se plantean dos secundarios que se desprenden
del análisis.

El primer objetivo secundario es plantear la evolución de los
\emph{booktoks} desde modelos anteriores.

El último objetivo secundario es identificar algunas de las diferencias
de los \emph{booktoks} con los modelos anteriores.
\section{Metodologia}\label{sec-metodologia}

Para avaliar o desempenho dos modelos no processamento de ambiguidades quanto aos parâmetros da ambiguidade lexical, sintática e semântica, conduzimos tarefas utilizando um conjunto de dados que foi elaborado e analisado pelos autores do presente artigo, um grupo de seis estudantes de Letras e Linguística, cada um com um conhecimento maior em áreas distintas da linguística, como fonética, tradução, sintaxe e análise do discurso. Esses estudantes serão referidos no trabalho como juízes-humanos por se demonstrarem aptos a produzir e julgar de forma adequada os dados, além disso, vale salientar que os dados e resultados das tarefas foram avaliados, posteriormente, por uma coautora que é especialista na área de variação linguística.

Para garantir a consistência e a confiabilidade das frases geradas pelo grupo, foi adotado o procedimento de \textit{inter-annotator agreement}. As sentenças que não receberam consenso absoluto entre os juízes-humanos foram revisadas ou excluídas do corpus. Como resultado, não foi necessário calcular o coeficiente de concordância {\footnote{Ao criar o corpus, as frases foram revisadas por todos os autores do artigo, e apenas aquelas com 100\% de aprovação no procedimento de inter-annotator agreement foram disponibilizadas para os testes, com o objetivo de minimizar a presença de frases problemáticas. As frases geradas pelos modelos também passaram por uma revisão conjunta e foram discutidas entre os autores para garantir consenso absoluto a partir do mesmo procedimento. Por exemplo, na Tabela 4, que corresponde à geração de frases por parte do modelo, a sentença \enquote{O homem viu o acidente com os próprios olhos} contém um componente favorável à criação de ambiguidade sintática, o adjunto \enquote{com os próprios olhos}, no entanto, com base na teoria de Cançado, essa ambiguidade é válida quando acompanhada por uma ambiguidade semântica, que permite o duplo sentido na compreensão pragmática do contexto dentro do enunciado. Já na Tabela 3, a frase \enquote{Pedi o prato principal ao garçom, era filé!} é parte do corpus das sentenças distratoras, pois a estrutura do enunciado fornece informações suficientes para evitar outras interpretações, ou seja, entendemos que \enquote{prato} não se refere ao utensílio doméstico e \enquote{filé} não é um elogio metafórico. Sendo assim, essa frase foi considerada não ambígua pelos seis juízes-humanos e pela especialista em linguística que revisou o trabalho.}}, uma vez que todas as sentenças, tanto as ambíguas quanto as não ambíguas, só foram incluídas após alcançarem 100\% de aprovação. Esse critério rigoroso assegurou que o conjunto de dados utilizado nas tarefas fosse  confiável ao máximo para minimizar divergências.

O experimento foi composto por um grupo de 120 sentenças, distribuídas de forma balanceada entre os três tipos de ambiguidade. Dessas, 60 apresentam algum tipo de ambiguidade, seja semântica, lexical ou sintática. As frases ambíguas foram elaboradas com o objetivo de provocar especificamente um dos três tipos de ambiguidade (semântica, lexical ou sintática). No entanto, é possível que algumas sentenças apresentem mais de um tipo de ambiguidade, que não foi avaliado durante a criação das mesmas. No decorrer de nossa análise, buscamos isolar ao máximo cada frase, de modo que apenas um elemento causador de ambiguidade estivesse presente. Essa abordagem nos permite um controle mais rigoroso das variáveis observadas nos experimentos, uma vez que a ambiguidade é um fenômeno complexo e multifacetado.

É importante ressaltar que, dentro de uma única sentença, podem existir múltiplos fatores que contribuem para a ambiguidade, o que torna a sua identificação e análise ainda mais desafiadoras, então, para um estudo inicial, de caráter pioneiro com LLM's formamos dados linguísticos mais artificiais, mas planejamos trabalhar futuramente com dados provenientes de corpora ou textos reais, incorporando materiais autênticos que nos permitam uma compreensão mais abrangente das ambiguidades linguísticas em contextos variados.

Foram criadas 20 sentenças com ambiguidade lexical — que abrangem casos de homonímia e polissemia, sem distinção de categoria na análise das frases —, 20 sentenças com ambiguidade semântica, nas quais o referente dos pronomes não está claro, e, por fim, 20 sentenças com ambiguidade sintática, envolvendo adjuntos adnominais ou adverbiais ambíguos que provocam duplo sentido devido às diferentes organizações estruturais que a frase pode ter (Tabela \ref{tab:amostras_sentencas}).

As outras 60 sentenças distratoras tiveram sua ambiguidade barrada ao máximo, não sendo verificadas pelos juízes-humanos durante a elaboração dos dados linguísticos. As sentenças distratoras foram criadas com a intenção de evitar qualquer forma de ambiguidade. Da mesma forma que foi feito o julgamento das frases ambíguas, as frases não ambíguas foram avaliadas em um nível de significado mais isolado, sem considerar diversos contextos enunciativos figurativos, na maioria das vezes, nas quais ela poderia fazer sentido, logo assumimos um posicionamento de interpretação mais pragmático linguístico, que corrobora com Cançado. Essas mesmas considerações são válidas para as avaliações feitas em relação às frases que foram geradas pelos modelos de linguagem na Tarefa 4.

Um exemplo para ilustrar essa forma de análise é a frase \enquote{Pedi o prato principal ao garçom, era filé!} (Tabela 3) em que em um contexto muito específico, poderia significar que o cliente pediu o utensílio principal de servir comida e este utensílio era \enquote{filé}, um termo popular para se referir a algo bom, porém, a princípio, a frase foi escrita tendo em vista o significado mais óbvio, que se determinou a partir das pistas interpretativas deixadas dentro da frase na qual um sentido lexical confirmava o outro sem a necessidade de buscar condições exofóricas à sentença que justificassem uma polissemia.





\begin{table}[htpb]
\centering
\begin{threeparttable}
\caption{}
\label{tab:amostras_sentencas}
\begin{tabular}{llp{6cm}}
\toprule
Id & Sentença & Classe \\
\midrule
1 & A rede caiu. & Ambiguidade lexical \\
2 & Ela não gosta da amiga dela. & Ambiguidade semântica \\
3 & O menino viu o incêndio do prédio. & Ambiguidade sintática \\
\bottomrule
\end{tabular}
\source{\url{https:XXXXXXXXX}.}
%\notes{Se necessário, poderá ser adicionada uma nota ao final da tabela.}
\end{threeparttable}
\end{table}

%\source{\url{https://osf.io/u7wre/?view_only=572c74eb4c634d47a02ad25485ea8caa}.}

Para responder as nossas perguntas de pesquisa, foram conduzidas quatro tarefas {\footnote{As respostas dadas pelos modelos de linguagem nas tarefas foram coletadas entre julho de 2023 até janeiro de 2024. Coletas posteriores podem levar a resultados diferentes devido às atualizações dos modelos de linguagem. Para evitar ao máximo o enviesamento dos modelos, as perguntas foram formuladas da forma mais objetiva possível, evitando dar pistas sobre a resposta correta ou sinalizando qual era o resultado esperado. A preocupação em minimizar testes com viés tinha o objetivo de garantir uma avaliação mais precisa do desempenho dos modelos de linguagem, afastar casos de generalização nas respostas, favorecer a imparcialidade dos resultados e alcançar um conjunto de respostas o mais transparente possível.}} distintas com as sentenças criadas. Em todas, foram realizadas coletas duplicadas das interações para cada frase, reiniciando o \textit{console} entre cada coleta para evitar qualquer influência do contexto que pudesse gerar respostas tendenciosas. Essa abordagem permitiu avaliar a consistência dos modelos nas respostas fornecidas.


A tarefa 1 visava identificar se os modelos conseguem detectar a presença de ambiguidade em cada sentença por meio da seguinte instrução: \textbf{A sentença ``[sentença]'' é ambígua? Responda, sim, não ou não sei}. Foram apresentadas individualmente todas as sentenças e registradas as respostas dos modelos, comparando-as com a nossa classificação prévia. As respostas foram cuidadosamente avaliadas quanto à correção e abrangência das explicações fornecidas por seis juízes-humanos que as julgaram independentemente. A partir dos resultados, foi gerada uma matriz de confusão para computar a quantidade de verdadeiros positivos (sentenças que são ambíguas e que os modelos classificaram como ambíguas), falsos positivos (sentenças que não o são e que os modelos classificaram como ambíguas), verdadeiros negativos (sentenças que não são ambíguas e que os modelos assim classificaram como não ambíguas), e falsos negativos (sentenças que são ambíguas e que os modelos classificaram como não ambíguas).


Na tarefa 2, foi realizado um teste para avaliar a capacidade dos modelos em distinguir corretamente entre as três classes de ambiguidade estudadas neste trabalho, formulando a seguinte pergunta para cada modelo: \textbf{``Qual o tipo de ambiguidade?''}. A tarefa consistiu em perguntar qual o tipo de ambiguidade da sentença que foi classificada anteriormente como ambígua ou não ambígua. Na tarefa 3, foi verificada a capacidade dos modelos em desambiguar as sentenças que foram fornecidas a eles. Com esse propósito, foram apresentadas frases que incluem tanto sentenças ambíguas, quanto sentenças não ambíguas, e solicitado aos modelos a seguinte instrução: \textbf{Faça a desambiguação da frase: ``[sentença]''}. A tarefa busca testar a habilidade dos modelos em compreender e interpretar o contexto, escolhendo a interpretação mais apropriada quando a ambiguidade está presente. 

Na tarefa 4, foi avaliada a capacidade dos modelos em gerar frases ambíguas na categoria solicitada. Para isso, pedimos para cada modelo gerar frases da seguinte forma: \textbf{Gere 20 frases com ambiguidade ``[categoria]''}. Em seguida, as respostas obtidas foram avaliadas por juízes-humanos, buscando compreender quão preciso é o ChatGPT o Gemini ao criarem frases que apresentam múltiplas interpretações contextuais.

Para mensurar quantitativamente o desempenho dos modelos, foi utilizada a métrica de acurácia, a qual já é amplamente empregada na área de aprendizado de máquina  \cite{naser2021error, freitag2021funccao}. A acurácia, no contexto da classificação, representa a proporção de frases corretamente classificadas pelos modelos em relação ao total de frases apresentadas na tarefa, como apresentado na equação \ref{eq_1}.

\begin{equation}
    \text{acc} = \frac{\text{Número de previsões corretas}}{\text{Total de previsões}}
    \label{eq_1}
\end{equation}

Todas as sentenças criadas por nós e geradas pelos modelos durante as tarefas estão disponíveis no Apêndice \ref{sec-apendice}. As respostas dos modelos durante as tarefas estão disponíveis para download\footnote{\url{https://docs.google.com/spreadsheets/d/1AOff1GJmh3oWIuKdfBGGHWeox-7HtuHQW0LkaT5yFVI/edit?usp=sharing}} em nosso repositório.


%\footnote{\url{https://osf.io/u7wre/?view_only=572c74eb4c634d47a02ad25485ea8caa}}
\section{Resultados}\label{sec-experimentos-resultados}

Nesta seção, são apresentados os resultados obtidos. Cada seção responde uma das nossas perguntas de pesquisa através de quatro tarefas. Todas as análises dos resultados descritos foram realizadas por seis juízes-humanos do curso de Letras ou Linguística com conhecimento na área.

\subsection{Qual é a precisão dos modelos na detecção de ambiguidade linguística em frases do Português Brasileiro?}\label{resultados-q-1}

Para avaliar a precisão dos modelos na detecção de ambiguidade seguimos a proposta de \cite{freitag2021funccao}, foram comparadas as acurácias e as matrizes de confusão obtidas, usando os dados da tarefa 1. Esta análise concentrou-se exclusivamente na detecção da presença ou ausência de ambiguidade, sem levar em consideração o tipo específico identificado pelos modelos posteriormente. Assim, os dados foram divididos em dois grupos distintos, totalizando 60 frases com ambiguidade e 60 frases distratoras. Foram realizadas duas coletas por frase, sendo obtidas 240 predições para cada modelo.


\begin{figure}[htb]
    \centering
    \begin{subfigure}[b]{0.45\textwidth}
        \includegraphics[width=\textwidth]{matriz_confusao_ChatGPT.pdf}
        \caption{Matriz de confusão do ChatGPT.}
        \label{fig:matriz_confusao_chatgpt}
    \end{subfigure}
    \hfill
    \begin{subfigure}[b]{0.45\textwidth}
        \includegraphics[width=\textwidth]{matriz_confusao_Bard.pdf}
        \caption{Matriz de confusão do Gemini.}
        \label{fig:matriz_confusao_bard}
    \end{subfigure}
    \caption{Matrizes de Confusão dos modelos ChatGPT e Gemini.}
    \label{fig:matriz_confusao_1}
\end{figure}

Os resultados da matriz de confusão na Figura \ref{fig:matriz_confusao_1} revelam que o ChatGPT registrou uma acurácia de apenas 28,75\%, enquanto o Gemini alcançou 49,58\%, indicando que estas versões dos modelos não conseguem detectar ambiguidade com precisão confiável. Os resultados revelam que ambos os modelos exibem uma quantidade significativa de falsos positivos, identificando ambiguidade em frases que carecem dela. Enquanto o ChatGPT demonstra erros distribuídos em todos os quadrantes da matriz, o Gemini tende a rotular quase todas as frases como ambíguas, resultando em uma taxa maior de falsos positivos. Para computar as matrizes de confusão e a acurácia foram consideradas apenas as respostas em que os modelos responderam \enquote{Sim} e \enquote{Não}, de modo que todas as respostas \enquote{Não Sei} foram descartadas. Assim, o ChatGPT destacou-se por apresentar mais dúvidas, conseguindo responder 196 perguntas enquanto o Gemini respondeu 236.


A diferença de acurácia entre os dois modelos pode ser atribuída ao fato do ChatGPT expressar dúvidas ao detectar frases ambíguas, declarando não saber ou negando a presença de ambiguidade. Analisando os três tipos de ambiguidade, observa-se que o ChatGPT lida melhor com ambiguidades semânticas e sintáticas, cometendo mais erros quando a ambiguidade é apenas lexical. Uma explicação é devida à estrutura da ambiguidade sintática ser descrita em estudos de processamento linguístico \cite{maiadimensoes} (1) e de processamento de linguagem natural \cite{padovani2022metodo}(2). Em contrapartida, o Gemini apresentou apenas um caso de falso negativo, acertando todos os outros testes em frases ambíguas. Entretanto, o Gemini tem a tendência de não distinguir entre frases ambíguas e não ambíguas, pois, em todos os testes, indica a presença de ambiguidade.


%\textbf{Sintática}.Dentre os 40 experimentos com ambiguidade feitos com o ChatGPT e Bard, a primeira inteligência constatou ambiguidade em 22 frases, mas por outro lado, o Gemini identificou integralmente as sentenças ambíguas ainda que nem todas estivessem com a classificação correta. No entanto, no experimento feito com as frases sem ambiguidade, o ChatGPT reconheceu alguma ambiguidade em 8 casos, alcançando uma acurácia final de 37,5\%, enquanto que o Gemini identificou ambiguidade em todos os testes, resultando, nessa etapa, 100\% de falsos positivos e chegando a uma acurácia de 50\%.  



\subsection{Qual dos modelos percebe melhor os fenômenos de homonímia e polissemia?}\label{resultados-q-4}


Neste estudo, foram examinadas as explicações fornecidas pelos modelos na tarefa 2, analisando apenas as respostas do conjunto de frases com ambiguidade lexical. Durante a análise, foi considerado que os modelos perceberam a homonímia e a polissemia através da explicação dada pelos modelos para justificar o tipo de ambiguidade identificada. Se a explicação dada pelos modelos foi referente a ambiguidade gerada devido aos diferentes significados que o item lexical pode assumir na frase, e se realmente os juízes-humanos enxergariam os diferentes significados do item lexical da mesma maneira, foi considerado acerto, caso contrário foi tido como um erro por parte dos modelos. 

Para demonstrar temos as frases \textbf{Isso não é legal!} e \textbf{A carteira foi danificada.} que foram classificadas como ambiguidade lexical de homonímia pelos juízes-humanos. Tais frases foram testadas no ChatGPT de modo que a primeira frase foi considerada correta, pois recebeu, em um dos testes, a seguinte resposta \textbf{A frase \enquote{Isso não é legal!} pode apresentar ambiguidade de sentido, pois a palavra \enquote{legal} possui múltiplas interpretações, dependendo do contexto em que é usada.} e a segunda frase foi classificada como incorreta, em um dos testes, por apresentar uma explicação incoerente \textbf{A frase \enquote{a carteira foi danificada} pode ser considerada ambígua devido à ambiguidade estrutural. Isso ocorre porque não está claro se a carteira sofreu dano físico ou se está se referindo a uma carteira de identidade ou pertencente a alguém. Portanto, a ambiguidade está relacionada à interpretação da frase em termos de sua estrutura sintática.}


Ambos os modelos demonstraram uma boa compreensão dos fenômenos de homonímia e polissemia. O ChatGPT obteve sucesso em 75\% das vinte frases testadas, enquanto o Gemini alcançou uma taxa de acerto de 80\%. Vale ressaltar a notável similaridade na forma como os modelos interpretaram esses fenômenos, uma vez que ambos obtiveram sucesso quase no mesmo subconjunto de frases.






\subsection{Os modelos conseguem desambiguar adequadamente as sentenças?}\label{resultados-q-3}

Nesta seção, a análise foi baseada nas respostas obtidas na tarefa 3, na qual foi solicitada a desambiguação de todas as sentenças. As respostas obtidas foram divididas em três categorias: 1) Correta ocorreu quando o modelo explica corretamente duas ou mais interpretações possíveis para a sentença, de forma semelhante à interpretação que um ser humano teria. No caso das sentenças sem ambiguidade, o modelo esclarece a ausência de ambiguidade e observa que não é necessário executar a tarefa solicitada; 2) Incompleta ocorreu quando o modelo apresentou pelo menos uma interpretação correta para a sentença, errando nas demais interpretações possíveis; e 3) Incorreta ocorreu quando o modelo apresentou opções de desambiguação sem nenhum sentido. 

Os resultados mostrados na Figura \ref{fig:matriz-confusao-desamb} revelam que ambos os modelos enfrentam dificuldades ao tentar desambiguar corretamente as sentenças, uma vez que propõem interpretações equivocadas, identificando ambiguidade em elementos pouco ou nada relacionados à ambiguidade real. Por exemplo, no conjunto de frases sem ambiguidade, ambos os modelos tentaram desambiguá-las oferecendo opções de interpretação. 

A frase \enquote{\emph{Como o pássaro estava cansado, Ricardo o alimentou.}} obteve uma desambiguação incorreta tanto na tarefa do ChatGPT como do Gemini. No primeiro modelo, a resposta foi a seguinte: \enquote{\emph{Para desambiguar a frase  e torná-la mais clara, podemos reescrevê-la de forma a eliminar a ambiguidade de referência. Uma maneira de fazer isso é substituindo o pronome \enquote{o} por seu referente explícito. Vejamos: 1. Como Ricardo estava cansado, o pássaro foi alimentado por ele. Nesta versão reescrita, fica claro que o pronome \enquote{o} se refere ao pássaro, e não a Ricardo. Agora a frase não possui mais ambiguidade de referência.}} 

O segundo modelo apresentou a seguinte explicação: \enquote{\emph{Claro, aqui estão duas possíveis desambiguações da frase \enquote{Como o pássaro estava cansado, Ricardo o alimentou}: Ricardo alimentou o pássaro porque estava cansado. Ricardo alimentou o pássaro porque estava cansado de ver o pássaro cansado. A primeira interpretação é a mais provável, pois é mais comum alimentar alguém quando está cansado. No entanto, a segunda interpretação também é possível, pois é possível que Ricardo estivesse preocupado com o bem-estar do pássaro e quisesse ajudá-lo a se recuperar. Sem mais informações, não é possível dizer com certeza qual é a interpretação correta.}}

Os modelos desambiguaram sem necessidade mais de 90\% dos casos de teste do conjunto de frases sem ambiguidade. É relevante destacar que, mesmo o ChatGPT - que afirmou em perguntas anteriores que algumas frases não tinham ambiguidade - ofereceu opções de desambiguação, sendo incongruente com suas respostas anteriores. Uma exceção notável ocorreu em uma resposta relacionada à frase \enquote{Gostamos da cor rosa}, na qual o ChatGPT afirmou: \enquote{\emph{Peço desculpas, mas a frase 'Gostamos da cor rosa' não é intrinsecamente ambígua e, portanto, não requer desambiguação. Ela expressa claramente uma preferência pela cor rosa. Se houver um contexto específico causando ambiguidade, por favor, forneça mais informações para que eu possa ajudar a esclarecer a situação}}. Observa-se que os modelos têm a tendência de realizar superinterpretações das sentenças sem ambiguidade, oferecendo soluções sem lógica apenas para cumprir a tarefa solicitada, demonstrando ainda não haver o conhecimento explícito de regras do funcionamento da língua.

%\begin{table}[h]
%\caption{Estatísticas obtidas na tarefa de desambiguação.}
%\label{tab:estat_desambiguacao}
%\begin{tabular}{lcccccc}
%\hline
%Grupo                 & \multicolumn{2}{c}{Corretas} & \multicolumn{2}{c}%{Incompletas} & \multicolumn{2}{c}{Incorretas} \\ \hline
%                      & ChatGPT        & BARD        & ChatGPT          & BARD         & ChatGPT         & BARD         \\
%Sem ambiguidade       & 12               & 3            & 0                 & 6             & 108                & 108             \\
%Ambiguidade Sintática & 15             & 20              & 19               & 13             & 6               & 7              \\
%Ambiguidade Semântica & 31               & 23              & 3                 & 3             & 6                & 14             \\
%Ambiguidade Lexical   & 28               & 29             & 7                 & 5           & 5                 & 6              \\ \hline
%\end{tabular}
%\source{Própria.}
%\notes{Se necessário, poderá ser adicionada uma nota ao final da tabela.}
%\end{table}


\begin{figure}[htb]
    \centering
    \begin{subfigure}[b]{0.45\textwidth}
        \includegraphics[width=\textwidth]{matriz_confusao_desamb_ChatGPT.pdf}
        \caption{Resultados quantitativos da tarefa de desambiguação realizada pelo ChatGPT.}
        \label{fig:matriz_confusao_chatgpt}
    \end{subfigure}
    \hfill
    \begin{subfigure}[b]{0.45\textwidth}
        \includegraphics[width=\textwidth]{matriz_confusao_desamb_Bard.pdf}
        \caption{Resultados quantitativos da tarefa de desambiguação realizada pelo Gemini.}
        \label{fig:matriz_confusao_bard}
    \end{subfigure}
    \caption{Estatísticas obtidas na tarefa de desambiguação. Os resultados mostram que ambos os modelos apresentam muita dificuldade para identificar que as sentenças distratoras não tem ambiguidade. Em 108 casos de teste, ambos os modelos forneceram explicações que não refletiam a interpretação humana, justificando a presença de ambiguidade em sentenças que, na verdade, não a possuíam.}
    \label{fig:matriz-confusao-desamb}
\end{figure}



Na desambiguação das frases com ambiguidade sintática, os modelos também apresentaram erros na maioria dos testes. Em 62,5\% dos testes, as análises apresentadas estavam incorretas ou incompletas e os critérios usados nas frases classificadas com ambiguidade sintática, também, pendiam para uma abordagem semântica, lexical ou voltada para a necessidade de um contexto favorável para desambiguar as sentenças. No caso da sentença \enquote{Pedro me mandou um cartão postal de Veneza}, o ChatGPT apresentou a seguinte desambiguação para a frase: \enquote{Pedro me mandou um cartão postal de Veneza}; \enquote{Pedro mandou um cartão postal de Veneza para a minha amiga Maria, e ela me mostrou o cartão}. Contudo, a ambiguidade no adjunto \enquote{de Veneza} permanece. Outra solução proposta é mudar a alocação do pronome: \enquote{Pedro mandou um cartão postal de Veneza para mim}, mantendo a frase ambígua. Nesse caso, o Gemini ofereceu melhores opções, como \enquote{Pedro está atualmente em Veneza e me enviou um cartão postal que foi produzido lá.} ou \enquote{Pedro não está atualmente em Veneza, mas me enviou um cartão postal que mostra uma imagem da cidade.},  demonstrando aplicar os princípios da aposição mínima e da aposição local \cite{maiadimensoes} para resolver a ambiguidade. No entanto, o modelo, na maioria das vezes, não conseguiu identificar com precisão qual era o elemento causador da ambiguidade.

Por outro lado, no grupo de frases com ambiguidade semântica, na maior parte dos testes, ambos os modelos de linguagem sugeriram frases adequadas para a desambiguação das sentenças, sempre destacando a importância do contexto para a correta interpretação. Um exemplo pode ser visto com a frase \enquote{Ana me contou um segredo sobre ela}, em que o ChatGPT propôs a desambiguação esperada, compreendendo que \enquote{ela} pode se referir tanto à Ana, quanto a uma outra pessoa. Em relação às respostas do Gemini, a maioria também foi apropriada porém com precisão menor que a ChatGPT. Por exemplo, em \enquote{Paulo não entrou na universidade de novo}, o Gemini indicou corretamente que as possíveis reescritas da sentença seriam \enquote{Paulo não foi aceito na universidade de novo} e \enquote{Paulo não entrou na universidade de novo depois de ter se formado}. Já em \enquote{A carteira foi danificada}, o Gemini sugeriu opções que especificassem informações não necessariamente ambíguas, visto que ele interpretou que a ambiguidade estaria em \enquote{danificado}, assim, deveria ser especificado se \enquote{A carteira foi danificada fisicamente} ou \enquote{A carteira foi danificada financeiramente}. Estas interpretações são bastante distantes do que se encontra em estudos psicolinguísticos sobre processamento \cite{machado1996sintaxe, brito2013processamento}.

Os modelos tiveram, simultaneamente, melhor desempenho nos casos de ambiguidade lexical (Figura \ref{fig:matriz-confusao-desamb}), entretanto, o ChatGPT não conseguiu desambiguar corretamente algumas frases que envolvem polissemia. Na frase \enquote{Pedi um prato ao garçom.}, o modelo apresentou duas opções de desambiguação com a seguinte resposta à tarefa 3: \enquote{Claro, aqui estão as duas possíveis interpretações desambiguadas da frase:  Eu pedi um prato ao garçom.  Eu pedi ao garçom um prato. Essas reestruturações eliminam a ambiguidade, deixando claro quem realizou a ação de pedir e o que foi pedido.}. Percebe-se, então, que o modelo tende a encontrar problemas na estrutura, ao trocar a posição dos elementos sintáticos e não nas palavras isoladamente. O Gemini apresentou mais facilidade de identificação e explicação nos casos de homonímias, haja vista que todas as suas explicações nestes casos estão corretas. Por outro lado, também predominaram justificativas equivocadas em alguns casos de polissemia.

\subsection{Quais padrões de ambiguidade os modelos ChatGPT e Bard demonstram conhecer na geração de frases ambíguas?}
\label{resultados-q-5}

A investigação sobre os padrões de ambiguidade usados pelos modelos ChatGPT e Gemini durante a geração de frases ambíguas é abordada por meio da análise das respostas obtidas na tarefa 4. Para conduzir essa análise, seis juízes-humanos qualificados julgaram se as frases geradas pelos modelos continham elementos que induzem ambiguidade perceptível por seres humanos e, em caso afirmativo, se esses elementos se alinhavam corretamente com a categoria de ambiguidade.


\textbf{Ambiguidade Lexical.} Na geração de frases com ambiguidade lexical, tanto o ChatGPT quanto o Gemini revelaram não conhecer os padrões geradores desse tipo de ambiguidade, resultando, em sua maioria, em frases sem qualquer forma de ambiguidade identificada pelos avaliadores humanos. O ChatGPT, em particular, não gerou nenhuma frase com ambiguidade lexical, tendo 18 frases avaliadas sem qualquer notificação de ambiguidade. Por exemplo, foram geradas frases como \enquote{O pássaro voou em direção à árvore mais alta.} e \enquote{Ela encontrou uma bela maçã na floresta.}, em que para um observador humano não há ambiguidade, revelando clareza na expressão. Em apenas duas das frases geradas, foram identificados padrões de adjunto ambíguo, referindo-se a ambiguidade estrutural ou sintática, não se enquadrando corretamente na categoria de ambiguidade lexical.

O Gemini também enfrentou dificuldades ao gerar frases com ambiguidade lexical. Das 20 frases, em 14 delas nenhum avaliador humano conseguiu identificar qualquer tipo de ambiguidade, resultando em uma taxa de sucesso de apenas 20\% para o Gemini e de 0\% para o ChatGPT. Um exemplo é a frase \enquote{O advogado defendeu o criminoso.} que não apresenta ambiguidade perceptível por um humano, pois com o contexto da frase é facilmente inferido que o advogado defendeu legalmente o criminoso. Foram observados apenas dois casos em que a ambiguidade residia na homonímia e polissemia, como na frase \enquote{O professor ensinou a classe.}. Além disso, o Gemini, em alguns casos, utilizou padrões que se alinham com a ambiguidade situacional, que é considerada parte da ambiguidade lexical por Cançado, por exemplo ao sugerir a frase \enquote{A casa está vazia} acompanhada da explicação que a casa pode estar vazia de pessoas ou de móveis. O desempenho nesta tarefa reforça o resultado de os modelos já possuem uma certa compreensão de ambiguidade, mas que ambos os modelos ainda enfrentam desafios na geração de frases em que o elemento gerador de ambiguidade é um item lexical.


\textbf{Ambiguidade Sintática.}
Na análise da geração de frases com ambiguidade sintática, o ChatGPT apresentou um desempenho relativamente superior, com uma taxa de sucesso de 65\%. Ou seja, das 20 frases geradas, 13 efetivamente incorporaram ambiguidade sintática por meio de complementos sintáticos, adjuntos adverbiais ou adnominais ambíguos, demonstrando que o modelo conseguiu aprender tais padrões de forma mais satisfatória. Por exemplo, na frase \enquote{Ela viu o homem com o telescópio,} o ChatGPT explorou a ambiguidade gerada pelo complemento sintático \enquote{com o telescópio,} permitindo interpretações tanto de ela ter utilizado um telescópio para ver o homem quanto de o homem estar com um telescópio quando foi visto. O modelo também utilizou ambiguidade oriunda de adjuntos adverbiais e adnominais, como em \enquote{A mãe deu um presente para a filha com uma fita bonita,} na qual o adjunto \enquote{com uma fita bonita} possibilita interpretações sobre o presente ter uma fita bonita ou a filha estar com uma fita bonita. Estes são os casos clássicos de ambiguidade sintática, amplamente descritos na literatura e explicados a partir dos princípios da aposição mínima e da aposição local \cite{maiadimensoes}, com descrições no português brasileiro \cite{maia2003processamento, maia2004compreensao, brito2013processamento}.

Apesar dos resultados satisfatórios obtidos na geração das frases, foram encontradas inconsistências na interpretação do elemento gerador da ambiguidade em 7 das 20 frases, mesmo quando a ambiguidade sintática estava de fato presente. Além disso, o modelo gerou 6 frases sem identificação de qualquer tipo de ambiguidade pelos juízes-humanos. Em 20\% das frases, o ChatGPT personificou elementos não-humanos, prejudicando a interpretação correta. Por exemplo, na frase \enquote{O jogador marcou um gol com a camisa amarela.}, o ChatGPT colocou o adjunto, gerando uma sentença com potencial de ser sintaticamente ambígua, no entanto, a compreensão de senso comum barra a possibilidade da frase ter ambiguidade para humanos, pois a interpretação mais intuitiva linguisticamente será que o jogador estava vestindo uma camisa amarela e descartará o sentido de que o gol vestia uma camisa amarela.

Embora a nossa comunicação seja potencialmente ambígua, a ambiguidade não parece ser um problema, pois quando há interferência, a ambiguidade é resolvida com o esclarecimento, reparo ou correção. A IA atua como o analista, como explicam \cite{freitag2021gramatica}, que dispõem de um grande conjunto de dados desprovidos de contexto, e que portanto são propensos a gerar ambiguidade na compreensão, mas que só existem do ponto de vista da IA.

O Gemini demonstrou um conhecimento ainda mais limitado dos padrões geradores de ambiguidade sintática, gerando apenas 4 das 20 frases com adjuntos adnominais e adverbiais como fontes de ambiguidade, resultando em uma taxa de acertos de apenas 20\%. Das 16 frases sem ambiguidade sintática, 11 não apresentam nenhum tipo de ambiguidade, como em \enquote{O homem comprou o livro que estava na prateleira,} em que erroneamente ela atribuiu ambiguidade à expressão \enquote{na prateleira.}, algo que um ser humano não faria, embora seja possível por conta dos princípios da aposição mínima e da aposição local \cite{brito2013processamento, maiadimensoes}. Além disso, em quatro sentenças, o Gemini confundiu o uso de palavras polissêmicas e homônimas, classificando erroneamente a ambiguidade lexical como sintática. Também é notável que, ao contrário do ChatGPT, o Gemini não incorporou a personificação de elementos inanimados em suas gerações.

Em 80\% das frases geradas, as explicações fornecidas pelo Gemini foram incoerentes demonstrando que o modelo ainda não consegue explicar corretamente a causa da ambiguidade, como na frase \enquote{O homem viu a mulher na janela,} onde atribuiu a ambiguidade ao verbo \enquote{ver,} ignorando que a verdadeira fonte era o adjunto \enquote{na janela,} permitindo que o homem ou a mulher estivessem na janela. Outro problema observado foi a dificuldade dos modelos em lidar com ambiguidades sintáticas envolvendo a concordância de gênero, como no caso de adjuntos que podem concordar tanto com sujeito como o objeto. Na frase \enquote{O rapaz encontrou a carteira perdida no chão do parque}, o adjetivo \enquote{perdida} só pode ocupar a função de adjunto adnominal por se referir à carteira, objeto da sentença, devido à flexão de gênero. O termo não pode ser adjunto adverbial do verbo \enquote{encontrar}, porque não concorda com o sujeito do verbo, que está no masculino. Essa ocorrência dificilmente ocorreria em inglês, por ser uma língua sem uso generalizado de gênero, logo essa ambiguidade seria mantida com o uso do termo \enquote{lost}. Isso mostra que os resultados poderiam ter sido melhores se as tarefas fossem realizadas em inglês, já que o modelo é majoritariamente treinado nessa língua e evidencia a necessidade de mais estudos e investimento sobre os dados linguísticos dos modelos de linguagem na língua portuguesa.


\textbf{Ambiguidade Semântica.}
É importante ressaltar que a ambiguidade semântica, em determinados referenciais teóricos, pode se assemelhar à ambiguidade lexical ou, em alguns casos, não é reconhecida como uma categoria distinta, resultando em uma linha tênue de separação entre ela e outros tipos de ambiguidade \cite{zavaglia2003ambiguidade}. Contudo, os resultados gerados pelo ChatGPT e pelo Gemini não se manifestam apenas na mistura de diferentes padrões de ambiguidade. Em vez disso, destaca-se na geração de frases que carecem de qualquer ambiguidade, as quais, foram consideradas como possuidoras de ambiguidade semântica. O ChatGPT, por exemplo, produziu 7 frases que não apresentaram ambiguidade para os juízes-humanos, enquanto o Gemini gerou 15 frases nessas condições. As demais sentenças, em sua maioria, foram geradas com base em padrões de adjunto ambíguo ou, residindo apenas em elementos lexicais.

Ambos os modelos geraram frases semelhantes ou idênticas às produzidas para ambiguidade sintática. O ChatGPT gerou a sentença \enquote{Ela viu o homem com o telescópio.}, e o Gemini, \enquote{A menina viu o homem com o binóculo}. Ambas continham adjunto ambíguo, e padrões similares foram reproduzidos quando solicitadas frases com ambiguidade sintática, evidenciando a falta de distinção clara entre os dois tipos de ambiguidade por parte dessas versões dos modelos.

\textbf{Caso particular.} Apesar do Manual de Semântica \cite{canccado2005manual} não tratar diretamente de substantivos compostos cristalizados pelo uso social e do nosso cotidiano, vale destacar duas sentenças geradas por ambos os modelos na Tarefa 4. O ChatGPT gerou a frase \enquote{Ele pegou a maçã verde do chão} e o Gemini produziu \enquote{O carro está parado no sinal vermelho.} Ambas as frases podem ser consideradas ambíguas devido aos termos compostos \enquote{maçã verde} e \enquote{sinal vermelho}. No primeiro caso, \enquote{maçã verde} pode se referir tanto à cor da fruta quanto a uma variedade específica de maçã, criando uma ambiguidade lexical. Já no segundo exemplo, \enquote{sinal vermelho} pode ser interpretado como a luz vermelha do semáforo, um alerta de perigo ou o poste do semáforo pintado de vermelho, resultando em diferentes interpretações.  

Esses dois exemplos evidenciam um padrão reconhecido pelos modelos na geração de sentenças classificadas como ambiguidade semântica, que classificamos como lexical no nosso julgamento, por estar em conformidade com o que a autora propõe em sua taxonomia em relação aos itens com múltiplas inferências de significado. Os componentes lexicais \enquote{maçã verde} e \enquote{sinal vermelho} provocam ambiguidade ao apresentarem mais de um sentido possível para o termo, mas essa distinção só é válida para fins de sistematização da análise dos dados. Reforçamos que não presumíamos que os modelos deveriam seguir a classificação adotada no estudos. Por conseguinte, não julgamos a nomenclatura como correta ou errada, mas somente a explicação que acompanhava as frases que foram geradas. Em vista dessas considerações, tal evidência pode servir como uma provocação para futuras pesquisas sobre a relação entre ambiguidade e modelos de linguagem, especialmente em como esses modelos lidam com a interpretação de termos cristalizados que carregam múltiplos sentidos lexicais.









\section{Discussion}\label{sec-discussion}

Our findings suggest that audio input does not necessarily fuse with the
subtitles to make for an overall improvement upon language retention as
we had initially hypothesized. It did seem reasonable to have considered
during the hypothesis forming step that the coinciding audio and visual
input of the language (i.e. Spanish audio with Spanish subtitles) would
bolster and reinforce comprehension of the scene. This is because the
audio would ideally complement the reading comprehension. However, the
no audio condition results, with its strikingly dispersed comprehension
scores, revealed that this reasoning is perhaps too optimistic and
elementary. In fact, almost the opposite seemed to happen as the No
Audio parameter concluded with the highest comprehension scores (95.0\%
for Arabic and 84\% for Spanish).

One of the factors that may have subversively impacted the results is
individual stress and anxiety levels during the viewing process.
Participants may have also been slightly discomforted by viewing videos
under conditions in which they are not used to (i.e. No Audio input for
participants who have always watched videos with audio videos). In
addition, the participants may have had an unconscious aversion towards
the voices of the characters and background sounds (music or character
movement and behavior sounds), thus reducing or straining their
attention. A mixed presentation approach using mismatched audio-visual
content goes against the age-old language teaching technique of full
language immersion. Moreover, variation in individual attention spans
could also explain discrepancy in comprehension scores. Cognitive input
levels may have also led to reduced comprehension if participants
experienced cognitive overload due to divided attention efforts. In this
study, cognitive overload is perhaps evidenced from lower fixation
duration as shown in Figure 4. The participants with the audio input
variable did not exceed the fixation duration of 450 ms and when we
conjecture upon this, we can perhaps argue that once the different
ongoing intake information that compete for cognitive attention reaches
an overload point, the participants' gaze then starts to stagger,
falter, or become sporadic. ``Gaze'' data may be misleading as one's
gaze does not always mean that the participant is on task. Some of the
gaze data may cover actual knowledge input time (in our case
``knowledge'' input would be considered actual ``reading time''). Figure
5 shows that reading subtitles only gets you so far since the highest
comprehension score did not take place during the highest reading
duration. Once again, the ``No Audio'' input reigns supreme. The No
Audio (Arabic) variable led to the highest comprehension scores falling
in the 95\% projectile even coupled with a reading time that was below
100,000 ms.

Gaze fatigue may also have had an impact upon the results since
individual gaze stamina will vary amongst participants. According to
\textcite{bafna2021} study, eye gaze fatigue shown in
eye metrics can also be connected to mental fatigue. Mental fatigue
would be a particularly debilitating variable in the comprehension and
retention levels of the participants. Their research highlighted that
indicators related to rapid eye movements, known as saccades, were the
most effective in identifying fatigue \cite{bafna2021}.

Furthermore, the languages themselves may have influenced the results of
the study because each language is characteristically unique, thus the
same parameters with the eye-ear relationship cannot be prescribed
universally. It is suggested that if visual media (movies, shows, etc.)
will be utilized in instructional language training, then discernment
should be applied especially as it relates to the complexity of the
scenes. What is meant by stating ``complexity'' here is the complexity
of the emotional aspect of content as well as the complexity of the
register being used (i.e. formal advanced language vs. colloquial or
basic language). To provide an example, the first video that was shown
had two characters talking about war and arguing over perceived
cowardice. The characters then talked about the intricacies of
fatherhood and affection. For language learners, this type of content
and heavy emotional sentiment that supports the scene may be too
advanced for the language learners to pick up on physical and audio cues
that are universally understood. In addition, language teachers and
scholars need to be cognizant of the differences of individual reader's
physical eye capabilities (mentioned prior in \Cref{sec-literaturereview}) and how readers
read, since not all people are taught the same reading strategies and
approach. Not all languages have the same presentation and pattern of
their parts of speech. A rather notorious difference is the placement of
the verb, which is essentially what informs the reader of the actual
activity being reported in the sentence. Spanish and English consist of
many shared cognates whereas languages that are totally isolated from
each other's language family tree may produce conflicting data, proving
a potential future hypothesis that language pairs also have an impact on
language acquisition through the eye-ear relationship. Students may not
be focusing their eyes on the part of the sentence which would arguably
be the verb. Any supplementary information can be better understood once
the reader knows what the action is.

Another key consideration to be aware of is that the overall ``reading
process could be affected by semantically relevant auditory input in the
context of reading English/L2 subtitles in video'' \cite[p. 260]{liao2022}. The importance of seeing saccade fluctuation because of
purposeful modulation between subtitle and audio input is that
researchers will be able to understand what language learners
essentially ``tune out'' or withdraw their attention from as a response
to the modulations. These results then reinforce assertions relating to
the efficacy of simultaneous audio and subtitle utilization and
hopefully indicate which sort of trend should be adopted in a learning
environment (i.e. language instructors should indeed use L2 subtitles
while L1 audio is present or not). \textcite[p. 260]{liao2022} proclaimed
that their data showed ``that readers adjusted the way they engaged in
the reading of subtitles in response to the varying needs to read the
subtitles in different audio conditions''. The student's own first
language may also affect the reading of the subtitles. Their first
language habits may dictate them to look towards the end of the sentence
or to focus on a particular linguistic unit (i.e. the verb unit is found
to be more prominent in East Asian languages such as Japanese). The
participant's first language was unfortunately not solicited thus we
cannot expand upon this variable. Successful corroboration would appear
inevitable in this experiment's case, however, there is less of an
emphasis on readers' reaction in this study than there is on new
knowledge acquisition for the reader.

There is perhaps a cognitive effort component that can be overlooked if
one is unaware that ``it takes much conscious effort to keep the eyes
fixated while attending to another location'' \cite[p. 85]{schotter2012}. This assertion raises several questions regarding language
acquisition, about the usage of subtitles and other audio-visual aids.
If effort is expended just on fixation during the concurrent saccades of
the eye, then shouldn't the content that is being presented be minimized
to the lowest amount possible and the introduction of new terms or
grammar points also be minimized? Teachers and other language promoters
do not want to cognitively overload or overstimulate the reader (content
receiver); thus, they should be actively engaged in the matter of
subtitles. Educators and even audio-visual translation trainers may want
to consider generating their own subtitles to fit the needs and learning
level of the content receivers (students, trainees, etc.).


\printbibliography\label{sec-bib}
%conceptualization,datacuration,formalanalysis,funding,investigation,methodology,projadm,resources,software,supervision,validation,visualization,writing,review
\begin{contributors}[sec-contributors]
\authorcontribution{José Rovira Collado}[conceptualization,supervision,writing]
\authorcontribution{Francisco Antonio Martínez Carratalá}[methodology,formalanalysis]
\authorcontribution{Sebastián Miras}[datacuration,validation,writing,review]
\end{contributors}
\end{document}
