\documentclass[spanish]{textolivre}

% metadata
\journalname{Texto Livre}
\thevolume{17}
%\thenumber{1} % old template
\theyear{2024}
\receiveddate{\DTMdisplaydate{2024}{3}{15}{-1}}
\accepteddate{\DTMdisplaydate{2024}{5}{20}{-1}}
\publisheddate{\DTMdisplaydate{2024}{9}{17}{-1}}
\corrauthor{Sebastián Miras}
\articledoi{10.1590/1983-3652.2024.51641}
%\articleid{NNNN} % if the article ID is not the last 5 numbers of its DOI, provide it using \articleid{} commmand 
% list of available sesscions in the journal: articles, dossier, reports, essays, reviews, interviews, editorial
\articlesessionname{articles}
\runningauthor{Rovira Collado, Martínez Carratalá y Miras}
%\editorname{Leonardo Araújo} % old template
\sectioneditorname{Hugo Heredia Ponce}
\layouteditorname{João Mesquita}

\title{Booktok: análisis de las estrategias discursivas multimodales para la promoción de la lectura en TikTok}
\othertitle{Booktok: análise das estratégias discursivas multimodais para a promoção da leitura no TikTok}
\othertitle{Booktok: analysis of multimodal discursive strategies for the promotion of reading in TikTok}

\author[1]{José Rovira Collado~\orcid{0000-0002-3491-8747}\thanks{Email: \href{mailto:jrovira.collado@ua.es}{jrovira.collado@ua.es}}}
\author[1]{Francisco Antonio Martínez Carratalá~\orcid{0000-0002-0587-5063}\thanks{Email: \href{mailto:famc@gcloud.ua.es}{famc@gcloud.ua.es}}}
\author[1]{Sebastián Miras~\orcid{0000-0002-4259-3890}\thanks{Email: \href{mailto:sebastian.miras@ua.es}{sebastian.miras@ua.es}}}
\affil[1]{Universidad de Alicante, Facultad de Educación, Alicante, España.}


\addbibresource{article.bib}
\usepackage{seqsplit}
\begin{document}
\maketitle
\begin{polyabstract}
\begin{abstract}
Los booktoks son vídeos breves que exponen contenidos sobre libros y literatura en la plataforma TikTok, como evolución de otras prácticas literarias en línea como los blogs o booktubers. Estos contenidos, con un formato único adaptado al móvil, han incrementado progresivamente desde 2020, convirtiéndose en una dinámica que ha alcanzado cifras millonarias, especialmente entre los lectores más jóvenes. El objetivo de esta investigación es identificar y diferenciar las estrategias multimodales que los booktoks utilizan para la promoción de la lectura. A partir de un análisis de contenido de una muestra previa de 500 vídeos, se identifican 10 dinámicas distintas de booktoks, algunas novedosas frente a modelos precedentes. Posteriormente, se realiza un análisis multimodal de diez booktoks concretos para mostrar las nuevas formas de hablar de libros a través de breves vídeos. Se describen los tipos de interacción entre el vídeo, el sonido y los contenidos que proponen los booktoks. Se identifican nuevas dinámicas, junto a otras heredadas de los booktuber, pero se confirma que el móvil, como plataforma de difusión de contenidos, impone una enorme velocidad en el consumo de vídeos sobre lectura.

\keywords{Educación \sep Internet \sep Lectura \sep Redes Sociales \sep Vídeos Interactivos}
\end{abstract}

\begin{portuguese}
\begin{abstract}
Os \textit{booktoks} são vídeos breves que apresentam conteúdos sobre livros e literatura na plataforma TikTok, como uma evolução de outras práticas literárias \textit{online}, como \textit{blogs} ou \textit{booktubers}. Esses conteúdos, com um formato único adaptado para dispositivos móveis, têm aumentado progressivamente desde 2020, tornando-se uma dinâmica que atingiu números milionários, especialmente entre os leitores mais jovens. O objetivo desta pesquisa é identificar e diferenciar as estratégias multimodais que os \textit{booktoks} utilizam para a promoção da leitura. A partir de uma análise de conteúdo de uma amostra anterior de 500 vídeos, foram identificadas dez dinâmicas diferentes de \textit{booktoks}, algumas inovadoras em relação a modelos anteriores. Posteriormente, foi realizada uma análise multimodal de dez \textit{booktoks} específicos para mostrar as novas formas de falar sobre livros por meio de vídeos breves. São descritos os tipos de interação entre o vídeo, o som e os conteúdos propostos pelos \textit{booktoks}. São identificadas novas dinâmicas, juntamente com outras herdadas dos \textit{booktubers}, mas confirma-se que o celular, como plataforma de difusão de conteúdos, impõe uma enorme velocidade no consumo de vídeos sobre leitura.

\keywords{Educação \sep Internet \sep Leitura \sep Redes Sociais \sep Vídeos Interativos}
\end{abstract}
\end{portuguese}

\begin{english}
\begin{abstract}
 Booktoks are short videos that showcase content about books and literature on the TikTok platform, evolving from other online literary practices such as blogs or booktubers. These contents, with a unique mobile-adapted format, have progressively increased since 2020, becoming a phenomenon that has reached millions, especially among younger readers. The aim of this research is to identify and differentiate the multimodal strategies that booktoks use for promoting reading. Through a content analysis of a previous sample of 500 videos, ten different booktok dynamics are identified, some of which are innovative compared to previous models. Subsequently, a multimodal analysis of ten specific booktoks is conducted to demonstrate the new ways of discussing books through short videos. The types of interaction between video, sound, and content proposed by booktoks are described. New dynamics are identified, along with others inherited from booktubers, but it is confirmed that the mobile platform imposes a significant speed in the consumption of reading-related videos.
 
\keywords{Education \sep Internet \sep Reading \sep Social Networks \sep Interactive Video}
\end{abstract}
\end{english}

\end{polyabstract}



\section{Introducción}\label{sec-introducción}

En la evolución de los espacios de promoción y mediación lectora en
Internet se encuentran los epitextos virtuales de lectura \cite{lluch2015} que aportan información sobre los
libros y la literatura, animando a leer uno u otro texto literario. El
último espacio de desarrollo de este ámbito son los \emph{booktoks},
breves vídeos sobre lectura y literatura en TikTok, una plataforma o red
social de vídeos para ser disfrutados y creados principalmente a través
del móvil. Estos epitextos son una evolución de modelos anteriores, como
los basados en el texto escrito, como los blogs o los foros literarios o
los basados en vídeos, como los \emph{booktubers} \cite{sorensen2014}
o fotografías y vídeos, como los \emph{bookstagramers} \cite{sánchezgarcía2020}.


Todos estos espacios permiten la \emph{lectura social} \cite{cordon2013}, donde cualquier persona puede compartir a través
de Internet y las redes sociales su opinión sobre sus lecturas e
interactuar con otras personas para hablar de literatura. El fenómeno
global de los \emph{booktubers} como espacio antecedente directo para la
animación lectora, y sus características en lengua española, ha
despertado gran interés en los últimos años \cite{vizcaino2019,tomasena2021}, incluso con
distintos análisis de sus posibilidades educativas \cite{roviracollado2016}. Ya no se escribe sobre libros, si no que se habla, se comenta o
se discute oralmente en vídeos de distinta extensión.

Una de las diferencias principales de estos vídeos con los
\emph{booktubers} es el móvil como espacio natural para su desarrollo,
pudiendo relacionar su uso con el \emph{Mobile} \emph{Learning} \cite{unesco2013}. Las redes sociales, desde su aparición, son espacio para el
desarrollo del aprendizaje informal o invisible, ya que están
constantemente ofreciendo información que se puede convertir en
conocimiento \cite{piscitelli2010}. Después de la
transformación digital provocada por la pandemia del covid-19 se deben
incorporar muchos de estos contenidos y estrategias de aprendizaje
informal a la hora de hablar de aprendizaje digital. También se deben
entender las estrategias discursivas como dinámicas de comunicación para
realizar el mensaje más comprensible y atractivo para los oyentes \cite{coll2021}. En estos espacios, estas estrategias deben aprovechar
todas las posibilidades de la multimodalidad (Martín Menéndez, 2020).
Posteriormente, se plantearán los elementos del análisis multimodal en
el presente estudio.

Instagram también ofrece espacios para los libros, y los
\emph{bookstragramer} publican tanto fotografías como vídeos
(\emph{reels} o \emph{stories}) para hablar de lecturas. Pero son los
\emph{booktoks} los que están adquiriendo mayor atención en este
momento. Actualmente, se encuentran distintas noticias e informes que
destacan el creciente uso que están adquiriendo estos vídeos en el mundo
de la literatura y la producción editorial como elementos de publicidad
que influyen en la elección de lecturas de sus espectadores \cite{nielsen2021}. A principios de esta década, la \textcite{penguin2020}
recomendaba los perfiles de \emph{booktokers} que había que seguir.
Solamente falta acercarse a los principales informes de usos de redes
sociales \cite{IABSpain2022} para comprobar que ambas redes tienen el mayor
crecimiento y audiencia en estos momentos.

Estos espacios generan nuevas figuras como los \emph{influencers}, las
personas con mucho éxito en redes sociales que se convierten en
celebridades \cite{abidin2020}. También ha aparecido el concepto de
``creadores de contenido'', para referirse a las personas que publican
vídeos elaborados, de cualquier temática, de forma periódica, ya que la
grabación y edición de estos vídeos supone en muchos casos una
especialización. Su visibilidad \cite{cordon2019} y
número de visionados permite que sus perfiles reciban una retribución
económica atendiendo al número de reacciones y visitas que tienen sus
vídeos por parte de la plataforma. En la sección de análisis de este
estudio se incluyen algunos de estos vídeos. En el ámbito de la
promoción literaria con esta plataforma se centran distintas
investigaciones sobre este fenómeno, principalmente, en el ámbito
anglosajón \cite{merga2021,martens2022} que detectan
el interés por esta nueva vía de comunicación literaria. \textcite{sanztejeda} han realizado una completa revisión sistemática sobre los
métodos y resultados de investigación sobre lectura en ambas plataformas
donde señalan las diferencias y novedades en su uso, y también los
elementos comunes con anteriores dinámicas.

Aunque TikTok e Instagram puedan parecer herramientas solamente para el
ocio, estas ofrecen múltiples aplicaciones educativas para promover
tanto aprendizajes formales como informales. Como herramientas
audiovisuales, estas ofrecen mensajes multimodales \cite{newlondongroup1996} que se analizarán detenidamente en el cuerpo de la presente
investigación. Muchos vídeos se replican en ambas plataformas, pero la
investigación se concreta exclusivamente en TikTok.

El objetivo principal de este estudio se centra en describir los
principales usos de \emph{booktok} en lengua española a través de un
análisis de contenido del discurso multimodal \cite{gomez2016}
empleado por las personas creadoras de los diferentes contenidos para
hacer los vídeos más atractivos.

A partir de este objetivo se plantean dos secundarios que se desprenden
del análisis.

El primer objetivo secundario es plantear la evolución de los
\emph{booktoks} desde modelos anteriores.

El último objetivo secundario es identificar algunas de las diferencias
de los \emph{booktoks} con los modelos anteriores.
\section{Metodologia}\label{sec-metodologia}

Para avaliar o desempenho dos modelos no processamento de ambiguidades quanto aos parâmetros da ambiguidade lexical, sintática e semântica, conduzimos tarefas utilizando um conjunto de dados que foi elaborado e analisado pelos autores do presente artigo, um grupo de seis estudantes de Letras e Linguística, cada um com um conhecimento maior em áreas distintas da linguística, como fonética, tradução, sintaxe e análise do discurso. Esses estudantes serão referidos no trabalho como juízes-humanos por se demonstrarem aptos a produzir e julgar de forma adequada os dados, além disso, vale salientar que os dados e resultados das tarefas foram avaliados, posteriormente, por uma coautora que é especialista na área de variação linguística.

Para garantir a consistência e a confiabilidade das frases geradas pelo grupo, foi adotado o procedimento de \textit{inter-annotator agreement}. As sentenças que não receberam consenso absoluto entre os juízes-humanos foram revisadas ou excluídas do corpus. Como resultado, não foi necessário calcular o coeficiente de concordância {\footnote{Ao criar o corpus, as frases foram revisadas por todos os autores do artigo, e apenas aquelas com 100\% de aprovação no procedimento de inter-annotator agreement foram disponibilizadas para os testes, com o objetivo de minimizar a presença de frases problemáticas. As frases geradas pelos modelos também passaram por uma revisão conjunta e foram discutidas entre os autores para garantir consenso absoluto a partir do mesmo procedimento. Por exemplo, na Tabela 4, que corresponde à geração de frases por parte do modelo, a sentença \enquote{O homem viu o acidente com os próprios olhos} contém um componente favorável à criação de ambiguidade sintática, o adjunto \enquote{com os próprios olhos}, no entanto, com base na teoria de Cançado, essa ambiguidade é válida quando acompanhada por uma ambiguidade semântica, que permite o duplo sentido na compreensão pragmática do contexto dentro do enunciado. Já na Tabela 3, a frase \enquote{Pedi o prato principal ao garçom, era filé!} é parte do corpus das sentenças distratoras, pois a estrutura do enunciado fornece informações suficientes para evitar outras interpretações, ou seja, entendemos que \enquote{prato} não se refere ao utensílio doméstico e \enquote{filé} não é um elogio metafórico. Sendo assim, essa frase foi considerada não ambígua pelos seis juízes-humanos e pela especialista em linguística que revisou o trabalho.}}, uma vez que todas as sentenças, tanto as ambíguas quanto as não ambíguas, só foram incluídas após alcançarem 100\% de aprovação. Esse critério rigoroso assegurou que o conjunto de dados utilizado nas tarefas fosse  confiável ao máximo para minimizar divergências.

O experimento foi composto por um grupo de 120 sentenças, distribuídas de forma balanceada entre os três tipos de ambiguidade. Dessas, 60 apresentam algum tipo de ambiguidade, seja semântica, lexical ou sintática. As frases ambíguas foram elaboradas com o objetivo de provocar especificamente um dos três tipos de ambiguidade (semântica, lexical ou sintática). No entanto, é possível que algumas sentenças apresentem mais de um tipo de ambiguidade, que não foi avaliado durante a criação das mesmas. No decorrer de nossa análise, buscamos isolar ao máximo cada frase, de modo que apenas um elemento causador de ambiguidade estivesse presente. Essa abordagem nos permite um controle mais rigoroso das variáveis observadas nos experimentos, uma vez que a ambiguidade é um fenômeno complexo e multifacetado.

É importante ressaltar que, dentro de uma única sentença, podem existir múltiplos fatores que contribuem para a ambiguidade, o que torna a sua identificação e análise ainda mais desafiadoras, então, para um estudo inicial, de caráter pioneiro com LLM's formamos dados linguísticos mais artificiais, mas planejamos trabalhar futuramente com dados provenientes de corpora ou textos reais, incorporando materiais autênticos que nos permitam uma compreensão mais abrangente das ambiguidades linguísticas em contextos variados.

Foram criadas 20 sentenças com ambiguidade lexical — que abrangem casos de homonímia e polissemia, sem distinção de categoria na análise das frases —, 20 sentenças com ambiguidade semântica, nas quais o referente dos pronomes não está claro, e, por fim, 20 sentenças com ambiguidade sintática, envolvendo adjuntos adnominais ou adverbiais ambíguos que provocam duplo sentido devido às diferentes organizações estruturais que a frase pode ter (Tabela \ref{tab:amostras_sentencas}).

As outras 60 sentenças distratoras tiveram sua ambiguidade barrada ao máximo, não sendo verificadas pelos juízes-humanos durante a elaboração dos dados linguísticos. As sentenças distratoras foram criadas com a intenção de evitar qualquer forma de ambiguidade. Da mesma forma que foi feito o julgamento das frases ambíguas, as frases não ambíguas foram avaliadas em um nível de significado mais isolado, sem considerar diversos contextos enunciativos figurativos, na maioria das vezes, nas quais ela poderia fazer sentido, logo assumimos um posicionamento de interpretação mais pragmático linguístico, que corrobora com Cançado. Essas mesmas considerações são válidas para as avaliações feitas em relação às frases que foram geradas pelos modelos de linguagem na Tarefa 4.

Um exemplo para ilustrar essa forma de análise é a frase \enquote{Pedi o prato principal ao garçom, era filé!} (Tabela 3) em que em um contexto muito específico, poderia significar que o cliente pediu o utensílio principal de servir comida e este utensílio era \enquote{filé}, um termo popular para se referir a algo bom, porém, a princípio, a frase foi escrita tendo em vista o significado mais óbvio, que se determinou a partir das pistas interpretativas deixadas dentro da frase na qual um sentido lexical confirmava o outro sem a necessidade de buscar condições exofóricas à sentença que justificassem uma polissemia.





\begin{table}[htpb]
\centering
\begin{threeparttable}
\caption{}
\label{tab:amostras_sentencas}
\begin{tabular}{llp{6cm}}
\toprule
Id & Sentença & Classe \\
\midrule
1 & A rede caiu. & Ambiguidade lexical \\
2 & Ela não gosta da amiga dela. & Ambiguidade semântica \\
3 & O menino viu o incêndio do prédio. & Ambiguidade sintática \\
\bottomrule
\end{tabular}
\source{\url{https:XXXXXXXXX}.}
%\notes{Se necessário, poderá ser adicionada uma nota ao final da tabela.}
\end{threeparttable}
\end{table}

%\source{\url{https://osf.io/u7wre/?view_only=572c74eb4c634d47a02ad25485ea8caa}.}

Para responder as nossas perguntas de pesquisa, foram conduzidas quatro tarefas {\footnote{As respostas dadas pelos modelos de linguagem nas tarefas foram coletadas entre julho de 2023 até janeiro de 2024. Coletas posteriores podem levar a resultados diferentes devido às atualizações dos modelos de linguagem. Para evitar ao máximo o enviesamento dos modelos, as perguntas foram formuladas da forma mais objetiva possível, evitando dar pistas sobre a resposta correta ou sinalizando qual era o resultado esperado. A preocupação em minimizar testes com viés tinha o objetivo de garantir uma avaliação mais precisa do desempenho dos modelos de linguagem, afastar casos de generalização nas respostas, favorecer a imparcialidade dos resultados e alcançar um conjunto de respostas o mais transparente possível.}} distintas com as sentenças criadas. Em todas, foram realizadas coletas duplicadas das interações para cada frase, reiniciando o \textit{console} entre cada coleta para evitar qualquer influência do contexto que pudesse gerar respostas tendenciosas. Essa abordagem permitiu avaliar a consistência dos modelos nas respostas fornecidas.


A tarefa 1 visava identificar se os modelos conseguem detectar a presença de ambiguidade em cada sentença por meio da seguinte instrução: \textbf{A sentença ``[sentença]'' é ambígua? Responda, sim, não ou não sei}. Foram apresentadas individualmente todas as sentenças e registradas as respostas dos modelos, comparando-as com a nossa classificação prévia. As respostas foram cuidadosamente avaliadas quanto à correção e abrangência das explicações fornecidas por seis juízes-humanos que as julgaram independentemente. A partir dos resultados, foi gerada uma matriz de confusão para computar a quantidade de verdadeiros positivos (sentenças que são ambíguas e que os modelos classificaram como ambíguas), falsos positivos (sentenças que não o são e que os modelos classificaram como ambíguas), verdadeiros negativos (sentenças que não são ambíguas e que os modelos assim classificaram como não ambíguas), e falsos negativos (sentenças que são ambíguas e que os modelos classificaram como não ambíguas).


Na tarefa 2, foi realizado um teste para avaliar a capacidade dos modelos em distinguir corretamente entre as três classes de ambiguidade estudadas neste trabalho, formulando a seguinte pergunta para cada modelo: \textbf{``Qual o tipo de ambiguidade?''}. A tarefa consistiu em perguntar qual o tipo de ambiguidade da sentença que foi classificada anteriormente como ambígua ou não ambígua. Na tarefa 3, foi verificada a capacidade dos modelos em desambiguar as sentenças que foram fornecidas a eles. Com esse propósito, foram apresentadas frases que incluem tanto sentenças ambíguas, quanto sentenças não ambíguas, e solicitado aos modelos a seguinte instrução: \textbf{Faça a desambiguação da frase: ``[sentença]''}. A tarefa busca testar a habilidade dos modelos em compreender e interpretar o contexto, escolhendo a interpretação mais apropriada quando a ambiguidade está presente. 

Na tarefa 4, foi avaliada a capacidade dos modelos em gerar frases ambíguas na categoria solicitada. Para isso, pedimos para cada modelo gerar frases da seguinte forma: \textbf{Gere 20 frases com ambiguidade ``[categoria]''}. Em seguida, as respostas obtidas foram avaliadas por juízes-humanos, buscando compreender quão preciso é o ChatGPT o Gemini ao criarem frases que apresentam múltiplas interpretações contextuais.

Para mensurar quantitativamente o desempenho dos modelos, foi utilizada a métrica de acurácia, a qual já é amplamente empregada na área de aprendizado de máquina  \cite{naser2021error, freitag2021funccao}. A acurácia, no contexto da classificação, representa a proporção de frases corretamente classificadas pelos modelos em relação ao total de frases apresentadas na tarefa, como apresentado na equação \ref{eq_1}.

\begin{equation}
    \text{acc} = \frac{\text{Número de previsões corretas}}{\text{Total de previsões}}
    \label{eq_1}
\end{equation}

Todas as sentenças criadas por nós e geradas pelos modelos durante as tarefas estão disponíveis no Apêndice \ref{sec-apendice}. As respostas dos modelos durante as tarefas estão disponíveis para download\footnote{\url{https://docs.google.com/spreadsheets/d/1AOff1GJmh3oWIuKdfBGGHWeox-7HtuHQW0LkaT5yFVI/edit?usp=sharing}} em nosso repositório.


%\footnote{\url{https://osf.io/u7wre/?view_only=572c74eb4c634d47a02ad25485ea8caa}}
\section{Resultados}\label{sec-resultados}

En la siguiente tabla se recogen los principales datos de los 10 vídeos
analizados para posteriormente proceder a un análisis multimodal
centrado principalmente en el vídeo, y en menor medida, en el uso del
texto. Se recogen solamente tres variables: \emph{Duración}, \emph{Me
	gusta} y \emph{Comentarios} para confirmar que son vídeos mucho más
breves y destacar que en nuestro análisis no se prioriza en la audiencia
y en la interacción.

\begin{table}[htbp]
\centering
\begin{threeparttable}
\caption{Análisis descriptivo del vídeo}
\label{tab-03}
\begin{tabular}{p{3cm} l l l l}
\toprule
Dinámica & Perfil & Duración & Me gusta & Comentarios \\
\midrule
1. What are you reading? & \seqsplit{@claudiacp\_books} & 05 seg. & 56 & 1 \\
2. Readers follow party & \seqsplit{@laslecturasdeloli} & 15 seg. & 3047 & 717 \\
3. Bibliotecas & \href{https://www.tiktok.com/@bibliotecaugena/video/6971466004585057542}{\seqsplit{@bibliotecaugena}} & 13 seg. & 2365 & 14 \\
4. Tipos de Booktoker & \href{https://www.tiktok.com/@pandi.book/video/7199333140748438790}{\seqsplit{@pandi.book}} & 1min. 54 seg. & 83.4K & 973 \\
5. Promoción lectora & \href{https://www.tiktok.com/@iriayselene/video/7211269180484750598}{\seqsplit{@iriayselene}} & 10 seg. & 21.4K & 194 \\
6. Editoras & \href{https://www.tiktok.com/@vreditorasya/video/7223140662022114565}{\seqsplit{@vreditorasya}} & 08 seg. & 243.8K & 487 \\
7. Bookinfluencer & \href{https://www.tiktok.com/@patriciafedz/video/7222221715819072774}{\seqsplit{@patriciafedz}} & 2 min, 15 seg. & 23K & 150 \\
8. Día del libro & \href{https://www.tiktok.com/@javierruescas/video/7224553867566910747}{\seqsplit{@javierruescas}} & 1 min. 35 seg. & 30.4K & 86 \\
9. Videopoemas & \seqsplit{@marinalcuadrado} & 1 min. 9 seg. & 12.5K & 719 \\
10. Así dijo\ldots & \href{https://www.tiktok.com/@solo.palabrqs/video/7190817595401030918}\seqsplit{@solo.palabrqs} & 20 seg. & 138.2K & 181 \\
\bottomrule
\end{tabular}
\source{Elaboración propia.}
\end{threeparttable}
\end{table}

Como se puede observar, muchos de estos vídeos tienen una enorme
audiencia, con miles de ``Me gusta''. Solamente la primera categoría
tiene muchos menos, pero se ha incluido este vídeo porque es el que más
claramente ejemplifica una nueva dinámica. El análisis de las diez
dinámicas analizadas en la plataforma TikTok permite una primera
diferenciación en la comunicación visual que se realiza entre aquellas
que adoptan un punto de vista mediado frente a las que no. Este punto de
vista mediado implica que las acciones que ven los espectadores son
vivenciadas como si fueran la persona que ha creado el vídeo (mediada:
inferida) o bien desde dentro de la imagen (mediada: inscrita) cuando se
observa alguna de las partes de su cuerpo (e inclusive desde sus
espaldas). Las dos dinámicas que emplean este tipo de estrategias de
comunicación visual son 1. \emph{¿Qué estás leyendo? (What are you
	reading?)} y 2. \emph{Fiesta de seguidores-lectores (Readers Follow
	Party)}, en la que se aprecian diferentes matices con algunos ejemplos
concretos. Estas dinámicas buscan la mayor intervención del espectador
al ubicarlo al mismo nivel experiencial que la persona que crea el
contenido. Y el protagonista de ambas son los libros o las estanterías
de los usuarios, siendo vídeos de corta duración acompañados de una
pieza de audio concreta que pone título a estas dinámicas en inglés.
Además, ambos son muy breves e indican nuevas dinámicas hiperbreves de
esta red social.

Siguiendo un ejemplo de la dinámica 1. \emph{¿Qué estás leyendo?} (de la
usuaria @claudiacp\_books) la perspectiva mediada e inscrita persigue
generar un efecto de suspense (inicialmente solo se muestra la parte
inversa al lomo del libro, donde solamente se ven unas páginas y no se
puede identificar la obra) y solo aparece la mano de la usuaria
sosteniéndolo unos segundos hasta revelarlo y generando una metonimia
visual \cite{moya-Guijarro}. Este tipo de dinámicas reducen la
distancia social entre el espectador y la autora del vídeo, para generar
una sensación de intimidad y adopta un plano cenital donde tanto la
persona creadora del contenido como el espectador se ubican en una
relación de superioridad frente al objeto-libro representado. En estos
vídeos, ambas personas, \emph{booktoker} y espectadores comparten la
misma perspectiva respecto al vídeo. En este caso, esta metonimia visual
que acompaña al giro para presentar la cubierta del libro va acompañada
de un audio específico, titulado: \emph{What this person reading right
	now?} Este audio dura solamente cinco segundos e inclusive se pueden
encontrar en la plataforma y usarlo en nuestras creaciones
\url{https://www.tiktok.com/music/Whats-this-person-reading-right-now-6904077601871170309}
y cuenta con más de cien mil vídeos que usan este misma dinámica y
sonido, que lo convierten en tendencia.

El ejemplo de la dinámica 2. \emph{Fiesta de seguidores-lectores} (de la
usuaria @laslecturasdelol) se basa en la creación de un solapamiento
entre la perspectiva del usuario y del espectador, dejando que el plano
tenga una mayor distancia para mostrar la panorámica de los espacios
lectores que recorre el usuario como sus estanterías. La implicación del
espectador también opta por una perspectiva mediada, pero en este caso
inferida, dado que ambos contemplan la escena en primera persona. Se
opta por una angulación donde prima la horizontalidad (y el aspecto
frontal) para que el espectador se sienta partícipe de la misma forma
que el usuario. La transcripción del vídeo de 15 segundos, que aparece
también en el vídeo es la siguiente:
\begin{quote}
		Fiesta de seguidores: Si sos \emph{booktoker} y tenés menos de cinco mil
		seguidores, seguime y así te puedo seguir. Deja un comentario con tu
		número de seguidores actual y a cada persona que deje un comentario
		seguila, No te olvides de compartir el link para que más booktokers lo
		vean. ¡Que lluevan seguidores!
		
		Fuente:
		\url{https://www.tiktok.com/@laslecturasdeloli/video/6972178596152511749}.
\end{quote}

Este vídeo es una adaptación al español de la dinámica \emph{Reader Follow Party}\footnote{\url{https://www.tiktok.com/tag/readerfollowparty}}
que se ha usado en más de diez mil ocasiones, aunque generalmente por
público anglófono. Por ejemplo, la usuaria @maddiesreadss ofrece un
vídeo con subtítulos automáticos de este audio, donde se observa que la
interacción entre cámara y elementos es idéntica al ejemplo seleccionado
en español.

\begin{quote}
	Reader Follow Party. If You're not already, follow me and I'll follow
	you back. Then, go to the comments and tell us what book you're
	currently reading. When someone likes your comment follow them and
	they're follow you back. Yay for more books friends!
	
	Fuente:
	\url{https://www.tiktok.com/@maddiesreadss/video/7236953611547184386}.
\end{quote}


El análisis se inicia con ambos vídeos porque son dos de las dinámicas
más novedosas de \#\emph{Booktok} y el protagonista central de ambas
prácticas son los libros. Aunque la dinámica con las que se identifican
estos vídeos son de usuarias hispanohablantes, las etiquetas y audios
originales son en inglés, con mucha mayor proyección, siendo claras
tendencias en el ámbito de TikTok.

Frente a estas dos dinámicas que adoptan una perspectiva subjetiva en su
grabación y montaje, se encuentran aquellas que optan por una
perspectiva no mediada en la que la implicación visual para el
espectador se compensa con otras estrategias como el contacto visual
directo o la creación de dinámicas con un componente humorístico. Entre
los casos analizados, se destaca un ejemplo por parte de una biblioteca
pública (@bibliotecaugena) 3. \emph{Bibliotecas} donde el efecto del
vídeo es ubicar en tercera persona al espectador y combinar las opciones
de focalización desde la alternancia de las miradas de oferta (aquella
que se dirigen por la persona protagonista, la bibliotecaria, hacia los
objetos y elementos que quiere que el espectador se fije) y las de
demanda donde prima el contacto visual (con las que interpela al
lector-espectador a seguir la dinámica). En este sentido, el plano se
aleja y se contempla a la persona con una mayor distancia respecto al
espectador (dado que prima la sucesión breve de acciones) y se ubica
desde el uso de ángulos frontales para que se facilite la sensación de
intervención del usuario. También tiene una duración breve, 13 segundos
para presentar varios libros. Este perfil tiene muchos vídeos similares,
con la bibliotecaria como protagonista y contenidos sobre la función e
importancia de las bibliotecas y es un claro ejemplo para la mediación
lectora en redes sociales.

El componente humorístico también forma parte de montajes como la
presentación de la dinámica 4. \emph{Tipos de Booktoker} (de
@pandi.book), que se basa de nuevo en la alternancia entre el contacto
visual de la joven protagonista del vídeo con el espectador (para
acrecentar la sensación de cercanía) y las miradas de demanda (ahora,
invitada) cuando se reproduce (simula, en este caso) un diálogo con otra
persona para darle réplica. En esos intercambios, cuando se dirige a
otra persona, una voz en \emph{off}, la posición corporal se angula
horizontalmente para crear en ese diálogo una mayor sensación de
desapego con el espectador (para compartir la reacción de la usuaria que
aparece en el vídeo). Cuando ese breve diálogo (en forma de reproche en
ocasiones) finaliza, se busca nuevamente la cercanía con la protagonista
al emplear preferentemente primeros planos (alternando con planos
americanos) para generar una situación de intimidad, como también ocurre
cuando se ubica en una relación neutral y frontal con el espectador.
Además, este vídeo muestra los tipos de nuevos lectores que se
identifican con estas dinámicas.

Otro ejemplo recurrente en la plataforma es la creación de un efecto
paródico a partir de la contradicción \cite{bateman2014} entre los modos
visuales y textuales. Por ejemplo, en esta presentación de su nuevo
libro por parte de sus autoras de LIJ en el vídeo 5. \emph{Promoción}
\emph{lectora} (de @iriayselene), se observa cómo la información textual
pasa a un segundo plano, sobreimpresionadas en la pantalla, mientras que
las protagonistas vocalizan la letra de la canción que suena de fondo
(\emph{Naughty} - Alisha Weir \& The Cast of Roald
Dahl\textquotesingle s Matilda The Musical). El texto dice y la
interacción con los libros es la siguiente:

\begin{quote}
	Cartela 1 Verde. Cuando nos criticaron un montón por hacer un rettelling
	de \emph{Ana de las Tejas Verdes} con representación LGTB+. (Muestran su
	libro \emph{Anne sin filtros})
	
	Cartela 2 Ocre. Así que hemos hecho lo mismo, pero con El Mago de Oz.
	(Presentan nuevo libro Seremos huracán)
\end{quote}


En este caso Iria y Selene ya eran famosas \emph{booktubers} y autoras
de literatura juvenil que han empezado una nueva aventura en
\emph{booktok}. Como se ha comentado anteriormente, en la plataforma se
opta con asiduidad al reemplazo de la voz de los usuarios por dejar que
sea la música la que complete el mensaje audiovisual, siendo este audio
del musical \emph{Matilda} muy habitual para presentar bromas o actos
traviesos.

Otro claro ejemplo de dinámicas paródicas es el vídeo de @vreditorasya
6. \emph{Editoras} (de @vreditorasya) donde se opta por no aparecer su
cabeza en la pantalla y se da respuesta a la pregunta sobreimpresionada
(``Cuando dice que me quiere regalar un libro, pero no sabe cuál'') con
la intención de generar un \emph{sketch} humorístico en la que poner de
manifiesto el interés desmedido por la lectura, porque luego presenta un
papel con una lista muy larga de libros deseados (\emph{Wish List}),
aunque en este caso, apenas aparecen libros y no se leen los títulos,
pero sí se destaca el gusto por la lectura. Además, este vídeo aprovecha
una tendencia humorística ``Ding dong'', donde después de oírse un ruido
sordo, una mujer imita el sonido de una campanilla para presentarse.
Muchas veces este sonido se usa de forma de incitación sexual, al usarse
para presentar a una persona semidesnuda se usa de forma sexista, pero
este \emph{booktok} lo aprovecha para presentar el interés por los
libros de la protagonista.
\url{https://www.tiktok.com/music/suono-originale-6830486689857784582}

Frente a las dinámicas anteriores (tanto de perspectiva mediada como no
mediada), en la plataforma también se reproducen dinámicas heredadas de
YouTube en el modo de presentar la información. En estos casos, las
personas creadoras de contenido dan relevancia a su figura como
``presentadoras de contenido''. En primer lugar, sobre contenidos como
``influencers'' \cite{establés2019}.
\emph{Bookinfluencer} (de @patriciafedz) en la que comparten las
anécdotas o vivencias de las personas usuarias, como es el caso de
@patriciafedz, donde se opta por primeros planos y el contacto visual
directo para compartir esa intimidad del relato con el espectador. Esta
tipología incluso explota la imagen de la protagonista, una mujer joven
y atractiva, como en otras redes sociales \cite{calvo2018,dezuanni2022}. Este vídeo tiene más de doscientas
mil visualizaciones y veinte mil me gusta, con varios comentarios que
incluso señalan algún error que realiza la protagonista al citar obras
literarias. Además, Patricia Fernández tiene más de trescientos mil
seguidores y una producción constante de vídeos, por lo que se puede
destacar como una ``\emph{bookinfluencer}''. En este tipo de vídeos, la
ubicación de la angulación vuelve a situarse levemente contrapicada,
dejando al espectador de nuevo en inferioridad frente a la persona que
narra dicha anécdota. Esta circunstancia también indica cómo el cambio
de grabación (principalmente el móvil) en una plataforma como TikTok
frente a Youtube donde la cámara suele ubicarse en una posición más
elevada en la pantalla del ordenador (siendo el espectador el que se
encuentra en superioridad, habitualmente, frente al creador del
contenido). Además, estos vídeos tienen mayor duración (más de dos
minutos) y edición, intercalando otros vídeos en el montaje. En este
caso, no se opta por efectos sonoros, y lo importante es la explicación
de la protagonista.

Además de vivencias relacionadas con experiencias alrededor de la
literatura, se han recogido vídeos sobre contenidos para compartir el
conocimiento sobre detalles relacionados con la lectura, como el de
@javirruescas 8. \emph{Día del libro} sobre el origen de esta efeméride.
Javier Ruescas es también un famoso bloguero, \emph{booktuber} y autor
de literatura juvenil \cite{ruescas2012} que ha sabido adaptarse al nuevo
medio, también con una gran audiencia. De nuevo, la importancia de estas
dinámicas es ubicarse frente al espectador ofreciéndole un plano cercano
que disminuya la distancia social, el contacto visual directo y una
angulación de neutralidad donde no ubicarse en superioridad con el
espectador. Otro efecto curioso es la sobreimpresión de los subtítulos
que permite simultáneamente leer la explicación. Como en el caso
anterior, no hay audio de acompañamiento, solamente la voz del
\emph{booktoker} y este vídeo como el precedente, también tiene mayor
duración, se ha grabado desde un ordenador sentado, y luego se ha subido
a la plataforma, pensada para las reproducciones en móviles.

En esta categoría también hay otros ejemplos que optan por crear
contenidos como recitados o lecturas en voz alta como la dinámica 9.
\emph{Videopoemas}. Pese a que pueden darse diferentes perspectivas, se
pone en relieve con el ejemplo de @marinalcuadrado cómo se diluye la
sensación de cercanía y la angulación vertical (contrapicada) juega para
ubicar a la persona que recita en superioridad respecto al espectador
que, además, encuentra la empatía de la mirada de la persona que recita
al emplear una focalización de contacto visual de demanda directo
(reclamando nuestra atención). Este formato con un protagonista
recitando también tiene una audiencia importante, con más ciento
cincuenta mil visualizaciones. Es un videopoema más cercano y con una
protagonista que hace atractiva la poesía. Las dinámicas tradicionales
de videopoemas en Youtube se basaban en montajes de diapositivas con el
recitado de fondo, como se verá a continuación.

Finalmente, con una estrategia más distante en la comunicación visual de
los protagonistas, también hay contenidos que optan por la desaparición
del creador para emplear una imagen que sirva de apoyo a un pasaje, cita
o fragmento de un autor, en este caso Mario Benedetti. La dinámica 10.
\emph{Así dijo\ldots{}} y se basa en reproducir citas y versos de
autores famosos. Se incluye esta categoría porque se considera que tiene
claras posibilidades didácticas para la enseñanza de la literatura y
confirma que grandes autores también tienen presencia en esta red.
Aunque también tienen mucha audiencia, estos vídeos muchas veces
atribuyen falsamente poemas o textos a autores, por lo que es necesario
usarlos con atención \cite{roviracollado-hernandez-ortega-2023}. Este tipo
de montajes intentan plasmar dinamismo mediante la sucesión de imágenes
acompañadas de un fragmento musical que amplifica de manera emotiva el
mensaje verbal escogido por el usuario @solo.palabrqs. Así, se trata de
un tipo de dinámica más impersonal y en la que se ofrece un videoclip o
sucesión de imágenes con música o recitado, más que la interacción con
los espectadores. En este caso se citan unas palabras concretas, pero
hay muchos videopoemas que aprovechan las voces originales de estos
autores.

\section{Discusión y conclusiones}\label{sec-discussion}

Los resultados de este análisis multimodal tenían como objetivo
describir los principales recursos discursivos multimodales de los
\emph{booktok} en lengua española a través de un análisis de contenido,
basado tanto en la imagen presentada en la grabación y la interacción
personal como en los textos y elementos sonoros y voces utilizados. Una
de las cuestiones clave se centra en la inmediatez y cercanía producida
por el hecho de que la mayoría de las grabaciones se realiza a través
del móvil y transforma la perspectiva de los \emph{booktoks} y que su
breve duración permite un consumo rápido al ampliarse su contenido a
través de diferentes elementos para la edición. Incluso hay vídeos donde
se pierde la voz del protagonista por una música de acompañamiento y un
mensaje sobreimpreso en la pantalla. Otro aspecto esencial es que el
libro y la literatura reclaman su espacio en esta serie de dinámicas,
siendo el protagonista de muchos de ellos, aunque también se encuentran
\emph{bookinfluencers} centrados en personas concretas. En este sentido,
se señala que se diferencian aquellos vídeos que optan por una
perspectiva subjetiva, donde el foco central está en los libros o en las
librerías, como los dos primeros ejemplos 1. \emph{¿Qué estás leyendo?}
(What are you reading?) y 2. \emph{Fiesta de seguidores-lectores}
(Readers Follow Party), con respecto a las demás, que son más
convencionales, y algunas son variaciones directas de los
\emph{booktubers}. En líneas generales, se observa que las opciones
empleadas en estas dinámicas de TikTok para la promoción de la lectura
son más atractivas y producen una mayor implicación con la audiencia al
reducir la distancia con su audiencia y con una gramática visual \cite{kress2006} más persuasiva al reclamar con diferentes estrategias
su implicación y un mayor tono humorístico. Ese dinamismo y discurso
multimodal provoca que, en muchas de estas formas de promoción, el
contenido reflexivo se relega a un segundo plano y la creación busca una
integración de recursos (textuales, musicales y visuales) que priman una
presentación atractiva y efectiva para las personas que siguen dicha
cuenta, más que un espacio de divulgación o de encuentro. Al no
necesitar la imagen de una persona en estas dinámicas, sino de un libro,
son además propuestas didácticas que se pueden aprovechar con jóvenes
lectores. Además, estas nuevas dinámicas buscan principalmente fomentar
la interacción entre los usuarios y conseguir más seguidores, siendo
virales como prácticas iniciales de Twitter como \emph{Follow}
\emph{Friday} (\#FF) \cite{cui2012}.

Respondiendo al segundo objetivo de la investigación, esta permite hacer
un recorrido sobre las plataformas digitales de promoción lectora y la
evolución de las herramientas en este siglo. Entre 2001 y 2017 se usaron
distintos foros, como los usados por Laura Gallego \cite{lluch2012}
para promocionar sus lecturas y conocer la opinión de los lectores.
Desde 2006 en adelante aparecen los \emph{blogs} literarios, con
muchísimas variedades \cite{garcia2014}. Entre 2011
y 2018 fue el momento de los \emph{Booktubers}, transformando las
recomendaciones hacia productos audiovisuales \cite{tomasena2021}. Desde
2016 hasta la actualidad es el momento de \emph{Bookstragram} \cite{quilescabrera2020} y también en 2020 comienza la etapa de los
\emph{Booktok}, conviviendo ambas dinámicas, muchas veces confluyendo en
el producto audiovisual final. En esta investigación no se ha centrado
en su comparación. Aunque en Instagram todavía se encuentran
publicaciones centradas en la fotografía de libros y breves comentarios,
prácticas que están entre los blogs y el \emph{microblogging}, son las
\emph{stories} y los \emph{reels} los espacios de mayor desarrollo y
audiencia. También Facebook incorpora la sección de \emph{Watch} para
incorporar vídeos de distinta extensión y Twitter permite insertar
vídeos, después de descartar otras opciones audiovisuales. En cada uno
se encuentran distintos contenidos, audiencias y cada algoritmo
proporciona el acceso a distintos vídeos. Pero son las dinámicas
audiovisuales de TikTok, incorporadas también a Instagram, las que se
convierten en virales en la actualidad. Y siempre es el móvil el
dispositivo de acceso y creación de estos breves vídeos. Además, cabe
destacar el cambio generacional. Mientras que los \emph{Booktubers} casi
han desaparecido, porque las grandes celebridades han crecido o han
tenido que migrar a las nuevas plataformas, los \emph{Booktokers} están
en la cresta de la ola. Brevedad y uso del móvil como características
principales de los \emph{booktoks} son las principales diferencias con
modelos anteriores, respondiendo al último objetivo planteado.

Siguiendo la máxima de Marshall Mcluhan de ``el medio es el mensaje'' se
asume que se está ante una categoría diferente, mediada siempre por el
móvil como dispositivo para crear y acceder a estos vídeos. Es cierto
que se debe asumir la superficialidad y rapidez de estos contenidos, que
se consumen en masa, uno detrás de otro y pocas veces dejan rastro en la
memoria. Pero en el caso de los \emph{booktoks} se aprecia su enorme
importancia en el mercado editorial y en el sistema de recomendación de
libros entre iguales, conformando un nuevo canon de lecturas accesible
en Internet \cite{lluch2021}. Si hace pocos años era \emph{Goodreads} la
plataforma preferida para acceder a sugerencias de otras personas
\cite{garcia-roca2020}, TikTok ya ha ocupado ese espacio rápidamente como
bien señalan editoriales y ferias del libro \cite{penguin2020}.

En Internet va todo muy rápido y ya hoy puede que sea \emph{ChatGPT} y
otras inteligencias artificiales las que ocupen ese espacio. Pero ahora
las y los lectores más jóvenes están siempre conectados a estos vídeos y
hay que asumir su importancia. Son obvias las diferencias de la mayoría
de estos breves vídeos frente a las reseñas escritas (blogs) y los
vídeos largos (\emph{booktubers}) que profundizan en los temas
literarios. Pero estas mismas críticas de superficialidad o falta de
profesionalidad los recibieron aquellos espacios en su momento de auge,
creados por jóvenes en su momento. Aunque no se ha focalizado en el
presente análisis en este aspecto, cabe señalar la juventud de muchos de
los protagonistas, tanto creadores de contenido como espectadores. La
mayoría personas que están todavía formándose como lectores competentes.
También se quiere mencionar el predominio del libro en papel como objeto
y el indiscutible protagonismo de las mujeres, ya que las
\#\emph{booktokers} son chicas cada vez más jóvenes.

La creación de vídeos y el análisis de otras producciones puede ayudar
al desarrollo de las competencias digitales \cite{allué2023} y los
modelos anteriores ya demostraron su utilidad pedagógica \cite{paladines_Paredes_Aliagas_Marín_2021}, así como los actuales \emph{Booktoks} ofrecen nuevas
posibilidades didácticas \cite{dezuanni2021,acevedo2022}.

Cabe resaltar la transmedialidad y la interacción con otras plataformas,
como \emph{Goodreads}, donde la reseña literaria es más profunda
\cite{roviracollado2021}(Rovira-Collado, 2021) o \emph{Wattpad} \cite{garciaroca2019},
como espacio de creación y remezcla literaria. Si los \#\emph{booktook}
interactúan con estos espacios, las posibilidades de reflexión lectora
son mayores.

Se asumen las limitaciones de este estudio, con un corpus concreto de
vídeos y con unas características asignadas principalmente por el
algoritmo de la plataforma, pero se considera que es una completa
descripción de las dinámicas de \emph{booktoks} en español que completa
otras investigaciones \cite{guinez-cabara2022}. Tampoco
se ha realizado un análisis especifico de las obras literarias citadas
en cada vídeo. Sería interesante identificar cuáles son las tendencias
literarias que promueven los \emph{booktoks}, aunque se augura que serán
los superventas de la literatura juvenil de cada momento. Queda por
confirmar cuánto influyen estos vídeos en las ventas de cada género
\cite{merga2021,martens2022}. En este sentido, las
interacciones entre creadores y espectadores a través de comentarios
emisiones en directo, \emph{replys} (respuestas a otros vídeos),
\emph{duets} (grabar vídeos entre dos personas desde dos dispositivos) o
\emph{Stitch} (pegar vídeos) y otros tipos de interacciones son un
espacio todavía por analizar.

La generalización de programas generativos de Inteligencia Artificial
puede cambiar todo en pocos meses, pero mientras tanto, \emph{Booktok}
es el espacio más dinámico para la promoción de la lectura en Internet,
con todas su carencias y éxitos, y lo seguirá siendo hasta que no
aparezca una herramienta con mayor éxito entre el gran público.

\section{Agradecimientos}\label{sec-agradecimientos}

Esta investigación está dentro de la Red de Investigación en Docencia
Universitaria \emph{Multimodalidad y alfabetización transmedia en
	asignaturas de Didáctica de la Lengua y la Literatura en Educación
	Infantil (5741)}, de la Universidad de Alicante.



\printbibliography\label{sec-bib}
%conceptualization,datacuration,formalanalysis,funding,investigation,methodology,projadm,resources,software,supervision,validation,visualization,writing,review
\begin{contributors}[sec-contributors]
\authorcontribution{José Rovira Collado}[conceptualization,supervision,writing]
\authorcontribution{Francisco Antonio Martínez Carratalá}[methodology,formalanalysis]
\authorcontribution{Sebastián Miras}[datacuration,validation,writing,review]
\end{contributors}
\end{document}
