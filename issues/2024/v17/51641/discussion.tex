\section{Discusión y conclusiones}\label{sec-discussion}

Los resultados de este análisis multimodal tenían como objetivo
describir los principales recursos discursivos multimodales de los
\emph{booktok} en lengua española a través de un análisis de contenido,
basado tanto en la imagen presentada en la grabación y la interacción
personal como en los textos y elementos sonoros y voces utilizados. Una
de las cuestiones clave se centra en la inmediatez y cercanía producida
por el hecho de que la mayoría de las grabaciones se realiza a través
del móvil y transforma la perspectiva de los \emph{booktoks} y que su
breve duración permite un consumo rápido al ampliarse su contenido a
través de diferentes elementos para la edición. Incluso hay vídeos donde
se pierde la voz del protagonista por una música de acompañamiento y un
mensaje sobreimpreso en la pantalla. Otro aspecto esencial es que el
libro y la literatura reclaman su espacio en esta serie de dinámicas,
siendo el protagonista de muchos de ellos, aunque también se encuentran
\emph{bookinfluencers} centrados en personas concretas. En este sentido,
se señala que se diferencian aquellos vídeos que optan por una
perspectiva subjetiva, donde el foco central está en los libros o en las
librerías, como los dos primeros ejemplos 1. \emph{¿Qué estás leyendo?}
(What are you reading?) y 2. \emph{Fiesta de seguidores-lectores}
(Readers Follow Party), con respecto a las demás, que son más
convencionales, y algunas son variaciones directas de los
\emph{booktubers}. En líneas generales, se observa que las opciones
empleadas en estas dinámicas de TikTok para la promoción de la lectura
son más atractivas y producen una mayor implicación con la audiencia al
reducir la distancia con su audiencia y con una gramática visual \cite{kress2006} más persuasiva al reclamar con diferentes estrategias
su implicación y un mayor tono humorístico. Ese dinamismo y discurso
multimodal provoca que, en muchas de estas formas de promoción, el
contenido reflexivo se relega a un segundo plano y la creación busca una
integración de recursos (textuales, musicales y visuales) que priman una
presentación atractiva y efectiva para las personas que siguen dicha
cuenta, más que un espacio de divulgación o de encuentro. Al no
necesitar la imagen de una persona en estas dinámicas, sino de un libro,
son además propuestas didácticas que se pueden aprovechar con jóvenes
lectores. Además, estas nuevas dinámicas buscan principalmente fomentar
la interacción entre los usuarios y conseguir más seguidores, siendo
virales como prácticas iniciales de Twitter como \emph{Follow}
\emph{Friday} (\#FF) \cite{cui2012}.

Respondiendo al segundo objetivo de la investigación, esta permite hacer
un recorrido sobre las plataformas digitales de promoción lectora y la
evolución de las herramientas en este siglo. Entre 2001 y 2017 se usaron
distintos foros, como los usados por Laura Gallego \cite{lluch2012}
para promocionar sus lecturas y conocer la opinión de los lectores.
Desde 2006 en adelante aparecen los \emph{blogs} literarios, con
muchísimas variedades \cite{garcia2014}. Entre 2011
y 2018 fue el momento de los \emph{Booktubers}, transformando las
recomendaciones hacia productos audiovisuales \cite{tomasena2021}. Desde
2016 hasta la actualidad es el momento de \emph{Bookstragram} \cite{quilescabrera2020} y también en 2020 comienza la etapa de los
\emph{Booktok}, conviviendo ambas dinámicas, muchas veces confluyendo en
el producto audiovisual final. En esta investigación no se ha centrado
en su comparación. Aunque en Instagram todavía se encuentran
publicaciones centradas en la fotografía de libros y breves comentarios,
prácticas que están entre los blogs y el \emph{microblogging}, son las
\emph{stories} y los \emph{reels} los espacios de mayor desarrollo y
audiencia. También Facebook incorpora la sección de \emph{Watch} para
incorporar vídeos de distinta extensión y Twitter permite insertar
vídeos, después de descartar otras opciones audiovisuales. En cada uno
se encuentran distintos contenidos, audiencias y cada algoritmo
proporciona el acceso a distintos vídeos. Pero son las dinámicas
audiovisuales de TikTok, incorporadas también a Instagram, las que se
convierten en virales en la actualidad. Y siempre es el móvil el
dispositivo de acceso y creación de estos breves vídeos. Además, cabe
destacar el cambio generacional. Mientras que los \emph{Booktubers} casi
han desaparecido, porque las grandes celebridades han crecido o han
tenido que migrar a las nuevas plataformas, los \emph{Booktokers} están
en la cresta de la ola. Brevedad y uso del móvil como características
principales de los \emph{booktoks} son las principales diferencias con
modelos anteriores, respondiendo al último objetivo planteado.

Siguiendo la máxima de Marshall Mcluhan de ``el medio es el mensaje'' se
asume que se está ante una categoría diferente, mediada siempre por el
móvil como dispositivo para crear y acceder a estos vídeos. Es cierto
que se debe asumir la superficialidad y rapidez de estos contenidos, que
se consumen en masa, uno detrás de otro y pocas veces dejan rastro en la
memoria. Pero en el caso de los \emph{booktoks} se aprecia su enorme
importancia en el mercado editorial y en el sistema de recomendación de
libros entre iguales, conformando un nuevo canon de lecturas accesible
en Internet \cite{lluch2021}. Si hace pocos años era \emph{Goodreads} la
plataforma preferida para acceder a sugerencias de otras personas
\cite{garcia-roca2020}, TikTok ya ha ocupado ese espacio rápidamente como
bien señalan editoriales y ferias del libro \cite{penguin2020}.

En Internet va todo muy rápido y ya hoy puede que sea \emph{ChatGPT} y
otras inteligencias artificiales las que ocupen ese espacio. Pero ahora
las y los lectores más jóvenes están siempre conectados a estos vídeos y
hay que asumir su importancia. Son obvias las diferencias de la mayoría
de estos breves vídeos frente a las reseñas escritas (blogs) y los
vídeos largos (\emph{booktubers}) que profundizan en los temas
literarios. Pero estas mismas críticas de superficialidad o falta de
profesionalidad los recibieron aquellos espacios en su momento de auge,
creados por jóvenes en su momento. Aunque no se ha focalizado en el
presente análisis en este aspecto, cabe señalar la juventud de muchos de
los protagonistas, tanto creadores de contenido como espectadores. La
mayoría personas que están todavía formándose como lectores competentes.
También se quiere mencionar el predominio del libro en papel como objeto
y el indiscutible protagonismo de las mujeres, ya que las
\#\emph{booktokers} son chicas cada vez más jóvenes.

La creación de vídeos y el análisis de otras producciones puede ayudar
al desarrollo de las competencias digitales \cite{allué2023} y los
modelos anteriores ya demostraron su utilidad pedagógica \cite{paladines_Paredes_Aliagas_Marín_2021}, así como los actuales \emph{Booktoks} ofrecen nuevas
posibilidades didácticas \cite{dezuanni2021,acevedo2022}.

Cabe resaltar la transmedialidad y la interacción con otras plataformas,
como \emph{Goodreads}, donde la reseña literaria es más profunda
\cite{roviracollado2021}(Rovira-Collado, 2021) o \emph{Wattpad} \cite{garciaroca2019},
como espacio de creación y remezcla literaria. Si los \#\emph{booktook}
interactúan con estos espacios, las posibilidades de reflexión lectora
son mayores.

Se asumen las limitaciones de este estudio, con un corpus concreto de
vídeos y con unas características asignadas principalmente por el
algoritmo de la plataforma, pero se considera que es una completa
descripción de las dinámicas de \emph{booktoks} en español que completa
otras investigaciones \cite{guinez-cabara2022}. Tampoco
se ha realizado un análisis especifico de las obras literarias citadas
en cada vídeo. Sería interesante identificar cuáles son las tendencias
literarias que promueven los \emph{booktoks}, aunque se augura que serán
los superventas de la literatura juvenil de cada momento. Queda por
confirmar cuánto influyen estos vídeos en las ventas de cada género
\cite{merga2021,martens2022}. En este sentido, las
interacciones entre creadores y espectadores a través de comentarios
emisiones en directo, \emph{replys} (respuestas a otros vídeos),
\emph{duets} (grabar vídeos entre dos personas desde dos dispositivos) o
\emph{Stitch} (pegar vídeos) y otros tipos de interacciones son un
espacio todavía por analizar.

La generalización de programas generativos de Inteligencia Artificial
puede cambiar todo en pocos meses, pero mientras tanto, \emph{Booktok}
es el espacio más dinámico para la promoción de la lectura en Internet,
con todas su carencias y éxitos, y lo seguirá siendo hasta que no
aparezca una herramienta con mayor éxito entre el gran público.

\section{Agradecimientos}\label{sec-agradecimientos}

Esta investigación está dentro de la Red de Investigación en Docencia
Universitaria \emph{Multimodalidad y alfabetización transmedia en
	asignaturas de Didáctica de la Lengua y la Literatura en Educación
	Infantil (5741)}, de la Universidad de Alicante.
