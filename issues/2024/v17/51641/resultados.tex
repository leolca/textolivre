\section{Resultados}\label{sec-resultados}

En la siguiente tabla se recogen los principales datos de los 10 vídeos
analizados para posteriormente proceder a un análisis multimodal
centrado principalmente en el vídeo, y en menor medida, en el uso del
texto. Se recogen solamente tres variables: \emph{Duración}, \emph{Me
	gusta} y \emph{Comentarios} para confirmar que son vídeos mucho más
breves y destacar que en nuestro análisis no se prioriza en la audiencia
y en la interacción.

\begin{table}[htbp]
\centering
\begin{threeparttable}
\caption{Análisis descriptivo del vídeo}
\label{tab-03}
\begin{tabular}{p{3cm} l l l l}
\toprule
Dinámica & Perfil & Duración & Me gusta & Comentarios \\
\midrule
1. What are you reading? & \seqsplit{@claudiacp\_books} & 05 seg. & 56 & 1 \\
2. Readers follow party & \seqsplit{@laslecturasdeloli} & 15 seg. & 3047 & 717 \\
3. Bibliotecas & \href{https://www.tiktok.com/@bibliotecaugena/video/6971466004585057542}{\seqsplit{@bibliotecaugena}} & 13 seg. & 2365 & 14 \\
4. Tipos de Booktoker & \href{https://www.tiktok.com/@pandi.book/video/7199333140748438790}{\seqsplit{@pandi.book}} & 1min. 54 seg. & 83.4K & 973 \\
5. Promoción lectora & \href{https://www.tiktok.com/@iriayselene/video/7211269180484750598}{\seqsplit{@iriayselene}} & 10 seg. & 21.4K & 194 \\
6. Editoras & \href{https://www.tiktok.com/@vreditorasya/video/7223140662022114565}{\seqsplit{@vreditorasya}} & 08 seg. & 243.8K & 487 \\
7. Bookinfluencer & \href{https://www.tiktok.com/@patriciafedz/video/7222221715819072774}{\seqsplit{@patriciafedz}} & 2 min, 15 seg. & 23K & 150 \\
8. Día del libro & \href{https://www.tiktok.com/@javierruescas/video/7224553867566910747}{\seqsplit{@javierruescas}} & 1 min. 35 seg. & 30.4K & 86 \\
9. Videopoemas & \seqsplit{@marinalcuadrado} & 1 min. 9 seg. & 12.5K & 719 \\
10. Así dijo\ldots & \href{https://www.tiktok.com/@solo.palabrqs/video/7190817595401030918}\seqsplit{@solo.palabrqs} & 20 seg. & 138.2K & 181 \\
\bottomrule
\end{tabular}
\source{Elaboración propia.}
\end{threeparttable}
\end{table}

Como se puede observar, muchos de estos vídeos tienen una enorme
audiencia, con miles de ``Me gusta''. Solamente la primera categoría
tiene muchos menos, pero se ha incluido este vídeo porque es el que más
claramente ejemplifica una nueva dinámica. El análisis de las diez
dinámicas analizadas en la plataforma TikTok permite una primera
diferenciación en la comunicación visual que se realiza entre aquellas
que adoptan un punto de vista mediado frente a las que no. Este punto de
vista mediado implica que las acciones que ven los espectadores son
vivenciadas como si fueran la persona que ha creado el vídeo (mediada:
inferida) o bien desde dentro de la imagen (mediada: inscrita) cuando se
observa alguna de las partes de su cuerpo (e inclusive desde sus
espaldas). Las dos dinámicas que emplean este tipo de estrategias de
comunicación visual son 1. \emph{¿Qué estás leyendo? (What are you
	reading?)} y 2. \emph{Fiesta de seguidores-lectores (Readers Follow
	Party)}, en la que se aprecian diferentes matices con algunos ejemplos
concretos. Estas dinámicas buscan la mayor intervención del espectador
al ubicarlo al mismo nivel experiencial que la persona que crea el
contenido. Y el protagonista de ambas son los libros o las estanterías
de los usuarios, siendo vídeos de corta duración acompañados de una
pieza de audio concreta que pone título a estas dinámicas en inglés.
Además, ambos son muy breves e indican nuevas dinámicas hiperbreves de
esta red social.

Siguiendo un ejemplo de la dinámica 1. \emph{¿Qué estás leyendo?} (de la
usuaria @claudiacp\_books) la perspectiva mediada e inscrita persigue
generar un efecto de suspense (inicialmente solo se muestra la parte
inversa al lomo del libro, donde solamente se ven unas páginas y no se
puede identificar la obra) y solo aparece la mano de la usuaria
sosteniéndolo unos segundos hasta revelarlo y generando una metonimia
visual \cite{moya-Guijarro}. Este tipo de dinámicas reducen la
distancia social entre el espectador y la autora del vídeo, para generar
una sensación de intimidad y adopta un plano cenital donde tanto la
persona creadora del contenido como el espectador se ubican en una
relación de superioridad frente al objeto-libro representado. En estos
vídeos, ambas personas, \emph{booktoker} y espectadores comparten la
misma perspectiva respecto al vídeo. En este caso, esta metonimia visual
que acompaña al giro para presentar la cubierta del libro va acompañada
de un audio específico, titulado: \emph{What this person reading right
	now?} Este audio dura solamente cinco segundos e inclusive se pueden
encontrar en la plataforma y usarlo en nuestras creaciones
\url{https://www.tiktok.com/music/Whats-this-person-reading-right-now-6904077601871170309}
y cuenta con más de cien mil vídeos que usan este misma dinámica y
sonido, que lo convierten en tendencia.

El ejemplo de la dinámica 2. \emph{Fiesta de seguidores-lectores} (de la
usuaria @laslecturasdelol) se basa en la creación de un solapamiento
entre la perspectiva del usuario y del espectador, dejando que el plano
tenga una mayor distancia para mostrar la panorámica de los espacios
lectores que recorre el usuario como sus estanterías. La implicación del
espectador también opta por una perspectiva mediada, pero en este caso
inferida, dado que ambos contemplan la escena en primera persona. Se
opta por una angulación donde prima la horizontalidad (y el aspecto
frontal) para que el espectador se sienta partícipe de la misma forma
que el usuario. La transcripción del vídeo de 15 segundos, que aparece
también en el vídeo es la siguiente:
\begin{quote}
		Fiesta de seguidores: Si sos \emph{booktoker} y tenés menos de cinco mil
		seguidores, seguime y así te puedo seguir. Deja un comentario con tu
		número de seguidores actual y a cada persona que deje un comentario
		seguila, No te olvides de compartir el link para que más booktokers lo
		vean. ¡Que lluevan seguidores!
		
		Fuente:
		\url{https://www.tiktok.com/@laslecturasdeloli/video/6972178596152511749}.
\end{quote}

Este vídeo es una adaptación al español de la dinámica \emph{Reader Follow Party}\footnote{\url{https://www.tiktok.com/tag/readerfollowparty}}
que se ha usado en más de diez mil ocasiones, aunque generalmente por
público anglófono. Por ejemplo, la usuaria @maddiesreadss ofrece un
vídeo con subtítulos automáticos de este audio, donde se observa que la
interacción entre cámara y elementos es idéntica al ejemplo seleccionado
en español.

\begin{quote}
	Reader Follow Party. If You're not already, follow me and I'll follow
	you back. Then, go to the comments and tell us what book you're
	currently reading. When someone likes your comment follow them and
	they're follow you back. Yay for more books friends!
	
	Fuente:
	\url{https://www.tiktok.com/@maddiesreadss/video/7236953611547184386}.
\end{quote}


El análisis se inicia con ambos vídeos porque son dos de las dinámicas
más novedosas de \#\emph{Booktok} y el protagonista central de ambas
prácticas son los libros. Aunque la dinámica con las que se identifican
estos vídeos son de usuarias hispanohablantes, las etiquetas y audios
originales son en inglés, con mucha mayor proyección, siendo claras
tendencias en el ámbito de TikTok.

Frente a estas dos dinámicas que adoptan una perspectiva subjetiva en su
grabación y montaje, se encuentran aquellas que optan por una
perspectiva no mediada en la que la implicación visual para el
espectador se compensa con otras estrategias como el contacto visual
directo o la creación de dinámicas con un componente humorístico. Entre
los casos analizados, se destaca un ejemplo por parte de una biblioteca
pública (@bibliotecaugena) 3. \emph{Bibliotecas} donde el efecto del
vídeo es ubicar en tercera persona al espectador y combinar las opciones
de focalización desde la alternancia de las miradas de oferta (aquella
que se dirigen por la persona protagonista, la bibliotecaria, hacia los
objetos y elementos que quiere que el espectador se fije) y las de
demanda donde prima el contacto visual (con las que interpela al
lector-espectador a seguir la dinámica). En este sentido, el plano se
aleja y se contempla a la persona con una mayor distancia respecto al
espectador (dado que prima la sucesión breve de acciones) y se ubica
desde el uso de ángulos frontales para que se facilite la sensación de
intervención del usuario. También tiene una duración breve, 13 segundos
para presentar varios libros. Este perfil tiene muchos vídeos similares,
con la bibliotecaria como protagonista y contenidos sobre la función e
importancia de las bibliotecas y es un claro ejemplo para la mediación
lectora en redes sociales.

El componente humorístico también forma parte de montajes como la
presentación de la dinámica 4. \emph{Tipos de Booktoker} (de
@pandi.book), que se basa de nuevo en la alternancia entre el contacto
visual de la joven protagonista del vídeo con el espectador (para
acrecentar la sensación de cercanía) y las miradas de demanda (ahora,
invitada) cuando se reproduce (simula, en este caso) un diálogo con otra
persona para darle réplica. En esos intercambios, cuando se dirige a
otra persona, una voz en \emph{off}, la posición corporal se angula
horizontalmente para crear en ese diálogo una mayor sensación de
desapego con el espectador (para compartir la reacción de la usuaria que
aparece en el vídeo). Cuando ese breve diálogo (en forma de reproche en
ocasiones) finaliza, se busca nuevamente la cercanía con la protagonista
al emplear preferentemente primeros planos (alternando con planos
americanos) para generar una situación de intimidad, como también ocurre
cuando se ubica en una relación neutral y frontal con el espectador.
Además, este vídeo muestra los tipos de nuevos lectores que se
identifican con estas dinámicas.

Otro ejemplo recurrente en la plataforma es la creación de un efecto
paródico a partir de la contradicción \cite{bateman2014} entre los modos
visuales y textuales. Por ejemplo, en esta presentación de su nuevo
libro por parte de sus autoras de LIJ en el vídeo 5. \emph{Promoción}
\emph{lectora} (de @iriayselene), se observa cómo la información textual
pasa a un segundo plano, sobreimpresionadas en la pantalla, mientras que
las protagonistas vocalizan la letra de la canción que suena de fondo
(\emph{Naughty} - Alisha Weir \& The Cast of Roald
Dahl\textquotesingle s Matilda The Musical). El texto dice y la
interacción con los libros es la siguiente:

\begin{quote}
	Cartela 1 Verde. Cuando nos criticaron un montón por hacer un rettelling
	de \emph{Ana de las Tejas Verdes} con representación LGTB+. (Muestran su
	libro \emph{Anne sin filtros})
	
	Cartela 2 Ocre. Así que hemos hecho lo mismo, pero con El Mago de Oz.
	(Presentan nuevo libro Seremos huracán)
\end{quote}


En este caso Iria y Selene ya eran famosas \emph{booktubers} y autoras
de literatura juvenil que han empezado una nueva aventura en
\emph{booktok}. Como se ha comentado anteriormente, en la plataforma se
opta con asiduidad al reemplazo de la voz de los usuarios por dejar que
sea la música la que complete el mensaje audiovisual, siendo este audio
del musical \emph{Matilda} muy habitual para presentar bromas o actos
traviesos.

Otro claro ejemplo de dinámicas paródicas es el vídeo de @vreditorasya
6. \emph{Editoras} (de @vreditorasya) donde se opta por no aparecer su
cabeza en la pantalla y se da respuesta a la pregunta sobreimpresionada
(``Cuando dice que me quiere regalar un libro, pero no sabe cuál'') con
la intención de generar un \emph{sketch} humorístico en la que poner de
manifiesto el interés desmedido por la lectura, porque luego presenta un
papel con una lista muy larga de libros deseados (\emph{Wish List}),
aunque en este caso, apenas aparecen libros y no se leen los títulos,
pero sí se destaca el gusto por la lectura. Además, este vídeo aprovecha
una tendencia humorística ``Ding dong'', donde después de oírse un ruido
sordo, una mujer imita el sonido de una campanilla para presentarse.
Muchas veces este sonido se usa de forma de incitación sexual, al usarse
para presentar a una persona semidesnuda se usa de forma sexista, pero
este \emph{booktok} lo aprovecha para presentar el interés por los
libros de la protagonista.
\url{https://www.tiktok.com/music/suono-originale-6830486689857784582}

Frente a las dinámicas anteriores (tanto de perspectiva mediada como no
mediada), en la plataforma también se reproducen dinámicas heredadas de
YouTube en el modo de presentar la información. En estos casos, las
personas creadoras de contenido dan relevancia a su figura como
``presentadoras de contenido''. En primer lugar, sobre contenidos como
``influencers'' \cite{establés2019}.
\emph{Bookinfluencer} (de @patriciafedz) en la que comparten las
anécdotas o vivencias de las personas usuarias, como es el caso de
@patriciafedz, donde se opta por primeros planos y el contacto visual
directo para compartir esa intimidad del relato con el espectador. Esta
tipología incluso explota la imagen de la protagonista, una mujer joven
y atractiva, como en otras redes sociales \cite{calvo2018,dezuanni2022}. Este vídeo tiene más de doscientas
mil visualizaciones y veinte mil me gusta, con varios comentarios que
incluso señalan algún error que realiza la protagonista al citar obras
literarias. Además, Patricia Fernández tiene más de trescientos mil
seguidores y una producción constante de vídeos, por lo que se puede
destacar como una ``\emph{bookinfluencer}''. En este tipo de vídeos, la
ubicación de la angulación vuelve a situarse levemente contrapicada,
dejando al espectador de nuevo en inferioridad frente a la persona que
narra dicha anécdota. Esta circunstancia también indica cómo el cambio
de grabación (principalmente el móvil) en una plataforma como TikTok
frente a Youtube donde la cámara suele ubicarse en una posición más
elevada en la pantalla del ordenador (siendo el espectador el que se
encuentra en superioridad, habitualmente, frente al creador del
contenido). Además, estos vídeos tienen mayor duración (más de dos
minutos) y edición, intercalando otros vídeos en el montaje. En este
caso, no se opta por efectos sonoros, y lo importante es la explicación
de la protagonista.

Además de vivencias relacionadas con experiencias alrededor de la
literatura, se han recogido vídeos sobre contenidos para compartir el
conocimiento sobre detalles relacionados con la lectura, como el de
@javirruescas 8. \emph{Día del libro} sobre el origen de esta efeméride.
Javier Ruescas es también un famoso bloguero, \emph{booktuber} y autor
de literatura juvenil \cite{ruescas2012} que ha sabido adaptarse al nuevo
medio, también con una gran audiencia. De nuevo, la importancia de estas
dinámicas es ubicarse frente al espectador ofreciéndole un plano cercano
que disminuya la distancia social, el contacto visual directo y una
angulación de neutralidad donde no ubicarse en superioridad con el
espectador. Otro efecto curioso es la sobreimpresión de los subtítulos
que permite simultáneamente leer la explicación. Como en el caso
anterior, no hay audio de acompañamiento, solamente la voz del
\emph{booktoker} y este vídeo como el precedente, también tiene mayor
duración, se ha grabado desde un ordenador sentado, y luego se ha subido
a la plataforma, pensada para las reproducciones en móviles.

En esta categoría también hay otros ejemplos que optan por crear
contenidos como recitados o lecturas en voz alta como la dinámica 9.
\emph{Videopoemas}. Pese a que pueden darse diferentes perspectivas, se
pone en relieve con el ejemplo de @marinalcuadrado cómo se diluye la
sensación de cercanía y la angulación vertical (contrapicada) juega para
ubicar a la persona que recita en superioridad respecto al espectador
que, además, encuentra la empatía de la mirada de la persona que recita
al emplear una focalización de contacto visual de demanda directo
(reclamando nuestra atención). Este formato con un protagonista
recitando también tiene una audiencia importante, con más ciento
cincuenta mil visualizaciones. Es un videopoema más cercano y con una
protagonista que hace atractiva la poesía. Las dinámicas tradicionales
de videopoemas en Youtube se basaban en montajes de diapositivas con el
recitado de fondo, como se verá a continuación.

Finalmente, con una estrategia más distante en la comunicación visual de
los protagonistas, también hay contenidos que optan por la desaparición
del creador para emplear una imagen que sirva de apoyo a un pasaje, cita
o fragmento de un autor, en este caso Mario Benedetti. La dinámica 10.
\emph{Así dijo\ldots{}} y se basa en reproducir citas y versos de
autores famosos. Se incluye esta categoría porque se considera que tiene
claras posibilidades didácticas para la enseñanza de la literatura y
confirma que grandes autores también tienen presencia en esta red.
Aunque también tienen mucha audiencia, estos vídeos muchas veces
atribuyen falsamente poemas o textos a autores, por lo que es necesario
usarlos con atención \cite{roviracollado-hernandez-ortega-2023}. Este tipo
de montajes intentan plasmar dinamismo mediante la sucesión de imágenes
acompañadas de un fragmento musical que amplifica de manera emotiva el
mensaje verbal escogido por el usuario @solo.palabrqs. Así, se trata de
un tipo de dinámica más impersonal y en la que se ofrece un videoclip o
sucesión de imágenes con música o recitado, más que la interacción con
los espectadores. En este caso se citan unas palabras concretas, pero
hay muchos videopoemas que aprovechan las voces originales de estos
autores.
