\section{Introducción}\label{sec-introducción}

En la evolución de los espacios de promoción y mediación lectora en
Internet se encuentran los epitextos virtuales de lectura \cite{lluch2015} que aportan información sobre los
libros y la literatura, animando a leer uno u otro texto literario. El
último espacio de desarrollo de este ámbito son los \emph{booktoks},
breves vídeos sobre lectura y literatura en TikTok, una plataforma o red
social de vídeos para ser disfrutados y creados principalmente a través
del móvil. Estos epitextos son una evolución de modelos anteriores, como
los basados en el texto escrito, como los blogs o los foros literarios o
los basados en vídeos, como los \emph{booktubers} \cite{sorensen2014}
o fotografías y vídeos, como los \emph{bookstagramers} \cite{sánchezgarcía2020}.


Todos estos espacios permiten la \emph{lectura social} \cite{cordon2013}, donde cualquier persona puede compartir a través
de Internet y las redes sociales su opinión sobre sus lecturas e
interactuar con otras personas para hablar de literatura. El fenómeno
global de los \emph{booktubers} como espacio antecedente directo para la
animación lectora, y sus características en lengua española, ha
despertado gran interés en los últimos años \cite{vizcaino2019,tomasena2021}, incluso con
distintos análisis de sus posibilidades educativas \cite{roviracollado2016}. Ya no se escribe sobre libros, si no que se habla, se comenta o
se discute oralmente en vídeos de distinta extensión.

Una de las diferencias principales de estos vídeos con los
\emph{booktubers} es el móvil como espacio natural para su desarrollo,
pudiendo relacionar su uso con el \emph{Mobile} \emph{Learning} \cite{unesco2013}. Las redes sociales, desde su aparición, son espacio para el
desarrollo del aprendizaje informal o invisible, ya que están
constantemente ofreciendo información que se puede convertir en
conocimiento \cite{piscitelli2010}. Después de la
transformación digital provocada por la pandemia del covid-19 se deben
incorporar muchos de estos contenidos y estrategias de aprendizaje
informal a la hora de hablar de aprendizaje digital. También se deben
entender las estrategias discursivas como dinámicas de comunicación para
realizar el mensaje más comprensible y atractivo para los oyentes \cite{coll2021}. En estos espacios, estas estrategias deben aprovechar
todas las posibilidades de la multimodalidad (Martín Menéndez, 2020).
Posteriormente, se plantearán los elementos del análisis multimodal en
el presente estudio.

Instagram también ofrece espacios para los libros, y los
\emph{bookstragramer} publican tanto fotografías como vídeos
(\emph{reels} o \emph{stories}) para hablar de lecturas. Pero son los
\emph{booktoks} los que están adquiriendo mayor atención en este
momento. Actualmente, se encuentran distintas noticias e informes que
destacan el creciente uso que están adquiriendo estos vídeos en el mundo
de la literatura y la producción editorial como elementos de publicidad
que influyen en la elección de lecturas de sus espectadores \cite{nielsen2021}. A principios de esta década, la \textcite{penguin2020}
recomendaba los perfiles de \emph{booktokers} que había que seguir.
Solamente falta acercarse a los principales informes de usos de redes
sociales \cite{IABSpain2022} para comprobar que ambas redes tienen el mayor
crecimiento y audiencia en estos momentos.

Estos espacios generan nuevas figuras como los \emph{influencers}, las
personas con mucho éxito en redes sociales que se convierten en
celebridades \cite{abidin2020}. También ha aparecido el concepto de
``creadores de contenido'', para referirse a las personas que publican
vídeos elaborados, de cualquier temática, de forma periódica, ya que la
grabación y edición de estos vídeos supone en muchos casos una
especialización. Su visibilidad \cite{cordon2019} y
número de visionados permite que sus perfiles reciban una retribución
económica atendiendo al número de reacciones y visitas que tienen sus
vídeos por parte de la plataforma. En la sección de análisis de este
estudio se incluyen algunos de estos vídeos. En el ámbito de la
promoción literaria con esta plataforma se centran distintas
investigaciones sobre este fenómeno, principalmente, en el ámbito
anglosajón \cite{merga2021,martens2022} que detectan
el interés por esta nueva vía de comunicación literaria. \textcite{sanztejeda} han realizado una completa revisión sistemática sobre los
métodos y resultados de investigación sobre lectura en ambas plataformas
donde señalan las diferencias y novedades en su uso, y también los
elementos comunes con anteriores dinámicas.

Aunque TikTok e Instagram puedan parecer herramientas solamente para el
ocio, estas ofrecen múltiples aplicaciones educativas para promover
tanto aprendizajes formales como informales. Como herramientas
audiovisuales, estas ofrecen mensajes multimodales \cite{newlondongroup1996} que se analizarán detenidamente en el cuerpo de la presente
investigación. Muchos vídeos se replican en ambas plataformas, pero la
investigación se concreta exclusivamente en TikTok.

El objetivo principal de este estudio se centra en describir los
principales usos de \emph{booktok} en lengua española a través de un
análisis de contenido del discurso multimodal \cite{gomez2016}
empleado por las personas creadoras de los diferentes contenidos para
hacer los vídeos más atractivos.

A partir de este objetivo se plantean dos secundarios que se desprenden
del análisis.

El primer objetivo secundario es plantear la evolución de los
\emph{booktoks} desde modelos anteriores.

El último objetivo secundario es identificar algunas de las diferencias
de los \emph{booktoks} con los modelos anteriores.