\documentclass[portuguese]{textolivre}

% metadata
\journalname{Texto Livre}
\thevolume{17}
%\thenumber{1} % old template
\theyear{2024}
\receiveddate{\DTMdisplaydate{2024}{5}{8}{-1}}
\accepteddate{\DTMdisplaydate{2024}{7}{28}{-1}}
\publisheddate{\today}
\corrauthor{Sandra Miranda}
\articledoi{10.1590/1983-3652.2024.52551}
%\articleid{NNNN} % if the article ID is not the last 5 numbers of its DOI, provide it using \articleid{} commmand 
% list of available sesscions in the journal: articles, dossier, reports, essays, reviews, interviews, editorial
\articlesessionname{articles}
\runningauthor{Cordeiro e Miranda}
%\editorname{Leonardo Araújo} % old template
\sectioneditorname{Daniervelin Pereira}
\layouteditorname{João Mesquita}

\title{O efeito da Responsabilidade Social Corporativa (RSC) na construção da imagem de marca: o caso Disney}
\othertitle{The effect of Corporate Social Responsibility (CSR) in building brand image: the Disney case}

\author[1]{Rita Cordeiro~\orcid{0000-0001-8713-6370}\thanks{Email: \href{mailto:ritacordeironl@gmail.com}{ritacordeironl@gmail.com}}}
\author[2]{Sandra Miranda~\orcid{0000-0002-5544-5942}\thanks{Email: \href{mailto:smiranda@escs.ipl.pt}{smiranda@escs.ipl.pt}}}
\affil[1]{Escola Superior de Comunicação Social, IPL, Lisboa, Portugal.}
\affil[2]{Escola Superior de Comunicação Social, IPL, LIACOM, Lisboa, Portugal.}

\addbibresource{article.bib}

\begin{document}
\maketitle
\begin{polyabstract}
\begin{abstract}
Neste artigo é investigado o papel da Responsabilidade Social Corporativa (RSC) na construção da imagem de marca, utilizando a Disney como caso de estudo. Foi adotada uma abordagem mista, combinando métodos qualitativos e quantitativos. No método qualitativo, foram analisados e comparados dois relatórios corporativos da The Walt Disney Company, datados de 2008 e 2022. No método quantitativo, foi aplicado um questionário a membros das Gerações X e Y (N=172). A análise comparativa dos relatórios revelou mudanças significativas na abordagem da Disney à RSC, com destaque para a priorização da diversidade, equidade e inclusão (DEI), demonstrando também a fidelidade da empresa aos seus valores. Quanto aos resultados do inquérito por questionário, não foram identificadas diferenças estatisticamente significativas nas perspetivas das gerações. No entanto, confirmou-se a existência de uma relação positiva entre RSC e imagem de marca. Conclui-se que o investimento em RSC é crucial para construir e manter uma imagem de marca sólida, destacando a crescente importância da RSC como uma ferramenta estratégica para as empresas.

\keywords{RSC \sep Sustentabilidade \sep Tendências \sep Valores \sep Impacto}
\end{abstract}

\begin{english}
\begin{abstract}
This article investigates the role of Corporate Social Responsibility (CSR) in building brand image, using Disney as a case study. A mixed-method approach was adopted, combining qualitative and quantitative methods. In the qualitative method, two corporate reports from The Walt Disney Company, dated 2008 and 2022, were analyzed and compared. In the quantitative method, a questionnaire was administered to members of Generations X and Y (N=172). The comparative analysis of the reports revealed significant changes in Disney's approach to CSR, particularly emphasizing diversity, equity, and inclusion (DEI), also demonstrating the company's commitment to its values. Regarding the questionnaire survey results, no statistically significant differences were identified in the perspectives of the generations. However, a positive relationship between CSR and brand image was confirmed. It is concluded that investing in CSR is crucial for building and maintaining a strong brand image, highlighting the growing importance of CSR as a strategic tool for companies.

\keywords{CSR \sep Sustainability \sep Trends \sep Values \sep Impact}
\end{abstract}
\end{english}
\end{polyabstract}

\section{Introdução}
A Responsabilidade Social Corporativa (RSC) é um tema incontornável no mundo empresarial. Refere-se ao compromisso da adoção de comportamentos socialmente responsáveis, considerando questões económicas, sociais e ambientais. Na última década, registou-se o aumento substancial na sua relevância, presumivelmente impulsionado por problemas emergentes como as alterações climáticas, as crises económicas, as injustiças sociais, ou a mais recente pandemia mundial de COVID-19 \cite{quezado_corporate_2022}. 

O conceito basilar de imagem de marca sugere que as ligações emocionais e a relevância simbólica atribuída a uma marca desempenham um papel vital na jornada do consumidor. Crê-se que a RSC contribua significativamente para a construção de uma melhor imagem de marca, uma vez que a corporação é continuamente associada a “bens maiores”, conferindo uma possível vantagem competitiva \cite{he_effect_2014,quezado_corporate_2022}.  Por isso, a RSC e a imagem de marca emergem como componentes fundamentais de uma estratégia corporativa. 

Dentro deste contexto, a presente investigação teve como principal objetivo analisar o impacto da RSC na imagem de marca, recorrendo ao caso Disney. Esta baseou-se exclusivamente na perceção de dois grupos geracionais (a Geração X e a Geração Y) com o propósito de perceber se consumidores de gerações distintas interpretam a RSC de uma forma diferente. 

Com um extenso catálogo de entretenimento infantil, a Disney é um ícone mundial do entretenimento e uma das primeiras marcas a estar presente na vida de um recém-nascido. Com o propósito de aperfeiçoar o seu papel social, a The Walt Disney Company e todas as suas respetivas marcas, desenvolveram a fundo diversos campos da sua Responsabilidade Social (RS). Consequentemente, a empresa adotou novos princípios orientadores como, por exemplo, a integração de histórias mais representativas no seu catálogo de entretenimento \cite{dubois_disney_2021}. Sendo que, o principal objetivo é aferir a influência da RSC na imagem de marca da Disney segundo a Geração X e a Geração Y. 

Para a sua efetivação, recorreu-se a um método misto, quantitativo, através da aplicação de um inquérito por questionário \textit{online} a elementos destes grupos geracionais e qualitativo, através da análise e comparação de dois relatórios corporativos com o propósito de caracterizar efetivamente as principais mudanças da RSC da The Walt Disney Company. Optou-se por aplicar este estudo à marca de entretenimento infantil Disney, preenchendo assim uma lacuna na literatura acadêmica.

\section{A responsabilidade social corporativa na atualidade}\label{sec-normas}
Apesar de ter sido um conceito concebido nos anos 50, através do livro \textit{Social responsibilities of the businessman} \cite{bowen_social_1953}, a RSC permanece como uma das áreas mais relevantes no mundo empresarial. 

Embora haja falta de consenso quanto à sua definição, a RSC fundamenta-se na convicção de que uma empresa deve desempenhar um papel positivo, indo além dos interesses dos seus \textit{stakeholders} (públicos), e considerando também o impacto nas comunidades em que opera \cite{camilleri_corporate_2017,stobierski_what_2021}.  A maioria das definições acadêmicas para este conceito estabelecem uma ligação direta entre ele e a filantropia corporativa \cite{carroll_carrolls_2016}. Segundo \textcite[p.~494]{kotler_marketing_2022}, o desenvolvimento desta dimensão filantrópica melhora substancialmente a imagem corporativa, a percepção da qualidade dos seus produtos e/ou serviços e a fidelização dos consumidores.   
Na perspetiva de gestão de marcas, a RSC posiciona-se como uma forte ferramenta de \textit{marketing} estratégico. \textcite{kotler_marketing_2022} pressupõem que muitas empresas adotem práticas socialmente responsáveis com o propósito de estabelecer um historial de ações éticas, tornando-as capazes de enfrentar possíveis críticas e controvérsias. \textcite{kell_five_2014} caracteriza-a como um mecanismo de defesa que se transformou num movimento mundial. 

Na última década, observou-se um crescente interesse pela RSC, possivelmente impulsionado por problemáticas mundiais como o aquecimento global, crises econômicas, injustiças sociais e, mais recentemente, a pandemia COVID-19 \cite{quezado_corporate_2022}. Além disso, acredita-se que os consumidores estejam mais propensos a adotar práticas de consumo responsáveis, mostrando-se mais sensíveis às consequências das suas escolhas de consumo \cite{faria_um_2019,quezado_corporate_2022}. Ademais, há uma crescente resistência ao consumo de marcas associadas a práticas irresponsáveis \cite{aftab_impact_2014}. À medida que os consumidores se tornam mais conscientes, procuram alinhar os seus valores pessoais com as corporações das quais escolhem consumir. 

Ao recensearmos os estudos que têm como caso de estudo a Walt Disney, verificamos que é unânime a constatação de que corporação está atenta ao desiderato da RSC e integra-a como valores e dimensões centrais da sua estratégia. \textcite{luiten_corporate_2020} verificou que a Disney está perfeitamente alinhada com os princípios da RSC e que para cumprir as suas metas é fundamental que a empresa pense cuidadosamente de que forma o consumidor é afetado e educado por esses valores e princípios. Entende que com o aumento do número de consumidores ambientalmente conscientes, que pesquisam e se informam antes de efetuar as suas compras, as empresas devem estar muito atentas à forma como abordam e comunicam a sua estratégia e política de RSE e ao modo como se alinham com os ODS (Objetivos de Desenvolvimento Sustentáveis).

Também \textcite{silva_junior_formas_2017}, ao estudar as formas de expressão animatorial na animação da Disney, coligiu que o programa narrativo revela nitidamente os valores da corporação, estando bem representantes os que se compaginam com a RSC. 


\subsection{Tipos de responsabilidade social corporativa}\label{sec-conduta}
Na esfera acadêmica, \textcite{carroll_carrolls_2016} destaca-se como um dos maiores defensores da RSC, sendo reconhecido como o autor de uma das suas principais definições. Entre as suas contribuições, destaca-se a teoria em que concebe a RSC como uma pirâmide, na qual delineia quatro dimensões: econômica, legal, ética e filantrópica. 

Por outro lado, \textcite{kotler_marketing_2022}, apresentam dois modelos de RSC alternativos aos propostos por \textcite{carroll_carrolls_2016}; o primeiro foca-se em três bases: a comunidade, o meio-ambiente e o mercado de atuação; enquanto o segundo assenta-se também ele em 3 bases: as pessoas, o planeta e a sua rentabilidade. Em ambos estes enfoques, assume-se que a RSC é composta por uma componente social, uma componente ambiental e uma componente financeira. 

Estes modelos são fundamentados na abordagem \textit{Triple Bottom Line} (TBL), elaborada por John Elkington em 1994, posteriormente reconhecida como os três P’s da Sustentabilidade (\textit{People, Planet, Profit} - Pessoas, Planeta e Lucro).  Esta abordagem oferece uma visão mais clara das áreas inerentes à RSC \cite{ksiezak_triple_2018}.


\subsubsection{Componente ambiental}\label{sec-fmt-manuscrito}
O modelo empresarial capitalista baseia-se na extração de matéria-prima, na sua manufaturação e posterior distribuição, originando um consumo excessivo e a acumulação de desperdício, revelando-se insustentável a longo prazo devido à disponibilidade limitada de recursos \cite{lambin_global_2020}.

A dimensão ambiental da RS assumiu uma nova importância entre o final da década de 60 e o início da década de 70 nos Estados Unidos, em resposta a vários desafios ambientais. Este despertar da consciência ecológica resultou no aprofundamento desta área de investigação. Entre as principais contribuições acadêmicas, destacam-se \textcite{dunlap_new_1978}, que alertam para a capacidade que a humanidade tem de perturbar o equilíbrio do Planeta Terra, sendo que o seu crescimento deve ser limitado, não lhe cabendo qualquer direito de apropriação sobre os recursos naturais. A preservação do meio ambiente tornou-se numa prioridade devido ao risco que a instabilidade dos ecossistemas representa para a humanidade \cite{berkowitz_most_2021}. No contexto empresarial, a escassez de recursos, os desastres naturais e outras adversidades ambientais traduzem-se em complicações, por vezes irreversíveis, podendo resultar em interrupções na produção e problemas nas cadeias de fornecimento, entre outras complicações. 

A adoção de uma economia mais sustentável acelerou a popularização de práticas corporativas ambientalmente responsáveis, como a obtenção de certificações ecológicas. As empresas procuram orientação de entidades competentes para apoiar e validar os seus esforços ambientais \cite{lambin_global_2020}. Apesar dos investimentos financeiros significativos em pesquisa, inovação e atualização, as práticas sustentáveis resultam na poupança monetária e na redução de riscos a longo-prazo.


\subsubsection{Componente social}\label{sec-formato}
A componente social do TBL abrange os comportamentos éticos, a criação de benefícios sociais e a filantropia, engloba todas as pessoas que afetam ou são afetadas pela atividade comercial de uma corporação \cite{ksiezak_triple_2018}. Esta dimensão comunica uma imagem mais ética e confiável das marcas e das suas corporações, transmitindo ainda uma sensação de proximidade aos seus consumidores \cite{kotler_marketing_2022}. Sem esta dimensão relacional, uma compra é exclusivamente a troca de uma transação monetária por um bem de consumo; é através dela que se fortalece o vínculo entre a corporação e os seus \textit{stakeholders} \cite{conley_como_2010}.  

As empresas não operam isoladamente da sociedade, por isso, esta componente baseia-se na promoção contínua do bem-estar social \cite{porter_competitive_2002}. Historicamente, esta dimensão pode ser vista como o elemento central na criação do conceito de RSC. Quando \textcite{bowen_social_1953}, desenvolveu a versão primordial do conceito de RSC, o autor aludiu à contribuição das corporações na ajuda a todas as comunidades afetadas significativamente pelas consequências da Segunda Guerra Mundial. 

A ética é um dos princípios fundamentais do TBL, abrangendo o tratamento ético do universo de \textit{stakeholders}. Esta componente pode ser subdividida em outras áreas como a conservação das condições e direitos laborais, direitos humanos, o envolvimento comunitário e promoção da diversidade, equidade e inclusão (DEI), abrangendo um conjunto de medidas de cariz externo e interno \cite{ksiezak_triple_2018}. No que diz respeito aos funcionários, visa-se a criação de um ambiente justo, seguro e saudável, onde é garantida a remuneração adequada ao cargo. Outras responsabilidades corporativas incluem o equilíbrio entre a vida profissional e pessoal, bem como o investimento na melhoria das condições de trabalho \cite{kotler_marketing_2022}.


\subsubsection{Componente econômica}\label{sec-modelo}
É inegável que o crescimento financeiro das corporações possa ter impactos desestabilizadores \cite{polman_net_2022}. Neste sentido, \textcite{bachnik_corporate_2022} consideram a restrição voluntária do mercado como um comportamento socialmente responsável. Assim, a componente econômica da RSC reforça a ideia que as normas de mercado não devem sobrepor-se às normas morais \cite{bachnik_corporate_2022}.

Embora se assuma que as empresas estejam cientes da necessidade de garantirem a sua rentabilidade financeira, esta continua a ser uma parte crucial da componente econômica \cite{ksiezak_triple_2018}. Ademais, uma empresa apenas poderá investir em iniciativas ambientais e sociais se tiver um modelo operacional que lhe assegura sustentabilidade financeira a longo prazo \cite{conley_como_2010}.

Uma parte significativa da responsabilidade econômica de uma empresa está na criação de lucro para os seus acionistas e trabalhadores \cite{daily_bankruptcy_1994}. A componente financeira reconhece a necessidade de gerar valor monetária para a sociedade em que as empresas operam. Isto inclui investimentos contínuos na economia local para combater a pobreza \cite{cezarino_dynamic_2019}.

Os \textit{stakeholders} estão cada vez mais atentos e exigem o acesso a informação financeira relevante \cite{francis_new_2022}; isto implica a adoção de princípios de transparência corporativa na área econômica. Ao garantir essa transparência financeira, as empresas contribuem para o combate à corrupção e outras práticas antiéticas \cite{rodriguezfernandez_sustainable_2020,ksiezak_triple_2018}. Em suma, a componente econômica do TBL tem como objetivo garantir que as empresas geram lucro de forma sustentável e responsável. 

\subsection{Tendências da responsabilidade social corporativa}\label{sec-organizacao}
\subsubsection{Diversidade, equidade e inclusão}
A diversidade, equidade e inclusão social (frequentemente reconhecida pela sua sigla DEI) emergiu como uma tendência da RSC no século XXI.  A diversidade (D) refere-se às diferenças físicas e socioculturais, enquanto a equidade (E) ao tratamento justo e a atribuição uniforme de remuneração, oportunidades, direitos e deveres e, por fim, a inclusão (I) refere-se à implementação e promoção de um ambiente que promova o sentido de pertença de grupos heterogêneos \cite{ferraro_embracing_2023}. \textcite{ferraro_embracing_2023} consideram-na uma obrigação moral, legal e econômica das corporações. 

A DEI não se limita apenas ao contexto interno das organizações, deve ser promovida externamente. Esta perspetiva é ilustrada por \textcite{barnhart_who_2013} através do exemplo do mercado de consumidores seniores, que muitas vezes perpetua uma imagem estigmatizada do envelhecimento. A representação dos diferentes grupos tem um impacto significativo nas suas experiências de consumo. Adicionalmente, de acordo com \textcite{ferraro_embracing_2023}, os consumidores tendem a absorver inconscientemente a mensagem de uma marca quando se identificam com ela. 

A pandemia de COVID-19 teve um impacto significativo nas comunidades marginalizadas \cite{creary_improving_2021,govindji_google_2023}; evidenciou as injustiças sociais através das taxas de infecção e mortalidade desproporcionais entre diferentes grupos raciais \cite{addo_double_2020}. Este contexto levou ao alargamento do foco da DEI, que historicamente se centrava principalmente na luta pela igualdade de gênero, para que abrangesse também a luta contra a injustiça e o preconceito racial \cite{creary_improving_2021}.

Além disso, o movimento \textit{Black Lives Matter}, que cresceu significativamente após o assassinato de George Floyd, desencadeou uma onda de protestos globais. Por isso, o impacto global da luta pela justiça social e racial, juntamente com a pandemia COVID-19, tornou a DEI numa prioridade corporativa \cite{human_human_2022}.

Esta mudança cultural afetou a forma como os consumidores percebem as marcas, tornando-os tão propensos a apoiar como a desafiar suas ideologias \cite{ferraro_embracing_2023}. Agora, exigem que as marcas sejam responsáveis pelos valores que prometem promover \cite{ferraro_embracing_2023}; estão mais atentos às empresas e às suas contribuições na criação de um mundo mais justo \cite{human_human_2022}.

A consciencialização crescente levou as marcas a atender às expectativas dos consumidores, que esperam que estas promovam iniciativas autênticas com impacto positivo e significativo \cite{ferraro_embracing_2023}.

A DEI é vista como uma estratégia de \textit{marketing}, pois melhora o relacionamento com os consumidores, que estão mais propensos a comprar produtos quando as marcas demonstram uma consciência social \cite{human_human_2022}. Para que tenham sucesso, as marcas devem ampliar continuamente o seu propósito e integrar os seus valores na sua estratégia, garantindo a sua diferenciação e a atração de novos clientes \cite{human_human_2022}.

\subsubsection{Ação climática}\label{sec-organizacao-latex}
Atualmente, as empresas e as suas marcas têm vindo a estabelecer uma série de compromissos ambientais \cite{salnikova_engaging_2022}. Mais de dois terços das empresas afirmam considerar os problemas climáticos durante a planificação das suas estratégias \cite{lambin_global_2020}.

As empresas estão também a investir em soluções inovadoras para enfrentar os problemas ambientais \cite{berkowitz_most_2021}. O desafio das empresas é serem capazes de criar estratégias que incentivem os seus consumidores a envolverem-se ativamente em iniciativas ambientalmente sustentáveis \cite{salnikova_engaging_2022}.

De acordo com \textcite{gonzalez-arcos_how_2021}, a maioria das intervenções corporativas promovem a mudança de comportamentos individuais, não se concentram na alteração de práticas sociais. Estas iniciativas podem despertar o sentimento de culpa nos consumidores, uma vez que lhes é atribuída a “grande” responsabilidade de salvarem e preservarem o meio ambiente. Para evitar a resistência, estas soluções têm de ser continuamente estudadas e aperfeiçoadas pelas corporações. Globalmente, o coletivismo desempenha um papel importante na promoção e adoção de atitudes ecológicas, por isso, as empresas devem reforçar a sua importância \cite{leonidou_consumers_2022}.

\subsubsection{Investimento de Impacto}\label{sec-titulo}
O investimento de impacto tem recebido uma maior atenção na última década \cite{ioannou_impact_2015}. De acordo com a \textcite{gsg_impact_new_2022}, os ativos europeus do investimento de impacto aumentaram cerca de 26\% entre 2020 e 2021, tornando-o numa tendência em ascensão. Atualmente, os acionistas não só consideram a sua margem de lucro, mas também reconhecem a importância de investir financeiramente no ambiente e na sociedade \cite{barber_impact_2017}.

O investimento de impacto tem como objetivo obter retornos financeiros enquanto produz, simultaneamente, um impacto social e/ou ambiental positivo \cite{barber_impact_2017,global_impact_investing_network_what_2023}. Para tal, parte dos lucros gerados por uma corporação devem ser redirecionados na resolução de problemas emergentes. Os desafios do século XXI são demasiado complexos para serem enfrentados exclusivamente por governos e instituições sociais e/ou ambientais, destacando assim a importância do envolvimento das corporações \cite{gsg_impact_new_2022}. Além disso, há uma crescente expectativa de que o capital privado seja investido na resolução dos principais desafios sociais e ambientais \cite{addy_calculating_2019}.

Esta tendência da RSC levou à criação de iniciativas bancárias, grupos de consultoria e outras instituições financeiras dedicadas a orientar as empresas nos seus investimentos \cite{barber_impact_2017}. Conforme a Gsg (2014), o "impacto" agora faz parte da dimensão de risco e retorno inerente à atividade comercial de uma empresa. Para ser eficaz, as intenções por trás de todos os investimentos devem ser genuínas \cite{barber_impact_2017,arora_understanding_2021,global_impact_investing_network_what_2023}. A transparência também é crucial para que as empresas tomem decisões mais informadas \cite{gsg_impact_new_2022}.

O termo impacto não está claramente definido, o que significa que nem sempre é possível quantificá-lo com métricas \cite{arora_understanding_2021}. Não existe uma fórmula estabelecida para avaliar o impacto social e ambiental positivo gerado \cite{addy_calculating_2019}; cabe às empresas escolher as métricas mais relevantes para avaliar a sua contribuição \cite{global_impact_investing_network_what_2023}. Por exemplo, a monitorização da redução das emissões de carbono pode ser uma métrica utilizada para calcular o impacto ambiental \cite{tora_three_2023}.

Em resumo, é o investimento de impacto que permitirá o financiamento da maioria das iniciativas de RSC, sendo também responsável pela mobilização de todo o capital para a resolução dos desafios ambientais e sociais.

\subsubsection{\textit{Ativism advocacy}}\label{sec-autores}
Talvez por ser uma nova área de investigação, o ativismo de marca ainda não possui de uma definição oficial. \textcite{kotler_brand_2018} descrevem-no como a participação voluntária das corporações em questões sociais, econômicas, ecológicas e/ou políticas; as corporações utilizam a influência das marcas para tomar uma posição e estimular debates sobre questões externas, frequentemente consideradas controversas \cite{kotler_marketing_2022}. Esta abordagem difere de outros comportamentos da RSC, uma vez que as corporações assumem uma posição proativa com o objetivo de aumentar a notoriedade de problemáticas socialmente relevantes \cite{korschun_brand_2021}. Desta forma, o ativismo é visto como uma evolução natural da RSC e uma tendência atual \cite{korschun_brand_2021,kotler_marketing_2022}, conforme destacado por \textcite{kotler_brand_2018}, este é igualmente o futuro do \textit{branding}.

É essencial que as empresas saibam identificar quando devem passar de observadoras a ativistas \cite{ghosh_when_2021}. Idealmente, devem consultar os seus \textit{stakeholders} e envolverem-se em questões que estes valorizam \cite{korschun_brand_2021}. Para alcançar o sucesso, as corporações precisam de estudar, acompanhar e compreender o seu público-alvo, tornando-as capazes de identificar quais as questões sociais e políticas que lhes são mais importantes. Além disso, a decisão de participar em ativismo corporativo é um compromisso irreversível \cite{kotler_marketing_2022}.

O ativismo é um tema complexo que apresenta alguns riscos \cite{mukherjee_brand_2020}. Embora não seja uma tendência nova, ganhou maior visibilidade nos últimos anos devido a crises mundiais, como a pandemia de COVID-19 e as injustiças raciais \cite{korschun_brand_2021}. Além disso, o ativismo implica o envolvimento em inúmeras questões polarizadoras. \textcite{lekakis_consumer_2022} exemplificam-no com a invasão do Capitólio nos Estados Unidos da América, motivada por divisões políticas opostas no país. Ao tomar partido em questões polarizadoras, as marcas correm o risco de alienar parte do seu público-alvo, o que pode prejudicar a sua sustentabilidade \cite{ortegon_danger_2019}. De acordo com \textcite{mukherjee_brand_2020}, marcas que reconsideram as suas posições ativistas e, posteriormente, pedem desculpas por as terem assumido podem enfrentar uma reação negativa.

O ativismo é fundamental dado que as expectativas dos consumidores aumentaram consideravelmente. Como \textcite{korschun_brand_2021} aponta, os consumidores não se preocupam apenas com o produto/serviço em si, mas também com a identidade e os valores da corporação que o comercializa.

\subsubsection{\textit{Marketing} de causa}\label{sec-idioma}
O \textit{marketing} de causa é uma prática que combina a RSC com as atividades comerciais, visando alinhar os interesses sociais com os interesses corporativos; é considerado uma das tendências da RSC. \textcite{beise-zee_cause-related_2013} define-o como uma categoria mais ampla de iniciativas da RSC, permitindo não só a associação da corporação a uma causa, mas também a ampliação da voz de uma necessidade emergente \cite{conley_como_2010}. Este tipo de ação está intrinsecamente ligado à filantropia corporativa e é realizado quando uma corporação opta por apoiar uma causa social ou ambiental alinhada com o perfil dos seus consumidores, permitindo assim o estabelecimento de relações mais estreitas com os \textit{stakeholders} \cite{beise-zee_cause-related_2013}; através de iniciativas deste cariz é destacada a sua preocupação com questões relevantes que afetam a sociedade \cite{kotler_marketing_2022}.

O \textit{marketing} de causa tem uma natureza dupla, uma vez que melhora o desempenho financeiro da corporação ao mesmo tempo que contribui para uma causa social \cite{robinson_choice_2012}. A corporação e a instituição por trás da causa estabelecem uma aliança mútua da qual ambas beneficiam \cite{fill_marketing_2016}. Através desta parceria, a corporação tem a oportunidade de comprovar e promover os seus esforços socialmente responsáveis \cite{beise-zee_cause-related_2013}. Além disso, os consumidores tendem a reagir positivamente à publicidade associada a uma causa social \cite{schiffman_consumer_2019}.

Este tipo de iniciativa oferece vários benefícios: a curto prazo, potencializa as vendas, já que os consumidores expressam as suas preferências através das suas compras; a longo prazo, melhora a imagem da corporação e adiciona valor às suas marcas, criando associações mais favoráveis \cite{conley_como_2010,beise-zee_cause-related_2013}. Adicionalmente, em mercados saturados onde características como preço ou atributos tangíveis não são tão significativos, esta parceria pode ser a melhor forma de garantir a diferenciação \cite{fill_marketing_2016}. Porém, é crucial que o compromisso com a causa seja genuíno \cite{conley_como_2010}.

Normalmente, o marketing de causa é realizado através de campanhas publicitárias para aumentar a consciencialização sobre uma causa específica. Por exemplo, durante a pandemia de COVID-19, muitas marcas investiram na criação de anúncios televisivos que promovessem as diretrizes de saúde do governo ou que homenageassem os profissionais de saúde \cite{brainstory_agency_como_2020}. Geralmente, o marketing de causa também envolve a doação monetária de uma quantia específica \cite{robinson_choice_2012}.

Apesar dos seus benefícios, este tipo de iniciativa também apresenta riscos. Para ser credível e ter um impacto significativo, não pode ser uma prática esporádica. Além disso, esta associação pode tanto atrair como afastar consumidores \cite{kotler_marketing_2022}. Por isso, a escolha da causa deve ser criteriosa, e os consumidores devem entender o motivo desta associação \cite{kotler_marketing_2022}. Portanto, a campanha deve estar alinhada com a estratégia de posicionamento da corporação e/ou da marca para evitar que seja interpretada como desonesta \cite{schiffman_consumer_2019}.

\section{Método}\label{sec-resumo}
O modelo proposto visa a investigar o relacionamento entre duas variáveis: a responsabilidade social corporativa (a variável independente) e a imagem de marca (a variável dependente). Embora este tema já tenha sido abordado em várias pesquisas, a maioria dos investigadores não consideraram as diferenças geracionais na interpretação dos seus resultados. Por isso, esta investigação adota uma abordagem específica dado que se foca exclusivamente na Geração X e na Geração Y. São consideradas as perspetivas de dois grupos geracionais distintos com base no contributo acadêmico de \textcite{wu_impact_2014}; na sua investigação, os autores concluem que diferentes gerações priorizam diferentes dimensões da RSC. Consideram este um aspeto relevante a ser explorado tanto pelas marcas como pela academia.  

Dada a natureza dos objetivos da investigação, foi adotado um plano misto, recorrendo-se simultaneamente ao método qualitativo através da análise de conteúdo e ao método quantitativo por meio da aplicação de um inquérito por questionário \cite{elo_qualitative_2008}. Por isso, contempla duas técnicas de recolha de dados.

Primeiramente, para o método qualitativo, foi feita uma análise de conteúdo a dois relatórios de 2008 e 2022 publicados pela The Walt Disney Company no seu \textit{website} relativamente à sua RSC. 

Esta investigação centraliza-se no método quantitativo, sendo o instrumento de recolha de dados um inquérito por questionário \textit{online}. O aproveitamento da internet para criar e promover o questionário considerou-se fundamental devido à amplitude do universo em estudo \cite{wright_researching_2005}. A presença \textit{online} do questionário conferiu ainda o anonimato aos respondentes \cite{ritter_introduction_2007}.  

O questionário desenvolvido é predominantemente composto por perguntas de resposta fechada. Relativamente ao \textit{design}, foi estruturado de modo a atender ao objetivo previamente delineado, consistido de três secções onde são utilizadas 2 variantes da escala de Likert. 

A primeira secção engloba um conjunto de itens destinados a avaliar a percepção da RS da The Walt Disney Company. A escala utilizada aborda as três componentes teorizadas por John Elkington em 1994. É composta por afirmações de elaboração própria e afirmações retiradas de escalas validadas e publicadas por \textcite{martinez_effect_2004}.

Na segunda secção, os inquiridos são apresentados às principais iniciativas de responsabilidade social da The Walt Disney Company. Nesta secção, avalia-se o grau de familiaridade com as principais ações da corporação. Os itens foram criados especificamente para este fim, sendo que a sua elaboração teve por base relatórios e informações atualmente disponíveis no \textit{website} da corporação.

Na terceira e última secção do questionário, é feita a avaliação à imagem de marca da Disney. Devido à natureza intangível do seu principal produto, o entretenimento infantil, optou-se pela utilização de uma escala que avalia os seus atributos, benefícios e atitudes, alinhado com o modelo \textit{Customer-based brand equity} \cite{heding_brand_2020}. Esta secção é composta por afirmações retiradas de escalas validadas e publicadas por \textcite{lien_online_2015,martinez_effect_2004}.

Como referido, a população desta investigação está restrita a dois grupos geracionais. Dado isto, para a Geração X e a Geração Y assumiu-se o intervalo temporal proposto pela \textcite{paw_research_center_generations_2019}. 

Para a obtenção das respostas, foi escolhido o método de amostragem de conveniência por bola de neve. Nesta abordagem, o investigador aproveita as suas redes sociais e a sua esfera de contatos pessoais para convidar indivíduos a participar voluntariamente na pesquisa. Alguns dos participantes desta investigação foram convidados intencionalmente por se enquadrarem no perfil da população em estudo. 

Em síntese, o método de amostragem adotado combinou a conveniência proporcionada pela amostragem de conveniência por bola de neve, assim como uma abordagem intencional na seleção de participantes. Procurou-se garantir a inclusão de inquiridos relevantes e representativos para os propósitos desta pesquisa. 

\section{Análise de resultados}\label{sec-secoes}
\subsection{Análise qualitativa}
Os relatórios analisados e comparados são referentes ao ano de 2008 (sendo este o primeiro disponibilizado \textit{online}) e 2022 (o mais recente até à data). Em 2008 o relatório está dividido nas seguintes áreas: crianças e família, conteúdo e produtos, ambiente, comunidade e locais de trabalho. Já em 2022, o relatório está dividido nas áreas: diversidade, equidade e inclusão, ambiente e conservação, comunidade, investimento nos trabalhadores e, por fim, operar de forma responsável. É relevante salientar a existência de um intervalo temporal de 14 anos entre os mesmos, o que se releva significativo na estruturação do documento: o contexto sociocultural ou até mesmo os fenômenos naturais resultam em diferentes áreas, iniciativas e objetivos que nem sempre são diretamente comparáveis. 

Através da comparação dos relatórios foi possível identificar pontos em comum, bem como práticas que permanecem inalteradas ao longo dos 14 anos que os separam. Em algumas áreas, as principais diferenças podem ser observadas na expansão de propostas, na criação de novos programas, na atualização de práticas e políticas estabelecidas e/ou no aumento do investimento financeiro. Em ambos os relatórios, a corporação ilustra os seus esforços socialmente responsáveis explorando detalhadamente as suas principais ações. 

No relatório referente a 2008, muitas das problemáticas foram apresentadas pela primeira vez ao público. Uma parte significativa dos dados disponibilizados não representam soluções já estabelecidas, mas sim explicações e contextualizações das medidas que foram tomadas. Esta decisão permitiu à corporação identificar áreas de atuação, comunicar e explorar os seus objetivos a médio e longo-prazo. Um exemplo disso é a área do ambiente, a corporação não documenta extensivamente o que já foi feito, opta por privilegiar a projeção de iniciativas futuras. Nesta seção, a corporação mostra o seu empenho em medir, pela primeira vez, métricas essenciais relacionadas com a sustentabilidade ambiental como, por exemplo, as emissões de gases de efeito de estufa nas suas infraestruturas (medição realizada e publicada em 2007 é referente a 2006, primeiro ano da iniciativa). Em 2022, observa-se um crescente compromisso com as práticas ambientais, refletindo-se em dados mensuráveis que não estavam ainda disponíveis em 2008. Neste ano, a corporação introduziu um conjunto de soluções inovadoras com o intuito de promover a sua sustentabilidade ambiental. 

Em 2022, é dado um maior destaque à transparência corporativa. Progressivamente, a corporação inclui dados representativos, dedicando um capítulo a uma variedade de valores numéricos das suas áreas de foco. Esta iniciativa envolveu a disponibilização de novas métricas e objetivos mensuráveis, permitindo aos leitores acompanhar o progresso da The Walt Disney Company de forma mais próxima. 

Existe uma discrepância na abordagem à DEI. Em 2008, esta é mencionada periodicamente, sendo incluída em alguns pontos de diversos capítulos. No relatório de 2022, a DEI emerge como uma das áreas prioritárias de atuação. Neste capítulo, a corporação destaca todas as suas iniciativas, assim como as mudanças implementadas para tornar a suas marcas mais inclusivas e representativas. A orientação para a DEI, não se reflete apenas nos seus produtos (como os filmes, as séries, as experiências, entre outros), mas também na sua força de laboral. Esta abordagem está alinhada com as perspetivas de \textcite{human_human_2022,hessekiel_2023_2023}, que argumentam que as corporações começaram a prestar uma maior atenção à forma como podem contribuir na criação de um mundo mais justo. Um exemplo desta evolução é a comunidade LGBTQIA+ que não é mencionada no relatório de 2008, havendo apenas uma breve menção a casais do mesmo sexo; já em 2022, a sigla é utilizada diversas vezes, inclusive em outros capítulos do relatório. 

Em 2022, os valores das doações e investimentos em diversos contratos, departamentos e ações são significativamente mais expressivos, indicando um compromisso ampliado com a sua RSC. Por exemplo, as doações para os ecossistemas (a fauna e a flora), testemunharam um aumento significativo, passando de 1.8 milhões de dólares em 2008, para 6.7 milhões de dólares em 2022, o que representa um crescimento percentual de aproximadamente 272\%.  De maneira semelhante, o investimento em empresas lideradas por minorias ou mulheres também aumentou consideravelmente; em 2008, esse investimento atingiu os 425 milhões de dólares, enquanto em 2022 elevou-se para os 800 milhões de dólares. Estes números demonstram não apenas o crescimento econômico da corporação, mas também o reconhecimento da importância de um compromisso mais robusto com diversas comunidades. 

Em ambos os relatórios, a corporação referencia frequentemente o seu fundador, Walt Disney, e os valores e legados deixados por si na criação da sua marca Disney. Os valores de Walt Disney servem como pilares orientadores para as ações da empresa; a corporação utiliza-os para justificar algumas das iniciativas e projetos explorados nos relatórios. 

Apesar dos 14 anos que separam estes relatórios, é possível identificar uma doutrina consistente que destaca a identidade corporativa da The Walt Disney Company. A continuidade na visão, nos valores e legados deixados por Walt Disney reforçam o seu compromisso de longo prazo, criando uma base sólida para todas as suas futuras ações. 

\subsection{Análise quantitativa}\label{sec-format-simple}
A amostra utilizada é composta por 172 inquiridos, pertencentes a um dos dois grupos geracionais em estudo, familiarizados com a corporação e a marca de entretenimento infantil Disney. Dos 172 inquiridos, 44,8\% são pertencentes da Geração X e 55,2\% da Geração Y. Dos 172 inquiridos, 71,5\% identificam-se com o gênero feminino, 27,9\% com o gênero masculino e, por fim, 0,6\%  com outro gênero. Relativamente à formação acadêmica, destaca-se que 80,8\% frequentou e completou pelo menos um grau do Ensino Superior. Quanto ao seu estado civil, 54,1\% indivíduos estão casados ou em união de fato, 36,0\% estão solteiros, 9,3\% estão divorciados/separados e, por fim, 0,6\% é viúvo. Por fim, a amostra é constituída predominantemente por indivíduos empregados (92,4%). 

Primeiramente, analisa-se a RSC. Como anteriormente referido, esta secção teve por base a teoria do \textit{Triple Bottom Line} e, por isso, avalia a percepção que os inquiridos têm da componente econômica, social e ambiental da The Walt Disney Company. A segunda avalia o nível de familiaridade com as principais ações socialmente responsáveis da corporação e, por fim, a terceira e última, a imagem de marca da Disney. Todas estes itens eram de resposta obrigatória. 

\subsubsection{Componente econômica}\label{sec-links}
A componente econômica revelou um valor médio consideravelmente alto (4.18; d.p=0,91). Dado que foi utilizada uma variante da escala de Likert, esta média indica uma inclinação para a opção de resposta “Concordo parcialmente”. 

Os inquiridos demonstram concordância quanto à responsabilidade econômica da The Walt Disney Company. Observa-se uma homogeneidade nas médias, dado que estão situadas entre o 4,01 e 4,31. Devido ao alto nível de concordância, assume-se que os inquiridos acreditam que a The Walt Disney Company é uma corporação economicamente responsável. 

\subsubsection{Componente social}\label{sec-outras-estr}
A componente social revela um valor médio de 3,40 (d.p.=0,97). Esta média encontra-se próxima do valor 3 na escala de Likert, que corresponde à opção de resposta "Não concordo, nem discordo". Isto indica que as respostas para esta componente foram predominantemente neutras dado situarem-se no centro da escala.

Avaliando a componente social com base nos 9 itens (\Cref{tab01}), as médias variam entre 2,95 e 3,95. Destes, os itens 2.2, 2.3, 2.8 e 2.9 registam médias próximas de 4, refletindo uma concordância parcial. Os itens 2.8 e 2.9, relacionados com a diversidade racial, étnica e a inclusão de personagens LGBTQ+, obtiveram as médias mais altas (3,95 e 3,69, respetivamente). Estes resultados refletem o crescente foco da The Walt Disney Company na DEI, evidenciado na análise do relatório corporativo de 2022.

\begin{table}[h!]
\centering
\begin{threeparttable}
\caption{Análise descritiva dos itens da componente social da RSC.}
\label{tab01}
\begin{tabular}{p{8cm} p{1cm} p{1cm} p{1cm}}
\toprule
 Itens & N & \multicolumn{1}{p{1cm}}{Média} & \multicolumn{1}{p{1.2cm}}{Desvio Padrão} \\
\midrule
2.1 Está empenhada em melhorar o bem-estar das comunidades & 172 & 3,34 & 1,09 \\
2.2 Participa ativamente em eventos sociais e culturais & 172 & 3,58 & 0,91 \\
2.3 Promove um papel na sociedade que vai além da geração de lucro & 172 & 3,56 & 1,07 \\
2.4 Proporciona um tratamento justo aos seus funcionários & 172 & 2,95 & 0,82 \\
2.5 Oferece oportunidades de treinamento e promoção aos seus funcionários &
172 & 3,24 & 0,82 \\
2.6 Preocupa-se com a atribuição de flexibilidade na expressão individual dos seus funcionários & 172 & 3,13 & 0,84 \\
2.7 Ajuda a resolver problemas sociais & 172 & 3,23 & 1,02 \\
2.8 Preocupa-se com a diversidade racial e étnica no seu entretenimento &
172 & 3,95 & 1,02 \\
2.9 Preocupa-se com a inclusão de personagens LGBTQ+ no seu entretenimento &
172 & 3,69 & 1,02 \\
\bottomrule
\end{tabular}
\source{Elaboração própria.}
\end{threeparttable}
\end{table}


\subsubsection{Componente ambiental}\label{sec-listas}
A componente ambiental, apresenta uma média de 3,23 (d.p.=0,98). A resposta mais registada foi a opção 3 “Não concordo, nem discordo” mostrando, novamente, um posicionamento neutro ou indecisivo. 

As médias da componente ambiental (\Cref{tab02}) situam-se em torno do valor 3, variando entre 3,11 e 3,41. A exceção é o último item, 3.7, que apresenta um desvio padrão de 1, indicando uma ligeira dispersão das respostas registadas. Dado que a resposta neutra é predominante nesta secção, assume-se que a amostra apresenta alguma neutralidade e/ou indecisão em relação à avaliação dos comportamentos ambientalmente responsáveis da The Walt Disney Company. 

\begin{table}[h!]
\centering
\begin{threeparttable}
\caption{Análise descritiva dos itens da componente ambiental da RSC.}
\label{tab02}
\begin{tabular}{p{8cm} lll}
\toprule
 Itens & N & \multicolumn{1}{p{1cm}}{Média} & \multicolumn{1}{p{1.2cm}}{Desvio Padrão} \\
\midrule
3.1 Protege o meio ambiente & 172 & 3,36 & 0,99 \\
3.2 Reduz o consumo de recursos naturais & 172 & 3,11 & 0,90 \\
3.3 Recicla & 172 & 3,37 & 0,85 \\
3.4 Comunica as suas práticas ambientais aos seus clientes & 172 & 3,12 & 0,99 \\
3.5 Explora energias renováveis num processo de produção amigo do meio-ambiente & 172 & 3,13 & 0,84 \\
3.6 Participa em certificações ambientais & 172 & 3,13 & 0,76 \\
3.7 Preocupa-se em promover comportamento ecológico no seu entretenimento  &
172 & 3,41 & 1,00 \\
\bottomrule
\end{tabular}
\source{Elaboração própria.}
\end{threeparttable}
\end{table}

\subsubsection{Nível de familiaridade}\label{sec-listas}
No que diz respeito à familiaridade com as principais ações socialmente responsáveis da corporação, os resultados são baixos. Embora a média se aproxime da opção 3 “nem muito, nem pouco”, a resposta mais frequentemente registada nestes 5 itens é a 1 “Nada familiarizado”. 
Todos os itens desta secção apresentam alguma dispersão de dados, tornando-a na variável menos consensual da investigação e indicando alguma heterogeneidade na distribuição dos dados (\Cref{tab03}). As médias variam entre 2,02 e 3,85, tendo uma inclinação para o valor 2, correspondente à opção de resposta “Pouco familiarizado”. No entanto, é possível destacar a parceria entre a Disney e a Fundação Make-A-Wish como a iniciativa com maior reconhecimento. A média deste item está próxima do valor 4 (3,98 (d.p.=0,92), “Familiarizado”. Globalmente, a maioria dos inquiridos demonstra pouco conhecimento sobre as principais iniciativas da The Walt Disney Company. 


\subsubsection{Nível de familiaridade}\label{sec-figuras-tabelas}
No que diz respeito à familiaridade com as principais ações socialmente responsáveis da corporação, os resultados são baixos. Embora a média se aproxime da opção 3 “nem muito, nem pouco”, a resposta mais frequentemente registada nestes 5 itens é a 1 “Nada familiarizado”. 

Todos os itens desta secção apresentam alguma dispersão de dados, tornando-a na variável menos consensual da investigação e indicando alguma heterogeneidade na distribuição dos dados (\Cref{tab03}). As médias variam entre 2,02 e 3,85, tendo uma inclinação para o valor 2, correspondente à opção de resposta “Pouco familiarizado”. No entanto, é possível destacar a parceria entre a Disney e a Fundação Make-A-Wish como a iniciativa com maior reconhecimento. A média deste item está próxima do valor 4 (3,98 (d.p.=0,92), “Familiarizado”. Globalmente, a maioria dos inquiridos demonstra pouco conhecimento sobre as principais iniciativas da The Walt Disney Company. 

\begin{table}[h!]
\centering
\begin{threeparttable}
\caption{Análise descritiva dos itens do nível de familiaridade.}
\label{tab03}
\begin{tabular}{p{8cm} lll}
\toprule
 Iniciativas & N & \multicolumn{1}{p{1cm}}{Média} & \multicolumn{1}{p{1cm}}{Desvio Padrão} \\
 \midrule
4.1 Disney Worldwide Conservation Fund & 172 & 2,02 & 1,19 \\
4.2 Disney VoluntEARS & 172 & 2,08 & 1,27 \\
4.3 Diversidade, Equidade e Inclusão & 172 & 2,88 & 1,40 \\ 
4.4 Make-A-Wish & 172 & 3,85 & 1,22 \\
4.5 Future Storytellers & 172 & 2,27 & 1,25 \\
\bottomrule
\end{tabular}
\source{Elaboração própria.}
\end{threeparttable}
\end{table}

\subsubsection{Imagem de marca}
A respeito da imagem de marca, a sua média está próxima do valor 4 (3,88; (d.p.=0,97), opção de resposta “Concordo parcialmente”. No entanto, a resposta mais registada nesta variável é a 5 “Concordo totalmente”. 

É possível verificar que as médias dos itens que avaliam a imagem de marca são elevadas (\Cref{tab04}), variando entre o 4,17 e o 4,63. Observa-se uma concordância geral em relação ás afirmações apresentadas. Considerando que todos os itens utilizados para avaliar a imagem de marca da Disney são afirmações de cariz positivo, conclui-se que esta tem uma boa imagem.

\begin{table}[h!]
\centering
\begin{threeparttable}
\caption{Análise descritiva dos itens da imagem de marca.}
\label{tab04}
\begin{tabular}{p{8cm} lll}
\toprule
 Itens & N & \multicolumn{1}{p{1cm}}{Média} & \multicolumn{1}{p{1cm}}{Desvio Padrão} \\
 \midrule
5.1 A Disney é uma marca de confiança & 172 & 4,23 & 0,89 \\
5.2 A Disney é uma marca atrativa & 172 & 4,63 & 0,66 \\
5.3 A Disney é uma marca agradável & 172 & 4,44 & 0,80 \\
5.4 A Disney é uma marca com boa reputação & 172 & 4,31 & 0,88 \\
5.5 A Disney tem personalidade & 172 & 4,45 & 0,78 \\
5.6 A Disney é interessante & 172 & 4,43 & 0,78 \\
5.7 A Disney é diferente da sua concorrência & 172 & 4,17 & 9,93 \\
\bottomrule
\end{tabular}
\source{Elaboração própria.}
\end{threeparttable}
\end{table}

Foi aplicado o teste t de Student às diferentes componentes da RSC. As médias observadas em relação à perspetiva de cada geração sobre as componentes socialmente responsáveis da The Walt Disney Company não apresentam diferenças estatisticamente significativas (econômica $p=0,54>0,05$; social $p=0,27>0,05$). Importa notar que a componente ambiental da RSC regista o único valor próximo a 0,05 sendo, ainda assim, ligeiramente elevado o suficiente para não se considerar uma diferença estaticamente significante ($p=0,055>0,050$).

Em resumo, não foram identificadas diferenças estatisticamente significativas nas médias das diferentes componentes da RSC. Isto significa que, independentemente da geração, os respondentes expressam opiniões semelhantes (\Cref{tab05}).  

\begin{table}[h!]
\centering
\begin{threeparttable}
\caption{\textit{Teste t} na comparação das médias das componentes da RSC, familiaridade e imagem de marca.}
\label{tab05}
\begin{tabular}{ll}
\toprule
\textbf{Variável} & \textbf{Significância} \\
 \midrule
Componente económica da RSC & 0,545 \\
Componente social da RSC & 0,278 \\
Componente ambiental da RSC & 0,055 \\
Familiaridade & 0,770 \\
Imagem de marca & 0,102 \\
\bottomrule
\end{tabular}
\source{Elaboração própria.}
\end{threeparttable}
\end{table}

Relativamente à variável da familiaridade com as ações socialmente responsáveis desenvolvidas pela corporação, o mesmo padrão é observado ($p=0,77>0,05$). Não existem diferenças estatisticamente relevantes nas médias registadas das gerações nesta dimensão. 

Por fim, com a variável imagem de marca, observa-se mais uma vez a inexistência de diferenças estatisticamente significativas entre as médias das gerações em relação à sua percepção da imagem de marca da Disney ($p=0,10>0,05$). 

\subsubsection{Análise de correlação}
Através da análise de correlação, observa-se que a imagem de marca e a percepção da RSC têm um relacionamento estatisticamente significativo ($sig=0,00<0,05$). O valor da correlação é de 0,534, indicando uma correlação positiva e moderada. Isto significa que, à medida que a percepção da RSC aumenta, a imagem de marca tende a melhorar, e vice-versa. 

O nível de familiaridade e a imagem de marca também apresentam um relacionamento estatisticamente significativo ($sig=0,04<0,05$). O valor da sua correlação é de 0,218, indicando uma correlação positiva e fraca, dada a sua proximidade ao valor 0. Isso sugere que, à medida que o nível de familiaridade com a marca aumenta, a imagem de marca tende a melhorar, embora o seu impacto seja relativamente modesto. 

Conclui-se que tanto a RSC e a imagem de marca, como o nível de familiaridade e a imagem de marca, têm uma relação positiva e significativa. Isto confirma que a imagem de marca é positivamente influenciada pela percepção da RSC e que a imagem de marca é também positivamente influenciada pelo nível de familiaridade em relação às ações socialmente responsáveis de uma corporação.

\section{Conclusões}
Na presente investigação conclui-se que, embora a RSC influencie a imagem de marca, não foram identificadas diferenças entre a Geração X e a Geração Y. A aferição e identificação de uma relação positiva entre a RSC e a imagem de marca neste estudo reforça a ideia de que a percepção dos consumidores sobre a RSC influencia positivamente a imagem de marca. Isto é, quanto mais positiva for a percepção da responsabilidade social de uma corporação, melhor será a sua imagem de marca

Ao avaliarmos o efeito da RSC na imagem de marca, junto de dois grupos geracionais: a Geração X e a Geração Y, juntamente com o destaque dado à RSC no contexto empresarial de uma marca de entretenimento infantil a Walt Disney, comprovamos, \textit{per si}, a relevância do tema. A The Walt Disney Company, trata-se de uma corporação com mais de um século de história, cuja influência transcende fronteiras geográficas e culturais. Dada a sua notoriedade, considerou-se apropriado aprofundar a caracterização da sua RSC, através da comparação de relatórios corporativos; esta análise proporcionou uma compreensão mais profunda sobre como uma empresa tão emblemática lida com questões sociais, ambientais e econômicas.

A análise dos relatórios corporativos evidenciou a importância da RSC não apenas como um fator isolado capaz de influenciar a imagem de uma marca, mas como parte integrante da estratégia e identidade corporativa. A The Walt Disney Company utiliza a sua RSC como forma de reforçar a sua identidade.  Ao incorporar a RSC nas suas operações, a empresa demonstra também um compromisso genuíno com os seus \textit{stakeholders}, consolidando a sua posição como uma entidade capaz de se adaptar às mudanças socioculturais. Esta abordagem está alinhada com a perspetiva de \textcite{he_effect_2014}, que caracterizam a RSC como uma ferramenta de \textit{marketing} estratégico. A Disney parece reconhecer o seu papel enquanto ícone cultural e, como tal, assume as obrigações sociais, culturais e ambientais perante a sociedade \cite{heding_brand_2020}.

Embora historicamente enraizada em tradições culturais e estereótipos, a Disney tem investido consideravelmente na promoção da DEI. Este compromisso reflete-se de forma tangível nos seus produtos através da criação de personagens e narrativas que representam diversas culturais, etnias e experiências. Verificou-se também um aumento significativo nas suas doações e investimentos, bem como um aprimoramento no seu compromisso com a sustentabilidade. Conclui-se que a corporação reflete algumas das principais tendências da RSC. De resto, embora tenhamos identificado algumas discrepâncias entre os relatórios, a The Walt Disney Company tem uma identidade corporativa coerente, guiando-se predominantemente nos valores e legados deixados pelo seu criador, Walt Disney. 

Esta investigação abordou vários construtos, incluindo a perceção da RSC da Walt Disney, o nível de familiaridade com as suas principais atividades socialmente responsáveis e a sua imagem de marca. A escolha da sua aplicação a duas gerações distintas baseou-se na teoria de \textcite{wu_impact_2014}, que sugere que as diferenças geracionais podem influenciar as dimensões valorizadas pelos consumidores, incluindo a sua percepção global da responsabilidade social de uma corporação. No entanto, não foram identificadas diferenças significativas entre as gerações X e Y. A ausência de divergências nas suas perspetivas pode ser interpretada como uma convergência de valores ou, alternativamente, com algum cepticismo em relação aos valores que são declarados pela The Walt Disney Company e as práticas que, efetivamente, são levadas a cabo pela empresa. 

Quando examinada a relação entre a RSC, a imagem de marca e o nível de familiaridade com as iniciativas, verificou-se uma correlação positiva entre os construtos. Assim, este estudo ecoa algumas das contribuições de diversos investigadores para o tema. Reforça a importância da RSC não apenas como uma escolha ética, mas também como uma estratégia eficaz para o sucesso corporativo.

Sugere-se com principal pista para futura investigação o recurso a uma amostra representativa, definida de forma aleatória. Sugere-se também o alargamento da amostra, através da sua aplicação a outras gerações para que, também elas, possam dar novos \textit{insights} sobre o tema. 

%Editar funções dos autores

\printbibliography\label{sec-bib}
%conceptualization,datacuration,formalanalysis,funding,investigation,methodology,projadm,resources,software,supervision,validation,visualization,writing,review
\begin{contributors}[sec-contributors]
\authorcontribution{Rita Cordeiro}[conceptualization,datacuration,formalanalysis,investigation,methodology,writing]
\authorcontribution{Sandra Miranda}[investigation,supervision,projadm,visualization,validation,review]
\end{contributors}
\end{document}
