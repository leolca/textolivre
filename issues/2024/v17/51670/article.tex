\documentclass[english]{textolivre}

% metadata
\journalname{Texto Livre}
\thevolume{17}
%\thenumber{1} % old template
\theyear{2024}
\receiveddate{\DTMdisplaydate{2024}{3}{19}{-1}}
\accepteddate{\DTMdisplaydate{2024}{10}{21}{-1}}
\publisheddate{\today}
\corrauthor{Aseel Alshbeekat}
\articledoi{10.1590/1983-3652.2024.51670}
%\articleid{NNNN} % if the article ID is not the last 5 numbers of its DOI, provide it using \articleid{} commmand 
% list of available sesscions in the journal: articles, dossier, reports, essays, reviews, interviews, editorial
\articlesessionname{articles}
\runningauthor{Alshbeekat and Awwad}
%\editorname{Leonardo Araújo} % old template
\sectioneditorname{Daniervelin Pereira}
\layouteditorname{João Mesquita}

\title{Metadiscourse markers in online promotions: exploring linguistic and visual metadiscourse in Jordanian school promotions}
\othertitle{Marcadores de metadiscurso em promoções on-line: explorando o metadiscurso linguístico e visual nas promoções escolares da Jordânia}

\author[1]{Aseel Alshbeekat~\orcid{0000-0001-7327-8482}\thanks{Email: \href{mailto:aseel.shbeekat@iu.edu.jo}{aseel.shbeekat@iu.edu.jo}}}
\author[2]{Anas Awwad~\orcid{0000-0002-5458-0181}\thanks{Email: \href{mailto:anas.awwad@iu.edu.jo}{anas.awwad@iu.edu.jo}}}
\affil[1]{Isra University, Faculty of Arts, Department of English Language / Translation, Amman, Jordan.}
\affil[2]{Isra University, Faculty of Arts, Department of English Language \& Literature, Amman, Jordan.}

\addbibresource{article.bib}

\begin{document}
\maketitle
\begin{polyabstract}
\begin{abstract}
This study examines the use of visual and linguistic metadiscourse markers in academic posters of private schools in Jordan and their role in persuasion. To this end, the study analyses a corpus of 40 advertisements for the use of visual and linguistic metadiscourse markers. The advertisements were published between 2019 and 2023 and were chosen randomly. The data were analysed qualitatively following \posscite{kumpf_visual_2000} framework of visual metadiscourse and \textcite{hyland_metadiscourse:_2005} model of linguistic metadiscourse. The study compared the use of visual and linguistic metadiscourse markers in the academic posters and examined how the use of these metadiscourse markers plays a crucial role in the process of persuasion and in encouraging parents to choose a particular school over another. The results showed that all types of visual metadiscourse, particularly attraction, convention and consistency, were highly marked in the data examined. The results also showed that directives had the highest frequency compared to other linguistic metadiscourses and were used as essential elements of persuasive language. The study concluded that the role of visual and linguistic metadiscourse in reflecting the goals of academic institutions, engaging audiences, and attracting parents’ interest cannot be ignored and plays an effective role in persuasive writing. The study holds implications for advertising effectiveness, emphasizing cultural sensitivity, strategic appeal deployment, and continuous adaptation. Acknowledging limitations in sample size and temporal specificity, the study recommends a balance of explicit and implicit information, exploration of humor, and collaborative research initiatives for industry growth.

\keywords{Directives \sep Visual metadiscourse \sep Linguistic metadiscourse \sep School promotions \sep Jordanian}
\end{abstract}

\begin{portuguese}
\begin{abstract}
Este estudo examina o uso de marcadores metadiscursivos visuais e linguísticos em cartazes acadêmicos de escolas particulares na Jordânia e seu papel na persuasão. Para tanto, o estudo analisa um \textit{corpus} de 40 anúncios publicitários de uso de marcadores metadiscursivos visuais e linguísticos. Os anúncios foram publicados entre 2019 e 2023 e foram escolhidos aleatoriamente. Os dados foram analisados qualitativamente seguindo a estrutura de metadiscurso visual de \textcite{kumpf_visual_2000} e o modelo de metadiscurso linguístico de \textcite{hyland_metadiscourse:_2005}. O estudo comparou o uso de marcadores metadiscursivos visuais e linguísticos em cartazes acadêmicos e examinou como o uso desses marcadores metadiscursivos desempenha um papel crucial no processo de persuasão e no incentivo aos pais a escolherem uma escola específica em detrimento de outra. Os resultados mostraram que todos os tipos de metadiscurso visual, particularmente atração, convenção e consistência, foram altamente marcados nos dados examinados. Os resultados também mostraram que as diretivas tiveram maior frequência em comparação com outros metadiscursos linguísticos e foram utilizadas como elementos essenciais da linguagem persuasiva. O estudo concluiu que o papel do metadiscurso visual e linguístico na reflexão dos objetivos das instituições acadêmicas, no envolvimento do público e na atração do interesse dos pais não pode ser ignorado e desempenha um papel eficaz na escrita persuasiva. O estudo tem implicações para a eficácia da publicidade, aumentando a sensibilidade cultural, a implantação do apelo estratégico e a adaptação contínua. Reconhecendo as limitações no tamanho da amostra e na especificidade temporal, o estudo recomenda um equilíbrio entre informações explícitas e implícitas, exploração do humor e iniciativas de pesquisa colaborativa para o crescimento da indústria.

\keywords{Diretivas \sep Metadiscurso visual \sep Metadiscurso linguístico \sep Promoções escolares \sep Jordaniano}
\end{abstract}
\end{portuguese}
\end{polyabstract}

\section{Introduction}
Language and culture are intricately intertwined, shaping the way we communicate, express ourselves, and understand the world around us. The relationship between language and culture has been the subject of research by scholars and linguists and has led to various insights and perspectives. According to \textcite[p.~192]{Halliday_1985}, language can be defined as “a semiotic system”. This definition emphasizes the diverse forms of language used by different cultural groups. Language serves as a means of communication within a specific community, enabling individuals to express their thoughts, emotions, and experiences. It includes not only the spoken or written word but also the use of gestures, signs, and other modes of nonverbal communication. Language not only reflects culture but also plays a central role in its preservation and transmission. \textcite[p.~42]{clark_using_1996} described language as “a type of collaborative action that functions as a facilitator and coordinator in daily interactions”. Through language, cultural norms, traditions, stories, and knowledge are passed from one generation to the next. Language acts as a vessel for cultural heritage, allowing individuals to connect with their roots, express their identities, and maintain a sense of belonging within their cultural community.

Therefore, online advertising is now considered one of the most important areas for scientific research in many fields such as: linguistics, mass media and economics. \posscite[p.~441]{okeeffe_impact_2011} defines media discourse as “interactions that take place through a broadcast platform, whether spoken or written, in which the discourse is oriented to a non-present reader, listener or viewer”. Advertising is one of the most influential tools that sellers and companies use to persuade the target group to buy a specific product. \textcite[p.~4]{wahl_persuasion_2021} define persuasion as “the process of attempting to change or reinforce attitudes, values, beliefs, or behavior”.

Education in Jordan constitutes a basic building block of society. Furthermore, many factors play a vital role in choosing the right school; one of these factors is advertising. It is a powerful instrument used by sellers and companies to persuade their target audience and promote specific products. In the field of advertising, linguistic and visual metadiscourse markers play a crucial role in capturing consumers' attention, shaping their attitudes, and influencing their purchasing decisions \cite{moriarty_advertising:_2014}.

Academic posters as a type of advertising are considered one of the most important influential tools used in the field of education. Hence, they have a great impact in the persuasion process, especially in Jordan, where education is an essential part of life and is one of the things that occupy a large part of the minds of parents.

\section{Literature review}\label{sec-normas}
\subsection{Visual and Linguistic Metadiscourse in online advertisements}
Metadiscourse is used as a method for analyzing different types of texts known as genres. According to \textcite[p.~56]{hyland_discourse_2013}, metadiscourse is “the cover term for the self-reflective expressions used to negotiate interactional meanings in a text, assisting the writer (or speaker) to express a viewpoint and engage with readers as members of a particular community”. The innovative work of \textcite{crismore_talking_1989} of metadiscourse has inspired many researchers to focus more on the field of analyzing metadiscourse in many types of texts. \textcite[p.~17]{hyland_metadiscourse:_2022} defined metadiscourse as “a coherent set of interpersonal resources used to organize a discourse or the writer’s stance towards either its content or the reader”.

There are many functions for advertising discourse. \textcite{cook_discourse_2001} stated that the functions of discourse of advertising are divided into the following categories: persuasion, inform, worry, and misinform. However, it can be seen that the main function is the persuasive function. \textcite[p.~25]{leech_linguistic_1969} claims that the language of advertising “comes under the broader heading of ‘loaded language’; that is, it aims to change the will, opinions, or attitudes of its audience”. For example, \textcite{rababah_persuasive_2016} investigated the use of persuasive appeals in English and Arabic television advertisements. The study analyzed the linguistic and visual features of advertising to identify the most commonly used persuasive appeals in advertising. The results of the study suggest that advertisers use various persuasive tools, such as emotional appeals, celebrity endorsements, and humor to influence audiences. Many studies have addressed metadiscourse of linguistics (see \textcite{longo_role_1994,crismore_talking_1989,hyland_metadiscourse:_2010,Povolna_2020}). However, few studies have focused on the connection between visual and linguistic metadiscourses and online advertising \cite{fuertes-olivera_persuasion_2001,gustafsson_metadiscourse_nodate}.

\textcite{al-subhi_metadiscourse_2022} examined the use of both linguistic and visual metadiscourse in social media advertising. The social media platforms involved were Snapchat, Instagram, and Twitter. The study examined the effectiveness of various metadiscourse features used in online advertising, such as rhetorical questions, hedges, and intensifiers in persuading audiences. The analysis of the visual metadiscourse was conducted based on \posscite{kumpf_visual_2000} model. \textcite{kumpf_visual_2000} shed light on the concept of visual metadiscourse by stating that it can reflect the expectations of the receivers and fulfil their needs. \textcite{kumpf_visual_2000} claimed that there is a more creative approach to visual metadiscourse that consists of 10 categories to simulate the previous approaches \cite{crismore_talking_1989,hyland_metadiscourse:_2005,kopple_exploratory_1985,Hyland_2019}. The findings of the study highlighted the importance of using metadiscourse to increase persuasion. Such analysis of metadiscourse markers in this study was conducted based on \posscite{hyland_metadiscourse:_2005} models of metadiscourse and \posscite{kumpf_visual_2000}. According to Hyland’s models, metadiscourse consists of an interactional dimension and an interactive dimension. These dimensions were incorporated into the analysis of the linguistic metadiscourse. The analysis of visual metadiscourse in this study is based on \posscite{kumpf_visual_2000}, who believed that the visual metadiscourse markers are classified into 10 categories: first impression, style, heft, convention, chunking, external skeleton, consistency, expense, attraction, and interpretation. The study of online advertising should include both linguistic and visual elements that play a vital role in convincing parents and attracting their attention to register their children in a particular school. So, we can say that a successful advertisement is the one that can create a mixture between visual and linguistic elements. \textcite[p.~165]{de_groot_picture_2016} defined visual metadiscourse as “a set of devices that designers use to convey meaning in order to influence the audience’s interpretation of the text”. 

\textcite{carrio-pastor_multimodal_2021} examined multimodal metadiscourse in digital academic journals by categorizing these multimodal discourses into textual and visual metadiscourse. The corpus consisted of 252 scientific papers with a focus on medicine, linguistics, and engineering. The results showed that the scientific authors do not follow the same multimodal metadiscourse. For example, the linguistic authors prefer the use of textual metadiscourses, while the authors who specialize in writing medical and engineering articles prefer the use of visual metadiscourse. \textcite{carrio-pastor_multimodal_2021} stated that academic culture plays a very important role in creating a difference between authors specializing in linguistics, engineering and medicine. \textcite{aguilar_metadiscourse_2017} examined the functions of textual and visual metadiscourse in academic posters by using a multimodal approach. The findings of the study showed that visual and textual metadiscourse are used in academic posters to attract the attention of the receivers and audiences.
        
\textcite{xia_engaging_2021} examined the strategies used in TED Talks (ideas worth spreading) and highlighted the connection between the multimodal semiotic resources in digital media. The corpus consisted of 28 highly viewed TED talks. The results of this study show that there are five strategic multimodal configurations. \textcite[p.~52]{xia_engaging_2021} stated that “the speech is captured in the format of a video, visual resources such as the size of image, the horizontal perspectives and the height of shots can also convey interpersonal meaning”. According to \textcite[p.~1]{halliday_language_1978}:

\begin{quote}
    If the context is theorized in linguistic terms as another stratum in the organization of language itself, this enables us to model its variation and complexity, taking account of the differing situational contexts for different levels and kinds of teaching/learning activities, as well as the processes and the institutions of education and the different cultures within which this are located.
\end{quote}

It can be noticed that there is a strong tie among the content and the interpersonal meaning as both of them affect the way the readers receive the intended message. Given all that has been mentioned previously, it can be stated that there is a lack of previous evidence on the use of visual and linguistic metadiscourse in school promotions in Jordan. For this reason, this study seeks to examine the use of visual and linguistic metadiscourse in Jordanian school promotions.

\section{Methodology}\label{sec-conduta}
\subsection{Research questions}
The present study aims to analyse visual and linguistic metadiscourse markers in a series of academic posters from private schools in Amman. The study answers the following research questions:

\begin{enumerate}
    \item What visual metadiscourses are used in promoting the Jordanian academic school?
    \item What linguistic metadiscourse markers are used in the Jordanian academic school promotions?
\end{enumerate}

\section{Development}\label{sec-fmt-manuscrito}
To answer the previously raised research questions, 40 scientific posters were analyzed. The posters were collected from the official websites of eight private schools in Amman. They were published between 2019 and 2023. In order to ensure that the used data is representative, the used posters have been chosen after ensuring that they meet the criteria which include: the use of colours, design, creativity, neatness and the written material. The reliability and authenticity of the data sources were ensured by using a software called MAXQDA. It is worth mentioning that the eight schools located in west Amman. A qualitative approach was used in data analysis to reveal frequencies, visual and linguistic metadiscourse markers and to present examples of analyzed multimodal texts. Based on \posscite{hyland_metadiscourse:_2005} models of metadiscourse and \posscite{kumpf_visual_2000} visual metadiscourse, the study compared the use of these two different types of metadiscourse, and explained how the use of visual metadiscourse plays a crucial role, hand by hand with the linguistic metadiscourse, in conveying persuasive messages and enticing parents to enroll their children in one school rather than another. The analyzed data were validated by three experts in linguistics and discourse analysis. The experts have PhDs in applied linguistics. Rigorous classification, aligned with the visual and linguistic metadiscourse markers, is ensured through validation by three experts, enhancing objectivity. The process of validation consisted of two phases. Phase 1 focuses on ensuring that the selected posters meet the criteria of selection and phase 2 focuses on ensuring that the analyzed data have been examined correctly based on \posscite{hyland_metadiscourse:_2005} models of metadiscourse and \posscite{kumpf_visual_2000} visual metadiscourse.

\section{Procedure of analysis}\label{sec-formato}
In this study, the data were analyzed qualitatively by using a software called MAXQDA, which is available freely for the academic staff at Isra University. This study attempts to shed light on the importance of visual and linguistic metadiscourse in the genre of online advertising. The interpretation of the visual and linguistic metadiscourse are considered according to the context in which they appear. According to the systemic functional linguistics by \textcite{halliday_categories_1961}, the language is seen as a semiotic social system which means that the meaning is based on the social context in which it appears. To achieve this aim, the visual and linguistic metadiscourses in a number of Jordanian school promotions were analyzed carefully and systematically. The analysis was conducted in two phases: the first phase examines the use of visual metadiscourse and the second phase scrutinizes the use of linguistic metadiscourse. The analysis of visual metadiscourse was conducted based on \posscite{kumpf_visual_2000} model and the analysis of linguistic metadiscourse was conducted based on \posscite{hyland_metadiscourse:_2005} two metadiscourse models.

\subsection{Findings and discussion of visual metadiscourse}

The analysis of the corpus yielded significant results that indicated how the visual and linguistic metadiscourse devices were constructed in school promotions. \Cref{tbl1} below shows the results of the analysis based on the frequency of occurrence of each visual and linguistic metadiscourse device.

\begin{table}[h!]
\centering
\begin{threeparttable}
\caption{Frequency of occurrence of the visual and linguistic metadiscourse devices in the corpus.}
\label{tbl1}
\footnotesize
\begin{tabular}{
l 
>{\raggedright\arraybackslash}p{1.5cm} 
>{\raggedright\arraybackslash}p{1.5cm} 
>{\raggedright\arraybackslash}p{1.5cm} 
>{\raggedright\arraybackslash}p{1.5cm} 
>{\raggedright\arraybackslash}p{1.5cm} 
l}
\toprule
&  & Visual Metadiscourse based on \posscite{kumpf_visual_2000} model &  \multicolumn{3}{p{4.5cm}}{Linguistic Metadiscourse based on \posscite{hyland_metadiscourse:_2005} model} & \\
\midrule
& Device & Occurrence & Interactive dimension & Occurrence & Interactional dimension & Total \\
1 & First impression & 3 & Transitions & 16 & Hedges & 0 \\
2 & Heft & 3 & Frame markers & 0 & Boosters & 0 \\
3 & Convention & 7 & Endophoric markers & 0 & Attitude markers & 16 \\
4 & External skeleton & 2 & Evidential & 0 & Self-mentions & 11 \\
5 & Chunking & 4 & Code glosses & 1 & Engagement markers & 41 \\
6 & Consistency & 6 & & & & \\
7 & Expense & 2 & & & & \\
8 & Attraction & 7 & & & & \\
9 & Interpretation & 4 & & & & \\
10 & Style & 4 & & & & \\
\bottomrule
\end{tabular}
\source{Own elaboration.}
\end{threeparttable}
\end{table}

The results in the table above show the visual and linguistic metadiscourse devices found in the corpus. It can be seen that conventions, attraction and consistency categories of visual metadiscourse were used more frequently than the other categories.

The findings also show that there are minimal occurrences of interactive categories and high occurrences of interactional metadiscourse, particularly Engagement markers and Attitude markers. In what follows, we present a qualitative analysis of the school promotions by analysing how each visual metadiscourse marker and linguistic metadiscourse marker were employed in the data.

\subsubsection{First impression}\label{sec-modelo}
First impression generally affects the choice of the parents about the school, and is therefore considered one of the most important factors in the persuasion process. First impression as one of the visual metadiscourses features is directly related to the expectations of the parents, and thus schools in Jordan pay great attention to it. \textcite[p.318]{kostenlnick_designing_1998} write, “We see documents before we read them: this initial encounter evokes an aesthetic response but one with immediate practical consequences”. In other words, the first impression can respond to the receptors either positively or negatively, which means that a great portion of the efforts should be devoted by school to the school promotions. \textcite[p.~406]{kumpf_visual_2000} stated that “[a] document cannot have a first impression. It may fail to move the reader at all, may evoke a dislike, or may make a positive impression, but it will have some effect on the reader”. The following advertisement, taken from the school corpus, shows how to use this feature.

\begin{figure}[htbp]
\centering
\begin{minipage}{0.5\textwidth}
\includegraphics[width=\linewidth]{Fig1.png}
\caption{Advertisement 1}
\label{fig1}
\source{\url{https://www.bing.com/}}
\end{minipage}
\end{figure}

\Cref{fig1} is extracted from one of the private schools in Jordan. The first impression is captured here through the use of the image of a child flying above the clouds on a book, and the written statement “let them spread their wings with bia”. The aim of this advertisement is to persuade the parents to choose this school over another by explaining that students will develop their ideas at this school. At first glance, parents can guess that the type of education in this school is different and does not follow the traditional way of education.

\subsubsection{Heft}\label{sec-organizacao}
\textcite[p.~408]{kumpf_visual_2000} claimed that heft and first impression are related to each other and added that “[h]eft has size, shape, quantity, and often weight, factors that readers may expect in a document, a circumstance that then imposes constraints on the writer to fulfil them”. The following \Cref{fig2} comes from the school promotions of a private school in Jordan. It can be seen that the idea of this promotion is to celebrate Labour Day in Jordan. The use of heft as one of the features in this context can give parents an idea of the school’s interest in celebrating international and national events. \textcite[p.~268]{niemi_discussing_2014} stated that “It can be concluded that festivals are important parts of school life and due to their multiple social dimensions, they encompass significant opportunities for intercultural learning if they are carried out in a meaningful way”.

\begin{figure}[htbp]
\centering
\begin{minipage}{0.5\textwidth}
\includegraphics[width=\linewidth]{Fig2.png}
\caption{Advertisement 2}
\label{fig2}
\source{\url{https://www.bing.com/}}
\end{minipage}
\end{figure}

\subsubsection{Convention}\label{sec-organizacao-latex}
Parents’ expectations of school promotion are considered essential and the visual metadiscourse feature that reflects these expectations is called Convention. According to \textcite[p.~97]{mancini_cinematic_2005}, convention “describes what readers expect from the appearance of a document in relation to what they actually have before their eyes, which influences their perception of it”. \textcite{kumpf_visual_2000} stated that what applies to the paper advertisements applies to the online advertisements. \textcite[p.~31]{al-subhi_metadiscourse_2022} claimed that “any reader can identify a document as an ad as a result of using verbal and visual conventions. These conventions include images, color, font, language, and integration of text and images”. The following figures are from school promotions and show many conventions of advertising texts.

\begin{figure}[htbp]
\centering
\begin{minipage}{0.5\textwidth}
\includegraphics[width=\linewidth]{Fig3.png}
\caption{Advertisement 3}
\label{fig3}
\source{\url{https://www.bing.com/}}
\end{minipage}
\end{figure}

\textcite{janoschka_web_2004} stated that the combination of linguistic and visual features in online advertisements plays a vital role in producing a very persuasive advertisement. Many schools adopt the strategy of combining these features in order to attract the attention of the parents. In \Cref{fig3}, the use of the personal pronoun (we) in the sentence “we are closed”, which is considered a linguistic feature, and the use of the image of the calendar, which is considered a visual feature, give the reader an indication that there is an off day. In other words, the combination of the linguistic feature and visual feature in \Cref{fig3} has been used to deliver a message.

\subsubsection{Chunking}\label{sec-titulo}
\textcite[p.~32]{al-subhi_metadiscourse_2022} claimed that “Chunking can be considered as the most conventional format for social media advertising in which targeted audience can run their eyes across the chunked parts of an ad and still get the gist”. In other words, it can be noticed that the idea of chunking revolves around connecting different visual parts in the advertisement to help the receptor identify the gist of it. Chunking is one of the visual metadiscourses markers that has a great role in the way that the readers handle the content. According to \textcite[p.~419]{kumpf_visual_2000} “Chunking as interpreted through visual metadiscourse helps provide visual relief in a document by allowing the readers to process the content in parts rather than as a continuous flow of text without breaks”.

\Cref{fig4} shows how chunking has been used as a visual metadiscourse marker to capture the attention of the parents and to reinforce the importance of the fun day. The division of the texts into more than one visual part reflects how chunking can be employed as a visual metadiscourse marker to identify the constituent parts of the advertisement.

\begin{figure}[htbp]
\centering
\begin{minipage}{0.5\textwidth}
\includegraphics[width=\linewidth]{Fig4.png}
\caption{Advertisement 4}
\label{fig4}
\source{\url{https://www.facebook.com/share/ezbJjNx1nhsC867k/?mibextid=ox5AEW}}
\end{minipage}
\end{figure}

\subsubsection{Consistency}\label{sec-autores}
\textcite[p.~412]{kumpf_visual_2000} explained that “consistency helps readers discern cohesion, seeing the document as a unified whole whose parts support a common theme instead of wandering and wavering among embellishments and stylistic choices”. In other words, one can notice that consistency as a visual metadiscourse can be viewed as a notification to the receptors that all the elements in the advertisement form a unity that functions in the same way to reflect the intended message. \textcite{al-subhi_metadiscourse_2022} added that consistency as a visual metadiscourse feature can be noticed in the use of the same colors, the size of images, and in maintaining a hierarchy between headings or information. 

\Cref{fig5} is used to celebrate the Independence Day of the Hashemite Kingdom of Jordan. The decoration of the number 77 in the advertisement, with the Historical Sites inside the number, shows how consistency as a visual metadiscourse feature can be used.

\begin{figure}[htbp]
\centering
\begin{minipage}{0.5\textwidth}
\includegraphics[width=\linewidth]{Fig5.png}
\caption{Advertisement 5}
\label{fig5}
\source{\url{https://www.bing.com/}}
\end{minipage}
\end{figure}

\subsubsection{Expense}\label{sec-idioma}
\textcite[p.~413]{kumpf_visual_2000} found that “The physical and aesthetic realms of expense affect the reader’s reception of a document, and expense is the part of visual metadiscourse most influenced by money; that is, the expense of paper, printing, and visuals, among other things”. In other words, the connection between the parents’ reception of the school advertisements and the aesthetic value cannot be ignored as a visual metadiscourse. \textcite[p.~34]{al-subhi_metadiscourse_2022} added that “SM [social media] channels appear to be excellent venues for advertisers to reach a variety of consumers and boost their sales at the least possible expense”.

\Cref{fig6} is directed to the parents to persuade them to let their children join the campus for a day filled with nonstop amusement and fun experiences. In this advertisement, the expense shows the aesthetic and economic value of the advertisement that may influence the parents’ perception of joining this day greatly. The Extracurricular activities are considered very important in the parents’ choice of the school.

\begin{figure}[htbp]
\centering
\begin{minipage}{0.5\textwidth}
\includegraphics[width=\linewidth]{Fig6.png}
\caption{Advertisement 6}
\label{fig6}
\source{\url{https://www.bing.com/}}
\end{minipage}
\end{figure}


\subsubsection{Attraction}\label{sec-resumo}
It can be said that Attraction as a visual metadiscourse feature can be considered one of the most important visual metadiscourses, since the main goal of school promotions is to attract the attention of parents so that they make a decision and enrol the student in a particular school. \textcite{Bgdan_2020} argued that the visual resources employed in visual representations organize the content, guide users, attract their attention and establish direct communication between the author/presenter and the viewer. \textcite{kumpf_visual_2000} found that the use of visual metadiscourse can drive away the readers who are not attracted to advertising by its content. \textcite[p.~34]{al-subhi_metadiscourse_2022} stated that “attraction concerns the setting and maintaining of a visual standard to capture and keep the attention of the reader through the document to its end”.

\Cref{fig7} shows how the use of the different font sizes and lines can attract the attention of the readers, in addition to the image of an open book and arrows that reflects the message the school is trying to convey. This message, by saying that “Good Education saves lives”, may mean that choosing a school that offers good education can help students set the right path for their future.

\begin{figure}[htbp]
\centering
\begin{minipage}{0.5\textwidth}
\includegraphics[width=\linewidth]{Fig7.png}
\caption{Advertisement 7}
\label{fig7}
\source{\url{https://www.bing.com/}}
\end{minipage}
\end{figure}


\subsubsection{Interpretation}\label{sec-secoes}
Interpretation as a visual metadiscourse feature is considered very important for understanding the core of the school promotions. \textcite{kumpf_visual_2000} stated that interpretation plays a vital role in establishing the connection between the image and the text within an advertisement. \textcite[p.~34]{al-subhi_metadiscourse_2022} explained that interpretation helps combine text and image and also allows text to support visual elements in a document. In most of the data analyzed, the interpretation as a text plays a crucial role.

\Cref{fig8} shows how interpretation is used as a visual metadiscourse marker. It can be seen that the image of the yellow bus next to the written sentence ‘The safety of our students is our responsibility’ form a unit, and helps parents understand that the transportation service at this school is very safe. Therefore, this can be seen as a persuasive tool to choose this school over another.

\begin{figure}[htbp]
\centering
\begin{minipage}{0.5\textwidth}
\includegraphics[width=\linewidth]{Fig8.png}
\caption{Advertisement 8}
\label{fig8}
\source{\url{https://mostaql.com/}}
\end{minipage}
\end{figure}

\subsubsection{Style}\label{sec-format-simple}
\textcite{kumpf_visual_2000} stated that style as a metadiscourse marker refers to the use of specific features to indicate a particular brand. \textcite[p.~36]{al-subhi_metadiscourse_2022} explained that “the style and the art of texts can be fundamental in creating a successful identity for the brand resulting in the success of the ad”.

\begin{figure}[htbp]
\centering
\begin{minipage}{0.5\textwidth}
\includegraphics[width=\linewidth]{Fig9.png}
\caption{Advertisement 9}
\label{fig9}
\source{\url{https://web.facebook.com/JordanIIS/posts/2703699719661176/}}
\end{minipage}
\end{figure}

The data collected from this school show that all its promotions were written in the same colours of orange, white and blue which reflect the colours of the school logo. So, it can be seen that the use of these three colours represents a milestone for this school. People may guess that these promotions belong to this school even before reading its name.

\subsubsection{External skeleton}\label{sec-links}
The external skeleton refers to the way the images are arranged in the school promotions. \textcite[p.~32]{al-subhi_metadiscourse_2022} explained that “it shows how a document is assembled and in the case of an ad, it would be, for instance, the placement of the heading or the core message versus the location of the surrounding information”. According to \textcite[p.~411]{kumpf_visual_2000}, “the external skeleton relies much on chunking because the visual separation caused by chunking helps identify the parts of the skeleton”.

\Cref{fig10} is used to celebrate the birthday of the Islamic Prophet Mohammad. It shows that the position of the name of prophet “Mohammad” compared to other words plays a vital role in showing the importance of this part of the image compared to other parts of the image.

\begin{figure}[htbp]
\centering
\begin{minipage}{0.5\textwidth}
\includegraphics[width=\linewidth]{Fig10.png}
\caption{Advertisement 10}
\label{fig10}
\source{\url{https://www.bing.com/}}
\end{minipage}
\end{figure}

\subsection{Findings and discussion of Linguistic Metadiscourse}\label{sec-outras-estr}
According to \posscite{hyland_metadiscourse:_2005} model, metadiscourse consists of two dimensions; the interactive dimension and the interactional dimension. The first one focuses on “the ways writers signal the arrangement of their texts based on their appreciation of the reader’s likely knowledge and understandings” \cite[p.~43]{hyland_metadiscourse:_2005}. The second one focuses on “the writer’s explicit interventions to comment on and evaluate material” \cite[p.~44]{hyland_metadiscourse:_2005}. In other words, the interactive dimensions refer to the methods that the writer uses in order to attract the readers towards his or her own notions through the text, while the interactional dimension refers to the responses of the readers to the content, and to the strong tie that the writers try to establish among their propositions and the readers.

\textcite{ebrahimi_role_2018} stated that interactive metadiscourse includes transitions which show the connection between the main clauses such as: and, but; frame markers that relate to the actions and stages of the discourse such as: in conclusion, first; endophoric markers that relate to the secondary parts of the text such as: see \Cref{tbl1}, see \cref{fig1}; evidential that refer to data from other contexts such as mention of the documentary sources; and code glosses that form the propositional meanings, e.g., such as, for example.

\textcite{al-subhi_metadiscourse_2022} explained that the interactional dimension includes the following. Hedges refers to open dialogue such as: maybe and might. Boosters focuses on certainty and a close dialogue such as: indeed and in fact. Attitude markers reflect the author’s attitude towards a particular idea such as: I believe, in my opinion. Engagement markers try to connect with the receptors such as: note, look. Self-mentions refer to mentioning the author through the use of pronouns that refer to author such as: I, we, my, me, our.

\textcite{hyland_metadiscourse:_2005} suggests that the interactional dimension should shed light on the expression of speaker and engagement with audience. Therefore, he believes that interaction can be divided into stance and engagement. According to \textcite[p.~5]{hyland_stance_2005b}, stance can be seen as “an attitudinal dimension and includes features which refer to the ways writers present themselves and convey their judgements, opinions, and commitments”. Engagement, on the other hand, is 

\begin{quote}
    An alignment dimension in which authors acknowledge and connect with others, recognizing the presence of their readers, engage them with their argument, focusing their attention, acknowledging their uncertainties, including them as discourse participants, and guiding them to interpretations \cite[p.~176]{hyland_stance_2005b}.
\end{quote}


\subsubsection{Interactive resources}\label{sec-listas}
As already mentioned, there are five categories for interactive metadiscourse: Transitions, Frame markers, Endophoric marker, Evidential and Code glosses. Analysis of the data revealed that the different categories of interactive resources of the metadiscourse did not appear. \textcite[p.~27]{al-subhi_metadiscourse_2022} stated that the minimal use of interactive resources is “due to the genre conventions and the short texts of advertising discourse”.

The transitions occurred most frequently in the data. There are 16 transitions and only one code gloss. Below are some examples:

\begin{itemize}
    \item Nonstop amusement and fun. (transition)
    \item Get up close and personal with adorable animals at our petting zoo. (transition)
    \item Our school has many facilities such as: swimming pool, language club… (code gloss)
\end{itemize}

\subsubsection{Interactional resources}\label{sec-listas}
\textcite{kotler_marketing_2007} clarified that the genre of advertisement has specific requirements. The language used should be precise and clear, conveying specific messages aimed at influencing people. The data showed that engagement markers and attitude markers were the most common, followed by self-mentions, while no occurrences were observed for hedges and boosters. \textcite[p.~28]{al-subhi_metadiscourse_2022} justified that “the highest frequency of Engagement markers and attitude markers suggest that the main concern of advertisers is the consumer as these features are an effective means of persuasion”.

The following sections will present the results in more details.


\subsubsection{Self-mentions}\label{sec-figuras-tabelas}
Self-mentions as interactional resources were used in the selected data and appeared frequently. Data show that self-mentions are recognized through the adoption of possessive pronouns, and first-person pronouns in the singular and plural. The total number of self-mentions is 11.

\begin{itemize}
    \item We are closed.
    \item Happy labor day to our future workers.
\end{itemize}

\subsubsection{Attitude markers}
\textcite[p.~31]{abdul-qadir_attitude_2015} stated that “Attitude markers refer to certain expressions that are used in a text that reflect writers’ position toward both the content in the text and the reader”. There are important grammatical categories used in the genre of advertising such as attitudinal nouns, attitudinal verbs, attitudinal adjectives and attitudinal adverbs. In the selected data, the attitudinal adjectives were more common than the other grammatical categories.

\textcite[p.~174]{adel_metadiscourse_2008} defines attitude markers as “the importance of something, the interest of something, its appropriateness, and the personal emotional concomitants of linguistic material”. The total number of occurrences of attitude markers is 16. Below are some examples:

\begin{itemize}
    \item Good education.
    \item Creative, new generation.
\end{itemize}

\subsubsection{Engagement markers}
\textcite{hyland_metadiscourse:_2005} noted that engagement markers aim to shed light on the role of the receptors as discourse participants, adding that there are four types of engagement markers: readers pronouns, personal asides, questions and directives. Readers pronouns, according to \textcite{hyland_metadiscourse:_2005}, refer to the use of reader pronouns such as (you, yourself, our, your) to give an indication that there is a strong connection between the producer and the customer. The total number of occurrences of readers pronouns is six.

\begin{itemize}
    \item It is your decision.
    \item Design your future.
\end{itemize}

\textbf{Personal asides}: \textcite[p.~28]{al-subhi_metadiscourse_2022} stated that they refer to “interrupt the argument to offer a comment on what has been said”. The total number of occurrences of personal asides is two.

\textbf{Questions}: \textcite{hyland_metadiscourse:_2005} explained that questions are used as engagement markers to encourage the receiver to engage with the core of the ads and, in many cases, reflect the view of the producer or writer. The total number of questions is six. 

\begin{itemize}
    \item Have you chosen a school for your children or not?
    \item Why our school?
\end{itemize}

\textbf{Directives}: \textcite{hyland_metadiscourse:_2005} stated that directive markers play an important role in motivating receptors to engage in the situation as the producer or writer prefers and that they are formed in the imperatives, and modals of obligations. The data showed that the directive markers had the highest frequency compared to other linguistic metadiscourse and were used as essential elements of persuasive language. The total number of occurrences of directive markers is 27. Below are some examples: 

\begin{itemize}
    \item Follow us on Facebook and Instagram
    \item Tag us on Facebook and Instagram
    \item Take a selfie with our ad in City mall.
\end{itemize}

The use of imperative verbs as directive markers reflects the school’s intention to attract the attention of students and parents towards its social media platforms. \textcite[p.~29]{al-subhi_metadiscourse_2022} claimed that directives are used to “instruct the targeted audience to perform certain actions arranged by the advertisers”. And this function can be observed in the above-mentioned examples.

\section{Discussion}

\textcite[p.~264]{rohim_marketing_2019} explained that:

\begin{quote}
    Marketing has an important role in a company or educational institution, because marketing is the main activity of the company to distribute products or services produced to the hands of consumers, therefore the school is required to make the right strategy in marketing products that have an impact on the promotion mix, one the way that can be done is by doing a marketing mix and also a promotional mix.
\end{quote}

The quote above shows that there is a big trend in schools to have a marketing plan, and the use of promotions is a part of this plan. Therefore, these promotions should be distinctive, and the use of both visual and linguistic metadiscourse markers can make school promotions more attractive.

The results presented above demonstrate the visual metadiscourse and the linguistic metadiscourse used in Jordanian school promotions. The Jordanian school promotions used all types of visual metadiscourse in the data, particularly attraction and convention. The findings also showed that directives had the highest frequencies compared to other linguistic metadiscourse because they were used as essential elements of persuasive language.

The findings also showed that Jordanian schools resort to the use of visual and linguistic metadiscourse to attract the attention of the parents, so these metadiscourse markers can be said to have persuasive functions. \textcite{duranti_linguistic_1997} stated that the main goal of language use is not only to communicate with one another but it is also considered an important part of cultural practices; and it can be seen that the use of school promotions on social media platforms has become an essential element in the Jordanian community. As \textcite[p.~214]{sapir_status_1929} stated, “Language is not merely a reproducing instrument for voicing ideas, but rather is the shaper of ideas”. This quote emphasizes that the role of language doesn’t have a limit, it can be used for transferring ideas, plans, thought and even cultures.

\section{Conclusion}
The study has examined the use of both visual and linguistic metadiscourse markers in the academic posters and has analysed how the use of this metadiscourse may impact the process of persuasion and encouraging of parents to choose a particular school over another. The results showed that all types of visual metadiscourse, particularly attraction, convention, and consistency, were highly marked in the data investigated. The results also showed that directives had the highest frequency compared to other linguistic metadiscourse, and were used as essential elements of persuasive language. The study concluded that the role of visual and linguistic metadiscourse in reflecting the goals of academic institutions, engaging audience, and attracting parents’ interest, cannot be ignored. Consequently, visual and linguistic metadiscourse plays an effective and important role in persuasive writing. The study has some implications for the development of the analysis of multimodal media discourse. Despite the comprehensive nature of this study, certain limitations should be acknowledged, influencing the interpretation and generalization of findings. Firstly, the study primarily focused on the content of advertisements, lacking detailed information about the specific cultural, economic, or political contexts that might influence advertising strategies. A more holistic understanding would require supplementary data. Secondly, the reliance on websites of the schools as primary data sources introduces platform-specific biases. Other digital platforms, traditional media, or offline channels may contribute additional dimensions to the persuasive strategies employed by the schools.

\printbibliography\label{sec-bib}
%conceptualization,datacuration,formalanalysis,funding,investigation,methodology,projadm,resources,software,supervision,validation,visualization,writing,review
\begin{contributors}[sec-contributors]
\authorcontribution{Aseel Alshbeekat}[conceptualization,datacuration,methodology,writing,review]
\authorcontribution{Anas Awwad}[formalanalysis,investigation,resources,validation,writing]
\end{contributors}
\end{document}
