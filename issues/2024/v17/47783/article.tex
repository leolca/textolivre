% !TEX TS-program = XeLaTeX
\documentclass[portuguese]{textolivre}

%\usepackage{enumerate} % To create numbered lists.
%\usepackage[inline]{enumitem}  %To create numbered in line lists.
%\usepackage{easyReview} %To make reviews on specific parts of the text
\usepackage{array} %It is necessary to the tables of this work.
\usepackage{threeparttable} %It is necessary to the tables of this work.
\usepackage{tabularx} %It is necessary to the tables of this work.
\usepackage{longtable} %It is necessary to the tables of this work.

% metadata
\journalname{Texto Livre}
\thevolume{17}
%\thenumber{1} % old template
\theyear{2024}
\receiveddate{\DTMdisplaydate{2023}{8}{20}{-1}}
\accepteddate{\DTMdisplaydate{2023}{10}{15}{-1}}
\publisheddate{\today}
\corrauthor{Lisidna Almeida Cabral}
\articledoi{10.1590/1983-3652.2024.47783}
%\articleid{NNNN} % if the article ID is not the last 5 numbers of its DOI, provide it using \articleid{} commmand 
% list of available sesscions in the journal: articles, dossier, reports, essays, reviews, interviews, editorial
\articlesessionname{dossier}
\runningauthor{Sampaio et al.}
%\editorname{Leonardo Araújo} % old template
\sectioneditorname{Daniervelin Pereira}
\layouteditorname{João Mesquista}

\title{Adaptação e evidências de validade do Questionnaire for Assessing Educational Podcasts (QAEP) para o português brasileiro: um estudo indisciplinar em letramento em saúde}
\othertitle{Adaptation and validity evidence of the Questionnaire for Assessing Educational Podcasts (QAEP) into Brazilian Portuguese: an undisciplined study in health literacy}

\author[1]{Helena Alves de Carvalho Sampaio~\orcid{0000-0003-2704-3265}\thanks{Email: \href{mailto:dr.hard2@gmail.com}{dr.hard2@gmail.com}}}
\author[2]{Nukácia Meyre Silva Araújo~\orcid{0000-0003-1951-0417}\thanks{Email: \href{mailto:nukacia.araujo@uece.br}{nukacia.araujo@uece.br}}}
\author[1]{Patrícia Cândido Alves~\orcid{0000-0002-2113-7423}\thanks{Email: \href{mailto:patricia7alvess@gmail.com}{patricia7alvess@gmail.com}}}
\author[1]{Lisidna Almeida Cabral~\orcid{0000-0002-1622-9577}\thanks{Email: \href{mailto:lissidna@yahoo.com.br}{lissidna@yahoo.com.br}}}
\affil[1]{Universidade Estadual do Ceará, Programa de Pós-Graduação em Saúde Coletiva, Fortaleza, CE, Brasil.}
\affil[2]{Universidade Estadual do Ceará, Programa de Pós-Graduação em Linguística Aplicada, Fortaleza, CE, Brasil.}

\addbibresource{article.bib}

\begin{document}
\maketitle

\begin{polyabstract}
\begin{abstract}
Adotando a perspectiva interdisciplinar da linguística aplicada
\cite{moita2006} e assumindo a relação intrínseca entre linguagem e práticas
sociais, neste artigo, apresentamos a validação de um instrumento de avaliação
de \textit{podcast} educacional cujo conteúdo vise ao desenvolvimento do
letramento em saúde. O uso de \textit{podcasts} como estratégia de educação em
saúde vem aumentando nos últimos anos. Os profissionais dessa área precisam,
por outro lado, analisar a adequação desse tipo de recurso educacional digital
(RED). Para análise deste tipo de RED, foi desenvolvido, no contexto
brasileiro, um instrumento destinado à avaliação de \textit{podcast} educativo
por profissionais de saúde. No entanto, no Brasil, não existe ainda um
instrumento específico para avaliação de \textit{podcast} pelo público ao qual
se destina esse tipo de recurso educacional. O objetivo deste estudo então foi
traduzir, adaptar e validar, para o português brasileiro, um instrumento com
esta finalidade, no caso, o Questionnaire for Assessing Educational Podcasts
(QAEP). Trata-se de um estudo metodológico, que se desenvolveu a partir das
seguintes fases: 1) foi obtida autorização do autor principal para tradução e
uso do instrumento; 2)  em seguida, foram executadas etapas de tradução por
tradutor profissional e de retrotradução do instrumento por falantes nativos da
língua-cultura de partida, que passou a ser chamado de Instrumento de Avaliação
de \textit{Podcast} Educativo (IAPE); 3) um comitê de especialistas (três
profissionais da área de saúde e uma linguista aplicada) discutiu a
conformidade do protocolo com os objetivos da pesquisa e da área após a
tradução e 4) finalmente, foi feita a validação da adaptação por juízes
especialistas.

\keywords{Letramento em saúde \sep Podcast \sep Podcast educativo \sep Validação \sep Recurso educacional digital}
\end{abstract}

\begin{abstract}
Adopting the interdisciplinary perspective of applied linguistics
\cite{moita2006} and assuming the intrinsic relationship between language and
social practices, in this article, we present the validation of an educational
podcast assessment instrument whose content aims to develop health literacy.
The use of podcasts as a health education strategy has increased in recent
years. Professionals in this area need, on the other hand, to analyze the
suitability of this type of digital educational resource (DER). To analyze this
type of DER, an instrument was developed in the Brazilian context for the
evaluation of educational podcasts by health professionals. However, in Brazil,
there is still no specific instrument for evaluating podcasts by the audience
for which this type of educational resource is intended. The objective of this
study was to translate, adapt and validate, into Brazilian Portuguese, an
instrument for this purpose, in this case, the Questionnaire for Assessing
Educational Podcasts (QAEP). This is a methodological study, which developed
from the following phases: 1) authorization was obtained from the main author
to translate and use the instrument; 2) then, translation steps were carried
out by a professional translator and back-translation of the instrument by
native speakers of the source language-culture, which came to be called the
Educational Podcast Assessment Instrument (EPAI); 3) a committee of experts
(three health professionals and an applied linguist) discussed the compliance
of the protocol with the research and area objectives after translation and 4)
finally, the adaptation was validated by expert judges.

\keywords{Health literacy \sep Podcast \sep Educational podcast \sep Validation \sep Digital educational resource}
\end{abstract}
\end{polyabstract}


\section{Introdução}\label{sec-introdução}
A produção e o consumo de \textit{podcasts}, tal como o \textit{podcast} educativo, vêm aumentando nos últimos anos, tanto no mundo, como no Brasil. Esse tipo de recurso educacional digital é de fácil utilização devido a não pressupor tempo e locais determinados para a visualização, à facilidade de armazenamento e de acesso em plataformas gratuitas e à possibilidade de papel ativo do aluno no processo de aprendizagem [quando os \textit{podcast} são produzidos pelos próprios alunos, por exemplo]. Embora seja um tipo de arquivo conhecido por usuários do ambiente digital, ainda não há um consenso para sua definição \cite{viana2020}. Há mais de uma década, \textcite{bottentuit2007} afirmaram que \textit{podcasts} seriam arquivos digitais de áudio, facilmente baixados e ouvidos em diferentes dispositivos. Segundo \textcite{ballsberry2018}, seu conteúdo pode englobar várias temáticas e seu objetivo é a transmissão da informação.

Considerando a noção de gêneros do discurso como tipos relativamente estáveis de enunciados que funcionam nas diferentes esferas da atividade humana \cite{bakhtin2016}, admite-se aqui que o \textit{podcast} é enunciado relativamente estável, construído social e culturalmente, e que como tal procede de alguém e se dirige a alguém \cite{volochinov2013}. Em outras palavras, entendemos o \textit{podcast} como um gênero discursivo oral que se caracteriza por sua dinamicidade multissemiótica, \enquote{uma vez que essas produções podem explorar diferentes recursos/mecanismos que são indiciadores de sentido, promovendo possibilidades de análise dos usos da modalidade oral} \cite[p. 53]{villarta2022}.

Sobre o surgimento e a circulação do gênero, a primeira produção de \textit{podcast} no Brasil aconteceu no ano de 2004. No ano seguinte, 2005, ocorreu a Conferência Brasileira de \textit{Podcasters}, como são chamados aqueles que produzem [do ponto de vista da linguagem] esse tipo de enunciado. Atualmente, podem ser encontradas produções bem diversificadas, com os mais diferentes objetivos (entreter, informar, ensinar, resenhar, entre outros) e sobre os mais variados temas em forma de \textit{podcast} \cite{abpod2021}. Entretanto, apesar da popularização desse gênero nas últimas décadas, parece limitado, ainda, o número de \textit{podcasts} educativos direcionados à população em geral.

O \textit{podcast} educativo é um tipo de recurso educacional digital (RED). Os REDs, segundo \textcite{araujo2019}, são entidades digitais que têm como objetivo o ensino. Os formatos em que se apresentam os REDs são bastante variados: \textit{softwares}, videoaulas, vídeos, áudios, \textit{podcast}, textos, infográficos, jogos, por exemplo. As principais características técnicas dos REDs são a granularidade (apresentação de pequenos recortes de conteúdo) e a reusabilidade (capacidade de ser usado diversas vezes e em diferentes situações e contextos de aprendizagem). 

Existem entidades digitais que já nascem como REDs, isto é, já são concebidas com o objetivo de ensinar, tal como um jogo educacional digital, por exemplo, ou como um \textit{podcast} educativo. Existem, por outro lado, recursos digitais que são, por assim dizer, convertidos em REDs, ou seja, que são transformados em algo para ensinar. Seria, por exemplo, o caso de um documentário do tipo jornalístico, que pode se converter em uma entidade digital para ensinar a respeito de determinado tema.

Assim como todo recurso pedagógico, os REDs, como recursos complementares de ensino, devem ser avaliados. Especificamente sobre instrumentos de avaliação de REDs em forma de \textit{podcast}, estudos como os de \textcite{semakula2017,semakula2020} e os de \textcite{sulistiawati2022} avaliam \textit{podcasts} educativos por meio de instrumentos criados para verificar qual foi a aprendizagem do conteúdo exposto no(s) episódio(s). É pertinente que se faça isso e que sejam avaliados conteúdos específicos, por exemplo.  No entanto, é necessário que haja um instrumento geral que permita a percepção do usuário sobre sua própria aprendizagem. Com a avaliação facilitada, tanto criadores de RED em forma de \textit{podcast}, como professores ou, no caso da área de saúde, profissionais da rede de atenção básica poderiam entender melhor como, para quem e por que utilizarem \textit{podcast} para promover a aprendizagem.

Por outro lado, mesmo considerando a avaliação específica importante, reconhecemos que, operacionalmente, é difícil desenvolver e validar diferentes instrumentos cujos conteúdos sejam específicos, mesmo considerando-se um recorte de área, como a da saúde, a cada \textit{podcast} construído. Por isso mesmo, propomos aqui a  adaptação e validação de um instrumento de avaliação geral. 

Destaque-se que os REDs em forma de \textit{podcast} educativo são recursos complementares a outras atividades educativas, o que permite que a avaliação específica de conteúdo possa ser realizada em outros momentos do contato educador-educando.

Especificamente quando se trata de letramento em saúde, os REDs normalmente são produzidos por equipes multidisciplinares, cujos responsáveis são profissionais de saúde. Esses recursos então já são produzidos com o objetivo de ensinar um conteúdo sobre saúde a leigos ou a profissionais.

Admitindo-se, numa perspectiva bakhtiniana, que a compreensão dialógica ativa exige a inserção do objeto a ser depreendido em um contexto dialógico \cite{bakhtin2017}, explicamos brevemente o contexto a partir do qual se concebe o letramento em saúde, exemplificando-o no cenário de funcionamento da Atenção Primária em Saúde (APS), que é a porta de entrada de usuários do sistema público e universal de saúde brasileiro, o Sistema Único de Saúde (SUS).

No âmbito do SUS, de acordo com {\textcite{marques2022}}, a APS é elemento fundamental para a qualidade do sistema, especialmente quando estruturada a partir da Estratégia de Saúde da Família. Este nível de atenção, por sua vez, apresenta como um dos atributos essenciais a longitudinalidade ou vínculo longitudinal da assistência.

 Na construção do vínculo ao longo do tempo entre usuários do SUS e profissionais de equipes multidisciplinares de APS, a comunicação é essencial para que se estabeleça relação de confiança e responsabilidade mútua. Sendo assim, como afirmam \textcite{sampaio2015}, é fundamental refletir sobre como as pessoas compreendem e utilizam as orientações da equipe profissional [de APS] para tomar decisões e agir no cuidado consigo mesmas. É este então o contexto das discussões que perpassam a noção de letramento em saúde.

Esse tipo específico de letramento diz respeito ao conjunto de conhecimentos e habilidades desenvolvidos pelas pessoas em interações sociais mediadas, segundo a \textcite{who2021}, por estruturas organizacionais [de saúde] e pela disponibilidade de recursos que permitem às pessoas acessar, entender, avaliar e usar informações e serviços de maneira a promover e manter boa saúde e bem-estar para si e para os que estão ao seu redor. Considerando-se a definição de letramento em saúde que acabamos de mencionar, os \textit{podcasts}, como REDs, podem atuar como recursos em prol do letramento em saúde de quem oo escuta.

Elencando características de \textit{podcast} educativo na área de letramento em saúde, \textcite{lopes2015} e \textcite{sampaio2021} afirmam que esse tipo de RED deve apresentar alguns atributos previamente definidos, antes de sua oferta ao público-alvo. Dentre elas, as principais seriam: 
\begin{enumerate}%[label=(\alph*)]
	\item tipo (orientação/instrução; expositivo/informativo; \textit{feedback}/comentário);
	\item formato (áudio, vídeo com locução ou apenas vídeo, \textit{videocast} e \textit{screencast});
	\item duração (curto – entre 1 a 5 minutos; moderado – 6 a 15 minutos; longo – mais de 15 minutos);
	\item estilo (linguagem utilizada: formal ou informal);
	\item funcionalidade (informar, divulgar, motivar, orientar);
	\item integralidade (envolvimento do autor (\textit{podcaster}) e ouvinte);
	\item definição do conteúdo (livre escolha do \textit{podcaster}).
\end{enumerate}

O \textit{podcast} educativo, como um RED, se diferencia dos demais porque sua finalidade é o ensino de temas/conteúdos neles contidos e não apenas a divulgação de informações. Nesta perspectiva, esses tipos de \textit{podcasts} necessitam ser avaliados e, para tanto, há a necessidade de instrumentos delineados para essa avaliação.

Recentemente, \textcite{muniz2021} desenvolveram um instrumento para validação de \textit{podcast} por profissionais da saúde, que nesse caso eram enfermeiros. O instrumento foi desenvolvido para avaliação de \textit{podcast} cujo conteúdo era referente à hanseníase, mas pode ser utilizado para validação de \textit{podcasts} com diferentes conteúdos educativos em saúde, substituindo-se o tema hanseníase pelo tema que estiver sendo avaliado. O instrumento desenvolvido por \textcite{muniz2021} já foi aplicado por \textcite{mota2021} e por \textcite{leite2022}.

Por outro lado, há poucos instrumentos desenvolvidos para avaliação de \textit{podcast} pelo público-alvo. Um desses instrumentos, que foi desenvolvido por \textcite{alarcon2020}, traz essa proposta. Trata-se do \textit{Questionnaire for Assessing Educational Podcasts} (QAEP). Os autores realizaram avaliação de episódios de \textit{podcast} sobre métodos e estatísticas de pesquisa, disciplina ministrada em um curso de Psicologia, da Universidade de Málaga, Espanha. O público que avaliou os episódios foi representado pelos alunos deste curso. O QAEP mostrou boas propriedades psicométricas e os autores sugeriram sua aplicação em outros contextos, envolvendo pessoas de diferentes graus de escolaridade. No Brasil, porém, não há ainda um instrumento validado para avaliação de \textit{podcasts} educativos por público-alvo.

Assim, o objetivo deste estudo foi traduzir, adaptar e validar o QAEP para o português brasileiro. Dividimos assim este escrito em quatro partes: esta introdução, na  qual apresentamos o \textit{podcast} como um recurso educacional digital utilizado para o letramento em saúde e objetivo do estudo aqui apresentado; a metodologia, em que descrevemos o percurso metodológico, bem como os aspectos éticos, empreendidos na  adaptação e validação do instrumento; os resultados e discussão, em que apresentamos as avaliações e sugestões feitas pelos juízes especialistas a respeito da tradução, assim como a discussão a respeito do Instrumento de Avaliação de \textit{Podcast} Educativo (IAPE), denominação dada ao instrumento em português a partir da adaptação e validação descritas neste artigo e, finalmente, trazemos a conclusão do estudo.

\section{Metodologia}\label{sec-metodologia}

Este estudo se caracteriza como metodológico. Estudos deste tipo são bastante utilizados na área de saúde e têm como \enquote{finalidade de: elaborar novos instrumentos ou ferramentas, criar protocolos assistenciais, além de traduzir, validar e adaptar instrumentos preexistentes}\cite[p. 1]{galvao2022}. A adaptação e validação do instrumento de avaliação de \textit{podcast} aqui apresentada é uma das etapas do projeto intitulado \enquote{Plano Conecta Saúde: aliando inovação tecnológica e letramento em saúde na luta contra as doenças crônicas não-transmissíveis}. O projeto é financiado pelo CNPq e foi aprovado pelo Comitê de Ética em Pesquisa da Universidade Estadual do Ceará, sob parecer 3.795.260, CAAE 69459317.0.0000.5534. Um dos objetivos específicos do projeto é a produção de \textit{podcasts} educativos. Um dos passos metodológicas para atingir esse objetivo é a adaptação do protocolo criado por \textcite{alarcon2020}. Os procedimentos adotados foram os que seguem.

Inicialmente, foi obtida autorização do autor principal, Rafael Alarcón, professor do curso de Psicologia da Universidade de Málaga, para adaptação e uso do QAEP. A autorização foi solicitada e também posteriormente concedida por \textit{e-mail}. No que diz respeito à adaptação, foi seguido o protocolo de \textcite{guillemin1993}, pesquisadores na área de saúde. O protocolo proposto pelos autores foi desenvolvido para atender uma demanda da própria área, no que diz respeito à necessidade de tradução e adaptação de instrumentos clínicos (originalmente criados em língua inglesa) em detrimento da criação de novos instrumentos para avaliar a qualidade de saúde de populações não falantes de inglês. O protocolo, chamado pelos autores de \enquote{Diretrizes para adaptação transcultural}, é composto três fases: 
\begin{enumerate}[label=(\alph*)]
	\item tradução, por profissionais especialistas em tradução, para a língua-cultura alvo;
	\item retrotradução, por tradutores nativos da língua-cultura de partida, para a língua inglesa;
	\item revisão, por um comitê multidisciplinar.
\end{enumerate}

Nos estudos da tradução, qualquer tradução implica uma leitura e uma interpretação. Por causa disso, é preciso que seja analisado o contexto em que essa tradução vai funcionar. Por essa razão, ela não é ditada por regras (objetividade) e nem também é dependente das idiossincrasias dos tradutores (subjetividade). É justamente por se levar em conta as questões que influenciam a tradução que afirmamos que toda tradução é intersubjetiva \cite{toury1995}. Foi a partir dessa perspectiva que a tradução do protocolo foi realizada. Destaque-se também que o olhar dialógico pode ser lançado sobre a tradução. Nas palavras de \textcite{sobral2008}, o ato de tradução 
\begin{quote}
envolve uma atividade de leitura de um texto numa dada língua que difere da leitura em geral porque é feita do ponto de vista de um profissional que, em vez de apenas entender o que lê ou responder/reagir ao que lê, deve enunciá-lo a outros interlocutores, tem de reconstituir/reconstituir/restituir o que lê em outra língua e que, portanto tem de penetrar em dois universos de discurso e colocá-los numa relação de interlocução [...] \cite[p. 7–8]{sobral2008}. 
\end{quote}

Neste artigo, enfoca-se principalmente a adaptação e validação feita pelo comitê multidisciplinar (composto pelas pesquisadoras responsáveis pela adaptação do instrumento e pelas juízas e juízes que participaram da etapa de validação). Na etapa de validação, foram analisadas a relevância e a pertinência de cada item do instrumento, considerando-se objetivo do próprio instrumento (avaliar \textit{podcasts} educativos) e o público a quem o documento se destina (população em geral), assumindo-se assim o enunciado \enquote{protocolo de avaliação de \textit{podcast}} como procedente de alguém e dirigido a alguém \cite{volochinov2013}.

Como primeira etapa do processo de adaptação, o instrumento foi traduzido, para o português brasileiro, por dois tradutores profissionais. No caso, foram escolhidos dois tradutores juramentados. Depois, essa tradução passou por um comitê de especialistas, composto por três nutricionistas e uma linguista aplicada, todas pesquisadoras responsáveis pela adaptação do instrumento. Esse comitê discutiu a conformidade do protocolo com os objetivos da pesquisa e da área. Assim, as pesquisadoras avaliaram as duas traduções, aprovaram uma tradução de consenso e encaminharam para dois tradutores nativos de língua inglesa, para realizar a retrotradução, a qual consiste em traduzir de volta o instrumento para a língua-cultura de partida. Após a retrotradução, o comitê reuniu-se novamente, definindo a versão final do instrumento, que passou a ser chamado Instrumento de Avaliação de \textit{Podcast} Educativo (IAPE). Este é composto por 20 questões, distribuídas em quatro fatores relativos ao \textit{podcast}: acesso e uso; \textit{design} e estrutura; adequação de conteúdo; e importância como recurso de aprendizagem.

Os critérios para seleção dos profissionais que participaram como juízes especialistas no comitê multidisciplinar seguiram aspectos discutidos por \textcite{alexandre2011} em relação a esse tipo de escolha. Definiu-se que o especialista deveria ter pelo menos uma produção científica e/ou prática sobre o tema (letramento em saúde e/ou \textit{podcast}) nos últimos cinco anos. Como produção científica ou prática, por sua vez, foram consideradas as seguintes situações: autoria de monografia, dissertação ou tese sobre o tema; orientação de monografia, dissertação ou tese sobre o tema; autoria ou coautoria de livros ou capítulos de livros sobre o tema; autoria ou coautoria de artigo sobre o tema; responsabilidade por disciplina de pós-graduação sobre o tema; atuação como \textit{podcaster}.

A busca pelos especialistas foi efetuada na Plataforma Lattes (disponível em \url{https://lattes.cnpq.br/}). Aplicados os critérios, foram escolhidos os primeiros seis profissionais que apareceram na busca. Além desses, foi pedida a indicação de mais um [que atendesse os mesmos critérios] a um dos próprios profissionais já selecionados. Em relação à quantidade de juízes que compõem o comitê multidisciplinar, não há uma recomendação única quanto ao número de juízes, mas há certo consenso em que deva haver ao menos seis especialistas avaliando o material \cite{pasquali2010}. Para este estudo então foram selecionados sete juízes. 

Os juízes receberam, por meio de \textit{e-mail}, um \textit{link} para acesso à avaliação do instrumento, além do termo de consentimento livre e esclarecido e um questionário de caracterização complementar aos dados disponíveis na Plataforma Lattes.

O instrumento foi avaliado quanto: 
\begin{enumerate}[label=(\alph*)]
	\item à relevância da presença de cada item no questionário
	\item quanto à sua clareza, através de uma escala tipo Likert, com as opções: não, baixa, média, alta e muito alta, sendo a elas atribuídas pontos, respectivamente, de 1 a 5.
\end{enumerate}

Adotou-se o valor acima de 3 pontos para aceitação e inclusão do item avaliado \cite{alarcon2020}. Os autores do instrumento original fizeram, ainda, uma avaliação do nível de concordância dos juízes na atribuição de pontos, estabelecendo um índice maior que 0,70 para aprovação.

\input{tex-resultadosediscussões.tex}
\section{Considerações finais}\label{sec-consideraçõesfinais}

Tivemos como objetivo neste estudo adaptar e validar um instrumento de avaliação de \textit{podcast} educacional cujo conteúdo vise ao desenvolvimento do letramento em saúde. Uma limitação deste estudo é relativa à escolha do instrumento a ser adaptado e avaliado. Como há poucos instrumentos psicométricos disponíveis para a avaliação de \textit{podcasts} educativos, não foi possível escolher em um universo maior um instrumento que seria mais apropriado para tal. Destaque-se que embora tenha sido pensado no contexto do letramento em saúde, o IAPE pode ser aplicado em diferentes contextos e públicos-alvo. Estudos futuros permitirão identificar a necessidade ou não de modificação do instrumento.  

Enfocamos neste escrito principalmente a adaptação e a validação do instrumento. As etapas da adaptação e validação contaram com a tradução por tradutor profissional e retrotradução do instrumento por falantes nativos da língua-cultura de partida; adaptação do instrumento, após a tradução, por um comitê de especialistas e a validação do instrumento em português brasileiro por juízas e juízes especialistas. Na adaptação e validação do instrumento foram considerados os aspectos intersubjetivos da tradução, destacando-se especialmente um olhar para o caráter interacional da linguagem. No processo de construção da arquitetura textual-discursiva do IAPE várias vozes se entrecruzam: a dos autores do QEAP, a dos tradutores, a das especialistas e a das juízas e juízes. Essas vozes e as dos novos interlocutores que usarão o protocolo continuarão o tecido da cadeia ininterrupta de enunciados que compõe a interação verbal.

\printbibliography\label{sec-bib}

%conceptualization,datacuration,formalanalysis,funding,investigation,methodology,projadm,resources,software,supervision,validation,visualization,writing,review
\begin{contributors}[sec-contributors]
\authorcontribution{Helena Alves de Carvalho Sampaio}[conceptualization,formalanalysis,projadm,methodology,supervision,writing,review]
\authorcontribution{Nukácia Meyre Silva Araújo}[conceptualization,formalanalysis,methodology,writing,review]
\authorcontribution{Patrícia Cândido Alves}[conceptualization,formalanalysis,methodology,writing,review]
\authorcontribution{Lisidna Almeida Cabral}[conceptualization,formalanalysis,methodology,writing,review]
\end{contributors}

\end{document}
