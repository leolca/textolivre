\section{Metodologia}\label{sec-metodologia}

Este estudo se caracteriza como metodológico. Estudos deste tipo são bastante utilizados na área de saúde e têm como \enquote{finalidade de: elaborar novos instrumentos ou ferramentas, criar protocolos assistenciais, além de traduzir, validar e adaptar instrumentos preexistentes}\cite[p. 1]{galvao2022}. A adaptação e validação do instrumento de avaliação de \textit{podcast} aqui apresentada é uma das etapas do projeto intitulado \enquote{Plano Conecta Saúde: aliando inovação tecnológica e letramento em saúde na luta contra as doenças crônicas não-transmissíveis}. O projeto é financiado pelo CNPq e foi aprovado pelo Comitê de Ética em Pesquisa da Universidade Estadual do Ceará, sob parecer 3.795.260, CAAE 69459317.0.0000.5534. Um dos objetivos específicos do projeto é a produção de \textit{podcasts} educativos. Um dos passos metodológicas para atingir esse objetivo é a adaptação do protocolo criado por \textcite{alarcon2020}. Os procedimentos adotados foram os que seguem.

Inicialmente, foi obtida autorização do autor principal, Rafael Alarcón, professor do curso de Psicologia da Universidade de Málaga, para adaptação e uso do QAEP. A autorização foi solicitada e também posteriormente concedida por \textit{e-mail}. No que diz respeito à adaptação, foi seguido o protocolo de \textcite{guillemin1993}, pesquisadores na área de saúde. O protocolo proposto pelos autores foi desenvolvido para atender uma demanda da própria área, no que diz respeito à necessidade de tradução e adaptação de instrumentos clínicos (originalmente criados em língua inglesa) em detrimento da criação de novos instrumentos para avaliar a qualidade de saúde de populações não falantes de inglês. O protocolo, chamado pelos autores de \enquote{Diretrizes para adaptação transcultural}, é composto três fases: 
\begin{enumerate}[label=(\alph*)]
	\item tradução, por profissionais especialistas em tradução, para a língua-cultura alvo;
	\item retrotradução, por tradutores nativos da língua-cultura de partida, para a língua inglesa;
	\item revisão, por um comitê multidisciplinar.
\end{enumerate}

Nos estudos da tradução, qualquer tradução implica uma leitura e uma interpretação. Por causa disso, é preciso que seja analisado o contexto em que essa tradução vai funcionar. Por essa razão, ela não é ditada por regras (objetividade) e nem também é dependente das idiossincrasias dos tradutores (subjetividade). É justamente por se levar em conta as questões que influenciam a tradução que afirmamos que toda tradução é intersubjetiva \cite{toury1995}. Foi a partir dessa perspectiva que a tradução do protocolo foi realizada. Destaque-se também que o olhar dialógico pode ser lançado sobre a tradução. Nas palavras de \textcite{sobral2008}, o ato de tradução 
\begin{quote}
envolve uma atividade de leitura de um texto numa dada língua que difere da leitura em geral porque é feita do ponto de vista de um profissional que, em vez de apenas entender o que lê ou responder/reagir ao que lê, deve enunciá-lo a outros interlocutores, tem de reconstituir/reconstituir/restituir o que lê em outra língua e que, portanto tem de penetrar em dois universos de discurso e colocá-los numa relação de interlocução [...] \cite[p. 7–8]{sobral2008}. 
\end{quote}

Neste artigo, enfoca-se principalmente a adaptação e validação feita pelo comitê multidisciplinar (composto pelas pesquisadoras responsáveis pela adaptação do instrumento e pelas juízas e juízes que participaram da etapa de validação). Na etapa de validação, foram analisadas a relevância e a pertinência de cada item do instrumento, considerando-se objetivo do próprio instrumento (avaliar \textit{podcasts} educativos) e o público a quem o documento se destina (população em geral), assumindo-se assim o enunciado \enquote{protocolo de avaliação de \textit{podcast}} como procedente de alguém e dirigido a alguém \cite{volochinov2013}.

Como primeira etapa do processo de adaptação, o instrumento foi traduzido, para o português brasileiro, por dois tradutores profissionais. No caso, foram escolhidos dois tradutores juramentados. Depois, essa tradução passou por um comitê de especialistas, composto por três nutricionistas e uma linguista aplicada, todas pesquisadoras responsáveis pela adaptação do instrumento. Esse comitê discutiu a conformidade do protocolo com os objetivos da pesquisa e da área. Assim, as pesquisadoras avaliaram as duas traduções, aprovaram uma tradução de consenso e encaminharam para dois tradutores nativos de língua inglesa, para realizar a retrotradução, a qual consiste em traduzir de volta o instrumento para a língua-cultura de partida. Após a retrotradução, o comitê reuniu-se novamente, definindo a versão final do instrumento, que passou a ser chamado Instrumento de Avaliação de \textit{Podcast} Educativo (IAPE). Este é composto por 20 questões, distribuídas em quatro fatores relativos ao \textit{podcast}: acesso e uso; \textit{design} e estrutura; adequação de conteúdo; e importância como recurso de aprendizagem.

Os critérios para seleção dos profissionais que participaram como juízes especialistas no comitê multidisciplinar seguiram aspectos discutidos por \textcite{alexandre2011} em relação a esse tipo de escolha. Definiu-se que o especialista deveria ter pelo menos uma produção científica e/ou prática sobre o tema (letramento em saúde e/ou \textit{podcast}) nos últimos cinco anos. Como produção científica ou prática, por sua vez, foram consideradas as seguintes situações: autoria de monografia, dissertação ou tese sobre o tema; orientação de monografia, dissertação ou tese sobre o tema; autoria ou coautoria de livros ou capítulos de livros sobre o tema; autoria ou coautoria de artigo sobre o tema; responsabilidade por disciplina de pós-graduação sobre o tema; atuação como \textit{podcaster}.

A busca pelos especialistas foi efetuada na Plataforma Lattes (disponível em \url{https://lattes.cnpq.br/}). Aplicados os critérios, foram escolhidos os primeiros seis profissionais que apareceram na busca. Além desses, foi pedida a indicação de mais um [que atendesse os mesmos critérios] a um dos próprios profissionais já selecionados. Em relação à quantidade de juízes que compõem o comitê multidisciplinar, não há uma recomendação única quanto ao número de juízes, mas há certo consenso em que deva haver ao menos seis especialistas avaliando o material \cite{pasquali2010}. Para este estudo então foram selecionados sete juízes. 

Os juízes receberam, por meio de \textit{e-mail}, um \textit{link} para acesso à avaliação do instrumento, além do termo de consentimento livre e esclarecido e um questionário de caracterização complementar aos dados disponíveis na Plataforma Lattes.

O instrumento foi avaliado quanto: 
\begin{enumerate}[label=(\alph*)]
	\item à relevância da presença de cada item no questionário
	\item quanto à sua clareza, através de uma escala tipo Likert, com as opções: não, baixa, média, alta e muito alta, sendo a elas atribuídas pontos, respectivamente, de 1 a 5.
\end{enumerate}

Adotou-se o valor acima de 3 pontos para aceitação e inclusão do item avaliado \cite{alarcon2020}. Os autores do instrumento original fizeram, ainda, uma avaliação do nível de concordância dos juízes na atribuição de pontos, estabelecendo um índice maior que 0,70 para aprovação.
