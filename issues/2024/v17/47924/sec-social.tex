\subsection{Social Life}

When we analyze agency from an ecological and complex perspective, it
becomes clear how this construct takes place in the interaction and
interrelationship between systems. In addition to the academic and
professional environments, mobile technology seems to mediate
participants' actions in their social lives, as it
allows them \textbf{to learn about various subjects and topics of
interest.}

I don't use apps to study for college. I use apps to
study things that interest me personally. For example, on the Amazon
Music app, I follow some podcasts about minimalism, umbanda and
psychoanalysis. On the YouTube app, I watch videos on these subjects as
well as interior design, nutrition and painting. (P6)

In addition, today we have countless language apps, cooking apps, in
short\ldots{} countless possibilities to learn new activities without
necessarily being at school. (P19)

Mobile technology allows participants to circulate in many different
milieus and learn from media about diverse knowledge, such as
"minimalism, \emph{umbanda} and psychoanalysis", "interior design,
nutrition and painting" (P6), and use apps "for languages, cooking"
(P19) etc. In this way, participants appear to recognize the potential
of mobile technology to learn and act in social practices that are
important to them beyond formal environments.

In their social lives, mobile devices allow individuals to act according
to their interests and needs, helping them \textbf{to connect with other
people and institutions}.

It's a great way to get in touch with people and
companies from different areas who offer free, quality content. From a
personal perspective, I learn a lot from TikTok. Anyone who thinks that
the app only offers dances and "content-free" accounts is mistaken.
Various content experts are present on the networks, even government
accounts, to encourage young people to learn about their rights.
It's possible to learn science, technology, languages,
have contact with different cultures, all through the app. (P1)

In the broad context, the cell phone has become a necessary tool for
day-to-day infrastructure. It is used to contact people, but also to
secure means of transportation, communicate with vans, make payments.
Banning a cell phone would make it so unfeasible that
it's completely unrealistic today. (P5)

But I have been able to realize and accept that the use of technology
can be very useful and reduce the distances between people. (P18)

In addition to acquiring knowledge that is relevant to social life, such
as "sciences, technologies, languages", participants recognize
opportunities to "have contact with people and companies from different
areas" and "with different cultures" (P1). Social engagement, through
platforms such as TikTok, proves to be an affordance perceived on mobile
devices. Acting on this perception can be viewed as a way for the
individual to become a better citizen (knowing their rights) and to
"contact people", entering other systems, providing ways of "securing
means of transportation, communicating with vans, making payments" (P5).
As a result, technology seems to reduce the "distances between people",
making its systems increasingly interrelated.
