\section{Data analysis}\label{sec-dataanalysis}

In this section we present the results of our data analysis and categorization process. We have therefore chosen to present the emerging patterns identified in the social spaces in which the participants circulate, underscoring patterns that seem to influence their agency in relation to mobile technologies. First, we present instances of agency in the academic lives of pre-service teachers based on the affordances perceived in mobile technology for the participants' routines as university students. Next, we discuss excerpts that show agency in their professional lives, in their daily lives as practicing teachers. Finally, we present some instances of agency that reinforce the ubiquity of mobile technology and show how it influences the participants' actions in social life.

\subsection{Academic Life}\label{subsec-academiclife}

As \textcite{mercer2012} points out, learners perceive affordances and act in meaningful ways as a result of their interactions with the environment. In the participants' narratives, the exercise of agency appears to be related to the affordances of mobile technology to access information, as the following excerpts illustrate.

It is necessary to use cell phones at the university because they allow students to access materials that are fundamental to their education. (P9)

Access to PDF texts also makes reading in class much easier, as we avoid excessive use of paper. I use adobe acrobat, adobe scan, siga… (P19)

Because of the lack of face-to-face and virtual teachers, I've tried to delve deeper into the content through video lessons and websites that provide supplemental materials. With the Internet, I was able to study and learn different interesting content, as well as search for different materials on the same content, but by different authors. (P13)

The use of cell phones to study outside of school is essential these days because it allows you to access a variety of websites and have information in the palm of your hand at all times. (P15)

[$\ldots$] Everyone has the autonomy to research what they want. We also have YouTube with lots of content, classes, lectures, etc. If you know what to research and use the right apps, you can learn a lot. (P10)

But it can also be harmful, because it's so easy to find information. I see the need to use more conventional methods as well. (P14)

Access to information, texts, content, and supplementary materials is an action that is frequently mentioned in the participants' narratives. In the data analyzed, the access to "materials that are fundamental to their [students'] education" (p.9) and "PDF texts" in the university context seem to be strongly associated with the affordances of mobile technology. In addition, the use of printed technology in place of digital technology also appears to facilitate reading in the classroom by mitigating the "excessive use of paper" (P19).

The use of mobile devices to find information from various sources also seems to be a common practice among the participants. They acknowledge the use of their mobile phones as a technology tool to "search for different materials on the same content" (P13), thereby enhancing their research consultations and enabling them to check facts or even validate different pieces of information on a given topic. This multiple and powerful search seems to empower learners in that it makes them perceive information as something that can be touched and/or controlled, something that is "always in the palm of [one’s] hand" (P15).

Affordances are “either for good or ill” \cite[p. 127]{gibson1986}. The ability to search through infinite sources is perceived both positively and negatively by different participants. While one person believes that it is possible to "learn a lot" from this type of autonomous research if well-informed and guided (P10), another believes that easy access to diverse information "can also be harmful" and that it would be necessary to "use more conventional methods as well" (P14). These excerpts not only reinforce the complementary relationship between agent and environment \cite{gibson1986}, but also emphasize that agency is intertwined with contextual possibilities, emerging through the self-organized interplay of factors both internal and external to the system \cite{larsen2019}. Given the sociocultural nature of agency, affordances or constraints can render an action (un)likely and (im)possible for a given agent depending on temporal and/or spatial contexts \cite{lantolf2006}.

Another instance of agency identified in the data from this study that seems to emerge in/from the academic context relates to the perceived affordance of mobile devices to study.

I usually use my tablet to study because I absorb knowledge through active reading and I find the tablet to be an excellent tool for this: I can highlight and take notes on theoretical texts. (P12)

I have a more traditional way of studying. I like paper, doodling, making diagrams and summaries. As a last resort, I can open a document in Word or Adobe, but I never use them to study. Or I do a quick Google search on my phone to look up the meaning of a word, check a simple piece of information, or look for a synonym when I write a text. (P6)

In delineating the affordance of studying, we highlight the conceptualization of mobile devices as mediating tools for engaging in study activities, such as facilitating "active reading," enabling the learner-user to "highlight and take notes on theoretical texts" (P12), for instance. It is important to underscore how, once again, mobile technology seems to empower the participant since, from his perspective, it would allow him not only to read, but also to "absorb knowledge" by interacting with texts. Highlighting and annotating are not actions strictly limited to digital environments and devices, as another participant who claims to have "a more traditional way of studying" and likes "paper, doodling" (P6) shows. From a complex perspective, agency is also related to intrapersonal aspects \cite{mercer2012}; in this case, to the learner's belief that one or the other technology (the mobile device or paper) would be more effective for studying.

For \textcite{larsen2019}, the learner's agency spans the learner's life experiences, their present and their future, and its effectiveness is perceived in the present. Interacting with mobile devices in the academic context seems to have an impact on participants' conceptions of time, making them realize that their phones enable them to speed time up.

A complement or extension and enhancement of activities and learning. It's through the cell phone that we do assignments, send files, print, read texts... The cell phone has become a quick tool that is always available to the student. (P3)

So I started using digital media more to take notes and do assignments. I believe that cell phones at university make it easier to search for information and speed up learning, if they are used prudently and correctly of course. (P16)

That mobile phones enable multiple actions seems to give new meaning to learners' notions of time when they implicitly establish a counterpoint between their previous and current academic experiences. The fact that one can "do assignments, send files, print, read texts" (P3) using a single device "that is always available to the student" means that the mobile phone is seen as an instrument that can "speed up" (P16) everyday actions in the life of a university student. In so far as actions are carried out more quickly, this is also perceived as improving and optimizing learning, further supporting the belief that faster equals more effective.

In addition to speed, the just-in-time aspect of mobile technology \cite{pegrum_mobile_2014,godwin-jones_smartphones_2017}, along with its ubiquity and multifunctionality, appears to offer the opportunity to maximize time.

Today, after going through this 100\% online experience, I see that cell phones and technology make it much easier to study, as they avoid spending time commuting to university, offer more comfort to study at home and make the process much easier. (P10)

I also organize my pending university activities on my tablet and cell phone using an agenda app so that I can make better use of time. (P12)

After going through a formal academic experience entirely online and contrasting their experiences, the participants show that they have realized that mobile devices allow them to better organize their routines, to the extent that they can, for example, "avoid spending time commuting to the university" (P10), allowing them to "make better use of time" (P12). In this sense, mobile technology has the potential to give a new meaning to time, allowing the participants, pre-service teachers to organize themselves and exercise their agency in order to study and fulfill their academic obligations satisfactorily.

Another instance of agency mediated by mobile phones and tablets in academic life seems to emerge from the perceived potential of these devices to provide opportunities for distraction.

In the classroom, cell phones are a potential source of distraction. That's the simplest analysis, but over the course of the pandemic I've come to change my mind. I have ADHD, but I no longer see the cell phone as a source of distraction, far from it, it is a possibility for autonomy and independence of research within the classroom. The teacher no longer needs to be the sole, authoritative source of information. I'm not even talking about checking whether what the teacher says is true or not, it's more about being able to expand tangents of knowledge in the middle of the lesson when opportunities arise. I really value students' autonomy in learning. (P5)

I mostly use my cell phone at university to get away from it all. I go on Instagram, Pinterest or OLX in search of visual stimulation. I find it hard to concentrate on one thing for long, so I end up using apps in class out of anxiety. I need these breaks. (P6)

Reflecting on the presence of cell phones in the classroom, one participant recognizes that a simplistic analysis could consider that "cell phones are a potential source of distraction" (P5). However, despite dealing with ADHD, the participant points out that the cell phone is a "possibility for autonomy and independence", allowing him to seize opportunities for research and expand his search for knowledge during class. In this sense, by carrying a device during class, the learner recognizes its capacity to foster engagement and connection with the learning process in an autonomous way, expanding his learning opportunities in the classroom and beyond the school walls.


Distraction is also perceived from another angle, as in the narrative of a participant who states that the use of cell phones at university is "mostly to get away from it" (P6). Again, the compression of time-space is perceived, as the participant emphasizes the potential of the cell phone to be in other places beyond the spatial context in which one is physically present. Because she feels anxious and has difficulty concentrating, the participant admits that she needs visual stimuli to stay in the classroom. In this way, the pre-service teacher seems to imply that using the cell phone to distract herself is important for her academic life.

The narratives of P5 and P6 seem to corroborate that the exercise of agency is influenced by intrapersonal factors, such as ADHD and anxiety, and is subject to adaptations and interactions with other systems. In the case of the two would-be teachers, there is interaction with the teacher, the classroom, and online environments. These issues bring us back to the idea that the exercise of agency is dynamic, open, and susceptible to change. The relational and emergent dimensions of agency are also present in the discourses of the participants in this study, since the exercise of agency emerges from interactions with factors, elements, and systems that make up the situated context of each of these participants. For example, we can mention interpersonal and intrapersonal factors, the relationships established with cultural artifacts (videos, PDF documents, mobile devices etc.) as well as systems such as the university, the classroom, online environments, and social media, to name but a few.
