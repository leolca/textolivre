\section{Ecological approach and complexity theory}\label{sec-Ecological approach and complexity theory}

According to \textcite[p. 11]{vanlier2004}, the ecological approach “looks at the entire situation and asks: What is it in this environment that makes things happen the way they do?” For the author, this approach encompasses the study of the context and all the things going on around it, at school, in the classroom, at the desk, and so on. In addition, the ecological approach is not limited to spaces; it encompasses movement, process, and action. As \textcite{vanlier2010a} explains, ecology studies the relationships and interactions between elements in the environment or ecosystem. For both the ecological perspective and complexity theory, the context and the system are inseparable. The environment influences — and is influenced by — other systems in which it is nested. For the purposes of this investigation, we consider it essential to look at these relationships and their influence on the actions that unfold in a situated context, particularly those regarding agency.


\textcite[p. 43]{mercer2012} reminds us that contexts are not static or monolithic. Rather, “they need to be understood as representing dynamic systems composed of a multitude of components which can combine and interact in complex, unique ways.” Drawing on \textcite{vanlier2004,mercer2012} emphasizes that the concept of affordance is fundamental for understanding agency since affordances “represent the interaction between contextual factors (micro and macro-level structures, artifacts) and learners' perceptions of them and the potential for learning inherent in this interaction.” As \textcite[p. 127]{gibson1986} points out, affordances are: "what it [the environment] 'offers' to the animal, what it 'provides' or 'furnishes', either for good or ill”. For \textcite[p. 43]{mercer2012} , “[a]gency thus emerges from the interaction between resources and contexts and the learners’perceptions and use of them.”


The adoption of the ecological approach to understanding human systems is not novel. It has gained momentum, however, with discussions on the principles of complexity theory, which perceives the world as integrated, based on established relationships. Like the ecological perspective, complexity theory involves agents in an interaction process in which the dynamics of systems, agents, and context are inextricably intertwined.

Complexity theory has come to consolidate its importance in different areas, including Education. There is consensus that although there are different approaches to looking into a phenomenon, all the perspectives seem to share the idea that “all complexity perspectives embrace organic, holistic models composed of complex dynamic systems as opposed to more traditional linear models” \cite[p. 43]{mercer2012}. In essence, a complex system usually has elements or agents that interact with each other, is open, and adapts based on contextual demands, whether internal or external. From these interactions, there is constant action and reaction between agents. The system evolves dynamically, that is, nothing in it is fixed. According to \textcite[as quoted in Mercer, 2012, p. 43]{cilliers2010}
“one of the most defining features of complex systems is that they have emergent properties, for instance, properties which cannot simply be reduced to properties of characteristics of components in the system.” Interactions and adaptations enable the agents of a system to self-organize, leading to the emergence of new patterns and behaviors. A new order spontaneously arises, and it does not seem to be governed by known physical laws.

We consider that discussions that recognize the relationships established in their contexts and surroundings are relevant to this study because they investigate systems that learn, that is, systems that organize and reorganize themselves based on their contextual demands, their network relationships, and dynamics. We also believe that such discussion can shed light on social practices mediated by mobile devices in the context of language teacher education.