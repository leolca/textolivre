\section{Conclusion}\label{sec-conclusion}

The findings of this study suggest that the exercise of agency in initial education contexts is multifaceted. Like the findings of \posscite{mercer2011,mercer2012} empirical work focusing on language learners, the analysis of the narratives in this study indicates that agency is influenced by the intricate interconnectedness of various factors and elements coexisting in the participants' systems. It was possible to identify interpersonal factors, such as the perceived opportunities to interact with agents (human and non-human) in formal and informal online environments, as well as intrapersonal factors, such as the impact on emotions, attitudes, and beliefs about the best ways to learn. The data also show that the exercise of agency is dynamic, subject to change, and open, as it can be influenced by other systems.

Throughout the analysis, the relational nature of agency \cite{larsen2019} proved to be quite salient, as the data unveiled a reciprocal interaction between internal and external factors and the actions emerging from these relationships within contexts and their possibilities.

The results point to the potential of mobile devices in facilitating the exercise of agency among the participating pre-service teachers. These devices allow teachers to access information, speed up time, study, and even provide opportunities for distraction. When it comes to their praxis, these teachers also recognize the possibilities of mobile technology to motivate, bring the classroom to the 21st century, and engage learners. The potential for mobile technologies to impact social life was also evident from the data, as participants at various points in their stories emphasized how pervasive and important technology is to everyday life and citizenship.

In terms of the possible implications of this study for teacher education, one of the possible insights may stem from the recognition that in order to understand the possibilities of teacher agency – pre-service and in-service – it is important to consider the environments in which these agents circulate, the technologies and other agents with which they interact, the nested systems that make up their ecosystems, and the ways in which these dynamics affect and feed back into the interaction between intra- and interpersonal aspects.

We recognize the complexity of agency in the context studied, and although some dynamics and the complex fabric of teachers' agency have been highlighted in the data analyzed, further research can highlight other units of analysis in relation to inter- and intrapersonal aspects, elements, and systems that can further the understanding of the role of these "agents", thus contributing to research on teacher agency.