\section{Mobile learning}\label{sec-mobilelearning}

Current language learning methodologies are strongly influenced by pedagogies that emphasize digital literacies \cite{dudeney2016} and multiliteracies \cite{cazden1996}, integrating a multitude of technologies and societal language practices that coexist within the cosmopolitan milieu. The advent of mobile technology, along with its linguistic facets, aligns seamlessly with this trend.

Understanding the advantages of mobile learning is pivotal for elucidating its conceptual framework. As outlined by \textcite[p. 1]{kukulska2010}, the main advantages of this mode of learning include:

\begin{itemize}
	\item immediate access to information, social networks, and situation-relevant help;
	\item flexible use of time and space for learning;
	\item continuity of learning between different settings;
	\item good alignment with personal needs and preferences;
	\item easy creation and sharing of simple content like photos, videos and audio recordings; e
	\item greater opportunity for sustained language practice while carrying out activities such as walking, waiting, or commuting.
\end{itemize}

Learning with mobility can afford individuals opportunities to use the language as they create, use, and share digital content within situated and contextualized practices.

Regarding language learning, \textcite{braga2017a} emphasizes that the ubiquitous presence of personal mobile devices facilitates the alignment of these objects with the perception of affordances, giving rise to spontaneous and local activities made up of resources available on the Internet. Furthermore, \textcite[p. 49–50]{braga2017b} consider mobile learning as

\begin{quote}
	a movement that goes beyond a mere intervention 'in' pedagogical practice. More than that, this mode of teaching and learning points to a change 'of' the practice itself, in such a way that the multiple contexts in which students and teachers circulate become sources of resources explored via mobile technologies for the teaching and learning of languages.
\end{quote}


Hence, it is important to emphasize that the mere integration of technologies does not ensure the productivity and authenticity of mobile language learning practices. A paradigm shift in teaching and learning is necessary, particularly one that acknowledges the pervasive influence of mobility on our cognitive processes, behaviors, and language use.

Despite possible negative ethical ramifications in mobile learning, such as cyberbullying, disinformation, and hate speech, the affordances for interactivity and multimedia communication facilitated by mobile technology hold significant promise for triggering agency. As emphasized by \textcite[p. 299]{andrews2011}, “mobile learning has the potential to subvert teacher dominance in the classroom and the didactic approaches to education which arose during the industrial era, such as the transmission model.” Instead, it has the potential to introduce a learner-centered paradigm, allowing students to exert more control over their individual learning processes.