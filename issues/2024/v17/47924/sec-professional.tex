\subsection{Professional Life}

Throughout their narratives, the
participants' narratives appear to indicate that the exercise of agency
is related to the use of mobile technologies, geared to their
professional everyday lives as language teachers. The participants
provide several examples showing how they can use mobile devices
\textbf{to teach}, as can be seen in the excerpts below.

As a teacher, I believe that the use of cell phones will always be
present in a classroom, of course, in a controlled way. We can use it
for consultations, which is very useful, in a language class for
example, when we have vocabulary questions, we have apps like Word
Reference, Duolingo, where you can learn a lot using technology.
Recently I've been using Kahoot with my elementary
school students. The app allows you to review content in a more fun way,
as if it were a game, and the students really like it and are very
involved in the lesson. (P18)

Before the pandemic, everything was very manual and with the adaptation
process {[}to the pandemic{]}, the processes have become automated in
some respects, making it easier to streamline lessons. (P13)

Again, the action of consulting reference sources in the classroom is
mentioned as something that the use of mobile devices makes possible.
Projecting himself as a teacher-learner, P18 (above) recognizes an
occasion when he resorted to mobile phones: "when we have vocabulary
questions" (P18), for example, giving insights aimed at the agency of
his (future) students. In addition, the participant highlights another
possibility for mediated pedagogical action: reviewing "content in a
more fun way". The fact that the Kahoot app allows gamified activities
seems to attract students and get them "very involved in the lesson." In
this sense, the participant shows that he believes that the pedagogical
use of mobile technology seems to influence intrapersonal aspects of his
students, affecting their attitudes and emotions in the classroom.

Another instance of agency related to teaching practice refers to the
possibility of catalyzing interaction in the classroom, which is
perceived as an affordance of mobile technology. Contrasting the
teaching processes before and after the Covid-19 pandemic, P13
recognizes that the adaptation of teaching generated by social isolation
has automated teaching and learning actions, "making it easier to
streamline lessons" (P13).

Changing classroom dynamics seems to be an ability perceived by
participants in digital technologies, who recognize that by using mobile
devices with their students, they are able \textbf{to bring the
classroom to the 21st century}.


{[}It's{]} a way to contextualize the learning
environment with the reality of the students, because we live in a
completely technological society. (P14)

It's a way to recover a past that could not be learned
with the technologies available at the time, and to learn how best to
guide a digital native in his or her process of evolution and
development. As a teacher who is not a digital native, immersing myself
in this world brings me closer to my students and allows me to
understand them better. Acquiring knowledge in a way that is current and
coherent with the digital transformation makes me feel that I belong and
want to be part of the experience. (P4)

The incorporation of mobile technology in language teaching is seen as a
way to bring about temporal changes in the classroom, as stated by the
participant who recognizes the use of mobile devices as "a way to
contextualize the learning environment with the reality of the
students." In this understanding, the appropriation of mobile technology
translates into an opportunity to offer students school experiences that
are true to reality and aligned with their social experiences.

The data generated in this study shows that the reconceptualization of
time generated by the exercise of agency also influences the notion of
space. The use of mobile devices in the classroom seems to bend
time/space, altering the objective qualities of these two systems,
causing the teacher to "recover a past," "immerse {[}himself{]} in this
world {[}of digital natives{]}\footnote{ Although the concept of
  ``digital natives'' \cite{prensky2001a} was briefly mentioned by a
  participant, it will not be further explored in this paper due to the
  specific focus of the research. However, it is noteworthy that over
  the years, researchers in the educational field have contested the
  term, asserting that it oversimplifies the competencies and skills of
  younger generations, perpetuating a false and prejudiced dichotomy
  between digital natives and digital immigrants. In response to all the
  criticism, \textcite{prensky2011} introduced the term ``digital wisdom'' to
  advocate for a more comprehensive approach, going beyond technical
  competence to encompass critical thinking, ethical behavior, and
  responsible attitudes in digital environments. As highlighted by
  \textcite[p.~8]{prensky2011} ``the question we should ponder for that future
  is no longer whether to use the technologies of our time but rather
  how to use them to become better, wiser people.''}," "get closer to
{[}his{]} students" in order to "be part of the experience" (P4). Thus,
by integrating mobile technology into his teaching practice, the teacher
seems to be projecting himself into a time warp that allows him to use
contemporary practices even though he wasn't trained to
do so.

Acting on the affordances perceived in mobile technologies seems to
provide opportunities \textbf{to motivate learners}.


Besides, the use of cell phones to study is a way to motivate students
because there are countless ways to learn. (P14)

Confirming \posscite{mercer2012} reflections, agency does
indeed seem to be intertwined with intrapersonal factors, as its
exercise, as highlighted above, seems to influence
learners' beliefs, attitudes, and emotions. The use of
mobile devices in teaching practice proves to be a way to "motivate
students because there are countless ways to learn" (P14). From this
perspective, the use of mobile devices in the classroom and their
acknowledgement as legitimate learning technologies appear to be
perceived as motivating for students.
