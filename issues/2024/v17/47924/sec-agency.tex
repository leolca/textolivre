\section{Agency}\label{sec-agency}

The concept of agency has been gaining prominence in Education, especially in Applied Linguistics, as it deals with issues aimed at possible actions by the learner, the teacher, the educator, or other participants in the educational community. It is present in discussions related to autonomy, identity, motivation among other constructs in this area. In his lecture at the 19th World AILA Congress in 2021, Benson, in a brief retrospective of discussions on agency, conflates autonomy and agency when considering “[a]utonomy: a more comprehensive capacity to purposefully control or direct agency.” We endorse the understanding that autonomy refers to the capacity for self-governance and self-direction and it may enable and trigger agency: the capacity of individuals to act intentionally to achieve specific goals.


Agency has a variety of definitions. One of the most-widely circulated definitions in Applied Linguistics is that of \textcite[p. 112]{ahearn2001} for whom agency is "the socioculturally mediated capacity to act".  \posscite{ahearn2001} discussions were considered a milestone in underscoring the sociocultural nature of the agency. Aligned with \posscite{ahearn2001} position, \textcite{lantolf2006} consider that agency is socioculturally mediated and dialectically exercised. This implies that within a specific timeframe and spatial context, certain limitations and affordances make some actions probable, possible, and even impossible. \textcite{stetsenko2020} advances by proposing the exercise of agency through a process in which agents are co-authors of the world and contribute to social transformation in a movement that involves individual and collective agency. \textcite{picard2010} emphasizes the need for new definitions of agency since modern definitions of this concept tend not to contemplate the interconnectivity between humans, non-humans, and environments.


According to \textcite[p. 1]{sang2020}, the notion of teacher agency is related to the ability to make active choices in educational practices. In his words:

\begin{quote}
    The notion of teacher agency is being used to describe the agentic capacity to make active choices in educational practices. Teacher agency refers to a teacher's competence to plan and enact educational change, direct, and regulate their actions in educational contexts. Becoming a crucial variable in studying teaching behavior and teacher education, researchers have investigated agency of student teachers, novice teachers, and experienced teachers to examine how agency affects learning to teach and acting to reform. In educational contexts, policy implementation and reform innovation rely on individual and/or collective agency of teachers.
\end{quote}

\textcite{sang2020} emphasizes that studies on teacher agency need to take account of the social structure, considering that teachers' actions are influenced by this structure, and may even be institutionally restrained. We agree with \textcite[p. 2]{sang2020} that “[i]n the ecological view, agency positions it within the contingencies of contexts in which agents act upon their beliefs, values, and attributes they mobilize in relation to a particular situation”.


Agency has also been discussed under the lens of complexity theory. For \textcite{mercer2012}, due to the complex nature of agency, conclusive and widely accepted definitions are difficult to find. \textcite{mercer2012} emphasizes the need to recognize and define agency from the data generated in an investigation. In one noteworthy case, \textcite[p. 42]{mercer2011} observed agency from two overlapping dimensions: the first points to the sense of the learner's agency, that is, “how agentic an individual feels both generally and in respect to particular contexts”, and the second concerns “learner's agentic behavior in which an individual chooses to exercise their agency through participation and action, or indeed through deliberate nonparticipation or non-action”. In the words of \textcite[p. 43]{mercer2012}, “[a]gency is therefore not only concerned with what is observable but it also involves non-visible behaviors, beliefs, thoughts and feelings; all of which must be understood in relation to the various contexts and affordances from which they cannot be abstracted”.


\textcite{larsen2019} also recognizes agency as a complex system and, supported by \textcite{mercer2011, mercer2012} and other researchers, asserts that agency is relational, emergent, spatially and temporally situated. It can be achieved, and it changes through iteration and co-adaptation; furthermore, it is multidimensional and heterarchical. For the purpose of this study, we explore possible dimensions of teacher agency considering the following characteristics:


\begin{itemize}
	\item Agency is relational
\end{itemize}


Agency is not inherent to the individual, that is, there is not something internally innate responsible for the observed behavior. Rather, “agency is interpellated from the self-organized dynamic interplay of factors internal and external to the system, persisting only through their constant interaction with each other” \cite[p. 65]{larsen2019}. In this sense, agency is relational and “is always related to possibilities in context and therefore inseparable from them, and possibilities, in turn, are ecological and not merely physical characteristics of the world, defined in terms of the relationship of 'systems' between the organism and its environment”.

\begin{itemize}
	\item Agency is emergent
\end{itemize}


\textcite{larsen2019} describes agency as a spontaneous activity connected to the world, forming coordinative structures that are units of selection in evolution and intentional change. Complex systems organize themselves harmoniously in functional synergies, sensitively adapting to the context and providing selective advantages. Agency emerges early in human life when children realize they can bring about change and understand their ability to transform the world. Drawing on \cite{kelso_self-organizing_2016}, \textcite[p. 65]{larsen2019} tells of a child who, at an early age, “becomes aware that it can make things happen; for example, by kicking its legs, it can make a mobile move or shaking its fist, a rattle sound”.

\begin{itemize}
	\item Agency is situated in space and time
\end{itemize}

Agency is influenced by the past, present, and future, responding to changing historical situations. Language learner agency is holistic, spanning a learner’s life story, past experiences, present and future goals. Its effectiveness occurs in the present, as observed by \textcite{biesta2007,mercer2012}.

\begin{itemize}
	\item Agency is Multidimensional
\end{itemize}

Agency goes beyond simple actions. As discussed by \textcite{lantolf2006,deters2015}, it involves the ability to assign relevance and meaning to various elements. Joana, a research participant in \posscite{mercer2012} study, exemplifies the multifaceted nature of agency, which is intertwined with intrapersonal aspects such as emotions, beliefs about language acquisition, self-image, personality, and motivation. Undoubtedly, these factors can collectively influence a person's agency.


In addition to these features,\textcite[p. 8]{mairitsch2023} also point out that agency can be both socially constructed and socially distributed. The social construct feature is related to “the dynamic interplay between a learner, their psychologies (in particular, how learners learn and interact with their environments), and their social contexts.” As for the socially distributed feature, \textcite[p. 5]{mairitsch2023} emphasize that learners’agency is not only socially determined, but also socially distributed. The authors \cite[p. 7]{mairitsch2023} claim that the distributed dimension of agency “is not just a characteristic of an individual. Rather, it can be viewed as a collective attribute promoted when groups work well together and feel empowered in a supportive and psychological safe climate”.


Although the features of agency in \textcite{mercer2012}, \textcite{larsen2019,mairitsch2023} certainly contribute to the discussions in this study, we share \posscite{mercer2012} thoughts that it is important to identify features of agency when examining data. With that in mind, we hope to contribute to the discussions on agency as an ecological and complex system by identifying some of the features discussed by both researchers, as well as other features yet to be identified when it comes time to analyze the data.
