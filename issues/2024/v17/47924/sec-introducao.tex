\section{Introduction}\label{sec-intro}
The educational context is seen as a social ecosystem, exhibiting characteristics of a complex adaptive system, composed of nested systems, such as schools, classrooms, families, and so on. Agents within these systems, such as educators, students, and teachers contribute to the emerging dynamics of their relationships within these systems. An investigation of ecological systems requires lenses that contemplate the possible relationships and dynamics among their agents. The ecological approach is considered a way of thinking and studying organisms in their relationship with the environment. \textcite{vanlier2004} reminds us that this approach is based on systems theory, complexity theory, chaos theory, and cybernetics, recognizing the complexity and interrelationship of the processes that lead to the creation of an environment.


Underpinned by the construct of ecological perspective and coupled with the concept of affordance, this study attempts to better understand how pre-service teachers exercise their agency using mobile digital technologies. The concept of affordance can serve to identify actions arising from these teachers’perceptions. Moreover, we resort to discussions on agency as a complex system in Applied Linguistics \cite{mercer2012, larsen2019} because they can advance our understanding of patterns that emerge from the perception-action relations as these pre-service teachers exercise agency.


The concept of agency is present in discussions of several areas of knowledge and has been on the agenda of Applied Linguistics. Discussions by \textcite{vanlier2004,vanlier2010a,mercer2011,mercer2012,mercer2018,larsen2019}, however, stand out from those of other scholars because they rely on the ecological perspective and complexity theory.


Although the field of Applied Linguistics is not devoid of discussions on these perspectives, the dimensions of pre-service teachers’ exercise of agency mediated by mobile digital technologies have been underexplored, which justifies the development of this study. With that in mind, we seek to identify: 

\begin{enumerate*}[label=\roman*)]
	\item Instances of the agency exercised by pre-service teachers;
	\item Actions mediated by the use of cell phones by these teachers;
	\item Intrapersonal issues (emotions, beliefs, motivation, etc.) that can influence their agency.
\end{enumerate*}


We briefly present the theories and concepts undergirding this study, namely: the ecological approach, complexity theory, mobile learning, and agency.

