\section{Conclusão}\label{sec-conclusão}

As tecnologias 3D são primordiais para a visualização 3D dos sólidos
geométricos, promovem a VE no estudo de vários temas de Desenho Técnico,
oferecendo simulações de experiências virtuais em tempo curto, assim
sendo, permitindo repetições das experiências e com baixo custo. Essas
experiências devem-se ao rápido desenvolvimento das tecnologias nos
últimos anos, em que se assistiu a muitas invenções de dispositivos
tecnológicos e de informação e criação de programas espaciais de
modelação 3D.

O AT Q3DM é uma ferramenta com potencial para o estudo das POs, porque é
adequado aos conceitos de representação das VOs resultantes da sua
projeção. Dessa forma, estimula a aprendizagem dos alunos, por meio da
sua apresentação interativa com alta resolução de nitidez dos elementos
geométricos.

O AT de geometria dinâmica GeoGebra é um meio didático que gerou grande
interesse e motivação nos alunos nas atividades de resolução de SCs. Por
conseguinte, promoveu a VE da seção produzida em qualquer posição do
plano secante no cilindro.

Respondendo à questão de investigação, os alunos melhoraram a VE no
estudo das POs e Scs, por meio de simulações computacionais de
exercícios práticos em 3D em ATs. Assim sendo, os alunos simularam os
exercícios de forma interativa e dinâmica, o que facilitou o transporte
para a folha de desenho. Portanto, foi observada uma redução de
dificuldades e um desenvolvimento de habilidades de VE. Os pressupostos
da pesquisa interpretativa possibilitaram adentrar no mundo social dos
alunos, para compreender, em diversas situações, qual o significado de
suas ações. A tecnologia na sala de aula pode garantir bons resultados,
porque permite uma representação de todos os elementos de DT e GD, mas
não substitui o professor, porque o professor é o agente que desenvolve
situações no ensino e integra a tecnologia relacionada com o
conhecimento pretendido na aprendizagem.


