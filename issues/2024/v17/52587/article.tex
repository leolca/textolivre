\documentclass[portuguese]{textolivre}

% metadata
\journalname{Texto Livre}
\thevolume{17}
%\thenumber{1} % old template
\theyear{2024}
\receiveddate{\DTMdisplaydate{2024}{5}{11}{-1}}
\accepteddate{\DTMdisplaydate{2024}{5}{11}{-1}}
\publisheddate{\DTMdisplaydate{2024}{5}{12}{-1}}
\corrauthor{Kleber Aparecido da Silva}
\articledoi{10.1590/1983-3652.2024.52587}
%\articleid{NNNN} % if the article ID is not the last 5 numbers of its DOI, provide it using \articleid{} commmand 
% list of available sesscions in the journal: articles, dossier, reports, essays, reviews, interviews, editorial
\articlesessionname{editorial}
\runningauthor{Silva; Silva; Salomão}
%\editorname{Leonardo Araújo} % old template
\sectioneditorname{Daniervelin Pereira}
\layouteditorname{João Mesquita}

\title{Dossiê 2024: Educação linguística e cultural mediada por tecnologias digitais}
\othertitle{Dossier 2024: Linguistic and cultural education mediated by digital technologies}

\author[1]{Eduardo Viana da Silva~\orcid{0000-0002-3651-3524}\thanks{Email: \href{mailto:evsilva@uw.edu}{evsilva@uw.edu}}}
\author[2]{Kleber Aparecido da Silva~\orcid{0000-0002-7815-7767}\thanks{Email: \href{mailto:kleberunicamp@yahoo.com.br}{kleberunicamp@yahoo.com.br}}}
\author[3]{Ana Cristina Biondo Salomão~\orcid{0000-0002-1531-8551}\thanks{Email: \href{mailto:ana.salomao@unesp.br}{ana.salomao@unesp.br}}}
\affil[1]{University of Washington, Seattle, EUA.}
\affil[2]{Universidade de Brasília, Brasília, DF, Brasil.}
\affil[3]{Universidade Estadual Paulista ``Júlio de Mesquita Filho'', FCL, Araraquara, SP, Brasil.}

\addbibresource{article.bib}

\begin{document}
\maketitle

\section{Editorial}
As diversas propostas trazidas nos últimos anos pelo uso de tecnologias no ensino de línguas e nas práticas de multiletramentos ensejaram as primeiras reflexões sobre a necessidade de um dossiê que abrisse espaço para discussões sobre o presente e o futuro da educação linguística com a utilização de tradutores digitais e plataformas de inteligência artificial. Em meio à epidemia do COVID, vimos uma crescente interação com tecnologias digitais em cursos de línguas, tal qual a aprendizagem e ensino de línguas por meio de cursos virtuais síncronos e/ou assincrônicos. Também percebemos um aumento em intercâmbios virtuais na aprendizagem de línguas adicionais, nomeadamente: Teletandem e \textit{Collaborative Online International Learning} (COIL), além de outros tipos de telecolaborações \cite{salomao2020,salomao2023}. Estas mudanças nas dinâmicas de ensino geram pedagogias, ideologias e comportamentos culturais a serem estudados. Diante deste cenário de políticas linguísticas com a expansão e inclusão de novas tecnologias no ensino de línguas, propomos este dossiê, intitulado: “Educação linguística e cultural mediada por tecnologias digitais”\footnote{Os artigos deste dossiê representam várias tendências atuais no ensino de línguas mediado por tecnologias digitais e em intercâmbios virtuais. Recebemos mais de 100 propostas de artigos, das quais dez foram selecionadas para publicação. A demanda de artigos indica também a relevância do tema. Agradecemos imensamente a todas as pessoas que contribuíram com este dossiê, de autoras/autores aos revisores que leram atentamente os artigos e propuseram sugestões e alterações. Finalmente, somos gratos aos editores da revista \textit{Texto Livre} da Universidade Federal de Minas Gerais por terem aceitado essa proposta de dossiê para o ano de 2024 e por terem trabalhado assiduamente conosco em um processo altamente seletivo de publicação.}.

As tecnologias digitais têm mediado o ensino de línguas mais intensamente nas últimas três décadas por meio de intercâmbios virtuais e telecolaboração (ex. MIT Cultura e Teletandem Brasil), uso de aplicativos (ex. Duolingo, Babel, HelloTalk), dicionários digitais (ex. Wordreference e Priberam), sistemas de tradução on-line (ex. Google translator), cursos on-line (ex. Coursera), entre outras ferramentas e recursos digitais. Nos Estados Unidos, um dos primeiros projetos de intercâmbio linguístico e cultural mediados por computadores começou em 1997 através do \textit{Cultura Project} do Massachusetts Institute of Technology (MIT), incluindo programas síncronos e assíncronos, principalmente no formato de fóruns, em inglês, francês, russo, espanhol e alemão, entre outras línguas. \textcite{english_going_2023} descreve a experiência de duas universidades francesas que participaram do projeto entre 1998 e 2007, com ênfase nos tipos de atividades propostas, que eram estruturadas seguindo uma progressão em sua complexidade. Iniciava-se com um questionário de associação de palavras, como por exemplo, \textit{family/famille, individualism/individualisme, suburbs/banlieue}, cujos significados apresentariam diferenças veladas que talvez não fossem consideradas pelos participantes à primeira vista. A segunda tarefa consistia em um questionário aberto para completar sentenças que envolveriam valores sobre determinados termos (como um bom amigo, um bom vizinho). O questionário final envolvia a tomada de ações imaginárias em situações hipotéticas. Os participantes então discutiam os resultados das tarefas nos fóruns. Segundo \textcite[p. 31]{english_going_2023}, os fóruns eram o ponto central do projeto uma vez que eram o lugar no qual os estudantes analisavam os dados que tinham produzido, como ‘médicos discutindo suas próprias imagens de raio-x’. 

Em 2006, o Teletandem Brasil - \url{http://www.teletandembrasil.org/} - desenvolvido pela Universidade Estadual Paulista (Unesp), teve seu início de atividades com interações virtuais que na época eram conduzidas por MSN e Skype \cite{telles_teletandem_2009}. \textcite{telles_foreign_2006} contam a história do desenvolvimento do Teletandem apontando para a expansão da Internet e do uso de \textit{e-mails} como essenciais para o aumento da popularidade da aprendizagem em tandem no ensino de línguas mediado por computador no país. Segundo os autores, os avanços tecnológicos da comunicação síncrona abriram novas possibilidades de interação para sessões de tandem a distância, levando à criação do projeto \textit{Teletandem Brasil: línguas estrangeiras para todos}, que visou oferecer um contexto para alunos e professores construírem a experiência telecolaborativa de ensino e aprendizagem, caracterizada pela autonomia e pela reciprocidade, e de refletirem juntos acerca desse processo, assim como acerca do uso das novas ferramentas de videoconferência na aprendizagem a distância. 

Por conta da pandemia do COVID, o processo de tecnologias digitais aplicadas ao ensino de línguas se acelerou significativamente, acentuando o uso de ferramentas de videoconferência, como Google Meets e Zoom para aulas síncronas, assim como o uso de aplicativos de mensagens instantâneas para comunicação e realização de tarefas à distância. A fim de discutir tais modos de interação no ensino formal, o artigo “Conferência online entre pares: uma janela de oportunidade a favor do \textit{feedback} para a escrita”, de \textcite{xhafaj_online_2024}, da Universidade Federal de Santa Catarina (UFSC), analisa três pares de estudantes universitários de inglês em suas interações virtuais através de Google Meet e WhatsApp, na qual os estudantes provêm \textit{feedback} sobre textos que escreveram em inglês para seus cursos. Os autores concluem que o processo de \textit{peer review} dos participantes deste estudo envolve uma nuance de ferramentas digitais, de WhatsApp a corretores e tradutores de texto, replicando as práticas digitais dos participantes em suas vidas diárias. O estudo também demonstra uma falta de consciência em uma fase inicial sobre a importância da escrita do inglês por parte dos participantes. 

Contribuindo para a mesma discussão, o artigo “‘Na Palma da mão’: a natureza ecológica e complexa da agência do professor” de \textcite{braga_palm_2024}, da Universidade Federal de Minas Gerais (UFMG), analisa a relação entre a agência de professores e o uso de tecnologia de celulares/móveis em um estudo com 20 participantes em um curso universitário de treinamento de professores e de integração de recursos tecnológicos da UFMG. O estudo demonstra o relacionamento dos professores-estudantes com aparelhos digitais, especialmente celulares e \textit{tablets}, concluindo que o uso dos mesmos facilita o exercício da agência de professores. Ainda que o uso destas tecnologias móveis possam gerar distração, os participantes do estudo, em sua maioria, veem muitos benefícios, de recursos de apoio online a leituras em PDF do material de ensino e às aulas virtuais, entre outros fatores.

Envolto em uma reflexão sobre a escrita em aplicativos de comunicação instantânea, o estudo “Norma digital e competência ortográfica de adolescentes em contextos vulneráveis: um estudo de caso”, de \textcite{fernandez-julia_norma_2024}, da Universidade de Sevilla, Espanha, compara a competência ortográfica e competência digital em WhatsApp de estudantes de escolas do ensino médio em áreas carentes com estudantes em escolas em áreas mais privilegiadas. O estudo realizado com 126 estudantes entre as idades de 14 e 16 anos indica que em ambos grupos de estudantes a escrita digital não prejudica a ortografia acadêmica, apesar de diferenças entre os dois grupos ao que tange \textit{textismos} léxicos, semânticos e multimodais, e erros ortográficos.

É importante refletir que apesar de algumas das tecnologias digitais como o WhatsApp serem amplamente utilizadas nos contextos educacionais do Brasil e da Espanha, como demonstrado nos artigos anteriores, o mesmo não se aplica ao contexto de outros países como no caso dos Estados Unidos e do Canadá, por exemplo. As práticas culturais e educacionais de países europeus e de países na América Latina, por vezes, se diferem muito das práticas culturais de países anglófonos ou de outras culturas. No caso dos Estados Unidos e do Canadá, especificamente, o uso do WhatsApp é inviável nos contextos educacionais destes países, onde estudantes e professores não costumam compartilhar  números de telefone por uma questão de privacidade e de segurança pessoal e profissional. Em outras palavras, a facilidade, mobilidade e agilidade de um recurso como o WhatsApp é praticamente impensável no contexto estadunidense e canadense. Estes tipos de variantes culturais não foram consideradas neste dossiê \emph{per se}, mas são um caminho para investigações futuras que comparem o uso de recursos digitais e suas contextualizações culturais, históricas e políticas.

No que tange aos materiais didáticos digitais, alguns artigos discutiram questões relacionadas ao seu desenvolvimento, descrevendo o processo de criação, e a sua curadoria, refletindo sobre a necessidade de se analisar criticamente as bases teóricas que compõem as escolhas dos recursos. Assim, em “Laboratório virtual de pesquisa escolar com gramática: educação científica em aulas de língua materna”, \textcite{silva_densidade_2024}, da Universidade do Norte de Tocantins, Araguaína, descrevem o desenvolvimento de materiais didáticos digitais para cursos de português como língua materna em uma escola pública do ensino básico no Brasil. Este estudo faz parte do projeto Conscientização Gramatical pela Educação Científica (ConGraEduC) apoiado pelo Programa Ciência na Escola (PCE) do Governo Federal brasileiro. Em um processo colaborativo entre professores e 30 estudantes do nono ano do ensino básico, os autores desenvolveram um jogo digital de fábulas, objetivando a alfabetização e o letramento científico de estudantes. O artigo descreve em detalhes o processo de criação do jogo Gramática das Fábulas e a experiência dos estudantes. 

Em “Curadoria digital e decolonial de vídeos e podcasts na educação linguística em francês”, \textcite{abreu_curadoria_2024}, da Universidade Federal do Piauí (UFPI), analisam o processo de curadoria de materiais digitais para curso de francês como língua adicional. O estudo aponta para o provável apagamento da criticidade em planejamentos de sequências didáticas que envolvem a busca, seleção, análise, adaptação, organização e compartilhamento de materiais digitais. As autoras propõem a criação de referenciais teórico-metodológicos que possam decolonizar a curadoria-autoria de materiais didáticos digitais, promovendo o pensamento crítico. Através de uma análise minuciosa de um episódio de um \textit{podcast} e de uma plataforma com vídeos destinados ao ensino de francês, este estudo demonstra as visões eurocêntricas e por vezes racistas dos recursos de ensino apresentados.

Ainda sobre o tema de \textit{podcasts}, em “Adaptação e evidências de validade do \textit{Questionnaire for Assessing Educational Podcasts} (QAEP) para o português brasileiro: um estudo indisciplinar em letramento em saúde”, \textcite{sampaio_adaptacao_2024}, da Universidade Estadual do Ceará, as autoras apresentam um estudo metodológico, no qual um instrumento de avaliação de \textit{podcasts}, QAEP, é traduzido ao português, adaptado às realidades brasileiras e validado com a participação de um comitê avaliador de sete participantes. O QAEP destina-se a \textit{podcasts} com fins educativos na área da saúde.

Os estudos previamente descritos sobre o uso de \textit{podcasts} e curadoria de recursos digitais nos levam a uma reflexão sobre o pensamento crítico e decolonial nas práticas de ensino. A ênfase de uma prática de ensino decolonial e reflexiva invoca a pedagogia crítica freireana e a construção do conhecimento com a participação ativa de estudantes, professores e demais pessoas envolvidas. Ou seja, o conhecimento deve ser construído e reconstruído, criticado e analisado em uma comunidade de aprendizes, da qual professores e estudantes colaboram mutuamente. O conceito de \emph{conscientização} de Paulo Freire é inerente a esta discussão. Nas palavras de \textcite[p. 35, tradução nossa]{macedo_introduction_2005}, \textit{conscientização} como um processo de aprendizado propõe “uma percepção social, política e de contradições econômicas para se tomar uma ação contra os elementos opressores da realidade”. Isto implica também uma análise da seleção e produção de recursos digitais desde uma perspectiva de raça, classe e identificação de gênero, por exemplo, citando-se apenas alguns dos elementos que são constantemente e sistematicamente utilizados como opressores sociais. Mais estudos nesta área, a exemplo dos artigos deste dossiê, são de grande importância para um melhor entendimento dos fatores socioculturais envolvidos na produção e curadoria de recursos digitais, de \textit{podcasts} e na construção de identidades culturais. 

Ainda considerando a importância da formação identitária e dos elementos de diversidade,  o artigo “\textit{Photovoice}, pedagogia psicodramática e multiletramentos para a formação crítica de adolescentes no contexto pandêmico”, de \textcite{bomfim_photovoice_2024}, da Universidade Federal do Mato Grosso do Sul, apresenta registros fotográficos de estudantes do ensino médio técnico por meio do método de \textit{Photovoice} utilizando recursos digitais. O artigo analisa as representações do “eu” em relação ao “outro” e ao meio-ambiente em atividades colaborativas com 70 estudantes entre as idades de 14 e 17 anos. A pedagogia psicodramática aplicada a este estudo objetiva contribuir para a formação de estudantes participativos, criativos e com pensamento crítico por meio de atividades interdisciplinares e colaborativas no contexto digital e da pandemia do COVID. Neste estudo, os participantes possuem a liberdade de criarem suas identidades digitais e representações que são autênticas e orgânicas. Este tipo de pesquisa nos lembra também o papel de avatares no ensino de línguas e em interações virtuais, como por exemplo no uso de equipamentos de realidade virtual. Esperamos que mais estudos sobre esta linha de formação e representação identitária sejam explorados em futuros artigos na área da linguística aplicada e em estudos culturais.

Um outro aspecto da mediação de tecnologias digitais na educação linguística e cultural são os artigos sobre o \textit{corpus} do português presentes neste dossiê. Os mesmos nos levam a questionar práticas culturais e pedagogias de ensino, oferecendo uma gama de informações sobre padrões e interferências sociolinguísticas nas produções escritas de estudantes de português como língua adicional. No artigo “Corpus de aprendizes de português da Universidade de Macau e ensino de português L2”, de \textcite{zhang_corpus_2024}, da Universidade de Macau, na China, apresenta-se um \textit{corpus} anotado destacando padrões linguísticos de aprendizes chineses de português com base em 933 redações escritas por 122 estudantes chineses. Por meio da análise qualitativa e quantitativa deste \textit{corpus}, os autores propõem aspectos linguísticos a serem futuramente trabalhados com os estudantes, baseando-se em uma análise contrastiva com a produção de falantes nativos de português. Este \textit{corpus} anotado é, nomeadamente, o primeiro \textit{corpus} de português que se centra em falantes de chinês. Percebe-se neste estudo que muitas vezes a pesquisa quantitativa por si só não é capaz de responder a todas perguntas de pesquisa, o que leva os autores a recorrerem também à pesquisa qualitativa. Observa-se também que o padrão do falante-nativo como referência linguística pode, ou deve, ser questionado dependendo do contexto sociocultural em que este/esta está inserido/a. Cabe aqui uma reflexão sobre este tema e a necessidade de estudos adicionais sobre o assunto.

Em “Densidade lexical em textos gerados pelo ChatGPT: implicações da inteligência artificial para a escrita em línguas adicionais”, de \textcite{silva_densidade_2024}, da Universidade de Essex, na Inglaterra e da Universidade Federal do Rio Grande do Sul (UFRS), respectivamente, analisa-se um corpus com tarefas de redação em alemão, espanhol, francês, italiano e português, em um total de 2991 textos. A pesquisa é realizada com base na linguística sistêmico-funcional e no conceito de complexidade lexical, apontando que o ChatGPT não é capaz de produzir textos que demonstram diferenças significativas de densidade lexical entre as línguas adicionais e os níveis de proficiência. Ou seja, os textos produzidos em ChatGPT não correspondem ao nível de complexidade e diversidade léxica que se espera em relação aos vários níveis de proficiência. Assim como o estudo anterior de Jing Zhang e Mu You, o artigo de Antonio Marcio da Silva e de Lucia Rottava apresenta um estudo inédito e o primeiro corpus sobre o tema em questão.   

O ChatGPT tem sido um das temáticas mais em voga sobre o ensino e aprendizagem de línguas com o auxílio de ferramentas digitais desde seu surgimento em novembro de 2022. O estudo anterior demonstra, porém, que tanto a densidade e complexidade lexical de textos produzidos atualmente em ChatGPT estão aquém do ideal. Pode-se, no entanto, questionar como estes aspectos serão aprimorados em um futuro próximo e até que ponto um recurso como o ChatGPT poderá contribuir ou até mesmo substituir algumas das funções de professores de línguas. Existe também um risco de empobrecimento do conhecimento e de uma repetição de mesmices com palavras distintas ou com uma densidade lexical maior. Se por um lado, os riscos envolvidos com o ChatGPT podem ser vários, por outro, os benefícios são também inquestionáveis. Sistemas operacionais de algumas empresas, por exemplo, têm utilizado o ChatGPT e a inteligência artificial como um recurso para trabalhos mecânicos e repetitivos, como o fornecimento de informações pelo telefone sobre um determinado serviço ou produto. Em ambientes educacionais, o ChatGPT tem auxiliado professores na criação de planos de aula por meio da geração de ideias, por exemplo, e tem também contribuído com a organização curricular de programas de línguas, além de outros fatores. Quando utilizado de forma crítica e criteriosa, o uso do ChatGPT parece auxiliar professores e estudantes de línguas. Os exemplos são vários, de correções de textos, produção de diálogos interativos com estudantes, e a geração de ideias e sugestões para a produção escrita, entre outros. No entanto, ficam ainda muitos questionamentos a serem estudados e analisados à medida que as tecnologias de Inteligência Artificial (IA) se desenvolvem. Qual deveria ser a posição ética de professores e estudantes diante da utilização de IA e de recursos como o ChatGPT? Até que ponto a utilização de IA incentiva a repetição de ideias, o plagiarismo e a geração de \textit{fake news} ou a reprodução de desinformação no contexto acadêmico? A comunicação em si mediada por máquinas produz também um efeito retroativo em nossa capacidade cognitiva? Enfim, por mais significativo que venha a ser o impacto de IA em nossas vidas, as questões seguem sobre a interação humana com IA e sobre seus riscos e benefícios. 

Finalmente, um dos aspectos propulsores deste dossiê foi o papel de intercâmbios virtuais no contexto acadêmico. Em “Virtual exchange in teacher education programs from Brazil and USA: outcomes and challenges”, de \textcite{calvo_virtual_2024}, da Universidade Estadual de Maringá (UEM) e da The Pennsylvania State University, respectivamente, as autoras compartilham a experiência de um projeto de oito semanas de intercâmbio virtual, realizado em 2022 e 2023, com professores em formação inicial matriculados em disciplinas relacionadas à aquisição de segunda língua no Brasil e nos EUA. Para isto, foram utilizados questionários e narrativas, assim como interações e tarefas em diferentes plataformas digitais. Esta pesquisa identifica os desafios e benefícios de interações virtuais, incluindo o processo colaborativo, a comunicação em um língua estrangeira e as dificuldades de agendamento das interações. Este estudo demonstra que estudantes recorrem a vários recursos digitais em interações virtuais. Graças a estes recursos, as interações têm sido facilitadas, ainda que haja uma barreira linguística e cultural. 

A exemplo do artigo anterior, a interculturalidade e a formação de cidadãos globais ocupam um espaço central em interações virtuais e, portanto, não devem ser ignoradas. Além disso, acreditamos que a inteligência emocional dos estudantes é um aspecto que deve também ser considerado em interações mediadas por tecnologia. Estudos sobre a inteligência emocional demonstram a importância do reconhecimento e da regulação das emoções pelos sujeitos envolvidos em uma determinada interação ou situação \cite{salomao2023}. Desta maneira, estudantes e professores exercem a capacidade de reconhecimento de suas emoções e de suas competências pessoais e sociais, facilitando a comunicação, o gerenciamento de conflitos e o trabalho colaborativo \cite{goleman_ei-based_2001}. Cabe aqui ressaltar a importância destes aspectos no ambiente educacional, dado as tendências de isolamento social e as dificuldades de socialização geradas também pelo uso excessivo de tecnologias digitais e da mídia social. Tanto em ambiente virtuais como em nosso dia-a-dia, as tecnologias têm nos oferecido atalhos, vantagens e acesso imediato a informações. No entanto, estes mesmos recursos tecnológicos têm gerado também percalços, desumanização, desinformação e por vezes isolamento. Terminamos esta introdução sinalizando a importância de se cultivar a inteligência emocional de estudantes e professores em ambientes virtuais, a fim de gerar um equilíbrio emocional maior para participantes do  ensino de línguas e culturas mediadas pelas tecnologias digitais.



\printbibliography\label{sec-bib}

\end{document}
