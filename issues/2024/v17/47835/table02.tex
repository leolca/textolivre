\begin{table}[!htpb]
\centering
\small
	\begin{threeparttable}
		\caption*{\textbf{Exemplo 1.}}
		\label{tab-02}
		\begin{tabular}[]{@{} 
				>{\raggedright\arraybackslash}p{(\columnwidth - 0\tabcolsep) * \real{1.0000}}@{}}
			\toprule\noalign{}
			Oficina de jogo: legendas com cores 
			\\
			\midrule\noalign{}
			
			\textbf{127. AC:} {[}...{]} você tá vendo a legenda?
			
			\textbf{128. KS:} estamos
			
			\textbf{129. AC:} Dá uma olhadinha na legenda\ldots{}
			
			\textbf{130. KS:} Vermelho substantivo---
			
			\textbf{131. GR:} Verde... verbo... azul circunstância... cinza
			
			\textbf{132. AC:} caracterizadores
			
			\textbf{133. GR:} caracterizadores e marrom {[}
			
			\textbf{134. AC:} Determinantes ...vai \ldots{} chega um pouquinho o
			quadro de vocês ... Isso...Olha... aí você observa as cores... \emph{porque
				\emph{às vezes você pode fazer alguma mistura e olha o que tá te
					pedindo}}
			
			\textbf{135. KS: }humrum\textbf{ }((estudantes balançando a cabeça))
			
			\textbf{136. AC: }Então \emph{se eu colocar um substantivo aqui... eu já
					sei que não tá me pedindo um substantivo... nem a ordem é essa na
					frase}... Vamos lá... Olha pra imagem\ldots. Olha pra imagem olha pras
			palavras que você tem \ldots{} ((os estudantes estão pensando... põem a
			mão no queixo)) \phantomsection\label{anchor-7}{} \\
			\bottomrule
		\end{tabular}
	\end{threeparttable}
\end{table}
