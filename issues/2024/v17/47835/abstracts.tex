\begin{polyabstract}
	\begin{abstract}
		Este artigo descreve uma estratégia inovadora do estudo da gramática, em aulas de Língua Portuguesa, em uma escola pública brasileira. Tal estratégia foi mediada por materiais didáticos produzidos colaborativamente por representantes de escola básica e de	universidade, orientados pela abordagem investigativa e pedagógica da	educação  científica, desenvolvida no campo indisciplinar da Linguística Aplicada. Trata-se de uma investigação interventiva caracterizada como
		uma pesquisa baseada em \emph{design}, pois se assume o esforço
		colaborativo para o aprimoramento da prática de ensino e para o
		fortalecimento de teorias de referência. É uma pesquisa financiada pelo
		governo federal do Brasil, dentro do Programa Ciência na Escola (PCE). A
		análise comparativa de respostas compartilhadas por estudante, em
		entrevistas orais realizadas por suas professoras, antes e depois da
		intervenção pedagógica, revelou uma expressiva ressignificação de
		representações discentes sobre aulas de língua materna, gramática,
		pesquisa e ciência. Isso resultou na superação de compreensões pelos
		estudantes, as quais foram compartilhadas previamente de forma ampla e
		equivocada.	
		
		\keywords{Jogos didáticos \sep Estudo da gramática \sep Educação científica}
	\end{abstract}
	
	\begin{english}
		\begin{abstract}
			This article describes an innovative strategy for studying grammar in Portuguese language classes in a Brazilian public school. This strategy was mediated by didactic materials collaboratively
			produced by representatives of elementary schools and universities,
			guided by the investigative and pedagogical approach of scientific
			education, developed in the interdisciplinary field of Applied
			Linguistics. This is an interventional investigation characterized as
			research based on design, as it assumes a collaborative effort to
			improve teaching practice and to strengthen reference theories. This is research funded by the federal government of Brazil, within the Science at School Program (SSP). The comparative analysis of answers shared by students, in oral interviews carried out by their teachers, before and after the pedagogical intervention, revealed a significant redefinition of student representations about mother language classes, grammar, research and science. This resulted in students overcoming understandings that were previously shared widely and mistakenly.
			
			\keywords{Didactic games \sep Study of grammar \sep Scientific education}
			
			
		\end{abstract}
	\end{english}
	% if there is another abstract, insert it here using the same scheme
\end{polyabstract}