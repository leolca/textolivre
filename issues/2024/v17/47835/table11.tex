\begin{table}[!htpb]
	\centering
	\small
	\begin{threeparttable}
		\caption*{\textbf{Exemplo 10.} }
		\label{tab-11}
		\begin{tabular}{@{} 
			>{\raggedright\arraybackslash}p{(\columnwidth - 2\tabcolsep) * \real{0.0300}} 
			>{\raggedright\arraybackslash}p{(\columnwidth - 2\tabcolsep) * \real{0.9700}} @{}}
			\toprule\noalign{}
			\multirow{2}{*}{Luiza Bairros: compreensão de pesquisa} \linebreak \\
			\midrule\noalign{}
			
			
			
			
			\textbf{1ª} & \textbf{171. LB:} \emph{buscar informação... saber mais sobre o assunto} {[}...{]} \\
			
			& \textbf{185. LB: }sim às vezes ela coloca por exemplo ela coloca assim é::... faça alguma coisa \emph{faça uma pesquisa sobre a guerra mundial a gente vai lá e pesquisa é::... guerra mundial e vai aparecer um textão né a gente vai ler resumir e fazer} \\
			
			& \textbf{186. KA: }ah muito bem {[}...{]} \\
			
			& \textbf{187. LB: }é isso  \\
            \midrule
			
			\textbf{2ª } & \textbf{102. LB:} a pesquisa é todo o \emph{percurso} ... que a gente \emph{percorre} é::: estudando livros ...tal ... então resumindo ... pesquisa é todo o \emph{percurso} ... toda uma \emph{trajetória} que a gente faz para chegar a um certo resultado ... \phantomsection\label{anchor-17}{} \\
			\bottomrule
		\end{tabular}
	\end{threeparttable}
\end{table}
