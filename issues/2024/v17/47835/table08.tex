\begin{table}[h]
	\centering
	\small
	\begin{threeparttable}
		\caption*{\textbf{Exemplo 7.} }
		\label{tab-08}
		\begin{tabular}{@{} 
				>{\raggedright\arraybackslash}p{(\columnwidth - 2\tabcolsep) * \real{0.0300}} 
				>{\raggedright\arraybackslash}p{(\columnwidth - 2\tabcolsep) * \real{0.9700}} @{}}
			\toprule\noalign{}
			\multirow{2}{*}{Milton Santos: compreensão de ciência}
			
			\\
			\midrule\noalign{}
			
			
			
			
			\textbf{1ª } & \textbf{218. KA.} {[}...{]} a outra pergunta essa também é pessoal tá Milton? É o que você acha ne? É o que você compreende... é é... o que você compreende por ciência? quando vem a palavra ciência na sua cabeça o quê que você entende?... não é a resposta certa que eu estou querendo eu estou querendo ver o que você entende \\
			
			& \textbf{219. MS:} ~até porque eu \emph{não ia saber dar} ela né? ((rindo)) {[}...{]} \\
			
			& \textbf{223. MS:} ~humm olha assim... eu tenho eu sempre nossa... eu vou uma coisa pessoal aqui exemplo eu tenho muita coisa tipo \emph{eu sempre coloco ciência acima de tudo não em certas coisas do tipo bora se dizer que:: ciência pra mim é::... é:: como eu posso explicar gente... ((pensando)) gente sem palavras sem reação}... pera {[}...{]} \\
			
			& \textbf{228. KA: }nas suas palavras é o que? nas suas palavras é o que? quando vem a palavra ciência na sua cabeça o quê que o quê que vem à mente? \\
			
			& \textbf{229. MS:} ~\emph{primeiro algo necessário} \\
			
			& \textbf{230. KA: }aham \\
			
			& \textbf{231. MS:} ~porque assim gente bora fingir que não existe a ciência um exemplo pro que nós tamo passando agora... qual qual outros meio nós iria usar? \emph{sobrenatural?} \emph{crença?} \\
			
			& \textbf{232. KA: }ah ((ri)) muito bem:: exatamente então \\
			
			& \textbf{233. MS:} ~-\/- eu tenho... eu tenho uma que eu consigo eu sou eu tenho muita crença em certas coisas mas ai gente me subornam demais falam que qualquer coisa é só crença é só crença... ah eu acredito em tal crença tal Deus meu vai fazer tal coisa assim... e de me \emph{misturar até a crença junto com a ciência entendeu}? e e:: \\
			
			& \textbf{234. PD: }e você concorda com isso? \\
			
			& \textbf{235. MS:} ~\emph{não isso me irrita} \\
			
			& \textbf{236. PD: }por quê? \\
			
			& \textbf{237. MS:} ~porque assim eu acho exemplo... a chuva \\
			
			& \textbf{238. PD: }aham \\
			
			& \textbf{239. MS:} ~tem gente que chegou em mim e falou que tal pessoa lá em cima fez chover... \emph{não no meu caso já é o que ciência mostra entendeu}? \\
            \midrule
			
			\textbf{2ª} & \textbf{132. MS:} oh... uma coisa que eu descobri que foi \emph{um BUM} ... foi que \emph{vocês quebraram uma barreira que tinha em mim} ... sobre esse negócio de ciência... quando falou que iriamos \emph{fazer uma pesquisa cientifica da linguagem... fiquei tipo... QUE? COMO ASSIM?}... pois na minha cabeça infantil ainda...tinha aquele negócio de \emph{carinha de cabelo bagunçado}... \emph{aquele branquelo}... estadunidense e tudo mais... com aqueles vidros e tal ... mas... sobre ciência eu digo que é \emph{uma área que abrange ... basicamente tudo... que pode ser estudada e experimental .... tem o exemplo das linguagens também} \\
			\bottomrule
		\end{tabular}
	\end{threeparttable}
\end{table}
