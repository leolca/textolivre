\begin{table}[!t] %Manter assim para melhor alinhamento - João 14/01/24
	\centering
	\small
	\begin{threeparttable}
		\caption*{\textbf{Exemplo 6.}}
		\label{tab-07}
		\begin{tabular}{@{} 
				>{\raggedright\arraybackslash}p{(\columnwidth - 2\tabcolsep) * \real{0.0300}} 
				>{\raggedright\arraybackslash}p{(\columnwidth - 2\tabcolsep) * \real{0.97000}} @{}}
			\toprule\noalign{}
			\multirow{2}{*}{Luiza Bairros: compreensão de gramática} \linebreak
			\\
			\midrule\noalign{}
			
			\noalign{}
			
			
			\textbf{1ª} & \textbf{61. LB: }hum... algumas vezes mas eu \emph{não sei bem o que que é} \\
			
			& \textbf{76. KA: }sim com as suas palavras o que você entende por gramática? você já estudou gramática você estudou isso na escola? \\
			
			& \textbf{77. LB: }\emph{não} \\
			
			& \textbf{78. KA: }\emph{nunca} estudou isso na escola? \\
			
			& \textbf{79. LB:} \emph{não não} que eu lembre {[}...{]} \\
			
			& \textbf{85. KA:} pra falar a verdade eu \emph{não sei muito bem} o que é isso \\
            \midrule
			
			\textbf{2ª } & \textbf{54. LB:} a gente estudava muito ... a gramática era dada pra::... pra dar ordem acho que era para dar ordem se não me engano ao texto ... enfim não estou lembrando muito .... mas agora depois do projeto deu pra entender que:.... \emph{a gramática ela tá pra dá sentido ao texto à fala ou algo...que está sendo mostrada} \\
			\bottomrule
		\end{tabular}
	\end{threeparttable}
\end{table}
