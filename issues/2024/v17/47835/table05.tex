\begin{table}[!htpb]
	\centering
	\small
	\begin{threeparttable}
		\caption*{\textbf{Exemplo 4.} }
		\label{tab-05}
		\begin{tabular}{@{} 
				>{\raggedright\arraybackslash}p{(\columnwidth - 2\tabcolsep) * \real{0.0300}} 
				>{\raggedright\arraybackslash}p{(\columnwidth - 2\tabcolsep) * \real{0.97000}} @{}}
			\toprule\noalign{}
			\multirow{2}{*}{Luiza Bairros: compreensão de aula de LP} \linebreak
			\\
			\midrule\noalign{}
			
			\multirow{8}{*}{1ª} & \textbf{35. KA: }ah perfeito... é:: e o que que você \emph{mais gosta nas aulas de língua portuguesa?} \\
			& \textbf{36. LB: }\emph{os textos} \\
			
			& \textbf{37. KA: }os textos? você gosta? é os textos que você lê que a professora leu os textos que você ouve que a professora é:: lê ... conta pra conta \\
			
			& \textbf{38. LB: }os dois \\
			
			& \textbf{39. KA:} é?.. beleza... O que você \emph{menos gosta nas aulas de língua portuguesa?} \\
			
			& \textbf{41. LB:} \emph{às vezes as atividades} \\
			
			& \textbf{42. KA:} e essas atividades incluem o quê? \\
			
			& \textbf{43. LB:} é que as vezes a gente \emph{não compreende bem o texto e acaba ficando complicado responder as atividades} \\
            \midrule
			
			\multirow{2}{*}{2ª} & \textbf{10. LB:} as aulas \emph{práticas ajudou bastante}... pra que a gente compreendesse melhor o conteúdo ... porque geralmente na sala de aula... é \emph{aquele conteúdo pratico né ...explica, explica}..... e o projeto de vocês .... foi muito é:: \emph{JOGOS É: AULA PRÁTICAS mesmo} .... e isso ajudou bastante o conhecimento {[}...{]} \\
			
			& \textbf{40. LB:} ((risos)) porque com os jogos a gente geralmente a gente está ... \emph{naquela expectativa} ... ah eu preciso ganhar eu quero ganhar e tal e.... acaba que \emph{a gente tem uma empolgação maior e acaba se empenhando mais} ... pra fazer aquele jogo ... eu quero ganhar e tal ... \emph{aí acaba que a gente se concentrando mais... e isso ajuda a gente aprender mais a raciocinar} \\
			
			\bottomrule
		\end{tabular}
	\end{threeparttable}
\end{table}
