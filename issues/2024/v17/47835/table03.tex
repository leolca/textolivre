\begin{table}[!htpb]
	\centering
	\small
	\begin{threeparttable}
		\caption*{\textbf{Exemplo 2.} }
		\label{tab-03}
		\begin{tabular}{@{} >{\raggedright\arraybackslash}p{(\columnwidth - 0\tabcolsep) * \real{1.0000}} @{}}
			\toprule\noalign{}
			Oficina de jogo: mediação da professora
			\\
			\midrule\noalign{}
			
			\textbf{287. AC:} Só uma perguntinha... olha-\/- pode colocar adormeceu-\/- Porque aí se tiver tudo certinho já vai fechar aí o jogo\ldots{} ((GR arrasta o verbo)) Deixa só eu perguntar pra vocês \\emph{oh\ldots{} tremia... tonteou-se... assustou-se adormeceu e morreu} ((KS inclinando olhar para AC)) \emph{É uma ação mental? (...) É uma ação involuntária? (...) É uma ação física?} ---
			
			\textbf{288. GR:} ((simulando tremor)) Tremia... acho que é uma ação involuntária\ldots{}
			
			\textbf{289. AC:} Olha aí pros verbos \ldots{}
			
			\textbf{290. KS:} tonteou-se... ((fazendo movimentos de dúvida com a boca e repetindo o verbo)) tonteou-se
			
			\textbf{291. GR:} ((inclina-se o corpo para o alto)) Tremia é uma ação involuntária que a gente num {[}
			
			\textbf{292. AC:} Olha aí pra esses verbos se eles são---
			
			\textbf{293. GR:} ((inclinando o olhar na direção de KS)) Tonteou- e \ldots{}
			
			\textbf{294. KS:} ((com a mão no queixo)) Pode ser também\ldots{}
			
			\textbf{295. AC:} Em que grupo eles estão?... É do agir do pensar ou do --- Tá certo ...Tonteou - se também
			
			\textbf{296. GR:} adormeceu e morreu também é uma ação involuntária
			
			\textbf{297. AC:} Então qual é o grupo aqui desses verbos?... Que grupo é esse?.... Do agir?... Do pensar?...
			
			\textbf{298. GR:} Do comportar\ldots{}
			
			\textbf{299. KS:} \emph{Do comportar\ldots{} nota 10 ... É isso aí} ((KS arrasta a palavra restante e encerra a fase 2 com a pontuação do JD))
			
			\textbf{300. AC:} Aeee\ldots((risos)) Gente vocês lembram que nós fizemos até uma atividade com esse não é?
			
			\textbf{300. GR:} Anram... com a KA\phantomsection\label{anchor-8}{} \\
			
			\bottomrule
		\end{tabular}
	\end{threeparttable}
\end{table}
