\begin{table}[!t] %Manter assim para melhor alinhamento - João 14/01/24
	\centering
	\small
	\begin{threeparttable}
		\caption*{\textbf{Exemplo 8.} }
		\label{tab-09}
		\begin{tabular}{@{} 
				>{\raggedright\arraybackslash}p{(\columnwidth - 2\tabcolsep) * \real{0.0300}} 
				>{\raggedright\arraybackslash}p{(\columnwidth - 2\tabcolsep) * \real{0.9700}} @{}}
			\toprule\noalign{}
			\multirow{2}{*}{Luiza Bairros: compreensão de ciência}
			\linebreak
			\\
			\midrule\noalign{}
			
			\noalign{}
			
			
			\textbf{1ª } & \textbf{128. KA:} quando vem na sua cabeça -\/- espontâneo assim quando é espontâneo quando você quando você vem alguém fala a palavra ciência isso é uma coisa é:: da ciência o que que vem na sua cabeça? \\
			
			& \textbf{129. LB:} vem \emph{muita matéria} né? \\
			
			& \textbf{130. KA: }{[}a matéria {[} \\
			
			& \textbf{131. LB:} a matéria de ciência \\
			
			& \textbf{132. KA:} \emph{é?} \\
			
			& \textbf{133. LB:} ((olhando para o alto com mão na cabeça e pensando)) é ciência é tudo... né? -\/- mais ou menos... é:: animais faz parte da ciências essas coisas... né? \\
            \midrule
			
			\textbf{2ª } & \textbf{98. LB:} \emph{é o conhecimento}! ((sorrindo))... que \emph{a gente estuda para ter um certo conhecimento} ... \emph{no final adquirimos conhecimento} ... antes nas aulas normais ... \emph{a ciência estava voltada ao corpo humano a essas coisas do tipo}...e... \\
			
			& \textbf{99. AC:} mais especifico uma disciplina \\
			
			& \textbf{100. LB:} isso:... \emph{exatamente específico para uma disciplina} ... e agora com todo o projeto todo estudo ... \emph{a gente entendeu que a ciência num é ... só voltada a uma disciplina ... mas sim voltada ao conhecimento ... porque a gente estuda e no final a gente adquiri um conhecimento} \\
			\bottomrule
		\end{tabular}
	\end{threeparttable}
\end{table}
