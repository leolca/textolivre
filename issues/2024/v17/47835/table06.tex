\begin{table}[!htpb]
\centering
\small
	\begin{threeparttable}
		\caption*{\textbf{Exemplo 5.} }
		\label{tab-06}
		\begin{tabular}{@{} 
				>{\raggedright\arraybackslash}p{(\columnwidth - 2\tabcolsep) * \real{0.0300}} 
				>{\raggedright\arraybackslash}p{(\columnwidth - 2\tabcolsep) * \real{0.9700}} @{}}
			\toprule\noalign{}
\multirow{2}{*}{Milton Santos: compreensão de gramática} \linebreak
			\\
			\midrule\noalign{}
			
			
			
			
			\textbf{1ª } & \textbf{99. KA: }a outra pergunta ela ela não é pra testar seus conhecimentos é o que você sabe com base na sua experiência também tá bom? pra você o que é gramática?... Milton o que é gramática? na sua visão você já ouviu essa palavra alguma vez na escola? pra você o que é? como você diria que é gramática?... \\
			
			& \textbf{100. MS: }eu \emph{não} diria \\
			
			& \textbf{101. KA: }\emph{não} diria? \\
			
			& \textbf{102. MS: }\emph{não} \\
			
			& \textbf{103. KA: }\emph{não}? você \emph{nunca} ouviu falar nessa palavra? \\
			
			& \textbf{104. MS: }sim já mas \emph{creio que estudei há muitos anos atrás} {[}...{]} \\
			
			& \textbf{115. MS: } professora, aí ficou difícil eu posso dá uma sugestão de que que eu estou imaginando que é eu tô (...) {[}...{]} \\
			
			& \textbf{119. MS: } porque eu \emph{não sei se é realmente isso}, espero que também tenho certeza que não vai ser isso (...) {[}...{]} \\
			
			& \textbf{121. MS: }mas é \emph{algo relacionado a histórias do tipo HQ em um livro} sei lá \\
            \midrule
			
			\textbf{2ª } & \textbf{56. MS:} MEU DEUS... olha quero dizer uma coisinha... se minha resposta não for o que você está esperando... não é culpa de vocês e nem NADA ... por que \emph{eu aprendi muita coisa durante isso} ... MAS OHH... \emph{é um conjunto de regra que pode ditar como a estrutura de um texto funciona ... para que ele seja compreensível ... essa é a base de uma coisa bem mais resumida o que é gramática} ... eu sei \\
			
			& \textbf{57. KA:} olha você falou uma palavra aí legal REGRAS ... eu queria que você dissesse \emph{que regras ...são essas que você está se referindo?} \\
			
			& \textbf{58. MS:} aí \emph{depende né ((risos)) se o texto é formal ou informal} \\
			
			& \textbf{59. KA:} é... mas eu quero dizer assim ... são regras de certo ou errado ou regras de funcionamento? \\
			
			& \textbf{60. MS:} \emph{funcionamento ... até porque não existe certo ou errado... existe adequado e inadequação} {[}...{]} \\
			
			& \textbf{75. KA:} {[}...{]} então qual é o papel e a função da gramática ... na sua vida o que ela vai te ajudar?? \\
			
			& \textbf{76. MS:} ah... \emph{Nossa... Em tudo...porque praticamente TUDO é linguagem ... em todo canto em todo lugar... a gramática infelizmente... não sei se eu diria infelizmente ou felizmente... mas dependendo de cada situação e principalmente nesse meio acadêmico.... e coisa de trabalho... toda coisa que vamos precisar é a formal... entendeu?... então a gramática faz a gente compreender muito bem... o posicionamento das coisas e onde usá-las} ... travou? \\
			
			& \textbf{77. KA:} não! Eu estou ouvindo perfeitamente! \\
			
			& \textbf{78. MS:} ah tá... e justamente ... \emph{por tudo ser linguagem a gramática é:: a que vai tá com o pé inicial pra tudo ...sabe que vai te dá uma base de compreensão e de mundo eu diria}  \\
			\bottomrule
		\end{tabular}
	\end{threeparttable}
\end{table}
