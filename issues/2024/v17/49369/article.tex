\documentclass[english]{textolivre}

% metadata
\journalname{Texto Livre}
\thevolume{17}
%\thenumber{1} % old template
\theyear{2024}
\receiveddate{\DTMdisplaydate{2024}{1}{2}{-1}}
\accepteddate{\DTMdisplaydate{2024}{2}{18}{-1}}
\publisheddate{\DTMdisplaydate{2024}{5}{20}{-1}}
\corrauthor{Francisco Javier Olivar de Julián}
\articledoi{10.1590/1983-3652.2024.49369}
%\articleid{NNNN} % if the article ID is not the last 5 numbers of its DOI, provide it using \articleid commmand 
% list of available sesscions in the journal: articles, dossier, reports, essays, reviews, interviews, editorial
\articlesessionname{articles}
\runningauthor{Olivar de Julián}
%\editorname{Leonardo Araújo} % old template
\sectioneditorname{Daniervelin Pereira}
\layouteditorname{João Mesquita}

\title{Automated journalism in news about traffic accidents: emergency services information dump}
\othertitle{Jornalismo automatizado em notícias sobre acidentes de trânsito: despejo de informações de serviços de emergência}

\author[1]{Francisco Javier Olivar de Julián~\orcid{0000-0002-2030-2458}\thanks{Email: \href{mailto:franciscojavier.olivar@unir.net}{franciscojavier.olivar@unir.net}}}

\affil[1]{Universidad Internacional de La Rioja, Facultad de Empresa y Comunicación / ESIT, Logroño, La Rioja, España.}

\addbibresource{article.bib}

\begin{document}
\maketitle
\begin{polyabstract}
\begin{abstract}
In this work, the process of producing news about
events in the Spanish digital press has been studied. A content analysis
was carried out on news related to the main external causes of mortality
(traffic accidents, accidental falls, drowning and suicides) published
in the six main Spanish digital media (\url{elpais.com}, \url{ elmundo.es}, \url{abc.es}, \url{lavanguardia.com}, \url{elconfidencial.com} and \url{20minutos.es})
during the period 2010-2020. The selected news items ($n=5,727$) were
obtained through the digital newspaper library \emph{Mynewsonline}. A
qualitative study has also been carried out (in-depth interviews with
professionals and a focus group). The results show a high number of news
items published on traffic compared to other external causes of death
with higher mortality. It is also appreciated that the media
 automatically transfer the information they receive from the emergency
services. It has even been detected that most of the news events coded
in this study are exact copies that are published on the same day in
different media (headlines and the body of the news).

\keywords{Automated journalism \sep News \sep Traffic accidents \sep Digital press \sep Mynewsonline}
\end{abstract}

\begin{portuguese}
\begin{abstract}
Neste trabalho foi estudado o processo de produção de
notícias sobre acontecimentos na imprensa digital espanhola. Foi
realizada uma análise de conteúdo de notícias relacionadas às principais
causas externas de mortalidade (acidentes de trânsito, quedas
acidentais, afogamentos e suicídios) publicadas nos seis principais
meios de comunicação digitais espanhóis (\url{elpais.com}, \url{ elmundo.es}, \url{abc.es}, \url{lavanguardia.com}, \url{elconfidencial.com} e \url{20minutos.es})
durante o período 2010-2020. As notícias selecionadas ($n=5.727$) foram
obtidas por meio da joalheria digital \emph{Mynewsonline}. Também foi
realizado um estudo qualitativo (entrevistas em profundidade com
profissionais e grupo focal). Os resultados mostram elevado número de
notícias publicadas sobre trânsito em comparação com outras causas
externas de morte com maior mortalidade. Também é apreciado que os meios
de comunicação transfiram automaticamente as informações que recebem dos
serviços de emergência. Foi ainda detectado que a maior parte das
notícias codificadas neste estudo são cópias exatas que são publicadas
no mesmo dia em diferentes meios de comunicação (manchetes e corpo da
notícia).

\keywords{Jornalismo automatizado \sep Notícias \sep Acidentes de trânsito \sep Imprensa digital \sep Mynewsonline}
\end{abstract}
\end{portuguese}
\end{polyabstract}

\section{Introduction}\label{sec-introduction}

Over the past decades there has been a growing interest in the
development of new methodologies for second language (L2) learning and
teaching, to fulfil the requirements of the new legislation, as defined
by the Common European Framework of Reference for Languages (CEFR),
which emphasizes the communicative aspect of languages.

One of these new methodologies is Didactic Audiovisual Translation
(DAT), i.e., the application of different modes of audiovisual
translation (subtitling, revoicing, audio description and voice over) to
L2 teaching. This approach has been implemented and its results analysed
in different educational stages. Nevertheless, this approach has yet
been used in primary education within alternative methodologies, such as
the Montessori Method, in blingual contexts (in this case,
Basque-Spanish). The present study aims to address this research gap.

This paper shows the results of an intervention combining subtitling and
dubbing in a class of 11-12-year-old pupils following the Montessori
Method, at a Basque-immersion language primary school. While our study
could not assess language acquisition improvements due to the
school's methodology, our findings regarding children
satisfaction and the alignment of DAT principles with Montessori
principles, as demonstrated in the text, suggest promising avenues for
successful DAT use in this environment.



\section{Theoretical background}\label{sec-theoreticalbakcground}

The use of automation in the production of journalistic articles
confronts journalism with threats, opportunities, and ambiguities.
Research has been found indicating that news articles written by humans
differ from news articles written through automation processes. News
written using algorithms shares some similarities with news written by
humans, such as the focus on current events or the use of the inverted
pyramid. But there are also differences in terms of news value; Articles
written by humans tend to exhibit more negativity and impact than
articles written by algorithms. News articles written by humans are more
likely to include interpretation, while articles written by algorithms
tend to be shorter and do not use human sources \cite{tandoc2022noticias}.

Studies have also been found that review the practice of automated
journalism and identify an important limitation on the potential of
automating journalistic writing, such as the absence of sufficient data
models to encode the journalistic knowledge necessary to automatically
write narratives based on events \cite{caswell2018automated}.

On the other hand, in the interview with Eric Aislan Antonelo \cite{rocha2019inteligencia} he warns of the existence of AI, as an emulation of the
intelligence of human beings with applications such as automatic
translation, voice recognition and question and answer systems.

Likewise, articles have also been found based on the discourse of the
information values of the elites and the personalization in the source
of information \cite{manoso2020news} and journalism studies inspired by
Bourdieu that try to explain how journalists make sense of the change
related to technology in the journalistic field \cite{lindblom2022digitalizing}. This seems to link with the idea of the precariousness
to which journalists have been subjected, the main consequence of which
has been to turn their profession into a mere transmitter of messages
instead of being a builder of news with the proper framing work.

\section{Methodology}\label{sec-methodology}

In this paper, a technology developed by Yandex in 2021 that translates
a live stream from English, Spanish, French, Italian, German, or Chinese
into Russian has been used. The translation of a live stream is a
challenging task that has been addressed by the development of a novel
technique based on neural networks. Our study is devoted to the
evaluation of this new technique and focuses on its ability to preserve
the subtleties of semantic meaning and cultural connotations inherent in
phraseological expressions in different linguistic contexts. Although
our study does not address the technical details underlying this tool,
we address these issues as they may help to inform translation. This
technology incorporates advanced machine learning algorithms to enable
the instantaneous translation of language during audiovisual broadcasts.
In essence, this algorithm comprises five fundamental steps, each
executed by a distinct neural network model.

Initially, the audio stream is captured and transcribed into plain text
using automatic speech recognition. The video may contain extraneous
sounds such as noise and music, people may speak with different accents,
speeds and diction, and there may be many speakers, so the technology
must ensure that context and coherence are maintained during the
translation process. Therefore, the algorithm takes a sequence of audio
chunks as input, extracts acoustic features, and passes them into the
neural network. The neural network in turn produces a set of word
sequences from which the language model selects the most plausible
hypothesis.

Subsequently, a machine translation model is employed to translate the
text into the desired target language. There are several problems here:
if you translate word-by-word or phrase-by-phrase, the quality will
suffer, and if you wait for a long pause to guarantee the end of a
sentence, there will be a long delay. So, the technology groups words
into sentences without losing meaning or making sentences too long. For
correct translation at this stage, it is also necessary to determine the
gender of the speaker to determine to whom a particular line belongs and
to reproduce the voice correctly. After selecting individual sentences
and lines, the translation is performed, for which Yandex uses its own
translator.

Once the translation is completed, the translated text is processed
through text-to-speech technology, which converts the written content
into spoken audio. This step ensures that the generated speech sounds
natural and coherent, considering various linguistic features such as
tone, rhythm, and inflection. Additionally, the gender of the speaker,
which was previously identified during the initial stages of the
process, is incorporated into the synthesis to ensure the voice matches
the intended speaker's profile. This level of
customization helps improve the overall quality and authenticity of the
synthesized speech, making it more relatable and contextually
appropriate for the target audience.

Furthermore, the algorithm ensures that the translated speech is
accurately synchronized with the corresponding segments of the video
stream, aligning seamlessly with the visual content, and maintaining
synchronization with the video frames. This process is crucial to ensure
that the audio matches the timing of the speaker's lip movements and
actions in the video. Additionally, the neural network addresses several
challenges in this stage, such as when the speaker delivers a sentence
rapidly, or when the translated sentence is significantly longer than
the original. In these cases, the system must dynamically adjust the
synthesized audio by compressing or shortening it to fit within the
allotted time frame, ensuring smooth and natural speech flow that aligns
with the visual context.

Finally, the translated speech is seamlessly integrated into the live
video stream, replacing the original audio with the newly generated
translated audio. This newly created audio is then encapsulated into an
audio stream, which is embedded directly into the browser interface of
the viewer, allowing them to hear the translated speech while watching
the video in real-time. The technology used for this process is
currently functional exclusively within the Yandex browser, which was
the platform selected for the study. \Cref{fig-01} displays a detailed
screenshot of the Yandex browser interface, clearly highlighting the key
components involved in the translation and audio synchronization
process. It provides a visual representation of how the translated audio
is integrated into the video stream, showing the alignment between the
original video content and the overlaid translated speech. The
screenshot also highlights the user interface elements that facilitate
the viewer's interaction with the translated video, such as volume
controls, language options, and playback features.

\begin{figure}[htpb]
  \centering
  \begin{minipage}{\textwidth}
  \caption{Screenshot of the Yandex browser interface when utilizing 
  the automatic translation features.}
  \label{fig-01}
  \includegraphics[width=\textwidth]{image1.png}
  \source{Author's own work}
  \end{minipage}
\end{figure}

To enhance the study's transparency and validity, it is
crucial to provide comprehensive details regarding the sample selection,
data collection methods, and the analytical techniques employed. The
live video functionality within the interface is activated via a clearly
visible and easily accessible button, allowing for straightforward
interaction with the system. A notable feature of the translation
process is the 40-50 second temporal delay between the original live
video stream and its translated counterpart in Spanish. This delay, as
derived from a detailed evaluation of the system's operation, is
purposefully incorporated to allow sufficient time for the system to
process the content contextually, which is essential for delivering an
accurate translation in real-time, especially when dealing with live
broadcasts. Such a delay also enables the system to manage the
complexities inherent in real-time translation, such as adjusting for
idiomatic expressions, speech nuances, and varying speaking speeds.

Moreover, the Yandex browser's functionality extends beyond mere passive
translation by allowing users to actively customize their experience.
This includes features like adjusting the volume of the original audio
track, which could be crucial in environments where background noise or
other factors might interfere with the audio quality. Additionally, the
availability of subtitles provides an extra layer of understanding and
flexibility for users, especially in cases where visual cues or context
may not fully suffice to ensure a clear understanding of the translated
content. These customizable features add a significant level of
adaptability to the translation process, catering to various user
preferences and enhancing the overall user experience. Such details are
integral to understanding the technical infrastructure of the study and
ensure its findings can be accurately interpreted and replicated in
future research.

This study employed the Yandex browser and a range of accessible
technological tools to explore the automated translation of live news
streams. The analysis focused on two YouTube channels, ``RTVE Noticias''
and ``Canal Sur Andalucía,'' both of which provide continuous news
coverage. These channels were selected as the primary subjects of
investigation due to their ongoing news broadcasts, providing a robust
sample for examining the translation of live content. The methodology
for the study, as outlined in \Cref{fig-02}, was structured to facilitate a
systematic comparison of the automated translation of phraseological
expressions in real-time news streams.

The approach for comparing the automated translation of phraseological
expressions involved several key stages:

\begin{enumerate}
\def\labelenumi{\arabic{enumi}.}
\item
  \emph{Audio recording:} The first step in the methodology involved
  recording both the original live stream (audio recording \#1) and its
  corresponding translation (audio recording \#2) simultaneously,
  ensuring that both recordings occurred at the same intervals during
  the live broadcast. Two separate devices were used to capture these
  audio recordings concurrently, ensuring precise alignment of the
  original and translated content.
\item
  \emph{Detection of phraseologisms (verbal idioms):} In the second
  stage, instances of phraseologisms, commonly used idiomatic
  expressions, were identified in audio recording \#1, which captured
  the live stream in the original language (Spanish). The identification
  process was carefully carried out to ensure that the expressions
  detected were contextually relevant and representative of the language
  used in the broadcast.
\item
  \emph{Translation matching:} Once the phraseologisms were identified
  in the original audio, the next step involved matching the
  corresponding translations of these expressions in audio recording
  \#2, which was in Russian. This step was crucial for establishing
  whether the automated translation system accurately rendered the
  idiomatic expressions in a culturally appropriate and linguistically
  accurate manner.
\item
  \emph{Comparison and analysis:} The final stage entailed a detailed
  comparison of the accuracy and correctness of the translations. A
  comprehensive comparison table was developed to facilitate this
  process, allowing for a side-by-side evaluation of the original and
  translated expressions. The total duration of the audio recordings in
  both Spanish (the original language) and Russian (the translated
  language) amounted to 243 minutes for each language. These recordings
  were made across different intervals, ranging from 15 to 60 minutes,
  and were captured on multiple occasions throughout the day over four
  non-consecutive days. This sampling strategy was employed to ensure a
  diverse representation of news topics covered during the broadcasts.
  As part of the analysis, 52 verbal idioms were identified and
  thoroughly examined within the context of the live news stream,
  providing insights into the effectiveness of the automated translation
  system in handling phraseological expressions in a dynamic, real-time
  setting.
\end{enumerate}

\begin{figure}[htpb]
  \centering
  \begin{minipage}{\textwidth}
  \caption{General outline of the methodology presented.}
  \label{fig-02}
  \includegraphics[width=\textwidth]{image2.png}
  \source{Author's own work.}
  \end{minipage}
\end{figure}




\section{Results}\label{sec-results}

\subsection{First Expectations}\label{sub-sec-firstexpectations}

The authors hypothesised that pupils would be able to achieve a higher
linguistic level as a result of the unquestionable stimulation provided
by the audio-visual tools, which have been demonstrated to address the
learning needs of students. Additionally, in accordance with the
findings of previous research, it was anticipated that the
children\textquotesingle s level of motivation for the subject would
increase.

Nevertheless, it is reasonable to anticipate certain challenges
associated with the computer skills and linguistic proficiency of the
pupils. With regard to the latter, it is important to recognise that
children are simultaneously acquiring three languages without formal
instruction, which could potentially lead to some difficulties.

\subsection{Results on Acquisition of the Language}\label{sub-sec-resultsonacquisition}

As previously stated, the Montessori Method does not include exams,
therefore it was not feasible to propose a pre-test and post-test
framework to collect quantitative data. Consequently, it is not possible
to provide evidence of language improvement, if any. Nonetheless, it was
possible to anticipate certain improvements in their productions, given
that the children were engaged in activities such as translating,
dubbing and subtitling videos, practising listening comprehension,
pronunciation, writing and speaking.

A further defining feature of Montessori is that children select the
areas of study that they wish to pursue. Thus, this approach presented a
significant challenge for them, as their choices were based on the
scenes they enjoyed, irrespective of the level of L2 proficiency
required. This is the reason why the implementation resulted in
unforeseen language difficulties for the children, which obliged the
teacher to provide linguistic support by scaffolding the language.

Although it was not possible to analyse in detail the language
improvement resulting from the implementation, some common mistakes
committed by the pupils could be identified. These can be divided into
two main groups: those related to written productions and those related
to oral ones. In general, the children committed grammatical mistakes in
order to fit the sentences with the speech, such as the removal of the
subjects or the elision of auxiliary verbs. The errors associated with
oral production were primarily related to the rhythm of the
conversations and the pronunciation, particularly that of the vocalic
phonemes, as the pupils read the words in a literal manner. It is
important to highlight the intricate nature of these phonemes for
speakers of Basque and Spanish. In comparison to English, which has
twelve vocalic sounds, the two languages in question possess only five.
Furthermore, the pace of the original dialogues also affected the
pupils\textquotesingle{} oral productions, necessitating adjustments in
their speaking rate to align with the tempo of the original dialogues.

\subsection{Results of the Questionnaire}\label{sub-sec-resultsofthequestionnaire}

Following the completion of the implementation, pupils completed a
survey about AVT and DAT (see \Cref{annex-01}). Unfortunately, as the teachers
did not compel the pupils to fill it, and children were being prepared
for their incorporation to the formal instruction of the next course
(1st of Secondary Education), only seven children answered to the
questionnaire. The questions addressed their opinions regarding the
utilisation of DAT, the motivation it provides and its potential
integration into future English classes. As the pupils are taught in the
Basque language, and given the inherent complexity of the vocabulary and
the dearth of knowledge about the subject matter, the survey was
developed in Basque. However, for the sake of clarity and accessibility,
the sentences have been translated into English in the title of the
figures.

The initial question (see \Cref{fig-01}) sought to ascertain the
children\textquotesingle s level of enjoyment in the workshop. All of
the participants provided a rating that was above the minimum passing
grade. Four children, representing over half of the sample, rated the
workshop with an 8 or 9, while two children assigned a 6, and one child
a 7. These ratings indicate that the children felt at ease and content.
It is noteworthy that none of the children reached the 10-point mark,
given that they live in a multimodal world surrounded by video and
gaming platforms, which are likely to exert a strong influence on their
preferences. It may be posited that the introduction of this novel
activity, coupled with the necessity to master the utilisation of
internet-based tools, has instilled a sense of unease and lack of
confidence amongst them.
\begin{figure}[htbp]
    \centering
    \begin{minipage}{.5\textwidth}
    \includegraphics[width=\textwidth]{fig01.png}
    \caption{Question 1. From 1 to 10, how much have you enjoyed
    the ambiance of the workshop?}
    \label{fig-01}
    \source{Owm elaboration.}
    \end{minipage}
\end{figure}

In question 2 (see \Cref{fig-02}), the respondents were asked about their
previous knowledge of AVT. Four of the participants were unaware of its
existence, while the remaining three were cognizant of it. These figures
were unanticipated, given that the children have access to audiovisual
products in three different languages. This may have led them to assume
that there is a process of translation behind that range of options. It
seems reasonable to posit that the most probable reason for this lack of
awareness is that, due to their age, they are accustomed to consuming
audiovisual content in those languages and have not considered the
processes involved.

\begin{figure}[htbp]
    \centering
    \begin{minipage}{.5\textwidth}
    \includegraphics[width=\textwidth]{fig02.png}
    \caption{Question 2. Did you know what audiovisual translation
    was? (green-yes, purple-no).}
    \label{fig-02}
    \source{Owm elaboration.}
    \end{minipage}
\end{figure}

The answers to question 3, if they had enjoyed learning the L2 by means
of the editing of videos (see \Cref{fig-03}) were also positive.
Three-quarters of the children, five, demonstrated a positive attitude
towards the methodology employed, while two of them expressed a negative
opinion. Once more, we may cite the necessity of learning something new
compulsory as the primary reason for their refusal. These results align
with the responses to questions 1, 5 and 6, which will be analysed
subsequently.

\begin{figure}[htbp]
    \centering
    \begin{minipage}{.5\textwidth}
    \includegraphics[width=\textwidth]{fig03.png}
    \caption{Question 3. Have you enjoyed learning English by
    means of video editing?}
    \label{fig-03}
    \source{Owm elaboration.}
    \end{minipage}
\end{figure}

In response to question 4, which pertains to the two modes implemented
in class and translation (see \Cref{fig-04}), it can be observed that data do
not provide a clear indication of the pupils\textquotesingle{}
preferences towards either mode. The preference for dubbing is indicated
by a single pupil, while the other two modes were selected by two pupils
each. It is noteworthy that two children selected translation, a written
activity that is not directly relevant to their lived experience, rather
than the other two modes, which are inherently more visual.

\begin{figure}[htbp]
    \centering
    \begin{minipage}{.5\textwidth}
    \includegraphics[width=\textwidth]{fig04.png}
    \caption{Question 4. Which tool have you liked the most?
    (Green: interlingual indirect translation of texts; purple: dubbing;
    blue: subtitling).}
    \label{fig-04}
    \source{Owm elaboration.}
    \end{minipage}
\end{figure}


In question 5, the children were asked about the process of language
acquisition. As illustrated in \Cref{fig-05}, all pupils demonstrated an
understanding that they would benefit from these educational
initiatives. It is noteworthy that five of the pupils awarded the
project an 8, a high rating that reflects their trust in DAT. One pupil
considered the quality to be satisfactory, while one rated it slightly
above average. Overall, the marks are deemed satisfactory, as none of
the responses were below the passing mark of 5. These responses are
consistent with those provided in question 2, in which two children
indicated a lack of knowledge regarding AVT.

\begin{figure}[htbp]
    \centering
    \begin{minipage}{.5\textwidth}
    \includegraphics[width=\textwidth]{fig05.png}
    \caption{Question 5. From 1 to 10, how much do you think you
    would learn?}
    \label{fig-05}
    \source{Owm elaboration.}
    \end{minipage}
\end{figure}

The final question (number 6) asked the pupils whether they would
like to engage in DAT activities within the classroom setting. Five
pupils responded in the affirmative, while the remaining two expressed
opposition to the proposal. However, their responses lacked sufficient
clarity or referenced their disapproval of the instructor rather than
the DAT activities themselves. It can be inferred that these two
students are the same individuals who, in question 1, rated the workshop
atmosphere as 6, who provided negative responses to question 3 regarding
their enjoyment along the process, and who rated the development as 6
and 7. This is somewhat surprising, given that these ratings are not
particularly low.

Question 6 (original answers in \Cref{annex-03})

\begin{itemize}
    \item \textbf{Pupil 1}: Yes, because I learn.
    \item \textbf{Pupil 2}: Yes, because I like making videos a lot.
    \item \textbf{Pupil 3}: Yes, because I have never tried it and I think it would be good.
    \item \textbf{Pupil 4}: Yes, because I like recording videos a lot.
    \item \textbf{Pupil 5}: Yes, because it is fun.
    \item \textbf{Pupil 6}: No, because I do not want X (a teacher) to appear.
    \item \textbf{Pupil 7}: No.
\end{itemize}

\section{Conclusions}\label{sec-conclusions}

One of the main types of evidence of this study has been the
confirmation of the great difference that exists between the number of
news published in all media about traffic accidents (89.14\%) compared
to the other three causes studied.

In addition to this important data, the great finding of this study has
been the confirmation of the high number of accurate headlines that also
fully coincide with the development of the news (27.15\%).

This result suggests that the media are using automated journalism,
directly uploading the information they receive from emergency services
or information agencies, without modifying, working on or adapting the
news at all.

This news production mechanism is used by all the media, to a greater or
lesser extent, with \url{lavanguardia.com} being the media that uses it
the most (66.23\% of all its news on traffic accidents are identical to
those of other media). In total, 663 duplicated, 17 tripled and 6
quadrupled news items have been detected, which means that the same
headline and body of the news item is repeated on up to six occasions in
exactly four media outlets at the same time.

In any case, an analysis has also been carried out on the exact news
published by all these media about other types of events, obtaining
similar results for drowning and accidental falls but observing a
different informative treatment in the case of suicides, where only
11.00\% of coincidences with other journalistic pieces (12 news).

Regarding the results of the qualitative analysis, a dependency of the
media on the notes issued by the emergency services is observed in the
in-depth interviews, although it is stated that a contrast of this
information is carried out before incorporating it as a publication.

It is indicated that the news of events are published to a greater
extent because they arouse interest in the audience and that the
sensitive data of the victims is taken into account in their production.

On the other hand, the discussion group believes that journalistic work
is carried out superficially and, in the case of suicides, without
social commitment. Regarding this first statement, the fact that more
and more traffic news is being published automatically seems to prove
them right.

Therefore, taking the agenda-setting theory as a basis for reflection,
where the media influence the issues that concern and speak of citizens
-in this case, events-, it seems that they have finally managed to shape
the public agenda so that the citizens talk preferentially about traffic
accidents over other types of events \cite{scheufele2007framing}.

In view of all these special characteristics described, it can also be
seen that there is automated journalism in the writing of traffic
accident news, which in most cases are automatically reported from the
emergency services to the media themselves without adapt the news
neither to its editorial line nor to its audience.

In short, attention must be drawn to the vicious circle that can be
generated if the media continue to offer exhaustive information on
traffic accidents, while ignoring the reality of some events that also
represent a social problem.


\printbibliography\label{sec-bib}

\end{document}
