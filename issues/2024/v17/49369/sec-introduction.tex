\section{Introductioon}\label{sec-introduction}

Event journalism mainly contemplates homicides, traffic accidents, drug
trafficking and robberies. Also, suicides, accidental falls, and
drowning, although the same informative treatment is not always applied
to all of them \cite{olivar2020tratamiento}.

Among all these topics, traffic accidents prevail in terms of
information compared to the rest due to their practically daily
frequency, the victims, and deaths they generate, the spectacularity and
visual impact of the images and the social alarm they arouse
\cite{rodriguez2011informacion}.

Event journalism understood until now as \enquote{specialized journalistic
information that deals with a varied subject centered on the commission
of crimes, accidents, catastrophes and curious and surprising facts}
\cite[p. 2]{rodriguez2011informacion}, faces in the 21st century the challenge
of a new technological dimension that allows the possibility of
producing news through automated journalism \cite{jamil2021automated}. In this
sense, people are already beginning to talk about \enquote{Robot Journalism},
\enquote{Automated Journalism} and even \enquote{Cognitive Journalism} where Artificial Intelligence (AI) has begun to occupy a field traditionally dominated by the human factor in the management of organizations. Also, in the media and in society through the application of Data Mining, to generate
algorithms that allow automation of management and refer it to the work
of bots in the production of news \cite{tunez2020from}.

It may seem that the journalistic profession is seemingly oblivious to
the robotization of newsrooms, but the origin of mass automation dates
to 2015, when the Associated Press (AP) automatically generated 3,000
\enquote{corporate earnings} stories in the US each quarter (IMT, 2019).

The editorial line, the agenda setting itself and the political
orientations of each communication media generate differences in the way
of covering the news about accidents \cite{arce2017accidentalidad}, but, within the information of events, it seems that the media
have shown a preference for certain types of news and events \cite{duran2020responsabilidad}. Specifically, it is expected to find a greater
number of news items referring to traffic accidents compared to other
types of events.

The agenda-setting theory maintains that the media influence the issues
that concern and speak about citizens, that is, that they can shape the
public agenda, in this case making it easier for citizens to talk
preferentially about traffic accidents over other types of events
\cite{scheufele2007framing}.

When the media publishes a greater amount of news about traffic
accidents, readers are not only informed about it, but they are also
influenced to think and give their opinion on that topic. In other
words, the media have a certain capacity to establish the agenda of the
set of issues about which a society speaks or is debated at a given
moment \cite{mccombs1972agenda}. In this way, it is intended to influence
the construction of news and information that they send to the public
\cite{vos2015how}.

It is also intended to obtain an answer on whether this greater number
of news publications on traffic accidents responds to a rigorous
elaboration and with its own elaboration or if the media are limited to
automatically transferring the information, they receive from the
emergency services. The media are in permanent contact with these
services, which are the ones who send them the necessary information to
produce news about traffic accidents, but the immediacy and permanent
updating of this information affects their journalistic rigor,
especially in digital newspapers.

According to González Ortiz, head of the \enquote{events and courts section} of
\emph{Diario de Navarra}, the main sources are the emergency services
and the police (personal communication, October 24, 2018). Crime
reporters draw on a variety of sources including:
\begin{itemize}
	
\item The police (through their press offices);
\item News agencies (mainly from \emph{EFE} and \emph{Europa Press});
\item Those of authors, victims, and witnesses (important sources but
difficult to obtain);
\item The judicial ones (the event journalist is usually also a court
journalist);
\item Undetermined sources (with and without attribution);
\item Other sources (forensic, penitentiary, neighborhood, union, health,
state, regional or municipal administrations, the media or traffic,
surveillance, emergency, or relief services \cite{rodriguez2015manual}.
\end{itemize}

Among all of them, journalists resort more frequently as a source of
information to prepare news about events to the police sources, which
include the National Police Corps, Local Police and Civil Guard, which
are the ones that make up the Security Forces and Corps of the State, as
established in Organic Law 2/1986, Security Forces and Corps (1986).
There are also specific services in each autonomous community.

The general objective of this study is to analyze the similarities
between news about traffic accidents (headlines and complete
journalistic pieces) published in different digital media.

The following are proposed as specific objectives:

Obtain information about the existence and repetition of exact news of
traffic accidents.

Obtain enough evidence to conclude that the media carry out a simple
dump of the news obtained from the emergency services and that,
therefore, there is an automated journalism model for the transmission
of news about traffic accidents.