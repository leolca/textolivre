\section{Conclusions}\label{sec-conclusions}

One of the main types of evidence of this study has been the
confirmation of the great difference that exists between the number of
news published in all media about traffic accidents (89.14\%) compared
to the other three causes studied.

In addition to this important data, the great finding of this study has
been the confirmation of the high number of accurate headlines that also
fully coincide with the development of the news (27.15\%).

This result suggests that the media are using automated journalism,
directly uploading the information they receive from emergency services
or information agencies, without modifying, working on or adapting the
news at all.

This news production mechanism is used by all the media, to a greater or
lesser extent, with \url{lavanguardia.com} being the media that uses it
the most (66.23\% of all its news on traffic accidents are identical to
those of other media). In total, 663 duplicated, 17 tripled and 6
quadrupled news items have been detected, which means that the same
headline and body of the news item is repeated on up to six occasions in
exactly four media outlets at the same time.

In any case, an analysis has also been carried out on the exact news
published by all these media about other types of events, obtaining
similar results for drowning and accidental falls but observing a
different informative treatment in the case of suicides, where only
11.00\% of coincidences with other journalistic pieces (12 news).

Regarding the results of the qualitative analysis, a dependency of the
media on the notes issued by the emergency services is observed in the
in-depth interviews, although it is stated that a contrast of this
information is carried out before incorporating it as a publication.

It is indicated that the news of events are published to a greater
extent because they arouse interest in the audience and that the
sensitive data of the victims is taken into account in their production.

On the other hand, the discussion group believes that journalistic work
is carried out superficially and, in the case of suicides, without
social commitment. Regarding this first statement, the fact that more
and more traffic news is being published automatically seems to prove
them right.

Therefore, taking the agenda-setting theory as a basis for reflection,
where the media influence the issues that concern and speak of citizens
-in this case, events-, it seems that they have finally managed to shape
the public agenda so that the citizens talk preferentially about traffic
accidents over other types of events \cite{scheufele2007framing}.

In view of all these special characteristics described, it can also be
seen that there is automated journalism in the writing of traffic
accident news, which in most cases are automatically reported from the
emergency services to the media themselves without adapt the news
neither to its editorial line nor to its audience.

In short, attention must be drawn to the vicious circle that can be
generated if the media continue to offer exhaustive information on
traffic accidents, while ignoring the reality of some events that also
represent a social problem.