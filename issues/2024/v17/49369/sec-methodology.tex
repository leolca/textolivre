\section{Methodology}\label{sec-methodology}

This work follows the recommendations that recommend the combined use of
quantitative and qualitative methodologies in studies of a social
nature. This mixed methodology provides a broader vision for the study
on the media treatment of an issue or event \cite{jensen2015comunicacion}, since it
reduces the limitations that would entail exclusively analyzing
subjectivity with the qualitative methodology and objectivity with the
quantitative one \cite{sanchezgomez2015}.

Therefore, this study combines a quantitative content analysis of the
informative treatment offered by the Spanish digital media on traffic
accidents and other main events with a qualitative analysis that
assesses, through in-depth interviews and a discussion group, the
perception on the publication of news about events according to the
different causes that motivate them. In this way, it is intended to
obtain final information that offers a more complete vision of the
object of study.

\subsection{Sample selection}\label{sub-sec-sampleselection}

The chosen unit of analysis is the journalistic piece published by the
selected digital newspapers, related to the causes of accidents
indicated in the study interval (2010-2020).

The choice of digital media to carry out this research is not by chance,
since more and more press is being consumed on the internet and less on
paper \cite{boasberg2019marco}.

To determine the sample, a mixture of immigrant and digital native
newspapers has been selected, considering that the combination of both
will collect the sample of all online newspapers with greater fidelity.
Digital immigrant newspapers are those that have made an \enquote{adaptation of
traditional newspapers to new digital media and their interface}
\cite[p. 27]{peña2016european}. On the other hand, in this work digital native newspapers are understood to be those that were directly born in the digital sphere, also considering
those that have become digital in a period not exceeding five years from
their birth. An example of this is \emph{20minutos.es}, which was born
on February 3, 2000, as a traditional newspaper and became digital in
2005, the year in which it became the first Spanish newspaper to have a
Creative Commons license (which allows copying literally their news
citing the source). It was also the first online newspaper to open all
its contents to the comments of its readers \cite{lopez2012tratamiento}.

Following criteria of relevance and popularity for this type of media
(number of visits/month and number of unique users), the digital
immigrant newspapers, \emph{elpais.com, elmundo.es, abc.es} and
\emph{lavanguardia.com} have been selected. and the digital native
newspapers \emph{elconfidencial.com} and \emph{20minutos.es}.

The audience data has been obtained from the General Media Study
\cite{boasberg2019marco} and from the internet
marketing research company \textcite{comscore2017rating}. These are entities that
offer data from offline and online media and that are the usual pattern
as data sources for both national \cite{galletero2018estudio} and international studies \cite{potvin2018frequency}.

The exclusion/inclusion criteria have been defined according to the
recommendations followed for other types of research, such as that of
\textcite{zimmermann2019content}.

Only news related to traffic accidents, drowning, accidental falls, and
suicides in the 2010-2020 study period were coded. News about accidents
with or without victims published by the media selected for the sample
and related to events that occurred anywhere in the world, were
included. Those journalistic pieces that had a character of current
event were considered valid. This current circumstance refers to the
present time but also to the immediate past, where the news relates an
event that is happening or that has just happened. For this reason, news
of an event that occurred within the same day of publication or the day
immediately before were included.

All those pieces not related to one (or several) specific aspects have
been left out of the study due to the causes analyzed.

\subsection{Extraction method}\label{sub-sec-extractionmethod}

The sample was retrieved from the digital newspaper library
\emph{Mynewsonline}, which includes material published since 2010. This
tool has been used in another research works that required a
chronological sample \cite{repiso2018universidades,garcia2018quality}. The keywords used in the search were the causes
of mortality (suicide, falls, drownings and traffic accidents).
According to this search procedure, a total of 46,987 news items were
collected for the period 2010-2015 alone. This volume of data, due to
its large size, was considered unfeasible to handle in an individual
study, so it was decided to opt for a constructed week sampling \cite{hester2007efficiency}, a scientific methodology commonly used in the field
of Communication used, for example, in the work of Valenzuela, \cite{valenzuela2017behavioral}.

For the construction of these weeks, the random number generator Random
Integer Set Generator has been used in this study:
(\url{https://www.random.org/integer-sets/}). For this, 7 sets per year were
requested, of 10 unique random integers in each one, taken from the
range [1, 52]. The integers in each set were sorted ascending in the
following order: Monday (Set 1), Tuesday (Set 2), Wednesday (Set 3),
Thursday (Set 4), Friday (Set 5), Saturday (Set 6) and Sunday (Set 7).
In each set the number of the week of the year was indicated, from 1 to
52.

In short, 70 days per year (770 days in total) were chosen, with the aim
of achieving a logical and adequate sample size for this study. In this
way, it was possible to reduce the volume of news to be analyzed while
maintaining the representativeness of the sample.

Once the sample was obtained, a content analysis was carried out using
different variables according to some blocks of categories, considering
aspects of location, sources, content, and others. In this way, it has
also been possible to select the exact news headlines for each of the
causes of events studied.

For the in-depth interviews, four people were selected according to
criteria of relevance in the exercise of the profession and
institutional authority within the specific field of communication in
Navarra.

In the case of professional media, both directors of traditional media
and digital natives have been combined, also considering the point of
view of a journalist from the events section. The association body for
journalists in the \emph{Comunidad Foral de Navarra} is integrated into
the Association of Journalists of Navarre (AJN), so it is considered
appropriate to obtain information from the president of the AJN.
According to these criteria, the people selected have been the
following:


\begin{itemize}
	\item D. Patxi Pérez Fernández (president of the Association of Journalists
of Navarre).
\item Mr. Jesús Morales (responsible for the Events Section of \emph{Diario
	de Noticias}).
\item Mr. Gabriel González Ortiz (journalist in the Events and Courts
Section of \emph{Diario de Navarra}).
\item D. Ignacio Murillo (director of \url{Navarra.com}, \emph{Glocal Influence,
	S.L}. He has also been editor of the Events, Society and Courts section
in \emph{Diario de Navarra}).
\end{itemize}

For the qualitative analysis, a discussion group (focus group) has been
created, made up of a heterogeneous group of eight people, with the aim
of collecting existing visions in society in general. It is considered
that a lower number would make it difficult to contrast opinions and a
higher it could suppose a fragmentation of the topics to be dealt with
\cite{krueger2002designing}. A double criterion has been followed that
contemplates people involved with the reality represented and people who
represent a public, without involvement in the subject of study. Efforts
have also been made that the discussion group is made up of people of
different socioeconomic levels, that there is gender parity, different
levels of training and a variety of ages. Individuals selected based on
these criteria have given their consent to include their names. However,
to protect them as much as possible, it has been decided to introduce a
code that indicates their sex and age:

\begin{itemize}
	\item F58 (female sex, married, 58 years old, president of the Besarkada --
Abrazo Association).
\item M68 (male sex, separated, 68 years old, poet, former vice president of
the Ateneo Navarro, currently retired).
\item F18 (female sex, single, 18 years old, high school student).
\item M18 (male sex, single, 18 years old, Engineering Degree student).
\item F57 (female sex, single, 57 years old, anesthesia nurse at Clínica
Universidad de Navarra).
\item M58 (male sex, married, 58 years old, forensic doctor and director of
the Forensic Clinic Service of the Navarre Institute of Legal Medicine).
\item F44 (female sex, married, 44 years old, assistant at the emergency
call center and housewife).
\item M70 (male sex, married, 70 years old, PhD in Information Sciences,
currently retired).
\end{itemize}

The structured interview is made up of three blocks: Practices and
routines of professionals, Audience, and Guidelines and ethical
criteria.

The discussion group script is structured in four parts: social
perception of accidents and suicides, social reaction to this type of
news, role of the media in disseminating news about events, and future
challenges.
