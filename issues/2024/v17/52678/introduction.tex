\section{Introduction}\label{sec-introduction}

Audiovisual translation (AVT) has emerged as an essential tool for
enhancing the accessibility of multimedia content to diverse audiences
across languages \cite{díaz-cintas2023}. Subtitling, a key
component of AVT, involves displaying spoken dialogues and sounds on
screen using synchronized written text snippets \cite{fernández-costales2023}. While conventional subtitling presents the translation in
the viewer\textquotesingle s native language, reversed subtitling offers
a less common approach by displaying foreign language subtitles
alongside the original audio in the viewer\textquotesingle s native
language \cite{bolaños2023}. For language
learners, especially those in the early stages of acquisition, watching
subtitled movies can support vocabulary development by coordinating
reading and listening. However, managing multiple information streams
simultaneously may tax limited cognitive resources, potentially
hindering comprehension, particularly for novice learners \cite{hornerocorisco2023}.

This study introduces a novel approach to investigating subtitle
processing by employing a web-based eye-tracking system, allowing for
remote data collection and expanding the potential participant pool
beyond traditional laboratory settings. The present study employs
eye-tracking measures and comprehension quizzes to investigate
linguistic processing during subtitle reading and comprehension in two
distinct languages: Arabic and Spanish. Unlike previous research, this
study uniquely examines the impact of L1 (English) audio, L2
(Spanish/Arabic) audio, and no audio on subtitle reading and
comprehension in novice learners of Spanish and Arabic, providing
insights into the role of audio in subtitle processing across different
language pairs. Specifically, vocabulary recognition assessments are
used to measure learning outcomes after the viewing of the subtitled
videos.


  The study\textquotesingle s within-subjects design allows for a more
  rigorous comparison of the different audio conditions, controlling for
  individual differences and enhancing the reliability of the findings.
  The researchers aim to shed light on the factors influencing subtitle
  reading strategies and the processing of various linguistic units in
  novice bilingual learners by comparing attentional coordination for
  vocabulary encoding across languages and cognitive manipulations. To
  achieve a comprehensive understanding of subtitle processing, the
  study conducts both global and local analyses of eye movements. This
  dual-level analysis approach, combining global subtitle reading
  patterns with local linguistic unit processing, represents a
  methodological advancement in the field of AVT research.

  The global analysis examines reading patterns across entire subtitles,
  considering measures such as average fixation duration, total number
  of fixations, saccade length, and percentage of skipped subtitles.
  These measures offer insights into general reading approaches and the
  extent of reliance on subtitles in the absence of L1 audio support.
  Complementing the global analysis, the local analysis focuses on
  specific linguistic units embedded in the subtitles, such as verbs,
  nouns, adjectives, adverbs, expressions, phrases, sentences, and
  questions. Assessing first fixation duration, gaze duration, total
  reading time, and the probability of refixations or regressions on
  these linguistic units allows for the investigation of the processing
  and integration efficiency of various language components.

  A key innovation of this study is the integration of eye-tracking
  measures with comprehension outcomes, providing a more
  empirically-based understanding of the relationship between attention
  allocation and learning in subtitle-supported language acquisition.
  The presence or absence of L1 or L2 audio is expected to
  differentially impact both global subtitle reading fluency and the
  processing of these linguistic units, which are crucial for language
  acquisition. Fixation and regression count on target linguistic units
  serve as indicators of overall integration success \cite{holmqvist_eye_2011}.

A primary objective of the study is to determine whether the presence or
absence of L1 or L2 audio leads to greater dependence on subtitles for
attentional allocation and the processing of linguistic units. There are
still many who believe that audio-visual outlets for learning such as
film or music don't set a good stage for learning, critisizing
audio-visual outlets as being more distracting than traditional lecture
style teaching. For instance, \textcite[p. 61]{borrás1994} indicated that:

\begin{quote}
opponents of the use of subtitled video in foreign/second language
teaching argue that the presence of subtitles is distracting and that
they slow down the development of learners\textquotesingle{} listening
abilities. Proponents of subtitles, on the other hand, contend that
subtitles may help develop language proficiency by enabling learners to
be conscious of language that they might -not otherwise understand.
\end{quote}

This research hopes to install more high beams of evidence to lengthen
the bridge between language learning and audio-visual stimulation. The
research also explores potential differences in the impact of audio
presence on the processing of various linguistic units for more
transparent Spanish compared to more opaque Arabic. Fixation and
regression counts provide validation of overall integration success. In
addition to the primary objective, the study investigates the links
between subtitle reading fluency metrics, as quantified by eye tracking,
and vocabulary learning outcomes, measured through recognition tests
administered after the viewing of subtitled videos. Combining these
measures allows the study to capture the multiliteracy impacts of audio
manipulations on language acquisition. The research apparatus involves
using RealEye, a webcam eye-tracking software, during authentic
subtitled video viewing, paired with L2 vocabulary recognition
assessments, aligning outcomes with specific research questions, and
elucidating the contributions of language proximity and audio support to
literacy development in novice learners.

The analysis of quantitative gaze metrics and comprehension scores
reveals how language proximity and cognitive load influence attentional
coordination efficiency, the processing of linguistic units, and overall
understanding. These findings underscore the importance of auditory
factors and language proficiency in subtitle design and delivery. To
facilitate engagement and comprehension, content creators and educators
should use native-language audio while supporting viewers with varying
language expertise. The insights gained from this research have
significant implications for language instruction and educational
material design used throughout the language education field. In
addition, the personal post survey is valuable to scholars of
translation, of which, the cognition focused researchers are
experiencing a time of flourishing of research interest and advancement.
The results suggest tailoring instructional strategies to the specific
needs and characteristics of the language being learned. For Arabic
learners, using L2 subtitles without audio is highly effective,
capitalizing on their ability to process visual information effectively.
In contrast, Spanish learners may benefit from a balanced approach
incorporating both L2 audio and subtitles, leveraging the linguistic
alignment between the two modalities. These insights highlight the
importance of considering each language\textquotesingle s unique
features and demands when developing instructional plans and resources.

The findings of this study have the potential to enrich theoretical
perspectives on the coordination of input streams during complex
literacy tasks and to inform the development of pedagogical guidelines
that leverage multimedia reversed subtitling to support novice language
growth, in line with multimedia learning principles. This research
contributes to the growing body of knowledge in the AVT field and
language acquisition, as well as translation studies by shedding light
on the factors influencing subtitle reading strategies, the processing
of linguistic units, and comprehension in novice bilingual learners.