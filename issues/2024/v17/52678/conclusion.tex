\section{Conclusion}\label{sec-conclusion}

For many years, subtitles were not utilized as a tool to either increase
language proficiency in language learners or were they the focus of
translation research. Now that there is a plethora of technological aids
to use for subtitle gaze analysis, it has opened many new possibilities
for prodigious findings. The rise of streaming services seems to have
eclipsed standard cable or television programming, but for language
accessibility and language exposure, this revolution could not have been
better. Streaming giants such as Netflix and Disney are localizing
through various audio-visual approaches such as subtitling, which are
certainly being utilized by the multicultural and multilingual
viewership at large. Furthermore, language learners and enthusiasts are
incorporating the multitude of language pair settings in their approach
to new language acquisition, which is why the research presented in this
paper is pertinent.

Like the parameters that were set in the experiments, subscribers can
mix and match differing language pair combinations that exclude the
source language altogether (i.e. watching \emph{Squid Game} (2021),
originally in Korean, with English-dubbed audio with Japanese
subtitles). This means that, while of course the source language will
always remain an important element at play when considering the central
cultural context of the show, the dialogue and plot can be
linguistically transformed in unforeseen ways. Though the results of
this experiment conclusively showed that variance supersedes supposition
when it comes to the eye-ear relationship, there were still positive
findings pertaining to the efficacy of the audio-visual educational
medium. Furthermore, language educators and curriculum creators do not
need to set a specific threshold for this type of medium to be
introduced into the curriculum. Many of the participants of the
experiment had just barely an elementary understanding of the Spanish
language, yet some of their comprehension scores did positively
fluctuate with the mixture of the conditions. More experiments that
follow the methodology used for this experiment would strengthen the
findings and would also uncover more efficacy differences amongst
differing language pair combinations. Future research should investigate
the role of language proficiency in subtitle reading and develop
evidence-based practices to improve the accessibility and effectiveness
of subtitled media, ultimately optimizing subtitle design and user
experience across diverse auditory contexts and languages.
