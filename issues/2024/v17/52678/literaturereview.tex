\section{Literature Review}\label{sec-literaturereview}

Enhancing students' reading comprehension skills has emerged as a
significant concern in both academic and non-academic settings \cite{kim2023}. Translation plays a crucial role in enhancing reading
comprehension, as noted by \textcite{alaboud2022}, who found that "translation
could be an effective instructional strategy in improving
learners\textquotesingle{} skills in reading comprehension in an EFL
setting" \textcite[p. 424]{alaboud2022}. In a case study of medical students,
\textcite{rushwan2017} indicated that the use of translation can facilitate and
enhance reading comprehension skills of ESP medical learning. Concerning
the utilization of subtitles to improve reading skills, several studies
\cite{omar2023,qazi2023} have endorsed the notion that
audio-visual materials enriched with subtitles manifest to enhance both
second language (L2) reading comprehension abilities. For instance,
\textcite{haider2022} indicate that the effects of viewing
captioned videos on EFL learners revealed a positive impact on reading
comprehension. \textcite{xu2022} also state that utilizing videos serves
as a potent and engaging educational resource for EFL learners. In their
study, Xu et al. provide further insights into the impact of viewing
captioned videos on the listening and reading skills of university EFL
learners. While the reading of subtitles has garnered increasing
attention in recent decades, our comprehension of the cognitive
processes involved in subtitle reading, and the variances from reading
static text, remains constrained \cite{baranowska2020}. Research in
subtitle reading has been significantly influenced by the intersection
of auditory input and visual processing. The relationship between what
viewers hear and what they read on screen presents intriguing questions
about cognitive workload and language comprehension.

To take in new information, our eyes must adjust their focal direction
to where it is most concerned next, ``in order to process information
most effectively we must move our eyes so that the fovea fixates the
location of that which we intend to process'' \cite[p. 84]{schotter2012}. One cannot presume that eye engagement behavior remains
static throughout the language learning endeavor \cite{blythe2005}.
\textcite[p. 86]{schotter2012} emphasize that ``Interestingly, the
amount of disparity {[}between individual eye focus{]} tends to be
greater in beginning readers than skilled readers''. Thus, our research
seeks to tackle this trend early in the language learning process for
second language learners. The pace at which they can become more fluent
may differ and be faster than the pace they had when they learned their
first language; however, such learners would be susceptible to
individual eye focus disparity. Subtitles presentation through the
purposeful utilization of colors, size, spacing, font and timing could
have a varying degree of impact when considering these factors. In
addition, video and subtitle speed is also an impactful factor upon the
viewer. According to \textcite{liao2020}, as subtitle speed increases,
word frequency and word length effects become less pronounced in local
eye movement measures marking lexical processing. The authors conclude
that increasing subtitle speed results in a shift "from local
(cognitive) eye-movement control towards heuristics informed by global
task constraints (e.g., subtitle speed)" \textcite[p. 430]{liao2020}.
This suggests that attention allocation during subtitle reading relies
less on characteristics of individual words when subtitles are presented
more rapidly.

Studies by \textcite{ross2023} and \textcite{szarkowska2018}
highlight a trend where the presence of audio, especially in a viewer's
native language, leads to reduced reliance on subtitles. This phenomenon
suggests that audio can support and enhance the subtitle reading
experience, particularly when it is semantically aligned with the text.
Another key consideration to be aware of is that the overall ``reading
process could be affected by semantically relevant auditory input in the
context of reading English/L2 subtitles in video'' \cite[p. 259]{liao2020}.

\textcite{liao2022} present evidence that in situations involving
multimodal reading, like reading subtitles in videos, eye movements are
not solely regulated by visual information. Rather, readers
simultaneously incorporate inputs from both visual and auditory
modalities in the moment to determine when, where, and even whether to
shift their eyes for subtitle reading. Thus, \textcite{liao2022}
contribute to this discourse by examining how auditory input, even when
partially redundant, influences the comprehension of subtitled content.
Their study moves beyond the mere tracking of eye movements to consider
how the audio-visual synergy affects viewers\textquotesingle{}
higher-level understanding of content.

A gap still exists in understanding the specific dynamics of how
auditory context facilitates subtitle reading. \textcite{ragni2020} partially
addressed this by investigating second-language subtitles with native
language audio. Yet, the absence of a no-audio condition in
Ragni\textquotesingle s research limits the ability to discern the
exclusive effects of auditory input on subtitle processing.

Given the widespread use of subtitles in second language education,
understanding how audio influences subtitle comprehension is critically
important. A range of prior research, as summarized by \textcite{liao2022}, has studied the impacts of native and foreign language audio on
global subtitle reading approaches using eye-tracking measures with
languages like Dutch, Swedish, and English. Other work also revealed the
effects of varied audio backing on patterns indicative of reading
strategies \cite{szarkowska2018}. Additionally, recent
studies measured local lexical and perceptual processing through
fixation metrics \cite{bisson2014,ragni2020}.

\textcite{bravo2008} extensively examined audiovisual-based language
learning. In her work, Bravo emphasizes that subtitles should not be
viewed as a panacea for foreign language acquisition. Despite the
wealth of research on how audiovisual configurations affect second
language learning, this topic remains relevant to related disciplines,
particularly linguistics and translation studies, which in turn
influence second language acquisition. Historically, translation has
been employed in second language instruction by formal educational
institutions. Furthermore, individual language learners have used
translation as a method to test and improve their comprehension of the
target language.

The current study builds upon these precedents by isolating the impacts
of native English audio, foreign Spanish/Arabic audio, and no audio
conditions during Spanish and Arabic subtitle reading specifically for
novice learners. The investigation goes beyond prior global analyses to
align macro reading strategies with granular dynamic time courses of
vocabulary integration efficiency. Through coordinated examination of
sentence-level and localized word-level processing, the work
comprehensively evaluates the contributions of audio presence and
language proximity factors to literacy development gains.

While \textcite{borrás1994} found that the opportunity of having
subtitles has a positive impact on comprehension, they also claim that
it improves the productive use of the foreign language. A factor that is
not underscored within this particular study, though it is of great
consequence upon second language acquisition, is individual writing
skill. In writing, phrasal usage is promoted as being a strategy that
enhances the quality of a text. This is not only true for English, but
also stressed in the Arabic writing system \cite{anis2022}. The
videos used in this experiment certainly had phrases, and the
participants may or may not have caught on to the phrases. \textcite{anis2022} found that the translations of Arabic literature (primarily
poetry, in their example) can be done more naturally if the Arabic
source text has phrases. Perhaps this finding then would also support a
notion that suggests that phrasal usage in oral speech in videos could
potentially be noticed by the listeners even if the language is their L2
language.

This eye-tracking study explores the impacts of native English language
(L1) versus foreign Spanish or Arabic language (L2) auditory input, or
its absence, on Spanish or Arabic subtitle reading and comprehension.
The research utilizes a 2x3 within-subjects design with native
English-speaking learners, manipulating audio condition (L1 English
audio present, L2 Spanish or Arabic audio present, or no audio) and
evaluating the effects on Spanish or Arabic subtitle processing.

The study aims to answer three questions:

1. How do L1 English audio, L2 Spanish or Arabic audio, and no audio
conditions influence global subtitle reading efficiency patterns and
local lexical processing in novice learners?

2. What is the relationship between L1 English or L2 Spanish or Arabic
auditory input and comprehension accuracy with Spanish or Arabic
subtitled foreign language content?

3. Does the presence or absence of L1 English versus L2 Spanish or
Arabic audio impact perceived cognitive load during Spanish or Arabic
subtitle reading?