\begin{polyabstract}
\begin{abstract}
O objetivo deste artigo é refletir sobre os discursos que circulam sobre o livro e a leitura e, especificamente, analisar sob a ótica da Análise de Discurso francesa vídeos de \textit{booktubers} com apresentações de estantes dos comentadores em uma prática chamada por eles de \textit{bookshelf tour}. Destaca-se os sentidos de mostrar que se é um leitor, de capital social para aquele que lê e de socialização das práticas leitoras por meio da fluidez e da dinamicidade da web e suas possíveis correlações com o ensino de leitura e escrita no Brasil. Questiona-se, com a regularidade de tais práticas, novas pluralidades para os significados da leitura, dando lugar a formatos audiovisuais e formas variadas de falar sobre livros e leituras. O conceito fundamental da Análise do Discurso para este estudo é o conceito de sujeito, na perspectiva foucaultiana. 

\keywords{Discurso \sep Leitura \sep Sujeito \sep \textit{Booktubers}}
\end{abstract}

\begin{abstract}
The objective of this article is to reflect on the discourses that circulate about books and reading and, specifically, to analyze, from the perspective of French Discourse Analysis, videos of booktubers with presentations of commentators' shelves in a practice they call bookshelf tour. The meanings of showing that one is a reader, of social capital for those who read and of socialization of reading practices through the fluidity and dynamism of the web and its possible correlations with the teaching of reading and writing in Brazil stand out. With the regularity of such practices, new pluralities for the meanings of reading are questioned, giving way to audiovisual formats and varied ways of talking about books and reading. The fundamental concept of Discourse Analysis for this study is the concept of subject, from the Foucauldian perspective.

\keywords{Discourse \sep Reading \sep Subject \sep Booktubers}
\end{abstract}
\end{polyabstract}
