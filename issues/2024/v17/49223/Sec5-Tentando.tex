\section{Tentando fechar ideias: uma conclusão inacabada}\label{sec-tentandofecharideias}

Exibir as estantes é uma forma de marcar-se como leitor e que, para o
espectador que assiste a estes vídeos, o sentido é escasso, foge,
escapa, e o que ele muitas vezes busca são indicações de leitura. Alguns
vídeos atentam para isso: destacar livros, comentá-los que estão na
estante e que podem ser uma leitura interessante para aquele que
assiste. Outros vídeos, porém, mostram o material que é feito a estante,
a montagem, práticas que não interferem de forma alguma na aprendizagem
da leitura e no desenvolvimento das habilidades leitoras. Se antes os
professores de ensino de leitura e escrita sofriam quando o aluno optava
por ler o resumo do livro em vez da obra em si, hoje muitos jovens
preferem \enquote{assistir} aos (comentários dos) livros a lê-los.

Os discursos analisados mostram a regularidade de uma prática que
intercala enunciados interdiscursivos, ou seja, enunciados que se
relacionam entre si discursivamente, retomando discursos que se
entrelaçam na formação da intimidade e da ostentação como subjetividades
enunciativas das leituras do eu. Dessa forma, apresenta-se um sujeito
marcando-se a si no discurso, pela circulação e veiculação, numa prática
do mostrar-se que é um leitor, muitos mais do que se é. O fato de ser
realmente ou não um leitor se perde no espaço e no tempo. Não há como
comprovar. O que se indicia, todavia, é que o mostrar antecede o ler e
que os formatos audiovisuais, outras formatações de obras em vídeos, em
cores, imagens, sons repercutem como sentidos estabilizados, congelados
de que o leitor só assim o será se mostrar a si como leitor na grande
rede.

A leitura está em todos os lugares, mesmo naqueles lugares colocados à
margem. Alguns lugares ora tidos como leituras à margem como é o caso
dos canais literários no YouTube, outras vezes se organizam entre si e
formam sociedades fechadas de leitura que não promovem, nem estimulam a
leitura, apenas mantêm a ordem e os sentidos desta prática. Toda e
qualquer leitura, contudo, produz sentido e deve ser analisada dentro de
uma história discursiva da leitura, porque a mídia e a grande massa
muitas vezes censuram e negam essas leituras. Verificar não se isso é
bom ou ruim, mas traduzir interpretações em condições próprias de
enunciados sujeitos imbricados na cultura de si, nos faz aproximar dos
mecanismos do saber, do poder e das tecnologias de si pensando numa
sociedade da resistência, como enfatizava Foucault.
