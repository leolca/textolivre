\section{\enquote{Passeando pela nova estante}: novas configurações da leitura e representações do leitor contemporâneo na cultura digital}\label{Sec-passeando}

Falar de livros, falar de filmes, falar dos mais variados objetos culturais não é prática nova. Contudo, com o advento das redes sociais e, mais precisamente, da plataforma audiovisual YouTube, comentar um livro tem se tornado prática rotineira e realizada, ao contrário da crítica literária profissional, por pessoas amadoras, ou estudantes do universo das Letras, denominadas \textit{booktubers}. Os sujeitos adeptos dessa prática, geralmente, são pessoas jovens que comentam livros em diversas ações. Marcam uma certa regularidade em não só resenhar os livros, mas apresentar estantes, leituras coletivas, compras de livros, inventários, entre tantas outras.

Observando diversos canais que têm por objetivo comentar livros por meio de resenhas, percebeu-se entre eles uma prática comum, que é chamada de \textit{bookshelf tour}, ou seja, um passeio pela estante onde o apresentador guarda seus livros. Além dessa, outras práticas surgem no universo dos \textit{booktubers} como, por exemplo, a prática do \textit{bookhaul}, que consiste em comentar os livros recebidos de editoras que os enviam para terem suas obras comentadas pelos resenhistas amadores. 

Neste artigo, pretende-se mostrar em análise de discurso, a partir das contribuições de Michel Foucault a este campo, algumas problematizações acerca dessa prática em dois vídeos que, antes de serem um \textit{bookshelf tour}, mostram a montagem da estante como um suporte, algo que questionamos se é da ordem da intimidade ou da ostentação, ou de ambas, e, ainda, sobre quais sentidos faz circular a respeito da leitura e dos livros na contemporaneidade.

Para a fundamentação teórica da análise, adotamos o conceito de sujeito em  \textcite{foucault2009arqueologia}, perpassado pelos mecanismos de saber e poder. Trazemos também a reflexão foucaultiana de escrita de si, para pensarmos na demonstração das estantes como uma “leitura de si”. Também é preciso considerar a história do livro e da leitura em \textcite{chartier1998aventura,chartier1999ordem,chartier2019lersem} e outros estudiosos da história cultural. 

O \textit{corpus} geral do estudo é composto por sete canais literários no
YouTube e, para este artigo, foram recortados dois canais, sem deixarmos
de fazermos uma apresentação geral das práticas que se repetem em
regularidades enunciativas \cite{foucault2009arqueologia}.

Baseados em outro estudo, \textcite{vizibeli_contrastes_2016}, partimos do pressuposto de que a crítica promovida pelos \textit{booktubers} guarda características próprias que se distinguem da crítica literária especializada. Avançando nesse e em outros estudos referentes à temática, percebemos que os leitores denominados como \textit{booktubers} são colocados à margem e que apresentar a estante é, ao mesmo tempo, que uma forma de gerar intimidade, também uma demonstração de quem são os leitores hoje, parafraseando a indagação foucaultiana de “Quem somos nós hoje?”.