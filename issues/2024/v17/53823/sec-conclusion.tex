\section{Conclusión}\label{sec-conclusión}

Las redes comunicativas sNOOC constituidas desde la formación de
posgrado y formadas por estudiantes que pasan a ser \emph{e-teacher} son
un ejemplo más de esfuerzo por la educación mediática inclusiva, en este
caso concreto, de la tercera edad. Gracias a plataformas como la de la
UNED, tmooc.es y el modelo sNOOC, se ha conseguido incentivar un modelo
formativo que pone de manifiesto un acceso equitativo y una
participación en la construcción colectiva del conocimiento. Esta
propuesta innovadora ha ayudado a consolidar un enfoque educativo y
comunicativo inclusivo adaptado a sectores vulnerables, a pesar de las
limitaciones del estudio como el tamaño de la muestra y la concreción de
los resultados en un contexto universitario determinado. La valoración
positiva de la experiencia formativa que hemos prestado en este artículo
no sólo destaca por su planteamiento didáctico o por la interacción de
las redes comunicativas creadas, sino que subraya el papel central de la
alfabetización mediática en la inclusión social de personas de la
tercera edad.

Es clave, como perspectiva futura, seguir fomentando la creación y
posterior investigación de redes comunicativas sNOOC, que asienten sus
proyectos formativos en acciones colaborativas y solidarias, potenciando
el uso de la inteligencia artificial, la gamificación y los entornos
inmersivos integrados en una pedagogía inclusiva. En un momento clave
para la formación a distancia, el reto de los agentes educativos y
sociales que se unen en red será mantener y mejorar la calidad del
modelo comunicativo y pedagógico, asegurando unas plataformas de
calidad, accesibles, adaptables y centradas en una comunicación
horizontal y bidireccional.