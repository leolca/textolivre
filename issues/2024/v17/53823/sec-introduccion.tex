% !TeX root = main.tex

\section{Introducción}\label{sec-introducción}

La educación mediática es un proceso pedagógico y comunicativo que busca
que la ciudadanía desarrolle habilidades críticas que permitan analizar
el universo de medios que se proyectan en la sociedad postdigital
\cite{Escaño_2023}. Esta perspectiva convierte este ámbito de la
alfabetización en un trabajo por la inclusión de todos los sectores de
la sociedad, especialmente los más vulnerables que, por unos u otros
motivos, no pueden acceder a determinada información o son más
susceptibles a los peligros de esta. La educación mediática apuesta, por
tanto, por la inclusión, al buscar la promoción de la equidad, la
diversidad y la participación en el proceso educativo cite{guillen2024}. La integración de los principios que
fundamentan la inclusión en un planteamiento mediático es fundamental
para conseguir que la ciudadanía desarrolle las competencias digitales y
mediáticas imprescindibles para poder sobrevivir a la brecha de la
sociedad posdigital, potenciando su pensamiento crítico y, en
consecuencia, su actitud crítica \cite{palacios-rodríguez2025}. En este contexto hipermediatizado, la educación mediática
proporciona oportunidades de aprendizaje a través de \emph{mass media} y
\emph{social media} de manera equitativa y accesible, incluidas aquellas
personas que, debido a su edad, han sido excluidas o marginadas del
proceso de digitalización.

La Educación Abierta y a Distancia (EAD) ha jugado un papel importante
en el ámbito de la educación mediática y, como consecuencia, en la
creación de redes comunicativas globales al posibilitar la conexión de
estudiantes de distintas ubicaciones geográficas y contextos culturales.
La UNED desempeña un papel destacado desde 1972 en España promocionando
y desarrollando este modelo formativo y el acceso a la educación
superior de calidad a un amplio abanico de personas. Más
específicamente, en el ámbito de la EAD, la UNED ha colaborado
activamente por la incursión de la educación mediática, a través de los
\emph{Massive Open Online Courses} (MOOC), ofreciendo éstos a través de
su plataforma Uned Abierta, o de otras conocidas como \emph{Coursera},
\emph{edX}, \emph{Miriadax}, \emph{EcoDigitalLearning,} tmooc.es, etc.,
contribuyendo así al intercambio de conocimiento y la
internacionalización de la universidad. Este modelo formativo, diseñado
desde hace más de quince años (Ratnasari; Chou; Huang, 2024), puede
colaborar en la promoción de la inclusión social, al ofrecer un acceso
global, flexibilidad, costos reducidos, adaptabilidad, diversidad de
temáticas, interacción, participación y construcción colectiva del
conocimiento. Se promueve así un aprendizaje a lo largo de la vida
buscando la convergencia mediática: vídeos, lecturas, cuestionarios,
foros, recursos gamificados, creaciones con IA \cite{Aparicio-Gómez2024,cárdeasbenavides2024} e incluso
entornos inmersivos incluyendo los metaversos \cite{huesoromero2024,chuchuca2024metaverso}.

En el presente estudio hemos optado, dentro de las diferentes tipologías
de los MOOC, por el modelo sNOOC \cite{quintana2024snooc}. Tradicionalmente,
se han presentado los SPOC, xMOOC, cMOOC, sMOOC, tMOOC, COMOOC, pero,
actualmente también nos encontramos con los \emph{Nano Open Online
Courses} - NOOC \cite{clark2013moocs,gomez2016,Osuna-Acedo2017,LopezdelaSerna_Garrido_2018,Escaño_Dewhurst_2024}, un planteamiento
concreto y personalizado que estructura su duración en horas y articula
su estructura en torno a un contenido, herramienta o habilidad concreta
\cite{MANANGÓN-CABRERA2023,huesoromero2024}; con una duración máxima de veinte
horas \cite{intef2016}. Una formación en módulos comunicativos e
interactivos más minimalista que se presenta como novedosa, líquida y
flexible de contenidos, de recursos, de espacio y de tiempo \cite{basantes2020}. Perfilando aún más
nuestra elección, hemos apostado en el estudio por los sNOOC o
\emph{social}NOOC que posibilitan más aún el empoderamiento social de
redes comunicativas de estudiantes gracias a la creación colectiva del
conocimiento, la base de la cultura participativa y la apuesta por la
fidelización-compromiso del alumnado convertido en \emph{e-teacher}.
Esta implementación de los sNOOC se presenta como herramienta de
evaluación continua en la EAD, centrándose en experiencias y
valoraciones del alumnado, los elementos comunicativos y pedagógicos,
así como el proceso de construcción colaborativa de los contenidos,
poniendo énfasis en pedagogías inclusivas, que se enriquecen de las
metodologías activas, la creación con IA y metaverso \cite{galíndez2024}. Estas prácticas pedagógicas inclusivas, desde un diseño universal
de aprendizaje \cite{SanchezFuentes2023}, valoran la diversidad de
habilidades, intereses, experiencias y estilos de aprendizaje,
garantizan la equidad, aseguran la adaptabilidad utilizando estrategias
flexibles adaptadas a las diversas capacidades, promueven la
participación, fomentan la colaboración de la comunidad virtual y la
creación de entornos seguros, acogedores y estimulantes.

En este artículo presentamos el análisis de una experiencia que parte de
la creación de redes comunicativas en estudiantes de posgrado de la
Universidad Nacional de Educación a Distancia (UNED) con la finalidad de
convertirlos en \emph{e-teacher,} implementando proyectos formativos
para que las personas de la tercera edad, en riesgo de exclusión
digital, adquieran competencias mediáticas. La creación de esta red
comunicativa sólida parte de un espacio académico de la asignatura de
``Escenarios virtuales para la participación'' y se proyecta como
vinculación y transferencia a la sociedad. Los recursos comunicativos
cobran protagonismo en estas redes, no sólo a través de la plataforma
del curso concreto, sino también en la creación con el uso de la
Inteligencia Artificial (IA) o de espacios inmersivos de una plataforma
sNOOC, donde no sólo las personas participantes puedan formarse, sino
también puedan conectarse, discutir temas relevantes, compartir recursos
o difundir por el \emph{software} social ideas originales. Este proyecto
formativo lleva a la potencialización de la mentoría entre iguales,
unidos en red de intereses y en diversidad de propósitos, donde se
establecen relaciones más cercanas entre el alumnado y el aprendizaje
colaborativo.

La formación mediática de personas de un sector, que se puede considerar
a nivel mediático en peligro de exclusión, como es el de la tercera
edad, es un reto que ya han iniciado otros agentes educativos y sociales
desde diferentes instituciones \cite{abad-acala2014,abad-acala2017,Leal-Maridueña2017,heredia-sánchez2023}. En este caso
concreto, se ha visto beneficiada gracias a la solidaridad y
empoderamiento de redes de estudiantes unedianos \cite{Swan2015,Reich2015}, y el
compromiso por la formación mediática, promoviendo un mejor sentido de
la responsabilidad cívica \cite{bringle1996}, convirtiéndose en un
planteamiento pionero en esta etapa de formación. En esta experiencia,
el compromiso se materializa como un puente tangible que vincula la
formación a través de sNOOC con las problemáticas sociales como es la
alfabetización mediática de las personas de la tercera edad, pocas veces
productoras o creadoras de contenidos y muchas más consumidoras pasivas
de las redes sociales, en comparación con generaciones más jóvenes,
sobre todo en plataformas como Instagram, TikTok, YouTube, X o Facebook.
Los adultos mayores suelen comunicarse a través de estas interfaces con
amistades y familiares, comparten recuerdos, siguen noticias o temas
interesantes, e incluso, las y los más valientes, se animan a formar
parte de comunidades. Las plataformas, conocedoras de esta situación,
están adaptando sus interfaces y servicios para ser más accesibles y
fáciles de usar para personas de estas edades. Partiendo de esta
realidad inequitativa se puede y se debe impulsar, a través de proyectos
formativos de alfabetización mediática como el que presentamos en esta
investigación, una mejora de las habilidades, competencias y capacidades
de este grupo demográfico, especialmente afectado por la brecha digital
generacional.
