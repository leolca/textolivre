% !TeX root = main.tex
\section{Discusión}\label{sec-discusión}

La educación inclusiva y la formación en mediática están relacionadas,
al promover la equidad, la diversidad y la participación de sectores en
riesgo de exclusión apostando por una ciudadanía más crítica \cite{Xie2021}. En una sociedad posdigital como la nuestra, es fundamental
integrar estos principios inclusivos en la educación mediática para que
la ciudadanía, especialmente las personas de la tercera edad desarrollen
competencias necesarias para superar la brecha digital en la que, la
mayoría de las ocasiones, se encuentran inmersos. El acceso, la equidad
y la participación son características propias de este proyecto de
creación de redes comunicativas para la alfabetización mediática de las
personas de la tercera edad, un reto formativo aún presente en la
sociedad postdigital \cite{hernando2013}. El acceso a la
plataforma de EAD (UNED) se ha desarrollado de forma potencial, tanto a
nivel de tiempo como de número de sesiones, hecho que ha posibilitado
unas interacciones y una participación activas. La valoración positiva
de la asignatura ``Escenarios Virtuales para la Participación'' nos hace
ver la importancia de un buen planteamiento didáctico de empoderamiento
del alumnado con el fin de conseguir una mayor equidad basada en una
adecuación, coherencia y aumento del grado de satisfacción global por el
aprendizaje adquirido; el esfuerzo por conseguir un trato igualitario,
acceso a recursos, apoyo, además de un mayor desempeño y evaluación
eficaz, ha sido corroborado por las personas participantes como
conseguido.

La creación de redes comunicativas favorece el entusiasmo de los
\emph{e-teachers} y sobre todo la ilusión por proyectar, más allá de los
muros del aula, la inclusión educativa en base a la alfabetización
mediática de sectores vulnerables como la tercera edad tanto en
contextos formativos formales como no formales. A través de distintas
herramientas comunicativas como los foros, que han tenido una alta
frecuencia, los pasos a seguir para la creación de la red han pasado por
la búsqueda y formación de grupos interdisciplinares, la presentación
personal, la diversidad de participantes, los intereses comunes, la
inclusión, colaboración y el apoyo mutuo. Aspectos todos ellos
fundamentales para formar grupos colaborativos y solidarios para el
proyecto formativo.

La plataforma tmooc.es se ha presentado una vez más como un espacio de
aprendizaje servicio que nos ayuda a generar proyectos formativos sNOOC
\cite{huesoromero2024}, dentro del modelo
NOOC \cite{clark2013moocs,cabrera2021}, para los
diferentes sectores de la ciudadanía. Gracias a la accesibilidad de
Chamilo y de otros espacios para generar recursos comunicativos, se ha
garantizado el diseño universal de aprendizaje \cite{SanchezFuentes2023};
el enriquecimiento de los recursos comunicativos, la planificación y el
itinerario pedagógico, han convertido a los sNOOC en espacios accesibles
y equitativos para una formación específica \cite{cabrera2021} basada en una comunicación horizontal y bidireccional.
Estos y otros aspectos han sido valorandos por el juicio de expertos
internacionales desde la perspectiva de las pedagogías inclusivas, cuyos
principios se fundamentan en la valoración de la diversidad, la equidad,
la adaptabilidad, la colaboración, la participación y los ambientes de
apoyo. Este itinerario sNOOC ayudado por la capacidad creadora atractiva
\cite{watters2022}, formado por una convergencia de recursos de creaciones
con IA, juegos, espacios inmersivos \cite{cárdeasbenavides2024} tanto evaluables (lecciones o tareas) como complementarios
(documentos, foros, glosario o metaverso), ha puesto de manifiesto que
el alumnado que se adentra en la aventura, demanda más información y
utiliza más recursos, además de permanecer más tiempo en estos espacios,
evitando el abandono \cite{Ratnasari2024} y la necesidad de
recibir una acreditación \cite{Rahimi2024}.

Este proyecto formativo ha buscado cerrar la brecha digital que afecta
al sector de la tercera edad quienes, aunque a menudo son consumidores
pasivos de medios, pueden beneficiarse enormemente de adquirir
competencias mediáticas. A través de iniciativas como las presentadas en
este artículo, se fomenta la alfabetización mediática y se impulsa la
participación de redes comunicativas que parten de las propias
instituciones.

