\section{Discusión}\label{sec-discusión}

Las personas entrevistadas en ambas comunas de Chile han integrado las
tecnologías digitales a sus experiencias cotidianas. Sin embargo, esas
experiencias son afectadas por las brechas digitales estructurales
existentes, lo que genera como consecuencia, limitaciones para usar
tecnologías en contextos laborales \cite[P.11]{MORALES2020}.

La tendencia a no hacer usos complejos de tecnología y concentrarse en
experiencias de entretenimiento, se conecta con investigaciones previas
respecto a los ambientes digitales como mecanismos de evasión de la
realidad y con posibles efectos negativos, que orientan hacia un uso
excesivo y unívoco de redes sociales, con posibles efectos sobre la
salud mental de los usuarios \cite{CHARITSIS2022,XU2023}. Al mismo tiempo, estas prácticas, aunque actúan como una
limitación, tienen un potencial de convertirse en herramientas para
avanzar hacia procesos de alfabetización digital más complejos, en
particular las que se vinculan con búsqueda de información o
comunicación comunitaria.

En relación con las brechas digitales y sus causas estructurales, y del
mismo modo que los trabajos previamente citados de \textcite{PETHIG2021}
o \textcite{LONGORIA2022}, puede decirse que su manifestación es
multidimensional. El contexto territorial, la desigualdad económica,
educativa y de género, entre otros factores, generan efectos
acumulativos en el mundo digital. A lo anterior se suma la propia
situación de discapacidad debido a las brechas de diseño de
dispositivos, aplicaciones y sitios web \cite{EGARD2021,KELLY2016}.

Otra cuestión preocupante es la experiencia de temor a la tecnología que
algunas entrevistadas manifestaron. A diferencia de lo señalado por
\textcite{PETHIG2021}, que asocian la tecnología a la protección frente
al miedo de exponerse a la sociedad, en las entrevistas se manifiesta un
temor a los dispositivos y a su manipulación.

En este contexto, las entrevistas en ambas comunas dan cuenta de pocas
oportunidades formales de alfabetización digital, por lo que las
experiencias de uso de tecnologías en el trabajo son limitadas,
concentrándose en aquellos que tuvieron acceso a educación
técnico-profesional. De esta forma, se manifiesta una amplia disparidad
entre las competencias laborales demandas por el mercado laboral y las
posibilidades de alfabetización disponibles en sus contextos locales. El
contexto de brechas digitales estructurales y las limitaciones de acceso
a alfabetización digital, son coincidentes con lo señalado por \textcite{QU2022}
respecto a la precarización laboral, baja calificación y remuneración.

Debido al diseño de sitios web gubernamentales que no cumplen los
criterios establecidos en las propias normativas, y de forma similar a
lo señalado por \textcite{LIN2018}, puede decirse que hay políticas de
accesibilidad, pero baja implementación. En contraste con ese déficit en
la política nacional, la accesibilidad proviene de políticas comunales
que facilitan la conectividad, y que a su vez tienen un potencial de
vinculación comunitaria, en la que las PeSD pueden acceder a internet,
espacios de encuentro comunitario y con potencial para la formación de
competencias digitales para el trabajo.

Lo anterior implica el desafío de diseñar estrategias de política para
enfrentar la inevitable penetración de las tecnologías digitales en
todos los ámbitos de la vida, y especialmente en el mundo del trabajo
\cite{PEREZROLDAN2021}.

De acuerdo con \textcite[p. 4438]{LIN2018}, la inclusión no es
exclusivamente un esfuerzo de política pública, sino más bien de
transformación de la sociedad. En ese mismo sentido apunta el trabajo de
\textcite[p. 3]{KOLOTOUCHKINA2022} que identifican cuatro
áreas para la inclusión digital: políticas de equidad digital, políticas
de estandarización, construcción de una cultura de accesibilidad
universal y compromisos sociales compartidos por el Estado, el sector
privado y la sociedad civil.

Finalmente, es importante considerar las limitaciones de la
investigación. Como se ha señalado, la investigación previa sobre PeSD y
el contexto digital laboral ha sido poco explorado, especialmente en
América latina. Es por ello que inicialmente se optó por un trabajo
descriptivo, y que requiere ser profundizado en el futuro con muestras
más amplias en otros contextos dentro del país o en América Latina y que
consideren más en detalle cuestiones como género o interculturalidad,
entre otros factores a considerar.