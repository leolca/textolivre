\section[Personas en situación de discapacidad]{Personas en situación de discapacidad: desde la experiencia de
	uso a las posibilidades de inclusión digital}\label{sec-personasensituacióndediscapacidad}
	
El análisis que se presenta a continuación, se organiza en torno a cinco
categorías principales y sus respectivas subdimensiones. En los casos en
que aparezca pertinente, se incorpora la visión del sector productivo
y sus respectivas subdimensiones, las que son presentadas en la \Cref{tab-03}.
	
\begin{table}[htpb]
\centering
\small
\begin{threeparttable}
\caption{Categorías y subdimensiones de análisis.}
\label{tab-03}
\begin{tabular}{p{0.425\textwidth} p{0.425\textwidth}}
\toprule
\multicolumn{1}{c}{Categorías} & \multicolumn{1}{c}{Subdimensiones}\\
\midrule
Experiencia de uso: se describen las experiencias con tecnologías
digitales de PeSD. & Uso de tecnología, uso informativo, uso
comunicacional, uso para entretenimiento, seguridad y temor a la
tecnología. \\
Brechas digitales: se describen diversas manifestaciones de brechas
digitales en sus experiencias. En este caso, las brechas entendidas como
causas. & Brechas digitales territoriales, educativas, económicas,
brechas digitales de diseño de dispositivos, aplicaciones y sitios
web. \\
Alfabetización digital: se describen necesidades y posibilidades de
acceso a alfabetización digital en sus contextos. & Alfabetización
digital como autoaprendizaje, con colaboración de familia y amigos,
aprendizaje funcional, uso de video para el aprendizaje; alfabetización
laboral. \\
Contexto laboral: se describen las experiencias de las PeSD respecto al
vínculo entre tecnologías digitales y el mundo del trabajo. En este
apartado se identifican las brechas y oportunidades en el sentido de sus
consecuencias en el ámbito laboral. & Búsqueda de trabajo en línea,
trabajo remoto, capacitación digital laboral. \\
Accesibilidad digital: se describen las condiciones de accesibilidad
digital en sus respectivos contextos. & Trámites en línea, acceso a
beneficios sociales, políticas de conectividad locales, accesibilidad
digital en el trabajo. \\
\bottomrule
\end{tabular}
\source{Elaboración propia.}
\end{threeparttable}
\end{table}
	
\subsection{Experiencia de uso de tecnologías}\label{sub-sec-experiencedeusodetecnologías}

Todas las personas entrevistadas manifestaron tener experiencias
cotidianas de uso de tecnologías digitales, pero con escasa vinculación
con el contexto laboral. Una primera tendencia es la predilección por el
uso de dispositivos móviles, especialmente teléfonos, que se asocia a su
facilidad de uso en comparación con los computadores:

\begin{quote}
	Es que es más fácil para mí usarlo (\ldots) el computador no, me voy a
	perder. Y en el celular, uno con el celular va a comprar material y te
	llaman, a un computador no {[}\ldots es{]} más accesible tener internet en
	el celular, en el uso diario, claro, ya el computador no lo uso.
	(SJ-M-7).
\end{quote}

Una experiencia de uso habitual es la búsqueda de información. Algunas
tecnologías facilitan estos procesos adaptándose a las necesidades de
las personas: \enquote{Cuando necesito buscar algo. Claro que me enseñaron a
	hablarle al parlante, porque de repente para escribir me cuesta}
(PT-F-3). El acceso a información, resulta clave, porque les permite
conocer también sobre beneficios sociales: \enquote{\ldots sabemos los beneficios y
	todo, porque nos enteramos por el Internet\ldots} (PT-M-5).

Otro uso habitual es para la comunicación interpersonal, con diferencias
significativas entre las personas entrevistas. Por una parte, usuarios
que debido a su situación de discapacidad tienen capacidades limitadas,
y por ello, las funciones de audio resultan de gran ayuda, tal como se
indica en los siguientes comentarios: \enquote{\ldots yo me manejo solo con
	WhatsApp y nada más, aplicaciones y esas cosas no tengo idea}
(SJ-F-11). \enquote{Mensaje de texto sí, o sino le mando audio (\ldots) no soy
	rápido para escribir, soy lento, entonces para hacerlo más rápido, mejor
	le mando audio} (PT-M-1).

Otras personas entrevistadas hacen un uso más amplio de aplicaciones y
redes sociales como Messenger, Facebook e Instagram. Solo una persona
entrevistada se comunicaba habitualmente por videoconferencia con su
familia.

Un aspecto relevante de destacar con relación a la comunicación, es la
participación en grupos de WhatsApp, que implican la vinculación con su
entorno de vecinos, amigos, o la participación en programas sociales u
otras iniciativas. Una entrevistada indica: \enquote{Yo tengo mis vecinas, mi
	WhatsApp con ellas, con mis vecinas de allá} (SJ-F-6). En algunos
casos, la participación es activa, que implica la gestión de los grupos
o procesos de intermediación:

\begin{quote}
	Estoy en un grupo aquí, de gestión vecinal de aquí de la comuna, donde
	me llega toda la información (\ldots) y yo como tengo al pasaje,
	entonces yo les mando todo, lo que hay, cuándo es el operativo, cuándo
	hay un operativo para los perros, para gatos, todo lo que hay, les mando
	todo a mis vecinos (SJ-F-11).
\end{quote}

Asociado a lo anterior están las prácticas sociales de entretenimiento
que se constituyen en el uso principal para la mayoría de las personas
entrevistadas, las que se vinculan con necesidades de conocimiento y
formación:

\begin{quote}
	Básicamente, lo que hago en el internet es, aunque suene absurdo, ver,
	ver la mayor cantidad de documentales o cosas de mi interés, como
	ciencia, que piensan los otros países, me gusta la antropología, pero me
	gusta más la historia y ciencia\ldots(SJ-F-4).
\end{quote}

En otros casos, el entretenimiento funciona como mecanismo de
distracción personal, el uso de YouTube, Facebook o Tik Tok que se
asocia a la idea de: \enquote{\ldots me gusta estar en el celular\ldots} (SJ-M-5); o
como expresa otro entrevistado: \enquote{Me meto a Facebook para ver las cosas
	que ponen, videos. Cosas así para entretenerme, más que nada} (SJ-M-9).
Esta tendencia podría constituirse en una barrera para la adquisición de
competencias digitales más complejas.

Finalmente, parece relevante destacar las experiencias de temor a la
propia tecnología:

\begin{quote}
	\ldots yo, a mí me da miedo\ldots Me da miedo porque a ver, yo siempre, como
	que yo misma me\ldots{} no voy a saber, no sé, voy a mandarme la
	embarrada y entonces, por ejemplo el celular, yo me manejo solo con
	WhatsApp y nada más, aplicaciones y esas cosas no tengo idea. (\ldots)
	hasta tocarlo me da miedo, porque, me dicen mamá mire, \enquote{hay no}, le
	digo yo, \enquote{pero mamá si no te va comer mamá}, \enquote{no por favor}, le digo
	yo, \enquote{que me puedo mandar una embarrada} (SJ-F-11).
\end{quote}

Aunque, no hay datos concluyentes al respecto, la tendencia de las
entrevistas sugiere que la búsqueda de entretenimiento está más
vinculada con el género masculino y las expresiones de temor hacia la
tecnología, asociadas al femenino.

\subsection{Brechas digitales}\label{sub-sec-brechasdigitales}

Una primera cuestión se refiere a lo que podríamos denominar brechas
digitales territoriales; las que en ambas comunas son asociadas con
dificultades de conectividad a internet, ya que las empresas que proveen
este servicio no tienen presencia en ciertos sectores de la comuna. Para
el caso de San Joaquín está vinculado con la percepción de seguridad:
\enquote{\ldots las compañías que ofrecen internet dentro de la comuna, igual
	existen sectores que son más peligrosos y las empresas ponen un poco de
	resistencia para entrar} (SJ-F-2).

En el caso de Pitrufquén, se vincula tanto con la distancia del centro
de la ciudad, como con contextos rurales. Un entrevistado indica que:
\enquote{\ldots para las poblaciones, al final, no hay muy buena conexión de red
	internet} (PT-M-4). En relación con los sectores rurales se indica:

\begin{quote}
	{[}He{]} probado por tres compañías más o menos, las más conocidas, y en
	algunos lugares dentro de la comuna no hay acceso a internet o no hay
	señal para hacer una llamada telefónica. Ahora, si me voy al sector
	rural es aún más complejo el tema de la conectividad\ldots{} (PT-M-7)
\end{quote}

Las personas entrevistadas tienen diversos grados de educación, desde
educación básica o secundaria incompleta hasta personas con formación
técnica o universitaria. Quienes no han podido finalizar sus estudios
son más afectados por las brechas digitales y no pueden hacer usos
complejos de tecnología:

\begin{quote}
	Sí, porque, por ejemplo, aquí me dicen: \enquote{tienes que entrar (\ldots) para
		hacer el formulario, para mandar el currículum}, pero no tengo ni idea
	(\ldots) cómo mandarlo. Entonces ya hace como cuatro meses que estoy
	tratando (\ldots) de mandar algo para poder tener una entrada de plata, y
	me piden que tengo que entrar a una cuestión, a una página y no tengo ni
	idea (SJ-F-6).
\end{quote}

Asociado a las brechas educativas, están las limitaciones para continuar
con su formación en el contexto actual. La formación en línea podría
generar muchas más oportunidades, ya que permite acceso a contenidos que
en los contextos locales no están disponibles. Pero debido a las
dificultades para desenvolverse en entornos digitales, prefieren la
presencialidad: \enquote{Si fuera presencial, sí, sí, pero si es \emph{online},
	no. Porque no, no me manejo mucho, pero presencial sí} (PT-F-3).

La última causa estructural tiene que ver con las brechas económicas. La
mayor parte de las personas entrevistadas señalan la existencia de
problemas relacionados con el costo de dispositivos y conexiones a
internet:

\begin{quote}
	Es muy caro tener computador o internet, y no todas las personas pueden
	tenerlos. Pero al día de hoy, casi todo se tiene que hacer por internet
	o el celular, es súper importante manejarlo (SJ-F-2).
\end{quote}

Las necesidades de uso, vinculadas a los costos, generan otros
problemas: \enquote{{[}Los costos{]} Son altísimos, un problema es que algunas
	{[}personas{]} dejan de hacer cosas para poder pagar el celular y el
	internet} (SJ-F-4). Frente a ello, los principales soportes económicos,
cuando existen, vienen de las propias familias: \enquote{Sí, no acá como los
	hijos se ponen. (\ldots) hasta mi teléfono me lo pagan}(SJ-M-9).
		
		Parece relevante comentar otra brecha que surge de la propia tecnología,
		cuando dispositivos, sitios web y otras plataformas digitales no son
		diseñados con criterios de accesibilidad. Son afectados, por ejemplo,
		por la constante actualización de programas:
		
		\begin{quote}
			\ldots me arreglaron el computador y me pusieron un programa actualizado.
			(\ldots) Yo estaba acostumbrado a un programa antiguo. Entonces, ahora
			me cuesta, porque no tengo los íconos de, que yo tenía antes (SJ-M-9).
		\end{quote}
		
		Otras personas indican que los teléfonos móviles resultan más adaptables
		a las diversas situaciones de discapacidad:
		
		\begin{quote}
			\ldots tengo mi compu, tengo mi teléfono, que ya ahí con él me gusta más
			trabajar, en el teléfono, que es más sensible.
			
			{[}en el{]} teléfono o computadores, yo le modifico el brillo, le pongo
			el modo nocturno que le dicen, entonces, en el modo nocturno yo lo
			manejo en el 50 \%, entonces, así mi vista se enfoca bien (PT-M-4).
		\end{quote}
		
		Más allá de los dispositivos, surgen problemas vinculados con los sitios
		web, especialmente de los sitios públicos, ya que las brechas de diseño
		en ellas impiden, no solo el acceso a información, sino también a
		beneficios sociales. Al respecto una persona entrevistada con
		discapacidad visual indica que:
		
		\begin{quote}
			\ldots una página de ministerio o estos típicos, \enquote{no soy un robot} que
			tenemos que rellenar y esa imagen hasta el día de hoy los lectores no
			son capaces de poder leerla y decirme: \enquote{H mayúscula, y griega}, o
			imágenes de escalera, o esos típicos \enquote{no soy un robot} que que no
			tiene sentido, y que si uno va a la opción que dice en audio, está en
			inglés y con una voz que no sé quién la pueda entender. Entonces eso
			todavía está complicado (PT-M-7).
		\end{quote}
		
\subsection{Alfabetización digital}\label{sub-sec-alfabetizacióndigital}

Una primera experiencia de alfabetización digital es el autoaprendizaje:
\enquote{Yo solo, metiendo las manos en la masa\ldots} (PT-M-4), por una
necesidad funcional de saber cómo utilizar un dispositivo, o por la
necesidad de aprender o de profundizar en diversos ámbitos de interés.
Otra persona entrevistada indica: \enquote{\ldots cada vez que tengo internet entro
	a lo mío, que es la electricidad y ver cómo avanza para estar al
	día\ldots} (PT-M-5).

Una segunda relación que posibilita la alfabetización es el
acompañamiento de familiares y amigos: \enquote{\ldots pero mi señora, mi señora me
	lo enseña, que yo le digo a ella enséñame. Ella me enseña toda la
	tecnología porque ella se mueve más\ldots} (SJ-M-8). Sin embargo, a
veces se pasa del aprendizaje a la dependencia, la constante presencia
de la familia para colaborar con las demandas tecnológicas cotidianas:

\begin{quote}
	\ldots ellos me dicen, mamá para pedir un Uber o algo, un auto, yo no tengo
	idea (\ldots) ahí lo suplen los niños, todo eso me lo solucionan los niños.
	Si tengo que hacer algo en el banco, (\ldots) llenar formulario (\ldots)
	todo eso, ellos me tienen que acompañar\ldots{} (SJ-F-11).
\end{quote}

Aunque hay algunos casos de personas entrevistadas que tienen
habilidades tecnológicas más avanzadas, la mayor parte solo han
adquirido competencias básicas o generales. La experiencia habitual es:
\enquote{\ldots lo común y corriente no más, WhatsApp, Facebook y Messenger}
(SJ-M-7). Estos conocimientos básicos de internet, orientan la
experiencia de uso hacia las aplicaciones que demandan menos
conocimientos y que son de uso común: \enquote{Me meto a Tik Tok, a Facebook,
	WhatsApp} (PT-F-8). Otra persona indica que:

\begin{quote}
	\ldots para mirar un video en YouTube, aunque te parezca tirado de las
	mechas, pero hasta para eso ya me cuesta un poco, meterme, buscar en
	Google no, no soy de meterme mucho ni en el teléfono. Mi teléfono
	funciona con Facebook, WhatsApp y llamadas (PT-M-2).
\end{quote}

Para otras personas, la experiencia de ver videos en Internet tiene una
orientación más centrada en procesos de aprendizaje:

\begin{quote}
	\ldots el computador lo usó para hacer investigaciones, de muchos temas que
	a mí me importan. Te voy a contar los temas que estoy investigando\ldots
	Investigué sobre el autismo, el asperger, el \emph{bullying}, sobre la
	discriminación, las vacunas. Sobre el tipo de mascarillas que hay que
	usar ahora\ldots{} (PT-F-11).
\end{quote}

Utilizar tecnologías en el trabajo es algo que depende fundamentalmente
del nivel educacional de las personas. Las ayudas del Estado para que
PeSD adquieran tecnologías se vinculan a la educación formal, y no con
la formación para el trabajo:

\begin{quote}
	En su momento cuando estuve estudiando sí, FONADIS era en ese entonces,
	que después pasó a ser el servicio de los programas de apoyo a la
	educación superior, y el aporte era para adquisición de software,
	computador\ldots (PT-M-7).
\end{quote}

La mayor parte de las personas entrevistadas no han recibido una
alfabetización digital para el trabajo, y cuando la respuesta es
positiva, sólo han tenido alfabetización básica:

\begin{quote}
	Básico de escribir. Como se prende, se apaga. Y después para poder hacer
	el currículo, de escribir. No, si esa parte estuvo buena igual. Y eso es
	lo que me gustaría igual aprender más en la parte del computador
	(PT-M-9).
\end{quote}

Esta experiencia de limitaciones con relación a las posibilidades de
formación digital para el trabajo, es coincidente con las discusiones de
los representantes del sector productivo:

\begin{quote}
	\ldots muchas veces las empresas quieren dar trabajo, quieren dar
	oportunidades, pero la persona que llega a buscar esa oportunidad tiene
	muchas brechas y lamentablemente la empresa no puede dar esa opción no
	porque no quiera, sino que se da cuenta que estas brechas no son acordes
	con la empresa (RM-M-1).
\end{quote}

Esto nos lleva a la experiencia específica de uso de tecnologías en el
contexto laboral, que se discute en la siguiente sección.

\subsection{Contexto laboral}\label{sub-sec-contextolaboral}

Una experiencia más habitual en las personas entrevistadas con más
estudios, es la búsqueda de empleo virtual, que implica promoción a
través de sus propias redes, participación en ferias virtuales de
empleabilidad y envíos de currículum en línea. Pero, a pesar de aparecer
como una vía expedita e inclusiva, no han encontrado empleo por estos
medios:

\begin{quote}
	Sentirme discriminado no, porque no alcanzo a verle las caras, pero
	porque ahora como es todo \emph{online} ven tu currículum y ahí no, no
	hay\ldots{} ahora de hecho creo tener una entrevista dentro estos días,
	\emph{online} con la empresa, pero no me han llamado y tampoco me llegó
	el correo y ahí estoy, esperando, y mientras tanto voy postulando a otro
	trabajo, pero ninguno sale, ninguno sale (PT-M-5).
\end{quote}

Hay situaciones donde las ofertas laborales implican complejos procesos
de desplazamiento fuera de la comuna o en otras regiones. Respecto a la
consulta a una entrevistada por la búsqueda de trabajo a través de
internet indica que: \enquote{Si ha buscado, pero la mandaban muy lejos a
	trabajar\ldots} (SJ-F-3)\footnote{La transcripción de esta entrevista
	corresponde a las notas de la entrevistadora, debido a las
	dificultades de comunicación verbal de la persona entrevistada.}. Otra
entrevistada señala:

\begin{quote}
	\ldots lo utilizaba para todo y para buscar páginas e inscribirme en
	búsqueda de trabajo, estoy en esta cuestión de Computrabajo, pero
	no\ldots{} me lo envió de Santiago la coordinadora que nos hizo los
	cursos, me envió esa página, pero no ha resultado nada (PT-F-10).
\end{quote}

Por otra parte, hay algunos relatos de experiencias de autoempleo en que
se utilizan habilidades digitales básicas para contactar clientes por
redes sociales, especialmente WhatsApp. Sin embargo, no apareció ninguna
conversación sobre trabajo remoto, algo que para el sector productivo
aparece como una posibilidad ampliamente compartida:

\begin{quote}
	\ldots quizás antes no se veía el teletrabajo como algo posible y era como
	una opción así como lejana, sin embargo y aunque suene fuerte gracias a
	la pandemia nos dimos cuenta que realmente se pueden hacer ciertas cosas
	desde la casa y bajo esa perspectiva sentimos que se abren nuevas
	perspectivas; como ese es un cambio cultural se debe conversar entre las
	áreas de las empresas (\ldots) y sin duda que eso abre el espectro para
	otro tipo de discapacidades y muchas veces lo he visto que me han dicho,
	(\ldots), yo no puedo tomar metro, porque el metro generalmente va lleno
	y yo no puedo caminar mucho, entonces esas cosas se pueden generar como
	nuevas opciones de trabajo\ldots{} (RM-M-1).
\end{quote}

Para poder efectuar trabajo desde casa, las PeSD deben contar tanto con
conectividad como con competencias digitales de mayor complejidad. Ello
implica que el uso de tecnologías en el trabajo está habitualmente
orientado a quienes tienen formación técnica o profesional, y en los que
se supone mayor autonomía para el uso de aplicaciones y para la gestión
de información. Como en muchos ámbitos laborales actuales, una pregunta
habitual es por las planillas de cálculo: \enquote{\ldots a veces suelen decir que
	si manejo Excel, pero sí, así que todo bien} (SJ-M-5).

La experiencia de capacitación de las PeSD entrevistadas es muy diversa,
y depende tanto de su educación previa como del acceso a oportunidades
en sus comunas, y fue posible identificar relatos de personas que nunca
se han capacitado, así como también de personas que al terminar su
educación universitaria han efectuado diplomados y están planificando la
realización de posgrados.

La experiencia habitual implica limitaciones, la necesidad de adquirir
competencias digitales para el trabajo, o de utilizar tecnologías para
adquirir competencias laborales en un ámbito específico de interés. En
este sentido, hay diversos relatos de uso de YouTube como herramienta de
auto-formación:

\begin{quote}
	\ldots en cocina, yo de repente, yo por ejemplo, antes siempre he
	comprado la masa para pizza y la aprendí a hacer, y las empanadas de
	horno también aprendí a hacer la masa, así que voy siempre como
	aprendiendo cosas, busco en YouTube y encuentro cosas y las hago
	(PT-F-8).
\end{quote}

\subsection{Accesibilidad digital}\label{sub-sec-accesibilidaddigital}

En relación con la accesibilidad digital, las conversaciones con las
personas entrevistadas no permiten distinguir en ninguna de las dos
comunas prácticas habituales de realización de trámites o de visitas a
páginas web del Estado o municipales. Como se ha indicado previamente,
la orientación habitual es hacia usos sociales, y por ello, usar
tecnologías para el vínculo con el Estado no aparece como prioridad.

Aunque algunos entrevistados tengan la intención de actuar con
autonomía, algunos aspectos del proceso generan barreras: \enquote{\ldots por
	ejemplo, por cosas de papeles, de trámites, necesito ayuda, pero la
	mayoría de las veces las trato de hacer todo yo solo} (PT-M-4). Esto se
asocia con la discusión previa sobre las brechas digitales de diseño,
especialmente de los \emph{captcha}, que resultan especialmente
complejos para personas en situación de discapacidad visual.

Como se ha comentado, las políticas que otorgan beneficios directos de
acceso a tecnologías digitales están centradas en la educación y no en
el mundo del trabajo. Sólo en un caso se indica que: \enquote{por intermedio de
	acá, del sistema de capacitación, pude postular a tener un computador
	(\ldots) una herramienta para poder estudiar} (PT-M-4). Otro
entrevistado con estudios universitarios indica que: \enquote{En su momento,
	cuando estuve estudiando, sí (\ldots) y el aporte era para adquisición
	de software, computador y software} (PT-M-5).

Parece relevante considerar para futuras políticas de accesibilidad
algunas ideas que surgen de los propios entrevistados, que identifican
ámbitos en los que podrían desarrollarse programas que faciliten la
generación de capacidades u oportunidades laborales: tarifas de internet
rebajadas para PeSD; mejorar la utilización de redes sociales para la
promoción de empleos; y la incorporación de comunicación inclusiva en
las diversas plataformas estatales.

Por otra parte, y pensando en el ámbito comunal, especialmente debido a
las dificultades de conectividad detectadas en ambas comunas, aparece
como una experiencia muy valorada por las personas entrevistadas los
servicios de internet que otorgan las respectivas municipalidades, por
ejemplo, conexiones en lugares públicos como plazas, consultorios o
bibliotecas: \enquote{Las veces que he ido al consultorio y ahí me conectaba en
	el consultorio, sabía la clave y todo y ahí me conectaba\ldots} (PT-F-3).

Finalmente, desde la perspectiva del sector empresarial, la
accesibilidad se vincula, por una parte, con las posibilidades de
trabajo remoto:

\begin{quote}
	(\ldots) hoy día la mayoría de los trabajos sobre todo en esas áreas son
	\emph{online} {[}comercio y negocios digitales{]}. Hay gente desde la
	casa, no es necesario que vayan a la planta o a la oficina, entonces se
	facilita mucho más, creo yo, la incorporación porque hay un espacio
	cómodo para ti, conocido por ti y preparado desde el punto de vista de
	una persona en situación de discapacidad, para ti (RA-M-5).
\end{quote}

Se relaciona, también, con las regulaciones internas existentes en torno
a los criterios de accesibilidad. En el caso de una empresa
transnacional, que se guía por estándares más complejos que los exigidos
en Chile, da cuenta de una preocupación por la accesibilidad que está
institucionalizada:

\begin{quote}
	\ldots por ejemplo, cuando existen los \emph{webinar}, los \emph{webinar}
	para las personas con discapacidad visual, nosotros sabemos que nos
	tenemos que describir, por ejemplo; cuando nosotros hacemos videos
	corporativos, tienen lengua de señas, pero sabemos que la lengua de
	señas es local, o sea he incluso dentro del mismo territorio, no es una
	lengua internacional, entonces ahí nos vamos adaptando. Ahora mientras
	vayan ingresando personas a la compañía con distintos tipos y grados de
	discapacidad, se van adaptando justamente los programas, que si por
	ejemplo, alguien necesita que el computador escriba a través de la
	reproducción de la voz, se gestiona, si se requiere un software para
	personas como te decía personas con discapacidad visual, también, o sea
	todo se va adaptando de acuerdo a la realidad de la persona (RM-F-3).
\end{quote}

Esta experiencia da cuenta de condiciones ideales de accesibilidad e
inclusión en el mundo del trabajo, y que tienen una notoria diferencia
con las posibilidades que habitualmente encuentran las personas en
situación de discapacidad al intentar ingresar al mundo laboral en
Chile.



