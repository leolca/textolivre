\section{Conclusiones}\label{sec-conclusiones}

Se ha presentado una descripción de experiencias y necesidades de PeSD
en la búsqueda de su inserción laboral en dos comunas de Chile. A partir
de las entrevistas efectuadas fue posible trazar una ruta que identifica
experiencias de uso, situaciones de brecha digital, posibilidades de
alfabetización, de acceso al mundo laboral, así como también de
diagnosticar las políticas de accesibilidad.

Aunque todas las personas entrevistadas usan tecnología, esas
experiencias de uso generan limitaciones que se entrelazan con las
brechas digitales existentes. Las causas se vinculan con condiciones
estructurales y sociales, es decir, con las decisiones legales,
políticas, económicas que determinan el diseño y la implementación
social de tecnologías. Desde esta perspectiva hay que considerar el
diseño de dispositivos y aplicaciones, así como su implementación
social, es decir, la infraestructura por medio de la cual se ponen en
funcionamiento y que las orienta hacia un uso social que reduce
considerablemente las posibilidades de usar las tecnologías para el
desarrollo de capacidades laborales.

Las causas de las brechas estarían vinculadas, además, con una serie de
características socioculturales que hacen menos factible el
aprovechamiento de los beneficios tecnológicos. Entre las más relevantes
pueden identificarse la falta de educación formal, la situación de
pobreza, la discriminación por edad o género, la ausencia/presencia de
vínculos sociales, la situación de ruralidad y las condiciones de salud.
Es decir, una brecha digital con causas multidimensionales.

Pueden considerarse acá la falta de regulación respecto a la
implementación de políticas de accesibilidad, la orientación excesiva de
las tecnologías a públicos que pueden consumir productos de alto costo,
o las decisiones sobre la construcción de infraestructuras tecnológicas
(como la presencia o ausencia de conectividad en zonas rurales o en
poblaciones con mayor pobreza y desigualdad).

Respecto a las consecuencias, estas se vinculan fundamentalmente con las
restricciones a las posibilidades de usar la tecnología, y a la
generación de posibilidades de empleabilidad. Es decir, que la capacidad
de uso de tecnologías se ve afectada por la educación, la situación
socioeconómica, la edad, el género, la ausencia/presencia de vínculos
sociales, la ruralidad y la salud, cuestiones a las que se suma la
situación de discapacidad.

Pero el aspecto más relevante de las consecuencias es el para qué se usa
la tecnología, y ello, no depende exclusivamente de las personas
usuarias, sino del contexto social donde las tecnologías se implementan,
y las posibilidades de integración social que ahí se ponen de
manifiesto. En este sentido, hay un primer nivel de implementación
social de tecnologías donde se espera un uso en cuestiones funcionales
como prender o apagar un dispositivo, hacer una búsqueda de información,
o el uso básico de redes sociales (como, por ejemplo, enviar un audio).
Las brechas digitales emergen, por tanto, con relación a cuestiones más
complejas como el trabajo y el consumo: por ejemplo, el uso de
aplicaciones ofimáticas específicas o la realización de transacciones
como transferencias o una compra en línea. Podría considerarse todavía
un nivel mayor de complejidad que se refiere a los límites para la
creación de páginas web o aplicaciones digitales, esto es, la capacidad
de afectar los procesos de diseño e implementación.

Ello implica la detección y preocupación por la definición legal y de
política en relación con la accesibilidad digital; lo que, a su vez,
permite caracterizar las diversas brechas digitales existentes, y las
necesidades de alfabetización. Sin embargo, la superación de las brechas
digitales no se resuelve de forma exclusiva con la formación en
competencias digitales, sino con transformaciones mucho más profundas
que implican considerar tanto cambios de política como culturales, que
abran el camino hacia la inclusión digital en el mundo del trabajo.

En este sentido, la construcción de las condiciones de accesibilidad
digital no puede resolverse exclusivamente con un contenido legal, sino
con un proceso que involucra a los organismos públicos, al sector
productivo (tanto como creadores de tecnología, como en su rol de
empleadores), y a las propias PeSD.

En términos de política, los datos de las entrevistas plantean también
oportunidades de pensar nuevas políticas de accesibilidad específicas
que posibiliten la integración más efectiva de PeSD al mundo laboral.
Considerando una oferta de alfabetización para el trabajo más amplia,
que considere las experiencias de uso de tecnologías que ya poseen las
personas (ver videos de YouTube o de canales de Facebook) como un primer
peldaño en un proceso gradual de adquisición de competencias digitales.

Frente a una tendencia de uso de tecnologías individual, las políticas
de accesibilidad digital y de alfabetización pueden enfatizar la
dimensión comunitaria que ha sido valorada por las personas
entrevistadas. No se trata únicamente de usar tecnología, sino del
sentido o de los intereses que las tecnologías buscan potenciar, en este
caso, la construcción de capacidades de liderazgo y de identificación
con la comunidad.
