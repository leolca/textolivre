\section{Metodología}\label{sec-Metodología}

El proyecto de investigación se orientó desde un diseño descriptivo
orientado al diagnóstico y consideró entrevistas semi-estructuradas de
PeSD de la comuna de San Joaquín, en la región Metropolitana, así como
también de la comuna de Pitrufquén en la región de la Araucanía. Esta
primera selección se debió a un criterio de oportunidad ya que el equipo
de trabajo conformado por investigadores de la Pontificia Universidad
Católica de Chile, la Universidad Central y la Universidad Católica de
Temuco, se encuentran ubicados en las regiones seleccionadas. Ambas
comunas tienen características distintivas, especialmente en relación
con las diferencias urbano-rurales y lo que ello implica en términos de
conectividad e infraestructura digital. Los vínculos locales del equipo
de investigación, facilitaron un primer contacto con los municipios de
cada comuna y con las Oficinas Municipales de Información Laboral (OMIL)
y de Discapacidad, las que facilitaron la vinculación con las personas
entrevistadas.

A partir de ello, se pudo contactar a 23 PeSD, constituyéndose una
muestra por conveniencia orientada por un criterio de casos típicos
\cite[p. 82]{FLICK2007}. La \Cref{tab01} presenta la distribución que implica la identificación de la comuna (SJ: San Joaquín; PT: Pitrufquén), el género (Femenino: F; Masculino: M), y un código numérico único con el que se
garantiza el anonimato. Se incluye en la tabla información general sobre
el tipo de discapacidad declarada durante las entrevistas.

%tabela grande demais para não usar o longtable, ou comando análogo.
\begin{table}[!htpb]
\centering
\small
\begin{threeparttable}
\caption{Muestra PeSD.}\label{tab01}
\begin{tabular}{l>{\raggedright}p{0.2\textwidth}>{\raggedright}p{0.2\textwidth}l}
\toprule
\multicolumn{4}{c}{Comuna de San Joaquín (SJ)}\\
\multicolumn{1}{p{0.2\textwidth}}{Código de Identificación (comuna, género, número de entrevista)} &
\multicolumn{1}{p{0.2\textwidth}}{Situación de Discapacidad} &
Educación & 
\multicolumn{1}{p{0.2\textwidth}}{Edad al momento de la entrevista} \\
\midrule
SJ-M-1 & Física moderada & Media completa & 32 \\
SJ-F-2 & Física moderada & Media completa & 25 \\
SJ-F-3 & Física leve & Básica completa & 38 \\
SJ-M-4 & Psíquica moderada & Universitaria completa & 40 \\
SJ-M-5 & Física moderada & Secundaria técnica profesional completa &
25 \\
SJ-F-6 & Física moderada & Secundaria incompleta & 52 \\
SJ-M-7 & Física severa & Secundaria incompleta & 54 \\
SJ-M-8 & Física moderada & Primaria incompleta & 71 \\
SJ-M-9 & Física severa & Universitaria completa & 60 \\
SJ-M-10 & Física moderada & Secundaria técnica profesional completa &
46 \\
SJ-F-11 & Física moderada & Primaria completa & 51 \\
SJ-M-12 & Física moderada & Secundaria completa & 60 \\

\midrule
\multicolumn{4}{c}{Comuna de Pitrufquén (PT)} \\
 
\multicolumn{1}{p{0.2\textwidth}}{Código de Identificación(comuna, género, número de entrevista)} &
\multicolumn{1}{p{0.2\textwidth}}{Situación de Discapacidad} &
Educación & 
\multicolumn{1}{p{0.2\textwidth}}{Edad al momento de la entrevista} \\
\midrule
PT-M-1 & Física e intelectual leve & Básica Incompleta & 47 \\
PT-M-2 & Física leve & Secundaria técnica profesional completa & 45 \\
PT-F-3 & Física leve & Secundaria completa & 50 \\
PT-M-4 & Física leve & Secundaria técnica profesional completa & 23 \\
PT-M-5 & Física leve & Secundaria completa & 43 \\
PT-F-6 & Intelectual leve & Secundaria completa & 25 \\
PT-M-7 & Física leve & Universitaria completa & 38 \\
PT-F-8 & Física leve & Secundaria completa & 40 \\
PT-M-9 & Física leve & Secundaria incompleta & 40 \\
PT-F-10 & Física leve & Secundaria incompleta & 48 \\
PT-F-11 & Física leve & Secundaria incompleta & 27 \\
\bottomrule
\end{tabular}
\source{Elaboración propia.}
\end{threeparttable}
\end{table}

Se efectuaron, además 5 entrevistas a representantes del sector
productivo (SP), con el objetivo de incluir en el análisis información
relevante desde la perspectiva de los posibles empleadores. En la \Cref{tab-02} se caracteriza a este segundo grupo de entrevistados.

\begin{table}[!htpb]
\centering
\small
\begin{threeparttable}
\caption{Muestra Sector Productivo.}\label{tab-02}
\begin{tabular}{l>{\raggedright}p{0.2\textwidth}>{\raggedright}p{0.2\textwidth}l}
\toprule
\multicolumn{4}{c}{Región Metropolitana (RM) - Región de la Araucanía (RA).}\\		
\multicolumn{1}{p{0.2\textwidth}}{Código de Identificación(región, género, número de entrevista)} &
Educación & Ámbito de Trabajo & \multicolumn{1}{p{0.2\textwidth}}{Edad al momento de la entrevista} \\
\midrule
RM-M-1 & Técnico Profesional & Selección de personal. Empresa chilena. &
42 \\
RM-M-2 & Universitaria completa & Inclusión laboral.	
Organización gremial. & 59 \\
RM-F-3 & Universitaria completa & Área de sustentabilidad, diversidad e
inclusión.		
Empresa transnacional. & 42 \\
RM-M-4 & Técnico Profesional & Dueño.
Pyme. & 47 \\
RA-M-5 & Universitaria completa & Selección de personal. Empresa
chilena. & 43 \\
\bottomrule
\end{tabular}
\source{Elaboración propia.}
\end{threeparttable}
\end{table}


Se escogió efectuar entrevistas semiestructuradas, ya que ello permitió
la flexibilidad necesaria para adaptar la conversación a las necesidades
de comunicación de las personas entrevistadas. Es importante consignar
que, durante las entrevistas, y a lo largo de todo el proceso de
investigación, se siguieron los protocolos éticos para la investigación
con seres humanos considerando el resguardo de la identidad,
consentimientos informados y compromisos de confidencialidad por parte
del equipo investigador. Previo al trabajo de campo, el proyecto de
investigación fue aprobado por el Comité de Ética de la Pontificia
Universidad Católica de Chile.

Las entrevistas fueron grabadas en formato de audio digital, excepto en
los casos en que debido a la situación de discapacidad no pudo grabarse
la entrevista y la transcripción fue realizada a partir de notas de
campo. Para el análisis, se utilizó el software Atlas.ti 23, por medio
del cual se organizó el análisis que recogió tanto las preocupaciones
identificadas previamente en la discusión teórica, como los temas
emergentes que surgieron desde los datos.
