\section{Introducción}\label{sec-Introducción
}

Se presenta un diagnóstico orientado a describir experiencias y
necesidades de personas en situación de discapacidad (PeSD). Esta
discusión se sitúa en el contexto de la búsqueda de su inserción laboral
en el marco de una cada vez más amplia digitalización, y de la
implementación de la Ley n° 21.015 \citeauthor{CHILE2017}, en la que se fijan cuotas de
contratación de PeSD.

La investigación se efectuó en dos comunas de Chile: San Joaquín,
ubicada en la Región Metropolitana; y Pitrufquén, en la Región de la
Araucanía, situada en un contexto urbano-rural en la zona centro-sur del
país. Con ello, fue posible identificar las similitudes y diferencias en
sus respectivos entornos digitales para su caracterización respecto a
sus posibilidades de inserción laboral.

Aunque el campo de los estudios sobre las transformaciones digitales y
el mundo digital laboral es muy amplio, las investigaciones sobre PeSD y
su inclusión laboral es, por el contrario, más limitado. Estudios
recientes han procurado identificar aquellos grupos sociales afectados
por procesos de exclusión digital. \textcite[p. 2]{PETHIG2021}
mencionan que, para el contexto europeo, estos grupos serían personas
con desventajas económicas, personas mayores, veteranos de guerra,
personas en contextos rurales y PeSD. Por su parte, \textcite{LONGORIA2022},
%Longoria, Bustamante, Ramírez-Montoya y Molina (2022), 
agregan a las
primeras naciones o pueblos originarios, y a las mujeres.

La investigación en relación con estos grupos, habitualmente excluidos
de los beneficios de las transformaciones digitales, tiene
significativas diferencias. Puede distinguirse una más amplia
investigación de la exclusión digital de personas mayores \cite{BINIMELISESPINOZA2023,MARTINEZHEREDIA2020}; y una preocupación reducida
respecto a las PeSD que se concentra en las brechas en educación formal
\cite{PORTE2021}, pero en el ámbito del trabajo los aportes son
reducidos en el contexto latinoamericano {\cites[p. 4439]{LIN2018}[p. 9]{PETHIG2021}{MORALES2020}{PEREZROLDAN2021}}.

\subsection{De la exclusión a la inclusión digital en el mundo laboral}\label{sub-sec-delainclusion}

En Chile, la accesibilidad universal, es definida y regulada por el
artículo n°3 de la Ley n° 20.422, el que en su letra b, especifica que
accesibilidad se refiere a:

\begin{quote}
	La condición que deben cumplir los entornos, procesos, bienes, productos
	y servicios, así como los objetos o instrumentos, herramientas y
	dispositivos, para ser comprensibles, utilizables y practicables por
	todas las personas, en condiciones de seguridad y comodidad, de la forma
	más autónoma y natural posible \cite{MINISTERIODEPLANIFICACION2010}.
\end{quote}

Estos nuevos mecanismos legales de protección, se sitúan en el cruce
entre las políticas de inclusión social y la búsqueda de cohesión social
en contextos democráticos. Según expresan \textcite{MALDONADO2020}, avanzar en derechos y en medidas de inclusión específicas se
sitúa en la búsqueda del cuidado de la democracia, enfrentado a diversas
tensiones sociales y políticas recientes. Ello implica el desarrollo de
mecanismos de protección social más efectivos, que den cuenta de los
procesos de transformación social, tecnológica y sus consecuencias sobre
el empleo y la calidad de vida.

Con relación al ámbito de la tecnología y el mundo laboral, la
accesibilidad se vincula, además, con el acceso a espacios laborales,
utilización de maquinarias o herramientas, y especialmente a las
posibilidades de utilización de dispositivos y aplicaciones en entornos
digitales. Como se ha señalado previamente, la implementación de la Ley,
21015, de inclusión laboral, genera la necesidad de indagar y
diagnosticar respecto al contexto y condiciones para el acceso al
mercado laboral de PeSD.

Las investigaciones sobre brecha digital y el mundo laboral referidas a
PeSD, son limitadas. Parece relevante la distinción entre \emph{gap} y
\emph{divide} que proponen \textcite[p. 2]{LONGORIA2022}. Ambos conceptos son traducidos indistintamente como
\enquote{brechas}, sin embargo, \emph{gap} se asociaría con aquello que causa
las brechas y \emph{divide} con sus consecuencias en las personas y en
la sociedad.

Los estudios sobre brecha digital y contexto laboral de PeSD, enfatizan
causas económicas y de implementación de políticas de accesibilidad
\cite[p. 4440]{LIN2018}, o la presencia de múltiples
discapacidades \cite[p. 734]{SCANLAN2022}. Respecto a las consecuencias han
sido relacionadas con el acceso a trabajo precario, poco calificado y de
baja remuneración \cite[p. 726]{QU2022}; el miedo o la ansiedad con
relación a la tecnología, y la utilización de entornos virtuales como
protección frente a los estigmas sociales \cite[p. 2]{PETHIG2021}.

Por su parte, la alfabetización digital puede comprenderse como
adquisición de capacidades para enfrentar tanto las causas como las
consecuencias de las brechas. Nuevamente se pone de manifiesto un vacío
en la investigación sobre alfabetización digital de PeSD, ya que se
concentran en procesos de alfabetización digital en contextos escolares
o universitarios \cite{IBRAIMKULOV2022}, con escasa
investigación sobre la vida adulta \cite{BARLOTT2021}, o el
trabajo \cite{GUPTA2021}.

En este último caso, deben considerarse las significativas diferencias
entre contextos desarrollados y América Latina, en relación con las
posibilidades de empleabilidad existentes, y las demandas de
requerimientos específicos como horarios de trabajo diferenciados,
teletrabajo, estaciones de trabajo y equipos computacionales adaptados.

Respecto al contexto chileno, según el Servicio Nacional de Discapacidad
(Senadis), actualmente en Chile existen dos millones setecientos tres
mil ochocientos noventa y tres personas adultas en situación de
discapacidad, lo que representa un 17,6 \% del total de la población del
país \cite[p. 14]{SENADIS2022}. De esa total de población adulta en
situación de discapacidad, un 63,5 \% son mujeres y un 36,5 \% hombres
\cite[p. 18]{SENADIS2022}. Senadis indica que las personas en situación de discapacidad son el 19,1 \% y un 22 \% de las regiones Metropolitana y de la Araucanía, aunque no hay datos comunales disponibles \cite[p. 30]{SENADIS2022}. Finalmente, parece relevante destacar que del total de personas desocupadas en el país, un 3,9 \% están en situación de discapacidad \cite[p. 52]{SENADIS2022}, sin embargo, un 56,1 \% del total de personas en situación de discapacidad no tiene actividad laboral \cite[p. 58]{SENADIS2022}.

Este trabajo es un esfuerzo por diagnosticar en Chile las necesidades de
las personas en situación de discapacidad (PeSD) y sus posibilidades de
inserción laboral en un entorno de transformaciones digitales.