\documentclass[portuguese]{textolivre}

% metadata
\journalname{Texto Livre}
\thevolume{17}
%\thenumber{1} % old template
\theyear{2024}
\receiveddate{\DTMdisplaydate{2024}{2}{23}{-1}}
\accepteddate{\DTMdisplaydate{2024}{4}{22}{-1}}
\publisheddate{\DTMdisplaydate{2024}{5}{18}{-1}}
\corrauthor{Fernando Lionel Quiroga}
\articledoi{10.1590/1983-3652.2024.51341}
%\articleid{NNNN} % if the article ID is not the last 5 numbers of its DOI, provide it using \articleid{} commmand 
% list of available sesscions in the journal: articles, dossier, reports, essays, reviews, interviews, editorial
\articlesessionname{essays}
\runningauthor{Quiroga e Bessa}
%\editorname{Leonardo Araújo} % old template
\sectioneditorname{Daniervelin Pereira}
\layouteditorname{João Mesquita}

\title{A educação em tempos de \textit{smartphones} e redes sociais: por uma crítica permanente no enfrentamento da dessubjetivação e monitoramento}
\othertitle{Education in times of smartphones and social networks: for a permanent criticism in addressing desubjectivation and monitoring}

\author[1]{Fernando Lionel Quiroga~\orcid{0000-0003-4172-2002}\thanks{Email: \href{mailto:fernando.quiroga@ueg.br}{fernando.quiroga@ueg.br}}}
\author[2]{Rosângela de Bessa~\orcid{0009-0007-1967-5760}\thanks{Email: \href{mailto:rdbessa01@gmail.com}{rdbessa01@gmail.com}}}
\affil[1]{Centro de Ensino e Aprendizagem em Rede, Programa de Pós-Graduação em Educação, Linguagens e Tecnologias, Anápolis, GO, Brasil.}
\affil[2]{Universidade Estadual de Goiás, Anápolis, GO, Brasil.}

\addbibresource{article.bib}

\begin{document}
\maketitle
\begin{polyabstract}
\begin{abstract}
O presente estudo tem como objetivo problematizar os sentidos decorrentes do uso de \textit{smartphones} e outros dispositivos móveis na sociedade contemporânea, de modo geral, e em âmbito educacional, de modo específico. Metodologicamente, tal estudo ampara-se na perspectiva teórica e conceitual, enquadrando-se no gênero ensaístico no tratamento das informações. Como principais resultados, observamos que tais dispositivos atuam como instrumentos de monitoramento e controle, impactando em um processo de dessubjetivação e perda cognitiva, além de suscitar questões éticas, sociais e de privacidade. Além disso, avançamos no debate acerca do uso dos dispositivos móveis no contexto escolar, bem como analisamos as relações entre educação e tecnologias digitais a partir de documentos como a Base Nacional Comum Curricular (BNCC), e o Relatório de Monitoramento Global da Educação (UNESCO). Concluímos, na direção da construção de respostas sobre os dilemas resultantes desta relação, a elaboração de uma \textit{crítica permanente} capaz de superar atitudes desatentas e ingênuas em face do teor ideológico, político e estratégico do paradigma da conectividade, cujos efeitos têm manifestado forte potencial de dessubjetivação, perda cognitiva, \textit{bulliyng}, cultura do ódio, bem como de outros problemas relacionados à saúde.

\keywords{Tecnologia e educação \sep Cultura digital \sep Disseminação da tecnologia educacional \sep Crítica}
\end{abstract}

\begin{english}
\begin{abstract}
The present study aims to problematize the meanings arising from the use of smartphones and other mobile devices in contemporary society, in general, and in the educational context, specifically. Methodologically, this study is based on a theoretical and conceptual perspective, falling within the genre of the essay in the treatment of information. As main results, we observed that such devices act as monitoring and control instruments, impacting a process of desubjectification and cognitive loss, in addition to raising ethical, social and privacy issues. Furthermore, we advance the debate about the use of mobile devices in the school context, as well as analyzing the relationships between education and digital technologies based on documents such as the National Common Curricular Base (BNCC), and the Global Education Monitoring Report (UNESCO). We conclude, in the direction of constructing answers to the dilemmas resulting from this relationship, the elaboration of a permanent critique capable of overcoming inattentive and naive attitudes in the face of the ideological, political and strategic content of the connectivity paradigm, whose effects have manifested a strong potential for desubjectification, cognitive loss, \textit{bullying}, hate culture, as well as other health-related problems.

\keywords{Technology and education \sep Digital culture \sep Dissemination of educational technology \sep Criticism}
\end{abstract}
\end{english}
\end{polyabstract}

\section{Introdução}
O dispositivo móvel, na definição de \textcite[p. 10]{canto_dispositivos_2018}, pode ser considerado “[...] um aparato de computação portátil, pequeno, geralmente equipado com um método de entrada e uma tela de exibição (tela sensível ao toque ou um mini teclado)”. Estes dispositivos, no cenário atual, tornaram-se elementos onipresentes em nossa vida cotidiana, desempenhando papéis multifacetados que vão além da mera comunicação. O uso destes equipamentos tecnológicos tem sido disseminados em todos os níveis da sociedade de maneira significativa desde o final do século XX.

Entretanto é imperativo reconhecer que essa inovação não é livre de consequências imprevisíveis, tais como a crescente dependência digital, a propagação de desinformação e a ameaça à privacidade individual.

Na busca de uma perspectiva teórico-crítica sobre o uso das tecnologias em nossa sociedade, autores como \textcite{assange_2013-1,desmurget_fabrica_2021}, tem chamado a atenção para o fato de que uma parcela expressiva da população, abrangendo diversas faixas etárias, ao fazerem uso das redes e dispositivos móveis, como \textit{smartphones}, \textit{tablets}, \textit{notebooks}, \textit{laptops}, entre outros, submetem-se, de um lado à expropriação, pelo mercado e governos, do uso de dados e informações sobre a vida privada e, de outro, expondo-se aos perigos relacionados ao uso excessivo de tais dispositivos, como a erosão do aparato cognitivo, cujas consequências, segundo um prognóstico mais pessimista, ameaçam a própria perpetuação da espécie humana da terra.

É importante destacar o papel do aparelho celular nesse contexto, que passa a fazer parte da vida cotidiana, problematiza a dualidade intrínseca desses aparelhos, que não apenas facilitam a conexão entre indivíduos, mas também operam como instrumentos de vigilância, extração de dados da vida privada, além de produzirem efeitos desastrosos à economia da atenção e do aparato cognitivo, especialmente de crianças em fase de desenvolvimento.

Portanto a reflexão parte da problemática central que reside na dualidade desses dispositivos, que, além de facilitarem a comunicação, atuam também como ferramentas de vigilância, suscitando preocupações diversas, tanto do ponto de vista social quanto educacional. Acerca desta tensão, entre o aspecto positivo e negativo das tecnologias na vida contemporânea, \textcite{bannell_uma_2017} desafia-nos a refletir sobre as seguintes questões:

\begin{quote}
    Vivenciar o mundo, no sentido mais pleno dessa palavra, em todas as formas que pode tomar, incluindo estética e ética, significa ampliar as muitas maneiras de ser no mundo. Significa, também, preservar um mundo no qual se tenha experiências. A Tecnologia Educacional ajuda a ampliar a experiência humana ou a reduzi-la? Ajuda a preservar o mundo e reduzir a desigualdade educacional e social, ou aumenta essa desigualdade? \cite[p. 49]{bannell_uma_2017}.
\end{quote}


Tais indagações, além da emergência quanto aos desafios impostos pela Inteligência Artificial, são urgentes na sociedade contemporânea. Ao refletirmos sobre as implicações das tecnologias, devemos ter em mente que, pelo caráter não neutral das mesmas, delas se pode obter tanto benefícios quanto malefícios, podem, como aprendemos com as lições da história, favorecer tanto a construção de aviões quanto de armas nucleares. Em face desse cenário, como argumenta o filósofo chinês Yuk Hui: “diante da doença da cultura global, temos uma necessidade urgente por reformas guiadas por novos pensamentos e novas estruturas que permitam que nos libertemos daquilo que foi imposto e ignorado pela filosofia” \cite[p. 212]{hui_tecnodiversidade_2020}.

Considerando tais aspectos, o objetivo central deste artigo consiste em analisar o sentido dual dos dispositivos tecnológicos aplicados à educação, bem como lançar luz sobre os aspectos problemáticos, em que pese a tomada de decisões acerca de tais usos. Tal estudo enquadra-se na categoria de pesquisa teórica e ensaística já que busca, por meio do manejo conceitual, avançar na compreensão acerca da dualidade dos dispositivos móveis na sociedade contemporânea, dando importância especial ao campo educacional.

Este artigo apresenta a seguinte estrutura: inicialmente, explora-se a influência resultante da interseção entre monitoramento e comunicação em dispositivos móveis na esfera da privacidade individual, analisando as complexidades e implicações desse entrelaçamento. Em seguida, a reflexão direciona-se para a inserção das tecnologias no ambiente escolar, com o objetivo de investigar as possibilidades e limites da integração de dispositivos móveis como instrumento de aquisição de conhecimento na escola. 

\section{A educação no contexto das comunicações e das tecnologias digitais}\label{sec-normas}
É necessário observar que na sociedade atual, o uso de aparelhos que estão conectados à internet permeiam quase todas as esferas da vida cotidiana, desempenhando um papel de protagonismo na comunicação, no acesso à informação, na vida doméstica e, sobretudo, na educação. O ponto ao qual não podemos deixar de apontar é o vínculo histórico das tecnologias digitais à indústria cultural e os efeitos que tal vínculo produz na reprodução das estruturas sociais, cujo pano de fundo, não podemos omitir, diz respeito aos estágios avançados do capitalismo neoliberal. A “cultura digital”, neste contexto, exerce forte poder ideológico na conformação das massas em correspondência ao poder econômico, demandando uma reflexão profunda acerca dos limites que ela impõe. Neste sentido, e considerando o contexto educacional, \textcite{maar_educacao_2014} observa:

\begin{quote}
    A educação para a resistência e a contradição deve ter como objetivo atual facultar uma tal experiência. É a experiência da contraposição entrepráticas “convocadas pela rede” e práticas avaliadas, discutidas e combinadas em interação viva e concreta das pessoas. Não se trata de recusar a rede social, mas criticar o modo irrefletido de assimilação de suas práticas. É a experiência do divórcio entre o tempo da rede e o tempo da experiência. Essa é a educação para a contradição e resistência à homogeneização mercantil que se requer \cite[p. 350]{maar_educacao_2014}.
\end{quote}


Ao utilizar aplicativos, redes sociais e serviços online, os usuários de modo inadvertido, compartilham informações sensíveis, contribuindo para a construção de perfis detalhados e por vezes, invasivos, que colocam suas informações pessoais e localização sob um escopo potencialmente extenso e em alguns casos, fora do controle consciente do próprio usuário. No âmbito educacional, o problema pode ser ainda mais sério, já que o uso de celular em sala de aula, descontadas as exceções exitosas de seu uso, pode servir como instrumentos de controle e monitoramento da atividade docente, além de favorecer a reprodução de discursos de ódio e racismo, violência entre alunos e professores, estimular o \textit{bullying}, dentre outras situações perniciosas ao ambiente escolar; em uma palavra, o celular, tem a capacidade de atuar como a uma lente de aumento dos conflitos escolares e das relações interpessoais por meio do potencial efêmero e ideológico das redes sociais. O celular, especialmente no espaço escolar, serve como uma ferramenta hiperbólica das relações sociais, atuando especialmente na captura e publicização das emoções individuais.

Não é por acaso o fundador do WikiLeaks\footnote{Organização transnacional sem fins lucrativos, com sede na Suécia, que tem como objetivo a publicação de assuntos sensíveis a partir de fonte anônimas, documentos e informações confidenciais vazadas de governos ou empresas.}, Julian Assange, conhecido por suas análises críticas sobre vigilância e segurança na era da informação tem insistido sobre o perigo do uso sem prudência de dispositivos móveis conectados à internet. Em suas palavras: “A internet, nossa maior ferramenta de emancipação, está sendo transformada no mais perigoso facilitador do totalitarismo que já vimos. A internet é uma ameaça à civilização humana” \cite[p. 21]{assange_2013-1}. Entretanto, revestida de uma aura contagiante de liberdade – conceito central da democracia – a internet, explorando sua poderosa capacidade de penetração na vida contemporânea, condiciona a totalidade da vida dos sujeitos ao seu uso coercitivo, indispensável, em todas as esferas, para a própria sobrevivência. Como discutiu um dos autores deste artigo em outro texto:

\begin{quote}
    A propalada “transparência” da vida contemporânea, fica claro, é apenas um conceito aplicado à população, nunca ao poder. E possui um sentido óbvio de espoliação da vida em todos os seus ângulos. Aos poucos, vamos compreendendo que a parafernália tecnológica que circunda nossas vidas, para além dos benefícios da praticidade e conforto, traz consigo a contrapartida de entrega total de nossas vidas por meio dos dados. Deste modo, estamos todos constrangidos a ser o doutor Fausto do capitalismo tardio: vendemos nossa alma por informação/conhecimento; entregamos nossa subjetividade pela autopromoção das redes sociais que chega via comentário e curtidas \cite{quiroga_o_2013}.
\end{quote}


A internet, como alerta \textcite{assange_2013-1}, que inicialmente foi considerada como a mais significativa ferramenta de emancipação, por sua capacidade de democratizar o acesso à informação e promover a interconectividade global, encontra-se em uma trajetória que suscita preocupações quanto ao seu potencial transformação em um facilitador extremamente perigoso e totalitário. Este fenômeno emergente incita reflexões acerca do impacto da internet na civilização humana, suscitando inquietações acerca de sua influência crescente na configuração sociopolítica global e, por extensão, na instituição escolar.
 
Com efeito, é crucial que o problema acerca dos desdobramentos que delineiam a internet como uma possível ameaça à integridade e estabilidade da civilização humana, volte ao centro do debate, problematizando aspectos que podem tanto contribuir na construção de novos paradigmas de pensamento, quanto chamando a atenção para os riscos resultantes da ausência de uma reflexão crítica permanente. Isso levanta preocupações éticas sobre o alcance da vigilância e os limites da privacidade em um mundo onde a comunicação pessoal e a coleta de dados estão intrinsecamente entrelaçadas.

Portanto, a privacidade individual é afetada pela dualidade desses dispositivos, que servem como canais de interação, ao mesmo tempo em que funcionam como ferramentas de coleta massiva de dados. A transparência limitada nas práticas de coleta e compartilhamento de dados por parte das empresas de tecnologia torna mais grave essa problemática, tornando crucial uma reflexão sobre as políticas de privacidade e segurança. Sendo que as trocas de mensagens, as chamadas telefônicas e até mesmo as atividades cotidianas registradas por sensores, surgem como objetos suscetíveis à vigilância em virtude da expansão das tecnologias de comunicação e monitoramento.

Este fenômeno reflete uma mudança significativa no panorama da privacidade individual, demandando uma análise crítica das implicações sociais, éticas e legais associadas à coleta e uso dessas informações. A compreensão dessas dinâmicas é fundamental para a formulação de políticas e estratégias que protejam os direitos individuais e promovam uma convivência harmoniosa entre avanços tecnológicos e preservação da liberdade e autonomia dos cidadãos. Isso gera um dilema, onde a praticidade da comunicação móvel é contrabalançada pela potencial exposição da esfera privada.
 
\begin{quote}
    A vigilância é muito mais óbvia atualmente… [...] E ela tem sido muito mais totalizadora agora, porque as pessoas divulgam suas ideias políticas, suas comunicações familiares e suas amizades na internet. Então a situação não inclui apenas uma maior vigilância das comunicações em relação ao que existia antes, mas também ao fato de que atualmente temos muito mais comunicação. E não é só uma questão do maior volume das comunicações, mas também de uma proliferação dos tipos de comunicação. Todos esses novos tipos de comunicação que antes eram privados agora são interceptados em massa \cite[p. 36]{assange_2013-1}.
\end{quote}

É interessante observar o aumento expressivo de produções científicas sobre cultura digital em relação a sua área de filiação epistemológica. Em estudo recente, \textcite{pereira_fas_2023}, destacaram o desnível no volume de estudos voltados à cultura digital, com uma superioridade expressiva na área da Comunicação em relação a área da Educação. Tal aspecto é essencial sobretudo se pensarmos nos efeitos que os diferentes tipos de comunicação tem exercido na sociedade contemporânea. Deste ponto de vista, tendo se tornado a comunicação a característica central da escola, fica a provocação para que novas reflexões possam contribuir na explicação e formulação de respostas aos desafios que se impõe à educação, sobretudo ao considerarmos a dimensão ideológica que perpassa os canais da comunicação em rede. Ao desvelar as complexidades dessa dinâmica, podemos estabelecer bases para a discussão de regulamentações mais eficazes, práticas transparentes e uma conscientização mais ampla sobre a preservação da privacidade em um mundo cada vez mais conectado.


\section{O que sobra às novas gerações ou as consequências do uso compulsivo das tecnologias}\label{sec-conduta}
O neurocientista francês Michel Desmurget, em seus estudos, tem alertado sobre o que ele chamou em, tom de provocação, de geração de “cretinos digitais”, isto é, uma geração que pela primeira vez tem um QI menor que a geração anterior. O neurocientista expõe criticamente os efeitos da exposição excessiva às telas e dispositivos digitais, especialmente em crianças e jovens. Ele destaca como a constante interação com dispositivos eletrônicos pode resultar em perdas cognitivas, prejuízos à saúde mental e potencial vulnerabilidade da privacidade individual. Sua pesquisa aponta que “as ferramentas digitais aqui consideradas afetam os quatro pilares constitutivos de nossa identidade: o cognitivo, o emocional, o social e o sanitário” \cite[p. 58]{desmurget_fabrica_2021}.

\begin{quote}
    [...] a questão do impacto da tecnologia digital nada tem de trivial, uma vez que a complexidade e a interatividade dos canais funcionais envolvidos favorecem a ocultação dos impactos produzidos. Mas isso não é tudo. O caso se complica ainda mais quando é levada em conta a existência de possíveis “fatores dissimulados”, que agem secretamente, sem considerar os saberes estabelecidos \cite[p. 60]{desmurget_fabrica_2021}.
\end{quote}


Deste modo, os impactos dos dispositivos móveis, vão além de uma simples distração e passam a ser um modulador de personalidades diversas ao mesmo tempo em que monitoram initerruptamente a privacidade individual dos usuários desses aparelhos.  Segundo \textcite{desmurget_fabrica_2021}, o uso compulsivo destes dispositivos afeta não apenas a vigilância, mas também viabiliza uma inesgotável autopromoção exacerbada das plataformas digitais, que apresentam conteúdos rasos predominantemente voltados ao entretenimento. O constante monitoramento das interações online contribui para a criação de um ambiente em que a privacidade se torna uma mercadoria valiosa, negociada em troca de serviços aparentemente gratuitos. Nas palavras do autor:

\begin{quote}
    O \textit{smartphone} nos segue o tempo todo, sem fraquejar nem nos dar trégua. Ele é o graal dos sugadores de cérebros. O derradeiro cavalo de Tróia de nosso embrutecimento cerebral. Quanto mais os aplicativos se tornam “inteligentes” mais eles substituem nossa reflexão e mais nos ajudam a nos tornar idiotas \cite[p. 68]{desmurget_fabrica_2021}.
\end{quote}

Ao levar em consideração a relevância dessas reflexões, esta pesquisa se apoia na perspectiva de autores como \textcite{assange_2013-1,desmurget_fabrica_2021,quiroga_o_2013}, entre outros, e oferece uma oportunidade para aprofundar a compreensão sobre o tema em questão. Considerando que a integração entre comunicação e monitoramento em dispositivos móveis não só compromete a privacidade individual, mas também influencia a construção de identidades e as relações sociais contemporâneas.


\section{A incorporação das tecnologias no espaço escolar: por quê e a serviço de quem?}\label{sec-fmt-manuscrito}
Em um contexto de constante evolução, torna-se importante investigar como a incorporação das tecnologias da informação e comunicação (TICs) pode ou não desempenhar um papel fundamental na ampliação das possibilidades de aprendizagem e na preparação dos estudantes para os desafios do século XXI. A incorporação dessas tecnologias no ambiente escolar não apenas representa um desafio, mas também uma oportunidade para a transformação do processo educacional.
A Base Nacional Comum Curricular (BNCC) do Brasil, em sua quinta competência geral da educação básica, destaca a importância da competência digital como parte integrante da formação do estudante. Destaca que a habilidade de utilizar, compreender e criar tecnologias deve ser desenvolvida de forma articulada com os demais componentes curriculares, promovendo uma educação mais alinhada às demandas da sociedade contemporânea. Segundo o documento, tal incorporação diz respeito a:

\begin{quote}
    Compreender, utilizar e criar tecnologias digitais de informação e comunicação de forma crítica, significativa, reflexiva e ética nas diversas práticas sociais (incluindo as escolares) para se comunicar, acessar e disseminar informações, produzir conhecimentos, resolver problemas e exercer protagonismo e autoria na vida pessoal e coletiva \cite[p. 11]{brasil_base_2018}.
\end{quote}

De acordo com a BNCC, o uso das TICs é considerado uma competência transversal, ou seja, uma habilidade a ser desenvolvida de forma integrada nas diversas áreas do conhecimento. A BNCC enfatiza a importância de incorporar as TICs no processo educativo, reconhecendo que essas ferramentas desempenham um papel significativo na formação dos estudantes para a cidadania e para o mundo do trabalho contemporâneo.

O documento ainda destaca a necessidade de desenvolver nos alunos habilidades relacionadas à pesquisa, seleção, análise e comunicação de informações por meio das TICs. Além disso, incentiva o uso dessas tecnologias como recursos pedagógicos para enriquecer práticas educativas, promover a colaboração, ampliar o acesso ao conhecimento e desenvolver o pensamento crítico e criativo. A integração das tecnologias na educação, conforme preconizado pela BNCC, supõe uma aprendizagem mais dinâmica e alinhada às demandas contemporâneas, preparando-os para lidar de forma crítica e responsável com as informações digitais.

\begin{quote}
    […] Contudo, também é imprescindível que a escola compreenda e incorpore mais as novas linguagens e seus modos de funcionamento, desvendando possibilidades de comunicação (e também de manipulação), e que eduque para usos mais democráticos das tecnologias e para uma participação mais consciente na cultura digital [...] \cite[p. 61]{brasil_base_2018}.
\end{quote}

A utilização das tecnologias no espaço escolar transcende a mera introdução de dispositivos eletrônicos nas salas de aula, envolvendo uma mudança mais profunda na abordagem pedagógica. A incorporação dessas ferramentas visa não apenas enriquecer o conteúdo, mas também promover uma aprendizagem mais ativa, participativa e personalizada. Dessa forma, a tecnologia se torna um instrumento para potencializar o desenvolvimento de habilidades essenciais, como o pensamento crítico, a resolução de problemas e a colaboração.

\begin{quote}
    Compreender e utilizar tecnologias digitais de informação e comunicação de forma crítica, significativa, reflexiva e ética nas diversas práticas sociais (incluindo as escolares), para se comunicar por meio das diferentes linguagens e mídias, produzir conhecimentos, resolver problemas e desenvolver projetos autorais e coletivos \cite[p. 65]{brasil_base_2018}.
\end{quote}

No entanto, o que se observa no texto é um sentido idealizado das tecnologias, uma concepção romântica – não sem intenções ocultas ligadas ao poder econômico – em que se negligenciam as consequências e perigos inerentes sobre o uso das TICs ou, de um modo ainda mais responsável, destacando suas consequências e riscos. E mesmo quando adverte sobre uma educação voltada para usos mais democráticos, não diz como fazer – o que seria imprescindível, considerando a elevada carga ideológica subjacente a comunicação em rede – restringindo-se a uma formalidade que termina por se dissolver em figura retórica.

Dessa forma, a incorporação das tecnologias no espaço escolar diz respeito a um processo dinâmico que visa preparar os alunos para um mundo cada vez mais digital e interconectado em total convergência aos interesses do poder econômico do neoliberalismo. Todavia, isto não quer dizer que, de maneira consciente e inclusiva, essa integração pode catalisar uma transformação positiva no ambiente educacional, promovendo uma educação mais alinhada com as necessidades e desafios contemporâneos. A questão, aqui, é a de que tal desafio esteja vivo, em primeiro lugar, no debate educacional, e não restrito ao campo da comunicação; em segundo lugar, que seja um debate permanente, e não localizado. Aqui, deparamo-nos com o problema desenvolvido por \textcite{hui_tecnodiversidade_2020}, quanto a “esgotar a fim de não esgotar”: “Tese: existe uma carta com um número limitado de símbolos plenamente capazes de expressar o mundo; antítese: o mundo não pode ser plenamente expressado pela escrita, já que, enquanto o mundo é infinito, a escrita é em si finita” \cite[p. 153]{hui_tecnodiversidade_2020}. É somente pelo propósito da educação destinar-se às novas gerações que as formas como e por quê se educa devem passar por um processo permanente de crítica. Consequentemente, abordagens que parecem oferecer respostas acabadas ou fórmulas definitivas colocam em xeque a própria potência do que entendemos como “novas gerações”, justamente porque aqui, parafraseando o problema levantado pelo filósofo chinês Yuk Hui, as novas gerações não podem ser plenamente contempladas pela mera apropriação de ferramentas tecnológicas, já que a escola se liga, em essência, aos saberes herdados historicamente, enquanto as novas gerações ligam-se à construção do futuro. É no hiato entre passado e futuro, portanto, que a escola deve ser pensada em um movimento de crítica permanente, onde nem sempre a inovação deve ocupar o papel de protagonismo\footnote{Exemplos disso tem ocorrido no mundo inteiro, especialmente em países como a Finlândia – um dos países com melhores sistemas educacionais do mundo – que tem recuperado a caligrafia como prática pedagógica cotidiana. Outras práticas que pareciam extintas, mas que, mesmo timidamente tem voltado ao debate (no Brasil essa é uma necessidade urgente), são o retorno do ditado, aos exercícios de memorização e a cópia. Ver mais em \textcite{desmurget_fabrica_2021,turcke_hiperativos!_2016}.}.

\section{Dispositivos móveis e educação: a aprendizagem em face da coerção ideológica}\label{sec-formato}
Antes de iniciamos esta reflexão, é fundamental que se reconheça o que é um dispositivo e sua função social, que consiste, segundo \textcite{agamben_o_2010}, em “uma estratégia concreta e se inscreve sempre numa relação de poder” e que resulta, finalmente, do cruzamento entre relações de poder e de saber \cite[p. 29]{agamben_o_2010}. De acordo com Agamben, o que os dispositivos têm em comum é o vínculo com a \textit{oikonomia}\footnote{Tal conceito é amplamente debatido na filosofia de Giorgio Agamben, e implica a administração da casa (do grego \textit{oikos}), embora possa ser entendido também de modo geral, como governo, gestão, \textit{management}.}, isto é, com saberes instituídos que tem como objetivo o governo, o controle, a gestão das cosias. Neste sentido, para Agamben, dispositivo é “qualquer coisa que tenha de algum modo a capacidade de capturar, de orientar, de determinar, de interceptar, modelar, controlar e assegurar os gestos, as condutas, as opiniões e os discursos dos seres viventes” \cite[p. 40]{agamben_o_2010}. Segundo o filosofo italiano, pelo fato dos dispositivos possuírem o poder da captura das subjetividades é que apenas sugerir que se faça o “uso correto” de tais dispositivos não passa de mera intenção ingênua. Segundo ele:

\begin{quote}
    [...] a estratégia que devemos adotar no nosso corpo a corpo com os dispositivos não pode ser simples, já que se trata de liberar o que foi capturado e separado por meio dos dispositivos e restituí-los a um possível uso comum \cite[p. 44]{agamben_o_2010}.
\end{quote}

E, portanto:
\begin{quote}
    Aquele que se deixa capturar no dispositivo “telefone celular”, qualquer que seja a intensidade do desejo que o impulsionou, não adquire, por isso, uma nova subjetividade, mas somente um número pelo qual pode ser, eventualmente, controlado; o espectador que passa suas noites diante da televisão recebe em troca da sua dessubjetivação apenas a máscara frustrante do \textit{zappeur} ou a inclusão do cálculo de um índice de audiência \cite[p. 48]{agamben_o_2010}.
\end{quote}

Neste sentido, a reflexão acerca dos usos de dispositivos tecnológicos deve presumir elaborações problemáticas como estas, já que, sob o efeito coercitivo pelo qual os dispositivos são incorporados à vida diária e ao ambiente escolar, a reflexão sobre a melhor forma de usá-los, para além do sentido fatalista apresentado pelo filósofo, deve estar sempre presente. Se há algo a ser feito no âmbito educacional acerca das tecnologias, considerando-se seus aspectos vinculantes (cognitivos, ideológicos, políticos, ideológicos), é o esforço na construção de uma “contra captura”, isto é, e insistimos, de um exercício \textit{permanente} de crítica sobretudo ao caráter ideológico presente na captura dos dispositivos e, mais profundamente, na edificação de uma “racionalidade democrática” \cite{feenberg_tecnologia_2018}.

A educação, possui um papel crucial não apenas na mediação de conhecimentos, mas na formação de cidadãos capazes de contribuir significativamente para a renovação constante de nossa sociedade. Por meio do desenvolvimento de competências cognitivas, sociais, emocionais e éticas, espera-se que a educação proporcione aos indivíduos as ferramentas necessárias para a compreensão crítica do mundo ao seu redor.

Na contemporaneidade, a presença ubíqua dos dispositivos móveis tem transformado radicalmente a dinâmica educacional. A interseção entre a tecnologia e a educação torna-se cada vez mais evidente, destacando-se presença dos \textit{smartphones}, \textit{tablets} e outros dispositivos como “ferramentas” no processo de aprendizagem. A influência desses dispositivos na educação e a integração tecnológica no ambiente escolar, impacta tanto educadores quanto alunos, o que nos remete ao pensamento de Freire, que sugere uma reflexão sobre o uso da tecnologia: “Depende de quem usa a favor de quê e de quem e para quê” \cite[p. 98]{freire_conversacao_1994}.

Logo, o crescente acesso a dispositivos móveis, aliado ao desenvolvimento de aplicativos educacionais inovadores, redefine a forma como o conhecimento é transmitido e assimilado. Professores e alunos estão imersos em um ambiente digital que oferece possibilidades ampliadas de aprendizagem, desde a personalização do ensino até a utilização de recursos multimídia. A mobilidade proporcionada por esses dispositivos amplia as fronteiras do espaço de aprendizagem, transformando salas de aula convencionais em ambientes dinâmicos e interativos.

Contudo, questões éticas relacionadas ao uso responsável, à privacidade dos alunos e à equidade no acesso às tecnologias surgem como pontos cruciais a serem considerados.

\begin{quote}
    Fazer com que os alunos se interessem, motivá-los, prender-lhes a atenção quando esta, o tempo todo é disputada pelas novidades tecnológicas (celulares, \textit{smartphones}, internet) é um dos desafios do professor dos dias atuais. Diante deste cenário, a relação entre professor e aluno não mais é representada pela oposição entre locutor e ouvinte. Essa concorrência, aliás, aparece como outro indicativo de mal-estar, sobretudo quando o público é composto por adolescentes (principais consumidores dessas tecnologias) \cite[p. 129]{quiroga_o_2013}.
\end{quote}

Com efeito, o olhar quanto a incorporação de dispositivos móveis nas escolas deve ser carregado de criticidade, portanto, deve colocar-se à margem do processo passivo de assimilação. \textcite{quiroga_o_2013}, destaca o desconforto e a angústia que o uso inadequado desses dispositivos pode gerar no ambiente escolar, especialmente para os professores. Além disso, ele aponta para a dificuldade ou resistência em adaptar o uso dessas tecnologias de forma eficaz, levando em conta o que é realmente aplicável em sala de aula, em contraste com o que pode se tornar apenas uma distração ou improvisação sem propósito definido.

\begin{quote}
    O professor, nesse sentido, encontra-se numa situação completamente oposta ao que tivera durante a sua formação, sobretudo no que diz respeito à condição ideal para o desenvolvimento do trabalho. O esvaziamento simbólico do espaço e dos materiais específicos para o desenvolvimento da prática, colocam o profissional na condição do puro improviso e responsabilidade. O sentimento de abandono e desprezo se materializam em expressões de angustia e mal-estar. Mas, se por um lado a falta de estrutura desmotiva o professor, quando ela existe, então, torna-se outro problema quando o professor encontra dificuldades em adequar-se a elas, como no caso das tecnologias. O desnível entre as reais necessidades do professor e a adesão de tecnologias (muitas vezes vinculadas a contratos e licitações por razões de interesse político, ou por modismos que dispensam uma reflexão sobre a sua real aplicabilidade) criam um desconforto, ou mesmo confundem ao invés de facilitar a prática docente \cite[p. 134-135]{quiroga_o_2013}.
\end{quote}

Portanto, a adaptação curricular para incluir o uso de dispositivos móveis, demanda uma revisão cuidadosa das práticas pedagógicas existentes, bem como a inserção de estratégias com engajamento que aproveitem a natureza interativa e colaborativa dos dispositivos que se apresentam essenciais nesse novo contexto escolar, onde os equipamentos tecnológicos fazem parte do cotidiano. Segundo \textcite{ferreira_educacao_2017}:

\begin{quote}
    Qualquer discussão na área, portanto, precisa reconhecer a natureza inerentemente política da educação e tecnologia. Vista por esse prisma, então, muitos dos questionamentos mais importantes em torno da educação na era digital são fundamentalmente questões políticas que devem sempre ser levantadas acerca da educação e da sociedade – ou seja, questões acerca do que é a educação e sobre o que ela deveria ser \cite[p. 89]{ferreira_educacao_2017}.
\end{quote}

Discussões como essa estão surgindo em várias regiões do mundo, como evidenciado pelo tema abordado no Relatório de Monitoramento Global da Educação intitulado "A Tecnologia na Educação: Uma Ferramenta a Serviço de Quem?", lançado no ano de 2023. Este evento aconteceu em Montevidéu, Uruguai, e foi organizado pela Organização das Nações Unidas para a Educação, a Ciência e a Cultura (UNESCO), juntamente com o Ministério da Educação e Cultura do Uruguai e a Fundação Ceibal, contando com a participação de 18 ministros da Educação de diversas nações ao redor do mundo.

O relatório procura colocar as reais necessidades dos alunos em primeiro lugar, mostrando que estudos que avaliam experiências de estudantes de diversas idades e aplicadas em contextos e realidades sociais diferentes são insuficientes e que:

\begin{quote}
    [...] embora haja muita pesquisa em geral sobre tecnologia educacional, a quantidade de estudos feitos sobre aplicações e contextos específicos é insuficiente, o que torna difícil provar que uma tecnologia específica enriquece um tipo específico de aprendizagem \cite[p. 11-12]{unesco_resumo_2023}.
\end{quote}

Na página de notícias da Campanha Nacional do Direito à Educação, foi publicado dia 27 julho de 2023, uma matéria com o título “UNESCO pede urgentemente o uso adequado da tecnologia na educação”. Nele podemos encontrar a declaração da então diretora geral da UNESCO, Audrey Azoulay, alertando sobre a importância de colocar as necessidades dos alunos em primeiro lugar, enquanto ressalta que as conexões online não podem substituir a interação humanas. Essa afirmação reflete uma preocupação com o equilíbrio entre o uso da tecnologia na educação e a preservação da qualidade das interações sociais e educacionais.

O relatório propõe quatro questões sobre as quais os formuladores de políticas e as partes interessadas na educação devem refletir à medida que as tecnologias na educação são implantadas. São elas: É adequado? É equitativa? É expansível? É sustentável?

Ao questionar: é \textit{adequado}? O relatório cita através de evidências que se a tecnologia for usada excessivamente ou sem o acompanhamento de um professor qualificado os benefícios do aprendizado não existem ou desaparecem. Destaca que um quarto dos países globalmente proíbem o uso de \textit{smartphones} na escola, fundamentados na percepção de que tais dispositivos servem primariamente como fonte de distração para os alunos.

Quando questiona: é \textit{equitativa}? O relatório de Monitoramento Global da Educação, infere que [...] o direito à educação é cada vez mais sinônimo de direito a uma conectividade significativa, embora metade das escolas primárias da América Latina ainda não tenha conexão com a Internet \cite[p. 2]{campanha_nacional_do_direito_a_educacao_nota_nodate}.

Na questão: é \textit{expansível}? Infere que mais do que nunca é necessário levantar evidências confiáveis e rigorosas e imparciais sobre o valor agregado da tecnologia na aprendizagem. E alerta que quando essas evidências podem ser tendenciosas quando são obtidas apenas das próprias empresas da tecnologia. Isso sugere a importância crescente de obter evidências confiáveis, rigorosas e imparciais sobre o valor agregado da tecnologia na educação. Adverte-se que tais evidências podem estar sujeitas a viés quando são obtidas exclusivamente das próprias empresas de tecnologia. Esta advertência ressalta a necessidade de uma abordagem crítica e transparente na avaliação do papel da tecnologia no processo educacional, garantindo que as conclusões sejam baseadas em dados objetivos e independentes.

\begin{quote}
    Muitos países ignoram os custos de longo prazo das aquisições de tecnologia, e o mercado de EdTech está se expandindo enquanto as necessidades de educação básica permanecem não atendidas. O custo de mudar para o aprendizado digital básico em países de baixa renda e conectar todas as escolas à internet em países de renda média baixa aumentaria em 50\% a lacuna de financiamento atual para atingir as metas nacionais do ODS 4. Uma transformação digital completa da educação com a internet, a conectividade em escolas e residências custaria mais de US\$ 1 bilhão por dia apenas para funcionar nos países mais pobres e US\$ 140 bilhões por ano nos países de renda média alta \cite[p. 2]{campanha_nacional_do_direito_a_educacao_nota_nodate}.
\end{quote}

Por fim, na questão: é \textit{sustentável}? O relatório aborda uma série de considerações pertinentes, incluindo a alfabetização digital e o desenvolvimento do pensamento crítico, aspectos que, como vimos até aqui podem ser problemáticos. Além disso, o documento evidencia que, além dessas competências, a alfabetização básica não deve ser subestimada, pois desempenha um papel crucial na aplicação eficaz da tecnologia digital. Por exemplo, evidências sugerem que alunos com habilidades sólidas de leitura estão menos suscetíveis a serem enganados ou manipulados online. Além disso, destaca a importância do treinamento adequado para o corpo docente nesse contexto; no entanto, é preocupante observar que menos de um terço dos países na América Latina e no Caribe não possuem padrões estabelecidos para o desenvolvimento das habilidades dos professores em Tecnologia da Informação e Comunicação (TIC). Esta constatação enfatiza a necessidade premente de investimentos em programas de capacitação e desenvolvimento profissional para os educadores, a fim de garantir a utilização eficaz e sustentável da tecnologia no ambiente educacional da região.

Dados como esses do Relatório de Monitoramento Global da Educação, acrescenta ainda mais importância de enfatizar a necessidade de uma abordagem equilibrada no uso de dispositivos móveis na educação, reconhecendo tanto as oportunidades transformadoras quanto os desafios inerentes a essa integração.
Desta maneira é possível perceber como as mudanças que ocorrem com o uso destes dispositivos móveis em nosso cotidiano, podem exercer uma influência sobre o comportamento das pessoas, provocando a necessidade de estar constantemente conectado e incorporar o acesso desregrado a eles, como uma parte inseparável do seu dia a dia.

\section{Conclusão}\label{sec-modelo}
Os dispositivos móveis, como discutido ao longo deste artigo, tornaram-se elementos indispensáveis na sociedade contemporânea, exercendo diversas funções além da simples comunicação. A rápida disseminação desses equipamentos ao longo das últimas décadas alterou substancialmente as formas de informação, comunicação e a realização de atividades cotidianas. A convergência entre comunicação e monitoramento, conforme destacado por Julian Assange, delineia um novo estilo de vida, no qual os \textit{smartphones}, \textit{tablets} e outros dispositivos não apenas facilitam a conexão entre indivíduos, mas também operam como instrumentos de vigilância e produzem outros tipos de comunicação. Resta saber de que modo este paradigma se coaduna no ambiente escolar. Será a incorporação tecnológica, de fato, uma demanda urgente, irrefreável, à qual estamos todos, para o bem ou para o mal, fadados a consentir? Sob quais argumentos? A serviço de quem? Que preço haveremos de pagar pela incorporação fortuita das tecnologias e dispositivos móveis ao ambiente escolar sem a pré-condição de um debate rigoroso e essencialmente crítico a respeito?

A dualidade destes aparelhos revela a necessidade de uma análise crítica de seu impacto na educação e na sociedade em geral. Na presente análise procuramos identificar como esses dispositivos atuam como facilitadores ou obstáculos no ambiente escolar. Enquanto facilitadores, eles oferecem (ao menos em potencial e desde que de modo crítico) inúmeras oportunidades de aprendizado, por outro lado, os prejuízos se manifestam em consequências diretas na constituição da própria cognição humana, atuando como forte instrumento de dessubjetivação e de controle.

Conclui-se, portanto, que a relação entre dispositivos móveis e educação é complexa e multifacetada. E que a discussão abordada nesse artigo não se esgota aqui, face a integração das tecnologias no ambiente educacional demandar uma abordagem cuidadosa e reflexiva, levando em consideração não apenas os benefícios evidentes, mas também os desafios e dilemas éticos associados. O entendimento crítico dessa dualidade é essencial para orientar políticas educacionais e práticas pedagógicas que maximizam os benefícios dos dispositivos móveis, ao mesmo tempo em que minimizam seus impactos negativos.

\printbibliography\label{sec-bib}
%conceptualization,datacuration,formalanalysis,funding,investigation,methodology,projadm,resources,software,supervision,validation,visualization,writing,review
\begin{contributors}[sec-contributors]
\authorcontribution{Fernando Lionel Quiroga}[conceptualization,investigation,writing,review]
\authorcontribution{Rosângela de Bessa}[writing,review]
\end{contributors}
\end{document}
