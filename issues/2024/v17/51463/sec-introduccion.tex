\section{Introducción}\label{sec-introducción}

Es innegable, a estas alturas, la importancia que la tecnología ha ido
adquiriendo en las últimas décadas en el desarrollo de nuestra vida
cotidiana. De este modo, cada día la tecnología, a través de diferentes
soportes, condicionan la manera en que la ciudadanía trabaja, se
relaciona y se forma \cite{gabarda2021}.

En este último campo, el educativo, se ha ido produciendo una
digitalización que ha impactado en diversas esferas: por un lado, los
centros educativos de cualquier etapa han ido modificando sus
instalaciones para incorporar diferentes dispositivos y equipamientos de
carácter tecnológico \cite{area2020}; por otro, la tecnología se ha
ido instalando en los currículos, convirtiendo a la competencia digital
en un aspecto clave de la formación de los ciudadanos \cite{hernandez2023}; y, por último, el diseño, implementación y evaluación
del acto formativo se ha visto enriquecido por la implementación de
diferentes metodologías mediadas por tecnología que han permitido
generar modos alternativos de concebir y desarrollar los procesos de
enseñanza y aprendizaje, así como a favorecer la atención a la
diversidad \cite{lopezgomez2022,romerorodrigo2021}.

Esta situación ha supuesto, en sí misma, un cambio de paradigma en el
modo de entender la educación, así como el papel de los agentes que
forman parte de ella. Además, este proceso de digitalización se ha visto
especialmente potenciado por la pandemia derivada de la COVID-19
\cite{marinsuelves2023}, que implicó el desencadenamiento de
procesos híbridos y no presenciales que, en el caso de la educación
superior, han permanecido durante más tiempo que en etapas anteriores.

Esta realidad ha generado mucho interés, tanto desde la esfera política,
como la comunidad científica por conocer diferentes aspectos sobre el
impacto de la implementación de la tecnología en el ámbito educativo
\cite{aguilar2023}, como cuál es el mejor modo de promover la
competencia digital del alumnado \cite{colomo2023}, qué formación
necesita el profesorado para atender de un modo efectivo su función \cite{mas2023} o qué impacto tiene la tecnología en el desarrollo de
competencias por parte del alumnado, siendo éste último el objeto de
estudio de la presente propuesta.