\section{Resultados}\label{sec-resultados}


A continuación, se recoge la exploración de las variables relacionadas
con los 28 estudios que son objeto de análisis de contenido (ver \Cref{tab-03}):

{
\setlength\LTleft{-1.4in}
\setlength\LTright{-1.4in}
\begin{footnotesize}
\begin{longtable}{p{3cm}p{1cm}p{2cm}p{4cm}p{2cm}p{4cm}}
%\begin{table}[htbp]
\caption{Análisis de contenido de los artículos.}
\label{tab-03}
%\begin{tabular}{p{3cm}p{1cm}p{2cm}p{4cm}p{2cm}p{4cm}}
\\
\toprule
Autor y año & Idioma y país & Muestra & Objetivos & Tecnología & Resultados \\
\midrule
\cite{perez2023} & Español España & Alumnos del Grado en Educación Primaria & Analizar el efecto de un juego formativo con realidad virtual en el alumnado universitario & Juegos & La aplicación de la tecnología ha favorecido tanto los resultados teórico-conceptuales como las 3 prácticas. \\
\cite{marques2023} & Español Chile & 41 estudiantes de Pedagogía & Analizar la percepción del alumnado respecto al desarrollo de un aprendizaje-servicio mediado por tecnología & No se detalla & El uso de la tecnología ha favorecido el trabajo autónomo, el trabajo en equipo, las competencias comunicativas y capacidad reflexiva. \\
\cite{fernandez2023b} & Español España & 77 estudiantes del Grado en Pedagogía & Conocer el nivel de conocimientos desarrollados por el estudiantado tras una experiencia formativa con realidad aumentada (RA) & Realidad Aumentada & La incorporación de la RA fomenta la motivación, favorece la incorporación de nuevas metodologías y potencia el desarrollo de conocimientos y destrezas tecnopedagógicas \\
\cite{lopeznoguero2023} & Español Nicaragua & 525 estudiantes de tres facultades & Analizar la percepción del alumnado sobre el uso de los Smartphones en los procesos de enseñanza-aprendizaje & Smartphone & La integración de esta tecnología ha favorecido el aprendizaje cooperativo, la gestión y organización
académica, un acceso rápido al contenido académico, una mayor interacción entre los discentes y entre ellos y el docente y una mejora de las competencias lingüísticas. Se destacan, asimismo, cuestiones negativas como la dependencia y la excesiva información que generan. \\
\cite{fernandez2023} & Español Argentina & 139 estudiantes de diversas titulaciones & Evaluar el impacto en el aprendizaje de diferentes Objetos Virtuales de Aprendizaje & Objetos Virtuales de Aprendizaje & Los resultados revelan el potencial innovador y motivador de los EVA. Consideran que mejoraron las habilidades de escucha y habla de los estudiantes. \\
\cite{area2023} & Español España & 136 estudiantes del Grado en Educación Primaria & Analizar el impacto de una experiencia en educación superior basada en el HyFlex (híbrido y flexible) & Aprendizaje híbrido & El alumnado valora positivamente la flexibilidad, la autonomía y la autorregulación del aprendizaje. Además, destacan las posibilidades de descubrimiento guiado y el aprendizaje activo, cuestiones que favorecen el desarrollo de metodologías innovadoras, competencias digitales y destrezas sociales y cívicas. \\
\cite{davila2023} & Español Ecuador & 565 estudiantes de Enfermería & Analizar el uso de la tecnología en la enseñanza de estadística & No se detalla & El alumnado reporta mejoras en cuanto a su actitud frente a la materia, una mejora en su conocimiento teórico y aplicado y una progresión en la competencia digital. \\
\cite{collazo2023} & Español Cuba & 78 estudiantes de Agronomía & Evaluar el impacto de los objetos de aprendizaje & Objetos de Aprendizaje & Los resultados apuntan a una mejora de las competencias y habilidades para resolver las actividades, el trabajo colaborativo y las estrategias para socializar el conocimiento \\
\cite{diazramirez2023} & Español Colombia & 97 estudiantes de Educación y Pedagogía & Explorar el impacto de las estrategias de aprendizaje móvil para el dominio del inglés & Aprendizaje móvil & Se constata el logro de las metas académicas y el enriquecimiento de su propio crecimiento profesional. \\
\cite{soncco2022} & Español Perú & 114 estudiantes de Lengua & Establecer la relación del aprendizaje móvil y las competencias del idioma inglés & Aprendizaje móvil & El aprendizaje móvil ha demostrado ser eficaz en el desarrollo de la expresión oral y escrita, la comprensión auditiva, lectora y audiovisual de la lengua. \\
\cite{villatoro2022} & Español España & 57 alumnos del primer ciclo del Grado de Educación Primaria & Evaluar el impacto de los itinerarios de aprendizaje en la formación del alumnado & No se detalla & Entre los aspectos positivos señalados destacan la evaluación continua, la libre elección de secuencias, el aprendizaje transferible a otros contextos, las tutorías y el seguimiento docente. En relación con los aspectos negativos, destacan la carga de trabajo o la preferencia por una evaluación no continuada. \\
\cite{nunez2022} & Español México & 73 estudiantes de Diseño para la Comunicación Gráfica y 59 de Derecho & Evaluar los procesos de comprensión de la lectura mediada por la tecnología & No se detalla & La lectura mediante las TIC presenta mayor diversidad y amplitud de medios y mejora la motivación. La comprensión de la lectura es similar a la de medios impresos \\
\cite{blasco2022} & Español España & 134 estudiantes de la carrera de Magisterio de Primaria y Diplomatura Profesional & Analizar el impacto del Flipped Classroom (FC) en el aprendizaje & Flipped Classroom & Se valora positivamente la retroalimentación durante el proceso de aprendizaje, el potencial de la metodología para la atención a la diversidad, la mejora de la comprensión y la individualización del aprendizaje \\
\cite{martinez2021} & Español España & 186 estudiantes y 4 profesores de titulaciones de Educación & Indagar sobre el efecto de la realidad aumentada en los procedimientos formativos & Realidad Aumentada & El estudio plantea ventajas de la RA como la atención educativa a todo el alumnado, el fomento del intercambio de conocimientos y el desarrollo de habilidades digitales. Además, se destaca su potencial para el trabajo autónomo, la relación de conceptos, la cooperación, la imaginación y la integración de conocimientos \\
\cite{rodrigo2022} & Español España & 86 estudiantes de los Grados en Educación Primaria y Educación Infantil & Evaluar el uso de herramientas digitales y redes sociales para la producción y difusión de trabajos académicos & Redes sociales & Los resultados resaltan el componente motivador y el desarrollo de competencias transversales como el trabajo cooperativo en red o la competencia digital. \\
\cite{andrade2022} & Español Ecuador & 44 estudiantes del cuarto ciclo de la carrera de Enfermería & Analizar las perspectivas del alumnado sobre el uso del aula invertida como una metodología activa en el aprendizaje & Aula Invertida & Destaca la motivación que genera la metodología y la potenciación de un rol activo en la generación de conocimiento. \\
\cite{ramos2022} & Español Ecuador & 98 estudiantes & Determinar la influencia del empleo de las redes académicas en la formación de la competencia profesional de los estudiantes & Redes académicas & Las redes académicas permiten socializar los resultados científicos y el nivel de formación de la competencia profesional de los estudiantes \\
\cite{cabero2022} & Español España & 102 estudiantes de Pedagogía & Comprobar los aprendizajes adquiridos mediante una experiencia formativa mediada por tecnología & No se detallan & Los resultados resaltan el desarrollo de competencias digitales y emprendedoras por parte del alumnado mediante el modelo
TAM \\
\cite{maldonadotorres2021}& Español Ecuador & 210 encuestas a estudiantes de administración de empresas de diferentes universidades & Determinar si los simuladores de negocios son una herramienta pedagógica & Simuladores & El alumnado destacó la posibilidad de poner en práctica sus conocimientos teóricos y
desarrollar competencias empresariales. En relación con los inconvenientes, desatacaron la falta de conocimientos técnicos por su parte y por la del docente, la escasez de tiempo y falta de instrucciones. \\
\cite{delacruz2021} & Español México & 50 estudiantes de psicología & Evaluar el impacto del uso de Zoom, Moodle y Facebook en el aprendizaje & Zoom, Moodle y Facebook & El análisis permitió corroborar una mejora de la autonomía en el aprendizaje, una mejor comprensión conceptual y una promoción del aprendizaje grupal. \\
\cite{lagossanmartin2022} & Español Chile & 50 estudiantes de Pedagogía en Educación Básica & Identificar el aporte del booktube como recurso pedagógico & Booktube & En el ámbito académico, el recurso facilita el desarrollo de habilidades de comunicación, promueve la expresión libre, permite practicar el lenguaje, favorece la comprensión lectora, facilita la internalización de un contenido y, permite analizar y comprender. Por otro lado, en el ámbito social, fomenta el compañerismo y la cooperación, el trabajo en equipo y sociabilización. En el ámbito emocional, ayuda a la autoestima, fomenta el desarrollo personal y permite profundizar aspectos del desarrollo emocional \\
\cite{cruz2021} & Colombia Español & 30 estudiantes pertenecientes al programa de diseño Industrial & Describir cómo el uso de la fabricación digital (FD) desarrolla competencias & Fabricación digital & La FD permite tener conocimientos tanto teóricos como aplicados. Además, dota al proceso educativo de libertad y flexibilidad. \\
\cite{quezadasarmiento2021} & Español España & 25 estudiantes de la titulación de Ciencias de la Computación & Evaluar la influencia de la Computación en la Nube en el proceso formativo. & Computación en la Nube & El uso de CN permite flexibilidad y adaptabilidad, ya que los estudiantes pueden seguir diferentes ritmos en su aprendizaje, mejorando sus competencias y habilidades educativas y profesionales. \\
\cite{higaldocajo2021} & Español Ecuador & 31 estudiantes de anatomía & Diseñar, implementar y evaluar una propuesta didáctica basada en la realidad aumentada (RA). & Realidad Aumentada & La RA despierta la motivación del alumnado por su fácil uso y la interacción que experimentan entre el contenido y los objetos virtuales, generando conocimiento con entretenimiento. Además, mejora del rendimiento académico. \\
\cite{gomez-gomez2021} & Español España & Alumnado de los Grados en Educación Infantil y Educación Social (n=66) & Analizar el impacto del Aprendizaje Servicio Solidario mediante escenarios virtuales en el aprendizaje & APS virtual & La modalidad híbrida permitió desarrollar competencias personales, profesionales y académicas, especialmente, su dimensión social. \\
\cite{astudillo2021} & Español México & 16 estudiantes de carreras técnicas & Explicar el proceso de enseñanza y aprendizaje llevado a cabo desde la virtualidad total & Videoconferencias y plataformas de educación a distancia & La tecnología no ha facilitado un aprendizaje significativo y un mejor entendimiento del tema, especialmente en alumnado de entornos vulnerables (brecha digital) \\
\cite{martinez2021} & España Español & Estudiantes de titulaciones de Educación & Valorar la utilización de la RA en los procesos formativos & Realidad Aumentada & Los resultados apuntan a la dificultad de uso de la RA, pero también su carácter motivador, atractivo y divertido. Concluyen que favorece la eflexión y la participación activa, el trabajo individual y grupal, la reflexión, la  reatividad y la innovación \\
\cite{cordero2021} & Español Costa Rica & 50 estudiantes & Evaluar el impacto de varias aplicaciones (Blogger, WordPress, Voki, PowToon y Canva) para fortalecer la habilidad de escritura en estudiantes de inglés como Lengua Extranjera (ILE) & APPs & Los hallazgos indicaron la utilidad de las aplicaciones para mejorar sus competencias de escritura por su motivación e interés, así como para mejorar su creatividad y capacidad de análisis, y para fortalecer el trabajo en equipo y su aprendizaje autónomo \\
\bottomrule
\source{Elaboración propia.}
%\end{tabular}
%\end{table}
\end{longtable}
\end{footnotesize}
}



Poniendo la atención en primer lugar a las variables de tipo
identificativo, cabe destacar que la producción científica en los tres
años se ha mantenido de manera estable. Así, hay 10 artículos publicados
en 2021, 9 en 2022 y también 9 en 2023.

En relación con el contexto geográfico donde se desarrollan las
investigaciones, destaca el caso de España, donde se contextualizan 11
de las 28 publicaciones. Destaca también Ecuador, que concentra 5 de las
investigaciones, México con 3 de ellas y Chile y Colombia con dos
propuestas cada una. Por último, hay también presencia de otros países
donde se desarrolla una investigación sobre el fenómeno de estudio:
Nicaragua, Argentina, Cuba, Perú y Costa Rica. Por otro lado,
independientemente del país, es reseñable que todas las investigaciones
se realizan en lengua española, siendo representativo el idioma oficial
en la publicación de los hallazgos.

Respecto a la autoría, la mayor parte de las propuestas (en concreto 23
de ellas) se realizan en coautoría (entre dos y siete autores), habiendo
solamente cuatro investigaciones firmadas por un único autor. Además, es
reseñable que, aunque hay una muestra muy diversa de orígenes y
filiaciones, hay algunos investigadores que son autores de más de una
propuesta como Julio Barroso-Osuna, Bárbara Fernández y Sandra
Martínez-Pérez.

Poniendo ahora el foco en el análisis de contenido de los artículos, hay
bastante variabilidad en relación con el tamaño de la muestra. Así, y
aunque la mayor parte de los estudios cuentan con un número de
participantes reducido, se pueden encontrar estudios con una muestra de
entre 25 y 50 estudiantes, como el caso de \textcite{higaldocajo2021,andrade2022,marques2023} donde la muestra supera el medio millar de estudiantes \cite{davila2023,lopeznoguero2023}. Por otro lado, es reseñable
también la variedad de titulaciones donde se contextualizan las
investigaciones (enfermería, agronomía, diseño industrial o anatomía
entre otras), aunque destacan aquellas que se desarrollan en
titulaciones del ámbito educativo, como Pedagogía, Magisterio o
Educación Social. Por ejemplo, este es el caso de los estudios de
\textcite{diazramirez2023}, \textcite{villatoro2022}, \textcite{blasco2022}, \textcite{martinez2021}, \textcite{area2023}, \textcite{cabero2022} o \textcite{lagossanmartin2022}.

Por otro lado, los objetivos de las investigaciones se orientan, de
manera generalizada, a evaluar el impacto de una determinada tecnología
en el aprendizaje del alumnado de educación superior. Difieren, sin
embargo, tanto los recursos como las competencias a desarrollar,
presentando un panorama bastante diverso en relación con estas dos
cuestiones.

Por un lado, en cuanto a los recursos, pueden identificarse dos
tipologías básicas: por un lado, herramientas tecnológicas y, por otro,
metodologías mediadas por tecnología. En relación con las primeras, es
destacable la presencia de la Realidad Aumentada (RA) en cuatro de las
investigaciones: \textcite{fernandez2023b}, \textcite{martinez2021},
\textcite{higaldocajo2021}. También hay tres
propuestas vinculadas con el aprendizaje móvil \cite{diazramirez2023,lopeznoguero2023,soncco2022} y otras tres centradas
en el uso de redes sociales o profesionales \cite{delacruz2021,ramos2022,rodrigo2022}. Aunque menos frecuentes, también hay
presencia de otras tecnologías como juegos de realidad virtual,
simuladores, booktube, software de fabricación digital o computación en
la nube. Respecto a las investigaciones que se fundamentan en la
implementación de metodologías mediadas por tecnología, hay dos
propuestas de flipped classroom \cite{andrade2022,blasco2022} y otras dos de aprendizaje-servicio
virtual \cite{cordero2021,gomez-gomez2021}.

Para terminar, y en relación con los resultados, hay bastante consenso
en atribuir beneficios al impacto de la tecnología en el alumnado de
educación superior para el desarrollo de competencias diversas. Entre
los que obtienen hallazgos negativos, \textcite{lopeznoguero2023}
concluyeron que los smartphones pueden crear dependencia y generan
demasiada información. Por su parte, \textcite{villatoro2022}
apuntaban que los itinerarios de aprendizaje suponían una gran carga de
trabajo para el alumnado y que exigían una retroalimentación constante
para ser efectivos. Por otro lado, \textcite{maldonadotorres2021}
señalaban que aspectos como la falta de conocimientos técnicos del
docente, la falta de tiempo y de instrucciones claras habían
condicionado la eficacia de una experiencia con simuladores. Por último,
\textcite{astudillo2021} concluía que la digitalización del proceso formativo no
había ayudado al alumnado a desarrollar aprendizajes significativos y,
además, había generado una brecha digital.

El resto de las investigaciones, como se apuntaba anteriormente,
concluyen que la tecnología puede contribuir a desarrollar diferentes
destrezas por parte del alumnado universitario. Así, algunos estudios
apuntan a la mejora de la comprensión teórico-conceptual de las
disciplinas \cite{cruz2021,davila2023,martinez2021}, así como la mejora de habilidades prácticas \cite{maldonadotorres2021,perez2023} y, por ende, del rendimiento académico
\cite{diazramirez2023,higaldocajo2021}. Asimismo, muchos
de los estudios inciden en el carácter motivador de la tecnología
\cite{fernandez2023,fernandez2023b,rodrigo2022}, poniendo de relieve la flexibilidad que ofrecen
para la autonomía y la autorregulación del aprendizaje \cite{area2023,delacruz2021,quezadasarmiento2021} y
la promoción de un rol activo del estudiante en su aprendizaje
\cite{andrade2022,martinez2021}. En la
dimensión social, las investigaciones ponen de relieve el potencial de
la tecnología para favorecer el trabajo en equipo \cite{cordero2021,lagossanmartin2022,marques2023}, 
la
cooperación entre iguales \cite{collazo2023,martinez2021,lopeznoguero2023,rodrigo2022} y la socialización
del conocimiento \cite{gomez-gomez2021,ramos2022}. Asimismo, la
tecnología ayuda a promover las competencias comunicativas \cite{marques2023,fernandez2023,soncco2022}, la
creatividad \cite{cordero2021} y la capacidad reflexiva
\cite{marques2023,martinez2021}.
