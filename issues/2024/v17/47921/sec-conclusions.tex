\section{Conclunsions} \label{sec-conclusions}

This study adds to the field aspects of the value of well-designed VE
projects to support PST's development of ICC, collaboration, and task
design skills, as well as an appreciation for SLA in other cultures
\cite{hanks2019research}. Overall, this VE project yielded many of the desired
learning outcomes for PSTs. Suggestions are also provided regarding
guidance for professor mentoring that is needed, types of ICT, and
processes for VE collaborations. While PSTs were successful on very
structured guided tasks; many needed additional scaffoldings for
initiative skills required for the cross-cultural projects which
required more open-ended decision-making, critical thinking, task
persistence, and time management.

As realized by \textcite{fuchs2022value}, in some aspects of the
process and product development, students may need more extensive
mediation and mentoring. Additional support and check-ins for the second
class of VE in 2023 -- opportunities through getting-to-know-you
activities, reminders with all PST emails, online reminders and
suggestions during their weekly modules, and some Zoom sessions -- were
all that some teams of PST needed to begin communications for their
cross-cultural projects. Even with those supports, some individual
Brazil PSTs in the in-person classes and US PSTs in the asynchronous web
needed more scaffolding, hints, guidelines, and push to spend the needed
time to work on the projects.

PSTs in the Brazil class asked for class time to work on their projects
and appreciated: \enquote{[BZ professor] took some time, during
class, to answer all our questions and give some advice. Their help was
really useful.} (BR 2023 ST2-58) and that there were
\enquote{\ldots computers to work on the project and the professor that
is always there with us to help.}(BR 2023 ST5-68).
		
Another challenge that will be addressed in future cross-cultural
collaborations also has to do with issues of time. The results from the
PST surveys and narratives seemed to be completed hastily and didn't
take into consideration specific information about what was personally
valuable to PST, for the entire process; lacking specifics about,
\emph{i.e}. Padlet interactions, autobiography of each person's
background, or specific communications between students. In \Cref{tab-03}, we
explore some features to continue and others to adapt in the third year
of the project.

\begin{table}[htpb]
\centering
\small
\setlength{\tabcolsep}{3pt}
\begin{threeparttable}
\caption{Consistency \& Adaptations for a third year of VE replication.}
\label{tab-03}
\begin{tabular}{*{2}{p{0.48\textwidth}}}
\toprule
Consistent Features to Continue & Adaptations/Additions \\
\midrule
Use of Google Classroom as the Plat-form for organizing the tasks and communicating with the students; & Helping the students with WhatsApp com-munication; \\ Organizing students in groups for in-teractions; & Designating a US \& Brazil PST leader for each team;\\
Use of Padlet for Introductions; & Providing common materials (readings and videos) for all the groups;\\ 
PST-lead \& faculty-guided discus-sions; & Schedule of synchronous meetings  for the discussion of the topic of the project;\\
Collaborative tasks; & Use of Zoom breakout rooms for synchro-nous groups interactions.\\
Reflections through a survey and a final narrative. & \\
\bottomrule		
\end{tabular}
\source{Own elaboration.}
\end{threeparttable}
\end{table}
	
As consistency for the next VE Projects, the professors will continue
using the Google Classroom for organizing the interactions and
materials; they will explore aspects related to VE for the PST to be
aware of the initiative; PST will continue with their introductions at
Padlet before developing the tasks and interacting in synchronous
platforms; also, professors' scaffolding for the development of joint
activities will be maintained as well as the final reflections using a
survey and writing a final narrative.
		
As adaptations and additions, the professor will help them with the
WhatsApp communication; also there will be leaders in each team and more
synchronous meetings for conversations in Zoom breakout rooms. Instead
of the groups developing projects on specific topics related to teacher
education, there will be only one theme for their conversations: Global
Education.
		
All those (re)organizations show that even working together for some
time, the professors need to constantly reflect and analyze the VE in
every year of its implementation. Working together is a continuous
development process in our career and teaching.
		
To finish, it is important to highlight that besides the challenges
faced and the (re)considerations professors needed to do and adapt for
each subsequent year of the project, the outcomes illustrated during
both years of the VE proved to be enriching experiences both for
professors and PSTs. They could interact with peers from different
contexts and cultures, experience an international activity \enquote{at home}
and develop many skills and attitudes important to their lives and
careers.
