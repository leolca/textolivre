\section{Introduction}\label{sec-intro}

Previous studies by Institutions of Higher Education (IHE) found ongoing
participation in online academic events, webinars, and Virtual Exchange
(VE) continued beyond the COVID-19 pandemic \cite{bowen2021virtual,garces2020upscaling,woicolesco2022internationalization}. Virtual
Exchange (VE) includes the engagement of groups of learners in extended
periods of online intercultural interaction and collaboration with
international peers as an integrated part of their educational programs
and under the guidance of educators and/or facilitators \cite[p.~1]{garces2020upscaling}. For students unable to physically travel for financial or
personal reasons, participation in virtual exchanges serves to
complement physical travel through cross-cultural \enquote{internationalization
at home} (IaH) and provide opportunities to develop intercultural
communicative competence and skills \cite{beelen2015redefining}.

One of these VE approaches is the Collaborative Online International
Learning (COIL) which brings together classes with a shared syllabus and
projects in which participants from universities in different parts of
the world come to collaborate, usually for four to eight weeks and
embedded as part of the classwork. Participants have opportunities to
discuss course materials, solve a problem, compare cultural norms, and
often create a gradable project. Participants can interact synchronously
(in real time) or asynchronous (not in real time) and can engage with
email, in a learning management platform (i.e., Google Classroom),
voice, social media, Padlet, or any other form of Information and
Communications Technology abbreviated as ICT \cite{odowd2018telecollaboration}.

This paper focuses on a VE/COIL project developed by the researchers in
2022 and 2023. The Brazilian professor, when looking for others
interested in projects based on the COIL model, found the corresponding
at a university in the US. The professors saw each
other’s interests and the VE director at the US
university arranged an E-meeting to discuss potential cross-cultural
collaborations. After a series of exchanges and realized common goals,
professors utilized their knowledge of VE task design \cite{kurek2017task} and this VE project was developed and implemented.

This exploratory Scholarship of Teaching and Learning (SoTL) qualitative
research project \cite{stake2005qualitative} studied the VE of students preparing to
be teachers (called Pre-service Teachers, abbreviated at PST) who were
enrolled in Second Language Acquisition (SLA) related courses, one in
the US and another in Brazil. In the two years of the VE Project, there
were 77 students participating in the initiative. PST in the US were
studying how to teach children in their classes who had immigrated to
the US and whose first language was not English. PST in Brazil were
studying to be English teachers.

In both years (2022 and 2023), professors planned an eight-week VE
Project embedded into our existing courses considering: the timeline for
the exchanges, the platforms to be used, the tasks to be developed,
introductory activities, the topics for the students'
cross-cultural projects, the forms of assessment, and the rubric. In
2022 with a class in the US and a class in Brazil and then again with
the research design revised in 2023 with two other classes of students
in respective courses, professors organized collaborative teams of PST
(in each team, PSTs from Brazil and US working together).

The aims of the joint research project\footnote{The research project was
	approved by the Ethics Committee of both universities.} carried out by
the authors of this paper were a) to develop and study a collaborative
inquiry project between students and professors from US and Brazil; b)
to study participants' virtual exchange experience and knowledge on
teacher education; language teaching and learning in both countries; c)
to analyze the development of the participants' intercultural
communicative competence (ICC) and learning and d) to investigate
the different resources they used when communicating with each other and
the benefits and challenges of information and communication
technologies (ICT) for the VE experience.

In this text, we focus on the outcomes and challenges of the VE,
analyzing data from periodic and end-of-semester surveys and narratives
as well as from the PST interactions and tasks in different digital
platforms.
