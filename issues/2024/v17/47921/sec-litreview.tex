\section{Literature Review}\label{sec-literaturereview}

According to \textcite{odowd2018telecollaboration}, educators involved in VE
initiatives offer their students the opportunity to develop skills such
as intercultural competence and critical thinking while working on
shared subject content through different cultural perspectives. In
teacher education programs, VE projects have been developed with the
following purposes: \enquote{to help the [Pre-service teachers (PST)]
develop digital, linguistic, and communicative skills; to work on
students' reflectivity and propensity for critical thinking, increased
openness, and social inclusion; and to prepare prospective language
teachers for facilitating their own VEs} \cite{grau2019experiential}.
International collaborative practices in teacher education programs have
the potential to develop global and intercultural attitudes in
pre-service teachers to prepare them to experience different
interactions and opportunities in their professional lives \cite{kopish2020leveraging}.
	
\textcite{sadler2016twelve} highlight that the experiences in VE in language
teacher education programs had an impact both on instructors and
pre-service teachers; it changed their mindset and understandings about
telecollaboration as well as their roles in terms of
instructor-pre-service teacher responsibilities in a way that they could
witness a learner-centered pedagogy. Additionally, the authors state
that the use of telecollaboration and the cooperation between the
teachers are also noticed by the students.
	
\textcite{hauck2019virtual} found that when tackling the communication, cultural,
linguistic and/or technical challenges pre-service teachers faced during
VE, they acquired new competencies in terms of behavioral flexibility,
interaction management, messaging skills, and language competence some
of which interrelate with digital competencies.
	
Incentives or opportunities, such as VE, can bring to light the
importance of the ability to communicate effectively and appropriately
in various cultural contexts in an increasingly global society
\cite{deardorff2020manual}. Some of the key components to developing
Intercultural Communication Competence (ICC) include motivation, self-
and other knowledge, mindfulness, cognitive flexibility, and tolerance
for uncertainty \cite{wilberschied_intercultural_2015}. Direct and thoughtful encounters with people, places, and foreign languages such as through VE with students at other colleges are effective ways to begin to develop ICC \cite{idris2019intercultural,lopez2017developing}.
	
Simply engaging PSTs in VE does not guarantee successful intercultural
outcomes. The types of activities, adequate time, and mentoring by the
instructors are critical to the value of virtual experiences and teacher
exchanges \cite{fuchs2022value}; for them to clearly understand
the content their peers from another culture share as well as be able to
critically, and with respect examine the content to discover
similarities and differences between that culture and their own \cite{roarty2021analysis}.
	
Generally, the first task in those projects is to exchange information
about each other to establish trust in the partnership. Then,
collaborative tasks are developed for analysis and comparison. In the
third phase, when it is included, the partners develop \enquote{some kind of
shareable product or artifact} \cite{godwin-jones2019telecollaboration}. A systematic and
well-planned framework with a constructivist lens that affords choices,
both synchronous and asynchronous platforms and tools but also guidance
with ICT can support the ease of cross-cultural collaboration, reduce
frustration, and facilitate student exploration of ideas \cite{calvo2023investigating,hauck2020approaches,kopish2020leveraging}.
		
\textcite{evaluate2019evaluating} found that students problem-solved when
certain tech problems occurred with synchronous tools and then switched
to a texting online social network -- WhatsApp to continue their
communications. When facing obstacles, pre-service teachers challenged
themselves and others to develop these new ICT collaboration skills. To
support their growth, another aspect of the project design should be a
built-in system for ongoing reflection so the PSTs can see the value of
skills and content learned and how they can apply their learning.
