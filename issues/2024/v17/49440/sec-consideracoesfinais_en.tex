\section{Final considerations}\label{sec-consideraçõesfinais}

Our study aimed to map the field of research on AL in Portuguese, Spanish, and English. For this, four axes were used: the definition of conceptual proposals, the identification of emerging pedagogical techniques and strategies, the description of the functions and domains of these different forms of knowledge, and the reflection on the challenges and difficulties revealed in the literature.

Regarding conceptual proposals (Q1), we observed a diverse set of approaches, which, although they make references to pre-existing literacies, have specificities that differentiate them. In general, what differentiates AL from predecessor literacy proposals is the critical focus on the agency of algorithmic systems, considering their performative and changeable nature.

As for pedagogical approaches (Q2), despite the limited number of publications on the topic, a diversity of strategies around teaching and awareness about algorithms is perceived. Among these approaches, content methodologies stand out, which seek to inform subjects about the consequences of the relationship with algorithms, as well as experiential methodologies, which encourage the exploration of experiences and knowledge developed in interactions with algorithmic systems. Proposals for critical design, interaction in games, textual production, and narrative dynamics also emerge as methodologies that aim to involve subjects in the process of understanding and criticism.

Regarding the areas of knowledge in which the debates on AL are located (Q3), it is possible to affirm that it is an eminently transdisciplinary research object. In our study, we noticed a significant breadth of fields of knowledge in which the topic has been studied, such as communication, computing, education, psychology and information science. Regarding the difficulties perceived in the analyzed studies (Q4), we consider that the central challenge for the field of investigation is the lack of a clear and consistent definition of the concept, along with its proximity to other forms of literacy. Regarding the development of critical consciousness, the studies analyzed refer to difficulties related to the often opaque and complex nature of algorithmic systems, especially regarding understanding and agency over their functioning and developments. Difficulties are also referred to as inequalities in access to digital technologies.

In conclusion, AL is an emerging and transdisciplinary area that needs development, both conceptually, for greater clarity, and methodologically, with the development of proposals that can assist in the process of promoting awareness and critical thinking regarding the actions of algorithms. We can place it as a series of knowledge and dynamics that aim to produce awareness about the agency of algorithms in relationships and that are established in the environments of platforms. The different approaches to AL commonly provide ways in which subjects can strategically use these systems to achieve personal and collective goals. Unlike other forms of literacy, AL requires a critical examination of the development practices of algorithmic systems, addressing issues such as performativity, opacity, bias, and social justice.
