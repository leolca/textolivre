\section{Introduction}\label{sec-introdução}

For many, a typical day involves several occasions in which online platforms are used to consume news, interact with friends, study, shop, or watch series and films. In all these activities, digital structures based on data and shaped by algorithms act by suggesting or determining which content and information are most relevant or which profiles are most compatible with our interests on a given platform. The insertion of these structures in the mediation of different social practices has profoundly reconfigured how subjects begin to constitute their relationships with the world and with the community \cite{Zuboff2020,Mbembe2021}.

Living in a datafied society, argues \textcite{Siles2024}, has been characterized by the increasing conversion of experiences, relationships, and identities into data that feed one of the most prosperous industries in the global economy. These dynamics depend dramatically on algorithms and recommendation systems, which play a key role in the business models of digital platforms. The algorithmic techniques that give logic to the functioning of these systems work by analyzing large volumes of data in order to identify patterns and trends. Based on this analysis, recommendation systems aim to personalize their functionalities by presenting content that is more likely to attract and engage subjects.

While we enjoy impressive access to content and services, we are exposed to aggressive processes of capturing data on human activity, used to quantify, analyze, and generate profits for companies in this sector \cite{Couldry2019,Siles2024}. The growing dependence on platforms and the consequent datafication of life \cite{VanEs2017} have raised relevant discussions about its developments, such as debates about the psychic and emotional effects of human behavior modulation tactics \cite{Bruno2019}, the dissemination of misinformation and online defamatory campaigns \cite{Rogers2023}, the dynamics of exploitation and precarious work \cite{Grohmann2023}, and regulatory policies on digital platforms \cite{Fletcher2023}.

In parallel, it is increasingly understood that responses from the field of education are essential to dealing with the consequences, some harmful, of these platformed social dynamics. Recently, the United Nations (2023, online) called for the need to invest in “robust digital literacy initiatives”, as a way of offering those in contact with these systems knowledge and skills to deal with such conditions, especially as concerns the circulation of false information and hate speech.

At the heart of this diagnosis is the call for the development of a critical consciousness concerning the logics of digital capitalism \cite{Buckingham2022}. In academic literature, this concern tends to be located in debates surrounding the notion of algorithmic literacy (AL), as a field of study focused on discussing the knowledge, skills, and abilities necessary for a more autonomous and critical relationship with the computational systems that operate in our lives \cite{Devito2021}.

The idea of literacy has a long history in discussions about information and media skills \cite{Mora2016}. However, in recent decades, the concept has been extended to designate the “skills and competencies involving the search, selection, analysis, evaluation, and process of information, considering the means, contexts, and environments in which it is found and knowledge is produced” \cite[p. 26]{Rosa2016}. Approaches that aim to promote literacy seek to develop reflective and critical capabilities to enable active involvement with the logics of the systems that shape contemporary societies \cite{Pangrazio2022}.

Specifically concerning the concept of AL, it can be stated that this is a notion that still contains significant inconsistencies \cite{Hargittai2020}. \textcite{DogrueL2022} indicate that, in the early publications related to AL, the majority focus on raising awareness about the use of algorithms or knowledge about them, but they do not actually detail how these aspects relate to or constitute AL as a concept presumably wider. In this sense, \textcite[p. 2]{Ridley2021} argue that:

\begin{quote}
without a clear, recognized, and actionable definition that differentiates it from concepts such as digital literacy, computational thinking and algorithmic thinking, algorithmic literacy will be relegated to a buzz phrase and the urgency of its recognition and application will be lost.
\end{quote}

Thus, our article aims to develop a systematic review on the concept of AL. The objective is to map the main studies related to AL in Spanish, English, and Portuguese, in the Scopus and Web of Science databases. Methodologically organized as a systematic literature review, the selection of publications follows three phases of the PRISMA statement (identification, screening, and eligibility), which guide the qualification of the selection of publications and the efficiency of the investigation \cite{Moher2009}.

Analysis of general publication data shows a significant increase in studies in recent years, especially in 2021 and 2022; a diversity of areas of study origin; and a low number of publications in Portuguese and Spanish. In relation to the analysis of the articles, a variety of conceptual models and pedagogical approaches were observed, reflecting the emerging and multidisciplinary nature of the field.