\section{Considerações finais}\label{sec-consideraçõesfinais}

Nosso estudo teve como objetivo mapear o campo de investigação sobre
literacia algorítmica (LA) nas línguas portuguesa, espanhola e inglesa.
Para isso, foram utilizados quatro eixos: a definição de propostas
conceituais, a identificação de técnicas e estratégias pedagógicas
emergentes, a descrição das funções e domínios dessas diferentes formas
de conhecimento e a reflexão sobre os desafios e dificuldades revelados
na literatura.

A respeito das propostas conceituais (Q1), observamos um conjunto
diversificado de abordagens, que, embora faça referências a literacias
preexistentes, possui especificidades que o diferenciam. De modo geral,
o que difere a LA de propostas predecessoras de literacia é o foco
crítico na agência dos sistemas algorítmicos, considerando sua natureza
performática e mutável.

Quanto às abordagens pedagógicas (Q2), apesar do número restrito de
publicações sobre o tema, é percebida uma diversidade de estratégias em
torno do ensino e da conscientização sobre algoritmos. Dentre essas
abordagens, destacam-se as metodologias conteudistas, que buscam
informar os sujeitos dos desdobramentos da relação com algoritmos, e as
metodologias experienciais, que incentivam a exploração das vivências e
conhecimentos desenvolvidos nas interações com sistemas algorítmicos.
Propostas de design crítico, interação em jogos, produção textual e
dinâmicas narrativas também emergem como metodologias que almejam
envolver os sujeitos no processo de compreensão e crítica. Sobre as
áreas de conhecimento nas quais se localizam os debates sobre LA (Q3),
consideramos ser possível afirmar que se trata de um objeto de pesquisa
eminentemente transdisciplinar. Em nosso estudo, percebe-se uma
significativa amplitude de campos de conhecimentos nas quais o tema vem
sendo estudado, como comunicação, computação, educação, psicologia e
ciência da informação. Quanto às dificuldades percebidas nos estudos
analisados (Q4), consideramos que o desafio central para o campo de
investigação está na falta de uma definição clara e consistente do
conceito, junto à sua proximidade com outras formas de literacia. No que
tange ao desenvolvimento de consciência crítica, os estudos analisados
referem dificuldades relacionadas à própria natureza frequentemente
opaca e complexa dos sistemas algorítmicos, sobretudo para compreensão e
agência sobre seu funcionamento e desdobramentos. Também são referidas
como dificuldades as desigualdades no acesso a tecnologias digitais.

Em conclusão, a LA é uma área emergente e transdisciplinar que carece de
desenvolvimento tanto conceitual, para maior clareza, quanto
metodológico, com o desenvolvimento de propostas que possam auxiliar no
processo de promoção de consciência e senso crítico diante das ações dos
algoritmos. Podemos posicioná-la como uma série de conhecimentos e
dinâmicas que almejam produzir consciência sobre a agência dos
algoritmos nas relações e que se estabelecem nas ambiências das
plataformas. As diferentes abordagens da LA costumam emprazar modos
pelos quais os sujeitos possam utilizar estrategicamente esses sistemas
para alcançar objetivos pessoais e coletivos. Diferente de outras formas
de literacia, a LA exige um exame crítico das práticas de
desenvolvimento dos sistemas algorítmicos, abordando questões como
performatividade, opacidade, viés e justiça social.
