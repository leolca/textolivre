\section{Introdução}\label{sec-introdução}

Para muitos, um dia comum envolve diversas ocasiões nas quais
plataformas \textit{on-line} são usadas para consumir notícias, interagir com
amigos, estudar, fazer compras ou assistir a séries e filmes. Em todas
essas atividades, estruturas digitais baseadas em dados e conformadas
por algoritmos atuam ao sugerir ou determinar quais são os conteúdos e
as informações mais relevantes ou os perfis mais compatíveis com nossos
interesses em uma determinada plataforma. A inserção dessas estruturas
na mediação de diferentes práticas sociais tem reconfigurado
profundamente como os sujeitos passam a constituir suas relações com o
mundo e com a coletividade \cite{Zuboff2020,Mbembe2021}.

Viver em uma sociedade dataficada, argumenta \textcite{Siles2024}, tem se
configurado pela crescente conversão de experiências, relações e
identidades em dados que alimentam uma das mais prósperas indústrias da
economia global. Essas dinâmicas dependem dramaticamente dos algoritmos
e dos sistemas de recomendação, que desempenham papel central nos
modelos de negócio de plataformas digitais. As técnicas algorítmicas que
dão lógica ao funcionamento desses sistemas atuam ao analisar grandes
volumes de dados a fim de identificar padrões e tendências. Com base
nessa análise, sistemas de recomendação almejam personalizar suas
funcionalidades a partir da apresentação de conteúdos que têm maior
probabilidade de atrair e engajar os sujeitos.

Ao passo que desfrutamos de um impressionante acesso a conteúdos e
serviços, estamos expostos a agressivos processos de captação de dados
da atividade humana, utilizados para que se possa quantificar, analisar
e gerar lucros às empresas desse setor \cite{Couldry2019,Siles2024}. A crescente dependência das plataformas e a consequente
dataficação da vida \cite{VanEs2017} têm levantado discussões
relevantes sobre seus desdobramentos, como os debates sobre os efeitos
psíquicos e emocionais das táticas de modulação do comportamento humano
\cite{Bruno2019}, a disseminação de desinformação e as
campanhas difamatórias \textit{on-line} \cite{Rogers2023}, as dinâmicas de
exploração e precarização do trabalho \cite{Grohmann2023} e as políticas de
regulação em plataformas digitais \cite{Fletcher2023}.

Em paralelo, cada vez mais, entende-se que respostas do campo da
educação são fundamentais para lidar com os desdobramentos, parte deles
nocivos, dessas dinâmicas sociais plataformizadas. Recentemente, a
\textcite{ONU2023} apelou pela necessidade de
investimento em ``iniciativas robustas de alfabetização digital'', como
forma de oferecer aos sujeitos em contato com esses sistemas
conhecimento e habilidades para lidar com tais condições, especialmente
no que toca à circulação de informação falsa e discursos de ódio.

Está no cerne desse diagnóstico o apelo para o desenvolvimento de uma
consciência crítica sobre as lógicas do capitalismo digital \cite{Buckingham2022}. Na literatura acadêmica, tal preocupação tende a localizar-se em
debates sobre a noção de literacia algorítmica (LA)\footnote{Neste
  estudo, adotamos o termo "literacia algorítmica" em vez de
  "alfabetização algorítmica" ou "letramento algorítmico", pois as
  traduções desses termos podem variar significativamente seu sentido em
  espanhol e português. Os termos "alfabetização" e ``letramento'' estão
  mais relacionados a uma ideia instrumental e tendem a desconsiderar as
  múltiplas linguagens dos sujeitos em suas culturas \cite{Mora2016}.
  Existem outras propostas conceituais, como educomunicação e
  mídia-educação, que podem abranger parcialmente ideias relacionadas à
  literacia \cite{Rosa2016}.}, enquanto campo de estudos voltado para
discussão de conhecimentos, competências e habilidades necessárias para
uma relação mais autônoma e crítica com os sistemas computacionais que
atuam em nossa vida \cite{DogrueL2021}.

A ideia de literacia tem uma longa trajetória nas discussões sobre
competências informacionais e midiáticas \cite{Mora2016}. Porém, nas
últimas décadas, o conceito tem sido estendido para designar as
``habilidades e competências envolvendo a busca, a seleção, a análise, a
avaliação e o processo da informação, considerando os meios, contextos e
ambientes em que se encontra e se produz o conhecimento'' \cite{Rosa2016}. As abordagens que almejam promover a literacia buscam
desenvolver capacidades reflexivas e críticas para propiciar o seu
envolvimento ativo com as lógicas dos sistemas que conformam as
sociedades contemporâneas \cite{Pangrazio2022}.

A respeito especificamente do conceito de LA, pode-se afirmar que essa é
uma noção ainda com inconsistências significativas \cite{Hargittai2020}. \cite{DogrueL2022} indicam que, nas primeiras
publicações relacionadas à LA, a maioria se concentra na conscientização
sobre o uso de algoritmos ou no conhecimento sobre eles, mas não detalha
como esses aspectos se relacionam ou constituem a literacia algorítmica
enquanto conceito presumivelmente mais amplo. Nesse sentido, \textcite[p.~2]{Ridley2021}
\footnote{Todas as citações em língua
  estrangeira foram traduzidas livremente pelos autores.} sustentam que:

\begin{quote}
sem uma definição clara, reconhecida e acionável que a diferencie de
conceitos como literacia digital, pensamento computacional e pensamento
algorítmico, a literacia algorítmica será relegada a uma frase da moda e
a urgência do seu reconhecimento e aplicação serão perdidas.
\end{quote}

Assim, nosso artigo almeja desenvolver uma revisão sistemática sobre o
conceito de literacia algorítmica. O objetivo é mapear os principais
estudos relacionados à LA em espanhol, inglês e português, nas bases de
dados Scopus e Web of Science. Organizada metodologicamente como uma
revisão sistemática da literatura, a seleção das publicações segue três
fases da declaração PRISMA (identificação, triagem e elegibilidade), que
orientam a qualificação da seleção de publicações e a eficiência da
investigação \cite{Moher2009}.

A análise dos dados gerais das publicações mostra aumento significativo
de estudos nos últimos anos, especialmente em 2021 e 2022, diversidade
de áreas de origem dos estudos e baixo número de publicações em
português e espanhol. Em relação à análise dos artigos, observa-se
variedade de modelos conceptuais e abordagens pedagógicas, reflexo da
natureza emergente e multidisciplinar do campo.