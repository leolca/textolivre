\section{Objetivo de la Investigación}\label{sec-objetivodelainvestigación}

La presente investigación tiene como objetivo conocer el nivel de
autoeficacia y actitudes de un grupo de estudiantes del grado de
Educación Primaria, respecto a la metodología basada en datos y el uso
de las herramientas digitales de corpus.

La novedad de nuestro estudio radica, por tanto, en dos aspectos. Por un
lado, explora la autoeficacia en un contexto de aprendizaje basado en
datos. Esta metodología de implantación aún incipiente en el contexto de
enseñanza de lengua extranjera hace uso de herramientas de corpus
digitales como recurso de aula manipulable para el alumnado.

Por otro lado, el perfil de los participantes es concreto. La muestra
está formada por estudiantes universitarios, futuros docentes de lengua
extranjera en la etapa de educación primaria. Si bien es cierto que los
estudios tanto de autoeficacia como de aprendizaje basado en datos se
dan en contextos de enseñanza superior, el hecho de que sean futuros
docentes de lengua extranjera puede aportar una visión particular, la
cual no pueden proporcionar otras investigaciones cuya muestra es
estudiantado de otros grados universitarios. Es decir, hablamos de
docentes de lengua extranjera en formación, que podrán aplicar la
metodología basada en datos y las herramientas de corpus en un futuro
cercano. La opinión y actitudes que tengan sobre dicha metodología y
recursos puede determinar el papel de los mismos en las aulas de lengua
extranjera.