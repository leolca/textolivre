\documentclass[spanish]{textolivre}

% metadata
\journalname{Texto Livre}
\thevolume{17}
%\thenumber{1} % old template
\theyear{2024}
\receiveddate{\DTMdisplaydate{2024}{3}{20}{-1}}
\accepteddate{\DTMdisplaydate{2024}{5}{4}{-1}}
\publisheddate{\today}
\corrauthor{Gema Alcaraz Mármol}
\articledoi{10.1590/1983-3652.2024.51693}
%\articleid{NNNN} % if the article ID is not the last 5 numbers of its DOI, provide it using \articleid{} commmand 
% list of available sesscions in the journal: articles, dossier, reports, essays, reviews, interviews, editorial
\articlesessionname{articles}
\runningauthor{Alcaraz Mármol}
%\editorname{Leonardo Araújo} % old template
\sectioneditorname{Hugo Heredia Ponce}
\layouteditorname{João Mesquita}

\title{Aprendizaje de lengua extranjera a través de herramientas digitales de corpus. Actitudes y autoeficacia en estudiantes universitarios del grado de Educación Primaria}
\othertitle{Aprendizagem de línguas estrangeiras através de ferramentas de corpus digital. Atitudes e autoeficácia em estudantes universitários do ensino básico}
\othertitle{Foreign language learning through digital corpus tools. Attitudes and self-efficacy in university students of Primary Education teacher training}

\author[1]{Gema Alcaraz Mármol~\orcid{0000-0001-7703-3829}\thanks{Email: \href{mailto:gema.alcaraz@uclm.es}{gema.alcaraz@uclm.es}}}
\affil[1]{Universidad de Castilla-La Mancha,  Departamento de Filología Moderna, Toledo, España.}

\addbibresource{article.bib}
\usepackage{array}

\begin{document}
\maketitle
\begin{polyabstract}
\begin{abstract}
El presente estudio tiene como objetivo explorar la
autoeficacia y actitudes sobre el uso de dos herramientas digitales de
corpus por parte de un grupo de estudiantes de inglés como lengua
extranjera que cursa el Grado en Educación Primaria en la Universidad de
Castilla-La Mancha. Los participantes realizaron una serie de
actividades en el contexto de la metodología basada en datos utilizando
como recurso dos herramientas de corpus: SKELL y BNC. Tras la
realización de las actividades y la manipulación de las herramientas de
corpus, los participantes respondieron un cuestionario donde debían
expresar su grado de acuerdo con cada una de las afirmaciones que
componían dicho cuestionario. El cuestionario hace referencia a aspectos
sobre la dificultad de uso, el nivel de utilidad, o la aplicación fuera
del aula de las herramientas digitales de corpus; también se pregunta a
los participantes por las actividades de aprendizaje basado en datos
para la adquisición de distintos aspectos de la lengua inglesa. Los
resultados apuntan a una actitud positiva y un nivel considerable de
autoeficacia hacia el corpus. Finalmente, se anima a adoptar este tipo
de metodología además de a una formación específica del profesorado para
su correcta y óptima aplicación.

\keywords{Autoeficacia \sep Aprendizaje basado en datos \sep Corpus \sep
	Actitudes \sep Lengua extranjera}
\end{abstract}

\begin{portuguese}
\begin{abstract}
O presente estudo tem como objetivo explorar a auto-eficácia e
as atitudes face à utilização de duas ferramentas de \textit{corpus} digital por parte de um grupo de estudantes de inglês como língua estrangeira que
frequentam uma licenciatura em Educação Primária na Universidade de
Castilla-La Mancha. Os participantes realizaram uma série de atividades
no contexto da metodologia orientada para os dados, utilizando duas
ferramentas de corpus: SKELL e BNC. Após a realização das atividades e
a manipulação das ferramentas de \textit{corpus}, os participantes responderam a um questionário no qual tinham de expressar o seu grau de concordância com cada uma das afirmações que compunham o questionário. O questionário refere-se a aspectos de dificuldade de utilização, nível de utilidade ou aplicação fora da sala de aula das ferramentas de \textit{corpus} digital; os participantes são também questionados sobre atividades de aprendizagem baseadas em dados para a aquisição de diferentes aspectos da língua inglesa. Os resultados apontam para uma atitude positiva e um nível considerável de auto-eficácia em relação ao corpus. Por último,
incentiva-se a adoção deste tipo de metodologia, bem como a formação
específica de professores para a sua aplicação correta e optimizada.

\keywords{Auto-eficácia \sep Aprendizagem baseada em dados \sep \textit{Corpus} \sep Atitudes \sep Língua estrangeira}
\end{abstract}
\end{portuguese}

\begin{english}
\begin{abstract}
The present study aims to explore the self-efficacy and
attitudes towards the use of two digital corpus tools by a group of
students of English as a foreign language studying for a degree in
Primary Education at the University of Castilla-La Mancha. The
participants carried out a series of activities in the context of the
data-driven methodology using two corpus tools: SKELL and BNC. After
carrying out the activities and manipulating the corpus tools, the
participants answered a questionnaire in which they had to express their
degree of agreement with each of the statements that made up the
questionnaire. The questionnaire refers to aspects such as difficulty of
use, level of usefulness, or application outside the classroom of the
digital corpus tools; participants are also asked about data-driven
learning activities for the acquisition of different aspects of the
English language. The results point to a positive attitude and a
considerable level of self-efficacy towards the corpus. Finally, the
adoption of this type of methodology is encouraged as well as specific
teacher training for its correct and optimal application.


\keywords{Self-efficacy \sep Data-driven learning \sep Corpus \sep Attitudes \sep Foreign language}
\end{abstract}
\end{english}
\end{polyabstract}

% !TeX root = main.tex

\section{Introducción}\label{sec-introducción}

La educación mediática es un proceso pedagógico y comunicativo que busca
que la ciudadanía desarrolle habilidades críticas que permitan analizar
el universo de medios que se proyectan en la sociedad postdigital
\cite{Escaño_2023}. Esta perspectiva convierte este ámbito de la
alfabetización en un trabajo por la inclusión de todos los sectores de
la sociedad, especialmente los más vulnerables que, por unos u otros
motivos, no pueden acceder a determinada información o son más
susceptibles a los peligros de esta. La educación mediática apuesta, por
tanto, por la inclusión, al buscar la promoción de la equidad, la
diversidad y la participación en el proceso educativo cite{guillen2024}. La integración de los principios que
fundamentan la inclusión en un planteamiento mediático es fundamental
para conseguir que la ciudadanía desarrolle las competencias digitales y
mediáticas imprescindibles para poder sobrevivir a la brecha de la
sociedad posdigital, potenciando su pensamiento crítico y, en
consecuencia, su actitud crítica \cite{palacios-rodríguez2025}. En este contexto hipermediatizado, la educación mediática
proporciona oportunidades de aprendizaje a través de \emph{mass media} y
\emph{social media} de manera equitativa y accesible, incluidas aquellas
personas que, debido a su edad, han sido excluidas o marginadas del
proceso de digitalización.

La Educación Abierta y a Distancia (EAD) ha jugado un papel importante
en el ámbito de la educación mediática y, como consecuencia, en la
creación de redes comunicativas globales al posibilitar la conexión de
estudiantes de distintas ubicaciones geográficas y contextos culturales.
La UNED desempeña un papel destacado desde 1972 en España promocionando
y desarrollando este modelo formativo y el acceso a la educación
superior de calidad a un amplio abanico de personas. Más
específicamente, en el ámbito de la EAD, la UNED ha colaborado
activamente por la incursión de la educación mediática, a través de los
\emph{Massive Open Online Courses} (MOOC), ofreciendo éstos a través de
su plataforma Uned Abierta, o de otras conocidas como \emph{Coursera},
\emph{edX}, \emph{Miriadax}, \emph{EcoDigitalLearning,} tmooc.es, etc.,
contribuyendo así al intercambio de conocimiento y la
internacionalización de la universidad. Este modelo formativo, diseñado
desde hace más de quince años (Ratnasari; Chou; Huang, 2024), puede
colaborar en la promoción de la inclusión social, al ofrecer un acceso
global, flexibilidad, costos reducidos, adaptabilidad, diversidad de
temáticas, interacción, participación y construcción colectiva del
conocimiento. Se promueve así un aprendizaje a lo largo de la vida
buscando la convergencia mediática: vídeos, lecturas, cuestionarios,
foros, recursos gamificados, creaciones con IA \cite{Aparicio-Gómez2024,cárdeasbenavides2024} e incluso
entornos inmersivos incluyendo los metaversos \cite{huesoromero2024,chuchuca2024metaverso}.

En el presente estudio hemos optado, dentro de las diferentes tipologías
de los MOOC, por el modelo sNOOC \cite{quintana2024snooc}. Tradicionalmente,
se han presentado los SPOC, xMOOC, cMOOC, sMOOC, tMOOC, COMOOC, pero,
actualmente también nos encontramos con los \emph{Nano Open Online
Courses} - NOOC \cite{clark2013moocs,gomez2016,Osuna-Acedo2017,LopezdelaSerna_Garrido_2018,Escaño_Dewhurst_2024}, un planteamiento
concreto y personalizado que estructura su duración en horas y articula
su estructura en torno a un contenido, herramienta o habilidad concreta
\cite{MANANGÓN-CABRERA2023,huesoromero2024}; con una duración máxima de veinte
horas \cite{intef2016}. Una formación en módulos comunicativos e
interactivos más minimalista que se presenta como novedosa, líquida y
flexible de contenidos, de recursos, de espacio y de tiempo \cite{basantes2020}. Perfilando aún más
nuestra elección, hemos apostado en el estudio por los sNOOC o
\emph{social}NOOC que posibilitan más aún el empoderamiento social de
redes comunicativas de estudiantes gracias a la creación colectiva del
conocimiento, la base de la cultura participativa y la apuesta por la
fidelización-compromiso del alumnado convertido en \emph{e-teacher}.
Esta implementación de los sNOOC se presenta como herramienta de
evaluación continua en la EAD, centrándose en experiencias y
valoraciones del alumnado, los elementos comunicativos y pedagógicos,
así como el proceso de construcción colaborativa de los contenidos,
poniendo énfasis en pedagogías inclusivas, que se enriquecen de las
metodologías activas, la creación con IA y metaverso \cite{galíndez2024}. Estas prácticas pedagógicas inclusivas, desde un diseño universal
de aprendizaje \cite{SanchezFuentes2023}, valoran la diversidad de
habilidades, intereses, experiencias y estilos de aprendizaje,
garantizan la equidad, aseguran la adaptabilidad utilizando estrategias
flexibles adaptadas a las diversas capacidades, promueven la
participación, fomentan la colaboración de la comunidad virtual y la
creación de entornos seguros, acogedores y estimulantes.

En este artículo presentamos el análisis de una experiencia que parte de
la creación de redes comunicativas en estudiantes de posgrado de la
Universidad Nacional de Educación a Distancia (UNED) con la finalidad de
convertirlos en \emph{e-teacher,} implementando proyectos formativos
para que las personas de la tercera edad, en riesgo de exclusión
digital, adquieran competencias mediáticas. La creación de esta red
comunicativa sólida parte de un espacio académico de la asignatura de
``Escenarios virtuales para la participación'' y se proyecta como
vinculación y transferencia a la sociedad. Los recursos comunicativos
cobran protagonismo en estas redes, no sólo a través de la plataforma
del curso concreto, sino también en la creación con el uso de la
Inteligencia Artificial (IA) o de espacios inmersivos de una plataforma
sNOOC, donde no sólo las personas participantes puedan formarse, sino
también puedan conectarse, discutir temas relevantes, compartir recursos
o difundir por el \emph{software} social ideas originales. Este proyecto
formativo lleva a la potencialización de la mentoría entre iguales,
unidos en red de intereses y en diversidad de propósitos, donde se
establecen relaciones más cercanas entre el alumnado y el aprendizaje
colaborativo.

La formación mediática de personas de un sector, que se puede considerar
a nivel mediático en peligro de exclusión, como es el de la tercera
edad, es un reto que ya han iniciado otros agentes educativos y sociales
desde diferentes instituciones \cite{abad-acala2014,abad-acala2017,Leal-Maridueña2017,heredia-sánchez2023}. En este caso
concreto, se ha visto beneficiada gracias a la solidaridad y
empoderamiento de redes de estudiantes unedianos \cite{Swan2015,Reich2015}, y el
compromiso por la formación mediática, promoviendo un mejor sentido de
la responsabilidad cívica \cite{bringle1996}, convirtiéndose en un
planteamiento pionero en esta etapa de formación. En esta experiencia,
el compromiso se materializa como un puente tangible que vincula la
formación a través de sNOOC con las problemáticas sociales como es la
alfabetización mediática de las personas de la tercera edad, pocas veces
productoras o creadoras de contenidos y muchas más consumidoras pasivas
de las redes sociales, en comparación con generaciones más jóvenes,
sobre todo en plataformas como Instagram, TikTok, YouTube, X o Facebook.
Los adultos mayores suelen comunicarse a través de estas interfaces con
amistades y familiares, comparten recuerdos, siguen noticias o temas
interesantes, e incluso, las y los más valientes, se animan a formar
parte de comunidades. Las plataformas, conocedoras de esta situación,
están adaptando sus interfaces y servicios para ser más accesibles y
fáciles de usar para personas de estas edades. Partiendo de esta
realidad inequitativa se puede y se debe impulsar, a través de proyectos
formativos de alfabetización mediática como el que presentamos en esta
investigación, una mejora de las habilidades, competencias y capacidades
de este grupo demográfico, especialmente afectado por la brecha digital
generacional.

\section{Marco Teórico}\label{sec-marcoteórico.tex}
\subsection{La lingüística de corpus en la enseñanza de lenguas extranjeras} \label{sub-sec-lingüísticadecorpus}

La lingüística de corpus está adoptando un papel cada vez más relevante
en la enseñanza y aprendizaje de lenguas extranjeras \cite{sinclair2004how,okeeffe2007corpus,romer2009corpus,szudarski2018corpus}. Así, ofrece herramientas y enfoques pedagógicos que
permiten al estudiantado manejar grandes cantidades de texto auténtico
en formato digital, obteniendo información útil sobre el significado de
palabras clave y determinadas estructuras gramaticales partiendo de su
contexto de uso \cite{hanks2013}. \textcite{mcenery2011what} diferencian entre
usos indirectos y directos del corpus en el marco de la enseñanza de
lenguas.

Tradicionalmente, la lingüística de corpus ha tenido un papel indirecto
en el aprendizaje, siendo fuente para el diseño de materiales didácticos
para lengua extranjera o para la confección de diccionarios de
aprendices donde, entre otros, se ofrece información concreta sobre la
frecuencia y contexto de uso de las palabras. Podemos encontrar otras
aplicaciones indirectas en el ámbito de las lenguas para fines
específicos, donde son comunes los glosarios derivados de corpus de
especialidad, por ejemplo. Además, y en la línea de las aplicaciones
indirectas, no podemos olvidar los corpus de aprendices, cuyo análisis
ofrece esclarecedores datos derivados de errores comunes, y que abren
una valiosa ventana a la que asomarse a la hora de investigar el proceso
de adquisición de lenguas extranjeras \cite{meunier2002pedagogical}.

Por otro lado, respecto a la aplicación directa de los corpus, también
denominado aprendizaje basado en datos, autores como \textcite{gabrielatos2005corpora}
y \textcite{bernardini2004corpora} destacan sus efectos positivos en la adquisición de
una lengua extranjera. En el aprendizaje basado en datos los aprendices
manejan herramientas de corpus, las cuales les permiten explorar el
comportamiento de la lengua meta en su contexto de uso real. Así,
observan la frecuencia de uso de las palabras y sus concordancias, entre
otros. Estos procedimientos desarrollan la capacidad del estudiantado
para identificar patrones lingüísticos, hacer generalizaciones
fundamentadas en la lengua que están aprendiendo y examinar en detalle
cierto léxico y estructuras morfosintácticas, contribuyendo así a su
competencia metalingüística. El proceso de enseñanza y aprendizaje
tiende así a convertirse en una experiencia más motivadora y auténtica,
puesto que el estudiante consolida conocimientos que ya posee, a la vez
que descubre e integra nuevos \cite{boulton2012what}.

\textcite{zapata-ros2018} destacan el papel que tienen las
herramientas digitales de corpus y el aprendizaje basado en datos en el
desarrollo de la capacidad reflexiva y de razonamiento del alumnado, que
se materializa en distintas destrezas cognitivas como analizar, hacer
inferencias, interpretar o verificar. Como bien apunta \textcite{boulton2009testing},
el aprendizaje basado en datos complementa el enfoque comunicativo ya
que fomenta, por un lado, el aprendizaje inductivo, y por otro la
autonomía del propio estudiantado. A este respecto, \textcite{flowerdew2015corpus}
menciona tres teorías de aprendizaje que fundamentan esta metodología:
la llamada Noticing Hypothesis o Hipótesis de la Captación, el
Constructivismo y la Teoría Sociocultural de \textcite{vygotsky1978mind}. Dichas teorías están presentes en el aprendizaje basado en datos y el uso de
corpus, puesto que mediante estos se construye conocimiento a partir de
la búsqueda y la observación, se aprende por notoriedad de patrones y se
refuerza la negociación de significado.

Al interactuar con los corpus, el estudiantado aplica destrezas de
pensamiento crítico que implican la observación, la conciencia
lingüística, la formulación, la comprobación de hipótesis, y la
sensibilidad a la variación lingüística. En una tarea típica de
aprendizaje basado en datos, el estudiantado examina un nodo (una
palabra o una cadena de palabras) y su contexto inmediato, tanto a la
izquierda como a la derecha, así como, si lo desea, el contexto más
amplio del párrafo. Cada nodo aparece en una denominada línea de
concordancia, y a partir de ahí se exploran las pruebas lingüísticas y
se construyen hipótesis sobre la naturaleza del uso de la lengua.

Se considera, por tanto, un enfoque constructivista, inductivo y
centrado en el alumnado, que fomenta la autonomía y el aprendizaje por
descubrimiento. En este contexto de enseñanza, el profesorado actúa como
asesor y guía, facilitando, más que transmitiendo, conocimientos
lingüísticos. Se presentan, por tanto, una combinación de innovaciones
tecnológicas, educativas y varios recursos en línea que promueven la
inclusión de diferentes perspectivas pedagógicas dentro y fuera del
aula. Así, contribuye a que el estudiantado universitario, actualmente
en su mayoría nativos digitales, aprecien los enfoques basados en corpus
y los adapten a su aprendizaje de lengua extranjera. Así, a través de
este tipo de aprendizaje el alumnado puede llegar a sus propias
conclusiones, con una retención más sólida y prolongada y un mayor
conocimiento de la lengua meta. En base a estas premisas, se podría
afirmar que el estudiantado es menos dependiente del profesorado, y
aprende directamente de la lengua en contexto real en lugar de hacerlo
de recursos tradicionales como los libros de texto, las gramáticas o los
diccionarios \cite{thomas2015deriving}.

Sin embargo, y a pesar de los numerosos estudios que defienden las
bondades del aprendizaje basado en datos, este enfoque no parece haber
recalado aún en las aulas de lengua extranjera. \textcite{callies2016towards} sugiere
que el profesorado requiere de una serie de competencias y conocimientos
para poder aplicar con éxito esta metodología: deben estar
familiarizados con la lingüística de corpus y las herramientas
digitales, además de tener una base de conocimientos lingüísticos en
general. Además, los profesores también deben poseer competencias
pedagógicas que les permitan aplicar los corpus en el proceso de
enseñanza, diseñando materiales adecuados y pedagógicamente sólidos,
combinándolos bien con otras técnicas de enseñanza e incorporándolas al
contexto educativo. Se puede afirmar que es necesario formar al
profesorado futuro y en activo para que desarrollen todas estas
competencias mencionadas arriba.

La reticencia de los docentes de lengua extranjera a aplicar el corpus
en sus clases ha sido bien documentada en los últimos años por autores
como \textcite{romer2009corpus} o \textcite{tribble2015teaching}. Estos autores realizaron encuestas a
un representativo número de docentes. Los resultados de sus
investigaciones apuntan, principalmente, a la falta de formación en este
tipo de metodología, aunque también se mencionan la falta de recursos y
herramientas de corpus accesibles, intuitivas y de bajo coste. La
mayoría de los docentes afirmaban desconocer las diferentes formas en
que los corpus se pueden explotar en el aula y las habilidades
necesarias para la aplicación de estos conocimientos. También es
interesante mencionar que el nivel educativo al que pertenezcan los
docentes influye en su visión sobre este enfoque. En un reciente estudio
llevado a cabo por \textcite{crosthwaite2021voices}, docentes de inglés de
primaria y secundaria en Indonesia participaron en un taller formativo
sobre corpus y aprendizaje basado en datos. Tras preguntar a los
participantes por sus impresiones, los investigadores detectaron una
clara diferencia entre las opiniones de docentes de distintas etapas
educativas. Mientras que los profesores de secundaria veían cierto
potencial en el enfoque, aunque con ciertas reticencias, los de primaria
se mostraban mucho menos atraídos por este tipo de enseñanza en general.

No obstante, la investigación sobre el corpus y el aprendizaje basado en
datos ha dado resultados satisfactorios en cuanto a su acogida por parte
de alumnado. \textcite{geluso2014discovering} estudiaron las opiniones de un
grupo de estudiantes de inglés como lengua extranjera que utilizaron
herramientas de corpus para trabajar sus destrezas orales. Tras realizar
varias actividades, respondieron a un cuestionario tipo Likert donde se
les preguntaba sobre su experiencia. La mayoría de ellos consideraban
dichas herramientas como muy útiles pese a la novedad y el tiempo
empleado en saber utilizarlas adecuadamente. En la misma línea, los
estudios de \textcite{asik2016lexical} y \textcite{lin2016effects} destacan la buena
actitud demostrada por el estudiantado tras haber experimentado la
metodología de aprendizaje basado en datos. En el primer estudio, además
de un cuestionario, los participantes se sometieron a una entrevista en
grupo, donde mostraron una actitud muy positiva hacia esta metodología,
especialmente a la hora de trabajar con sinónimos y colocaciones.

En \textcite{lin2016effects}, tanto estudiantes de grado como de posgrado afirmaron
sentirse más autónomos y protagonistas de su proceso de aprendizaje. En
esta investigación se aplicaron tres acciones formativas: en la primera,
únicamente se trabajó con herramientas de corpus, en la segunda y
tercera acción formativa se combinó el aprendizaje basado en datos con
una metodología tradicional en distintas proporciones para cada grupo.
Si bien la actitud positiva se detectó en las tres acciones formativas,
fue en el primer grupo donde se obtuvieron las actitudes más favorables.
Las herramientas de corpus se utilizaron como base para el aprendizaje
de lengua extranjera a través de aplicaciones móviles. En este sentido,
\textcite{perezparedes2019mobile} quisieron saber si el uso de
herramientas de corpus en este tipo de dispositivo tenían la misma
acogida que en contextos de aula. Los resultados mostraron que, si bien
los usuarios sugirieron mejoras referentes al diseño de algunas
actividades, quedaron muy satisfechos con este tipo de aprendizaje.

\subsection{Autoeficacia en el aprendizaje de lenguas extranjeras}\label{sub-sec-autoeficaciaenelaprendizajedelenguas extranjeras}

Uno de los conceptos relacionados con las creencias y actitudes respecto
al aprendizaje es la autoeficacia. La literatura de los últimos veinte
años ha resaltado la importancia de las diferencias individuales a la
hora de abordar el proceso de enseñanza y aprendizaje de una lengua
extranjera \cite{chamorro2021actitudes,dornyei2019towards,munoz2019actitudes,you2016}. Autores como \textcite{dornyei2019towards} sugieren que, además de las aptitudes
y capacidades cognitivas de los aprendices, factores como la motivación,
las creencias y otras individualidades tienen un papel fundamental para
que un aprendiz tenga éxito en su adquisición de una lengua extranjera.
No obstante, el concepto de autoeficacia ha sido de estudio
relativamente reciente, comparado con otros factores individuales.

La autoeficacia comenzó a despertar interés a finales de los años 80 con
estudios en el marco de la teoría sociocognitiva de la mano de
investigaciones como las de \textcite{bandura1986social}. Sus investigaciones se
centran en explorar cómo lo que los individuos piensan de sí mismos y
del entorno influye en su manera de comportarse y de afrontar la
realización de determinadas tareas \textcite{bandura1986social}. En una reciente
definición \textcite{lin2014learning} concibe la autoeficacia como la confianza que una
persona tiene sobre ella misma respecto a su propia capacidad para
realizar una tarea, determinando así la cantidad de esfuerzo y tiempo
que está dispuesta a dedicar a dicha tarea. Investigaciones como la de
\textcite{linnenbrink2003role} han observado cómo la autoeficacia ayuda
de manera significativa en los procesos de aprendizaje a nivel tanto
cognitivo como motivacional.

La autoeficacia se ha relacionado con el uso de estrategias de
aprendizaje. \textcite{li2010empirical}, \textcite{anam2016language} o \textcite{saito2020strategy}
observaron que los aprendices con mayores niveles de autoeficacia hacían
uso de más y mejores estrategias de aprendizaje. De la misma manera,
\textcite{haro2021soler2021} explora los beneficios de la autoeficacia para alumnado de
traducción. Más concretamente en el campo de las lenguas extranjeras, la
relación entre la autoeficacia y el uso de estrategias es una de las
líneas más exploradas. Entre los estudios más recientes encontramos a
\textcite{tengwangwu2021metacognitive} o \textcite{bai2023role}. El primero de los estudios
es especialmente significativo dado el número de estudiantes que
formaron parte del mismo. Más de 500 estudiantes de inglés de origen
chino realizaron una encuesta donde se les preguntaba por diferentes
estrategias y aspectos motivacionales. Los resultados revelaron que la
autoeficacia predecía especialmente el uso de estrategias metacognitivas
y aumentaba la motivación por aprender en una época particularmente
complicada para la docencia como fue la pandemia por covid-19. En un
estudio incluso más reciente, \textcite{bai2023role} comprobaron que no solo
las estrategias como la planificación o la automonitorización se veían
favorecidas por la autoeficacia, sino también la motivación intrínseca
de los participantes a la hora de realizar distintas tareas en lengua
inglesa.

También encontramos investigaciones que muestran cómo la autoeficacia es
un buen predictor para determinar el éxito del estudiantado a la hora de
realizar ciertas actividades. Por ejemplo, \textcite{waddington2019developing} exploró qué
tareas comunicativas estaban más afectadas por un bajo nivel de
autoeficacia. Llegó a la conclusión de que tanto las destrezas
receptivas como las productivas se vieron mermadas por esa falta de
autoeficacia. Por su parte, \textcite{sardegna2018self} observaron que
un alto nivel de autoeficacia mejoraba los resultados de aprendizaje en
términos de pronunciación. Es la destreza de producción escrita la que
ha sido más explorada en este contexto. Como ejemplos podemos mencionar
a \textcite{woodrow2011college}, \textcite{villalon2013high} o \textcite{teng2021individual}. El
primero observó que la mayoría de los participantes presentaba cierto
grado de ansiedad a la hora de enfrentarse a una tarea escrita en
inglés. Los resultados del estudio apuntaban a que esa relación estaba
condicionada por la autoeficacia. Es decir, aquellos estudiantes con más
nivel de autoeficacia manejaban mejor sus niveles de ansiedad a la hora
de realizar la producción escrita. Por su parte, \textcite{villalon2013high} introdujeron la variable del género, queriendo averiguar
si había diferencias significativas entre estudiantes del género
femenino y masculino a la hora de cómo lidiaban con sus tareas de
producción escrita según una mayor o menor autoeficacia. Los
investigadores no encontraron diferencias significativas entre ambos
géneros, si bien confirmaban estudios anteriores respecto al papel de la
autoeficacia en la producción escrita en lengua extranjera. En la línea
de los dos anteriores, el trabajo de \textcite{teng2021individual} muestra que el nivel de
autoeficacia está íntimamente ligado tanto a estrategias cognitivas como
metacognitivas utilizadas en las tareas escritas.

Como se ha discutido arriba, los estudios que se refieren a la
autoeficacia en el contexto de enseñanza y aprendizaje de una lengua
extranjera se centran en la relación de esta con ciertas destrezas
comunicativas y con otras diferencias individuales como son la ansiedad
o la motivación. Sin embargo, y a pesar del interés que despierta dicha
autoeficacia actualmente en el aprendizaje de lenguas extranjeras,
faltan estudios que exploren su papel en el contexto de distintas
metodologías de enseñanza como puede ser las que adoptan herramientas
digitales de corpus.
\section{Objetivo de la Investigación}\label{sec-objetivodelainvestigación}

La presente investigación tiene como objetivo conocer el nivel de
autoeficacia y actitudes de un grupo de estudiantes del grado de
Educación Primaria, respecto a la metodología basada en datos y el uso
de las herramientas digitales de corpus.

La novedad de nuestro estudio radica, por tanto, en dos aspectos. Por un
lado, explora la autoeficacia en un contexto de aprendizaje basado en
datos. Esta metodología de implantación aún incipiente en el contexto de
enseñanza de lengua extranjera hace uso de herramientas de corpus
digitales como recurso de aula manipulable para el alumnado.

Por otro lado, el perfil de los participantes es concreto. La muestra
está formada por estudiantes universitarios, futuros docentes de lengua
extranjera en la etapa de educación primaria. Si bien es cierto que los
estudios tanto de autoeficacia como de aprendizaje basado en datos se
dan en contextos de enseñanza superior, el hecho de que sean futuros
docentes de lengua extranjera puede aportar una visión particular, la
cual no pueden proporcionar otras investigaciones cuya muestra es
estudiantado de otros grados universitarios. Es decir, hablamos de
docentes de lengua extranjera en formación, que podrán aplicar la
metodología basada en datos y las herramientas de corpus en un futuro
cercano. La opinión y actitudes que tengan sobre dicha metodología y
recursos puede determinar el papel de los mismos en las aulas de lengua
extranjera.
% !TeX root = main.tex

\section{Metodología}\label{sec-metodología}

La metodología de la investigación la entendemos como el conjunto de
procedimientos y técnicas que el equipo investigador ha utilizado en el
diseño, desarrollo y análisis del estudio. En este caso concreto, el
método utilizado ha sido de corte mixto, utilizando técnicas
cualitativas y cuantitativas. Las cualitativas se han basado en la
etnografía virtual de los datos generados en los sNOOC y las
conclusiones del juicio del equipo de expertos. Las cuantitativas
provienen de los cuestionarios de satisfacción del alumnado y de los
datos de interacción del alumnado en la plataforma de aprendizaje.


\subsection{Objetivos e hipótesis}\label{sub-sec-objetivosehipotesis}

El objetivo general de este estudio es analizar el proceso de creación
de redes comunicativas de estudiantes para la implementación de sNOOC
como método de evaluación continua en la UAD y su repercusión en la capa
social como modelo de formación mediática en personas de la tercera
edad. Con base en este objetivo general, los objetivos específicos hacen
referencia a:

\begin{itemize}
\item
Objetivo Específico 1 (OE1): Investigar las percepciones y opiniones
de las redes comunicativas de estudiantes respecto a la utilidad y
efectividad de un sNOOC como método de evaluación continua y su
impacto en la motivación hacia el aprendizaje.
\item
Objetivo Específico 2 (OE2): Examinar el proceso de desarrollo de las
redes comunicativas de estudiantes para la creación de contenidos de
los sNOOC, centrándose en el impacto del uso de pedagogías inclusivas,
IA y Metaverso en EAD.
\item
Objetivo Específico 3 (OE3): Evaluar el nivel de implicación activa de
las redes comunicativas de estudiantes en la plataforma de la UNED y
en la creación colaborativa del sNOOC en tmooc.es.
\end{itemize}

A continuación, se formulan las hipótesis para dar respuesta a las
relaciones causales:

\begin{itemize}
\item
Hipótesis 1 (H1-OE1): Si las redes comunicativas de estudiantes
perciben el sNOOC como una herramienta efectiva y útil para la
evaluación continua, aumentará su motivación intrínseca hacia el
aprendizaje y su participación en las actividades y recursos del
itinerario de aprendizaje propuesto.
\item
Hipótesis 2 (H2-0E2): Si el modelo sNOOC es diseñado y aplicado
considerando criterios pedagógicos inclusivos y herramientas
tecnológicas adecuadas, mejorará la comprensión de los contenidos por
parte de las redes comunicativas de estudiantes, incrementando su
satisfacción general con la experiencia de EAD.
\item
Hipótesis 3 (H3-OE3): Si el itinerario de aprendizaje en sNOOC está
basado en pedagogías inclusivas, se incrementará el compromiso activo
de las personas participantes, reflejado en una mayor interacción,
colaboración en equipo y corresponsabilidad en la construcción
colectiva del conocimiento.
\end{itemize}


\subsection{Muestra, instrumentos y análisis de
	datos}\label{sub-sec-muestrainstrumentos}
	
	El objeto de estudio de esta investigación son las interacciones del
	alumnado en la plataforma ALF de la UNED, contando con la participación
	de 79 personas, 57 mujeres y 22 hombres; 1 de nacionalidad croata y, el
	resto, española. Estos participantes han sido estudiantes del Máster
	Universitario en Educación y Comunicación en la Red y, dentro de este,
	de la asignatura ``Escenarios Virtuales para la participación'', una
	disciplina con contenidos relacionados con la educación mediática. En
	este caso concreto, para estructurar el método cuantitativo, se han
	utilizado los cuestionarios con preguntas diseñadas para recopilar datos
	cuantitativos correspondiente al curso 2023/2024.
	
	Referido a la plataforma ``tmooc.es'' donde este grupo de estudiantes
	creó los sNOOC, se ha realizado un análisis de estas propuestas tomando
	también esos entornos como objeto de estudio. Se tuvieron en cuenta los
	registros de datos relacionados con la dedicación en la creación de los
	sNOOC. Los sNOOC seleccionados son los siguientes: ``Introdúcete al
	mundo de Facebook'' (sN1), ``Senior 3.0'' (sN2), ``Correo electrónico
	son misterios: alfabetización digital para personas mayores'' (sN3),
	``Enredados en la edad dorada: dominar Facebook e Instagram con
	confianza'' (sN4), ``Healthy seniors network'' (sN5), ``Estas a un clic
	de conocer el mundo digital'' (sN6), ``Familias y aprendizaje en red''
	(sN7) y ``Google e inteligencia artificial, tus compañeros digitales''
	(sN8). En cuanto al enfoque cualitativo, se consideraron los datos
	generados a través de los sNOOC y las conclusiones del juicio de 22
	personas expertas internacionales, con el fin de validar hipótesis y
	evaluar riesgos o problemáticas presentes en el proyecto formativo. Para
	analizar los datos cuantitativos y cualitativos se utilizaron los
	programas SPSS y Atlas.ti, respectivamente. Estos aspectos se han
	organizado en categorías que se ajustan a las dimensiones de la
	educación inclusiva.

\section{Resultados y discusión}\label{sec-resultadosydiscusión}

La mayoría de los participantes afirmaron que trabajar con herramientas
digitales de corpus y realizar actividades de aprendizaje basado en
datos aumentaba sus expectativas respecto a su aprendizaje del inglés.
Así, un 63,6~\% se mostró, de alguna manera, de acuerdo al afirmar que
sentían que las herramientas de corpus les hacían avanzar en su proceso
de aprendizaje, mientras que, de manera similar, un porcentaje del 75,7~\% comentó lo mismo respecto a las actividades de aprendizaje basado en
datos. En general, el 93,9~\% recomendaría el uso de herramientas de
corpus para aprender inglés, incluso más de un 27,3~\% afirmó que estaba
no sólo de acuerdo sino muy de acuerdo con esta afirmación. Estas
afirmaciones son positivas para el aprendizaje desde el punto de vista
motivacional. De hecho \textcite{raoofi2012self} argumentan que el
creer que se es capaz de hacer una tarea aumenta significativamente las
posibilidades de éxito en dicha tarea. De manera similar, tras un
meta-análisis de los estudios más relevantes sobre la relación entre
autoeficacia y nivel de lengua extranjera, \textcite{wang2020} concluyeron
que a mayor nivel de autoeficacia, mayor competencia lingüística.

Estas respuestas están en consonancia con el estudio de \textcite{lin2016effects},
donde tras la aplicación de la nueva metodología los participantes
afirmaban haber mejorado en cuanto a sus creencias de lo que podían ser
capaces de aprender y hacer en la lengua extranjera. Así, no se debe
pasar por alto el papel de la autoeficacia y el autoconcepto del
estudiante de cara al éxito en su proceso de aprendizaje.
Investigaciones como la de \textcite{sun2020college} y \textcite{genc2016exploring}
muestran cómo este tipo de factores afectivos pueden llegar a ser
importantes predictores del desarrollo de la adquisición de una segunda
lengua. Por tanto, el hecho de que la metodología basada en datos y las
herramientas de corpus parezcan contribuir a aumentar el nivel de
autoeficacia del estudiantado nos muestra el potencial de estas como
elemento de refuerzo en el aprendizaje de la lengua extranjera.

Al preguntar por su capacidad de aprendizaje de inglés a través de esta
nueva metodología y recursos, una amplia mayoría afirmó sentirse capaz
de hacerlo y de tener éxito (94~\%). De manera más específica, el
cuestionario planteó afirmaciones respecto a la adquisición de
vocabulario, gramática y expresiones idiomáticas en la lengua inglesa
(los tres contenidos que se habían trabajado a través de corpus en el
estudio). La mayoría se veía capaz de mejorar los tres aspectos gracias
a dichos recursos digitales. Los participantes se vieron especialmente
capaces de mejorar el aprendizaje de vocabulario (100~\%), seguido de
las expresiones idiomáticas (94~\%) y por último la gramática (81,8~\%).
Estudios anteriores están en la línea de los resultados obtenidos aquí.
Por ejemplo, \textcite{saeedakhtar2020effect} constataron un cambio de
actitud hacia el aprendizaje de colocaciones entre estudiantes
universitarios de inglés como lengua extranjera. Tras una intervención
didáctica basada en el uso de recursos digitales de corpus, los
participantes afirmaron tener más ánimo de seguir estudiando
colocaciones a través de dicha metodología. Otras dos investigaciones
llevadas a cabo por \textcite{yao2019vocabulary} y \textcite{sinha2021efl} se centraron en el uso de
herramientas de corpus para el aprendizaje de léxico en segunda lengua.
También en estos casos los participantes mostraron una actitud muy
positiva hacia este tipo de recursos. Por último, \textcite{charles2014getting} o
\textcite{guilquin2021using} realizaron encuestas de satisfacción y actitud tras
aplicar herramientas de corpus para el aprendizaje de ciertas
construcciones gramaticales. Descubrieron que no solo la actitud era
positiva sino que realmente esta metodología basada en datos y el corpus
como recurso didáctico suponían una diferencia significativa en cuanto
al aprendizaje de dichas estructuras gramaticales.

Los participantes valoraron ocho afirmaciones sobre la dificultad,
utilidad, uso fuera y dentro de clase, entre otras, referidas a las
herramientas digitales de corpus y la metodología de aprendizaje basado
en datos. En cuanto a la dificultad del uso de las herramientas de
corpus, más de la mitad de los participantes se mostraron en desacuerdo
(54,4~\%) con la afirmación de que las herramientas eran difíciles de
utilizar. Nuestros resultados coinciden con estudios anteriores como el
de \textcite{daskalovska2015corpus}, que observó que los participantes de su estudio
habían manejado las herramientas digitales de corpus con una destreza
bastante alta, a pesar de que habían recibido una formación muy básica
sobre ellas, como ocurre en este estudio. Incluso con ese grado mínimo
de familiarización, la autora pudo comprobar que el estudiantado había
sacado partido a las herramientas. Una de las razones podría ser, entre
otras, la edad de los participantes. Tanto los del estudio de
Daskalovska como los nuestros pertenecen a la generación de nativos
digitales, los cuales de manera intuitiva se hacen en poco tiempo con
todo tipo de recursos digitales.

Preguntados por si consideraban dichas herramientas más útiles que otros
recursos que ya utilizaban, el 87,9~\% estuvo de acuerdo en que sí.
Además, manifestaron su deseo de seguir aprendiendo más sobre estas
herramientas digitales de corpus el 90,9~\%, de los cuales un 27,3~\%
respondió estar ``muy de acuerdo'' con esta afirmación. En cuanto a su
opinión sobre las actividades de aprendizaje basado en datos, el 69,7~\%
afirmó no encontrarlas aburridas, además de preferirlas a actividades
más tradicionales que se solían hacer en clase con una amplia mayoría
del 87,8~\%.

Por otro lado, deseaban seguir utilizando dichas herramientas tanto
dentro como fuera de clase en la mayor parte de los casos. Preguntados
sobre hasta qué punto estaban de acuerdo en utilizar de forma regular
las herramientas de corpus en el aula, el 63,6~\% respondió estar de
acuerdo e incluso muy de acuerdo. En el caso de si les gustaría
incorporar dichas herramientas a su aprendizaje fuera del aula, la
respuesta también fue muy positiva con el 57,5~\% de participantes
respondiendo que estaban de acuerdo o muy de acuerdo con ello. Los datos
aquí recabados son especialmente relevantes en el sentido de que estos
mismos participantes van a ser docentes de inglés en un futuro próximo.
Por tanto, el haber tenido una buena experiencia con las herramientas de
corpus los puede animar a querer aplicarlas con sus propios estudiantes,
dando la oportunidad a esos futuros alumnos a beneficiarse de los buenos
resultados que la investigación comentada arriba ha constatado. Así, se
puede contribuir a despejar las posibles dudas y recelos que este tipo
de recurso y enfoque ha despertado entre los docentes, según estudios
anteriores como los de \textcite{boulton2017corpus}, \textcite{caliskan2018training} o
\textcite{poole2022corpus}.

\section{Conclusion}\label{sec-conclusion}

The findings of this study suggest that the exercise of agency in initial education contexts is multifaceted. Like the findings of \posscite{mercer2011,mercer2012} empirical work focusing on language learners, the analysis of the narratives in this study indicates that agency is influenced by the intricate interconnectedness of various factors and elements coexisting in the participants' systems. It was possible to identify interpersonal factors, such as the perceived opportunities to interact with agents (human and non-human) in formal and informal online environments, as well as intrapersonal factors, such as the impact on emotions, attitudes, and beliefs about the best ways to learn. The data also show that the exercise of agency is dynamic, subject to change, and open, as it can be influenced by other systems.

Throughout the analysis, the relational nature of agency \cite{larsen2019} proved to be quite salient, as the data unveiled a reciprocal interaction between internal and external factors and the actions emerging from these relationships within contexts and their possibilities.

The results point to the potential of mobile devices in facilitating the exercise of agency among the participating pre-service teachers. These devices allow teachers to access information, speed up time, study, and even provide opportunities for distraction. When it comes to their praxis, these teachers also recognize the possibilities of mobile technology to motivate, bring the classroom to the 21st century, and engage learners. The potential for mobile technologies to impact social life was also evident from the data, as participants at various points in their stories emphasized how pervasive and important technology is to everyday life and citizenship.

In terms of the possible implications of this study for teacher education, one of the possible insights may stem from the recognition that in order to understand the possibilities of teacher agency – pre-service and in-service – it is important to consider the environments in which these agents circulate, the technologies and other agents with which they interact, the nested systems that make up their ecosystems, and the ways in which these dynamics affect and feed back into the interaction between intra- and interpersonal aspects.

We recognize the complexity of agency in the context studied, and although some dynamics and the complex fabric of teachers' agency have been highlighted in the data analyzed, further research can highlight other units of analysis in relation to inter- and intrapersonal aspects, elements, and systems that can further the understanding of the role of these "agents", thus contributing to research on teacher agency.



\printbibliography\label{sec-bib}
%conceptualization,datacuration,formalanalysis,funding,investigation,methodology,projadm,resources,software,supervision,validation,visualization,writing,review

\end{document}
