\section{Marco Teórico}\label{sec-marcoteórico.tex}
\subsection{La lingüística de corpus en la enseñanza de lenguas extranjeras} \label{sub-sec-lingüísticadecorpus}

La lingüística de corpus está adoptando un papel cada vez más relevante
en la enseñanza y aprendizaje de lenguas extranjeras \cite{sinclair2004how,okeeffe2007corpus,romer2009corpus,szudarski2018corpus}. Así, ofrece herramientas y enfoques pedagógicos que
permiten al estudiantado manejar grandes cantidades de texto auténtico
en formato digital, obteniendo información útil sobre el significado de
palabras clave y determinadas estructuras gramaticales partiendo de su
contexto de uso \cite{hanks2013}. \textcite{mcenery2011what} diferencian entre
usos indirectos y directos del corpus en el marco de la enseñanza de
lenguas.

Tradicionalmente, la lingüística de corpus ha tenido un papel indirecto
en el aprendizaje, siendo fuente para el diseño de materiales didácticos
para lengua extranjera o para la confección de diccionarios de
aprendices donde, entre otros, se ofrece información concreta sobre la
frecuencia y contexto de uso de las palabras. Podemos encontrar otras
aplicaciones indirectas en el ámbito de las lenguas para fines
específicos, donde son comunes los glosarios derivados de corpus de
especialidad, por ejemplo. Además, y en la línea de las aplicaciones
indirectas, no podemos olvidar los corpus de aprendices, cuyo análisis
ofrece esclarecedores datos derivados de errores comunes, y que abren
una valiosa ventana a la que asomarse a la hora de investigar el proceso
de adquisición de lenguas extranjeras \cite{meunier2002pedagogical}.

Por otro lado, respecto a la aplicación directa de los corpus, también
denominado aprendizaje basado en datos, autores como \textcite{gabrielatos2005corpora}
y \textcite{bernardini2004corpora} destacan sus efectos positivos en la adquisición de
una lengua extranjera. En el aprendizaje basado en datos los aprendices
manejan herramientas de corpus, las cuales les permiten explorar el
comportamiento de la lengua meta en su contexto de uso real. Así,
observan la frecuencia de uso de las palabras y sus concordancias, entre
otros. Estos procedimientos desarrollan la capacidad del estudiantado
para identificar patrones lingüísticos, hacer generalizaciones
fundamentadas en la lengua que están aprendiendo y examinar en detalle
cierto léxico y estructuras morfosintácticas, contribuyendo así a su
competencia metalingüística. El proceso de enseñanza y aprendizaje
tiende así a convertirse en una experiencia más motivadora y auténtica,
puesto que el estudiante consolida conocimientos que ya posee, a la vez
que descubre e integra nuevos \cite{boulton2012what}.

\textcite{zapata-ros2018} destacan el papel que tienen las
herramientas digitales de corpus y el aprendizaje basado en datos en el
desarrollo de la capacidad reflexiva y de razonamiento del alumnado, que
se materializa en distintas destrezas cognitivas como analizar, hacer
inferencias, interpretar o verificar. Como bien apunta \textcite{boulton2009testing},
el aprendizaje basado en datos complementa el enfoque comunicativo ya
que fomenta, por un lado, el aprendizaje inductivo, y por otro la
autonomía del propio estudiantado. A este respecto, \textcite{flowerdew2015corpus}
menciona tres teorías de aprendizaje que fundamentan esta metodología:
la llamada Noticing Hypothesis o Hipótesis de la Captación, el
Constructivismo y la Teoría Sociocultural de \textcite{vygotsky1978mind}. Dichas teorías están presentes en el aprendizaje basado en datos y el uso de
corpus, puesto que mediante estos se construye conocimiento a partir de
la búsqueda y la observación, se aprende por notoriedad de patrones y se
refuerza la negociación de significado.

Al interactuar con los corpus, el estudiantado aplica destrezas de
pensamiento crítico que implican la observación, la conciencia
lingüística, la formulación, la comprobación de hipótesis, y la
sensibilidad a la variación lingüística. En una tarea típica de
aprendizaje basado en datos, el estudiantado examina un nodo (una
palabra o una cadena de palabras) y su contexto inmediato, tanto a la
izquierda como a la derecha, así como, si lo desea, el contexto más
amplio del párrafo. Cada nodo aparece en una denominada línea de
concordancia, y a partir de ahí se exploran las pruebas lingüísticas y
se construyen hipótesis sobre la naturaleza del uso de la lengua.

Se considera, por tanto, un enfoque constructivista, inductivo y
centrado en el alumnado, que fomenta la autonomía y el aprendizaje por
descubrimiento. En este contexto de enseñanza, el profesorado actúa como
asesor y guía, facilitando, más que transmitiendo, conocimientos
lingüísticos. Se presentan, por tanto, una combinación de innovaciones
tecnológicas, educativas y varios recursos en línea que promueven la
inclusión de diferentes perspectivas pedagógicas dentro y fuera del
aula. Así, contribuye a que el estudiantado universitario, actualmente
en su mayoría nativos digitales, aprecien los enfoques basados en corpus
y los adapten a su aprendizaje de lengua extranjera. Así, a través de
este tipo de aprendizaje el alumnado puede llegar a sus propias
conclusiones, con una retención más sólida y prolongada y un mayor
conocimiento de la lengua meta. En base a estas premisas, se podría
afirmar que el estudiantado es menos dependiente del profesorado, y
aprende directamente de la lengua en contexto real en lugar de hacerlo
de recursos tradicionales como los libros de texto, las gramáticas o los
diccionarios \cite{thomas2015deriving}.

Sin embargo, y a pesar de los numerosos estudios que defienden las
bondades del aprendizaje basado en datos, este enfoque no parece haber
recalado aún en las aulas de lengua extranjera. \textcite{callies2016towards} sugiere
que el profesorado requiere de una serie de competencias y conocimientos
para poder aplicar con éxito esta metodología: deben estar
familiarizados con la lingüística de corpus y las herramientas
digitales, además de tener una base de conocimientos lingüísticos en
general. Además, los profesores también deben poseer competencias
pedagógicas que les permitan aplicar los corpus en el proceso de
enseñanza, diseñando materiales adecuados y pedagógicamente sólidos,
combinándolos bien con otras técnicas de enseñanza e incorporándolas al
contexto educativo. Se puede afirmar que es necesario formar al
profesorado futuro y en activo para que desarrollen todas estas
competencias mencionadas arriba.

La reticencia de los docentes de lengua extranjera a aplicar el corpus
en sus clases ha sido bien documentada en los últimos años por autores
como \textcite{romer2009corpus} o \textcite{tribble2015teaching}. Estos autores realizaron encuestas a
un representativo número de docentes. Los resultados de sus
investigaciones apuntan, principalmente, a la falta de formación en este
tipo de metodología, aunque también se mencionan la falta de recursos y
herramientas de corpus accesibles, intuitivas y de bajo coste. La
mayoría de los docentes afirmaban desconocer las diferentes formas en
que los corpus se pueden explotar en el aula y las habilidades
necesarias para la aplicación de estos conocimientos. También es
interesante mencionar que el nivel educativo al que pertenezcan los
docentes influye en su visión sobre este enfoque. En un reciente estudio
llevado a cabo por \textcite{crosthwaite2021voices}, docentes de inglés de
primaria y secundaria en Indonesia participaron en un taller formativo
sobre corpus y aprendizaje basado en datos. Tras preguntar a los
participantes por sus impresiones, los investigadores detectaron una
clara diferencia entre las opiniones de docentes de distintas etapas
educativas. Mientras que los profesores de secundaria veían cierto
potencial en el enfoque, aunque con ciertas reticencias, los de primaria
se mostraban mucho menos atraídos por este tipo de enseñanza en general.

No obstante, la investigación sobre el corpus y el aprendizaje basado en
datos ha dado resultados satisfactorios en cuanto a su acogida por parte
de alumnado. \textcite{geluso2014discovering} estudiaron las opiniones de un
grupo de estudiantes de inglés como lengua extranjera que utilizaron
herramientas de corpus para trabajar sus destrezas orales. Tras realizar
varias actividades, respondieron a un cuestionario tipo Likert donde se
les preguntaba sobre su experiencia. La mayoría de ellos consideraban
dichas herramientas como muy útiles pese a la novedad y el tiempo
empleado en saber utilizarlas adecuadamente. En la misma línea, los
estudios de \textcite{asik2016lexical} y \textcite{lin2016effects} destacan la buena
actitud demostrada por el estudiantado tras haber experimentado la
metodología de aprendizaje basado en datos. En el primer estudio, además
de un cuestionario, los participantes se sometieron a una entrevista en
grupo, donde mostraron una actitud muy positiva hacia esta metodología,
especialmente a la hora de trabajar con sinónimos y colocaciones.

En \textcite{lin2016effects}, tanto estudiantes de grado como de posgrado afirmaron
sentirse más autónomos y protagonistas de su proceso de aprendizaje. En
esta investigación se aplicaron tres acciones formativas: en la primera,
únicamente se trabajó con herramientas de corpus, en la segunda y
tercera acción formativa se combinó el aprendizaje basado en datos con
una metodología tradicional en distintas proporciones para cada grupo.
Si bien la actitud positiva se detectó en las tres acciones formativas,
fue en el primer grupo donde se obtuvieron las actitudes más favorables.
Las herramientas de corpus se utilizaron como base para el aprendizaje
de lengua extranjera a través de aplicaciones móviles. En este sentido,
\textcite{perezparedes2019mobile} quisieron saber si el uso de
herramientas de corpus en este tipo de dispositivo tenían la misma
acogida que en contextos de aula. Los resultados mostraron que, si bien
los usuarios sugirieron mejoras referentes al diseño de algunas
actividades, quedaron muy satisfechos con este tipo de aprendizaje.

\subsection{Autoeficacia en el aprendizaje de lenguas extranjeras}\label{sub-sec-autoeficaciaenelaprendizajedelenguas extranjeras}

Uno de los conceptos relacionados con las creencias y actitudes respecto
al aprendizaje es la autoeficacia. La literatura de los últimos veinte
años ha resaltado la importancia de las diferencias individuales a la
hora de abordar el proceso de enseñanza y aprendizaje de una lengua
extranjera \cite{chamorro2021actitudes,dornyei2019towards,munoz2019actitudes,you2016}. Autores como \textcite{dornyei2019towards} sugieren que, además de las aptitudes
y capacidades cognitivas de los aprendices, factores como la motivación,
las creencias y otras individualidades tienen un papel fundamental para
que un aprendiz tenga éxito en su adquisición de una lengua extranjera.
No obstante, el concepto de autoeficacia ha sido de estudio
relativamente reciente, comparado con otros factores individuales.

La autoeficacia comenzó a despertar interés a finales de los años 80 con
estudios en el marco de la teoría sociocognitiva de la mano de
investigaciones como las de \textcite{bandura1986social}. Sus investigaciones se
centran en explorar cómo lo que los individuos piensan de sí mismos y
del entorno influye en su manera de comportarse y de afrontar la
realización de determinadas tareas \textcite{bandura1986social}. En una reciente
definición \textcite{lin2014learning} concibe la autoeficacia como la confianza que una
persona tiene sobre ella misma respecto a su propia capacidad para
realizar una tarea, determinando así la cantidad de esfuerzo y tiempo
que está dispuesta a dedicar a dicha tarea. Investigaciones como la de
\textcite{linnenbrink2003role} han observado cómo la autoeficacia ayuda
de manera significativa en los procesos de aprendizaje a nivel tanto
cognitivo como motivacional.

La autoeficacia se ha relacionado con el uso de estrategias de
aprendizaje. \textcite{li2010empirical}, \textcite{anam2016language} o \textcite{saito2020strategy}
observaron que los aprendices con mayores niveles de autoeficacia hacían
uso de más y mejores estrategias de aprendizaje. De la misma manera,
\textcite{haro2021soler2021} explora los beneficios de la autoeficacia para alumnado de
traducción. Más concretamente en el campo de las lenguas extranjeras, la
relación entre la autoeficacia y el uso de estrategias es una de las
líneas más exploradas. Entre los estudios más recientes encontramos a
\textcite{tengwangwu2021metacognitive} o \textcite{bai2023role}. El primero de los estudios
es especialmente significativo dado el número de estudiantes que
formaron parte del mismo. Más de 500 estudiantes de inglés de origen
chino realizaron una encuesta donde se les preguntaba por diferentes
estrategias y aspectos motivacionales. Los resultados revelaron que la
autoeficacia predecía especialmente el uso de estrategias metacognitivas
y aumentaba la motivación por aprender en una época particularmente
complicada para la docencia como fue la pandemia por covid-19. En un
estudio incluso más reciente, \textcite{bai2023role} comprobaron que no solo
las estrategias como la planificación o la automonitorización se veían
favorecidas por la autoeficacia, sino también la motivación intrínseca
de los participantes a la hora de realizar distintas tareas en lengua
inglesa.

También encontramos investigaciones que muestran cómo la autoeficacia es
un buen predictor para determinar el éxito del estudiantado a la hora de
realizar ciertas actividades. Por ejemplo, \textcite{waddington2019developing} exploró qué
tareas comunicativas estaban más afectadas por un bajo nivel de
autoeficacia. Llegó a la conclusión de que tanto las destrezas
receptivas como las productivas se vieron mermadas por esa falta de
autoeficacia. Por su parte, \textcite{sardegna2018self} observaron que
un alto nivel de autoeficacia mejoraba los resultados de aprendizaje en
términos de pronunciación. Es la destreza de producción escrita la que
ha sido más explorada en este contexto. Como ejemplos podemos mencionar
a \textcite{woodrow2011college}, \textcite{villalon2013high} o \textcite{teng2021individual}. El
primero observó que la mayoría de los participantes presentaba cierto
grado de ansiedad a la hora de enfrentarse a una tarea escrita en
inglés. Los resultados del estudio apuntaban a que esa relación estaba
condicionada por la autoeficacia. Es decir, aquellos estudiantes con más
nivel de autoeficacia manejaban mejor sus niveles de ansiedad a la hora
de realizar la producción escrita. Por su parte, \textcite{villalon2013high} introdujeron la variable del género, queriendo averiguar
si había diferencias significativas entre estudiantes del género
femenino y masculino a la hora de cómo lidiaban con sus tareas de
producción escrita según una mayor o menor autoeficacia. Los
investigadores no encontraron diferencias significativas entre ambos
géneros, si bien confirmaban estudios anteriores respecto al papel de la
autoeficacia en la producción escrita en lengua extranjera. En la línea
de los dos anteriores, el trabajo de \textcite{teng2021individual} muestra que el nivel de
autoeficacia está íntimamente ligado tanto a estrategias cognitivas como
metacognitivas utilizadas en las tareas escritas.

Como se ha discutido arriba, los estudios que se refieren a la
autoeficacia en el contexto de enseñanza y aprendizaje de una lengua
extranjera se centran en la relación de esta con ciertas destrezas
comunicativas y con otras diferencias individuales como son la ansiedad
o la motivación. Sin embargo, y a pesar del interés que despierta dicha
autoeficacia actualmente en el aprendizaje de lenguas extranjeras,
faltan estudios que exploren su papel en el contexto de distintas
metodologías de enseñanza como puede ser las que adoptan herramientas
digitales de corpus.