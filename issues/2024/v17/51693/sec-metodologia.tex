\section{Metodología}\label{sec-metodología}
\subsection{Participantes}\label{sub-sec-participantes}

En el estudio participaron 33 estudiantes, de las que 20 eran mujeres
(60,6~\%) y 13 eran hombres (39,4~\%) de último año del grado de
Educación Primaria de la Universidad de Castilla-La Mancha. Todos ellos
cursaban la mención en inglés como lengua extranjera, que los habilitaba
para impartir dicha asignatura en centros escolares de educación básica
en España. Sus edades comprendían entre los 20 y los 25 años, excepto
uno de los participantes que tenía más de 25 años. Todos ellos contaban
con un nivel medio-alto o avanzado de inglés certificado,
correspondiente a B2-C1 según el \textcite{VolumencomplementariodelMarcoComúnEuropeodeReferenciaparalasLenguas}. Todos eran hablantes nativos de la lengua española y
no conocían ninguna lengua adicional a excepción del inglés. Ninguno de
ellos contaba con experiencia previa en el manejo de herramientas
digitales de corpus.
\subsection{Herramientas de corpus y actividades}\label{sub-sec-herramientasdecorpusyactividades}

Para llevar a cabo el estudio se utilizaron dos herramientas de corpus:
el \textcite{bnc2002} y el \textcite{sketchengine2018}. El primero fue recopilado por la
Oxford University Press entre los años ochenta y principios de los
noventa. Contiene 100 millones de palabras de una amplia gama de
géneros, como el lenguaje hablado, la ficción, las revistas, los medios
de comunicación escritos y diversos textos académicos. Es de libre
acceso, y sólo requiere registrarse en la herramienta. Sin embargo, la
versión gratuita sólo permite 20 búsquedas al día y no permite comparar
textos. No obstante, estas limitaciones no afectaron al desarrollo de
esta investigación, puesto que las acciones formativas que se llevaron a
cabo suponían un número de búsquedas menor al límite establecido por la
herramienta.

En cuanto a SKELL, es una herramienta especialmente diseñada para el
aprendizaje de lenguas extranjeras. Además del inglés, trata el alemán y
el italiano, entre otros. Permite verificar fácilmente la búsqueda de
palabras y expresiones. Para acceder a esta herramienta no fue necesario
registrarse ni pagar. Contiene más de mil millones de palabras y 57
millones de frases. Así pues, el corpus no contiene documentos
completos, sino frases ordenadas según la calidad del texto.

Ambas herramientas digitales ofrecen un buen ejemplo de cómo se utiliza
el inglés en distintos contextos como puede ser el informal, el formal o
el académico. Así, al teclear una palabra o expresión, estas
herramientas ofrecen tres opciones: ejemplos reales en los que se ha
utilizado esa palabra, colocaciones de esa palabra en distintas
posiciones oracionales y sinónimos. Dichos corpus suponen un enorme
potencial didáctico para el aula de lengua extranjera.

La \Cref{tab-01} muestra las características de las dos herramientas de corpus
utilizadas en esta investigación. Ambas herramientas son muy intuitivas
y fáciles de manejar, especialmente para nativos digitales y personas
con una competencia digital básica.

\begin{table}[htbp]
\centering
\small
\begin{threeparttable}
\caption{Características principales de las herramientas de corpus.}
\label{tab-01}
\centering
\begin{tabular}{
  >{\raggedright\arraybackslash}p{4.5cm}
  >{\raggedright\arraybackslash}p{2.25cm}
  >{\raggedright\arraybackslash}p{2.25cm}
  >{\raggedright\arraybackslash}p{4cm}
  }
\toprule
Herramientas & Número de palabras & Tipo & Acceso \\
\midrule
\emph{British National Corpus} (BNC)\newline
{\url{www.english-corpora.org/bnc}}
& 100 millones & Inglés general\newline
Varios géneros textuales & Se requiere registro\newline
Limitación diaria de búsquedas\newline
No permite comparación de textos\\
\emph{Sketch Engine for Language Learning} (SKELL)\newline
{\url{https://skell.sketchengine.eu}}
& 1000 millones & Inglés general & Sin registro\newline
Sin limitación de búsqueda \\
\bottomrule
\end{tabular}
\source{Elaboración propia.}
\end{threeparttable}
\end{table}


Los participantes utilizaron estas herramientas de corpus en cuatro
actividades. En la primera actividad, a los participantes, de manera
individual, se les pedía acceder a SKELL y buscar distintas colocaciones
con los verbos \emph{make} y \emph{do}. Tras la búsqueda, se les animaba
a pensar en el tipo de palabras que acompañan a ambos verbos y la
función de esas palabras en una oración. La segunda actividad consistía
en familiarizarse con una serie de expresiones idiomáticas relacionadas
con partes del cuerpo. Se trabajaron cinco expresiones: \emph{all ears},
\emph{cold feet}, \emph{cold shoulder}, \emph{eagle eye} y \emph{eye to
eye}. En este caso los participantes realizaron la actividad en parejas.
Se les pidió que accedieran a la herramienta del BNC y que hicieran la
búsqueda de cada una de las expresiones idiomáticas, marcando la casilla
\emph{Context}. Las expresiones idiomáticas en el BNC aparecen
contextualizadas en oraciones reales, y los participantes tuvieron que
inferir el significado de cada una de ellas atendiendo al contexto en el
que aparecen. En segundo lugar, completaron una tabla donde debían
escribir ejemplos de esas expresiones. Como último paso en esta
actividad, se les pidió que usaran esas expresiones en un breve diálogo
con su compañero de trabajo. Se estimó una duración de 20 minutos para
completarla.

La tercera actividad versó sobre el estilo indirecto o \emph{reported
speech}, estaba diseñada para realizarse en 15 minutos, y está inspirada
en una propuesta de \textcite{brown2023using}. Los participantes utilizaron SKELL
para llevar a cabo la búsqueda de \emph{I told} y \emph{I said} en la
sección \emph{Examples}. Tras observar el comportamiento de los dos
verbos, en parejas, completaron un texto con algunas reglas y ejemplos
de uso de esas expresiones en el estilo indirecto. Por último, la cuarta
actividad consistió en diferenciar el significado de
adjetivos/participios terminados en \emph{-ed} o -\emph{ing}. Esta
actividad es una adaptación de una propuesta previa de Sutherland (2023)
y se llevó a cabo a través de la herramienta del BNC, con una duración
de 20 minutos. Se partió de la distinción entre \emph{bored} y
\emph{boring}. Tras discutir el significado de ambas, se les pidió a los
participantes que utilizaran el BNC para buscar ambos términos y
observar qué sustantivos acompañaban a cada una de ellas. Tras esto
categorizaron dichos sustantivos en ``humanos'', ``no-humanos'' y
``pronombres'', y discutieron con el resto de participantes qué tipo de
elementos acompañan a \emph{boring} y \emph{bored} atendiendo a su
clasificación. Como parte final de esta actividad, se les presentaron
una serie de adjetivos/participios, (\emph{annoying}/\emph{annoyed},
\emph{confusing}/\emph{confused}, \emph{exciting}/\emph{excited},
\emph{disappointing}/\emph{disappointed}, entre otros), cuyo significado
deberían debatir y reflexionar sobre ello, del mismo modo que lo habían
hecho con \emph{boring} y \emph{bored}. La \Cref{tab-02} recoge los datos
principales de las cuatro actividades.

\begin{table}[h!]
\centering
\small
\begin{threeparttable}
\caption{Actividades realizadas.}
\label{tab-02}
\begin{tabular}{ll>{\raggedright\arraybackslash}p{6cm}l}
\toprule
Actividad & Herramienta & Descripción & Duración\\
\midrule
Colocaciones con \emph{make} y \emph{do} & SKELL & Aprendizaje de
distintas colocaciones con los verbos \emph{make} y \emph{do} y
posterior reflexión sobre ciertas categorías de palabras & 10 minutos \\
Expresiones idiomáticas & BNC & Aprendizaje de expresiones idiomáticas
que contienen partes del cuerpo e inferencia de significado & 20
minutos \\
Estilo indirecto & SKELL & Distinción y uso de \emph{tell} y \emph{say}
en el estilo indirecto & 15 minutos \\
-ed vs. -ing & BNC & Aprendizaje de la diferencia entre las
terminaciones -ed e -ing en adjetivos/participios & 20 minutos \\
\bottomrule
\end{tabular}
\source{Elaboración propia.}
\end{threeparttable}
\end{table}

\subsection{Instrumento y procedimiento}\label{sub-sec-instrumento yprocedimiento}

Previamente a la intervención didáctica, se explicó a los participantes,
brevemente y de manera clara, qué es un corpus, qué es la metodología
ABD, y su utilidad en el aprendizaje de idiomas. También se presentaron
las herramientas digitales de corpus de las que se iba a hacer uso, y
unas pequeñas indicaciones de cómo manejarlas. Una vez los participantes
se familiarizaron con las herramientas, se procedió a realizar las
tareas propuestas. La intervención didáctica duró cuatro semanas. Cada
semana los participantes trabajaron una actividad. Esta actividad se
realizaba dentro del aula de lengua extranjera, y se integraba en la
lección de ese día.

Una vez realizadas las actividades por parte de los participantes, se
procedió a la recopilación de datos. Para ello se utilizó un
cuestionario diseñado \emph{ad hoc} que constaba de 14 ítems. Se tomaron
como referencia los instrumentos utilizados por \textcite{chen2019introducing}, \textcite{lin2016effects} y \textcite{yoon2004}. Los participantes
debían indicar su grado de acuerdo o desacuerdo en una escala Likert de
seis puntos, donde 1 indicaba ``muy en desacuerdo'' y 6 ``muy de
acuerdo''. Todos los ítems se presentaron en castellano, puesto que
queríamos asegurar que los participantes comprendían lo que se les
planteaba.

El cuestionario comienza dando unas sencillas indicaciones y pidiendo a
los participantes algunos datos identificativos como edad, género, años
estudiando inglés, y conocimiento de otras lenguas extranjeras. Los
ítems están organizados en dos secciones: una relacionada con las
actitudes sobre las herramientas de corpus y las actividades de
aprendizaje basado en datos, y otra sección donde los ítems pretenden
reflejar el nivel de autoeficacia de los participantes respecto a las
herramientas de corpus y las actividades de aprendizaje basado en datos.
El cuestionario hace referencia a aspectos sobre la dificultad de uso,
el nivel de utilidad, o la aplicación fuera del aula de las herramientas
digitales de corpus; también se pregunta a los participantes por las
actividades de aprendizaje basado en datos para la adquisición de
distintos aspectos de la lengua inglesa, además de compararlas con
actividades tradicionales. El diseño y la distribución del cuestionario
se realizó en formato electrónico a través de \emph{Google Forms}.
Además de las instrucciones escritas en el propio cuestionario, se
explicó a los participantes oralmente cómo proceder. Se realizó en clase
y se les informó de que las respuestas eran anónimas y de que no había
límite de tiempo.
