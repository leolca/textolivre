\section{Resultados y discusión}\label{sec-resultadosydiscusión}

La mayoría de los participantes afirmaron que trabajar con herramientas
digitales de corpus y realizar actividades de aprendizaje basado en
datos aumentaba sus expectativas respecto a su aprendizaje del inglés.
Así, un 63,6~\% se mostró, de alguna manera, de acuerdo al afirmar que
sentían que las herramientas de corpus les hacían avanzar en su proceso
de aprendizaje, mientras que, de manera similar, un porcentaje del 75,7~\% comentó lo mismo respecto a las actividades de aprendizaje basado en
datos. En general, el 93,9~\% recomendaría el uso de herramientas de
corpus para aprender inglés, incluso más de un 27,3~\% afirmó que estaba
no sólo de acuerdo sino muy de acuerdo con esta afirmación. Estas
afirmaciones son positivas para el aprendizaje desde el punto de vista
motivacional. De hecho \textcite{raoofi2012self} argumentan que el
creer que se es capaz de hacer una tarea aumenta significativamente las
posibilidades de éxito en dicha tarea. De manera similar, tras un
meta-análisis de los estudios más relevantes sobre la relación entre
autoeficacia y nivel de lengua extranjera, \textcite{wang2020} concluyeron
que a mayor nivel de autoeficacia, mayor competencia lingüística.

Estas respuestas están en consonancia con el estudio de \textcite{lin2016effects},
donde tras la aplicación de la nueva metodología los participantes
afirmaban haber mejorado en cuanto a sus creencias de lo que podían ser
capaces de aprender y hacer en la lengua extranjera. Así, no se debe
pasar por alto el papel de la autoeficacia y el autoconcepto del
estudiante de cara al éxito en su proceso de aprendizaje.
Investigaciones como la de \textcite{sun2020college} y \textcite{genc2016exploring}
muestran cómo este tipo de factores afectivos pueden llegar a ser
importantes predictores del desarrollo de la adquisición de una segunda
lengua. Por tanto, el hecho de que la metodología basada en datos y las
herramientas de corpus parezcan contribuir a aumentar el nivel de
autoeficacia del estudiantado nos muestra el potencial de estas como
elemento de refuerzo en el aprendizaje de la lengua extranjera.

Al preguntar por su capacidad de aprendizaje de inglés a través de esta
nueva metodología y recursos, una amplia mayoría afirmó sentirse capaz
de hacerlo y de tener éxito (94~\%). De manera más específica, el
cuestionario planteó afirmaciones respecto a la adquisición de
vocabulario, gramática y expresiones idiomáticas en la lengua inglesa
(los tres contenidos que se habían trabajado a través de corpus en el
estudio). La mayoría se veía capaz de mejorar los tres aspectos gracias
a dichos recursos digitales. Los participantes se vieron especialmente
capaces de mejorar el aprendizaje de vocabulario (100~\%), seguido de
las expresiones idiomáticas (94~\%) y por último la gramática (81,8~\%).
Estudios anteriores están en la línea de los resultados obtenidos aquí.
Por ejemplo, \textcite{saeedakhtar2020effect} constataron un cambio de
actitud hacia el aprendizaje de colocaciones entre estudiantes
universitarios de inglés como lengua extranjera. Tras una intervención
didáctica basada en el uso de recursos digitales de corpus, los
participantes afirmaron tener más ánimo de seguir estudiando
colocaciones a través de dicha metodología. Otras dos investigaciones
llevadas a cabo por \textcite{yao2019vocabulary} y \textcite{sinha2021efl} se centraron en el uso de
herramientas de corpus para el aprendizaje de léxico en segunda lengua.
También en estos casos los participantes mostraron una actitud muy
positiva hacia este tipo de recursos. Por último, \textcite{charles2014getting} o
\textcite{guilquin2021using} realizaron encuestas de satisfacción y actitud tras
aplicar herramientas de corpus para el aprendizaje de ciertas
construcciones gramaticales. Descubrieron que no solo la actitud era
positiva sino que realmente esta metodología basada en datos y el corpus
como recurso didáctico suponían una diferencia significativa en cuanto
al aprendizaje de dichas estructuras gramaticales.

Los participantes valoraron ocho afirmaciones sobre la dificultad,
utilidad, uso fuera y dentro de clase, entre otras, referidas a las
herramientas digitales de corpus y la metodología de aprendizaje basado
en datos. En cuanto a la dificultad del uso de las herramientas de
corpus, más de la mitad de los participantes se mostraron en desacuerdo
(54,4~\%) con la afirmación de que las herramientas eran difíciles de
utilizar. Nuestros resultados coinciden con estudios anteriores como el
de \textcite{daskalovska2015corpus}, que observó que los participantes de su estudio
habían manejado las herramientas digitales de corpus con una destreza
bastante alta, a pesar de que habían recibido una formación muy básica
sobre ellas, como ocurre en este estudio. Incluso con ese grado mínimo
de familiarización, la autora pudo comprobar que el estudiantado había
sacado partido a las herramientas. Una de las razones podría ser, entre
otras, la edad de los participantes. Tanto los del estudio de
Daskalovska como los nuestros pertenecen a la generación de nativos
digitales, los cuales de manera intuitiva se hacen en poco tiempo con
todo tipo de recursos digitales.

Preguntados por si consideraban dichas herramientas más útiles que otros
recursos que ya utilizaban, el 87,9~\% estuvo de acuerdo en que sí.
Además, manifestaron su deseo de seguir aprendiendo más sobre estas
herramientas digitales de corpus el 90,9~\%, de los cuales un 27,3~\%
respondió estar ``muy de acuerdo'' con esta afirmación. En cuanto a su
opinión sobre las actividades de aprendizaje basado en datos, el 69,7~\%
afirmó no encontrarlas aburridas, además de preferirlas a actividades
más tradicionales que se solían hacer en clase con una amplia mayoría
del 87,8~\%.

Por otro lado, deseaban seguir utilizando dichas herramientas tanto
dentro como fuera de clase en la mayor parte de los casos. Preguntados
sobre hasta qué punto estaban de acuerdo en utilizar de forma regular
las herramientas de corpus en el aula, el 63,6~\% respondió estar de
acuerdo e incluso muy de acuerdo. En el caso de si les gustaría
incorporar dichas herramientas a su aprendizaje fuera del aula, la
respuesta también fue muy positiva con el 57,5~\% de participantes
respondiendo que estaban de acuerdo o muy de acuerdo con ello. Los datos
aquí recabados son especialmente relevantes en el sentido de que estos
mismos participantes van a ser docentes de inglés en un futuro próximo.
Por tanto, el haber tenido una buena experiencia con las herramientas de
corpus los puede animar a querer aplicarlas con sus propios estudiantes,
dando la oportunidad a esos futuros alumnos a beneficiarse de los buenos
resultados que la investigación comentada arriba ha constatado. Así, se
puede contribuir a despejar las posibles dudas y recelos que este tipo
de recurso y enfoque ha despertado entre los docentes, según estudios
anteriores como los de \textcite{boulton2017corpus}, \textcite{caliskan2018training} o
\textcite{poole2022corpus}.
