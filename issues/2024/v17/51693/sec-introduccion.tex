\section{Introducción}\label{sec-introducción}

Las nuevas tecnologías van adquiriendo un papel cada vez más relevante
en la enseñanza de lenguas. La era digital ha calado también en las
aulas a través de dispositivos electrónicos como las pizarras
digitales, las tabletas electrónicas o incluso los dispositivos
móviles y las redes sociales. Pese a que la cultura digital ya es un
hecho, existen aún ciertos recursos y enfoques relacionados con ella
que se resisten a formar parte de la enseñanza de lenguas. Este es el
caso de las herramientas digitales de corpus y de la metodología
basada en datos derivada de dichas herramientas. A este respecto,
precisamente, se presenta este trabajo donde se investiga la
aplicación de tales recursos digitales para el aprendizaje de inglés
como lengua extranjera en el contexto universitario español.