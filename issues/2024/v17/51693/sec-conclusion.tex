\section{Conclusión}\label{sec-conclusión}

El presente estudio explora el nivel de autoeficacia y las actitudes de
un grupo de profesorado en formación, que estudia inglés como lengua
extranjera en el grado de Educación Primaria con respecto a dos
herramientas digitales de corpus y a la metodología de aprendizaje
basado en datos. Los resultados extraídos del cuestionario apuntan a una
buena acogida tanto del corpus como de la metodología. Además, las
respuestas aportadas muestran que el aprendizaje basado en datos y las
herramientas digitales de corpus favorecen la autoeficacia en los
estudiantes, demostrando predisposición, confianza y motivación en su
proceso de aprendizaje de la lengua extranjera. Los participantes
encontraron el corpus interesante e intuitivo, afirmando que aumentaba
su grado de autonomía y seguridad, reconociendo su utilidad, y la
facilidad en su manejo para seguir aplicándolos tanto en el aula como
fuera de ella en el estudio autónomo.

Pese a su desconocimiento previo, el interés y la buena actitud
expresados por los participantes hacia la metodología basada en corpus
resulta alentador para el profesorado que desea incorporarla a sus
clases de lengua extranjera. Por otro lado, si bien parece que los
participantes se han adaptado a esta metodología, sí que es importante
que el profesorado también esté preparado para ello. De esta manera,
para que la metodología basada en datos y el corpus como recurso
didáctico funcionen, debe existir una formación específica del
profesorado, además de un diseño minucioso de las actividades con un
adecuado andamiaje. Esto solamente sería posible si el profesorado que
opte por ello esté debidamente entrenado \cite{smart2014role}.

De cara a futuras investigaciones, se deberían abordar algunas
limitaciones que presenta nuestro estudio. Por un lado, se contó con un
grupo reducido de participantes que impiden realizar generalizaciones
respecto a los resultados obtenidos. Para próximos trabajos se espera
contar con un número mayor de participantes, donde se incluyan, además,
distintitos niveles de competencia lingüística y otros contextos
educativos como la educación básica. Por otro lado, también se debe
tener en cuenta la corta duración de la intervención didáctica, si bien
se decidió así para evitar el efecto de posibles factores como el
abandono de algunos participantes.
