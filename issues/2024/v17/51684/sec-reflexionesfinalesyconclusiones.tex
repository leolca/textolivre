\section{Reflexiones finales y conclusiones}\label{sec-reflexionesfinalesyconclusiones}

La experiencia desarrollada en este artículo tenía como objetivo
responder a preguntas vinculadas con los desafíos que tiene la lectura
en la actualidad. Dichos desafíos están implicados con las literacidades
digitales de los docentes en formación de lengua y literatura. Para
ello, interesaba observar de qué manera interactuaban sus literacidades
con el diseño de proyectos que integran la lectura literaria a través de
la pedagogía de las multiliteracidades.

Uno de los primeros objetivos de la experiencia pretendía que los
profesores en formación desarrollaran sus literacidades digitales en los
portafolios reflexivos digitales. En relación con este objetivo, si bien
la muestra es muy acotada, es significativa en cuanto a que proporciona
una idea del tipo de literacidades digitales que posee un grupo de
estudiantes de pedagogía en un contexto académico y en el que deben
reflexionar sobre sus experiencias de mediación de la literatura.

Dentro de las dimensiones de las literacidades digitales se puede
señalar que, en estos futuros docentes, el ámbito digital es natural
dentro de sus prácticas, considerando, además su rango de edad. Esto se
demuestra por el despliegue de herramientas utilizadas en la plataforma
Sites y la idea de diseño de cada uno de los portafolios. En ellos se
pudo ver prácticas próximas al lenguaje de las redes sociales y
culturales como el uso de memes, la gamificación e itinerarios
proporcionados por las diversas plataformas utilizadas, que podría ser
una aproximación a una competencia digital pedagógica como la han
propuesto Silva et al. (2019), Sandoval Rubilar; Rodríguez Alveal;
Maldonado Fuentes (2017) y Ayala (2015), aunque de manera indirecta, ya
que el despliegue de literacidades digitales podría aportar algunas
luces a la problemática sobre la competencia digital en la formación
inicial docente.

El diseño de los portafolios y los itinerarios propuestos pueden
entenderse dentro de la idea de multimodalidad, en el sentido de que,
tanto en los portafolios como en las experiencias de mediación, se pudo
observar rediseños que obedecían a un modo particular de entender la
experiencia de creación de los portafolios y la aproximación de textos
literarios en el aula. Esto último se asemeja en parte a lo señalado por
Fernandes (2018) en lo que refiere a la reconstrucción más amplia de los
textos, que, en estos casos, fue a través de los productos culturales
desarrollados y el uso que se dio a las plataformas digitales.

Sobre esto último, llamó la atención el uso que se dio a la plataforma
YouTube en varios de los portafolios observados, pues dicha herramienta
fue utilizada para incrustar vídeos realizados a través de la plataforma
Zoom, con un claro propósito expositivo. Esto último podría ser
consecuencia de la digitalización de la enseñanza en la pandemia, ya que
dicha plataforma se incorporó masivamente dentro de la docencia en la
universidad y, por tanto, formó parte de las prácticas digitales de ese
periodo.

El uso de la plataforma YouTube de esta manera, invita a reflexionar
sobre el conocimiento ético de su uso, así como del tema de la
privacidad en el ciberespacio. Se señala lo anterior, porque en la
mayoría de los vídeos compartidos a través de esta plataforma, están en
modalidad pública, siendo que existe la opción de ocultarlos o hacerlos
privados.

En la valoración de los portafolios se pudo observar que las
herramientas y plataformas incorporadas por los futuros docentes en sus
portafolios, tienen una finalidad predominantemente formativa. Es el
caso de Genially, Padlet, Slide y Docs de Google, que son herramientas
ampliamente utilizadas en el ámbito educativo y, por tanto, su uso
informa de las literacidades digitales de futuros docentes dentro del
ámbito académico. No obstante, lo anterior, dentro de las experiencias
de mediación se apreció el uso de YouTube y Padlet, pero no de Genially,
debido, probablemente, al nivel de conexión a internet que se necesita
para su funcionamiento y que, en escuelas vulnerables, es un recurso que
escasea o es débil.

Esto último pudo haber repercutido en el desarrollo de las experiencias,
ya que muchas de ellas utilizaron recursos analógicos más que digitales,
que afectó la agencia del estudiantado. Probablemente, la escasez de
dispositivos interactivos dificulta la diversificación de estrategias,
coincidiendo con el análisis de Ramada Prieto y Turrión Penelas (2019) y
lo señalado por Ryan (2004). Pese a lo anterior, la incorporación de la
cultura de internet en las experiencias pudo haber impactado el
desarrollo de las estrategias de mediación.

Se señala lo anterior, ya que se pudo apreciar que la incorporación de
los procesos de conocimiento, especialmente los procesos de
experimentación desde lo nuevo o lo conocido y el análisis crítico, dio
un énfasis menos tradicional al proceso de mediación literaria. Esto se
puede visualizar a partir de las actividades desarrolladas por los
profesores en formación, ya que, en su mayoría, estas se llevaban a cabo
a partir de artefactos y manifestaciones culturales-digitales conocidas
tanto por ellos como por sus estudiantes, y no necesariamente desde el
texto literario en sí. Esto último concuerda con la idea de sinestesia
propuesta por Kalantzis, Cope y Zapata (2019), ya que para interpretar
un texto literario, los profesores presentaban a sus estudiantes una
aproximación a los textos desde distintas modalidades que les
posibilitara realizar una interpretación estética o crítica.

En las estrategias de mediación se puede observar que se discutían y
analizaban las manifestaciones culturales conocidas a partir de
temáticas vinculas a los OA trabajados. A partir de dicha discusión, se
presentaba el texto literario, el cual, la mayoría de las veces se
analizaba a partir de estrategias de lectura convencionales. Sin
embargo, al llevar a cabo los procesos de conocimiento de las
multiliteracidades, los docentes en formación generaron instancias en
las que sus estudiantes pudieron aplicar de forma apropiada o creativa.
En este sentido, la creación se enfoca desde los productos culturales y
la interpretación de los textos literarios considera elementos más
emocionales y, en algunos casos, más estructurales, que puede deberse al
contexto de retorno de la pandemia y de los OA seleccionados para el
proyecto. Se podría decir que las experiencias se aproximan, levemente,
a lo señalado por Meneses, Maturana y Báez (2024), lo que indica que es
necesario seguir trabajando desde este enfoque.

Se puede señalar, por tanto, que la mediación de la literatura a través
de la pedagogía de las multiliteracidades en los portafolios observados,
integra parte de las literacidades digitales de los profesores en
formación, ya que el modo de aproximar la lectura literaria en los
discentes está matizado por la experiencia digital, que les permite
tener una visión diferente sobre la mediación de lectura literaria en el
aula. Esto último, se condice con las experiencias llevadas a cabo por
Godoy (2023) y Cabré Rocafort (2021) respecto al enfoque de las
multiliteracidades en la formación inicial docente y la mediación
lectora.

Respecto a las escuelas que tienen un alto índice de vulnerabilidad, se
puede decir que el uso de ciertas herramientas y plataformas conocidas
por este grupo de profesores en formación, facilitó la lectura de textos
más complejos en los estudiantes y ayudó a que estos realizaran una
interpretación o creación a partir del desarrollo de su punto de vista
sobre dichos textos; por lo tanto, el enfoque de las multiliteracidades
podría ser una oportunidad para avanzar en la disminución de la
desigualdad que presentan estas escuelas.

Como conclusión, se puede señalar que es necesario crear instancias
formativas que permitan a los futuros docentes desplegar sus
literacidades digitales y que estas puedan ser concientizadas por ellos,
de modo que puedan integrarlas en los procesos de mediación de la
literatura. Al respecto, la experiencia relatada quiso ser un aporte a
la comprensión de nuevas formas de mediación de la literatura que
integran prácticas ligadas a una visión del uso digital, a través del
enfoque pedagógico de las multiliteracidades.

Las limitaciones de la experiencia son variadas, está la cantidad de
portafolios valorados, el contexto de retorno a la pandemia y la
necesidad de datos más empíricos que pudo haber afectado la valoración.
También dentro de las limitaciones podemos señalar el acceso a la
recepción de los estudiantes de aula sobre esta experiencia, que puede
pensarse como futuras investigaciones sobre este tema.

Pese a lo anterior, se piensa que la descripción de esta experiencia
puede ser un insumo interesante para dar a conocer un enfoque que se
está trabajando dentro de la formación inicial docente de una
universidad pública en Chile. También la valoración de estos
portafolios, pretende ser una contribución a estudios sobre la pedagogía
de las multiliteracidades que abordan experiencias de mediación de la
literatura.

La experiencia descrita, además, es una invitación a profundizar en las
prácticas de literacidad digital de futuros docentes de lengua y
literatura para reflexionar sobre el impacto en su ejercicio docente, su
concepción de la literatura y la mediación en las aulas, que implicaría
un estudio más profundo que permita dilucidar conclusiones más robustas
sobre estos temas, que adquieren relevancia dentro del contexto
desafiante e incierto que deben enfrentar las escuelas en la actualidad.

 \section{Financiación}
Este estudio fue apoyado por la Vicerrectoría de Investigación y Desarrollo (VID) de la Universidad de Chile, a través del concurso “Ayuda de Viajes para Potenciar la Productividad Académica VID, 2024” (AYV077/01-23).