\section{Introducción} \label{sec-introducción}

Los desafíos sobre la mediación de la literatura en el aula han sido
abordados desde distintas perspectivas en la formación del profesorado.
Una de ellas refiere a los perfiles lectores de docentes en formación
\cite{caride2018leer,colomer2013,granadopuig2014}, cuyos estudios han señalado las dificultades que poseen los
futuros docentes al momento de enfrentar un texto literario. Otra
perspectiva refiere a su nivel de competencia literaria, en que se ha
señalado su dificultad en el acceso y nivel de intertexto lector
\cite{contreras20515}. Una de las alternativas para enfrentar dichas
problemáticas, ha sido la exploración de la lectura en formatos
digitales y multimodales \cite{arbonés2015localización} que, en la
actualidad, cobra una mayor relevancia después de lo que significó la
digitalización de la enseñanza en el confinamiento por la pandemia de la
COVID-19 \cite{ciae_covid19._2020,miras_implications_2023,sanchez-cruzado_teacher_2021}.

Las consecuencias más evidentes de las medidas adoptadas durante la
pandemia han sido los cambios en las dinámicas de aula, ya que los
estudiantes debieron volver a las rutinas que tenía la escuela antes del
COVID-19; y el aumento de la brecha digital \cite{rivera_polo_brecha_2023}, que ha
evidenciado la desigualdad que sufren los sectores más desfavorecidos y
que ha impactado en los resultados a nivel nacional. Según un estudio de
la Organización para la Cooperación y el Desarrollo Económico (OECD en
inglés), el 32 \% de los estudiantes en Chile no poseen las competencias
mínimas en lectura (2022). A nivel internacional, los resultados de las
pruebas del Programa para la Evaluación Internacional de Alumnos (PISA,
por sus siglas en inglés) de la OCDE, han señalado un descenso en la
lectura y la inequidad de género en el acceso al conocimiento
\cite{schleicher_2023}.

Frente a este panorama, en Chile se han propuesto medidas a través de un
plan de reactivación que fue diseñado a partir de un consejo
participativo. Dicho consejo recomendó el disfrute de la lectura en las
primeras etapas y su promoción, a través de manifestaciones transmedia y
digital con el fin de reactivar la lectura en el aula \cite{consejopara}.

Esto último, da a entender que la visión sobre cómo debe ser enseñada y
promovida la lectura dentro del aula está cambiando y que, por lo tanto,
es necesario desarrollar diversas formas para aproximar y dinamizar la
lectura literaria. En la formación inicial docente, implica atender a
perspectivas que ya han sido abordadas en diferentes estudios y
experiencias y que están vinculadas a la multimodalidad \cite{reyesTorres_batallerCatalà_2019,berríos2022,farias2019,ow_forster_2012} a la lectura digital \cite{pallares2014,calvovalios2015,ballester2016} o a su representación
en otros formatos para aproximarla a los estudiantes desde una
perspectiva más inclusiva.

Dentro de la formación inicial docente es relevante guiar y diversificar
la visión de la literatura y su mediación en los futuros profesores, ya
que estos deberán enfrentar la complejidad del siglo XXI, que incluye
las desigualdades de origen económico, de género, de condición económica
y de educación de los estudiantes \cite{oecd_2019}. Se espera que los
futuros profesores puedan desarrollar habilidades que permitan a sus
alumnos enfrentar el incierto y cambiante mundo de este siglo.

En este sentido, la lectura debe ser aproximada a los estudiantes de
manera que pueda responder a las exigencias actuales. En el aula,
significa diversificar las prácticas predominantes, para introducir
otras más cercanas a los aprendices y vinculadas con literacidades con
las que se desenvuelven cotidianamente, como son las digitales y
multimodales.

Este artículo relata una experiencia en la formación inicial docente,
relacionada con la mediación de la literatura, a partir de la valoración
de seis portafolios reflexivos digitales que fueron diseñados por
profesores en formación de lengua y literatura en su última etapa
formativa. La propuesta de creación de portafolios se desarrolló en la
asignatura de Proyectos Didácticos y Evaluativos Innovadores de la
carrera de Pedagogía en Educación Media con mención en Lenguaje de la
Universidad de Chile, durante el primer semestre de los años académicos
2022 y 2023.

La experiencia llevada a cabo intentó responder a los desafíos de la
lectura en la actualidad. Para ello, se invitó a un grupo de futuros
docentes a desplegar sus literacidades digitales en una plataforma
digital para narrar su propia experiencia en la elaboración de proyectos
que integran la lectura literaria, a través del enfoque de la pedagogía
de las multiliteracidades.

La experiencia, en este sentido, abordó varios objetivos. El principal
de ellos fue el desarrollo de las literacidades digitales de futuros
docentes en la creación de sus portafolios y la posibilidad de exponer
sus experiencias de mediación de la lectura a través de dicho formato. A
través de la valoración, se pudo observar cómo fue la interacción de las
literacidades digitales en la experiencia de mediación y la integración
del enfoque de la pedagogía de las multiliteracidades, como una
posibilidad de atender a la complejidad que presentan escuelas con alto
índice de vulnerabilidad.