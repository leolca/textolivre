\section{Prácticas de literacidad digital en la mediación de la
	literatura}\label{prácticasdeliteracidad}
	
	En la formación inicial docente, la mediación de la literatura a través
	de tecnologías digitales ha sido abordada en el último tiempo como una
	preocupación ante la forma de concebir la lectura por parte de las
	nuevas generaciones \cite{contrereasbarcelo2023,gonzalez_2016,rovira-collado2021}. Los estudios han señalado,
	especialmente, la relación de futuros docentes con internet, las redes
	sociales y sus posibilidades para la enseñanza de la literatura. Dichos
	estudios proponen un enfoque desde la multimodalidad, las
	multiliteracidades y la lectura en formatos digitales, como una forma de
	acoger a los nuevos lectores que definen su relación con la lectura a
	través de los dispositivos digitales, las redes sociales e internet.
	
	La discusión sobre el nuevo lector \cite{cerrillo2005,lluch_2010_new,montesa_et_al_2010}, ha advertido de la interacción de los
	niños y adolescentes con los formatos digitales y de la posibilidad de
	acceder a otra experiencia de lectura, cuyos fines son más prácticos
	\cite{oecd_2019b}. La pandemia pudo haber intensificado este aspecto, ya que
	la lectura que realizan los estudiantes se da, mayormente, en formatos
	digitales. Estudios señalan que los jóvenes realizan una lectura que se
	centra en la obtención de información, comunicación y socialización
	\cite{Valdivia_Brossi_Cabalin_Pinto_2019}, pero que no necesariamente desarrolla el
	pensamiento crítico y la capacidad de agencia \cite{cabero-almenara_valencia-ortiz_llorente-cejudo_palacios-rodríguez_2023}.

En este sentido, se ha señalado la necesidad de desarrollar un enfoque diferente para la lectura literaria a través de soportes digitales \cite{cordongarcias2015,garciaroca2016practicas}, ya que el estudiante sería un lector implícito \cite{turreonpenelas2014}. No obstante, uno de los principales problemas para llevar a cabo dicha lectura dentro del aula es la escasez de dispositivos interactivos digitales en las escuelas \cite{ramadaprietoturrion2019}, que dificultaría desarrollar estrategias de mediación diferentes.
	
Como una forma de atender a los problemas en torno a la mediación de la
literatura y la formación de lectores, en el último tiempo se ha
indagado en las trayectorias lectoras de futuros profesores, en que se
ha señalado que los docentes son las figuras más relevantes en la
mediación de la literatura en el aula \cite{contrerasbarcelo2021}. Dentro
de esta idea, también se ha indagado en futuros docentes como lectores
en medios digitales y creadores dentro de escenarios transmedia, donde
han revelado la necesidad de ser formados en interpretación, reflexión y
evaluación de contenidos transmedia en un contexto formal \cite{contrereasbarcelo2023}, que da cuenta del cambio de paradigma en
torno a la lectura literaria y la formación de lectores.
	
	Considerando los aspectos señalados anteriormente, cabe preguntarse por
	los cambios que requiere la enseñanza de la literatura y de cuál es el
	lugar de las literacidades digitales de profesores en formación dentro
	de la interacción didáctica. Uno de los aspectos que han sido estudiados
	en torno a dicha inquietud, tiene relación con el desarrollo de la
	competencia digital pedagógica en profesores en formación y sus
	limitaciones, que podrían dificultar el desarrollo de estrategias que
	puedan mediar textos literarios a través del uso de tecnología digital.
	Diversos autores han señalado la necesidad de formar pedagógicamente en
	el uso de TIC a los futuros docentes \cite{ayala2015,sandoval-rubilar_alveal_fuentes_2017,Silva_Morales_Lázaro-Cantabrana_Gisbert_Miranda_Rivoir_Onetto_2019},
	ya que, en general, existe un desconocimiento del uso pedagógico de las
	herramientas digitales.
	
	En este sentido, es necesario que las tecnologías digitales sean
	integradas de manera pedagógica en el desarrollo de estrategias y en la
	mediación de textos literarios, considerando lo que se ha señalado sobre
	los nuevos lectores y las prácticas digitales de los adolescentes, que
	inciden en el tipo de lectura que realizan. No obstante, dentro de las
	dificultades para la enseñanza de la literatura y su mediación, están
	las concepciones que tiene el profesorado sobre la literatura \cite{barra2022} y las estrategias que se requieren para enseñarla
	en soportes digitales, que son diferentes a la lectura en soporte
	analógico \cite{ryan_2004}.
	
	Una manera de abordar estos asuntos puede ser a través de la integración
	de las literacidades digitales de docentes en formación en las
	estrategias para mediar la literatura. Cabe precisar al respecto, que la
	definición de las literacidades digitales posee límites imprecisos
	\cite{manghi_haquin_2016,villar_onrubia_morini_marin_nascimbeni_2022},
	no obstante, para la valoración de los portafolios, se ha considerado
	desde una perspectiva sociocultural y con prácticas que van más allá de
	la lectura y escritura \cite{villar_onrubia_morini_marin_nascimbeni_2022}. Dichas
	prácticas se desarrollan dentro de un marco que participa tanto de lo
	cultural como de lo histórico e institucional \cite{gee2015} y que
	interactúan con formas post-tipográficas de textos y su producción
	\cite{lankshear_knobel_2008}.
	
	Las literacidades digitales, dentro de la experiencia desarrollada, se
	han acotado a lo que proponen \textcite[p. 9]{gillen_barton_2010}, que las definen como
	``prácticas en constante cambio, a través de las cuales las personas
	crean significados rastreables utilizando tecnologías digitales'' y, por
	tanto, ``implican una atención cuidadosa y sensible a lo que las
	personas hacen con los textos ``.
	
	Dentro del desarrollo de los portafolios se dio énfasis a la interacción
	de los futuros docentes con las tecnologías digitales y que, en dicha
	interacción, se reflejaran sus recursos e ideología, para trascender,
	así, el uso técnico de las TIC \cite{dowdall2021digital}.
	
	En la formación inicial y continua del profesorado, se ha dado
	importancia a la introducción de las prácticas de literacidad en la
	enseñanza, que están fuertemente ligadas al uso de tecnologías digitales
	\cite{thibaut_2020} y también al desarrollo de las multiliteracidades para
	la mediación de la literatura \cite{godoy2023,cabrerocafort2021}, que
	se ha visto como una posibilidad tangible y necesaria. En Chile es
	relevante su introducción en el aula, debido a la presencia que tiene la
	multimodalidad dentro del currículo nacional \cite{meneses_literacidad_2023}.
	
	En relación con lo anterior, hay experiencias que han abordado la
	mediación de la literatura a través de la noción de diseño de las
	multiliteracidades, la cual se entiende dentro de la gramática de la
	multimodalidad. La idea de diseño disponible (recursos), de diseñar
	(producir significado) y de rediseño (diseños disponibles nuevos)
	\cite{kalantzis_cope_new_learning}, han posibilitado a los estudiantes profundizar
	la lectura y reconstruir de forma más amplia el sentido de los textos
	\cite{Fernandes_2018}. Esto último va ligado a la idea de sinestesia, que
	permite expresar significado ``en una modalidad y luego en otra''
	\cite[p. 244]{kalantzis2019} que apoyaría la diversificación
	de estrategias para interpretar un texto literario que esté en un
	soporte digital y multimodal.
	
	


		
