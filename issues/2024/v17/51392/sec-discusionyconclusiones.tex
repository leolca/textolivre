\section{Discusión y conclusiones}\label{sec-discusiónyconclusiones}

Partiendo de los resultados obtenidos, consideramos que se refuerza un
planteamiento, ya apuntado por otros autores \cite{alfian_role_2021,sidorenko-bautista_use_2020}, que
sostiene que las narrativas deben cambiar para adaptarse a los nuevos
medios, y que los medios están cambiando continuamente para adaptarse a
las narrativas. En este sentido, se observa un gran interés de la
comunidad científica por adaptar las técnicas clásicas de la narración a
los nuevos medios, soportes, tecnologías y entornos, para logar conectar
con sus audiencias. La necesidad de vehicular contenidos a través de
estructuras narrativas creíbles resulta esencial ya que, de acuerdo con
\cite{vergara2020herramientas}, podemos decir que si algo nos
caracteriza desde el punto de vista cultural es que somos depredadores
de buenas historias.

Como podemos observar, el interés por el \emph{Storytelling} ha superado
la barrera de las Ciencias Sociales, posicionándose como método de
transmisión de conocimiento en diversas disciplinas. El
\emph{Storytelling} es una herramienta poderosa que ha sido utilizada
durante siglos en una variedad de ámbitos, desde la literatura hasta la
publicidad, desde la política hasta la psicología. Su capacidad para
inspirar, educar y entretener va mucho más allá de las ciencias sociales
\cite{haven_story_2007}.

Los hallazgos que hemos realizado nos permiten concluir que la narrativa
sigue siendo fundamental para la articulación de contenidos y el proceso
de enseñanza y aprendizaje. La comunidad científica muestra un gran
interés por adaptar la narrativas y técnicas del \emph{Storytelling} a
los nuevos escenarios educativos, metodologías y herramientas
tecnológicas. La continua exploración de las posibilidades de estas
técnicas está despertando el interés de diferentes disciplinas que
persiguen beneficiarse del carácter pedagógico de las historias.

Podemos establecer otra línea de conclusiones en relación con las
temáticas que se abordan en las historias. Hemos detectado un gran
interés de la comunidad científica por construir narrativas eficaces que
puedan ayudar a comprender las claves del cambio climático. En este
sentido, resulta interesante la vía abierta por diversos trabajos \cite{arnoud_climate_2018,bloomfield_climate_2021,de_meyer_transforming_2021,moezzi_using_2017} que recomiendan trasladar, utilizando
técnicas de comunicación efectiva, estas narrativas climáticas a todas
las esferas de la sociedad, y no solo al ámbito educativo.

Desde la perspectiva didáctica, se hace evidente el interés por crear
propuestas metodológicas específicas que permitan aprovechar todo el
potencial del \emph{Storytelling} en el aula. Los métodos activos son,
en todos los casos, las opciones preferidas. Entendemos que se trata de
una relación lógica, ya que el Storytelling guarda una relación de
coherencia con los rasgos propios del aprendizaje activo (implicación
emocional, cooperación, creatividad, búsqueda del conocimiento, etc.).
Hemos encontrado diferentes estrategias metodológicas para utilizar las
historias como recurso didáctico. Muchas se integran en el método
didáctico Aprendizaje Basado en Historias.

Otros de los fenómenos a los que la comunidad científica debe prestar
atención en el futuro es al de los \emph{Entornos Inteligentes de
Aprendizaje} y a \emph{la Narración transmedia}. De todas estas nuevas
vías, surgen fortalezas como el potencial pedagógico de las narrativas
aplicadas al pensamiento reflexivo o a los nuevos medios digitales \cite{kim_digital_2021,yasar__2022} y otros debates éticos, como la
moralidad de utilizar técnicas de manipulación, como podría considerarse
el \emph{Storytelling} dentro del contexto de las redes sociales tras
demostrarse los efectos negativos que estas producen en la salud mental
\cite{sheldon_dark_2019}.

En cuanto al uso de la tecnología que todas estas metodologías
combinadas con narrativas propician, se imponen con claridad
instrumentos como los dispositivos móviles y la realidad aumentada. En
este sentido, numerosos autores \cite{aurelia_survey_nodate,dunleavy_augmented_2014,nam_designing_2015,nobrega_mobile_2017} apuntan al potencial de la combinación de ambas tecnologías como un buen modelo de enseñanza debido a que cualquier lugar u objeto del mundo real puede integrarse en una historia, estimulando la imaginación del participante. Aunque también señalan que existen algunas barreras que impiden a los usuarios participar activamente en este tipo de actividades, como la
necesidad de disminuir la brecha digital y el elevado coste de los
dispositivos y la tecnología.

Al contrastar nuestra revisión con otras similares, podemos comprobar
que las conclusiones confluyen en varios aspectos. El primero de ellos
sería en la positiva percepción sobre la aplicación de la narración
digital de historias por parte de los estudiantes. Esta idea ha sido
destacada en otras revisiones que también señalan el efecto novedad como
posible artífice de esta aceptación. Se propone continuar investigando
para ver cómo se desarrolla esta relación en el futuro. El segundo
aspecto a tener en cuenta sería la necesidad de formar a los docentes en
el uso de los entornos digitales y la creación, ya que la narración de
historias continúa capturando la atención de audiencias, convirtiéndose
en un aliado indispensable de cualquier proceso didáctico y formativo,
por lo que necesitamos formar a nuestros docentes \cite{chikasanda_enhancing_2013,lawless_professional_2007}.

En relación con futuras líneas de investigación, cabe apuntar que el
importante índice de experimentación didáctica que hemos encontrado,
unido a la necesidad de una constante adaptación de las narrativas a los
nuevos entornos inteligentes y digitalizados, derivarán en nuevas formas
de enseñar y aprender en el futuro, a las que la comunidad científica
deberá prestar atención para evaluar su potencial y medir su eficacia.