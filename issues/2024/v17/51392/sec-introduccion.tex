\section{Introducción}\label{sec-introducción}

Desde la antigüedad, la narración de historias ha servido a los seres
humanos para transmitir información, jugando un papel fundamental en el
desarrollo de las sociedades y en la conservación del patrimonio
inmaterial y cultural de las mismas. Hoy en día, con el paso de los
siglos, los modelos clásicos de construcción de historias siguen estando
vigentes en diversos ámbitos como el cinematográfico, el televisivo o la
creación de videojuegos. La publicidad o la comunicación política
también recurren a las técnicas del \emph{Storytelling.}

En el ámbito de la educación, la narración tiene una gran presencia como
estrategia didáctica \cite{vergara_ramirez_narrar_2018}. La aplicación de técnicas de
\emph{Storytelling} mediadas por tecnología educativa son muy utilizadas
en el aula, dando lugar a métodos didácticos específicos como el
Aprendizaje Basado en Historias \cite{sanchez_rivas_experiencia_2023}.

El Aprendizaje Basado en Historias es un método activo que promueve la
investigación a partir de una historia (cuento, novela o película),
utilizando para el proceso diferentes recursos tecnológicos, como
\emph{Google Workspace} o \emph{Genially}. La secuencia didáctica parte
de la generación de un vínculo emocional hacia la historia y culmina con
un producto de aplicación del aprendizaje a partir de la creación de
historias propias \cite{sanchez-rivas_narrative-based_2022}.

En el plano didáctico, las estrategias metodológicas basadas en la
narración de historias y mediadas por tecnología presentan importantes
beneficios para el proceso de enseñanza y aprendizaje. Entre ellos, se
encuentran los siguientes:

\begin{itemize}
	\item
	Favorece la comprensión, ya que las historias pueden ser utilizadas
	para acercarnos a conceptos y temas complejos de una forma más fácil
	de entender mediante el empleo de metáforas, ejemplos y analogías
	\cite{isbell_effects_2004,lenhart_more_2020}.
	\item
	Fomenta la empatía, permitiendo a los receptores identificarse con las
	situaciones que se describen y poniéndose en el lugar de sus
	personajes \cite{bratitsis_digital_2016,skaraas_playing_2018,vaughan-lee_power_2019}.
	\item
	Estimula el análisis, la reflexión y el pensamiento crítico, a través
	de las diferentes perspectivas desde las que se plantean las
	situaciones narradas \cite{larry_crumbley_using_2000,young_introducing_1996}.
	\item
	Impulsa la creatividad, fomentando el uso de la imaginación, la
	recreación de espacios, argumentos y soluciones alternativas \cite{celume_fostering_2019,tabieh_effect_2021}.
\end{itemize}

En el contexto educativo actual, marcado por las tecnologías digitales,
los discursos narrativos en al aula se han ido redefiniendo hacia los
llamados \enquote{transmedia}. \textcite{dudacek_transmedia_2015} describe el transmedia como un proceso en el que elementos integrales de una ficción se comunican a través de diferentes canales de distribución para estructurar y contar una historia. A esto hay que sumarle el cambio metodológico que implican los métodos activos, como el Aprendizaje Basado en Proyectos, el
Aprendizaje Basado en Resolución de Problemas o el Aprendizaje
Cooperativo.

De esta nueva configuración de los escenarios didácticos basados en la
narración y la tecnología, se deriva la necesidad de generar una base de
conocimiento científico a partir de la cual fundamentar intervenciones
docentes que combinen el aprendizaje basado en historias y la
tecnología, y ampliar el conocimiento generado en torno a estos tópicos.

El objetivo de este artículo consiste en identificar y analizar
publicaciones científicas que aborden investigaciones y experiencias
didácticas basadas en la aplicación de una enseñanza a partir de
historias y mediada por recursos tecnológicos, de manera que las
conclusiones extraídas puedan servir de base a futuros estudios e
investigadores que pretendan aportar conocimiento a una línea de
investigación tan sustancial como la descrita, y que contribuyan a
mejorar las experiencias educativas.

Existe una amplia producción bibliográfica sobre el \emph{Storytelling}
y tecnología, aplicada principalmente a los campos de los individuos y
sociedades, y a la creación, teoría y crítica literaria. En lo que se
refiere a su vertiente educativa, dicha producción se centra en aspectos
como su utilidad para enseñar y asimilar conceptos \cite{alismail2015integrate,sadik2008digital}, fomentar la comprensión lectora \cite{al-shaye2021digital,bakar2019digital} y modelar y analizar estudios de casos reales, especialmente en
los ámbitos científicos y de la salud \cite{moreau_digital_2018,open2022digital,thompson_communicating_2018}.

Aunque la narración de historias siempre ha estado vinculada a la
pedagogía, la evolución de la tecnología y la aparición de nuevas
metodologías activas han influido en la redefinición y adaptación de las
técnicas de narración en los procesos de enseñanza y aprendizaje. Desde
principios del siglo XXI se ha observado un gran interés por la
mediación tecnológica en la narración de historias en el ámbito
educativo, especialmente por la mayor capacidad de las técnicas visuales
y auditivas para captar la atención de los estudiantes frente a la
comunicación escrita \cite{suwardy2013using}. Entre sus principales
beneficios, numerosos estudios han destacado la flexibilidad de la
narración digital para integrar mensajes instructivos, crear entornos de
aprendizaje más atractivos, fomentar la reflexión, el trabajo en equipo
y el debate \cite{beck2021digital,smeda2014effectiveness,tiba2015digital}. Fruto de estos hallazgos han surgido diversas iniciativas
académicas como la creación del \enquote{Centro de Narrativa Digital} por
parte de la Universidad de Berkeley, California (Estados Unidos), que
establece claves y elementos de éxito para construir narrativas
digitales eficaces \cite{robin2008effective}, o la plataforma creada por la Facultad
de Educación de la Universidad de Houston, Texas (Estados Unidos)
destinada a recopilar y ofrecer recursos a educadores interesados en
integrar las narrativas digitales en sus actividades educativas
\cite{rudnicki2006buzz}.

Aprovechar la producción científica resulta esencial para identificar
problemas y establecer posibles soluciones que nos permitan mejorar y
abordar nuevos retos en el futuro. Por ello, se considera pertinente
ofrecer a la comunidad científica este trabajo de revisión, un análisis
que recoge las principales ideas que han aportado otros científicos e
investigadores.

En la actualidad, la comunidad científica dispone de numerosas bases de
datos que permiten revisar, analizar y contrastar la producción y el
trabajo realizado en diferentes campos de investigación. Sin embargo, la
gran cantidad de archivos bibliográficos disponibles sobre cualquier
objeto de estudio requiere un cuidadoso filtrado, dirigido a optimizar
el provecho de la investigación en curso. Esta necesidad ya ha sido
señalada por diversos autores \cite{baas_scopus_2020,schotten_brief_2017,vila_gamificacion_2019} que también destacan las ventajas de SCOPUS frente a
otras bases de datos. Entre ellas, debemos señalar la mayor cobertura
global en investigación, la relevancia e impacto de las publicaciones,
así como herramientas que facilitan el seguimiento, el análisis y la
visualización de las investigaciones, como factores clave para la
utilización de SCOPUS en este trabajo \cite{alryalat2019comparing,chadegani2013comparison,singh2024large}.
