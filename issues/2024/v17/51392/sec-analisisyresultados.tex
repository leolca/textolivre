\section{Análisis y resultados}\label{sec-análisisyresultados}

Los resultados de este estudio se representan organizados según los
criterios de análisis establecidos. En primer lugar, se muestran los
números de artículos publicados por periodos temporales, es decir, por
años naturales. En ellos se evidencia un aumento progresivo anual de la
producción, y 2021 como el año con el incremento más pronunciado \Cref{tab-01}.

\begin{table}[htbp]
\centering
\begin{threeparttable}
\caption{Número de artículos por año.}
\label{tab-01}
\begin{tabular}{lll}
\toprule
Año & Artículos & Diferencia\\
\midrule
2019 & 26 & - \\
2020 & 28 & 2 \\
2021 & 40 & 12\\
2022 & 51 & 11\\
\bottomrule
\end{tabular}
\source{Elaboración propia.}
\end{threeparttable}
\end{table}

A partir de 2021 se observa una mayor preocupación por explorar nuevas
fórmulas narrativas. Encontramos una razón para ello en la crisis del
COVID-19. De acuerdo con diversos autores \cite{monroy_retos_2021,torras_virgili_emergency_2021,una_martin_aproximacion_2020,vilhelm_rios_didactica_2022}, consideramos que la
pandemia y la crisis sanitaria reformularon muchos planteamientos
pedagógicos tradicionales, generando la necesidad de experimentar con
modelos más flexibles, personalizados, con tecnologías y metodologías
más innovadoras.

El número de citas de los artículos es muy variable, y no puede
establecerse un patrón de comportamiento, ni correlación entre las
fechas de publicación y el impacto de los artículos. Aunque generalmente
los artículos más antiguos suelen tener mayor número de citas, en el
caso que nos ocupa los dos artículos más destacados fueron publicados en
2021. Entre toda la producción seleccionada, podemos destacar un trabajo
que alcanza más de sesenta citas, y otros siete con cifras superiores o
iguales a veinte \Cref{tab-02}

\begin{table}[htbp]
\centering
\begin{threeparttable}
\caption{Número de citas por año.}
\label{tab-02}
\begin{tabular}{lllll}
\toprule
\multicolumn{1}{c}{Artículo} & \multicolumn{4}{c}{Citas por año} \\
  &2019& 2020& 2021& 2022\\
\midrule
Secundo, Mele, Vecchio, Elia, Margherita, \& Ndou (2021) &-& - &16& 51\\
Agbo, Oyelere, Suhonen, \& Tukianen (2021)& -& - &12 &25\\
Kerr, \& Lawson (2020)& - &6& 16 &11\\
Flórez-Aristizábal \& Fardoun (2019)& 3& 4 &7& 12\\
Naul, \& Liu (2020) &- &1& 9& 15\\
Sarnok, Wannapiroon, \& Nilsook (2019) &1 &4& 8& 10\\
Kaeophanuek, Na-Songkhla, \& Nilsook (2019) & 2 & 4& 8& 7\\
\bottomrule
\end{tabular}
\source{Elaboración propia.}
\end{threeparttable}
\end{table}

Atendiendo al número de citas, debemos destacar tres artículos que están
liderando la tendencia en investigación sobre el \emph{storytelling}
mediado por el uso de tecnología y sus posibilidades pedagógicas. Se
trata de los trabajos de \cite{secundo_threat_2021,agbo_scientific_2021,kerr_augmented_2020}.

El primero de ellos surge en el contexto de crisis del COVID-19, y en él
se abordan los principales retos tecnológicos que la pandemia generó en
el ámbito educativo. Los autores ilustran el rediseño de un programa de
aprendizaje para emprendedores, aprovechando las tecnologías digitales y
a nivel práctico, aportan ideas para remodelar los programas
universitarios tradicionales, preparándolos para abordar emergencias
futuras.

El segundo artículo ofrece una amplia revisión bibliográfica en la que
se examina el panorama de la investigación sobre los Entornos
Inteligentes de Aprendizaje, mediante un análisis bibliométrico. El
resultado de este análisis indica que el primer artículo sobre esta
nueva realidad se publicó en 2002, lo que supuso el inicio en la
exploración de este campo. Los Entornos Inteligentes de Aprendizaje
aluden a entornos físicos enriquecidos con dispositivos digitales,
sensibles al contexto y adaptativos, para promover un aprendizaje más
rápido y eficaz \cite{koper_conditions_2014}. Su análisis temático muestra que la narración digital y sus componentes asociados, como la realidad virtual, el pensamiento crítico y los Juegos Serios, que son juegos creados para proporcionar un contexto de entretenimiento y auto-fortalecimiento con
el que motivar, educar y entrenar a los jugadores  \cite{chipia_lobo_juegos_2011}. Consideramos que se trata de dos líneas de investigación emergentes.

Por último, en el artículo de \cite{kerr_augmented_2020}, se describe el
desarrollo de un prototipo de Realidad Aumentada, creado para formar a
estudiantes en los principios básicos de la arquitectura paisajística.
El enfoque principal de los autores se centra en proponer nuevas
prácticas de narración digital a través de experiencias situadas. De
este trabajo inferimos la necesidad de mejorar los programas de
formación docente para capacitar a los futuros maestros en la
integración de las tecnologías inmersivas, como también indica \cite{figueroa_fusionando_2021}.

El tercer indicador hace referencia a las autorías, y nos permite
destacar a los investigadores que más han aportado en el campo objeto de
estudio. En la siguiente tabla mostramos sus nombres y filiación
institucional, a través de la cual podemos observar una gran
representación de Australia y Malasia país que, junto a Tailandia,
lidera sus investigaciones desde universidades centradas en el
desarrollo tecnológico \Cref{tab-03}.

Este dato se corrobora al confrontarlo con las instituciones destacadas
en la investigación sobre \emph{Storytelling} \Cref{tab-04}.

\begin{table}[htbp]
\centering
\begin{threeparttable}
\caption{Investigadores destacados.}\label{tab-03}
\begin{tabular}{lll}
\toprule
Investigador & Filiación & Artículos\\
\midrule
Nilsook, Prachyanun &King Mongkut´s University of Technology & 3\\
Buendgens-Kosten, Judith & Goethe-Universität Frankfurt am Main & 2\\
Chubko, Nadezhda & Edith Cowan University & 2\\
Cornillie, Frederik & KU Leuven & 2 \\
Girmen, Pınar &  Eskişehir Osmangazi Üniversitesi & 2\\
Kantathanawat, Thiyaporn K. & Mongkut's Institute of Technology Ladkrabang & 2\\
Kantathanawat, Thiyaporn K. & Mongkut\textquotesingle s Institute of Technology Ladkrabang & 2\\
Lummis, Geoff W. & Edith Cowan University& 2\\
Macleroy, Vicky & Goldsmiths, University of London &2\\
McKinnon, David H. & Edith Cowan University & 2\\
Morris, Julia Elizabeth & Edith Cowan University & 2\\
\bottomrule
\end{tabular}
\source{Elaboración propia.}
\end{threeparttable}
\end{table}


Al analizar las filiaciones (\Cref{tab-04}) de los principales autores y producción por
países, encontramos en Europa y América del Norte el mayor volumen de
aportaciones. El continente americano, lidera las investigaciones que
relacionan el \emph{Storytelling} con el ámbito de la medicina y la
salud, aunque también hay una gran representación europea debido a las
aportaciones de Reino Unido. En lo referente al campo de las ciencias
sociales, Europa es quien lidera y marca las tendencias. España, que
ocupa el cuarto puesto de la tabla en número de aportaciones por países
con un total de catorce artículos, destaca por su contribución a los
campos relacionados con la aplicación de la narrativa en entornos
digitales, nuevos soportes, y en el uso de metodologías activas. En este
sentido, podríamos señalar a este país como uno de los líderes en la
investigación sobre innovación y tecnología educativa. Este interés por
el \emph{Storytelling} aplicado a metodologías activas y recursos
digitales, continúa aumentando y ha sido identificado en otros estudios
recientes vinculados a universidades españolas \cite{sanchez-rivas_narrative-based_2022,sanchez_rivas_experiencia_2023}.

\begin{table}[htbp]
\centering
\begin{threeparttable}
\caption{Instituciones destacadas.}
\label{tab-04}
\begin{tabular}{lll}
\toprule
Institución & País & Artículos \\
\midrule
Universiti Kebangsaan & Malasia & 3\\
Goldsmiths, University of London & Reino Unido & 3\\
Universiti Teknologi Malaysia & Malasia & 3\\
King Mongkut\textquotesingle s Institute of Technology Ladkrabang &
Tailandia &3\\
Edge Hill University & Reino Unido & 3\\
National and Kapodistrian University of Athens & Grecia & 3\\
King Mongkut\textquotesingle s University of Technology North Bangkok&
Tailandia &3\\
Universitat Oberta de Catalunya & España & 2\\
Universiti Utara Malaysia & Malasia & 2\\
David Geffen School of Medicine at UCLA & Estados Unidos & 2\\
Universidad de Oviedo& España & 2\\
\bottomrule
\end{tabular}
\source{Elaboración propia.}
\end{threeparttable}
\end{table}

 En lo que respecta a la producción por países, Estados Unidos ocupa la primera posición con veintiuna aportaciones, seguido de Reino Unido y Australia. España ocupa la cuarta posición de la tabla con una aportación de catorce artículos, el mismo número que Australia \Cref{tab-05}.
 
\begin{table}[htbp]
\centering
\begin{threeparttable}
\caption{Producción por países.}
\label{tab-05}
\begin{tabular}{ll}
\toprule
País & Artículos \\
\midrule
Estados Unidos & 21\\
Reino Unido & 19\\
Australia & 14\\
España & 14\\
Grecia & 11\\
Malasia & 10\\
Canadá & 9 \\
Italia & 9\\ 
\bottomrule
\end{tabular}
\source{Elaboración propia.}
\end{threeparttable}
\end{table}

Si atendemos a las líneas de investigación abiertas en relación con el
\emph{Storytelling,} advertimos que su estudio se aborda desde
diferentes disciplinas. Aunque el enfoque de las investigaciones siempre
tiene una perspectiva pedagógica, podemos asegurar que el tratamiento
científico de las mismas se acomete desde una perspectiva
multidisciplinar \Cref{tab-06}.

\begin{table}[htbp]
\centering
\begin{threeparttable}
\caption{Producciones por disciplina científica.}
\label{tab-06}
\begin{tabular}{ll}
\toprule
País & Artículos \\
\midrule
Ciencias Sociales & 107\\
Ciencias Informáticas & 40\\
Artes y Humanidades & 22\\
Ingeniería & 16\\
Medicina & 15\\
Psicología & 15\\
Administración y negocios & 6\\
Ciencias Medioambientales & \\
Ciencia de Materiales & 5\\
Ciencias de la Salud &5\\
\bottomrule
\end{tabular}
\source{Elaboración propia.}
\end{threeparttable}
\end{table}

A la hora de abordar nuestro análisis cualitativo, y dar respuesta a
nuestra primera pregunta de investigación, se realizó un análisis de los
campos de estudio y de los contenidos desde los que se abordaban los
diferentes artículos publicados, es decir, aquellos ámbitos con mayor
interés en el uso de historias como recurso didáctico. Las Ciencias
Sociales, las Ciencias Informáticas y las Artes y Humanidades fueron las
áreas más productivas. Cabe destacar que, si a la medicina le sumamos
las publicaciones de otros campos afines como el de las ciencias de la
salud o el de la psicología, ocuparían el tercer puesto. Esto es debido
a la gran cantidad de estudio de casos que utilizan estas ciencias tanto
para enseñar, como para contrastar información en el desempeño de su
trabajo.

A esta revisión de naturaleza cuantitativa, se añadió otra de tipo
documental, para responder a nuestra segunda pregunta e identificar las
finalidades pedagógicas de las líneas temáticas más investigadas en cada
una de las principales áreas de estudio. Los trabajos analizados podrían
organizarse en tres grandes bloques (\Cref{tab-07}). En primer lugar, aquellos que ponen su foco en el proceso de aprendizaje y utilizan las técnicas vinculadas a las historias para mejorar el proceso didáctico. Dentro de esta vertiente, encontramos una gran tendencia a trasladar los modelos
clásicos de la narración a nuevos medios, soportes y tecnologías. En
segundo lugar, contamos con una gran producción de trabajos enfocados en
la transmisión de valores, concienciación con el medio ambiente, con la
inclusión y la necesidad de eliminar fronteras tecnológicas, sociales, y
étnicas. Por último, encontramos una línea dedicada a utilizar la
narración como cauce terapéutico, destinando sus estrategias y recursos
a mejorar la salud física y mental de las personas.

\begin{table}[htbp]
\centering
\begin{threeparttable}
\caption{Principales finalidades pedagógicas de la narración de historias en los artículos analizados.}
\label{tab-07}
\begin{tabular}{ll}
\toprule
Bloque 1 & Narración de historias como parte del proceso de enseñanza y
aprendizaje.\\
Bloque 2 & Narración de historias como medio de transmisión de valores.\\
Bloque 3 & Narración de historias como cauce terapéutico.\\
\bottomrule
\end{tabular}
\source{Elaboración propia.}
\end{threeparttable}
\end{table}

Al enfrentarnos a nuestra tercera pregunta de investigación, y analizar
los conceptos más mencionados en cada uno de los bloques anteriormente
descritos, en el primero de ellos descubrimos un creciente interés en la
comunidad científica por la necesidad de construir historias que
contribuyan al éxito en la aplicación de las denominadas metodologías
activas en el aula. De esta manera, los términos Aprendizaje Basado en
Proyectos, el Aprendizaje Basado en Resolución de Problemas y el
Aprendizaje Cooperativo, son habituales en los trabajos analizados. Del
mismo modo, se identifica un gran volumen de referencias sobre el uso de
la tecnología y los recursos digitales como herramientas de trabajo. En
este sentido, los términos \enquote{Robot}, \enquote{Narrativa Digital}, \enquote{Mapa Digital}, \enquote{Social Media}, \enquote{Dispositivos móviles}, o \enquote{Realidad Aumentada} son los más mencionados.

Con respecto a los trabajos centrados en la concienciación y transmisión
de valores, el concepto de \enquote{Desarrollo Sostenible} es el más repetido,
y también identificamos un gran interés en lo referido a crear
estrategias dirigidas a frenar el cambio climático, la inclusión de
minorías y la conservación del patrimonio inmaterial de las diferentes
culturas. También detectamos una notable producción enfocada en el
intento de disminuir la brecha digital y la igualdad de oportunidades.

En cuanto al tercer bloque, al analizar el área de la salud y el ámbito
terapéutico, los principales ejes de interés se posicionaron entorno a
conceptos como el acompañamiento a pacientes, el estudio de casos de
éxito en medicina, la atención a niños con trastornos del espectro
autista, las terapias psicológicas con animales y el arte, y el refuerzo
de la autoestima.

Además, cabe destacar que desde todas las áreas específicas de
conocimiento se hallaron dos grandes tendencias. La primera de ellas
referida al concepto \enquote{Narración Transmedia}, entendida como un
proceso en el que los elementos integrales de una historia se dispersan
sistemáticamente a través de múltiples canales de distribución con el
fin de crear una experiencia de entretenimiento unificada y coordinada
\cite{jenkins_transmedia_2010}. El otro núcleo de interés está relacionado con la pandemia del COVID-19, y numerosos artículos inciden en la necesidad de
estructurar la narración para mantener la atención de los estudiantes
durante el proceso de \enquote{e-Learning} o \enquote{Aprendizaje a distancia}, así como de la importancia de emplear la tecnología para conseguirlo. Por último, cabe destacar otros dos campos que están presentes en un gran número de estudios y generando tendencia, que son los \enquote{Entornos Inteligentes de Aprendizaje} y la \enquote{Realidad Aumentada}.

Desde el punto de vista metodológico, encontramos una mayoría de
estudios mixtos, que combinan técnicas cuantitativas y cualitativas,
entre las que destacan los estudios etnográficos. Con respecto a las
investigaciones que derivan de las ramas del área de la salud,
identificamos una gran producción en forma de estudios de casos.
