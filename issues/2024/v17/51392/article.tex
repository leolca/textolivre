\documentclass[spanish]{textolivre}

% metadata
\journalname{Texto Livre}
\thevolume{17}
%\thenumber{1} % old template
\theyear{2024}
\receiveddate{\DTMdisplaydate{2024}{2}{27}{-1}}
\accepteddate{\DTMdisplaydate{2024}{4}{17}{-1}}
\publisheddate{\DTMdisplaydate{2024}{6}{13}{-1}}
\corrauthor{Manuel Fernando Ramos Nunez}
\articledoi{10.1590/1983-3652.2024.51392}
%\articleid{NNNN} % if the article ID is not the last 5 numbers of its DOI, provide it using \articleid{} commmand 
% list of available sesscions in the journal: articles, dossier, reports, essays, reviews, interviews, editorial
\articlesessionname{articles}
\runningauthor{Sánchez Rivas et al.}
%\editorname{Leonardo Araújo} % old template
\sectioneditorname{Hugo Heredia Ponce}
\layouteditorname{João Mesquita}

\title{Revisión de la producción científica sobre Storytelling mediado por tecnología entre 2019 y 2022 a través de SCOPUS}
\othertitle{Revisão da produção científica sobre Storytelling mediado por tecnologia entre 2019 e 2022 através do SCOPUS}
\othertitle{Review of scientific production concerning Storytelling mediated by technology between 2019 and 2022 through SCOPUS}

\author[1]{Enrique Sánchez Rivas~\orcid{0000-0003-2518-2026}\thanks{Email: \href{mailto:enriquesr@uma.es}{enriquesr@uma.es}}}
\author[1]{Manuel Fernando Ramos Núnez~\orcid{0000-0002-5991-5454}\thanks{Email: \href{mailto:fernandelaware@gmail.com}{fernandelaware@gmail.com}}}
\author[1]{José Sánchez Rodríguez~\orcid{0000-0003-4525-8761}\thanks{Email: \href{mailto:josesanchez@uma.es}{josesanchez@uma.es}}}
\author[1]{María Rubio-Gragera ~\orcid{0000-0002-8311-8498}\thanks{Email: \href{mailto:jmrubiogr@uma.es}{mrubiogr@uma.es}}}

\affil[1]{Universidad de Málaga, Facultad de Ciencias de la Educación, Departamento de Didáctica y Organización Escolar, Málaga, España.}

\addbibresource{article.bib}
\usepackage{csquotes}

\begin{document}
\maketitle
\begin{polyabstract}
\begin{abstract}
El poder de la narración de historias, conocido como “Storytelling”, se ha revelado como una técnica eficaz e indispensable para la transmisión del conocimiento en el ámbito educativo. Este artículo ofrece un estudio de las publicaciones científicas indexadas que respaldan el uso didáctico de las historias mediado por tecnología. El periodo seleccionado abarca desde 2019 hasta 2022. Para realizar la revisión, se aplicó un algoritmo de búsqueda basado en criterios específicos, mediante los que se realizó una revisión de la base de datos SCOPUS. Los datos obtenidos se interpretaron desde una perspectiva cuantitativa, para después aportar un estudio cualitativo mediante el análisis e interpretación de las aportaciones de cada uno de los artículos. Se evidencia un aumento anual y progresivo en la producción científica, con un estancamiento en 2020, y se identifica a los autores, investigaciones e instituciones con mayores aportaciones en el campo objeto de estudio. En cuanto al análisis del contenido, en primer lugar, podemos destacar un creciente interés de la comunidad educativa por la necesidad de construir narrativas que contribuyan a mejorar las metodologías activas en el aula, así como de la integración de estas en entornos tecnológicos. Como fruto de esta tendencia identificamos el auge del uso de términos como “Narración Transmedia”, “Entornos Inteligentes de Aprendizaje” y “Realidad Aumentada”, que están proliferando y abriendo nuevas vías de investigación. Otra de las tendencias destacables puede enmarcarse en el ámbito medioambiental, con numerosos estudios que persiguen reinventar narrativas para concienciar sobre la necesidad de frenar el cambio climático. Por último, detectamos también la utilización de las narrativas para mejorar procesos médicos y terapéuticos, especialmente aplicados a la formación de nuevos profesionales.

\keywords{Ciencias sociales \sep Tecnologías \sep Educación \sep Creatividad \sep Estudio bibliométrico}
\end{abstract}

\begin{portuguese}
\begin{abstract}
O poder da narração de histórias tornou-se uma técnica eficaz e essencial para a transmissão de conhecimentos na Educação. Este artigo apresenta uma revisão de publicações científicas indexadas que apoiam o uso didático de histórias mediadas por tecnologia. A linha do tempo selecionada é de 2019 a 2022. Para esta revisão, foi aplicado um algoritmo de busca baseado em critérios específicos, que apoiou a realização de uma revisão da base de dados SCOPUS. Os dados obtidos foram interpretados a partir de uma perspetiva quantitativa. Em seguida, foi feito um estudo qualitativo através de uma análise e interpretação dos contributos de cada um dos artigos. Evidencia-se um aumento anual e progressivo da produção científica, embora também se observe uma parada em 2020. São também identificados os autores, os trabalhos de investigação e as instituições com maiores contributos nesse domínio. Quanto à análise de conteúdo, em primeiro lugar, podemos destacar um interesse crescente da comunidade educativa na necessidade de construir narrativas que contribuam para melhorar as metodologias ativas na sala de aula, bem como a integração destas em ambientes tecnológicos. Como resultado dessa tendência, identificamos o surgimento do uso de termos como "Narrativa Transmídia", "Ambientes Inteligentes de Aprendizagem" e "Realidade Aumentada", que estão proliferando e abrindo novas vias de pesquisa. Outra tendência destacável pode ser enquadrada no âmbito ambiental, com numerosos estudos que buscam reinventar narrativas para consciencializar sobre a necessidade de combater as mudanças climáticas. Por fim, também detectamos uma ampla utilização das narrativas para melhorar processos médicos e terapêuticos, especialmente aplicados à formação de novos profissionais.

\keywords{Ciências sociais \sep Tecnologias \sep Educação \sep Criatividade \sep Estudo bibliométrico}
\end{abstract}
\end{portuguese}

\begin{english}
\begin{abstract}
The power of storytelling has become an effective and essential technique for knowledge transmission in Education. This article provides a review of indexed scientific publications that support the didactic use of stories mediated by technology. The selected timeline is from 2019 to 2022. For this review, a search algorithm was applied based on specific criteria, that helped to make a review of the SCOPUS database. The obtained data were interpreted from a quantitative perspective. After this, a qualitative study was made by an analysis and interpretation of the contributions of each of the articles. A yearly and progressive increase in scientific production is evidenced, although it is also observed a standstill in 2020. Authors, research works, and institutions with the greatest contributions in this field are also identified. Regarding content analysis, firstly, we can highlight a growing interest from the educational community in the need to construct narratives that contribute to improving active methodologies in the classroom, as well as their integration into technological environments. As a result of this trend, we identify the rise of terms such as "Transmedia Narration," "Intelligent Learning Environments," and "Augmented Reality," which are proliferating and opening new avenues of research. Another noteworthy trend can be framed in the environmental field, with numerous studies seeking to reinvent narratives to raise awareness about the need to curb climate change. Finally, we also detect a significant use of narratives to enhance medical and therapeutic processes, especially applied to the training of new professionals.
    
\keywords{Social sciences \sep Technologies \sep Education \sep Creativity \sep Bibliometric study}
\end{abstract}
\end{english}
\end{polyabstract}

% !TeX root = main.tex

\section{Introducción}\label{sec-introducción}

La educación mediática es un proceso pedagógico y comunicativo que busca
que la ciudadanía desarrolle habilidades críticas que permitan analizar
el universo de medios que se proyectan en la sociedad postdigital
\cite{Escaño_2023}. Esta perspectiva convierte este ámbito de la
alfabetización en un trabajo por la inclusión de todos los sectores de
la sociedad, especialmente los más vulnerables que, por unos u otros
motivos, no pueden acceder a determinada información o son más
susceptibles a los peligros de esta. La educación mediática apuesta, por
tanto, por la inclusión, al buscar la promoción de la equidad, la
diversidad y la participación en el proceso educativo cite{guillen2024}. La integración de los principios que
fundamentan la inclusión en un planteamiento mediático es fundamental
para conseguir que la ciudadanía desarrolle las competencias digitales y
mediáticas imprescindibles para poder sobrevivir a la brecha de la
sociedad posdigital, potenciando su pensamiento crítico y, en
consecuencia, su actitud crítica \cite{palacios-rodríguez2025}. En este contexto hipermediatizado, la educación mediática
proporciona oportunidades de aprendizaje a través de \emph{mass media} y
\emph{social media} de manera equitativa y accesible, incluidas aquellas
personas que, debido a su edad, han sido excluidas o marginadas del
proceso de digitalización.

La Educación Abierta y a Distancia (EAD) ha jugado un papel importante
en el ámbito de la educación mediática y, como consecuencia, en la
creación de redes comunicativas globales al posibilitar la conexión de
estudiantes de distintas ubicaciones geográficas y contextos culturales.
La UNED desempeña un papel destacado desde 1972 en España promocionando
y desarrollando este modelo formativo y el acceso a la educación
superior de calidad a un amplio abanico de personas. Más
específicamente, en el ámbito de la EAD, la UNED ha colaborado
activamente por la incursión de la educación mediática, a través de los
\emph{Massive Open Online Courses} (MOOC), ofreciendo éstos a través de
su plataforma Uned Abierta, o de otras conocidas como \emph{Coursera},
\emph{edX}, \emph{Miriadax}, \emph{EcoDigitalLearning,} tmooc.es, etc.,
contribuyendo así al intercambio de conocimiento y la
internacionalización de la universidad. Este modelo formativo, diseñado
desde hace más de quince años (Ratnasari; Chou; Huang, 2024), puede
colaborar en la promoción de la inclusión social, al ofrecer un acceso
global, flexibilidad, costos reducidos, adaptabilidad, diversidad de
temáticas, interacción, participación y construcción colectiva del
conocimiento. Se promueve así un aprendizaje a lo largo de la vida
buscando la convergencia mediática: vídeos, lecturas, cuestionarios,
foros, recursos gamificados, creaciones con IA \cite{Aparicio-Gómez2024,cárdeasbenavides2024} e incluso
entornos inmersivos incluyendo los metaversos \cite{huesoromero2024,chuchuca2024metaverso}.

En el presente estudio hemos optado, dentro de las diferentes tipologías
de los MOOC, por el modelo sNOOC \cite{quintana2024snooc}. Tradicionalmente,
se han presentado los SPOC, xMOOC, cMOOC, sMOOC, tMOOC, COMOOC, pero,
actualmente también nos encontramos con los \emph{Nano Open Online
Courses} - NOOC \cite{clark2013moocs,gomez2016,Osuna-Acedo2017,LopezdelaSerna_Garrido_2018,Escaño_Dewhurst_2024}, un planteamiento
concreto y personalizado que estructura su duración en horas y articula
su estructura en torno a un contenido, herramienta o habilidad concreta
\cite{MANANGÓN-CABRERA2023,huesoromero2024}; con una duración máxima de veinte
horas \cite{intef2016}. Una formación en módulos comunicativos e
interactivos más minimalista que se presenta como novedosa, líquida y
flexible de contenidos, de recursos, de espacio y de tiempo \cite{basantes2020}. Perfilando aún más
nuestra elección, hemos apostado en el estudio por los sNOOC o
\emph{social}NOOC que posibilitan más aún el empoderamiento social de
redes comunicativas de estudiantes gracias a la creación colectiva del
conocimiento, la base de la cultura participativa y la apuesta por la
fidelización-compromiso del alumnado convertido en \emph{e-teacher}.
Esta implementación de los sNOOC se presenta como herramienta de
evaluación continua en la EAD, centrándose en experiencias y
valoraciones del alumnado, los elementos comunicativos y pedagógicos,
así como el proceso de construcción colaborativa de los contenidos,
poniendo énfasis en pedagogías inclusivas, que se enriquecen de las
metodologías activas, la creación con IA y metaverso \cite{galíndez2024}. Estas prácticas pedagógicas inclusivas, desde un diseño universal
de aprendizaje \cite{SanchezFuentes2023}, valoran la diversidad de
habilidades, intereses, experiencias y estilos de aprendizaje,
garantizan la equidad, aseguran la adaptabilidad utilizando estrategias
flexibles adaptadas a las diversas capacidades, promueven la
participación, fomentan la colaboración de la comunidad virtual y la
creación de entornos seguros, acogedores y estimulantes.

En este artículo presentamos el análisis de una experiencia que parte de
la creación de redes comunicativas en estudiantes de posgrado de la
Universidad Nacional de Educación a Distancia (UNED) con la finalidad de
convertirlos en \emph{e-teacher,} implementando proyectos formativos
para que las personas de la tercera edad, en riesgo de exclusión
digital, adquieran competencias mediáticas. La creación de esta red
comunicativa sólida parte de un espacio académico de la asignatura de
``Escenarios virtuales para la participación'' y se proyecta como
vinculación y transferencia a la sociedad. Los recursos comunicativos
cobran protagonismo en estas redes, no sólo a través de la plataforma
del curso concreto, sino también en la creación con el uso de la
Inteligencia Artificial (IA) o de espacios inmersivos de una plataforma
sNOOC, donde no sólo las personas participantes puedan formarse, sino
también puedan conectarse, discutir temas relevantes, compartir recursos
o difundir por el \emph{software} social ideas originales. Este proyecto
formativo lleva a la potencialización de la mentoría entre iguales,
unidos en red de intereses y en diversidad de propósitos, donde se
establecen relaciones más cercanas entre el alumnado y el aprendizaje
colaborativo.

La formación mediática de personas de un sector, que se puede considerar
a nivel mediático en peligro de exclusión, como es el de la tercera
edad, es un reto que ya han iniciado otros agentes educativos y sociales
desde diferentes instituciones \cite{abad-acala2014,abad-acala2017,Leal-Maridueña2017,heredia-sánchez2023}. En este caso
concreto, se ha visto beneficiada gracias a la solidaridad y
empoderamiento de redes de estudiantes unedianos \cite{Swan2015,Reich2015}, y el
compromiso por la formación mediática, promoviendo un mejor sentido de
la responsabilidad cívica \cite{bringle1996}, convirtiéndose en un
planteamiento pionero en esta etapa de formación. En esta experiencia,
el compromiso se materializa como un puente tangible que vincula la
formación a través de sNOOC con las problemáticas sociales como es la
alfabetización mediática de las personas de la tercera edad, pocas veces
productoras o creadoras de contenidos y muchas más consumidoras pasivas
de las redes sociales, en comparación con generaciones más jóvenes,
sobre todo en plataformas como Instagram, TikTok, YouTube, X o Facebook.
Los adultos mayores suelen comunicarse a través de estas interfaces con
amistades y familiares, comparten recuerdos, siguen noticias o temas
interesantes, e incluso, las y los más valientes, se animan a formar
parte de comunidades. Las plataformas, conocedoras de esta situación,
están adaptando sus interfaces y servicios para ser más accesibles y
fáciles de usar para personas de estas edades. Partiendo de esta
realidad inequitativa se puede y se debe impulsar, a través de proyectos
formativos de alfabetización mediática como el que presentamos en esta
investigación, una mejora de las habilidades, competencias y capacidades
de este grupo demográfico, especialmente afectado por la brecha digital
generacional.

% !TeX root = main.tex

\section{Metodología}\label{sec-metodología}

La metodología de la investigación la entendemos como el conjunto de
procedimientos y técnicas que el equipo investigador ha utilizado en el
diseño, desarrollo y análisis del estudio. En este caso concreto, el
método utilizado ha sido de corte mixto, utilizando técnicas
cualitativas y cuantitativas. Las cualitativas se han basado en la
etnografía virtual de los datos generados en los sNOOC y las
conclusiones del juicio del equipo de expertos. Las cuantitativas
provienen de los cuestionarios de satisfacción del alumnado y de los
datos de interacción del alumnado en la plataforma de aprendizaje.


\subsection{Objetivos e hipótesis}\label{sub-sec-objetivosehipotesis}

El objetivo general de este estudio es analizar el proceso de creación
de redes comunicativas de estudiantes para la implementación de sNOOC
como método de evaluación continua en la UAD y su repercusión en la capa
social como modelo de formación mediática en personas de la tercera
edad. Con base en este objetivo general, los objetivos específicos hacen
referencia a:

\begin{itemize}
\item
Objetivo Específico 1 (OE1): Investigar las percepciones y opiniones
de las redes comunicativas de estudiantes respecto a la utilidad y
efectividad de un sNOOC como método de evaluación continua y su
impacto en la motivación hacia el aprendizaje.
\item
Objetivo Específico 2 (OE2): Examinar el proceso de desarrollo de las
redes comunicativas de estudiantes para la creación de contenidos de
los sNOOC, centrándose en el impacto del uso de pedagogías inclusivas,
IA y Metaverso en EAD.
\item
Objetivo Específico 3 (OE3): Evaluar el nivel de implicación activa de
las redes comunicativas de estudiantes en la plataforma de la UNED y
en la creación colaborativa del sNOOC en tmooc.es.
\end{itemize}

A continuación, se formulan las hipótesis para dar respuesta a las
relaciones causales:

\begin{itemize}
\item
Hipótesis 1 (H1-OE1): Si las redes comunicativas de estudiantes
perciben el sNOOC como una herramienta efectiva y útil para la
evaluación continua, aumentará su motivación intrínseca hacia el
aprendizaje y su participación en las actividades y recursos del
itinerario de aprendizaje propuesto.
\item
Hipótesis 2 (H2-0E2): Si el modelo sNOOC es diseñado y aplicado
considerando criterios pedagógicos inclusivos y herramientas
tecnológicas adecuadas, mejorará la comprensión de los contenidos por
parte de las redes comunicativas de estudiantes, incrementando su
satisfacción general con la experiencia de EAD.
\item
Hipótesis 3 (H3-OE3): Si el itinerario de aprendizaje en sNOOC está
basado en pedagogías inclusivas, se incrementará el compromiso activo
de las personas participantes, reflejado en una mayor interacción,
colaboración en equipo y corresponsabilidad en la construcción
colectiva del conocimiento.
\end{itemize}


\subsection{Muestra, instrumentos y análisis de
	datos}\label{sub-sec-muestrainstrumentos}
	
	El objeto de estudio de esta investigación son las interacciones del
	alumnado en la plataforma ALF de la UNED, contando con la participación
	de 79 personas, 57 mujeres y 22 hombres; 1 de nacionalidad croata y, el
	resto, española. Estos participantes han sido estudiantes del Máster
	Universitario en Educación y Comunicación en la Red y, dentro de este,
	de la asignatura ``Escenarios Virtuales para la participación'', una
	disciplina con contenidos relacionados con la educación mediática. En
	este caso concreto, para estructurar el método cuantitativo, se han
	utilizado los cuestionarios con preguntas diseñadas para recopilar datos
	cuantitativos correspondiente al curso 2023/2024.
	
	Referido a la plataforma ``tmooc.es'' donde este grupo de estudiantes
	creó los sNOOC, se ha realizado un análisis de estas propuestas tomando
	también esos entornos como objeto de estudio. Se tuvieron en cuenta los
	registros de datos relacionados con la dedicación en la creación de los
	sNOOC. Los sNOOC seleccionados son los siguientes: ``Introdúcete al
	mundo de Facebook'' (sN1), ``Senior 3.0'' (sN2), ``Correo electrónico
	son misterios: alfabetización digital para personas mayores'' (sN3),
	``Enredados en la edad dorada: dominar Facebook e Instagram con
	confianza'' (sN4), ``Healthy seniors network'' (sN5), ``Estas a un clic
	de conocer el mundo digital'' (sN6), ``Familias y aprendizaje en red''
	(sN7) y ``Google e inteligencia artificial, tus compañeros digitales''
	(sN8). En cuanto al enfoque cualitativo, se consideraron los datos
	generados a través de los sNOOC y las conclusiones del juicio de 22
	personas expertas internacionales, con el fin de validar hipótesis y
	evaluar riesgos o problemáticas presentes en el proyecto formativo. Para
	analizar los datos cuantitativos y cualitativos se utilizaron los
	programas SPSS y Atlas.ti, respectivamente. Estos aspectos se han
	organizado en categorías que se ajustan a las dimensiones de la
	educación inclusiva.

\section{Análisis y resultados}\label{sec-análisisyresultados}

Los resultados de este estudio se representan organizados según los
criterios de análisis establecidos. En primer lugar, se muestran los
números de artículos publicados por periodos temporales, es decir, por
años naturales. En ellos se evidencia un aumento progresivo anual de la
producción, y 2021 como el año con el incremento más pronunciado \Cref{tab-01}.

\begin{table}[htbp]
\centering
\begin{threeparttable}
\caption{Número de artículos por año.}
\label{tab-01}
\begin{tabular}{lll}
\toprule
Año & Artículos & Diferencia\\
\midrule
2019 & 26 & - \\
2020 & 28 & 2 \\
2021 & 40 & 12\\
2022 & 51 & 11\\
\bottomrule
\end{tabular}
\source{Elaboración propia.}
\end{threeparttable}
\end{table}

A partir de 2021 se observa una mayor preocupación por explorar nuevas
fórmulas narrativas. Encontramos una razón para ello en la crisis del
COVID-19. De acuerdo con diversos autores \cite{monroy_retos_2021,torras_virgili_emergency_2021,una_martin_aproximacion_2020,vilhelm_rios_didactica_2022}, consideramos que la
pandemia y la crisis sanitaria reformularon muchos planteamientos
pedagógicos tradicionales, generando la necesidad de experimentar con
modelos más flexibles, personalizados, con tecnologías y metodologías
más innovadoras.

El número de citas de los artículos es muy variable, y no puede
establecerse un patrón de comportamiento, ni correlación entre las
fechas de publicación y el impacto de los artículos. Aunque generalmente
los artículos más antiguos suelen tener mayor número de citas, en el
caso que nos ocupa los dos artículos más destacados fueron publicados en
2021. Entre toda la producción seleccionada, podemos destacar un trabajo
que alcanza más de sesenta citas, y otros siete con cifras superiores o
iguales a veinte \Cref{tab-02}

\begin{table}[htbp]
\centering
\begin{threeparttable}
\caption{Número de citas por año.}
\label{tab-02}
\begin{tabular}{lllll}
\toprule
\multicolumn{1}{c}{Artículo} & \multicolumn{4}{c}{Citas por año} \\
  &2019& 2020& 2021& 2022\\
\midrule
Secundo, Mele, Vecchio, Elia, Margherita, \& Ndou (2021) &-& - &16& 51\\
Agbo, Oyelere, Suhonen, \& Tukianen (2021)& -& - &12 &25\\
Kerr, \& Lawson (2020)& - &6& 16 &11\\
Flórez-Aristizábal \& Fardoun (2019)& 3& 4 &7& 12\\
Naul, \& Liu (2020) &- &1& 9& 15\\
Sarnok, Wannapiroon, \& Nilsook (2019) &1 &4& 8& 10\\
Kaeophanuek, Na-Songkhla, \& Nilsook (2019) & 2 & 4& 8& 7\\
\bottomrule
\end{tabular}
\source{Elaboración propia.}
\end{threeparttable}
\end{table}

Atendiendo al número de citas, debemos destacar tres artículos que están
liderando la tendencia en investigación sobre el \emph{storytelling}
mediado por el uso de tecnología y sus posibilidades pedagógicas. Se
trata de los trabajos de \cite{secundo_threat_2021,agbo_scientific_2021,kerr_augmented_2020}.

El primero de ellos surge en el contexto de crisis del COVID-19, y en él
se abordan los principales retos tecnológicos que la pandemia generó en
el ámbito educativo. Los autores ilustran el rediseño de un programa de
aprendizaje para emprendedores, aprovechando las tecnologías digitales y
a nivel práctico, aportan ideas para remodelar los programas
universitarios tradicionales, preparándolos para abordar emergencias
futuras.

El segundo artículo ofrece una amplia revisión bibliográfica en la que
se examina el panorama de la investigación sobre los Entornos
Inteligentes de Aprendizaje, mediante un análisis bibliométrico. El
resultado de este análisis indica que el primer artículo sobre esta
nueva realidad se publicó en 2002, lo que supuso el inicio en la
exploración de este campo. Los Entornos Inteligentes de Aprendizaje
aluden a entornos físicos enriquecidos con dispositivos digitales,
sensibles al contexto y adaptativos, para promover un aprendizaje más
rápido y eficaz \cite{koper_conditions_2014}. Su análisis temático muestra que la narración digital y sus componentes asociados, como la realidad virtual, el pensamiento crítico y los Juegos Serios, que son juegos creados para proporcionar un contexto de entretenimiento y auto-fortalecimiento con
el que motivar, educar y entrenar a los jugadores  \cite{chipia_lobo_juegos_2011}. Consideramos que se trata de dos líneas de investigación emergentes.

Por último, en el artículo de \cite{kerr_augmented_2020}, se describe el
desarrollo de un prototipo de Realidad Aumentada, creado para formar a
estudiantes en los principios básicos de la arquitectura paisajística.
El enfoque principal de los autores se centra en proponer nuevas
prácticas de narración digital a través de experiencias situadas. De
este trabajo inferimos la necesidad de mejorar los programas de
formación docente para capacitar a los futuros maestros en la
integración de las tecnologías inmersivas, como también indica \cite{figueroa_fusionando_2021}.

El tercer indicador hace referencia a las autorías, y nos permite
destacar a los investigadores que más han aportado en el campo objeto de
estudio. En la siguiente tabla mostramos sus nombres y filiación
institucional, a través de la cual podemos observar una gran
representación de Australia y Malasia país que, junto a Tailandia,
lidera sus investigaciones desde universidades centradas en el
desarrollo tecnológico \Cref{tab-03}.

Este dato se corrobora al confrontarlo con las instituciones destacadas
en la investigación sobre \emph{Storytelling} \Cref{tab-04}.

\begin{table}[htbp]
\centering
\begin{threeparttable}
\caption{Investigadores destacados.}\label{tab-03}
\begin{tabular}{lll}
\toprule
Investigador & Filiación & Artículos\\
\midrule
Nilsook, Prachyanun &King Mongkut´s University of Technology & 3\\
Buendgens-Kosten, Judith & Goethe-Universität Frankfurt am Main & 2\\
Chubko, Nadezhda & Edith Cowan University & 2\\
Cornillie, Frederik & KU Leuven & 2 \\
Girmen, Pınar &  Eskişehir Osmangazi Üniversitesi & 2\\
Kantathanawat, Thiyaporn K. & Mongkut's Institute of Technology Ladkrabang & 2\\
Kantathanawat, Thiyaporn K. & Mongkut\textquotesingle s Institute of Technology Ladkrabang & 2\\
Lummis, Geoff W. & Edith Cowan University& 2\\
Macleroy, Vicky & Goldsmiths, University of London &2\\
McKinnon, David H. & Edith Cowan University & 2\\
Morris, Julia Elizabeth & Edith Cowan University & 2\\
\bottomrule
\end{tabular}
\source{Elaboración propia.}
\end{threeparttable}
\end{table}


Al analizar las filiaciones (\Cref{tab-04}) de los principales autores y producción por
países, encontramos en Europa y América del Norte el mayor volumen de
aportaciones. El continente americano, lidera las investigaciones que
relacionan el \emph{Storytelling} con el ámbito de la medicina y la
salud, aunque también hay una gran representación europea debido a las
aportaciones de Reino Unido. En lo referente al campo de las ciencias
sociales, Europa es quien lidera y marca las tendencias. España, que
ocupa el cuarto puesto de la tabla en número de aportaciones por países
con un total de catorce artículos, destaca por su contribución a los
campos relacionados con la aplicación de la narrativa en entornos
digitales, nuevos soportes, y en el uso de metodologías activas. En este
sentido, podríamos señalar a este país como uno de los líderes en la
investigación sobre innovación y tecnología educativa. Este interés por
el \emph{Storytelling} aplicado a metodologías activas y recursos
digitales, continúa aumentando y ha sido identificado en otros estudios
recientes vinculados a universidades españolas \cite{sanchez-rivas_narrative-based_2022,sanchez_rivas_experiencia_2023}.

\begin{table}[htbp]
\centering
\begin{threeparttable}
\caption{Instituciones destacadas.}
\label{tab-04}
\begin{tabular}{lll}
\toprule
Institución & País & Artículos \\
\midrule
Universiti Kebangsaan & Malasia & 3\\
Goldsmiths, University of London & Reino Unido & 3\\
Universiti Teknologi Malaysia & Malasia & 3\\
King Mongkut\textquotesingle s Institute of Technology Ladkrabang &
Tailandia &3\\
Edge Hill University & Reino Unido & 3\\
National and Kapodistrian University of Athens & Grecia & 3\\
King Mongkut\textquotesingle s University of Technology North Bangkok&
Tailandia &3\\
Universitat Oberta de Catalunya & España & 2\\
Universiti Utara Malaysia & Malasia & 2\\
David Geffen School of Medicine at UCLA & Estados Unidos & 2\\
Universidad de Oviedo& España & 2\\
\bottomrule
\end{tabular}
\source{Elaboración propia.}
\end{threeparttable}
\end{table}

 En lo que respecta a la producción por países, Estados Unidos ocupa la primera posición con veintiuna aportaciones, seguido de Reino Unido y Australia. España ocupa la cuarta posición de la tabla con una aportación de catorce artículos, el mismo número que Australia \Cref{tab-05}.
 
\begin{table}[htbp]
\centering
\begin{threeparttable}
\caption{Producción por países.}
\label{tab-05}
\begin{tabular}{ll}
\toprule
País & Artículos \\
\midrule
Estados Unidos & 21\\
Reino Unido & 19\\
Australia & 14\\
España & 14\\
Grecia & 11\\
Malasia & 10\\
Canadá & 9 \\
Italia & 9\\ 
\bottomrule
\end{tabular}
\source{Elaboración propia.}
\end{threeparttable}
\end{table}

Si atendemos a las líneas de investigación abiertas en relación con el
\emph{Storytelling,} advertimos que su estudio se aborda desde
diferentes disciplinas. Aunque el enfoque de las investigaciones siempre
tiene una perspectiva pedagógica, podemos asegurar que el tratamiento
científico de las mismas se acomete desde una perspectiva
multidisciplinar \Cref{tab-06}.

\begin{table}[htbp]
\centering
\begin{threeparttable}
\caption{Producciones por disciplina científica.}
\label{tab-06}
\begin{tabular}{ll}
\toprule
País & Artículos \\
\midrule
Ciencias Sociales & 107\\
Ciencias Informáticas & 40\\
Artes y Humanidades & 22\\
Ingeniería & 16\\
Medicina & 15\\
Psicología & 15\\
Administración y negocios & 6\\
Ciencias Medioambientales & \\
Ciencia de Materiales & 5\\
Ciencias de la Salud &5\\
\bottomrule
\end{tabular}
\source{Elaboración propia.}
\end{threeparttable}
\end{table}

A la hora de abordar nuestro análisis cualitativo, y dar respuesta a
nuestra primera pregunta de investigación, se realizó un análisis de los
campos de estudio y de los contenidos desde los que se abordaban los
diferentes artículos publicados, es decir, aquellos ámbitos con mayor
interés en el uso de historias como recurso didáctico. Las Ciencias
Sociales, las Ciencias Informáticas y las Artes y Humanidades fueron las
áreas más productivas. Cabe destacar que, si a la medicina le sumamos
las publicaciones de otros campos afines como el de las ciencias de la
salud o el de la psicología, ocuparían el tercer puesto. Esto es debido
a la gran cantidad de estudio de casos que utilizan estas ciencias tanto
para enseñar, como para contrastar información en el desempeño de su
trabajo.

A esta revisión de naturaleza cuantitativa, se añadió otra de tipo
documental, para responder a nuestra segunda pregunta e identificar las
finalidades pedagógicas de las líneas temáticas más investigadas en cada
una de las principales áreas de estudio. Los trabajos analizados podrían
organizarse en tres grandes bloques (\Cref{tab-07}). En primer lugar, aquellos que ponen su foco en el proceso de aprendizaje y utilizan las técnicas vinculadas a las historias para mejorar el proceso didáctico. Dentro de esta vertiente, encontramos una gran tendencia a trasladar los modelos
clásicos de la narración a nuevos medios, soportes y tecnologías. En
segundo lugar, contamos con una gran producción de trabajos enfocados en
la transmisión de valores, concienciación con el medio ambiente, con la
inclusión y la necesidad de eliminar fronteras tecnológicas, sociales, y
étnicas. Por último, encontramos una línea dedicada a utilizar la
narración como cauce terapéutico, destinando sus estrategias y recursos
a mejorar la salud física y mental de las personas.

\begin{table}[htbp]
\centering
\begin{threeparttable}
\caption{Principales finalidades pedagógicas de la narración de historias en los artículos analizados.}
\label{tab-07}
\begin{tabular}{ll}
\toprule
Bloque 1 & Narración de historias como parte del proceso de enseñanza y
aprendizaje.\\
Bloque 2 & Narración de historias como medio de transmisión de valores.\\
Bloque 3 & Narración de historias como cauce terapéutico.\\
\bottomrule
\end{tabular}
\source{Elaboración propia.}
\end{threeparttable}
\end{table}

Al enfrentarnos a nuestra tercera pregunta de investigación, y analizar
los conceptos más mencionados en cada uno de los bloques anteriormente
descritos, en el primero de ellos descubrimos un creciente interés en la
comunidad científica por la necesidad de construir historias que
contribuyan al éxito en la aplicación de las denominadas metodologías
activas en el aula. De esta manera, los términos Aprendizaje Basado en
Proyectos, el Aprendizaje Basado en Resolución de Problemas y el
Aprendizaje Cooperativo, son habituales en los trabajos analizados. Del
mismo modo, se identifica un gran volumen de referencias sobre el uso de
la tecnología y los recursos digitales como herramientas de trabajo. En
este sentido, los términos \enquote{Robot}, \enquote{Narrativa Digital}, \enquote{Mapa Digital}, \enquote{Social Media}, \enquote{Dispositivos móviles}, o \enquote{Realidad Aumentada} son los más mencionados.

Con respecto a los trabajos centrados en la concienciación y transmisión
de valores, el concepto de \enquote{Desarrollo Sostenible} es el más repetido,
y también identificamos un gran interés en lo referido a crear
estrategias dirigidas a frenar el cambio climático, la inclusión de
minorías y la conservación del patrimonio inmaterial de las diferentes
culturas. También detectamos una notable producción enfocada en el
intento de disminuir la brecha digital y la igualdad de oportunidades.

En cuanto al tercer bloque, al analizar el área de la salud y el ámbito
terapéutico, los principales ejes de interés se posicionaron entorno a
conceptos como el acompañamiento a pacientes, el estudio de casos de
éxito en medicina, la atención a niños con trastornos del espectro
autista, las terapias psicológicas con animales y el arte, y el refuerzo
de la autoestima.

Además, cabe destacar que desde todas las áreas específicas de
conocimiento se hallaron dos grandes tendencias. La primera de ellas
referida al concepto \enquote{Narración Transmedia}, entendida como un
proceso en el que los elementos integrales de una historia se dispersan
sistemáticamente a través de múltiples canales de distribución con el
fin de crear una experiencia de entretenimiento unificada y coordinada
\cite{jenkins_transmedia_2010}. El otro núcleo de interés está relacionado con la pandemia del COVID-19, y numerosos artículos inciden en la necesidad de
estructurar la narración para mantener la atención de los estudiantes
durante el proceso de \enquote{e-Learning} o \enquote{Aprendizaje a distancia}, así como de la importancia de emplear la tecnología para conseguirlo. Por último, cabe destacar otros dos campos que están presentes en un gran número de estudios y generando tendencia, que son los \enquote{Entornos Inteligentes de Aprendizaje} y la \enquote{Realidad Aumentada}.

Desde el punto de vista metodológico, encontramos una mayoría de
estudios mixtos, que combinan técnicas cuantitativas y cualitativas,
entre las que destacan los estudios etnográficos. Con respecto a las
investigaciones que derivan de las ramas del área de la salud,
identificamos una gran producción en forma de estudios de casos.

\section{Discusión y conclusiones}\label{sec-discusiónyconclusiones}

Partiendo de los resultados obtenidos, consideramos que se refuerza un
planteamiento, ya apuntado por otros autores \cite{alfian_role_2021,sidorenko-bautista_use_2020}, que
sostiene que las narrativas deben cambiar para adaptarse a los nuevos
medios, y que los medios están cambiando continuamente para adaptarse a
las narrativas. En este sentido, se observa un gran interés de la
comunidad científica por adaptar las técnicas clásicas de la narración a
los nuevos medios, soportes, tecnologías y entornos, para logar conectar
con sus audiencias. La necesidad de vehicular contenidos a través de
estructuras narrativas creíbles resulta esencial ya que, de acuerdo con
\cite{vergara2020herramientas}, podemos decir que si algo nos
caracteriza desde el punto de vista cultural es que somos depredadores
de buenas historias.

Como podemos observar, el interés por el \emph{Storytelling} ha superado
la barrera de las Ciencias Sociales, posicionándose como método de
transmisión de conocimiento en diversas disciplinas. El
\emph{Storytelling} es una herramienta poderosa que ha sido utilizada
durante siglos en una variedad de ámbitos, desde la literatura hasta la
publicidad, desde la política hasta la psicología. Su capacidad para
inspirar, educar y entretener va mucho más allá de las ciencias sociales
\cite{haven_story_2007}.

Los hallazgos que hemos realizado nos permiten concluir que la narrativa
sigue siendo fundamental para la articulación de contenidos y el proceso
de enseñanza y aprendizaje. La comunidad científica muestra un gran
interés por adaptar la narrativas y técnicas del \emph{Storytelling} a
los nuevos escenarios educativos, metodologías y herramientas
tecnológicas. La continua exploración de las posibilidades de estas
técnicas está despertando el interés de diferentes disciplinas que
persiguen beneficiarse del carácter pedagógico de las historias.

Podemos establecer otra línea de conclusiones en relación con las
temáticas que se abordan en las historias. Hemos detectado un gran
interés de la comunidad científica por construir narrativas eficaces que
puedan ayudar a comprender las claves del cambio climático. En este
sentido, resulta interesante la vía abierta por diversos trabajos \cite{arnoud_climate_2018,bloomfield_climate_2021,de_meyer_transforming_2021,moezzi_using_2017} que recomiendan trasladar, utilizando
técnicas de comunicación efectiva, estas narrativas climáticas a todas
las esferas de la sociedad, y no solo al ámbito educativo.

Desde la perspectiva didáctica, se hace evidente el interés por crear
propuestas metodológicas específicas que permitan aprovechar todo el
potencial del \emph{Storytelling} en el aula. Los métodos activos son,
en todos los casos, las opciones preferidas. Entendemos que se trata de
una relación lógica, ya que el Storytelling guarda una relación de
coherencia con los rasgos propios del aprendizaje activo (implicación
emocional, cooperación, creatividad, búsqueda del conocimiento, etc.).
Hemos encontrado diferentes estrategias metodológicas para utilizar las
historias como recurso didáctico. Muchas se integran en el método
didáctico Aprendizaje Basado en Historias.

Otros de los fenómenos a los que la comunidad científica debe prestar
atención en el futuro es al de los \emph{Entornos Inteligentes de
Aprendizaje} y a \emph{la Narración transmedia}. De todas estas nuevas
vías, surgen fortalezas como el potencial pedagógico de las narrativas
aplicadas al pensamiento reflexivo o a los nuevos medios digitales \cite{kim_digital_2021,yasar__2022} y otros debates éticos, como la
moralidad de utilizar técnicas de manipulación, como podría considerarse
el \emph{Storytelling} dentro del contexto de las redes sociales tras
demostrarse los efectos negativos que estas producen en la salud mental
\cite{sheldon_dark_2019}.

En cuanto al uso de la tecnología que todas estas metodologías
combinadas con narrativas propician, se imponen con claridad
instrumentos como los dispositivos móviles y la realidad aumentada. En
este sentido, numerosos autores \cite{aurelia_survey_nodate,dunleavy_augmented_2014,nam_designing_2015,nobrega_mobile_2017} apuntan al potencial de la combinación de ambas tecnologías como un buen modelo de enseñanza debido a que cualquier lugar u objeto del mundo real puede integrarse en una historia, estimulando la imaginación del participante. Aunque también señalan que existen algunas barreras que impiden a los usuarios participar activamente en este tipo de actividades, como la
necesidad de disminuir la brecha digital y el elevado coste de los
dispositivos y la tecnología.

Al contrastar nuestra revisión con otras similares, podemos comprobar
que las conclusiones confluyen en varios aspectos. El primero de ellos
sería en la positiva percepción sobre la aplicación de la narración
digital de historias por parte de los estudiantes. Esta idea ha sido
destacada en otras revisiones que también señalan el efecto novedad como
posible artífice de esta aceptación. Se propone continuar investigando
para ver cómo se desarrolla esta relación en el futuro. El segundo
aspecto a tener en cuenta sería la necesidad de formar a los docentes en
el uso de los entornos digitales y la creación, ya que la narración de
historias continúa capturando la atención de audiencias, convirtiéndose
en un aliado indispensable de cualquier proceso didáctico y formativo,
por lo que necesitamos formar a nuestros docentes \cite{chikasanda_enhancing_2013,lawless_professional_2007}.

En relación con futuras líneas de investigación, cabe apuntar que el
importante índice de experimentación didáctica que hemos encontrado,
unido a la necesidad de una constante adaptación de las narrativas a los
nuevos entornos inteligentes y digitalizados, derivarán en nuevas formas
de enseñar y aprender en el futuro, a las que la comunidad científica
deberá prestar atención para evaluar su potencial y medir su eficacia.


\printbibliography\label{sec-bib}
%conceptualization,datacuration,formalanalysis,funding,investigation,methodology,projadm,resources,software,supervision,validation,visualization,writing,review
\begin{contributors}[sec-contributors]
\authorcontribution{Enrique Sánchez Rivas}[conceptualization,methodology,resources,writing]
\authorcontribution{Manuel Fernando Ramos Núnez}[conceptualization,datacuration,investigation,validation,writing]
\authorcontribution{José Sánchez Rodríguez}[formalanalysis,review,supervision]
\authorcontribution{María Rubio-Gragera}[formalanalysis,review,supervision]
\end{contributors}
\end{document}
