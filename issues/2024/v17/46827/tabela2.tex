\begin{table}[htpb]
	\centering
	
	\begin{threeparttable}
		\caption{The categorisation processes}
		\label{tb2}
	
	\begin{tabularx}{\textwidth}{X X X X}
	
	
			\toprule
				
	\multirow{1}{*}{\textbf{Objective: To comprehend the medias as a critical pedagogy scenario to promote reexis-}}

     \multirow{1}{*}{\textbf{tence literacies in pre-service English teachers}}
		\\
		
	\textbf{Open coding} & \textbf{Axial coding} & \textbf{Category (ies)} & \textbf{Topic} \\
			

		
		Average (dis)advantages & \textbf{Scope/Media project} & Media as a critical pedagogy scenario & Media as a critical pedagogy scenario to explore social issues and challenge dominant power structures \\ 
		Coloniality & Critical pedagogy & Inclusive and resistant environments for English teacher education & Education for Equity: Building Resistant Environments for English Teacher Education \\
		Creation of resistance narratives & Critical media & English teachers and digital media role & Preservice English teachers’ roles in/from media scenarios \\
		Critical English teaching & Critical practice &  &  \\
		Critical literacies & Resistance narratives &  &  \\
		Critical media & Power structures &  &  \\
		Critical pedagogy & Promotion of reexistence literacies &  &  \\
		Critical media & Resistance &  &  \\
		Digital ethics & Social issues &  &  \\
		Digital media consumption & \textbf{Resistance Context and English Education} &  &  \\
		Digital media project & School &  &  \\
		Digital media & Critical English teaching &  &  \\
		Digital practices & English teaching &  &  \\
		English teaching & Negotiations &  &  \\
		Media user role & Realities of the participants &  &  \\
		Negotiations & Resistance &  &  \\
		Power structures & \textbf{Average (dis)advantages} &  &  \\
		Promotion of reexistence literacies & Digital media role &  &  \\
		Realities of the participants & Digital Ethics &  &  \\
		Resistance school & Digital media &  &  \\
		Social issues & Digital media consumption &  &  \\
				& Digital practices &  &  \\ 
				& Critical literacies &  &  \\ 
				&Language Teachers &  &  \\ 
		& Media user role &  &  \\ 
			\bottomrule
		\end{tabularx}
		\end{threeparttable}
		\source{Own elaboration}
	
\end{table}



	
