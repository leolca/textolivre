\section{Discussion}\label{sec-discussion}
The analysis of participants' use and experience with
AI tools (R.Q.1) provides valuable insights into their engagement with
emerging technologies in both academic and non-academic domains. The
data reveals a diverse range of use patterns among participants,
showcasing a spectrum of preferences and approaches towards AI adoption.
While some participants actively employ AI tools for tasks such as idea
generation, translation, and language instruction, others exhibit a more
reserved attitude, either abstaining from their use entirely or
employing them sparingly. Moreover, participants display variability in
their choice of AI platforms, with preferences ranging from widely
recognized tools like Deepl and ChatGPT, as outlined by \textcite{schmidt2022} and to lesser-known options such as TWEE and Canva.
Notably, there is an evident openness among participants to explore new
AI technologies, as indicated by intentions to adopt novel tools and
experiment with additional functionalities of existing platforms.
Furthermore, participants frequently integrate multiple AI tools into
their workflow, combining different platforms to address various aspects
of academic tasks effectively. This integration underscores a holistic
approach to leveraging AI technologies, wherein participants draw upon
the strengths of each tool to optimize their productivity and outcomes.
Overall, the findings suggest that participants'
utilization and experience with AI are characterized by diversity,
variability, openness to exploration, integration of multiple tools, and
varied levels of experience. These insights provide a comprehensive
understanding of how participants engage with AI technologies to support
their academic and non-academic endeavors, emphasizing the multifaceted
nature of AI utilization in contemporary educational contexts.

These findings highlight the high familiarity and proficiency of
participants with AI for general purposes, which may translate into an
appreciation for its potential professional applications. As most
pre-service teachers belong to the digital-native generation, they are
likely to adapt easily to AI technology and recognize its potential for
future practice.

Concerning the R.Q.2, the results of the SWOT analysis are summarised in
\Cref{tab-02}. The internal analysis shows a clear representation of strengths
and weaknesses related to the nature of the app: participants report
practical uses of the resource for the teacher's perspective such as the
different activities, the speediness and usefulness of the software.
Regarding the weaknesses, they factor diversity, both in terms of
students with special needs (no option to adapt the activities within
the app) as well as language constrictions (only in English), while also
mentioning interface issues (e.g. video conversion). These findings
align with those by \textcite{ravshanovna2023}.

The results found on the external analysis of opportunities and threats
resonate to some extent with the answers for the internal analysis; this
can be explained by the fact that TWEE is an educational tool itself so
teachers analyse it as such. Therefore, the quick creation of diverse
and innovative activities is reported once again along with the
opportunities it provides for paying attention to students' likes, which
can be linked to the idea of personalised learning thanks to AI \cite{chen2020}. As far as threats and dangers, participants focus on the
detriments it may pose for students in terms of ethical concerns (e.g.
cheating) and learning factors (e.g. effort and attention span), while
the major concern for teachers would be to become superfluous.


\begin{table}[!htbp]
\centering
\caption{Results of the SWOT Analysis.}
\label{tab-02}
\begin{tabular}{ll}
\toprule
Strengths & Weaknesses\\
- Variety of resources & -Students with special needs\\
- Usefulness and practicality & - Interface issues\\
- Immediacy & - English only\\
-Accessibility & - Premium tools\\
-Creativity & - Human dimension\\
\midrule
Opportunities & Threats\\
- Quick creation of activities & - Ethical concerns (e.g. misuse)\\
- Diverse and innovative activities & - Drawbacks to students (e.g. lack of effort)\\
- Students’ likes & - Abuse of AI\\
\bottomrule
\end{tabular}
\source[Own elaboration (2023).
\end{table}


In line with \textcite{hartono2023}, participants' attitudes and
perceptions (R.Q.3) towards the use of AI are overall favorable as they
report positive feelings concerning its use during classroom practice.
Similarly, when asked about their possible future use of AI, the
majority (90.5\%) of prospective FL teachers believe they will use AI
(in this case, TWEE) in their future careers, which resonates with the
openness towards new technologies registered in R.Q.1. It bears noting
that this positive overview is linked (to some extent) to participants'
use of AI in other realms.

Concerning their future use, it is clear they believe AI to be suited
for written tasks such as written comprehension activities and those
related to use of English (vocabulary and grammar): this is related to
the fact that TWEE does not provide the option for feedback to students
(speaking) and its main input is in written format, thus, being found
more suitable for the abovementioned activities. Furthermore, in line
with \textcite{yang2022}, students' motivation is one of the reported areas
which would benefit the most from the use of AI, as well as enhancing
the use of technological tools by prospective students