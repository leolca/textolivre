\section{Introduction}\label{sec-introduction}

The educational landscape is undergoing significant transformation owing
to technological advancements, with the incorporation of Artificial
Intelligence (AI) into language teaching practices serving as a notable
manifestation of this paradigmatic shift. This evolution traces its
roots back to the initial adoption of Computer Assisted Language
Learning (CALL), a term encompassing any utilization of technology for
language instruction \cite[p.~1841]{tafazoli2020}. Acknowledging the
crucial role of language educators in cultivating effective language
learning environments underscores the necessity of comprehending their
perceptions towards AI in language teaching.

As we find ourselves at the crossroads of conventional pedagogy and
technological progress, investigating the perspectives and attitudes of
prospective language instructors concerning the integration of AI
becomes imperative. Consequently, this study seeks to examine the
viewpoints and attitudes of aspiring foreign language teachers in the
secondary education sector in Spain, with a specific focus on the
utilization of AI in their instructional methodologies. This examination
will particularly explore the implementation and impact of TWEE, a
widely used AI tool among language instructors. To guide this inquiry,
three research questions have been formulated:

R.Q. 1: How do participants use and what is their level of experience
with AI?

R.Q. 2: What are the strengths, weaknesses, opportunities, and threats
of using the digital application "TWEE: A.I. Powered Tools For English
Teachers" based on prospective teachers´ perceptions?

R.Q.3: What are the prospective language teachers´ perceptions and
attitudes of using AI regarding their future practice?

To achieve this objective, participants' responses were
analyzed using the SWOT analysis method, which entails the scrutiny of
Strengths, Weaknesses, Opportunities, and Threats. This analysis delved
into internal and external factors with potential implications for the
implementation of the aforementioned tool. The inclusion criterion for
participants is restricted to individuals enrolled in the
Master's Degree program in Secondary Education with a
specialization in Foreign Language Teaching (FLT) at the University of A
Coruña (UDC). The research design employed in this study is
characterized by a quasi-experimental and exploratory approach, thereby
adopting a mixed-methods paradigm. This comprehensive strategy
incorporates both quantitative and qualitative methodologies for data
collection.