\section{Artificial Intelligence in Language Teaching}\label{sec-artificialintelligencein}
The integration of AI, particularly ChatGPT, in language teaching offers
numerous opportunities but comes with challenges that require thoughtful
consideration and strategic approaches. Teachers are encouraged to
embrace AI as a complementary tool, continually enhance digital
literacy, engage in human-machine collaboration, and shift towards
student-centered education to navigate the evolving landscape of
language teaching. \textcite[p.~167]{schmidt2022} discuss the
classification of key concepts in AI-powered language learning tools
scenarios as outlined by \textcite{baker2019}. They identify three
main categories:

\begin{itemize}
    \item Learner-facing AI tools: These tools are designed to assist students directly in improving their language skills. They typically incorporate features like practice patterns, feedback mechanisms, and drills aimed at enhancing comprehension and proficiency. An example cited is Babbel, an application that offers immediate feedback tailored to the learner's input, with a focus on areas such as mixed tenses and verb forms.
    \item Teacher-facing systems: This category encompasses tools intended to alleviate the burden on educators by automating various aspects of their work. These tools handle tasks such as grading assignments, providing feedback to students, managing classroom activities, and handling administrative duties. For instance, GradeScanner is highlighted as a tool capable of automatically grading multiple-choice tests, thereby saving teachers valuable time and effort.
    \item System-facing AI tools: These tools primarily serve institutional administrators or stakeholders by providing them with processed data. By analyzing data such as student transcripts and performance metrics, these tools offer insights that can inform strategic decisions at the institutional level. They may utilize algorithms to predict future student performance or identify trends within the learning environment, aiding in the development of effective policies and interventions.
\end{itemize}


\textcite[p.~119–120]{ravshanovna2023} discusses the benefits and drawbacks of
integrating AI into language teaching. AI offers personalized training
programs based on individual student needs, automates evaluation tasks
like grading essays and tests, and enhances feedback processes. It also
streamlines educational management and improves access to education,
particularly for remote learners through online courses. However,
challenges include AI's occasional lack of accuracy,
difficulty in adapting to diverse contexts, and limited ability to
interpret human emotions. Moreover, AI solutions may lack clarity and
struggle with creative thinking. Additionally, the implementation and
maintenance of AI systems require qualified personnel. While AI presents
opportunities for language teaching, it cannot entirely replace human
involvement, emphasizing the importance of balancing technology with
human intervention for optimal results.

The systematic review conducted by \textcite{sharadgah2022} provides a
comprehensive exploration of the literature on the integration of AI in
English Language Teaching (ELT) from 2015 to 2021. The review identifies
various AI applications in ELT, including automatic correction,
listening skill enhancement, oral training through AI-based robots,
machine translation, personalized teaching, and experimental studies
evaluating AI-based systems. While highlighting the positive impact of
AI on ELT efficiency and student satisfaction, the study also points out
challenges and gaps in the literature, emphasizing the need for clear
evaluation criteria and future research directions.

In a study by \textcite[p.~449]{hockly2023}, various AI-powered tools supporting
language development are discussed, such as Duolingo, Write \& Improve,
Grammarly, Google Translate, and chatbot apps. The focus is on the
effectiveness of chatbots, which have proven beneficial, especially for
lower-proficiency learners, offering clearly delimited contexts for
language practice. Despite some limitations, chatbots contribute to
improved learning confidence, motivation, and self-efficacy. Teachers
are encouraged to recommend chatbot apps for additional exposure to
English language content outside class time, enhancing overall learning
outcomes.

Moreover, \textcite[p.~78]{huang2023} highlight the extensive
opportunities presented by ChatGPT in foreign language teaching. For
learners, ChatGPT addresses limitations of traditional classrooms,
providing opportunities for communication, easy access to learning
resources, and individualized learning strategies. It fosters interest
in learning through interactive conversations. For teachers, ChatGPT
serves as a multifaceted tool, assisting in resource provision,
conversation practice, assessment, and virtual teaching. The combination
of teachers and AI tools is seen as a new model for effective teaching
and learning.

However, challenges identified by \textcite[p.~83]{huang2023} include
potential reductions in intercultural communication skills, the
weakening of self-correction abilities, difficulties in ensuring content
accuracy, concerns about cheating, and inherent limitations in AI tools.
To address these challenges, the authors propose teaching strategies,
emphasizing the acceptance of AI as a supportive tool, the importance of
continuous learning and digital literacy, human-machine collaboration,
and a shift towards student-centered, competency-focused education.