\documentclass[english]{textolivre}

% metadata
\journalname{Texto Livre}
\thevolume{17}
%\thenumber{1} % old template
\theyear{2024}
\receiveddate{\DTMdisplaydate{2024}{3}{20}{-1}}
\accepteddate{\DTMdisplaydate{2024}{4}{23}{-1}}
\publisheddate{\today}
\corrauthor{María Bobadilla-Pérez}
\articledoi{10.1590/1983-3652.2024.51692}
%\articleid{NNNN} % if the article ID is not the last 5 numbers of its DOI, provide it using \articleid{} commmand 
% list of available sesscions in the journal: articles, dossier, reports, essays, reviews, interviews, editorial
\articlesessionname{articles}
\runningauthor{Galán-Rodríguez; Bobadilla-Pérez; Barros-Grela}
%\editorname{Leonardo Araújo} % old template
\sectioneditorname{Hugo Heredia Ponce}
\layouteditorname{João Mesquita}

\title{Attitudes and perceptions: the role of artificial intelligence in the training of future secondary school foreign language teachers}
\othertitle{Atitudes e percepções: o papel da inteligência artificial na formação de futuros professores de línguas estrangeiras do ensino secundário}
\author[1]{Noelia Mª Galán-Rodríguez~\orcid{0000-0001-6662-7269}\thanks{Email: \href{mailto:noelia.galan@udc.es}{noelia.galan@udc.es}}}
\author[1]{María Bobadilla-Pérez~\orcid{0000-0002-4972-5980}\thanks{Email: \href{mailto:m.bobadilla@udc.es}{m.bobadilla@udc.es}}}
\author[2]{Eduardo Barros-Grela~\orcid{0000-0002-7533-5580}\thanks{Email: \href{mailto:ebarros@udc.es}{ebarros@udc.es}}}
\affil[1]{Universidade da Coruña, College of Education, Department of Specific Teaching Training and Research and Diagnosis Methods in Education, A Coruña, Spain.}
\affil[2]{Universidade da Coruña, College of Arts and Letters, Department of English, A Coruña, Spain.}

\addbibresource{article.bib}

\newcommand\subsubsubsection[1]{%
  \paragraph{#1}\mbox{}\\
}

\begin{document}
\maketitle
\begin{polyabstract}
\begin{abstract}
This study explores prospective foreign language teachers' attitudes towards integrating artificial intelligence (AI) into their teaching practices, focusing on the TWEE tool. Using a mixed-methods approach, data from 29 pre-service secondary school teachers in Spain were analyzed using SWOT analysis to assess TWEE's strengths, weaknesses, opportunities, and threats. Internal strengths include resource variety and practicality, while weaknesses involve interface issues and language constraints. External opportunities include quick activity creation, while threats center on ethical concerns and teacher redundancy risks. Despite concerns, participants generally express a positive attitude towards AI, acknowledging its potential to enhance written tasks, motivation, and technological literacy. Pedagogical implications underscore the need for a balanced AI integration approach, considering accessibility and ethical concerns. Further research is suggested to explore AI tool implementation in classrooms. This study contributes to understanding AI's role in language education and informs strategic planning for its integration.


\keywords{Foreign language teaching \sep Artificial intelligence \sep Secondary Education \sep TWEE \sep Pedagogical perspectives}
\end{abstract}

\begin{portuguese}
\begin{abstract}
Este estudo explora as atitudes dos futuros professores de línguas estrangeiras relativamente à integração da inteligência artificial (IA) nas suas práticas de ensino, centrando-se na ferramenta TWEE. Utilizando uma abordagem de métodos mistos, os dados de 29 professores de ensino secundário em início de carreira em Espanha foram analisados utilizando a análise SWOT para avaliar os pontos fortes, os pontos fracos, as oportunidades e as ameaças da TWEE. Os pontos fortes internos incluem a variedade de recursos e o carácter prático, enquanto os pontos fracos envolvem questões de interface e restrições linguísticas. As oportunidades externas incluem a criação rápida de atividades, enquanto as ameaças se centram em preocupações éticas e riscos de redundância de professores. Apesar das preocupações, os participantes expressam geralmente uma atitude positiva em relação à IA, reconhecendo o seu potencial para melhorar as tarefas escritas, a motivação e o letramento tecnológico. As implicações pedagógicas sublinham a necessidade de uma abordagem equilibrada da integração da IA, tendo em conta a acessibilidade e as preocupações éticas. Sugere-se mais investigação para explorar a implementação de ferramentas de IA nas salas de aula. Este estudo contribui para a compreensão do papel da IA no ensino das línguas e serve de base ao planeamento estratégico para a sua integração.

\keywords{Ensino de línguas estrangeiras \sep Inteligência artificial \sep Ensino secundário \sep TWEE \sep Perspectivas pedagógicas}
\end{abstract}
\end{portuguese}
\end{polyabstract}

\section{Introduction}\label{sec-introduction}

Over the past decades there has been a growing interest in the
development of new methodologies for second language (L2) learning and
teaching, to fulfil the requirements of the new legislation, as defined
by the Common European Framework of Reference for Languages (CEFR),
which emphasizes the communicative aspect of languages.

One of these new methodologies is Didactic Audiovisual Translation
(DAT), i.e., the application of different modes of audiovisual
translation (subtitling, revoicing, audio description and voice over) to
L2 teaching. This approach has been implemented and its results analysed
in different educational stages. Nevertheless, this approach has yet
been used in primary education within alternative methodologies, such as
the Montessori Method, in blingual contexts (in this case,
Basque-Spanish). The present study aims to address this research gap.

This paper shows the results of an intervention combining subtitling and
dubbing in a class of 11-12-year-old pupils following the Montessori
Method, at a Basque-immersion language primary school. While our study
could not assess language acquisition improvements due to the
school's methodology, our findings regarding children
satisfaction and the alignment of DAT principles with Montessori
principles, as demonstrated in the text, suggest promising avenues for
successful DAT use in this environment.



\section{Theoretical framework}\label{sec-theoreticalframework}
\subsection{Artificial Intelligence in Education}\label{sub-sec-artificialintelligenceineducation}

The integration of AI in education has garnered significant attention in
recent years, revolutionizing educational goals, practices, and the
learning environment. In this context, \textcite{roll2016} explore the
evolutionary and revolutionary aspects of AI in education, emphasizing
the transformative impact on educational objectives, classroom
practices, and the broader learning environment. Building upon this,
\textcite{chen2020} contribute by highlighting the positive
outcomes and potential of AI systems across administrative,
instructional, and learning domains, providing a comprehensive review of
AI applications in education. \posscite{roll2016} 
study underscores the shift in educational goals, moving away from rigid
knowledge preparation for the workforce towards equipping students as
adaptive experts and on-the-job learners. The ubiquity of smartphones
has transformed educational goals, emphasizing knowledge application,
collaboration, and self-regulated learning. This shift requires a
corresponding change in assessments from summative to ongoing formative
measures. In terms of practices, Roll and Wylie note the incorporation
of authentic elements in classrooms, resulting in increased complexity
and the challenge of personalization. The learning environment has
expanded beyond traditional classrooms to include informal and workplace
learning, transforming teachers from the "sage on the stage" to the
"guide on the side" \cite[p.~592]{roll2016}. Roll and Wylie also
acknowledge challenges posed to AI in education, raising questions about
effective technology support for teachers. \posscite{chen2020} study
complements this by providing a comprehensive review of AI applications
in education, emphasizing AI's positive impact on
administrative tasks, instructional quality, and learning experiences.
AI has the capability in assisting teachers in tasks such as exam
generation and grading. Chen et al. highlight the increasing integration
of AI into education, from early childhood to higher levels, showcasing
the adaptability of AI, including the use of robots (cobots) in teaching
routine tasks. It recognizes the growth in AI-related publications and
explores the diverse applications of AI, emphasizing its transformative
impact on various educational aspects. The positive outcomes of AI
deployment include improved learning quality, collaboration, global
access, enhanced academic integrity, and personalized learning plans.

On the other hand, the increasing integration of AI in education has
prompted extensive exploration into its challenges, implications, and
ethical considerations. \textcite[p.~4]{rodrigues2023} delve into
the philosophical, historical, and practical dimensions associated with
tools such as ChatGPT and AI in education, emphasizing the necessity for
a thoughtful and ethical approach to their integration within
educational frameworks. This discussion is crucial as AI technologies,
including natural language processing and datification, become more
pervasive in educational settings. At the same time, \textcite[p.~4222]{nguyen2023} explore the ethical principles governing AI in
education, focusing on K-12 settings, emphasizing the need for global
standards. The study critically evaluates existing ethical guidelines
and explores the applications of AI, such as personalized learning
systems, automated assessments, facial recognition, and predictive
analytics, in supporting both teachers and students. However, the study
acknowledges ethical challenges, ranging from systemic bias to privacy
concerns, which call for the urgent need to develop comprehensive
ethical guidelines in the field, emphasizing the need for global
standards and unified ethical principles for trustworthy AI. \textcite{rodrigues2023} emphasize the importance of an ethical and
considerate approach to AI integration, situating their study within the
broader context of AI technologies' increasing
influence, particularly in natural language processing. The authors
address concerns about AI's impact on professions,
particularly within education, questioning whether ChatGPT and AI pose a
threat or a challenge to the educational landscape. They highlight the
significance of datification in the Web 4.0 era, emphasizing ethical
considerations in the intertwining of human-machine interactions,
particularly in education. \textcite{nguyen2023} stress the urgent need
to educate teachers and students about ethical concerns and justify the
development of ethical guidelines in the field. Also, \textcite[p.~170]{regan2019} identify six privacy concerns for teachers and learners.
These are: information privacy, anonymity, surveillance, autonomy,
non-discrimination and ownership of information.

In summary, these studies highlight the transformative impact of AI
integration in education, emphasizing shifts towards adaptive learning
methodologies, personalized approaches, and the redefined role of
teachers. AI emerges as an evolving reality within the educational
realm, prompting considerations of ethical, social, and methodological
implications as discussed by the authors. Clear guidelines are deemed
necessary, particularly concerning assessment practices, privacy issues,
and technological infrastructure. Additionally, effective technology
support for teachers is identified as essential, especially regarding AI
tools such as cobots and chatbots.
\section{Artificial Intelligence in Language Teaching}\label{sec-artificialintelligencein}
The integration of AI, particularly ChatGPT, in language teaching offers
numerous opportunities but comes with challenges that require thoughtful
consideration and strategic approaches. Teachers are encouraged to
embrace AI as a complementary tool, continually enhance digital
literacy, engage in human-machine collaboration, and shift towards
student-centered education to navigate the evolving landscape of
language teaching. \textcite[p.~167]{schmidt2022} discuss the
classification of key concepts in AI-powered language learning tools
scenarios as outlined by \textcite{baker2019}. They identify three
main categories:

\begin{itemize}
    \item Learner-facing AI tools: These tools are designed to assist students directly in improving their language skills. They typically incorporate features like practice patterns, feedback mechanisms, and drills aimed at enhancing comprehension and proficiency. An example cited is Babbel, an application that offers immediate feedback tailored to the learner's input, with a focus on areas such as mixed tenses and verb forms.
    \item Teacher-facing systems: This category encompasses tools intended to alleviate the burden on educators by automating various aspects of their work. These tools handle tasks such as grading assignments, providing feedback to students, managing classroom activities, and handling administrative duties. For instance, GradeScanner is highlighted as a tool capable of automatically grading multiple-choice tests, thereby saving teachers valuable time and effort.
    \item System-facing AI tools: These tools primarily serve institutional administrators or stakeholders by providing them with processed data. By analyzing data such as student transcripts and performance metrics, these tools offer insights that can inform strategic decisions at the institutional level. They may utilize algorithms to predict future student performance or identify trends within the learning environment, aiding in the development of effective policies and interventions.
\end{itemize}


\textcite[p.~119–120]{ravshanovna2023} discusses the benefits and drawbacks of
integrating AI into language teaching. AI offers personalized training
programs based on individual student needs, automates evaluation tasks
like grading essays and tests, and enhances feedback processes. It also
streamlines educational management and improves access to education,
particularly for remote learners through online courses. However,
challenges include AI's occasional lack of accuracy,
difficulty in adapting to diverse contexts, and limited ability to
interpret human emotions. Moreover, AI solutions may lack clarity and
struggle with creative thinking. Additionally, the implementation and
maintenance of AI systems require qualified personnel. While AI presents
opportunities for language teaching, it cannot entirely replace human
involvement, emphasizing the importance of balancing technology with
human intervention for optimal results.

The systematic review conducted by \textcite{sharadgah2022} provides a
comprehensive exploration of the literature on the integration of AI in
English Language Teaching (ELT) from 2015 to 2021. The review identifies
various AI applications in ELT, including automatic correction,
listening skill enhancement, oral training through AI-based robots,
machine translation, personalized teaching, and experimental studies
evaluating AI-based systems. While highlighting the positive impact of
AI on ELT efficiency and student satisfaction, the study also points out
challenges and gaps in the literature, emphasizing the need for clear
evaluation criteria and future research directions.

In a study by \textcite[p.~449]{hockly2023}, various AI-powered tools supporting
language development are discussed, such as Duolingo, Write \& Improve,
Grammarly, Google Translate, and chatbot apps. The focus is on the
effectiveness of chatbots, which have proven beneficial, especially for
lower-proficiency learners, offering clearly delimited contexts for
language practice. Despite some limitations, chatbots contribute to
improved learning confidence, motivation, and self-efficacy. Teachers
are encouraged to recommend chatbot apps for additional exposure to
English language content outside class time, enhancing overall learning
outcomes.

Moreover, \textcite[p.~78]{huang2023} highlight the extensive
opportunities presented by ChatGPT in foreign language teaching. For
learners, ChatGPT addresses limitations of traditional classrooms,
providing opportunities for communication, easy access to learning
resources, and individualized learning strategies. It fosters interest
in learning through interactive conversations. For teachers, ChatGPT
serves as a multifaceted tool, assisting in resource provision,
conversation practice, assessment, and virtual teaching. The combination
of teachers and AI tools is seen as a new model for effective teaching
and learning.

However, challenges identified by \textcite[p.~83]{huang2023} include
potential reductions in intercultural communication skills, the
weakening of self-correction abilities, difficulties in ensuring content
accuracy, concerns about cheating, and inherent limitations in AI tools.
To address these challenges, the authors propose teaching strategies,
emphasizing the acceptance of AI as a supportive tool, the importance of
continuous learning and digital literacy, human-machine collaboration,
and a shift towards student-centered, competency-focused education.
\section{Methodology}\label{sec-methodology}

In this paper, a technology developed by Yandex in 2021 that translates
a live stream from English, Spanish, French, Italian, German, or Chinese
into Russian has been used. The translation of a live stream is a
challenging task that has been addressed by the development of a novel
technique based on neural networks. Our study is devoted to the
evaluation of this new technique and focuses on its ability to preserve
the subtleties of semantic meaning and cultural connotations inherent in
phraseological expressions in different linguistic contexts. Although
our study does not address the technical details underlying this tool,
we address these issues as they may help to inform translation. This
technology incorporates advanced machine learning algorithms to enable
the instantaneous translation of language during audiovisual broadcasts.
In essence, this algorithm comprises five fundamental steps, each
executed by a distinct neural network model.

Initially, the audio stream is captured and transcribed into plain text
using automatic speech recognition. The video may contain extraneous
sounds such as noise and music, people may speak with different accents,
speeds and diction, and there may be many speakers, so the technology
must ensure that context and coherence are maintained during the
translation process. Therefore, the algorithm takes a sequence of audio
chunks as input, extracts acoustic features, and passes them into the
neural network. The neural network in turn produces a set of word
sequences from which the language model selects the most plausible
hypothesis.

Subsequently, a machine translation model is employed to translate the
text into the desired target language. There are several problems here:
if you translate word-by-word or phrase-by-phrase, the quality will
suffer, and if you wait for a long pause to guarantee the end of a
sentence, there will be a long delay. So, the technology groups words
into sentences without losing meaning or making sentences too long. For
correct translation at this stage, it is also necessary to determine the
gender of the speaker to determine to whom a particular line belongs and
to reproduce the voice correctly. After selecting individual sentences
and lines, the translation is performed, for which Yandex uses its own
translator.

Once the translation is completed, the translated text is processed
through text-to-speech technology, which converts the written content
into spoken audio. This step ensures that the generated speech sounds
natural and coherent, considering various linguistic features such as
tone, rhythm, and inflection. Additionally, the gender of the speaker,
which was previously identified during the initial stages of the
process, is incorporated into the synthesis to ensure the voice matches
the intended speaker's profile. This level of
customization helps improve the overall quality and authenticity of the
synthesized speech, making it more relatable and contextually
appropriate for the target audience.

Furthermore, the algorithm ensures that the translated speech is
accurately synchronized with the corresponding segments of the video
stream, aligning seamlessly with the visual content, and maintaining
synchronization with the video frames. This process is crucial to ensure
that the audio matches the timing of the speaker's lip movements and
actions in the video. Additionally, the neural network addresses several
challenges in this stage, such as when the speaker delivers a sentence
rapidly, or when the translated sentence is significantly longer than
the original. In these cases, the system must dynamically adjust the
synthesized audio by compressing or shortening it to fit within the
allotted time frame, ensuring smooth and natural speech flow that aligns
with the visual context.

Finally, the translated speech is seamlessly integrated into the live
video stream, replacing the original audio with the newly generated
translated audio. This newly created audio is then encapsulated into an
audio stream, which is embedded directly into the browser interface of
the viewer, allowing them to hear the translated speech while watching
the video in real-time. The technology used for this process is
currently functional exclusively within the Yandex browser, which was
the platform selected for the study. \Cref{fig-01} displays a detailed
screenshot of the Yandex browser interface, clearly highlighting the key
components involved in the translation and audio synchronization
process. It provides a visual representation of how the translated audio
is integrated into the video stream, showing the alignment between the
original video content and the overlaid translated speech. The
screenshot also highlights the user interface elements that facilitate
the viewer's interaction with the translated video, such as volume
controls, language options, and playback features.

\begin{figure}[htpb]
  \centering
  \begin{minipage}{\textwidth}
  \caption{Screenshot of the Yandex browser interface when utilizing 
  the automatic translation features.}
  \label{fig-01}
  \includegraphics[width=\textwidth]{image1.png}
  \source{Author's own work}
  \end{minipage}
\end{figure}

To enhance the study's transparency and validity, it is
crucial to provide comprehensive details regarding the sample selection,
data collection methods, and the analytical techniques employed. The
live video functionality within the interface is activated via a clearly
visible and easily accessible button, allowing for straightforward
interaction with the system. A notable feature of the translation
process is the 40-50 second temporal delay between the original live
video stream and its translated counterpart in Spanish. This delay, as
derived from a detailed evaluation of the system's operation, is
purposefully incorporated to allow sufficient time for the system to
process the content contextually, which is essential for delivering an
accurate translation in real-time, especially when dealing with live
broadcasts. Such a delay also enables the system to manage the
complexities inherent in real-time translation, such as adjusting for
idiomatic expressions, speech nuances, and varying speaking speeds.

Moreover, the Yandex browser's functionality extends beyond mere passive
translation by allowing users to actively customize their experience.
This includes features like adjusting the volume of the original audio
track, which could be crucial in environments where background noise or
other factors might interfere with the audio quality. Additionally, the
availability of subtitles provides an extra layer of understanding and
flexibility for users, especially in cases where visual cues or context
may not fully suffice to ensure a clear understanding of the translated
content. These customizable features add a significant level of
adaptability to the translation process, catering to various user
preferences and enhancing the overall user experience. Such details are
integral to understanding the technical infrastructure of the study and
ensure its findings can be accurately interpreted and replicated in
future research.

This study employed the Yandex browser and a range of accessible
technological tools to explore the automated translation of live news
streams. The analysis focused on two YouTube channels, ``RTVE Noticias''
and ``Canal Sur Andalucía,'' both of which provide continuous news
coverage. These channels were selected as the primary subjects of
investigation due to their ongoing news broadcasts, providing a robust
sample for examining the translation of live content. The methodology
for the study, as outlined in \Cref{fig-02}, was structured to facilitate a
systematic comparison of the automated translation of phraseological
expressions in real-time news streams.

The approach for comparing the automated translation of phraseological
expressions involved several key stages:

\begin{enumerate}
\def\labelenumi{\arabic{enumi}.}
\item
  \emph{Audio recording:} The first step in the methodology involved
  recording both the original live stream (audio recording \#1) and its
  corresponding translation (audio recording \#2) simultaneously,
  ensuring that both recordings occurred at the same intervals during
  the live broadcast. Two separate devices were used to capture these
  audio recordings concurrently, ensuring precise alignment of the
  original and translated content.
\item
  \emph{Detection of phraseologisms (verbal idioms):} In the second
  stage, instances of phraseologisms, commonly used idiomatic
  expressions, were identified in audio recording \#1, which captured
  the live stream in the original language (Spanish). The identification
  process was carefully carried out to ensure that the expressions
  detected were contextually relevant and representative of the language
  used in the broadcast.
\item
  \emph{Translation matching:} Once the phraseologisms were identified
  in the original audio, the next step involved matching the
  corresponding translations of these expressions in audio recording
  \#2, which was in Russian. This step was crucial for establishing
  whether the automated translation system accurately rendered the
  idiomatic expressions in a culturally appropriate and linguistically
  accurate manner.
\item
  \emph{Comparison and analysis:} The final stage entailed a detailed
  comparison of the accuracy and correctness of the translations. A
  comprehensive comparison table was developed to facilitate this
  process, allowing for a side-by-side evaluation of the original and
  translated expressions. The total duration of the audio recordings in
  both Spanish (the original language) and Russian (the translated
  language) amounted to 243 minutes for each language. These recordings
  were made across different intervals, ranging from 15 to 60 minutes,
  and were captured on multiple occasions throughout the day over four
  non-consecutive days. This sampling strategy was employed to ensure a
  diverse representation of news topics covered during the broadcasts.
  As part of the analysis, 52 verbal idioms were identified and
  thoroughly examined within the context of the live news stream,
  providing insights into the effectiveness of the automated translation
  system in handling phraseological expressions in a dynamic, real-time
  setting.
\end{enumerate}

\begin{figure}[htpb]
  \centering
  \begin{minipage}{\textwidth}
  \caption{General outline of the methodology presented.}
  \label{fig-02}
  \includegraphics[width=\textwidth]{image2.png}
  \source{Author's own work.}
  \end{minipage}
\end{figure}




\section{Discussion}\label{sec-discussion}

In regard to the learning outcomes, it is not feasible to assess the
acquisition of the L2 content. However, we have identified recurring
errors in writing and pronunciation that illustrate potential areas for
guidance to facilitate pupil improvement in the language. Additionally,
the children committed mistales in pronunciation and intonation.
Although these issues require resolution, it is acknowledged that
mispronunciations of this nature are common among speakers of Basque and
Spanish, given the inherent challenges posed by the English vocalic
system for those with a limited number of vocalic phonemes.

Regarding the findings of the questionnaire, the outcomes are consistent
with our expectations. The pupils have demonstrated predisposition
towards this pedagogical approach, which integrates stimuli that align
with the fundamental principles of Montessori, namely the diverse
classroom environment to which they are accustomed. It is also
noteworthy that the method incorporates multimodality, which is a key
aspect of the world in which children live. The final products
demonstrate that pupils can successfully handle the L2. This is a
significant finding, as \textcite{alonso-perez2018} have
demonstrated that observing the final outcome and analysing the progress
that the students have achieved ``make everyone feel rewarded'' \cite[p. 21]{alonso-perez2018}, which implies an even stronger boost to motivation.

Conversely, children have demonstrated interest and engagement with the
DAT activities. A recurrent 5-2 parameter is evident when analysing
questions 3, 5 and 6. This signifies that five of the seven children
indicated a high level of enjoyment, with ratings of Bs and As, while
the remaining two children achieved a passing grade of C+. It is
accurate to conclude that the highest grade awarded was not an A+, which
can be attributed to the fact that this task was compulsory and required
the learning of internet tools, which served this purpose.

In examining the reasons for the implementation\textquotesingle s
success or failure, the indicators recur. It is presumed that the two
children who provided the lowest ratings were those who responded in the
negative. The issue is that the respondents did not provide reasons
related to their experiences with DAT activities. One respondent did not
provide any reasons, while the other provided a single negative response
regarding one of the teachers. This is a mishap, as their answers would
have been of great help for the research. Conversely, the remaining five
respondents provided positive feedback, citing enjoyment derived from
video recording, fun during implementation, and, most notably, two
respondents demonstrated appreciation for the activity\textquotesingle s
didactic potential, stating, "It would be good" and "I learn." This
latter response is particularly noteworthy, as it reflects a depth of
reflection that not all children possess. These statements are in
accordance with \textcite{fernandez-costeles2021} regarding the perceptions of
primary education pupils towards the didactic possibilities of DAT.

In light of the previous considerations, it can be posited that the
findings of this survey indicate a favourable outcome. In terms of
learning, this implementation has opened paths for the teacher to work
those aspects of language which have proven to be less developed in
learners. Conversely, with regard to motivation, the utilisation of DAT
has enhanced students motivation, as it is founded upon certain
customary activities among children, such as the recording of videos.
This results in a more enjoyable process of learning the L2, which
undoubtedly facilitates the work of both the teacher and the learners.
This research contributes to the existing body of literature on the
subject, building upon the findings of previous studies conducted by
scholars such as \textcite{neves2004language}, \textcite{talavan2009aplicaciones,talavan2010audiovisual}, \textcite{banos2015clipflair}, \textcite{talavan2015first}, \Textcite{BELTRAMELLO_2019}, \textcite{lertola2019audiovisual}, \textcite{talavan2022audiovisual}, \textcite{rodríguez-Arancón2023} and \textcite{talavan2024}.

\section{Conclusion}

This research has explored the relationship between the Montessori Method and DAT, demonstrating that both have a significant degree of overlap and that the latter can be integrated as an additional workshop within the Montessori classes. We encountered two main limitations. The first limitation is inherent to the methodology itself. Given that children do not take exams, it is not possible to adhere to the experimental scheme that would otherwise be appropriate for this type of research. Consequently, it has not been possible to measure the anticipated improvement in pupils' language acquisition.

A second limitation of the study is the low response rate to the satisfaction questionnaire, with only seven out of twenty-four students completing it. This is a potential issue, as the sample size is small and the responses are not fully representative. However, they do offer insights into a particular trend. Among the pupils who responded, we have been able to ascertain their motivations and levels of enjoyment regarding the implementation of the DAT. Five of them indicated that they found it motivating, while the other two expressed differing pedagogical opinions. The primary issue that emerged from the data was that learners lacked clarity regarding the specific applications of DAT.

In this particular instance, a class of twenty-four children at an educational institution that employs active methodologies and Basque as the L1 have demonstrated the efficacy of DAT from several perspectives. From the viewpoint of the teacher, the role differs from that of the traditional educator. DAT allows teachers to provide pupils with materials that align with their interests, thereby personalising their learning experience. Although pupils are permitted to select activities that exceed their current capabilities, the provision of effective scaffolding can facilitate their learning process. Furthermore, Montessori advocated the use of self-correcting materials, a concept that is exemplified by DAT. Audiovisual materials allow learners to listen to themselves and receive immediate feedback on their performance, which undoubtedly contributes to the improvement of their oral proficiency in the L2. DAT presents new opportunities for stimulating pupils, given its multimodal nature. It offers language in authentic contexts, which helps pupils comprehend the meaning of language in real situations. Thus, by providing a new lexical variety, children can expand their vocabulary to express themselves in different situations. Lastly, the integration of technology and multimodality has enabled learners to personalise their learning experience, allowing them to progress at their own pace and receive education that is specifically tailored to their needs. This is achieved through the creation of new teaching tools, which are based on interactive games, simulations or educational videos. In this regard, DAT has emerged as a highly valuable instrument in the context of this pedagogical approach, facilitating more engaging and efficacious learning, while fostering autonomous learning and self-regulation. 

Importantly, the implementation of DAT is in accordance with both the fundamental principles of the Montessori Method and current legislation. It has been demonstrated that DAT effectively implements the competencies and skills required by law with regard to the teaching of the L2. Furthermore, it facilitates the development of students' cognitive abilities, which is fundamental for lifelong learning.

Future research could encompass a longitudinal study to include an analysis of the role of multilingualism (learners work with three languages: Basque L1 of some pupils and of school, Spanish L1 of some children, and English as L2) in the general learning process and in the Montessori Method.
\section{Conclusion}\label{sec-conclusion}

The findings of this study suggest that the exercise of agency in initial education contexts is multifaceted. Like the findings of \posscite{mercer2011,mercer2012} empirical work focusing on language learners, the analysis of the narratives in this study indicates that agency is influenced by the intricate interconnectedness of various factors and elements coexisting in the participants' systems. It was possible to identify interpersonal factors, such as the perceived opportunities to interact with agents (human and non-human) in formal and informal online environments, as well as intrapersonal factors, such as the impact on emotions, attitudes, and beliefs about the best ways to learn. The data also show that the exercise of agency is dynamic, subject to change, and open, as it can be influenced by other systems.

Throughout the analysis, the relational nature of agency \cite{larsen2019} proved to be quite salient, as the data unveiled a reciprocal interaction between internal and external factors and the actions emerging from these relationships within contexts and their possibilities.

The results point to the potential of mobile devices in facilitating the exercise of agency among the participating pre-service teachers. These devices allow teachers to access information, speed up time, study, and even provide opportunities for distraction. When it comes to their praxis, these teachers also recognize the possibilities of mobile technology to motivate, bring the classroom to the 21st century, and engage learners. The potential for mobile technologies to impact social life was also evident from the data, as participants at various points in their stories emphasized how pervasive and important technology is to everyday life and citizenship.

In terms of the possible implications of this study for teacher education, one of the possible insights may stem from the recognition that in order to understand the possibilities of teacher agency – pre-service and in-service – it is important to consider the environments in which these agents circulate, the technologies and other agents with which they interact, the nested systems that make up their ecosystems, and the ways in which these dynamics affect and feed back into the interaction between intra- and interpersonal aspects.

We recognize the complexity of agency in the context studied, and although some dynamics and the complex fabric of teachers' agency have been highlighted in the data analyzed, further research can highlight other units of analysis in relation to inter- and intrapersonal aspects, elements, and systems that can further the understanding of the role of these "agents", thus contributing to research on teacher agency.

\printbibliography\label{sec-bib}

\begin{contributors}[sec-contributors]
\authorcontribution{Noelia María Galán-Rodríguez}[datacuration,formalanalysis,investigation]
\authorcontribution{María Bobadilla-Pérez}[conceptualization,methodology,writing,validation]
\authorcontribution{Eduardo Barros-Grela}[review,projadm,supervision,visualization]
\end{contributors}

\end{document}
