\section{Introdução}\label{sec-intro}

Nas últimas décadas, os avanços tecnológicos têm modificado não apenas o
mundo do trabalho, requerendo profissionais qualificados, bem como as
relações humanas e impactando o campo educacional. Neste sentido, muitos
trabalhos têm sido reportados na literatura acerca do uso das
tecnologias digitais da informação e da comunicação (TDIC), como
ferramentas para a mediação pedagógica \cite{Peripolli2022, Pires2024}.

Esta preocupação com o uso das TDIC e o desenvolvimento da fluência
tecnológico-pedagógica para o uso destas ficou ainda mais evidente, com
a pandemia do Covid-19 \cite{Joaquim2021, Cleophas2022}.

Nesse sentido, \textcite[p. 2]{Oliveira2022} ressaltam,

\begin{quote}
o ensinar e o aprender em meio à cultura digital demandam cuidados para
que não recaiam em práticas vazias ou sem a devida fundamentação, para
que, de algum modo, possam contribuir ao contemplarem necessidades do
cotidiano do século XXI.
\end{quote}


Este apontamento faz-se ainda mais importante quando se trata da
educação de jovens e adultos, visto que o contexto em que se desenvolve
a ação pedagógica destes é permeado de especificidades. Dentre estas
especificidades destacamos a heterogeneidade etária, indo desde o
público jovem até os adultos, assim como a experiência de vida no mundo
do trabalho bem como, em muitos casos, de um tempo muito longo de
distanciamento da escola.

Assim, embora a inserção das TDIC no contexto educacional seja bastante
discutida, encontra-se ainda uma lacuna quando buscamos informações
acerca do uso das tecnologias digitais no contexto da educação de jovens
e adultos da Educação Profissional. A educação profissional e
tecnológica visa preparar os sujeitos para o mundo do trabalho, nessa
seara, o PROEJA, denominado Programa de Integração da Educação
Profissional ao Ensino Médio na Modalidade Educação de Jovens e Adultos,
foi criado em 24 de junho pelo Decreto 5.478/2005 e visa possibilitar a
formação profissional técnica de Ensino Médio, da qual, segundo o
decreto, são comumente excluídos \cite{Brasil2005}.

\textcite[p. 387]{Medeiros2021} apontam que no ensino de jovens e adultos,

\begin{quote}
qualquer que seja o conteúdo, não se resume a fazer uma breve pesquisa
na internet e aplicá-la nas turmas da modalidade, ou ainda, tentar
adaptar atividades do ensino regular e levá-las para jovens e adultos
responderem. Assim, torna-se necessário conhecer as necessidades
formativas desses discentes para possibilitar que exista uma relação
entre as suas histórias de vida e o conteúdo de sala de aula.
\end{quote}


Neste contexto, a mediação pedagógica apoiada nas TDIC ganha um novo
contorno, pois não é possível meramente adaptar ferramentas comumente
utilizadas no ensino regular para o PROEJA. Faz-se imprescindível que
ocorra uma real transposição de saberes, iniciando pela compreensão de
quem são estes sujeitos, quais suas necessidades e vivências de forma a
planejar a mediação de uma forma mais alinhada aos anseios e
necessidades deste público.

\textcite{Barin.et.al.} apontam ainda que, considerando o objetivo da
Educação Profissional, modalidade de ensino em que o PROEJA encontra-se
inserido, a formação dos sujeitos para o mundo do trabalho, cada vez
mais imerso em tecnologias, cabe repensar o fazer pedagógico e as
políticas públicas de forma a desenvolver as competências digitais dos
estudantes e professores no intuito de minimizar as desigualdades de
acesso à informação.

\textcite{morais2023} destacam em uma perspectiva freiriana, que a
educação propicia aos aprendizes conceberem-se como sujeitos históricos.
Nesse contexto, \textcite[p.~10]{Gadotti2000} afirma que, ``{[}\ldots{]} A
tecnologia contribui pouco para a emancipação dos excluídos se não for
associada ao exercício da cidadania'', portanto não é o uso da
tecnologia em si, que propicia a minimização das desigualdades e a
emancipação dos excluídos, mas a forma como esta é utilizada no contexto
educacional.

Ainda nessa linha, \textcite{Alvarenga_Lemos_Neto_2020} sinalizam que é
imprescindível que o uso das tecnologias digitais da informação e da
comunicação sejam acompanhadas de um objetivo e propósito pedagógico
claro, que oriente a ação docente e a aplicabilidade desta para o
público de adultos do PROEJA.

Assim, a mensuração do progresso científico e da produção de
conhecimento sobre o uso das TDIC para o público de jovens e adultos da
Educação Profissional e Tecnológica é de grande valia para compreensão
das tendências e das necessidades de estudo (lacunas) desta temática.
Esta mensuração de tendências e lacunas é objeto de estudo da
cienciometria.

A cienciometria é um ramo da ciência que tem como objetivo compreender
como a produção de conhecimento tem se delineado, nos mais diversos
campos do saber. Neste sentido o presente trabalho visa traçar um perfil
cienciométrico das dissertações e teses disponíveis no Biblioteca
Digital Brasileira de Dissertações e Teses sobre o uso das tecnologias
digitais no contexto da Educação de Jovens e Adultos da Educação
Profissional e Tecnológica.