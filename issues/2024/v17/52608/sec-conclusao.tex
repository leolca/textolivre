\section{Conclusão}\label{sec-conclusão}

O presente trabalho apresenta o delineamento da produção de saberes
sobre as TDIC no contexto da Educação Profissional e Tecnológica, mais
especificamente no PROEJA a partir da perspectiva cienciométrica. Como
pode-se observar pelo número de trabalhos retornantes, embora as
tecnologias digitais venham ocupando seu espaço no contexto da sala de
aula, pouco ainda é discutido sobre seu uso por adultos maduros.

Com base nos resultados retornantes, pode-se inferir que as mulheres vêm
se destacando na pesquisa, superando o número de trabalhos desenvolvidos
por homens. Este dado está em consonância com os fomentos para inserção
das mulheres na Ciência e aponta, embora timidamente, para uma conquista
feminina.

A distribuição dos trabalhos quanto a regionalidade, demonstra pequenos
avanços nas políticas de incentivo a descentralização do eixo sudeste,
que em grande parte das áreas de pesquisa se sobressaem em número, pois
é a região com maior número de programas de pós-graduação e de fomento à
pesquisa. Embora não se tenha encontrado nenhum trabalho em programas de
pós-graduação na região norte, uma das dissertações é direcionada ao
público do PROEJA no Instituto Federal de Rondônia (IFRO), o qual fica
situado nesta região.

Quantos aos aspectos metodológicos, pode-se verificar que pelo fato de a
maior parte dos trabalhos ter como origem programas na área de Educação
e Ensino, as pesquisas são, em sua maioria, de caráter qualitativo,
tendo instrumentos de coleta de dados similares e priorizarem a análise
de conteúdo, característica deste campo do saber.

Embora o presente trabalho vise contribuir para sanar a lacuna de
estudos sobre o uso de tecnologias digitais no contexto educacional de
jovens e adultos da educação profissional, o mesmo é apenas uma gota de
água num oceano, visto que a compreensão das necessidades e os desafios
da formação integral que o mundo do trabalho requer não pode ser sanada
sob um único olhar. Neste sentido, aponta-se para a necessidade de
fomentar e incentivar estudos para propiciar o uso das tecnologias
digitais não apenas como elemento de mediação pedagógica, mas também
como possibilidade do desenvolvimento das competências digitais do
cidadão para sua inserção no o mundo do trabalho.

\section{Agradecimentos}\label{sec-agradecimentos}

À CAPES pela bolsa de pesquisa.