\section{Introdução}\label{sec-introdução}

Neste artigo, buscamos apresentar um relato de experiência sobre o
processo de elaboração de atividades didáticas para um curso
\emph{online} de natureza assíncrona, oferecido no ambiente
virtual Moodle, destinado ao ensino de francês língua estrangeira (FLE)
para alunos de um centro de extensão de uma universidade federal
brasileira.

O curso foi desenvolvido no quadro de um programa de formação de
professores em línguas clássicas e modernas que busca congregar três
vertentes da vida acadêmica: o ensino, por meio de práticas
didático-pedagógicas desenvolvidas pelos alunos-estagiários (doravante
estagiários); a pesquisa, com a possibilidade de os estagiários se
dedicarem ao estudo aprofundado de temas relacionados ao ensino e a
aprendizagem de línguas estrangeiras; a extensão, já que os cursos
ministrados são abertos à comunidade externa e interna da universidade
em questão. Neste programa de formação, há um projeto destinado ao
ensino do francês, em vários níveis, que propõe uma formação continuada
didático-pedagógica e linguística aos estagiários. Essa formação foi
ministrada por um professor efetivo da habilitação em francês que, no
centro de extensão, atuava tanto como supervisor do idioma francês
quanto como professor-formador. Para o curso que descreveremos, dois
estagiários mostraram-se interessados em elaborar o material e
receberam, durante o período de trabalho, uma bolsa acadêmica de
extensão.

Durante a formação, coube ao professor-formador atuar junto aos
estagiários e concentrar-se na elaboração do material didático para o
curso, em suas mais diferentes etapas, pelo prisma teórico-metodológico
proposto pelo quadro do Interacionismo Sociodiscursivo \cite{bronckart_atividade_1999,bronckart_teorias_2021,schneuwly_generos_2004,graca_da_2023,tocaia_leitura_2019,tocaia_letramento_2022}. Durante a formação, também foram realizados encontros semanais de orientação, momento em que foram
lidos e discutidos conceitos teórico-pedagógicos relativos à teoria
sociointeracionista. O objetivo central desses encontros era construir,
junto aos estagiários, um conjunto de conhecimentos comuns,
materializados não só pela discussão de propostas e práticas
pedagógicas, mas também pela discussão de conceitos da corrente teórica
em questão. Ao final desta etapa de estudo, um projeto de ação, que
compreendia a elaboração do material para o primeiro módulo do curso,
intitulado \emph{Français Online 1}, foi executado.

Neste projeto de ação, quatro grandes pressupostos nortearam a
elaboração do material para o curso. O primeiro relacionou-se ao fato de
que aprender uma língua, seja ela a própria língua materna (LM) ou uma
língua estrangeira (LE), é aprender a se comunicar, isto é, agir por
meio da linguagem, amparado pela produção e compreensão de gêneros
textuais diversos, nas múltiplas situações sociais. Neste primeiro caso,
a preocupação foi preparar os alunos que desejavam aprender o idioma
francês a dominar operações linguageiras que subsidiassem a compreensão
e a produção de gêneros textuais, uma vez expostos a uma variedade de
discursos orais e escritos em LE. Ao prepararmos o aluno para o domínio
da LE em diferentes situações sociais, buscávamos auxiliá-lo a construir
representações cada vez mais complexas da língua-alvo, encorajando-o a
transpor seus limites no aprendizado e a agir socialmente em LE. O
segundo pressuposto dizia respeito ao desenvolvimento de capacidades de
linguagem \cite{dolz-mestre_acquisition_1993,schneuwly_generos_2004}, que
poderiam ser mobilizadas não apenas para a produção dos gêneros textuais
propostos no material do curso, mas que possivelmente seriam transpostas
a outros gêneros textuais diferentes daqueles estudados, desde que
apresentassem aspectos contextuais, discursivos e
linguístico-discursivos similares. O terceiro era o desejo, enquanto
professor-formador, de que os próprios estagiários, por meio de uma
experiência de transposição didática \cite{chevallard_transposition_1981}, entendida, grosso modo, como um conjunto de transformações que um dado conhecimento de cunho científico (saber científico) sofre ao ser transposto para o ensino escolar (saber escolar), tivessem participação efetiva na
construção do material, o que lhes conferiria o papel de igualmente
responsáveis pelo processo de elaboração e condução do material e das
práticas didáticas a ele relacionadas. O quarto e último referia-se ao
desafio de se construir um curso virtual assíncrono para o ensino de
francês à distância que buscasse garantir a interatividade no processo
de ensino e aprendizagem do idioma, e que mesmo mediante a ausência
física do professor, propusesse um material didático organizado a partir
de textos autênticos na LE, de método indutivo, progressivo no
desenvolvimento dos conhecimentos e das capacidades de linguagem e,
principalmente, em permanente diálogo com os alunos participantes por
meio da plataforma Moodle.

Dessa maneira, para relatar nossa experiência, este texto organiza-se em
cinco partes, além desta introdução: primeiramente, faremos uma
apresentação da plataforma Moodle e de suas possíveis funcionalidades
destinadas ao ensino de línguas; em seguida, descreveremos os
pressupostos teórico-metodológicos na perspectiva sociointeracionista
que orientaram a elaboração do curso em questão; posteriormente,
elaboraremos algumas considerações sobre o contexto de produção do curso
para, então, descrevermos seu processo de elaboração, suas
características essenciais, sua organização estrutural, sua metodologia
de ensino e sua relação com o desenvolvimento das capacidades de
linguagem; por fim, seguem as considerações finais.