\section{A plataforma Moodle como ambiente virtual para o ensino de línguas}\label{sec-aplataformamoodlecomoambientevirtualparaoensinodelínguas}

Nas duas últimas décadas, percebemos um grande aumento da oferta de
cursos para o ensino de idiomas oferecidos em ambientes virtuais, seja
na modalidade síncrona, seja na modalidade assíncrona. Nesses cursos, os
processos de interação entre o professor/aluno e entre o aluno/conteúdo
proposto para o aprendizado de determinado idioma acontecem por meio de
ambientes virtuais de aprendizagem (AVA), que assumem um papel central
no que diz respeito à participação e ao aproveitamento do curso pelos
alunos e no processo de ensino-aprendizagem efetuado pelos professores.
De acordo com \textcite{lima_https://proceedings.science/isa-2017/trabalhos/metodos-mais-usados-para-avaliacoes--ambientes-virtuais--aprendizagem-avas_2018}, num AVA, o aluno inscrito em um curso tem acesso a múltiplos recursos de interação, ou seja, suportes necessários para o aprendizado à distância, como, por exemplo, chats, fóruns, aulas interativas, questionários, glossários, vídeos, entre outros, o que faz do professor um mediador do conhecimento entre o aluno inscrito no curso e os conteúdos ministrados. Além de propiciarem aos alunos uma forma mais ativa de lidar com seu aprendizado, permitindo uma melhor organização e a oportunidade do autoestudo, um AVA funciona, nos cursos síncronos e assíncronos, como a verdadeira sala de aula, porém, virtualizada. Para \textcite{bersch_moodle_2009}, os AVA buscam facilitar esse processo ao reunirem um conjunto de ferramentas que permite disponibilizar materiais diversificados, propor, realizar e avaliar diferentes tipos de atividades.

No caso do curso que descreveremos, o AVA escolhido para hospedar os
cursos de idiomas foi o Moodle (\emph{Modular Object-Oriented Dynamic
	Learning Environment}), desenvolvido por Martin Dougiamas. Trata-se de
uma ferramenta livre e de código aberto, muito utilizada pelas
instituições de ensino públicas e particulares, que foi desenvolvida
para propiciar a professores e alunos um sistema seguro e totalmente
integrado, que tem por objetivo criar ambientes de aprendizagem
personalizados, como explicam as palavras de \textcite[p. 31]{gervasio_nvestigacao_2019}:
\enquote{[$\ldots$] é inegável que o Moodle se tornou uma poderosa ferramenta
tanto para o ensino EAD como no presencial, sua variedade de plugins e
flexibilidade de customização o torna utilizável em qualquer curso e
modalidade de ensino}. Ainda, segundo o pesquisador, \enquote{a combinação de
metodologias de ensino, agrupadas pelas tecnologias disponíveis no
ambiente, é fundamental para o design de cursos direcionados
especificamente a área de atuação}. \cite[p. 31]{gervasio_nvestigacao_2019}.

De acordo com \textcite{thomaz_principios_2022}, a plataforma Moodle foi desenvolvida a partir de quatro princípios, o Construtivismo, o Construtivismo Social, o Construcionismo e o Comportamento Conectado e Separado, visto que a
interação social (a ação), por meio da linguagem, permite o
estabelecimento de uma corrente colaborativa em permanente comunicação,
síncrona e assíncrona, colhendo frutos para um aprendizado de cunho
social, histórico e cultural, que se define por novas formas autônomas
de aprender crítica e dinamicamente, construir ideias, comunicar e se
informar. Embora dependa de uma prática pedagógica igualmente
interacionista, a plataforma Moodle busca valorizar o processo de
aprendizagem ao disponibilizar um rol de ferramentas interativas de
diferentes qualidades que oferecem ao usuário a possibilidade de
construção de significados ao efetivarem as variadas propostas de
interação, negociação e construção colaborativa de conhecimento
disponíveis. Na \Cref{tab-01}, reproduzimos uma lista de ferramentas encontradas no AVA Moodle, elaborada por \textcite{mayrink_ensino_2017}, com
base em estudos de \textcite{valente_educacao_2011}, que ilustra a forma como vários desses recursos podem ser agrupados e utilizados para o desenvolvimento
de atividades didáticas em cursos de idiomas.

\begin{table}[h!]
\centering \small
\begin{threeparttable}
\caption{Agrupamento de ferramentas da Plataforma Moodle.}
\label{tab-01}
\begin{tabular}{>{\raggedright\arraybackslash}p{3cm} >{\raggedright\arraybackslash}p{3cm} >{\raggedright\arraybackslash}p{6cm}}
\toprule
Potencial de interatividade & Recurso & Tipo de atividade \\
\midrule
Baixo potencial de interatividade (aluno-máquina- conteúdo) &
\begin{minipage}[t]{\linewidth}
\begin{itemize}[leftmargin=*,topsep=-0pt,partopsep=0pt,parsep=0pt,itemsep=0pt]
    \item Questionário
    \item Escolha
    \item Pergunta
\end{itemize} 
\end{minipage}
&
\begin{minipage}[t]{\linewidth}
\begin{itemize}[leftmargin=*,topsep=-0pt,partopsep=0pt,parsep=0pt,itemsep=0pt]
    \item Completar frases, diálogos, canções, textos (com espaço ou close). 
    \item Ordenar um diálogo ou texto.
    \item Organizar o vocabulário em diferentes campos semânticos.
    \item Relacionar ou preencher colunas.
    \item Relacionar imagens a palavras/textos.
    \item Escolher a alternativa correta.
\end{itemize} 
\end{minipage}
\\
Potencial de interatividade médio (aluno-máquina-conteúdo-professor) &
\begin{minipage}[t]{\linewidth}
\begin{itemize}[leftmargin=*,topsep=-0pt,partopsep=0pt,parsep=0pt,itemsep=0pt]
    \item Tarefa
\end{itemize} 
\end{minipage}
&
\begin{minipage}[t]{\linewidth}
\begin{itemize}[leftmargin=*,topsep=-0pt,partopsep=0pt,parsep=0pt,itemsep=0pt]
    \item Mudar o final de um texto.
    \item Ler um texto narrativo e escrever o diálogo que o originou.
    \item Enviar textos gravados em vídeo ou áudio.
\end{itemize} 
\end{minipage}
\\
Alto potencial de Interatividade (aluno- aluno- professor- máquina-conteúdo) &
\begin{minipage}[t]{\linewidth}
\begin{itemize}[leftmargin=*,topsep=-0pt,partopsep=0pt,parsep=0pt,itemsep=0pt]
    \item Fórum
    \item Chat
    \item Diálogo
    \item Wiki
    \item Glossário
\end{itemize} 
\end{minipage}
&
\begin{minipage}[t]{\linewidth}
\begin{itemize}[leftmargin=*,topsep=-0pt,partopsep=0pt,parsep=0pt,itemsep=0pt]
    \item Realizar debates e discussões.
    \item Trabalhos em grupo (construção de texto coletivo, síntese de pesquisas, entre outros)
    \item Construção de glossários, linkotecas colaborativas.
\end{itemize}  
\end{minipage}
\\
\bottomrule
\end{tabular}
\source{\cite{mayrink_ensino_2017}}
\end{threeparttable}
\end{table}

Durante a formação pedagógica oferecida para a elaboração do curso,
sempre insistimos no fato de que o aprendizado de uma LE, seja na
modalidade presencial, seja na modalidade remota, deve se concentrar em
quatro eixos: a leitura (compreensão escrita); a escuta (compreensão
oral); a fala (produção oral); a escrita (produção escrita). A interação
entre esses eixos deve ser contínua durante a aprendizagem e, para isso,
buscamos adaptar o material elaborado a partir do que a plataforma
Moodle oferecia como possibilidades para o trabalho com os quatro eixos
elencados, de forma que nem todas as ferramentas propostas e elencadas
no quadro anterior foram efetivamente colocadas em prática nas
atividades formuladas. Como o curso seria oferecido na modalidade
assíncrona, buscamos nos concentrar nas ferramentas que, de fato,
colaborariam para o efetivo aprendizado do francês à distância,
oferecendo o máximo possível de atividades interativas e colaborativas
que pudessem ser geridas e realizadas de acordo com o ritmo, as
possibilidades e o tempo dos alunos, igualmente previstas em um percurso
pedagógico de logística facilitada, que tentava garantir o protagonismo
e a autonomia absolutos em um ambiente de imersão em francês como LE.

Chegamos ao final da formação com a impressão de que a plataforma Moodle
poderia ser uma ferramenta bastante proveitosa para o ensino de LE,
desde que dentro de uma proposta pedagógica padronizada, organizada e
customizada, com atividades e objetivos pedagógicos que contribuíssem
para um aprendizado dinâmico e significativo.
