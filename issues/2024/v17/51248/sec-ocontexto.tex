\section{O contexto de produção do curso}\label{sec-ocontexto}

O curso descrito fora elaborado no quadro de um projeto de cursos à
distância assíncronos de línguas clássicas e modernas, colocado em
prática pelo centro de extensão, que busca atender tanto a comunidade
interna quanto a comunidade externa da referida universidade,
estendendo-se, também, àqueles que residem fora da cidade de XXX, no
estado de XXX. Nesse sentido, o curso foi planejado tanto para um
público de jovens universitários brasileiros, adequado à sua faixa
etária e à sua disponibilidade para a aprendizagem, com vistas,
inclusive, à sua participação em programas de mobilidade acadêmica em
países francófonos oferecidos pela universidade, quanto para adultos que
se interessassem pelo aprendizado do francês para viagens, negócios,
oportunidades de trabalho ou de estudo em países francófonos, ou ainda
que gostariam de ampliar seu repertório cultural por meio do aprendizado
de uma LE. Procuramos propor a um público jovem e adulto um material
didático que não fosse demasiadamente infantil ou desinteressante,
destinado a mostrar a importância do francês como língua internacional,
falada por mais de 320 milhões de pessoas nos 5 continentes\footnote{Dados
	extraídos de:
	\url{https://www.francophonie.org/la-langue-francaise-rayonne-avec-321-millions-de-locuteurs-dans-le-monde-2140}},
buscando apresentar, sempre que possível, uma reflexão acurada em torno
do funcionamento da língua e das representações culturais não só dos
franceses, mas também de outros povos francófonos.

Além da formação pedagógica oferecida pelo professor-formador aos
estagiários, concebida para acompanhar, em termos de conteúdos, os
cursos de mestrado em FLE oferecidos em países francófonos, uma outra
formação foi oferecida ao professor-formador e aos estagiários pelo
Centro de Apoio à Educação à Distância (CAED) da universidade. Tal
formação previa um curso virtual de 60 horas, intitulado \enquote{Laboratório
de Criação de Materiais Didáticos para EaD}, durante o qual foram
preparadas as unidades de aprendizagem do guia didático, o
desenvolvimento da identidade visual do curso e a montagem do conteúdo
proposto para o ensino do francês no AVA Moodle. Ao término desta etapa,
a equipe de elaboração havia construído o desenho didático do curso e se
preparava para a construção da primeira unidade do curso.

No que diz respeito à sua organização prática, o curso \emph{Français
	Online 1} foi pensado para alunos iniciantes em francês, nível A1 do
Quadro Europeu Comum de Referência para as Línguas (Conselho da Europa,
2001). Sua carga horária total era de cinquenta e quatro horas,
divididas entre as três unidades apresentadas (dezoito horas para cada
unidade). Como o curso era de natureza assíncrona, as avaliações
relativas aos eixos da compreensão (oral e escrita) foram realizadas na
própria plataforma Moodle por meio da ferramenta \enquote{Questionário}. As
avaliações destinadas à produção escrita foram efetuadas por meio da
ferramenta \enquote{Tarefa}. Já as avaliações de produção oral e outros
exercícios de produção oral foram efetuados pelos alunos por gravações
de áudio que deveriam ser igualmente enviadas pela plataforma Moodle por
meio da ferramenta \enquote{Arquivo}. Seria considerado aprovado o aluno que
obtivesse o mínimo de setenta pontos na escala de avaliações, de acordo
com as normas avaliativas do centro de extensão.

Uma vez elaborado e inserido na plataforma Moodle, o curso foi oferecido
sob a forma de um projeto-piloto liderado pelos estagiários, com a
inscrição de 20 alunos. Durante a aplicação do curso piloto, o
professor-formador e os estagiários (efetivamente os professores das
aulas) elaboraram fichas de avaliação para serem preenchidas ao longo do
curso, buscando mapear, no material, eventuais problemas nas atividades
de interação, de construção de conhecimento, ou até mesmo dificuldades
técnicas em relação à plataforma Moodle. Após os ajustes efetuados e as
incorreções resolvidas, o curso foi liberado ao centro de extensão para
futuras ofertas.