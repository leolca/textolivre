\begin{polyabstract}
\begin{abstract}
Neste trabalho, analisamos uma vivência educacional que combinou a técnica
\textit{Photovoice} e a Pedagogia Psicodramática, aliadas ao trabalho com
linguagens na perspectiva dos multiletramentos, para a formação crítica de
adolescentes. A partir de um projeto interdisciplinar, desenvolvido por meio de TDICs, buscou-se ampliar e integrar a visão dos estudantes acerca do cuidado tridimensional: eu-outro-meio ambiente. O projeto foi desenvolvido com alunos de 1º e 2º anos da educação profissional técnica de nível médio, no interior do Mato Grosso do Sul, Brasil, em 2021, durante a pandemia da Covid-19, momento em que se evidenciou ainda mais a necessidade de olhar para o “eu”, para as relações “eu-outro” e para as relações “nós-meio ambiente”. A partir dos registros fotográficos com o \textit{Photovoice}, os participantes produziram textos multissemióticos utilizando a mixagem de linguagens e mídias digitais. Para este trabalho, selecionamos as etapas do projeto de ensino, o qual foi organizado em quatro fases: aquecimento inespecífico, aquecimento específico, dramatização e compartilhamento. Para efeito de análise, utilizamos a proposta dos multiletramentos e os quatro processos do conhecimento: experienciar o novo; conceitualizar com teoria; aplicar criativamente; e analisar criticamente. As análises permitiram observar o desenvolvimento de habilidades críticas acerca de questões socioambientais que impactam o cotidiano dos participantes.
		
\keywords{Photovoice \sep Multiletramentos \sep Pandemia}
\end{abstract}
	
\begin{english}
\begin{abstract}
In this paper, we analyzed an educational experience that combined the
Photovoice technique and Psychodramatic Pedagogy, combined with work with
languages from the perspective of multiliteracies, for the critical training of adolescents. Through an interdisciplinary project developed via ICTs, we aimed at broadening and integrating students' vision of three-dimensional care: self-other-environment. We developed the project with students from the 1st and 2nd grades of Technical High School Education in the countryside of Mato Grosso do Sul, Brazil, in 2021, during the Covid-19 pandemic, a moment when the need to look at myself, myself-yourself, and ourself-environment relationships became even more evident. Through Photovoice records, the participants produced multisemiotic texts using a mix of languages and digital media. For this article, we selected the stages of the teaching project, which was organized into four phases: non-specific warm-up, specific warm-up, dramatization and sharing. For analysis purposes, we used the multiliteracies proposal and the four knowledge processes: experiencing the new; conceptualize with theory; apply creatively; and critically analyze. The analyzes allowed us to observe the development of critical skills regarding socio-environmental issues that impact the daily lives of the participants.
		
\keywords{Photovoice \sep Multiliteracies \sep Pandemic}
\end{abstract}
\end{english}
% if there is another abstract, insert it here using the same scheme
\end{polyabstract}
