\section{Ciberespaço, cibercultura e multiletramentos: de leitor a navegador, de 
	usuário a designer}\label{sec-ciberespaço,cibercultura}
	
	Com a popularização da internet e das mídias digitais no final do século XX, entram em cena e em debate novas possibilidades de interações sociais por meio da linguagem, ou melhor, das linguagens. As mudanças são percebidas nas diversas esferas sociais humanas de maneira global; emissores e receptores já não precisam ocupar o mesmo espaço físico para um diálogo em tempo real. Rompem-se barreiras geográficas, pessoas de várias partes do mundo podem interagir, trocar ideias, experiências, compartilhar coisas interessantes, denunciar injustiças, dentre tantas outras coisas. O conhecimento já não ocupa espaço físico específico, mas está em todo e em nenhum lugar ao mesmo tempo, configurando, assim, a \textit{World Wide Web}, a grande rede mundial de computadores.
	
	Esse espaço de conexão global e simultânea é denominado por \textcite[p. 95]{levy_cibercultura_2000} de ciberespaço, “o espaço de comunicação aberta pela interligação mundial de computadores e das memórias informáticas”, que possibilita a configuração da sociedade em rede \cite{castells_sociedade_2003} e introduz a cultura digital ou cibercultura, a qual, conforme \textcite[p. 17]{levy_cibercultura_2000}, é o {\textquotedbl}conjunto de técnicas (materiais e intelectuais), de práticas, de atividades, de modos de pensamento e de valores que se desenvolvem juntamente com o crescimento do ciberespaço{\textquotedbl}.
	
	Para \textcite[p. 19]{lima_hipertexto:_2016}, “três princípios básicos orientam o ciberespaço e a cibercultura: a interconexão generalizada, a criação de comunidades virtuais e a inteligência coletiva”, dando destaque para este último, uma vez que na era digital as pessoas podem dar e receber, construir e pensar juntos, e, assim, vão produzindo na troca e na colaboração os “coletivos inteligentes”.
	
	Sendo assim, na era digital, a \textbf{participação} ocupa posição central; além disso, vemos que há uma mudança significativa acerca da noção de \textbf{autoria}, pois desloca-se o foco do “meu” para o “nosso”. Nesse contexto, faz-se importante refletirmos quanto à mudança de três conceitos muito caros à educação, em especial, ao ensino de línguas: leitor, autor e texto.
	
	Antes do advento da internet, a noção de leitor referia-se àquele(a) que tinha acesso a materiais impressos (como livros, jornais e revistas), que realizava a leitura linearmente, seguindo a ordem estabelecida pelo autor da obra, na sequência de cima para baixo, da esquerda para direita, linha após linha. Era aquele(a) que frequentava bibliotecas (não que isso ainda não aconteça), que carregava livros de um lado para o outro.
	
	Com o surgimento e evolução das mídias digitais, ampliam-se as possibilidades para leitor e autor, pois uma gama de informações está a seu dispor mediante o mais sutil clique de um mouse; rompe-se com a linearidade textual, o leitor já não precisa seguir o que o autor preestabeleceu, mas vai construindo, guiado pelo seu interesse/objetivo, um roteiro próprio de leitura. Aqui, o leitor é um navegador que salta de um ponto a outro da “grande teia”, sendo muito mais participativo, pois comenta, curte, compartilha, escreve junto, interage com outros leitores/autores, os chamados “lautores” \cite[p. 20]{rojo_escola_2013}, que juntos formam as comunidades virtuais que se conectam em torno de interesses comuns, constituindo a inteligência coletiva.
	
	Como vimos, os “lautores” estão a todo tempo consumindo e produzindo conhecimento. Nesse sentido, dissipa-se o poder do autor que, agora, sugere caminhos de produção de sentido, que podem ou não ser concretizados. Assim, os textos são construídos a várias mãos e cérebros. Dessa maneira, aqueles que navegam pelo labirinto do ciberespaço fazem contato com uma multiplicidade de referências midiáticas, culturais e linguísticas, o que requer mais do que as habilidades tradicionais.
	
	Para tanto, é necessário que a escola contemporânea priorize a formação de estudantes que saibam ler, ressignificar, construir e distribuir sentidos apoiando-se nas diversas formas linguageiras disponíveis, assumindo o protagonismo e tornando-se \textit{designers,} conforme \textcite{cope_letramentos_2020}, sendo aqueles que vão construindo de forma ativa, crítica e criativa os sentidos a partir dos recursos disponíveis e, nesse processo, vão conscientizando-se acerca de suas situações cotidianas, desenvolvendo, assim, condições para desenhar/transformar suas realidades.
	
	A partir do que refletimos sobre o perfil de leitor e autor dos espaços digitais, percebemos que se amplia significativamente a noção de texto, pois este é muito mais dinâmico, interativo, multissequencial. Já não há o predomínio da linguagem verbal, mas abrem-se infinitas possibilidades de mixagem e/ou remixagem de linguagens no processo de construção de sentidos, concretizadas por meio de uma multiplicidade de mídias digitais. Como argumenta \textcite[p. 51]{lima_hipertexto:_2016}, com as tecnologias digitais é possível apreciar:
	
	\begin{quote}
	[$\ldots$] não só as versões finais [dos textos], mas as versões penúltimas, antepenúltimas, os rascunhos, os esboços, as anotações escritas em papel \newline
	[$\ldots$], o processo de criação, o pensamento em formação, a forma ideal em busca de si mesma. E mais: não só os textos originais, mas os comentários, as críticas, as interpretações dos leitores; não só textos estanques, mas textos relacionados (‘linkados’)com outros textos, por sua vez relacionados com outros, numa cadeia de elos sem fim.
	\end{quote}
	Além disso, o texto característico dos espaços digitais integra uma multiplicidade de linguagens, não prioriza a linguagem verbal, mas soma-se a essas imagens estáticas e em movimento, sons, gráficos, infográficos etc. Na contemporaneidade, uma imagem pode dizer mais do que mil palavras, a depender do objetivo dos “lautores”. Sendo assim, vemos uma reconfiguração e ampliação dos perfis de leitores e autores, bem como da noção de texto, o que exige novas capacidades daqueles que navegam pelo ciberespaço.
	
	Trazendo essa reflexão para o contexto do ensino de línguas, o que cabe a nós (professores, pesquisadores, pensadores) diante dessa realidade complexa e em constante mutação? Quais as reverberações das mídias digitais para as práticas de ensino e aprendizagem de línguas? Quais desafios são impostos pela realidade digital em constante mutação para os que lidam com ensino de línguas? O trabalho aqui apresentado é parte do esforço que temos desenvolvido para ir ao encontro dessa tarefa.
	
	Diante dessas questões, acreditamos que a pedagogia dos multiletramentos pode potencializar o ensino de línguas em sala de aula. Primeiro, porque traz imbricada em sua proposta um conceito de língua que não é estática, mas sim dinâmica e em constante transformação, o que, a nosso ver, desafia o professor a pensar o ensino de línguas a partir de práticas situadas em um contexto que é social, histórico e cultural.
	
	Em segundo lugar, na atualidade, nós não utilizamos apenas a linguagem verbal para as interações nas diversas esferas sociais, pelo contrário, cada vez mais mixamos/remixamos as linguagens nos processos de produção de sentidos, principalmente nos espaços digitais. Sendo assim, focalizar a multiplicidade de linguagens associadas à diversidade de mídias (digitais ou não) exploradas dentro de uma realidade cultural é de suma importância para o processo de ensino aprendizagem de línguas, configurando, dessa forma, propostas pedagógicas mais significativas, conectadas à realidade estudantil.
	
	Terceiro, vemos diariamente que as interações estão cada vez mais complexas (no sentido de muitas possibilidades); então, ensinar línguas requer a mixagem não apenas de mídias e linguagens, mas também de áreas do conhecimento. A atuação em parceria entre os campos do saber propicia um aprendizado mais completo e significativo, pois já não se sustenta a tradicional divisão do conhecimento “em caixinhas” (disciplinas), assim como é muito mais promissor o trabalho colaborativo entre os alunos e professores, por meio de projetos que trabalhem assuntos levantados pelos grupos, a partir de suas realidades.
	
	Por fim, destacamos que termos como ciberespaço, cibercultura, texto digital (hipertexto), “lautores”, autoria participativa, coletivos inteligentes, comunidades virtuais e de afinidades, \textit{designers}, multiletramentos e tantos outros relacionados ao contexto interacional da contemporaneidade já vêm sendo discutidos há muito tempo, contudo, a nosso ver, ganham maior destaque e visibilidade na educação, em especial na educação básica, durante a pandemia, momento em que, como vimos, necessitou de uma ruptura brusca com modelos mais tradicionais de ensino.