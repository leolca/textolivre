\section{A pandemia, um projeto de ensino e os multiletramentos: um diálogo possível}\label{sec-apandemia,umprojetode}
No início de 2020, a instituição em que o projeto de ensino relativo a esta pesquisa foi desenvolvido, assim como as instituições educacionais em geral, precisou buscar alternativas para continuar com as suas atividades, em virtude da necessidade de isolamento físico imposto pela pandemia da Covid-19.

De fato, as instituições educacionais brasileiras, de todos os níveis de ensino, tiveram que reorganizar o calendário escolar, adotando, em muitos casos, o Ensino Remoto Emergencial (ERE)\footnote{ É uma mudança temporária para um modo de ensino alternativo, devido a circunstâncias de crise. Envolve o uso de soluções de ensino totalmente remotas para instrução ou educação que, de outra forma, seriam ministradas presencialmente ou como cursos combinados ou híbridos e que retornam a esse formato assim que a crise ou emergência diminui. O objetivo principal nessas circunstâncias não é recriar um ecossistema educacional robusto, mas, sim, fornecer um acesso temporário à instrução e suporte educacional de uma maneira que seja rápida de configurar e esteja disponível de forma confiável durante uma emergência ou crise \cite{hodges_difference_2020}}, utilizando as Tecnologias Digitais da Informação e Comunicação (TDICs) como suporte para que fosse possível a garantia e o cumprimento das oitocentas horas para o Ensino Médio, conforme prevê a Lei 9.394 – Lei de Diretrizes e Bases da Educação \cite{brasil_lei_1996}.

No caso do contexto educacional em questão, um \textit{campus} do Instituto Federal de Ensino Básico, Técnico e Tecnológico, no interior do estado de Mato Grosso do Sul (MS), para implementar as atividades remotas, os professores passaram a desenvolver “Atividades Não Presenciais” (ANPs), em caráter de excepcionalidade. Para isso, utilizaram o Moodle como plataforma oficial, bem como o \textit{e-mail} institucional para as atividades assíncronas, além de outras ferramentas de apoio, tais como: o WhatsApp, os recursos do GSuite, a Zoom Cloud Meetings para as atividades síncronas ou de interação rápida com os estudantes.

Na ocasião, vivíamos uma situação de medo, insegurança, aflição, desespero e dor. Nós, professores, nos vimos diante da demanda por trabalhar o que estava posto nos currículos e, ao mesmo tempo, acolher os estudantes com as suas necessidades e particularidades (alunos com infraestrutura tecnológica digital escassa ou precária, alunos e professores com problemas emocionais advindos da situação e da perda de cuidadores/familiares, entre outras situações).

Diante da nova e caótica realidade e almejando atender adequadamente aos estudantes da instituição, matriculados nos 1º e 2º anos do Ensino Médio Integrado ao Curso Técnico em Informática, durante o ano de 2021, a professora de língua portuguesa, em parceria com colegas das áreas de enfermagem, psicologia, sociologia, filosofia, biologia, geografia e tecnologia (análise de sistemas), elaborou e coordenou um projeto de ensino interdisciplinar intitulado “Eu, tu e o nosso ambiente: construindo uma consciência ecológica coletiva”\footnote{ Projeto contemplado em 1º lugar pelo edital 022/2021-PROEN/IFMS (\url{https://selecao.ifms.edu.br/edital/files/edital-no-022-2021-ifms-proen-projetos-de-ensino-com-fomento-edital-no-022-3-2021-ifms-proen-resultado-final.pdf}). Esse projeto também é objeto de uma tese de doutorado em andamento, sob a responsabilidade de uma das autoras deste artigo.}.

O objetivo do projeto foi discutir o cuidado tridimensional: eu-outro-meio ambiente, num momento em que, a nosso ver, havia intensa necessidade de olhar para o “eu”, para as relações “eu-outro”, bem como para as relações “nós-meio ambiente”, já que a própria situação de pandemia nos mostrou que as ações humanas têm ocasionado fortes impactos na natureza, trazendo consequências drásticas de nível global.

O projeto atendeu aproximadamente setenta adolescentes, com idades entre 14 e 17 anos, residentes no interior do MS, bastante acostumados com as rodinhas de tereré\footnote{ O Tereré é uma bebida feita de infusão da erva-mate, com origem guarani, podendo ser consumida com água, limão, entre outras coisas. A bebida possui um gosto um pouco amargo, porém costuma ser muito apreciada no Mato Grosso do Sul \cite{mato_grosso_do_sul_por_favor_historia_2020}.}, os passeios de bicicleta, as partidas de futebol e tantas outras interações face a face bastante comuns para a faixa etária e as realidades interioranas do estado. Para melhor engajamento dos estudantes envolvidos, propusemos a formação de sete equipes, organizadas por área do conhecimento (língua portuguesa, biologia, geografia, sociologia, filosofia, enfermagem e psicologia); os participantes puderam escolher via formulário eletrônico a área que tinham mais afinidade para discutir questões relacionadas ao cuidado do “eu”, do “outro” e do “meio ambiente”.

A proposta foi realizada entre abril e dezembro de 2021, sendo estruturada a partir de temáticas norteadoras. A primeira parte, compreendendo os meses de abril, maio e junho, foi reservada ao que nomeamos "aquecimento" dos participantes quanto à necessidade, a partir do contexto da pandemia, do cuidado do “eu”, do “outro” e do “meio ambiente”. Nesta etapa, trabalhamos a “valorização do eu, respeito ao outro”. Na segunda parte, de agosto a dezembro, trabalhamos a relação eu-outro e nós-meio ambiente, com as temáticas “quem determina o meu bem-estar”, “fazendo a diferença” e “teia ecossistêmica”.

Ainda na fase de aquecimento, realizamos com todos os participantes uma oficina virtual, que teve a duração de aproximadamente duas horas, realizada via plataforma Zoom. Na ocasião, trabalhamos a leitura coletiva de fotografias, o direito de imagem, o direito autoral, dicas de registro de fotografias, o aplicativo Comica e as características da técnica \textit{photovoice}. Esses conhecimentos foram utilizados, além do que foi trabalhado em atividades síncronas e assíncronas durante a primeira parte da proposta, para as produções multissemióticas feitas pelos alunos participantes ao final do projeto. Na oficina, usamos os seguintes recursos didáticos: vídeos, fotografias, \textit{smartphones}, computadores, aplicativo Comica\footnote{App gratuito e de fácil manuseio.} e Zoom Meeting.

O propósito da oficina foi instruir os alunos para que levantassem, por meio de fotografias, problemáticas socioambientais presentes em suas realidades cotidianas, durante aproximadamente um mês, norteados pela seguinte pergunta: \textit{Quais problemáticas socioambientais presentes em vossas realidades afetam o eu, o outro e o meio ambiente?} Os registros fotográficos feitos pelos estudantes serviram de base para as produções textuais multissemióticas (vídeos e histórias em quadrinhos digitais colaborativas).

Além disso, as fotografias foram sendo armazenadas pelos estudantes em um mural fotográfico, criado pela equipe de professores por meio da ferramenta Padlet. O mural era aberto a todos os participantes, que foram incentivados à utilização de fotografias uns dos outros, respeitando o direito autoral e dando o devido crédito ao(s) autor(es). O processo de registro fotográfico foi uma ação que fortemente contribuiu para despertar a atenção e sensibilizar os estudantes acerca das questões/problemas inerentes às suas realidades.

Sendo assim, o objetivo deste estudo é analisar como se dá o processo de produção de linguagens nessa atividade, que combina a técnica do \textit{Photovoice} e a Pedagogia Psicodramática com a proposta dos multiletramentos para a formação crítica de adolescentes. Para efeito de análise, utilizamos os quatro processos de conhecimento propostos por \textcite{cope_letramentos_2020}, a saber: experienciar o novo; conceitualizar com teoria; aplicar criativamente; e analisar criticamente.

