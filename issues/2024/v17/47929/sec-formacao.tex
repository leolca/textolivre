\section{Formação crítica: criatividade, colaboração e agência}\label{sec-formaçãocrítica-criatividade}

Quando pensamos no uso de tecnologias digitais na educação, caminhamos por uma linha tênue do neoliberalismo, na qual, é desafiador saber onde termina o educacional e começa o mercadológico \cite{ball_nova_2014}, pois a lógica neoliberal é que as duas instâncias caminham juntas. Como propõe \textcite{buzato_inteligencia_2023}, o propósito é uma análise crítica das relações entre o humano e o digital para que não caiamos nos jogos das relações mercadológicas de tecnologias digitais. Destacamos, ainda, que as mídias digitais por si só não são capazes de promover uma formação crítica e criativa, mas sim as ações, a criatividade, a sensibilidade dos professores e alunos envolvidos no processo que fazem a diferença.

Foi nesse sentido que a prática relatada no presente artigo buscou um esforço de atuação dentro de um novo \textit{ethos} \cite{knobel_new_2007}, na tentativa de contribuir para a formação de um sujeito crítico, participativo e criativo, dentro de práticas situadas, interdisciplinares e colaborativas conectadas à realidade a que pertence: digital e pandêmica.

A formação desse sujeito crítico a qual nos referimos incide sobretudo sobre  a agência do aluno, ligada à capacidade de avaliar e agir frente ao que foge do esperado \cite[p. 1]{monte_mor_development_2013}, a partir do desenvolvimento de consciência crítica social. É nesse sentido que buscamos uma prática situada ao contexto dos alunos e dos professores frente ao ciberespaço e a sociedade em rede na pandemia da Covid-19; conceitualizar com teoria a partir de pesquisa e trabalho teórico do projeto; analisar criticamente a realidade vivenciada a partir das atividades listadas ao longo da nossa discussão; além da prática transformada, aplicada criativamente por meio de vídeos e HQs divulgados à comunidade.

Pensando nas atividades desenvolvidas, é pertinente destacar a relação de um novo \textit{ethos} \cite{knobel_new_2007} ao prefixo “multi” de multiletramentos. \textcite{knobel_new_2007} consideram que as novas tecnologias proporcionam uma nova ordem social e não simplesmente atualizam as ferramentas já utilizadas. Tal posicionamento também serve aos multiletramentos que não tratam somente da construção de sentido por meio de tecnologias digitais, mas pela multiplicidade semiótica. Por isso, \textcite{duboc_delinking_2021} sugerem o {\textquotedbl}desprendimento{\textquotedbl}\footnote{ No original: \textit{delinking}.} dos multiletramentos à ligação fixa ao digital.

Essa discussão é pertinente, pois é por meio dos registros fotográficos que os alunos são sensibilizados e se deparam com a amplitude da problemática trabalhada no projeto, ou seja, não se trata somente de uma reconfiguração de aprendizagem a partir de tecnologias digitais, mas, sim, como o \textit{ethos} do aluno foi redesenhado para que a construção de sentido acontecesse, o que se deu a partir de um contexto amplo de caracteres do ciberespaço e da sociedade em rede.


	Na prática relatada e discutida neste trabalho, vemos um aluno sensibilizado pelos problemas sociais da sua cidade, do seu entorno, em um período pandêmico e de aulas remotas. Os próprios alunos destacam no projeto as relações entre as atividades propostas e questões sociais – por vezes do seu entorno –, não o uso de tecnologias digitais por si só.


	A nosso ver, o diferencial não está na tecnologia digital por si só, mas nas relações estabelecidas entre os sujeitos e as ferramentas digitais, dentro de um contexto amplo de multiletramentos, no qual ações colaborativas e criativas podem resultar em práticas significativas para o bem-estar da coletividade.


	Vemos também a agência do professor no projeto, ao considerar o contexto dos alunos, a diversidade das formas de aprender por meio dos diversos recursos utilizados, a mediação por questionamentos e a divulgação e reconhecimento dos trabalhos finais para a comunidade.

No caminho do projeto (experienciar o novo; conceitualizar com teoria; aplicar criativamente; e analisar criticamente.), conseguimos olhar para as relações entre humano/digital, múltiplas e hibridizadas, de forma crítica e por meio da agência do professor e do aluno, significativas às agendas não neoliberais de educação e tecnologias digitais.