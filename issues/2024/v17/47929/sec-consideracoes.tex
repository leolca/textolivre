\section{Considerações finais}\label{sec-consideraçõesfinais}

A partir da análise da vivência educacional descrita, visualizamos as interações propiciadas pelo novo \textit{ethos}, advindo do ciberespaço e da cultura digital, em especial, em contexto pandêmico, no qual as múltiplas semioses foram articuladas e propiciaram o exercício de consciência e agência dos alunos e dos professores.

Os participantes puderam olhar criticamente para as relações entre o eu, o outro e o meio ambiente em suas comunidades locais, problematizando questões socioambientais durante a pesquisa e o registro fotográfico, além de agir socialmente mediante a produção de vídeos e HQs que foram divulgados à comunidade em geral em eventos institucionais, bem como na internet.

Além disso, a técnica \textit{photovoice} aliada aos multiletramentos e à pedagogia psicodramática potencializou a mixagem de gêneros, mídias e linguagens, cara ao novo \textit{ethos} proveniente da cultura digital, na qual fontes, autorias e gêneros se articulam de forma rizomática.

Reconhecidos os limites deste texto e de todo trabalho educacional – que se estabelece na tênue linha educação-mercado –, esperamos que ele possa ajudar a pensar em alguns caminhos para o trabalho crítico com tecnologias digitais em sala de aula, de forma colaborativa e interdisciplinar e com foco em questões de linguagem.
