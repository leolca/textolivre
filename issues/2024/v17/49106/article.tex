\documentclass[spanish]{textolivre}

% metadata
\journalname{Texto Livre}
\thevolume{17}
%\thenumber{1} % old template
\theyear{2024}
\receiveddate{\DTMdisplaydate{2023}{12}{7}{-1}}
\accepteddate{\DTMdisplaydate{2024}{2}{3}{-1}}
\publisheddate{\DTMdisplaydate{2024}{4}{8}{-1}}
\corrauthor{Juan Fernando Gómez Paniagua}
\articledoi{10.1590/1983-3652.2024.49106}
%\articleid{NNNN} % if the article ID is not the last 5 numbers of its DOI, provide it using \articleid{} commmand 
% list of available sesscions in the journal: articles, dossier, reports, essays, reviews, interviews, editorial
\articlesessionname{articles}
\runningauthor{Gómez Paniagua et al.}
%\editorname{Leonardo Araújo} % old template
\sectioneditorname{Daniervelin Pereira}
\layouteditorname{João Mesquita}

\title{Autoeficacia y actitud para el uso de herramientas web 2.0 en el aprendizaje del inglés en línea: diferencias por sexo}
\othertitle{Auto-eficácia e atitude em relação à utilização de ferramentas Web 2.0 para a aprendizagem de inglês online: diferenças de género}
\othertitle{Self-efficacy and attitude towards the use of web 2.0 tools for online language learning: gender diferences}

\author[1]{Juan Fernando Gómez Paniagua~\orcid{0000-0003-1004-6793}\thanks{Email: \href{mailto:jgomez2@tdea.edu.co}{jgomez2@tdea.edu.co}}}
\author[1]{Jorge Emiro Restrepo~\orcid{0000-0001-8790-7454}\thanks{Email: \href{mailto:jorge.restrepo67@tdea.edu.co}{jorge.restrepo67@tdea.edu.co}}}
\author[2]{Claudio Díaz Larenas~\orcid{0000-0003-2394-2378}\thanks{Email: \href{mailto:claudiodiaz@udec.cl}{claudiodiaz@udec.cl}}}
\author[3]{Julio Antonio Álvarez Martínez~\orcid{0000-0001-9481-7422}\thanks{Email: \href{mailto:julio.alvarezm@udea.edu.co}{julio.alvarezm@udea.edu.co}}}
\affil[1]{Tecnológico de Antioquia, Institución Universitaria, Departamento de Ciencias Básicas y Áreas Comunes, Medellín, Colombia.}
\affil[2]{Universidad de Concepción, Facultad de Educación, Departamento de Curriculum e Instrucción, Bío Bío, Chile.}
\affil[3]{Universidad de Antioquia, Escuela de Idiomas, Medellín, Colombia.}

\addbibresource{article.bib}

\usepackage{multirow}
%\newcommand{\rotheader}[1]{\parbox[t]{2mm}{\multirow{2}{*}{\rotatebox[origin=c]{90}{#1}}}}
\usepackage{calc} % Include the calc package
%\newcommand{\rotheader}[1]{\parbox[t]{2mm}{\rotatebox[origin=bl]{90}{\makebox[\widthof{#1}][l]{#1}}}}
%\newcommand{\rotheader}[1]{\rotatebox[origin=bl]{90}{\begin{minipage}[t]{3cm}\raggedright #1\end{minipage}}}
%\newcommand{\rotheader}[1]{\rotatebox[origin=bl]{90}{\begin{minipage}[b][1cm][b]{3cm}\raggedright #1\end{minipage}}}
\newcommand{\rotheader}[1]{\rotatebox[origin=bl]{90}{\begin{minipage}[b]{3cm}\raggedright #1\end{minipage}}}



% Redefine column type with left alignment
\usepackage{array}
\newcolumntype{P}[1]{>{\raggedright\arraybackslash}p{#1}}

\usepackage{siunitx}
\sisetup{input-ignore={.},
     group-separator = {.},
     input-decimal-markers={,}, 
     output-decimal-marker = {,},
     group-minimum-digits=4
     }

\begin{document}
\maketitle
\begin{polyabstract}
\begin{abstract}
Las tecnologías Web 2.0 proveen diversos recursos para el aprendizaje del inglés en línea. La investigación tiene como propósito analizar las diferencias en las creencias sobre la autoeficacia percibida en las habilidades comunicativas y la actitud para el uso de las herramientas Web 2.0 en el aprendizaje del inglés en línea, con respecto a la variable sexo. Se realizó un estudio no experimental, ex post facto, descriptivo-inferencial y transversal con una muestra de 861 estudiantes en una universidad pública de Colombia. Los resultados indicaron que desde el Modelo de Aula Invertida en Encuentros Sincrónicos (MAIES), los hombres poseen una autoeficacia mayor en todas las habilidades comunicativas y emplean los recursos Web 2.0 en mayor medida. Las diferencias entre sexos se atribuyen a la motivación intrínseca, extrínseca e instrumental, a la forma en que las mujeres asumen el MAIES, y a un posible sentido de ‘autocomplacencia’ en la forma en que los hombres evaluaron su autoeficacia en el aprendizaje de una lengua extranjera (L2). En esta muestra, el sexo es un aspecto diferencial en el aprendizaje del inglés por medio del MAIES, en lo concerniente a las creencias de autoeficacia y al empleo de las herramientas Web 2.0. Se derivaron algunas implicaciones pedagógicas en consideración con los resultados.

\keywords{Aula invertida \sep Autoeficacia \sep Habilidades comunicativas \sep Herramientas web 2.0}
\end{abstract}

\begin{portuguese}
\begin{abstract}
As tecnologias da Web 2.0 fornecem vários recursos para a aprendizagem online da língua inglesa. A investigação tem como objetivo analisar as diferenças na percepção das crenças de auto-eficácia nas competências comunicativas e na atitude em relação à utilização de ferramentas da Web 2.0 na aprendizagem online da língua inglesa, tendo em conta a variável gênero. Foi realizado um estudo não-experimental, \textit{ex post facto}, descritivo-inferencial e transversal com uma amostra de 861 estudantes numa universidade pública na Colômbia. Os resultados indicaram que, a partir do Modelo de Sala de Aula Invertida em Encontros Síncronos (MAIES), os homens têm maior auto-eficácia em todas as competências comunicativas e utilizam mais os recursos da Web 2.0. As diferenças entre os gêneros são atribuídas à motivação intrínseca, extrínseca e instrumental, à forma como as mulheres adotam o MAIES e a um possível sentimento de "auto-complacência" na forma como os homens avaliaram a sua auto-eficácia na aprendizagem de línguas estrangeiras (L2). Nesta amostra, o gênero é um aspeto diferencial da aprendizagem da língua inglesa através do MAIES, no que diz respeito às crenças de auto-eficácia e à utilização de ferramentas da Web 2.0. Foram retiradas algumas implicações pedagógicas tendo em conta os resultados.

\keywords{Sala de aula invertida \sep Auto-eficácia \sep Competências de comunicação \sep Ferramentas web 2.0}
\end{abstract}
\end{portuguese}

\begin{english}
\begin{abstract}
Web 2.0 technologies offer different resources to learn English in a virtual environment. This study aims to analyze differences concerning self-efficacy beliefs perceived in both linguistic skills and the attitude towards the use of Web 2.0 tools when learning English online with respect to the gender variable. A non-experimental, ex post facto study was conducted following a descriptive-inferential and cross-sectional nature. The sample corresponded to 861 undergraduate students in a public university in Colombia. Findings demonstrated that, under the Flipped Classroom in Synchronous Settings Model (FCISSM) for English Language Learning, men have higher self-efficacy levels than women. Moreover, men use Web 2.0 resources to a larger extent to develop their linguistic skills. Gender differences are linked to intrinsic, extrinsic, and instrumental motivation to the way in which women have assumed the FCISSM and are also connected to a possible sense of ‘self-complacency’ regarding how men assessed their self-efficacy in a foreign language learning (L2). In this sample size, gender is a differential aspect in language learning through the FCISSM, regarding self-efficacy beliefs and the use of Web 2.0. Some pedagogical implications were provided in accordance with the results of the study.

\keywords{Flipped classroom \sep Self-efficacy \sep Communicative skills \sep Web 2.0 tools}
\end{abstract}
\end{english}
\end{polyabstract}

\section{Introducción}
A causa de la pandemia ocasionada por el COVID 19 en el año 2020, los escenarios físicos donde tienen lugar tradicionalmente los procesos de enseñanza-aprendizaje pasaron a ser mediados por plataformas digitales \cite{garcia_diaz_plataformas_2021}. Esto originó un modo de instrucción mediante eventos asincrónicos y sincrónicos \cite{dhawan_online_2020}, lo cual se constituyó en una propuesta pedagógica más flexible para los estudiantes. En este sentido, los recursos tecnológicos se convirtieron en una herramienta indispensable que permitió el aprendizaje del inglés bajo una modalidad de enseñanza denominada ‘Aula Invertida desde Encuentros Sincrónicos’ (MAIES) \cite{marshall_synchronous_2017}. 

En la literatura se encuentran diversos estudios que destacan la importancia del sexo en el aprendizaje de una lengua extranjera (L2) por encontrarse asociado al desempeño lingüístico \cite{gomez_paniagua_diferencias_2018,nguyen_relationship_2022,dong_influential_2023,csizer_gender-related_2024}. El sexo se encuentra determinado por las características biológicas y físicas que diferencian a un hombre de una mujer, lo cual incluye órganos, hormonas y cromosomas. En lo que atañe al aprendizaje de una segunda lengua, existen diferencias entre hombres y mujeres, entre las cuales se encuentran los estilos de aprendizaje, aspectos cognitivos, estructura cerebral \cite{gomez_paniagua_diferencias_2018}, el empleo de estrategias de aprendizaje \cite{alrabai_association_2018}, las aspiraciones académicas y aspectos culturales \cite{dong_influential_2023}, las creencias de autoeficacia \cite{mwaura_gender_2021,nguyen_relationship_2022} y la actitud en el empleo de las tecnologías web 2.0 \cite{jarrah_arab_2021,ningsih_gender-based_2022}. 

Específicamente, las variables que resultan de interés para esta investigación corresponden a (1) la autoeficacia, entendida como las creencias que un individuo tiene sobre su competencia en las habilidades lingüísticas, las cuales influyen positivamente sobre la motivación y práctica de la L2 \cite{bandura_self-efficacy:_1997,namaziandost_account_2020}, y que es fundamental para el aprendizaje en línea \cite{csizer_gender-related_2024}; (2) la actitud para el uso de las herramientas Web 2.0, debido a que la disposición e interés en la utilización de estos recursos digitales permite un aprendizaje de la L2 de manera independiente, colaborativa y flexible (Hassan y col., 2021), e incrementan la motivación \cite{asiksoy_elt_2018}; (3) y el sexo, por tratarse de una variable que no se ha investigado a profundidad en relación con la autoeficacia en el campo del aprendizaje del inglés por medio de recursos tecnológicos \cite{namaziandost_account_2020}, y por considerarse de interés para la comunidad científica \cite{jarrah_arab_2021,dong_influential_2023,csizer_gender-related_2024}. Conocer el nivel de percepción de autoeficacia percibida y las actitudes hacia el empleo de las tecnologías Web 2.0 de los estudiantes en el contexto universitario, podría servir para que los docentes de lengua extranjera diseñen experiencias de aprendizaje que incentiven la autoeficacia y favorezcan el desarrollo de la lengua por medio de los recursos digitales, en consideración con los aspectos que son diferenciadores entre ambos sexos. Basados en los expuesto anteriormente, se definió como objetivo analizar las diferencias en las creencias sobre la autoeficacia percibida en las habilidades lingüísticas y la actitud para el uso de las herramientas Web 2.0 en el aprendizaje del inglés con respecto al sexo, con una muestra de estudiantes universitarios.

En lo que respecta a la autoeficacia, este es un concepto que tiene un fundamento sociocognitivo y es definido por \textcite{bandura_self-efficacy:_1997} como el conjunto de creencias que una persona tiene sobre sus propias capacidades para organizar y ejecutar acciones necesarias para alcanzar resultados favorables. En el contexto escolar, la autoeficacia corresponde a las creencias que tienen los estudiantes sobre su propia habilidad para realizar las actividades académicas de manera exitosa y alcanzar las metas trazadas \cite{garcia_fernandez_predictive_2016}. \textcite{criollo_autoeficacia_2017} explican que la autoeficacia está conformada por la percepción que un individuo tiene sobre el grado de dificultad de una tarea y de su capacidad para llevarla a cabo, confianza que posea para realizarla satisfactoriamente y lograr el objetivo trazado, y capacidad para transferir los resultados logrados previamente a diversos ámbitos de la vida. Al respecto, \textcite{zhu_relationship_2020} sustentan que aquellos estudiantes que poseen un alto nivel de autoeficacia usualmente enfrentan los desafíos, se esfuerzan y son más persistentes en su desempeño académico, lo que es opuesto en aquellos que tienen un bajo un nivel de autoeficacia. 

Las creencias de autoeficacia afectan directamente el desempeño de aquellos que aprenden una L2, puesto que no basta con poseer las destrezas que se requieren para alcanzar la competencia comunicativa. En este sentido, \textcite{diaz_larenas_estudio_2013} sustentan que es esencial que los individuos confíen en sus capacidades sobre su desempeño en las habilidades lingüísticas correspondientes a la producción (hablar y escribir), y a la comprensión (leer y escuchar la lengua inglesa). Por consiguiente, la autoeficacia se constituye en un aspecto determinante en el aprendizaje de una segunda lengua dado que implica la persistencia y el interés por alcanzar la meta propuesta, y, además, influye sobre la motivación, la resiliencia y las reacciones emocionales \cite{goulao_relationship_2014}. En lo concerniente al aprendizaje del inglés por medio de la internet, \textcite{goulao_relationship_2014} afirma que es necesario tener las competencias necesarias y conocimientos tecnológicos que permitan el manejo de las distintas herramientas y recursos que provee este ambiente abierto y dinámico, para que la autoeficacia pueda surgir y generar un impacto positivo en el proceso de aprendizaje. En este sentido, \textcite{unveren_developing_2019} señala que los individuos que tienen un alto nivel de autoeficacia se esfuerzan en mayor manera para lograr el desarrollo en las cuatro habilidades comunicativas, y de acuerdo con \textcite{afifah_students_2021}, obtienen resultados positivos en su desempeño mediado por los diversos recursos digitales. 

Con relación a las tecnologías web 2.0, \textcite{moussaoui_integration_2020} explican que corresponden a la segunda generación de sitios web que posibilitan la transmisión de información, la interacción y colaboración entre las personas. Estos recursos virtuales permiten a los individuos el aprendizaje fuera del aula, en cualquier momento y desde cualquier lugar, a su ritmo de trabajo específico, y en consideración a sus necesidades particulares. De hecho, estas herramientas se han incorporado en el escenario educativo puesto que favorecen el aprendizaje colaborativo, promueven el aprendizaje flexible, y contribuyen al desarrollo de competencias concernientes al empleo de estos recursos digitales \cite{hassan_challenges_2021}. Al respecto, \textcite{asiksoy_elt_2018} señala que las tecnologías Web 2.0 son de gran utilidad para los docentes en diferentes disciplinas académicas, puesto que son utilizadas como un recurso educativo que promueve el aprendizaje autónomo. 

Igualmente, \textcite{asiksoy_elt_2018} argumenta que las herramientas Web 2.0 son una estrategia pedagógica que ofrece diversidad de recursos interactivos y de creación, los cuales fomentan el aprendizaje de una segunda lengua, y tiene un efecto positivo en la motivación. Al respecto, \textcite{mohammed_use_2020} explican que estas herramientas contribuyen al mejoramiento de las habilidades comunicativas en la L2 y al aprendizaje de vocabulario, puesto que permiten a los usuarios crear contenidos, compartirlos, e interactuar con otras personas. En este sentido, \textcite{moussaoui_integration_2020} señalan que los recursos Web 2.0 (redes sociales, procesadores de texto, Podcasts, Tools blogs, You Tube, presentaciones online, mapas conceptuales, entre otros), son elementos pedagógicos que favorecen el desarrollo de las habilidades lingüísticas por medio de contenidos de calidad y de interés general. \textcite{jarrah_arab_2021} afirman que en el momento en que las personas descubren la utilidad, accesibilidad y sencillez para manejar las tecnologías Web 2.0, desarrollan actitudes favorables hacia su empleo, lo que conlleva a incrementar la motivación, e impactar positivamente el aprendizaje de la L2. Esto indica que estos recursos tecnológicos funcionan como mediadores del aprendizaje y permiten la exposición a la lengua meta de una manera interactiva y real.  

En lo que respecta a las variables establecidas en esta investigación, la literatura reporta diversos estudios concernientes a la autoeficacia en el aprendizaje del inglés y su relación con el sexo en el contexto universitario. \textcite{wang_self-efficacy_2013} condujeron un estudio con 160 estudiantes alemanes y chinos y sus resultados revelaron que las mujeres tienen una percepción más favorable en cuanto a su autoeficacia en la L2. En el contexto egipcio, \textcite{abdelhafez_efl_2016} examinaron las creencias de autoeficacia con relación al sexo con una muestra de 320 estudiantes. Se encontró que las mujeres tienden a sentir una mayor confianza con relación a su autoeficacia en las cuatro habilidades comunicativas. \textcite{hasan_effect_2019} condujeron un estudio con una muestra de 400 estudiantes en India. Los resultados indicaron que el nivel de autoeficacia percibida en la lengua meta fue superior en las mujeres. 
 
Al respecto, \textcite{zhu_relationship_2020} investigaron la relación entre la autoeficacia en el inglés y su desempeño con una muestra de 387 estudiantes chinos. El estudio reveló que existen diferencias significativas en la autoeficacia entre ambos sexos, puesto que fueron las mujeres quienes evidenciaron un nivel de autoeficacia mayor. \textcite{nguyen_relationship_2022} estudiaron las influencias de las creencias de la autoeficacia en el aprendizaje del inglés con 128 estudiantes en Vietnam. Los análisis indicaron que las mujeres tienen una mayor autoeficacia que los hombres con relación a su desempeño lingüístico. \textcite{csizer_gender-related_2024} analizaron la diferencia entre ambos sexos referente a la autoeficacia y empleo de los recursos tecnológicos con una muestra de 467 hombres y 682 mujeres en Hungría. Los resultados demostraron que el grado de autoeficacia en las habilidades lingüísticas y confianza en la efectividad de los recursos digitales para el aprendizaje del inglés, es mayor en los hombres. Los estudios anteriores evidencian que, con relación a la dimensión lingüística en una L2, las creencias de autoeficacia se encuentran en ocasiones a favor de las mujeres o de los hombres. Para el caso de esta investigación, los hallazgos respaldan esta tendencia en los hombres.

En lo que atañe a la actitud para el uso de las herramientas Web 2.0, \textcite{kuznetsova_students_2019} examinaron el uso de estos recursos digitales para apoyar el proceso de aprendizaje de la L2 con una muestra de 137 estudiantes de una universidad de Estados Unidos. Se encontraron diferencias significativas con respecto al sexo, dado que los valores obtenidos indicaron que los hombres emplean estos recursos en mayor medida. En esta línea, \textcite{adibi_adoption_2019} se propuso hallar posibles diferencias entre ambos sexos en cuanto al empleo de estos recursos para el aprendizaje de una L2 con una muestra de 380 estudiantes universitarios en Nigeria. Se encontró que los hombres los utilizan en mayor proporción. \textcite{azak_analysis_2020} determinaron identificar las percepciones de hombres y mujeres sobre el empleo de estas herramientas para el aprendizaje del inglés con 427 estudiantes en Turquía. Los resultados revelaron que existe una diferencia significativa entre ambos sexos, y se encontró que los hombres dedican más tiempo y emplean estos recursos en mayor proporción.

Por su parte, \textcite{jarrah_arab_2021} examinaron los recursos Web 2.0 empleados por un grupo de 50 estudiantes en el contexto de Malasia. Se reportaron diferencias entre ambos sexos y se evidenció que las mujeres se encuentran mayormente involucradas y emplean en mayor medida estas tecnologías digitales. En el contexto de Indonesia, \textcite{ningsih_gender-based_2022} investigaron el empleo de las herramientas Web 2.0 con un grupo de 656 estudiantes. El estudio reveló que, a diferencia de las mujeres, los hombres tienen una actitud más favorable y consideran que estas herramientas digitales son de gran utilidad para el aprendizaje del inglés.

\section{Metodología}\label{sec-normas}
\subsection{Método}
Se realizó un estudio no experimental, ex post facto, de nivel descriptivo y transversal. Como objetivo se propuso analizar, con respecto al sexo, las diferencias en las creencias sobre la autoeficacia percibida en las habilidades lingüísticas y la actitud para el uso de las herramientas Web 2.0 en el aprendizaje del inglés en el contexto.

\subsection{Participantes}
La población estaba conformada por 2548 estudiantes matriculados en el curso de inglés. En el estudio participaron 861 estudiantes (nivel de confianza = 95 \%; margen de error = 3 \%) de la asignatura de inglés que fueron incluidos por conveniencia. De estos, hubo 296 hombres y 565 mujeres distribuidos en los niveles de dominio lingüístico A1 – Básico ($N = 307$), A2 – Básico superior ($N = 312$) y el B1 – intermedio ($N = 242$). Todos los participantes hacen parte de diversos programas técnicos, tecnológicos y profesionales en una universidad pública ubicada en la ciudad de Medellín, Colombia. Los niveles de dominio se han establecido en concordancia con los parámetros establecidos por el Marco Común Europeo de Referencia para las Lenguas \cite{consejo_de_europa_marco_nodate} y de acuerdo con los requisitos presentados en las mallas curriculares. Se considera necesario mencionar que el aprendizaje del inglés tiene lugar por medio de sesiones de trabajo realizadas bajo la modalidad de clases sincrónicas virtuales, a través de las plataformas digitales Microsoft Teams o Zoom.  En este escenario, se procura el desarrollo de las habilidades lingüísticas en la lengua meta bajo el ‘Modelo de Aula Invertida desde Encuentros Sincrónicos (MAIES)’, propuesto por \textcite{marshall_synchronous_2017}. Desde este escenario pedagógico, el docente provee a los estudiantes con diversos recursos de la internet y propone la realización de tareas a partir del empleo de las herramientas 2.0.

\subsection{Instrumentos de recolección de información}\label{sec-conduta}
El primer instrumento de recolección de datos correspondió al Cuestionario sobre la Autoeficacia en el Aprendizaje del Inglés propuesto por \textcite{wang_psychometric_2014}, conformado por 32 enunciados ponderados en una escala de medición Likert con 6 opciones, desde los rangos ‘Soy totalmente incapaz de hacerlo’ hasta ‘Lo puedo hacer muy bien’. El cuestionario mide la percepción que tiene una persona que aprende inglés en cuanto a su nivel de autoeficacia en cada una de las habilidades lingüísticas: comprensión auditiva (8 ítems); comprensión de lectura (8 ítems); expresión oral (8 ítems); producción escrita (8 ítems). \textcite{wang_psychometric_2014} indican que la bondad de ajuste del instrumento es adecuada, puesto que los resultados obtenidos correspondieron a: $\text{GFI} = \num{0,73}$; $\text{NFI} = \num{0,94}$; $\text{CFI} = \num{0,97}$; $\text{RMSEA} = \num{0,81}$. En el presente estudio, el coeficiente Alfa de Cronbach del instrumento correspondió a $\num{0,978}$. 

El segundo instrumento es el Cuestionario sobre Actitudes para el Uso de las Herramientas Web 2.0 diseñado por \textcite{seleviciene_university_2015}, empleado para determinar el grado de inclinación sobre las actitudes frente al uso de las herramientas educativas digitales Web 2.0 para el aprendizaje del inglés. Está conformado por tres dimensiones: Actitudes frente al empleo de las herramientas Web 2.0 (7 ítems); Percepciones respecto al uso de las plataformas digitales para mejorar las cuatro habilidades lingüísticas en una L2 (6 ítems); Empleo de herramientas Web 2.0 para el aprendizaje del inglés (7 ítems). En la primera dimensión, se presentan algunos enunciados que se orientan hacia la motivación. Por ejemplo, el Ítem 6: “El uso de las herramientas Web 2.0 hace que el aprendizaje del inglés sea más entretenido que con otros medios tradicionales de enseñanza”. Los ítems se presentan en una escala Likert con puntuaciones que van desde (1) ‘Totalmente en desacuerdo’ hasta (5) ‘Totalmente de acuerdo’, y otra que comprende rangos que van desde (1) ‘Para nada’ hasta (4) ‘bastante’. De acuerdo con \textcite{seleviciene_university_2015}, la validez de contenido del instrumento fue evaluada y refinada a través de un comité de expertos. En el presente estudio, el valor del Alpha de Cronbach correspondió a $\num{0,924}$.


\subsection{Procedimiento y análisis de la información}\label{sec-fmt-manuscrito}
Se invitó a los estudiantes a participar en la investigación por medio de su correo electrónicen el primer semestre del año 2023 en los meses de mayo y junio. Se incluyó un link de acceso a un formulario en línea de Google forms donde se presentó el propósito de la investigación, consentimiento informado y los instrumentos de recolección de información. Un lingüista experto verificó la debida traducción de los cuestionarios al castellano. Además, se diligenció el aval del Comité de Bioética de la institución donde se realizó el estudio. Los datos fueron analizados mediante el paquete estadístico SPSS v. 28. Se evaluaron los datos provenientes de los instrumentos a través de la prueba de Kolmogorov-Smirnov, y se comprobó que se ajustaban a una distribución normal. Se determinaron las medidas de resumen para las variables y se aplicó la prueba T de Student para muestras independientes para conocer si existían diferencias estadísticamente significativas ($p < \num{0,05}$) en las variables al comparar por sexo.


\section{Resultados}\label{sec-formato}
La prueba T-student reveló la existencia de diferencias estadísticamente significativas entre ambos sexos en todas las medias correspondientes a las cuatro dimensiones que conforman la autoeficacia en el aprendizaje del inglés, puesto que $p < \num{0,05}$ (\Cref{tab01}). La autoeficacia en la comprensión auditiva resultó ser superior en los hombres, ya que los resultados fueron $M = \num{3,3}$, $DE = \num{1,0}$ y en las mujeres $M = \num{3,1}$, $DE = \num{0,85}$. La comprensión de lectura también fue mayor en los hombres, dado que se presentó una $M = \num{3,5}$, $DE = \num{1,0}$ y en las mujeres fue $M = \num{3,2}$, $DE = \num{0,91}$. En cuanto a la expresión oral, el resultado delos hombres superó al de las mujeres, a razón de que los valores obtenidos correspondieron a $M = \num{3,4}$, $DE = \num{1,0}$ y $M = \num{3,2}$, $DE = \num{0,94}$ respectivamente. El nivel de autoeficacia en la producción escrita fue más alto en los hombres y los valores obtenidos fueron $M = \num{3,4}$, $DE = \num{1,0}$ y $M = \num{3,2}$, $DE = \num{0,94}$.

\begin{table}[htpb]
    \centering
    \footnotesize
    \begin{threeparttable}
    \caption{Diferencias entre el sexo con relación a las dimensiones que conforman la autoeficacia en la L2.}
    \label{tab01}
    \begin{tabular}{P{5cm} *{5}c}
    \toprule
    Dimensiones de la autoeficacia en las cuatro habilidades comunicativas en la lengua inglesa & \multicolumn{2}{c}{Hombres} & \multicolumn{2}{c}{Mujeres} & T-student \\
    & M & DE & M & DE & p \\
    \midrule
    Dimensión 1. Comprensión auditiva & 3,3 & 1,0 & 3,1 & 0,85 & 0,001 \\
    Dimensión 2. Comprensión de lectura & 3,5 & 1,0 & 3,2 & 0,91 & 0,000 \\
    Dimensión 3. Expresión oral & 3,4 & 1,0 & 3,1 & 0,91 & 0,000 \\
    Dimensión 4. Producción escrita & 3,4 & 1,0 & 3,2 & 0,94 & 0,001 \\
    \bottomrule
    \end{tabular}
    \notes{Nota. M = Media; DE = Desviación estándar. *$p<0,001$.}
    \source{Elaboración de los autores.}
    \end{threeparttable}
\end{table}

Los resultados presentados en la \Cref{tab02} indican la existencia de diferencias entre ambos sexos en lo que atañe a las cuatro dimensiones de la autoeficacia en el aprendizaje del inglés. Esto se evidencia en los resultados de cada enunciado que describe el grado de autoeficacia para cada habilidad comunicativa en la L2 en la dimensión 1 (ítems 1. 2. 3. 4. 6); dimensión 2 (ítems 11. 13. 14. 15. 16); dimensión 3 (ítems 17. 18. 19. 20. 21. 22. 23); dimensión 4 (ítems 25. 26. 29. 31. 32). considerando $p < \num{0,05}$. Se aclara que los ítems donde no se encontraron diferencias fueron removidos de la tabla (5. 7. 8. 9. 10. 12. 24. 27. 28 y 30).

\begin{table}[htpb]
    \centering
    \scriptsize
    \begin{threeparttable}
    \caption{Diferencias entre el sexo con relación a los ítems en cada dimensión de la autoeficacia en la L2.}
    \label{tab02}
    \begin{tabular}{P{5cm} *{11}c}
    \toprule
    Ítem & \rotheader{Soy totalmente incapaz de hacerlo} & \rotheader{No puedo hacerlo} & \rotheader{Posiblemente puedo hacerlo} & \rotheader{Lo puedo hacer de forma simple y al principio} & \rotheader{Puedo hacerlo} & \rotheader{Lo puedo hacer muy bien} & \multicolumn{2}{c}{Hombres} & \multicolumn{2}{c}{Mujeres} & \rotheader{T-student} \\
    & & & & & & & M & DE & M & DE & p \\
    \midrule
    1 Puedo entener historias narradas en inglés. & 2,9 & 18,1 & 45,6 & 14,6 & 15,3 & 3,4 & 3,4 & 1,2 & 3,2 & 1,0 & 0,002 \\
    2 Puedo enteder programas americanos presetados en inglés. & 4,9 & 26,7 & 42,3 & 12,9 & 10.3 & 2,9 & 3,2 & 1,2 & 2,9 & 1,0 & 0,001 \\
    3 Puedo entender programas de radio emitidos en inglés & 6,9 & 37,7 & 35,8 & 11,3 & 6,9 & 1,5 & 2,9 & 1,1 & 2,6 & 1,0 & 0,001 \\
    4 Puedo entender programas de televisión en inglés hechos en Colombia, & 4,4 & 23,7 & 45,2 & 11,4 & 13,2 & 2,1 & 3,3 & 1,2 & 3,0 & 1,0 & 0,000 \\
    6 Puedo comprender películas que no tengan ningún subtítulo, & 6,3 & 36,8 & 37,3 & 10,2 & 7,5 & 1,9 & 3,0 & 1,1 & 2,7 & 0,99 & 0,000 \\
    11 Puedo entender los mensajes o noticias publicadas en inglés en la internet. & 1,9 & 17,3 & 44,7 & 13,0 & 18,0 & 5,1 & 3,7 & 1,2 & 3,2 & 1,1 & 0,000 \\
    13 Puedo leer periódicos escritos en inglés, & 3,1 & 25,7 & 40,2 & 12,1 & 15,6 & 3,4 & 3,4 & 1,2 & 3,0 & 1,1 & 0,000 \\
    14 Puedo descubrir los significados de las palabras en un diccionario monolingüe, & 3,0 & 28,5 & 42,4 & 6,9 & 15,3 & 3,9 & 3,3 & 1,2 & 3,0 & 1,1 & 0,001 \\
    15 Puedo entender artículos en inglés sobre la cultura americana, & 3,5 & 30,7 & 44,4 & 8,5 & 11,0 & 2,0 & 3,2 & 1,1 & 2,8 & 1,0 & 0,000 \\
    16 Puedo comprender las lecturas que mi docente me entrega para una clase, & 1,9 & 10,2 & 45,4 & 13,2 & 22,0 & 7,3 & 3,8 & 1,2 & 3,5 & 1,1 & 0,001 \\
    17 Puedo describir mi universidad a otras personas en inglés, & 4,4 & 30,8 & 41,0 & 8,5 & 12,7 & 2,7 & 3,2 & 1,1 & 2,8 & 1,1 & 0,000 \\
    18 Puedo describir la manera de llegar a la universidad desde el lugar donde vivo, & 5,2 & 30,9 & 40,7 & 8,4 & 12,0 & 2,9 & 3,3 & 1,1 & 2,8 & 1,1 & 0,000 \\
    19 Puedo contar una historia en inglés & 5,1 & 37,6 & 37,6 & 8,7 & 8,6 & 2,3 & 3,0 & 1,1 & 2,7 & 1,0 & 0,000 \\
    20 Puedo hacer diversas preguntas a mi docente en inglés, & 3,1 & 24,0 & 44,1 & 7,0 & 17,2 & 4,5 & 3,4 & 1,2 & 3,1 & 1,1 & 0,000 \\
    21 Puedo presentar a mi docente a otra persona en inglés, & 3,1 & 18,1 & 46,8 & 7,2 & 19,4 & 5,3 & 3,5 & 1,2 & 3,2 & 1,2 & 0,010 \\
    22 Puedo discutir temas de interés general con mis compañeros en inglés. & 4,3 & 41,1 & 36,4 & 7,1 & 8.4 & 2,8 & 3,1 & 1,1 & 2,6 & 1,0 & 0,000 \\
    23 Puedo responder a las preguntas que hace mi docente en inglés, & 2,7 & 14,1 & 50,1 & 11,8 & 16,8 & 4,5 & 3,5 & 1,1 & 3,3 & 1,1 & 0,002 \\
    25 Puedo componer mensajes en redes sociales como Instagram y Facebook & 2,0 & 16,3 & 41,3 & 10,7 & 22,4 & 7,3 & 3,7 & 1,2 & 3,4 & 1,2 & 0,001 \\
    26 Puedo escribir un texto en inglés, & 2,1 & 21,6 & 42,6 & 12,0 & 17,2 & 4,5 & 3,5 & 1,2 & 3,2 & 1,1 & 0,000 \\
    29 Puedo escribir correos en inglés, & 3,0 & 28,5 & 39,0 & 10,7 & 15,7 & 3,1 & 3,4 & 1,1 & 3,0 & 1,1 & 0,000 \\
    31 Puedo escribir en un diario mis pensamientos, sentimientos y sucesos en inglés. & 3,3 & 36,0 & 38,9 & 8,1 & 10,9 & 2,8 & 3,1 & 1,1 & 2,8 & 1,0 & 0,001 \\
    32 Puedo escribir un ensayo de dos páginas sobre un tema en inglés. & 7,5 & 47,0 & 30,7 & 7,5 & 5,3 & 1,9 & 2,8 & 1,1 & 2,5 & 0,98 & 0,000 \\
    \bottomrule
    \end{tabular}
    \notes{Nota. M = Media; DE = Desviación estándar. *$p<0,001$.}
    \source{Elaboración de los autores.}
    \end{threeparttable}
\end{table}

Los resultados presentados en la \Cref{tab03} revelan que existen diferencias entre ambos sexos en relación con el empleo de las herramientas Web 2.0 para el aprendizaje del inglés, y los valores resultaron ser mayores en los hombres. No se presentaron diferencias estadísticamente significativas entre ambos sexos en la Dimensión 1 a nivel global (ítems 1 al 7), dado que $p = \num{0,018}$. Sin embargo, dentro de esta dimensión, las diferencias se presentaron en el ítem 1 ($\text{Hombres} - M = \num{4,3}$, $DE = \num{0,85}$ y $\text{Mujeres} - M = \num{4,2}$, $DE = \num{0,97}$, $p = \num{0,030}$), y en el ítem 4 ($\text{Hombres} - M = \num{4,0}$, $DE = \num{0,95}$ y $\text{Mujeres} - M = \num{3,8}$, $DE = \num{1,0}$, $p = \num{0,017}$) (\Cref{tab04}). La dimensión 2 (ítems 8 al 13) reveló diferencias entre ambos sexos, dado que se obtuvo un $\text{p valor} = \num{0,002}$ (\Cref{tab03}). Se hallaron diferencias específicamente en el ítem 8 ($\text{Hombres} - M = \num{3,7}$, $DE = \num{0,98}$ y $\text{Mujeres} - M = \num{3,4}$, $DE = \num{1,0}$, $p = \num{0,000}$); en el ítem 11 ($\text{Hombres} - M = \num{3,4}$, $DE = \num{1,0}$ y $\text{Mujeres} - M = \num{3,2}$, $DE = \num{1,0}$, $p = \num{0,009}$); en el ítem 12 ($\text{Hombres} - M = \num{3,6}$, $DE = \num{1,0}$ y $\text{Mujeres} - M = \num{3,3}$, $DE = \num{1,0}$, $p = \num{0,002}$); y finalmente en el ítem 13 ($\text{Hombres} - M = \num{3,8}$, $DE = \num{0,92}$ y $\text{Mujeres} - M = \num{3,6}$, $DE = \num{1,0}$, $p = \num{0,007}$) (\Cref{tab04}). En cuanto a la dimensión 3, se reportaron diferencias significativas en el empleo de dos herramientas digitales en particular: Blogs ($\text{Hombres} - M = \num{2,0}$, $DE = \num{8,3}$ y $\text{Mujeres} - M = \num{1,8}$, $DE = \num{0,75}$, $p = \num{0,000}$) y herramientas para dejar comentarios de voz, video o texto ($\text{Hombres} - M = \num{2,1}$, $DE = \num{0,88}$ y $\text{Mujeres} - M = \num{2,0}$, $DE = \num{0,79}$, $p = \num{0,018}$) (\Cref{tab03}).

\begin{table}[htpb]
    \centering
    \footnotesize
    \begin{threeparttable}
    \caption{Diferencias entre el sexo con relación a las herramientas web 2.0 por dimensiones.}
    \label{tab03}
    \begin{tabular}{P{5cm} *{5}c}
    \toprule
    Dimensiones de la autoeficacia en las cuatro habilidades comunicativas en la lengua inglesa & \multicolumn{2}{c}{Hombres} & \multicolumn{2}{c}{Mujeres} & T-student \\
    & M & DE & M & DE & p \\
    \midrule
    Dimensión 1 Actitudes hacia el uso de las plataformas digitales & 3,9 & 0,80 & 3,0 & 0,80 & 0,018 \\
    Dimensión 2 Percepciones sobre el uso de plataformas digitales para mejorar las habilidades comunicativas en inglés & 3,6 & 0,84 & 3,4 & 0,93 & 0,002 \\
    Dimensión 3 Herramientas digitales empleadas para desarrollar las habilidades comunicativas en inglés (blogs) & 2,0 & 0,83 & 1,8 & 0,75 & 0,000 \\
    Herramientas para dejar comentarios de voz, video o texto & 2,1 & 0,88 & 2,0 & 0,79 & 0,018 \\
    \bottomrule
    \end{tabular}
    \notes{Nota. M = Media; DE = Desviación estándar. *$p<0,001$.}
    \source{Elaboración de los autores.}
    \end{threeparttable}
\end{table}

\begin{table}[htpb]
    \centering
    \footnotesize
    \begin{threeparttable}
    \caption{Diferencias entre el sexo con relación a herramientas web 2.0 por ítems.}
    \label{tab04}
    \begin{tabular}{P{4cm} *{10}c}
    \toprule
    Ítem & \rotheader{Totalmente en desacuerdo} & \rotheader{En desacuerdo} & \rotheader{Neutral} & \rotheader{De acuerdo} & \rotheader{Totalmente de acuerdo} & \multicolumn{2}{c}{Hombres} & \multicolumn{2}{c}{Mujeres} & \rotheader{T-student} \\
    & & & & & & M & DE & M & DE & p \\
    \midrule
    1 Soy consciente de la existencia de plataformas virtuales para el aprendizaje del idioma inglés. & 3,1 & 2,3 & 8,0 & 35,5 & 51,0 & 4,3 & 0,85 & 4,2 & 0,97 & 0,030 \\
    4 Las plataformas virtuales son útiles para mis estudios del idioma inglés. & 3,0 & 5,7 & 19,0 & 41,2 & 31,0 & 4,0 & 0,95 & 3,8 & 1,0 & 0,017 \\
    5 El uso de plataformas virtuales es una buena estrategia para aprender inglés. & 4,3 & 6,7 & 20,6 & 40,1 & 28,3 & 3,8 & 0,99 & 3,7 & 1,0 & 0,102 \\
    8 El uso de plataformas digitales hace que mejore mi habilidad de comprensión lectora en inglés. & 5,1 & 10,3 & 29,7 & 37,0 & 17,8 & 3,7 & 0,98 & 3,4 & 1,0 & 0,000 \\
    11 El uso de plataformas virtuales hace que mejore mi habilidad para hablar en inglés. & 5,7 & 15,4 & 31,7 & 31,6 & 15,6 & 3,4 & 1,0 & 3,2 & 1,0 & 0,009 \\
    12 El uso de plataformas virtuales hace que mejore mis habilidades de pronunciación. & 4,8 & 14,5 & 27,9 & 35,5 & 17,3 & 3,6 & 1,0 & 3,3 & 1,0 & 0,002 \\
    13 El uso de plataformas virtuales hace que mejore mi vocabulario en inglés. & 3,6 & 7,4 & 24,3 & 41,1 & 23,6 & 3,8 & 0,92 & 3,6 & 1,0 & 0,007 \\
    \bottomrule
    \end{tabular}
    \notes{Nota. M = Media; DE = Desviación estándar. *$p<0,001$.}
    \source{Elaboración de los autores.}
    \end{threeparttable}
\end{table}

\section{Discusión}

En este estudio se propuso como objetivo analizar las diferencias en las creencias sobre la autoeficacia percibida en las habilidades lingüísticas y la actitud para el uso de las herramientas Web 2.0 en el aprendizaje del inglés con respecto al sexo. Los resultados indicaron que los hombres presentan un mayor nivel de autoeficacia en las cuatro habilidades comunicativas. Esto significa que los hombres podrían presentar mayor autoconfianza y/o motivación instrumental frente al estudio de una L2, además de, probablemente, dedicar más tiempo al desarrollo de las habilidades lingüísticas. Además, esto indica que los hombres posiblemente se perciben más capaces para realizar las tareas que requieren aprender una L2 a partir de las habilidades lingüísticas que poseen. Por consiguiente, se ha demostrado que las creencias de autoeficacia percibida en las habilidades lingüísticas son más favorables en los hombres en el contexto de la investigación. Esto resulta afín con la evidencia empírica reportada en el estudio de \textcite{csizer_gender-related_2024}, y es contradictorio con los hallazgos de \textcite{wang_self-efficacy_2013,abdelhafez_efl_2016,hasan_effect_2019,zhu_relationship_2020,nguyen_relationship_2022}, dado que en estas investigaciones las mujeres tienen un nivel de autoeficacia mayor en el aprendizaje de una segunda lengua. 

Este hallazgo respalda la idea de que existe una brecha entre ambos sexos en lo concerniente a la autoeficacia en el aprendizaje del inglés, en el contexto donde se realizó el presente estudio. Asimismo, este resultado podría indicar que los hombres han confiado en mayor medida en sus habilidades lingüísticas y que probablemente han tenido experiencias más positivas en el aprendizaje de la lengua meta con la ayuda de las herramientas tecnológicas. Este hallazgo causa sorpresa, dado que en la literatura se declara que las mujeres tienen una mayor autoeficacia y motivación en el aprendizaje de una L2, mientras que en los hombres se presenta en mayor proporción en el aprendizaje de las matemáticas, sistemas y ciencias sociales \cite{mwaura_gender_2021,nguyen_relationship_2022}. En este sentido, \textcite{zhu_relationship_2020,nguyen_relationship_2022} sustentan que las mujeres tienen mejor desempeño en las cuatro habilidades comunicativas puesto que su autoeficacia es mayor, y, en consecuencia, obtienen mejores resultados académicos en la lengua meta, a diferencia de los hombres. 

Por otro lado, los resultados obtenidos desafían de cierto modo el aspecto biológico. Al respecto, \textcite{zaidi_gender_2010} indica que el cerebro femenino procesa el lenguaje verbal de manera simultánea en ambos hemisferios cerebrales, mientras que el masculino emplea solamente el lado izquierdo. Adicionalmente, el mismo autor afirma que debido a que el tamaño del cuerpo calloso es más grande en las mujeres, se genera una mayor sinapsis entre ambos lados del cerebro. Los estudios de neuropsicología revisados por \textcite{lopez_rua_sex_2006} revelan que las mujeres poseen habilidades superiores para procesar el lenguaje oral, mientras que los hombres poseen un mejor desempeño en tareas que requieren el aspecto visual y espacial. En esta línea, \textcite{eriksson_differences_2012} afirman que las mujeres poseen habilidades lingüísticas más avanzadas que los hombres, que son notorias en gestos comunicativos, vocabulario productivo y en la cantidad de palabras que logran combinar, lo que indica mayor habilidad en la sintaxis. Sin embargo, debe tenerse en consideración que lo que se comparó no fue una habilidad cognitiva (dominio de una L2), sino un rasgo percibido de comportamiento (la autoeficacia). Así que estas diferencias neurobiológicas y neuropsicológicas entre sexos no serían determinantes.

Una primera explicación para este resultado contrastivo es que podría haber sido causado por la motivación intrínseca (intereses individuales) o extrínseca (acciones del entorno que movilizan la intrínseca) de los individuos, lo cual se evidencia en las áreas específicas de formación académica que los estudiantes se encuentran cursando. Posiblemente los hombres tuvieron un mayor nivel de autoeficacia debido a los programas académicos que predominaron en ellos: Tecnología en sistemas, Ingeniería en sistemas, Negocios internacionales, Comercio exterior, Ingeniería en software, e Ingeniería ambiental (70\%), en los cuales se requiere un dominio de la lengua extranjera correspondiente al B1 (intermedio) como requisito de grado. En las mujeres correspondieron a: Trabajo social, Psicología, Licenciatura en educación, Derecho y contaduría pública (61\%), y en estos programas se requiere acreditar el nivel A1 (básico) para el mismo propósito. Por lo tanto, podría presumirse que en las áreas de formación académica que los hombres han decidido estudiar, sus motivaciones personales, aspiraciones profesionales, y el nivel final de inglés requerido en cada programa justifican de alguna manera este resultado. En este tenor, \textcite{mwaura_gender_2021} sustenta que el éxito académico en determinado campo de estudio está influenciado por el nivel de autoeficacia que se tiene en este. 

Por otro lado, la revisión de la literatura realizada por Pajares (2002) respalda la idea de que los hombres tienen una tendencia a la ‘autocomplacencia’ en sus respuestas, mientras que las mujeres tienden a ser más modestas. Esto quiere decir que los hombres de esta muestra posiblemente perciben que poseen una alta autoeficacia en las habilidades lingüísticas en la L2, al igual que una alta autoconfianza sobre las mismas, pero en realidad su desempeño en la lengua meta no es tan satisfactorio como ellos presumen. Esta segunda hipótesis podría explicar de alguna manera el resultado contradictorio en lo referente a la autoeficacia con relación al sexo. Por otro lado, este resultado también podría indicar que posiblemente el modelo de MAIES no fue lo suficientemente atractivo y novedoso para las mujeres de esta muestra como para incentivar todo su potencial en lo concerniente al aprendizaje de la L2, dado que, en comparación a los hombres, las mujeres tienen habilidades superiores en la expresión oral y la comprensión auditiva \cite{abdelhafez_efl_2016,zhu_relationship_2020}.

Es importante aclarar que el éxito del aprendizaje de una lengua extranjera que se aprende por medio del MAIES depende de la práctica, preparación y estudio de los videos, actividades interactivas y recursos adicionales que el docente envía a los estudiantes antes de la clase. En este punto, los resultados conllevan a pensar que, de algún modo, el MAIES y la motivación han incrementado en mayor medida las creencias de autoeficacia en las cuatro habilidades lingüísticas en los hombres. Según \textcite{namaziandost_account_2020} la autoeficacia produce un efecto positivo en la confianza y motivación de los estudiantes. Por lo tanto, podría pensarse que esto fue lo que sucedió en el caso de los hombres que conforman esta muestra, y por tal razón, se presentó un efecto positivo en su desempeño en las habilidades comunicativas en la L2. Este resultado también demuestra que los hombres estuvieron más involucrados con el MAIES, lo que se traduce en mejores resultados académicos. No obstante, esto es contradictorio con la evidencia empírica de \textcite{chiquito_flipped_2020,namaziandost_account_2020}, puesto que estos autores sostienen que las mujeres son quienes sacan mayor provecho del MAIES. Este resultado podría atribuirse a la motivación instrumental que podrían tener los hombres de esta muestra, y que se relaciona con los programas académicos que se encuentran estudiando y con la prueba estandarizada que deben aprobar como requisito de graduación. De acuerdo con \textcite{alico_writing_2016}, este tipo de motivación se refiere a la utilidad de la lengua meta para aprobar exámenes, lograr metas profesionales, e incluso obtener beneficios financieros. 

En lo concerniente al uso de las herramientas Web 2.0 para el aprendizaje del inglés, los resultados revelaron que los hombres emplean estos recursos en mayor medida y esto se encuentra alineado con lo reportado por \textcite{adibi_adoption_2019,kuznetsova_students_2019,azak_analysis_2020,ningsih_gender-based_2022,csizer_gender-related_2024}, pero está en oposición con los resultados de \textcite{jarrah_arab_2021}. Este hallazgo significa que los hombres tienen una mayor conciencia frente al uso de los recursos Web 2.0 para trabajar en las habilidades de habla, lectura y escritura, y su preferencia está en los blogs y herramientas que permiten practicar la comunicación verbal. \textcite{kleanthous_collaboration_2016} explican que los blogs constituyen un recurso divertido y motivante, porque integra las habilidades de habla, lectura y escritura en un ambiente colaborativo de aprendizaje de una segunda lengua. Adicionalmente, las herramientas de voz hacen posible practicar la pronunciación, entonación, comprensión y retención del vocabulario requerido para el mensaje (estas preferencias se evidenciaron en los resultados obtenidos en la \Cref{tab04}). Las diferencias encontradas podrían haber ocurrido debido a una baja actitud de las mujeres hacia el empleo de los recursos tecnológicos web 2.0 para el aprendizaje de una L2. Al respecto, \textcite{cai_gender_2017} afirman que, con relación a las mujeres, los hombres demuestran una actitud más favorable hacia estas herramientas digitales para el aprendizaje de una lengua extranjera, lo cual explica de cierto modo este resultado.

\section{Conclusiones}\label{sec-modelo}
Esta investigación demostró que, en esta muestra, los hombres poseen un nivel de autoeficacia mayor al de las mujeres en el aprendizaje del inglés que tienen lugar por medio del MAIES. Esto significa que los hombres en este contexto tienden a tener un nivel de ansiedad más bajo, asumen con mayor persistencia los retos que se presentan en las tareas que apuntan a desarrollo de las habilidades lingüísticas y poseen una motivación más alta. Adicionalmente, se encontró que en el aprendizaje del inglés que se realiza por medio del MAIES, los hombres tienen una actitud más favorable y emplean en mayor medida los recursos Web 2.0 para el desarrollo de las habilidades lingüísticas, especialmente las concernientes a la expresión oral, escritura y comprensión de lectura. Los autores de esta investigación atribuyen estos resultados a factores que atañen la motivación intrínseca, motivación extrínseca, motivación instrumental, a la manera en que las mujeres han asumido el MAIES en el aprendizaje de la L2, y a un posible sentido de ‘autocomplacencia’, en la manera en que los hombres han evaluado su autoeficacia. 

De este estudio se derivan algunas implicaciones pedagógicas. En primer lugar, aquellos individuos que poseen un alto nivel de autoeficacia pueden alcanzar mejores resultados académicos, puesto que emplean más estrategias de aprendizaje y mantienen niveles de ansiedad más bajos \cite{hasan_effect_2019,zhu_relationship_2020}, lo que incide positivamente en el desarrollo de las habilidades lingüísticas en la lengua meta \cite{wang_self-efficacy_2013,alrabai_association_2018}, e incentiva la utilización de las herramientas Web 2.0 para lograr este propósito \cite{seleviciene_university_2015,csizer_gender-related_2024}. Se considera fundamental que los docentes generen experiencias de aprendizaje dentro y fuera del aula donde se aprende una L2, que apunten al desarrollo de las cuatro habilidades comunicativas a partir de experiencias motivantes, interesantes y significativas, para que los estudiantes alcancen la competencia comunicativa (hacer reportes orales, escuchar noticias, grabar videos, escribir blogs, leer diversos textos, entre otros). En este sentido, los recursos Web 2.0 se constituyen en herramientas adecuadas para alcanzar este propósito \cite{asiksoy_elt_2018,moussaoui_integration_2020,jarrah_arab_2021,ningsih_gender-based_2022}. 

En segunda instancia, se considera que una manera de minimizar la disparidad en las diferencias encontradas entre ambos sexos es a partir de una apropiada retroalimentación correctiva \cite{gomez_diferencias_2019}, de acuerdo con las necesidades de cada individuo en cada una de las habilidades lingüísticas. Por lo tanto, es fundamental que el docente ayude a los estudiantes a superar las dificultades que surgen en el proceso de aprendizaje de la lengua meta a partir de la exaltación de sus aciertos, promueva el desarrollo de la competencia comunicativa en cada encuentro sincrónico, y que emplee los recursos Web 2.0 de tal forma que el sentido de autoeficacia de cada individuo aumente. De acuerdo con \textcite{alrabai_association_2018}, una autoeficacia elevada tiene el poder de incrementar la autonomía en el proceso de aprendizaje de los estudiantes, y esto es fundamentan en aquellos que aprenden una nueva lengua. 

Entre las limitaciones del estudio, se considera importante mencionar que los resultados podrían complementarse si se considera la inclusión de diversas herramientas de recolección de información, tales como entrevistas y videos que permitan aportar nueva información para la comprensión del fenómeno abordado en este estudio. Los análisis realizados no consideraron la segmentación de los niveles de dominio en la L2 (A1 – Básico, A2 – Básico superior, B1 – intermedio), ni el tipo de programa académico cursado en el contexto de la investigación, lo que podría proporcionar nueva información que permita realizar análisis más profundos sobre el comportamiento de estas variables en el contexto de la presente investigación. Además, al ser una muestra por conveniencia, existe el sesgo de autoselección. Finalmente, se recomienda que en futuras investigaciones se analice si existe correlación entre las creencias de autoeficacia y la actitud en el uso de las herramientas Web 2.0, y se efectúen análisis en función del nivel de dominio lingüístico y el programa académico. Además, podría realizarse un estudio de nivel explicativo que incluya otras variables que inciden en el aprendizaje de una L2 en línea, como la autorregulación, ansiedad lingüística y motivación.



\printbibliography\label{sec-bib}
%conceptualization,datacuration,formalanalysis,funding,investigation,methodology,projadm,resources,software,supervision,validation,visualization,writing,review
\begin{contributors}[sec-contributors]
\authorcontribution{Juan Fernando Gómez Paniagua }[conceptualization,projadm,investigation,methodology,supervision,writing,review]
\authorcontribution{Jorge Emiro Restrepo}[methodology,projadm,datacuration]
\authorcontribution{Claudio Díaz Larenas}[formalanalysis,supervision,writing,review]
\authorcontribution{Julio Antonio Álvarez Martínez}[formalanalysis,writing,review]
\end{contributors}
\end{document}
