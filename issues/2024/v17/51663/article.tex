\documentclass[english]{textolivre}

% metadata
\journalname{Texto Livre}
\thevolume{17}
%\thenumber{1} % old template
\theyear{2024}
\receiveddate{\DTMdisplaydate{2024}{3}{18}{-1}}
\accepteddate{\DTMdisplaydate{2024}{6}{25}{-1}}
\publisheddate{\today}
\corrauthor{Liudmila Shafirova}
\articledoi{10.1590/1983-3652.2024.51663}
%\articleid{NNNN} % if the article ID is not the last 5 numbers of its DOI, provide it using \articleid{} commmand 
% list of available sesscions in the journal: articles, dossier, reports, essays, reviews, interviews, editorial
\articlesessionname{articles}
\runningauthor{Shafirova and Araújo e Sá}
%\editorname{Leonardo Araújo} % old template
\sectioneditorname{Daniervelin Pereira}
\layouteditorname{João Mesquita}

\title{Students making videos on social media: exploring the potential of online videos for language learning}
\othertitle{Produção de vídeos em redes sociais por estudantes: explorando o potencial de vídeos online para a aprendizagem de línguas}

\author[1]{Liudmila Shafirova~\orcid{0000-0003-3743-2029}\thanks{Email: \href{mailto:liudmila.shafirova@ua.pt }{liudmila.shafirova@ua.pt }}}
\author[1]{Maria Helena Araújo e Sá~\orcid{0000-0002-6623-9642}\thanks{Email: \href{mailto:helenasa@ua.pt}{helenasa@ua.pt}}}
	
\affil[1]{University of Aveiro, Department of Education and Psychology, CIDTFF, Aveiro, Portugal.}

\addbibresource{article.bib}

\begin{document}
	
\maketitle
\begin{polyabstract}

\begin{abstract}
In recent years, communication on social media has undergone major changes, shifting towards being more multimodal and video-centred. This study investigates the engagement of Portuguese university students in video production within social media platforms and examines their perceptions of language learning opportunities inherent to this practice. We conducted a questionnaire among 212 students at the University of Aveiro (Portugal), delving into the role of video-making as an interactive learning tool. Our results show that roughly 22\% of the respondents make videos, indicating that video production is not very common among them, especially when compared to video watching. Despite this, our analysis highlights several possible educational benefits of making online videos. Our data illustrate the interactive and multilingual nature of online video-making, encompassing practices such as searching for information, collaborating with others, or watching other similar videos in different languages. Finally, while this study highlights the infrequency of video production, it underscores its potential as a holistic approach to language learning and the development of multiliteracies.


\keywords{Online videos \sep Social media \sep Multiliteracies \sep Informal language learning}
\end{abstract}

\begin{portuguese}
\begin{abstract}
Nos últimos anos, a comunicação nas redes sociais passou por significativas transformações, evoluindo para uma abordagem mais multimodal e orientada para o vídeo. Este estudo investiga o envolvimento de estudantes universitários portugueses na produção de vídeos em plataformas de redes sociais e examina as suas percepções de oportunidades de aprendizagem de línguas inerentes a esta atividade social. Para atingir estes objetivos, realizamos um questionário com 212 estudantes da Universidade de Aveiro. Os resultados mostram que cerca de 22\% dos inquiridos fazem vídeos, o que indica que esta atividade não é muito comum entre eles, especialmente quando comparada com a visualização de vídeos. Apesar disso, os resultados destacam vários benefícios educativos possíveis da criação de vídeos online, como a sua natureza interativa e multilíngue. Essas práticas incluem a busca de informações, a colaboração com outros e a visualização de vídeos semelhantes em diferentes línguas. Por fim, embora este estudo destaque a pouca frequência da produção de vídeos, sublinha o seu potencial como abordagem holística à aprendizagem de línguas e ao desenvolvimento de multiletramentos.

\keywords{vídeos online \sep redes sociais \sep multiletramentos \sep aprendizagem de línguas fora da escola}
\end{abstract}
\end{portuguese}
\end{polyabstract}

\section{Introduction}\label{sec-introduction}

Over the past decades there has been a growing interest in the
development of new methodologies for second language (L2) learning and
teaching, to fulfil the requirements of the new legislation, as defined
by the Common European Framework of Reference for Languages (CEFR),
which emphasizes the communicative aspect of languages.

One of these new methodologies is Didactic Audiovisual Translation
(DAT), i.e., the application of different modes of audiovisual
translation (subtitling, revoicing, audio description and voice over) to
L2 teaching. This approach has been implemented and its results analysed
in different educational stages. Nevertheless, this approach has yet
been used in primary education within alternative methodologies, such as
the Montessori Method, in blingual contexts (in this case,
Basque-Spanish). The present study aims to address this research gap.

This paper shows the results of an intervention combining subtitling and
dubbing in a class of 11-12-year-old pupils following the Montessori
Method, at a Basque-immersion language primary school. While our study
could not assess language acquisition improvements due to the
school's methodology, our findings regarding children
satisfaction and the alignment of DAT principles with Montessori
principles, as demonstrated in the text, suggest promising avenues for
successful DAT use in this environment.


%
\section{Theoretical framework}\label{sec-theoframework}

In this section, we will review prior research related to video-making
and its connection to language learning. Also, we will outline our
approach to analysing video-making, employing the framework of
multiliteracies.

\subsection{Video-making and language learning}\label{sub-sec-video-making}

For the last 20 years, research has expanded into diverse aspects of
learners\textquotesingle{} engagement with videos. This research delved
into how video production was used in the classroom to foster students'
creativity and enhance social competence \cite{fethi2018}, to
increase active knowledge construction and students' cooperation
\cite{nikitina2010}, digital competence \cite{yeh2018}, critical digital
literacies \cite{delosríos2018} or to improve speaking competence \cite{devana2021}.

Several studies have extensively explored speaking competence in the
context of video production, as videos could be useful for educators due
to the ability to revise and practice the students' speech \cite{cassany2021}. This competence encompasses a spectrum of skills
including pronunciation, comprehension, fluency, vocabulary, and grammar
of oral production \cite{devana2021}. Several studies highlighted
videos as a valuable instrument to practice speaking fluency or
motivation to speak \cite{tan2022}. For instance, \textcite{devana2021}, in a quasi-experimental study conducted at an Indonesian
university, demonstrated that the students who followed tasks of making
video blogs in English achieved significantly higher speaking scores and
were more motivated to speak in comparison to a control group.
Similarly, in another study employing qualitative analysis, students'
speaking competence scores also improved after producing TikTok videos
\cite{zaitun2021}. Nevertheless, some studies yielded more
mixed findings. For instance, a qualitative study conducted in a high
school in the Philippines found no improvement in students' perceptions
of their speaking ability through the use of TikTok \cite{asio2023}. The authors suggest that this may be because the tasks needed to
be more contextualised to be meaningful.

A systematic review of studies on TikTok indicates that in-school
research on video production has predominantly focused on speaking
competence \cite{tan2022}. Meanwhile, the research on
out-of-school language learning through video production highlights the
value of written interaction as a source of learning \cite{cassany2021}. For instance, when users post videos on social media,
they also interact with their followers by reading and writing comments,
frequently in a foreign language. These interactions could be important
for language learning, as illustrated in the study of \textcite{vazquez-calvo2023}, where the comments under TikTok and YouTube videos
were analyzed and described as a valuable discussion environment for
Korean language learning. In addition, a study on Instagram revealed
that reading in English was the competence highlighted by the students
when asked about their perceptions of language learning on Instagram
\cite{gonulal2019}.

Reading, writing, chatting, and translating were found to be frequent
practices in another area of out-of-school learning -- fandom, meaning
the organisation of affectionate consumers \cite{sauro2017}. In this line
of research, the focus has been on video-making through fan translation,
i.e., fansubbing, creating subtitles for videos \cite{tee2022} and fandubbing, and recording voice acting in a different language
\cite{shafirova2019}. This research on subtitling and dubbing
highlights the pivotal role of community and feedback provided in these
learning spaces \cite{zhang2016}. Fans engaged in these spaces
self-reported improvement in various foreign language skills including
translation, written interaction, writing, reading and speaking \cite{benson2015,shafirova2019}.

Moreover, plurilingualism and intercultural learning have been important
topics in both contexts of social media and fandom research. Case
studies focused on YouTube described some results on cultural identity
development by creating vlogs in a foreign country \cite{chang2019}, and learning multiple languages by fandubbing, singing and
interacting with followers \cite{zhang2022}. Also, on
TikTok, \textcite{vazquez2022} focused on the platform's
language learning opportunities in different languages (Russian, Chinese
and Italian). The study highlights TikTok's potential for plurilingual
and pluricultural learning through the variety of languages used in the
videos and the focus on different cultural stereotypes usually discussed
in the comments.

In conclusion, our analysis suggests that the main focus of research
regarding in-school video-making is somewhat different from the
out-of-school context. In-school studies primarily focus on the
development of speaking competence \cite{asio2023,devana2021,tan2022,zaitun2021},
whereas out-of-school data mostly centres around interaction,
translation, and intercultural competence \cite{gonulal2019,vazquez2022,zhang2022}. Also, the majority
of the out-of-school research results come from case studies, so it is
difficult to understand if video-making is a frequent and beneficial
practice among the majority of the students. This study focuses on
contributing to minimising these gaps and offering valuable insights
into the field of out-of-school language learning.

\subsection{Multiliteracies and video-making}\label{sub-sec-multiliteracies}

We look at video production from a multiliteracies perspective in which
digital multimodal communication can disrupt traditional text-based
literacies and should be incorporated into the classroom and curriculum
\cite{cope2015,thorne2013}. Digital multimodal tasks are
suggested to enhance the students' learning by bringing education closer
to the out-of-school reality \cite{ito2013}, positioning the
students at the centre of the learning process \cite{cope2009}, and motivating them to engage in the tasks \cite{muñoz-basols2019}.

Similarly to \textcite{yeh2018}, we look at online video-making as a
multiliteracy practice, hence we focus our attention on the fact that
this practice could be multilingual, multimodal and interactive. As a
social and multimodal practice, it could be complex and include various
modalities \cite{cope2009}. Video production can include
various activities during its preparation, such as searching for
information, analysing similar videos, writing descriptions or scripts,
or collaborating with others, as highlighted by \textcite{yeh2018}. Also, video
production can involve a variety of interactive activities after the
video is completed, such as reading and responding to comments. In
addition, as a possibly multilingual practice, all of the activities of
preparation for the video and management of the feedback after posting a
video can be done in different languages, providing a field for
engagement among different languages in several modalities.

With this study, we are focusing on adding knowledge to the video
practices of the students from a multiliteracy perspective including the
use of different languages, and various interactive activities
undertaken before and after video production.%
\section{Materials and methods}\label{sec-materialsand}

In this section, the methodological aspects of the study will be
discussed including the procedure, participants, analysis and ethical
considerations. As an exploratory study, the majority of our inquiries
were inherently descriptive, intending to provide a wide picture of the
phenomenon and lay the groundwork for future studies. The main method of
the study is a quantitative questionnaire with some open questions
analysed qualitatively.

\subsection{Questionnaire}\label{sub-sec-questionnaire}

We started the data collection with a questionnaire which was
distributed to all the students in bachelor, master and PhD programs at
the University of Aveiro. Thus, on 17.02.2022 the Heads of all of the
departments sent the questionnaire to the students via email. Two
reminders were issued to obtain more responses. In total, 299 responses
were obtained from 11,932 students. Within this number, 212
questionnaires were fully answered, hence, the total N is 212. The low
rate of response to the questionnaire could be related to the fact that
there are many questionnaires distributed at the University so the
students are normally flooded with demands for participation in
research. The questionnaire consisted of two separate parts: (1) video
viewing (212 answers) and (2) video production (46 answers). In this
paper, we will mostly centre on the second part of the questionnaire,
which was answered by 46 students, with the use of the whole dataset
only for comparisons. This part has 14 questions including 4 yes/no
questions, 5 multiple choice questions, 3 matrix questions, 1 Likert
question and 1 open question. The questionnaire was validated by two
researchers in the areas of plurilingualism and technology-mediated
education who gave their written feedback. Moreover, a small pilot study
was conducted with 17 PhD students from the University of Aveiro,
Education and Psychology faculty, on the 28th of January 2022. It
consisted of a recorded one-hour focus-group session during which the
participants filled up the questionnaire, wrote their critique and
afterwards discussed it via ZOOM. Based on this feedback, subsequent
changes were made to it.

Also, the study received approval from the data protection committee of
the University of Aveiro. The consent forms were elaborated and
validated according to the Portuguese data protection laws. The consent
form appeared before the questionnaire, so the participants had to agree
with the conditions to fill in the questionnaire.

\subsection{Participants}\label{sub-sec-participants}

The participants of the questionnaire were mostly randomly sampled from
the students at the University. The main sampling criteria was to be
enrolled in the University. The descriptive statistics of the whole
questionnaire and the second part of the questionnaire regarding the age
and gender of the participants are presented in \Cref{tab-01}. Students'
mother tongues are shown in \Cref{tab-02}. These specific languages were
chosen based on the statistics of international students at the
university. \Cref{tab-03} shows how many students are currently learning an
additional language. Also, most of the departments are represented in
the video production part, however, most students are from the
departments of Education and Psychology (45.7\%), Biology (13\%) and
Languages and Cultures (10.9\%). It is important to mention that the
curriculum of the Education and Psychology department includes
Master\textquotesingle s programmes with language classes and language
didactics in the bachelor\textquotesingle s degree. In the Biology
department programmes, there are almost no obligatory language courses.


\begin{table}[htbp]
\centering
\begin{threeparttable}
\caption{Age and gender of the participants.}
\label{tab-01}
\begin{tabular}{llllll}
\toprule
\multicolumn{6}{c}{The whole questionnaire} \\
\midrule
\multicolumn{3}{c}{Gender}  & \multicolumn{3}{c}{How old are you?}\\
	& N & \% & & N& \% \\
Female & 149 & 70.3\% & From 18 to 23 & 93 & 43.9\% \\
Male & 61 & 28.8\% & From 24 to 35 & 75 &35.4\% \\
Prefer not to answer & 2 & 0.9\% & More than 36 & 44 & 20.8\%\\
\multicolumn{6}{c}{\rule{0pt}{4ex}The video production part of the questionnaire} \\
 & N & \% & & N & \% \\
Female & 26 & 56.5\% & From 18 to 23 & 13 & 28.3\% \\
Male & 20 & 43.5\% &From 24 to 35 & 15 & 32.6\% \\
Prefer not to answer & 0 & 0\% & More than 36 & 18 &39.1\% \\
\bottomrule
\end{tabular}
\source{Own elaboration. Information from the publication of \textcite{shafirova2023} is partly used.}
\end{threeparttable}
\end{table}

\begin{table}[htb]
\centering
\begin{threeparttable}
\caption{Mother tongues of the participants.}
\label{tab-02}
\begin{tabular}{llllll}
\toprule
\multicolumn{3}{>{\raggedright\arraybackslash}p{0.4\textwidth}}{Mother tongue(s) of the whole questionnaire} & 
\multicolumn{3}{>{\raggedright\arraybackslash}p{0.4\textwidth}}{Mother tongue(s) of the video production section of the questionnaire}\\
\midrule
	& \multicolumn{2}{l}{Responses} & & \multicolumn{2}{l}{Responses}\\
	& N & \% & & N & \% \\
Portuguese & 191 & 92.7\% & Portuguese & 39 & 82.6\%\\
Spanish & 8 & 3.9\% & Spanish & 2& 	4.3\% \\
English & 5 & 2.4\% & Russian & 1 & 2.2\% \\
Persian & 4 & 1.9\% & Persian & 1 & 2.2\% \\
Chinese & 2 & 1\% & Chinese & 1 & 2.2\% \\
French & 1 & 0.5\% & Other &3 & 6.6\% \\
Russian & 1 & 0.5\% & Total & 46 & 100.1\% \\
Other & 8 & 3.9\% &&& \\
\hline
Total & 212 & 106.8\%\tnote{1} &&&\\
\bottomrule
\end{tabular}
\begin{tablenotes}
\item[1] More than 100\% due to the overlap of mother tongues.
\end{tablenotes}
\source{Own elaboration. Information from the publication of \textcite{shafirova2023} is partly used.}
\end{threeparttable}
\end{table}

\begin{table}[htb]
\centering
\begin{threeparttable}
\caption{Participants as language learners.}
\label{tab-03}
\begin{tabular}{*6{l}}
\toprule
\multicolumn{6}{c}{Are you learning any language at the moment?}\\
\midrule
\multicolumn{3}{c}{The whole questionnaire} & \multicolumn{3}{c}{The video production part}\\
 & N & \% & & N & \% \\ Yes & 89 & 42\% & Yes & 15 & 32.6\% \\ No & 123 & 58\% & No & 31 & 67.4\%\\
\bottomrule
\end{tabular}
\source{Own elaboration.}
\end{threeparttable}
\end{table}

Following \Cref{tab-01}, we can see that in the video viewing section, we had
more students ranging from 18 to 23 years, than in the video production
section, in which the age of the participants is distributed more
towards the option ``more than 36'' (39.1\%). According to \Cref{tab-02}, most
of the students reported Portuguese as their mother tongue. It is
important to note that this question allowed multiple selections, though
only a few respondents chose multiple languages (hence, the percentages
in the ``total'' column slightly exceed 100\%). Moreover, \Cref{tab-03} shows
that slightly more participants are not currently learning any language,
particularly in the video production section of the questionnaire.

\subsection{Analysis}\label{sub-sec-analysis}

All of the answers provided were first analysed with descriptive
statistics of SPSS including frequencies. We also ran comparison or
cross-tabulation tests with age and gender regarding the platforms the
students chose, and regarding the students' mother tongues when
producing videos. When providing comparisons among age and gender
concerning the platforms the students chose, we considered the Pearson
chi-square test and significance.

We also analysed the open questions with a bottom-up content analysis.
We followed the topics of the responses which arrived from the data;
hence, the majority of the responses were centred on the descriptors of
language proficiency. Consequently, we centred on these descriptors
including pronunciation, vocabulary, grammar, comprehension,
communicative skills, language use in a context, and cultural aspects.
The codebook and examples of analysis are in Annex A.


%
\section{Results}\label{sec-results}

\subsection{First Expectations}\label{sub-sec-firstexpectations}

The authors hypothesised that pupils would be able to achieve a higher
linguistic level as a result of the unquestionable stimulation provided
by the audio-visual tools, which have been demonstrated to address the
learning needs of students. Additionally, in accordance with the
findings of previous research, it was anticipated that the
children\textquotesingle s level of motivation for the subject would
increase.

Nevertheless, it is reasonable to anticipate certain challenges
associated with the computer skills and linguistic proficiency of the
pupils. With regard to the latter, it is important to recognise that
children are simultaneously acquiring three languages without formal
instruction, which could potentially lead to some difficulties.

\subsection{Results on Acquisition of the Language}\label{sub-sec-resultsonacquisition}

As previously stated, the Montessori Method does not include exams,
therefore it was not feasible to propose a pre-test and post-test
framework to collect quantitative data. Consequently, it is not possible
to provide evidence of language improvement, if any. Nonetheless, it was
possible to anticipate certain improvements in their productions, given
that the children were engaged in activities such as translating,
dubbing and subtitling videos, practising listening comprehension,
pronunciation, writing and speaking.

A further defining feature of Montessori is that children select the
areas of study that they wish to pursue. Thus, this approach presented a
significant challenge for them, as their choices were based on the
scenes they enjoyed, irrespective of the level of L2 proficiency
required. This is the reason why the implementation resulted in
unforeseen language difficulties for the children, which obliged the
teacher to provide linguistic support by scaffolding the language.

Although it was not possible to analyse in detail the language
improvement resulting from the implementation, some common mistakes
committed by the pupils could be identified. These can be divided into
two main groups: those related to written productions and those related
to oral ones. In general, the children committed grammatical mistakes in
order to fit the sentences with the speech, such as the removal of the
subjects or the elision of auxiliary verbs. The errors associated with
oral production were primarily related to the rhythm of the
conversations and the pronunciation, particularly that of the vocalic
phonemes, as the pupils read the words in a literal manner. It is
important to highlight the intricate nature of these phonemes for
speakers of Basque and Spanish. In comparison to English, which has
twelve vocalic sounds, the two languages in question possess only five.
Furthermore, the pace of the original dialogues also affected the
pupils\textquotesingle{} oral productions, necessitating adjustments in
their speaking rate to align with the tempo of the original dialogues.

\subsection{Results of the Questionnaire}\label{sub-sec-resultsofthequestionnaire}

Following the completion of the implementation, pupils completed a
survey about AVT and DAT (see \Cref{annex-01}). Unfortunately, as the teachers
did not compel the pupils to fill it, and children were being prepared
for their incorporation to the formal instruction of the next course
(1st of Secondary Education), only seven children answered to the
questionnaire. The questions addressed their opinions regarding the
utilisation of DAT, the motivation it provides and its potential
integration into future English classes. As the pupils are taught in the
Basque language, and given the inherent complexity of the vocabulary and
the dearth of knowledge about the subject matter, the survey was
developed in Basque. However, for the sake of clarity and accessibility,
the sentences have been translated into English in the title of the
figures.

The initial question (see \Cref{fig-01}) sought to ascertain the
children\textquotesingle s level of enjoyment in the workshop. All of
the participants provided a rating that was above the minimum passing
grade. Four children, representing over half of the sample, rated the
workshop with an 8 or 9, while two children assigned a 6, and one child
a 7. These ratings indicate that the children felt at ease and content.
It is noteworthy that none of the children reached the 10-point mark,
given that they live in a multimodal world surrounded by video and
gaming platforms, which are likely to exert a strong influence on their
preferences. It may be posited that the introduction of this novel
activity, coupled with the necessity to master the utilisation of
internet-based tools, has instilled a sense of unease and lack of
confidence amongst them.
\begin{figure}[htbp]
    \centering
    \begin{minipage}{.5\textwidth}
    \includegraphics[width=\textwidth]{fig01.png}
    \caption{Question 1. From 1 to 10, how much have you enjoyed
    the ambiance of the workshop?}
    \label{fig-01}
    \source{Owm elaboration.}
    \end{minipage}
\end{figure}

In question 2 (see \Cref{fig-02}), the respondents were asked about their
previous knowledge of AVT. Four of the participants were unaware of its
existence, while the remaining three were cognizant of it. These figures
were unanticipated, given that the children have access to audiovisual
products in three different languages. This may have led them to assume
that there is a process of translation behind that range of options. It
seems reasonable to posit that the most probable reason for this lack of
awareness is that, due to their age, they are accustomed to consuming
audiovisual content in those languages and have not considered the
processes involved.

\begin{figure}[htbp]
    \centering
    \begin{minipage}{.5\textwidth}
    \includegraphics[width=\textwidth]{fig02.png}
    \caption{Question 2. Did you know what audiovisual translation
    was? (green-yes, purple-no).}
    \label{fig-02}
    \source{Owm elaboration.}
    \end{minipage}
\end{figure}

The answers to question 3, if they had enjoyed learning the L2 by means
of the editing of videos (see \Cref{fig-03}) were also positive.
Three-quarters of the children, five, demonstrated a positive attitude
towards the methodology employed, while two of them expressed a negative
opinion. Once more, we may cite the necessity of learning something new
compulsory as the primary reason for their refusal. These results align
with the responses to questions 1, 5 and 6, which will be analysed
subsequently.

\begin{figure}[htbp]
    \centering
    \begin{minipage}{.5\textwidth}
    \includegraphics[width=\textwidth]{fig03.png}
    \caption{Question 3. Have you enjoyed learning English by
    means of video editing?}
    \label{fig-03}
    \source{Owm elaboration.}
    \end{minipage}
\end{figure}

In response to question 4, which pertains to the two modes implemented
in class and translation (see \Cref{fig-04}), it can be observed that data do
not provide a clear indication of the pupils\textquotesingle{}
preferences towards either mode. The preference for dubbing is indicated
by a single pupil, while the other two modes were selected by two pupils
each. It is noteworthy that two children selected translation, a written
activity that is not directly relevant to their lived experience, rather
than the other two modes, which are inherently more visual.

\begin{figure}[htbp]
    \centering
    \begin{minipage}{.5\textwidth}
    \includegraphics[width=\textwidth]{fig04.png}
    \caption{Question 4. Which tool have you liked the most?
    (Green: interlingual indirect translation of texts; purple: dubbing;
    blue: subtitling).}
    \label{fig-04}
    \source{Owm elaboration.}
    \end{minipage}
\end{figure}


In question 5, the children were asked about the process of language
acquisition. As illustrated in \Cref{fig-05}, all pupils demonstrated an
understanding that they would benefit from these educational
initiatives. It is noteworthy that five of the pupils awarded the
project an 8, a high rating that reflects their trust in DAT. One pupil
considered the quality to be satisfactory, while one rated it slightly
above average. Overall, the marks are deemed satisfactory, as none of
the responses were below the passing mark of 5. These responses are
consistent with those provided in question 2, in which two children
indicated a lack of knowledge regarding AVT.

\begin{figure}[htbp]
    \centering
    \begin{minipage}{.5\textwidth}
    \includegraphics[width=\textwidth]{fig05.png}
    \caption{Question 5. From 1 to 10, how much do you think you
    would learn?}
    \label{fig-05}
    \source{Owm elaboration.}
    \end{minipage}
\end{figure}

The final question (number 6) asked the pupils whether they would
like to engage in DAT activities within the classroom setting. Five
pupils responded in the affirmative, while the remaining two expressed
opposition to the proposal. However, their responses lacked sufficient
clarity or referenced their disapproval of the instructor rather than
the DAT activities themselves. It can be inferred that these two
students are the same individuals who, in question 1, rated the workshop
atmosphere as 6, who provided negative responses to question 3 regarding
their enjoyment along the process, and who rated the development as 6
and 7. This is somewhat surprising, given that these ratings are not
particularly low.

Question 6 (original answers in \Cref{annex-03})

\begin{itemize}
    \item \textbf{Pupil 1}: Yes, because I learn.
    \item \textbf{Pupil 2}: Yes, because I like making videos a lot.
    \item \textbf{Pupil 3}: Yes, because I have never tried it and I think it would be good.
    \item \textbf{Pupil 4}: Yes, because I like recording videos a lot.
    \item \textbf{Pupil 5}: Yes, because it is fun.
    \item \textbf{Pupil 6}: No, because I do not want X (a teacher) to appear.
    \item \textbf{Pupil 7}: No.
\end{itemize}
%
\section{Discussion}\label{sec-discussion}

In regard to the learning outcomes, it is not feasible to assess the
acquisition of the L2 content. However, we have identified recurring
errors in writing and pronunciation that illustrate potential areas for
guidance to facilitate pupil improvement in the language. Additionally,
the children committed mistales in pronunciation and intonation.
Although these issues require resolution, it is acknowledged that
mispronunciations of this nature are common among speakers of Basque and
Spanish, given the inherent challenges posed by the English vocalic
system for those with a limited number of vocalic phonemes.

Regarding the findings of the questionnaire, the outcomes are consistent
with our expectations. The pupils have demonstrated predisposition
towards this pedagogical approach, which integrates stimuli that align
with the fundamental principles of Montessori, namely the diverse
classroom environment to which they are accustomed. It is also
noteworthy that the method incorporates multimodality, which is a key
aspect of the world in which children live. The final products
demonstrate that pupils can successfully handle the L2. This is a
significant finding, as \textcite{alonso-perez2018} have
demonstrated that observing the final outcome and analysing the progress
that the students have achieved ``make everyone feel rewarded'' \cite[p. 21]{alonso-perez2018}, which implies an even stronger boost to motivation.

Conversely, children have demonstrated interest and engagement with the
DAT activities. A recurrent 5-2 parameter is evident when analysing
questions 3, 5 and 6. This signifies that five of the seven children
indicated a high level of enjoyment, with ratings of Bs and As, while
the remaining two children achieved a passing grade of C+. It is
accurate to conclude that the highest grade awarded was not an A+, which
can be attributed to the fact that this task was compulsory and required
the learning of internet tools, which served this purpose.

In examining the reasons for the implementation\textquotesingle s
success or failure, the indicators recur. It is presumed that the two
children who provided the lowest ratings were those who responded in the
negative. The issue is that the respondents did not provide reasons
related to their experiences with DAT activities. One respondent did not
provide any reasons, while the other provided a single negative response
regarding one of the teachers. This is a mishap, as their answers would
have been of great help for the research. Conversely, the remaining five
respondents provided positive feedback, citing enjoyment derived from
video recording, fun during implementation, and, most notably, two
respondents demonstrated appreciation for the activity\textquotesingle s
didactic potential, stating, "It would be good" and "I learn." This
latter response is particularly noteworthy, as it reflects a depth of
reflection that not all children possess. These statements are in
accordance with \textcite{fernandez-costeles2021} regarding the perceptions of
primary education pupils towards the didactic possibilities of DAT.

In light of the previous considerations, it can be posited that the
findings of this survey indicate a favourable outcome. In terms of
learning, this implementation has opened paths for the teacher to work
those aspects of language which have proven to be less developed in
learners. Conversely, with regard to motivation, the utilisation of DAT
has enhanced students motivation, as it is founded upon certain
customary activities among children, such as the recording of videos.
This results in a more enjoyable process of learning the L2, which
undoubtedly facilitates the work of both the teacher and the learners.
This research contributes to the existing body of literature on the
subject, building upon the findings of previous studies conducted by
scholars such as \textcite{neves2004language}, \textcite{talavan2009aplicaciones,talavan2010audiovisual}, \textcite{banos2015clipflair}, \textcite{talavan2015first}, \Textcite{BELTRAMELLO_2019}, \textcite{lertola2019audiovisual}, \textcite{talavan2022audiovisual}, \textcite{rodríguez-Arancón2023} and \textcite{talavan2024}.

\section{Conclusion}

This research has explored the relationship between the Montessori Method and DAT, demonstrating that both have a significant degree of overlap and that the latter can be integrated as an additional workshop within the Montessori classes. We encountered two main limitations. The first limitation is inherent to the methodology itself. Given that children do not take exams, it is not possible to adhere to the experimental scheme that would otherwise be appropriate for this type of research. Consequently, it has not been possible to measure the anticipated improvement in pupils' language acquisition.

A second limitation of the study is the low response rate to the satisfaction questionnaire, with only seven out of twenty-four students completing it. This is a potential issue, as the sample size is small and the responses are not fully representative. However, they do offer insights into a particular trend. Among the pupils who responded, we have been able to ascertain their motivations and levels of enjoyment regarding the implementation of the DAT. Five of them indicated that they found it motivating, while the other two expressed differing pedagogical opinions. The primary issue that emerged from the data was that learners lacked clarity regarding the specific applications of DAT.

In this particular instance, a class of twenty-four children at an educational institution that employs active methodologies and Basque as the L1 have demonstrated the efficacy of DAT from several perspectives. From the viewpoint of the teacher, the role differs from that of the traditional educator. DAT allows teachers to provide pupils with materials that align with their interests, thereby personalising their learning experience. Although pupils are permitted to select activities that exceed their current capabilities, the provision of effective scaffolding can facilitate their learning process. Furthermore, Montessori advocated the use of self-correcting materials, a concept that is exemplified by DAT. Audiovisual materials allow learners to listen to themselves and receive immediate feedback on their performance, which undoubtedly contributes to the improvement of their oral proficiency in the L2. DAT presents new opportunities for stimulating pupils, given its multimodal nature. It offers language in authentic contexts, which helps pupils comprehend the meaning of language in real situations. Thus, by providing a new lexical variety, children can expand their vocabulary to express themselves in different situations. Lastly, the integration of technology and multimodality has enabled learners to personalise their learning experience, allowing them to progress at their own pace and receive education that is specifically tailored to their needs. This is achieved through the creation of new teaching tools, which are based on interactive games, simulations or educational videos. In this regard, DAT has emerged as a highly valuable instrument in the context of this pedagogical approach, facilitating more engaging and efficacious learning, while fostering autonomous learning and self-regulation. 

Importantly, the implementation of DAT is in accordance with both the fundamental principles of the Montessori Method and current legislation. It has been demonstrated that DAT effectively implements the competencies and skills required by law with regard to the teaching of the L2. Furthermore, it facilitates the development of students' cognitive abilities, which is fundamental for lifelong learning.

Future research could encompass a longitudinal study to include an analysis of the role of multilingualism (learners work with three languages: Basque L1 of some pupils and of school, Spanish L1 of some children, and English as L2) in the general learning process and in the Montessori Method.

\section*{Funding}\label{sec-funding}
This work is funded by national funds through FCT – Fundação para a Ciência e a Tecnologia,
I.P., under the Scientific Employment Stimulus - Individual Call – [2022.06443.CEECIND] with DOI: 
10.54499/\allowbreak{}2022.06443.\allowbreak{}CEECIND/\allowbreak{}CP1720/\allowbreak{}CT0040 
and the CIDTFF Research Centre (projects UIDB/00194/2020 and UIDP/00194/2020).
It was also partly supported by OralGrab. Grabar vídeos y audios para ensenyar y aprender (PID2022-141511NB-100, Ministry of Science and Innovation), Gr@el - Grup de recerca sobre aprenentatge i ensenyament de llengües (SGR471, AGAUR), and by DEFINERS: Digital language learning of language teachers (TED2021-129984A-I00, Ministry of Science and Innovation, Spain).


\printbibliography
\label{sec-bib}
%conceptualization,datacuration,formalanalysis,funding,investigation,methodology,projadm,resources,software,supervision,validation,visualization,writing,review
\begin{contributors}[sec-contributors]
\authorcontribution{Liudmila Shafirova}[conceptualization,datacuration,formalanalysis,investigation,methodology,projadm,resources,visualization,writing,review]
\authorcontribution{Helena Araújo e Sá}[conceptualization,funding,funding,investigation,methodology,supervision,validation,writing,review]
\end{contributors}


\newpage
\appendix
\section{Codebooks of video viewing and production}\label{annex-a}
\begin{table}[htpb]
\centering
\footnotesize
\begin{threeparttable}
\begin{tabular}{p{1.5cm} c p{3cm} p{3cm} p{3cm}}
\caption{Categories of language learning through video consumption.}
\label{tab-07}\\
\toprule
Category & Freq. & Examples in Portuguese & Our translation to English & Departments \\
\midrule
Pronunciation & 7 & Tentando repetir pronúncias e palavras, principalmente com músicas. & Trying to repeat pronunciations and words, especially with songs. & Languages and Cultures, Biology \\
Vocabulary (everyday language, slang) & 23 & Enriquecimento de vocabulário novo, expressões de gíria e fonética. Novas expressões, pesquisa depois de ver o vídeo, entre outras. & Learning new vocabulary, slang expressions, and phonetics. New expressions, research after watching the video, etc. & Education and Psychology, Languages and Cultures, Biology, Environment, Engineering \\

Grammar & 1 & Aprendi questões de gramática. & I learnt about grammar. & Languages and Cultures \\
Oral comprehension & 9 & Comecei por ver com legendas na minha língua materna e fui me apercebendo do que significavam as palavras/expres\newline -sões ditas. Mais tarde via sem legendas e tentava entender. & I started watching it with subtitles in my mother tongue and realised what the words/expressions meant. Later, I watched without subtitles and tried to understand. & Biology, Education and Psychology, Languages and Cultures \\
Speaking skills & 2 & Muitas vezes em conversa com os colegas fico firme e convicta do que falo. & Often, when talking to colleagues, I'm firm and convinced of what I'm saying. & Mathematics \\

Language use in context & 5 & Vendo os vídeos é possível compreender um pouco mais o contexto e o uso das expressões em outra língua. & By watching the videos you can understand a little more about the context and the use of expressions in another language. & Physics, Engineering, Education and Psychology, Languages and Cultures \\
Cultural aspects & 7 & As séries permitem aprender outra língua e ao mesmo tempo os contextos sociais/culturais do país ou região onde é falada. Por exemplo, vi uma série chinesa que mostrava a vida de uma funcionária de uma empresa chinesa para aprender o chinês em contexto laboral. & TV shows allow you to learn another language and at the same time the social/cultural contexts of the country or region where it is spoken. For example, I saw a Chinese TV show that showed the life of an employee of a Chinese company learning Chinese in a work context. & Biology, Education and Psychology, Environment, Languages and Cultures, Geosciences \\
\bottomrule
\end{tabular}
\source{Own elaboration.}
\end{threeparttable}
\end{table}




\begin{table}[htpb]
\centering
\footnotesize
\begin{threeparttable}
\begin{tabular}{p{1.5cm} c p{3cm} p{3cm} p{3cm}}
\caption{Video production.}
\label{tab-08}\\
\toprule
Category & Freq. & Examples in Portuguese & Our translation to English & Departments \\
\midrule
Communicative skills & 2 & A expressão oral & Speaking & Languages and Cultures; Education and Psychology \\
Vocabulary & 4 & Consegui aprender novas palavras & I was able to learn new words & Education and Psychology; Languages and Cultures \\
Cultural aspects & 1 & A pesquisa necessária para produzir a maioria dos meus conteúdos permitiu-me alargar a minha visão à cerca da língua e cultura inglesa (\ldots) & The research required to produce most of my content has allowed me to widen my vision of the English language and culture (\ldots) & Biology \\
\bottomrule
\end{tabular}
\source{Own elaboration.}
\end{threeparttable}
\end{table}






\end{document}
