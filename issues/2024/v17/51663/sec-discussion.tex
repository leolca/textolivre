\section{Discussion}\label{sec-discussion}

Our findings reveal that video production in general is relatively uncommon among the respondents, with less than a quarter of them producing videos. Also, older students (more than 36 years old) were found to be as frequent producers of video content as the younger generation. According to our data, age was a decisive factor in the choice of several platforms to publish videos, where YouTube was the most popular platform among older students and TikTok among younger students.

The respondents’ objectives when producing videos mostly correspond to previous studies on language learning in the wild and the idea of “rewilding” language education, indicating that the most popular objective of the practice was entertainment \cite{thorne2021}. However, a considerable number of students made videos as university tasks, which could be a field for further investigation. For instance, exploring the types of videos students create at the university and the languages used in these tasks could be an interesting topic for research, especially as it questions the boundaries between formal and informal learning at the university level.

Also, according to our data, students suggest that video visualization contributes to a greater extent to language learning compared to video production. Several reasons could explain this finding. First and foremost, nearly all the respondents were active and frequent consumers of video content, while fewer people were posting videos frequently (max. 13\% of frequent posts on Instagram). This discrepancy could lead to a certain interpretation in which the respondents place a higher value on frequent video exposure than on occasional video production in a foreign language. Another potential factor is our use of a broad definition of video production, including videos which are very different in length and complexity, from Instagram stories to lengthy YouTube videos. The language input and its impact on language learning could considerably differ between these diverse formats, which could also explain the different perceptions of the students of its effectiveness.

Interestingly, the data indicated that being a current language learner was not a decisive factor in students' perceptions of language learning through video production. This may be due to the requirement in the Portuguese school system for all high school students to attend foreign language classes, providing most of them with an intentional language learning experience. Instead, differences in perceptions regarding the impact of video production on language learning were found among students from different departments. Students from the Department of Languages and Cultures had a much more positive perception of video production. This suggests that the pedagogical approaches in the Languages and Cultures department may influence students' perceptions of their video production practices, both formal and informal. Nevertheless, more data needs to be collected to substantiate this point.

Overall, these data encourage us not to assume that all students highly value their social media activity in terms of language learning, potentially challenging some of the results of previous case studies \cite{junior2020,zhang2022}, and indicating a necessity for further quantitative research to improve our understanding of how students perceive video content in their language learning journeys on a broader scale.

Moreover, students highlighted that both video viewing and production contributed to the improvement of their vocabulary. This correlates with data on all types of out-of-school language learning \cite{gonulal2019,soyoof2023}, and with some research on video-making in the English classroom, in which vocabulary improvement was also the most valued aspect by the students \cite{yeh2018}. In addition, in the case of video production, speaking was also mentioned by the students, which was the main focus of the in-class video-making investigation \cite{devana2021}.

Our findings also illustrate that video production on social media could be considered a valuable multiliteracy practice and an interactive environment, as it frequently includes searching for information, writing the descriptions, collaborating with other creators, and reading and interacting in the comment section \cite{cope2009}.

According to students self-reporting, these communicative actions, involving text rather than speaking, had more linguistic diversity than actual video production (11 languages used after video production). In addition, a lot of these actions were made in foreign languages, with English being the most frequent foreign language of use. The students also mention in the open-answer questions the importance of searching for information for their overall cultural and language learning. Hence, our findings indicate the possible significance of all the multifaceted practices that surround video production on social media. This perspective aligns with previous research on learning outside the classroom \cite{zhang2022}, and underlines the value of having a multiliteracy perspective in the classroom context \cite{yeh2018}.


\subsection{Limitations and future research}
The main limitations of this study are the small number of respondents and the low response rate, which constrains the ability to extrapolate the results to wider contexts. We also think that analysing the students’ perceptions of learning through a quantitative perspective was methodologically challenging due to the students’ individual preferences and perceptions. Moreover, the main contribution of this study is opening the discussion about the communicative actions that surround video-making and students’ perception of video production as a practice that contributes to language learning. Our analysis also indicated that students from the Language and Cultures department had a more positive view of video production. Also, the study brings attention to the practices in which students engage before and after making videos, and how these practices can be perceived as multilingual and beneficial by language learners. Finally, future studies could be focused specifically on language students and future language teachers. In such a case,  we could have some data on whether the university’s pedagogy influences students' perceptions of the videos they make in or out of the classroom.

