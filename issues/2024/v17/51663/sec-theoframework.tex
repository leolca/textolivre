\section{Theoretical framework}\label{sec-theoframework}

In this section, we will review prior research related to video-making
and its connection to language learning. Also, we will outline our
approach to analysing video-making, employing the framework of
multiliteracies.

\subsection{Video-making and language learning}\label{sub-sec-video-making}

For the last 20 years, research has expanded into diverse aspects of
learners\textquotesingle{} engagement with videos. This research delved
into how video production was used in the classroom to foster students'
creativity and enhance social competence \cite{fethi2018}, to
increase active knowledge construction and students' cooperation
\cite{nikitina2010}, digital competence \cite{yeh2018}, critical digital
literacies \cite{delosríos2018} or to improve speaking competence \cite{devana2021}.

Several studies have extensively explored speaking competence in the
context of video production, as videos could be useful for educators due
to the ability to revise and practice the students' speech \cite{cassany2021}. This competence encompasses a spectrum of skills
including pronunciation, comprehension, fluency, vocabulary, and grammar
of oral production \cite{devana2021}. Several studies highlighted
videos as a valuable instrument to practice speaking fluency or
motivation to speak \cite{tan2022}. For instance, \textcite{devana2021}, in a quasi-experimental study conducted at an Indonesian
university, demonstrated that the students who followed tasks of making
video blogs in English achieved significantly higher speaking scores and
were more motivated to speak in comparison to a control group.
Similarly, in another study employing qualitative analysis, students'
speaking competence scores also improved after producing TikTok videos
\cite{zaitun2021}. Nevertheless, some studies yielded more
mixed findings. For instance, a qualitative study conducted in a high
school in the Philippines found no improvement in students' perceptions
of their speaking ability through the use of TikTok \cite{asio2023}. The authors suggest that this may be because the tasks needed to
be more contextualised to be meaningful.

A systematic review of studies on TikTok indicates that in-school
research on video production has predominantly focused on speaking
competence \cite{tan2022}. Meanwhile, the research on
out-of-school language learning through video production highlights the
value of written interaction as a source of learning \cite{cassany2021}. For instance, when users post videos on social media,
they also interact with their followers by reading and writing comments,
frequently in a foreign language. These interactions could be important
for language learning, as illustrated in the study of \textcite{vazquez-calvo2023}, where the comments under TikTok and YouTube videos
were analyzed and described as a valuable discussion environment for
Korean language learning. In addition, a study on Instagram revealed
that reading in English was the competence highlighted by the students
when asked about their perceptions of language learning on Instagram
\cite{gonulal2019}.

Reading, writing, chatting, and translating were found to be frequent
practices in another area of out-of-school learning -- fandom, meaning
the organisation of affectionate consumers \cite{sauro2017}. In this line
of research, the focus has been on video-making through fan translation,
i.e., fansubbing, creating subtitles for videos \cite{tee2022} and fandubbing, and recording voice acting in a different language
\cite{shafirova2019}. This research on subtitling and dubbing
highlights the pivotal role of community and feedback provided in these
learning spaces \cite{zhang2016}. Fans engaged in these spaces
self-reported improvement in various foreign language skills including
translation, written interaction, writing, reading and speaking \cite{benson2015,shafirova2019}.

Moreover, plurilingualism and intercultural learning have been important
topics in both contexts of social media and fandom research. Case
studies focused on YouTube described some results on cultural identity
development by creating vlogs in a foreign country \cite{chang2019}, and learning multiple languages by fandubbing, singing and
interacting with followers \cite{zhang2022}. Also, on
TikTok, \textcite{vazquez2022} focused on the platform's
language learning opportunities in different languages (Russian, Chinese
and Italian). The study highlights TikTok's potential for plurilingual
and pluricultural learning through the variety of languages used in the
videos and the focus on different cultural stereotypes usually discussed
in the comments.

In conclusion, our analysis suggests that the main focus of research
regarding in-school video-making is somewhat different from the
out-of-school context. In-school studies primarily focus on the
development of speaking competence \cite{asio2023,devana2021,tan2022,zaitun2021},
whereas out-of-school data mostly centres around interaction,
translation, and intercultural competence \cite{gonulal2019,vazquez2022,zhang2022}. Also, the majority
of the out-of-school research results come from case studies, so it is
difficult to understand if video-making is a frequent and beneficial
practice among the majority of the students. This study focuses on
contributing to minimising these gaps and offering valuable insights
into the field of out-of-school language learning.

\subsection{Multiliteracies and video-making}\label{sub-sec-multiliteracies}

We look at video production from a multiliteracies perspective in which
digital multimodal communication can disrupt traditional text-based
literacies and should be incorporated into the classroom and curriculum
\cite{cope2015,thorne2013}. Digital multimodal tasks are
suggested to enhance the students' learning by bringing education closer
to the out-of-school reality \cite{ito2013}, positioning the
students at the centre of the learning process \cite{cope2009}, and motivating them to engage in the tasks \cite{muñoz-basols2019}.

Similarly to \textcite{yeh2018}, we look at online video-making as a
multiliteracy practice, hence we focus our attention on the fact that
this practice could be multilingual, multimodal and interactive. As a
social and multimodal practice, it could be complex and include various
modalities \cite{cope2009}. Video production can include
various activities during its preparation, such as searching for
information, analysing similar videos, writing descriptions or scripts,
or collaborating with others, as highlighted by \textcite{yeh2018}. Also, video
production can involve a variety of interactive activities after the
video is completed, such as reading and responding to comments. In
addition, as a possibly multilingual practice, all of the activities of
preparation for the video and management of the feedback after posting a
video can be done in different languages, providing a field for
engagement among different languages in several modalities.

With this study, we are focusing on adding knowledge to the video
practices of the students from a multiliteracy perspective including the
use of different languages, and various interactive activities
undertaken before and after video production.