\section{Materials and methods}\label{sec-materialsand}

In this section, the methodological aspects of the study will be
discussed including the procedure, participants, analysis and ethical
considerations. As an exploratory study, the majority of our inquiries
were inherently descriptive, intending to provide a wide picture of the
phenomenon and lay the groundwork for future studies. The main method of
the study is a quantitative questionnaire with some open questions
analysed qualitatively.

\subsection{Questionnaire}\label{sub-sec-questionnaire}

We started the data collection with a questionnaire which was
distributed to all the students in bachelor, master and PhD programs at
the University of Aveiro. Thus, on 17.02.2022 the Heads of all of the
departments sent the questionnaire to the students via email. Two
reminders were issued to obtain more responses. In total, 299 responses
were obtained from 11,932 students. Within this number, 212
questionnaires were fully answered, hence, the total N is 212. The low
rate of response to the questionnaire could be related to the fact that
there are many questionnaires distributed at the University so the
students are normally flooded with demands for participation in
research. The questionnaire consisted of two separate parts: (1) video
viewing (212 answers) and (2) video production (46 answers). In this
paper, we will mostly centre on the second part of the questionnaire,
which was answered by 46 students, with the use of the whole dataset
only for comparisons. This part has 14 questions including 4 yes/no
questions, 5 multiple choice questions, 3 matrix questions, 1 Likert
question and 1 open question. The questionnaire was validated by two
researchers in the areas of plurilingualism and technology-mediated
education who gave their written feedback. Moreover, a small pilot study
was conducted with 17 PhD students from the University of Aveiro,
Education and Psychology faculty, on the 28th of January 2022. It
consisted of a recorded one-hour focus-group session during which the
participants filled up the questionnaire, wrote their critique and
afterwards discussed it via ZOOM. Based on this feedback, subsequent
changes were made to it.

Also, the study received approval from the data protection committee of
the University of Aveiro. The consent forms were elaborated and
validated according to the Portuguese data protection laws. The consent
form appeared before the questionnaire, so the participants had to agree
with the conditions to fill in the questionnaire.

\subsection{Participants}\label{sub-sec-participants}

The participants of the questionnaire were mostly randomly sampled from
the students at the University. The main sampling criteria was to be
enrolled in the University. The descriptive statistics of the whole
questionnaire and the second part of the questionnaire regarding the age
and gender of the participants are presented in \Cref{tab-01}. Students'
mother tongues are shown in \Cref{tab-02}. These specific languages were
chosen based on the statistics of international students at the
university. \Cref{tab-03} shows how many students are currently learning an
additional language. Also, most of the departments are represented in
the video production part, however, most students are from the
departments of Education and Psychology (45.7\%), Biology (13\%) and
Languages and Cultures (10.9\%). It is important to mention that the
curriculum of the Education and Psychology department includes
Master\textquotesingle s programmes with language classes and language
didactics in the bachelor\textquotesingle s degree. In the Biology
department programmes, there are almost no obligatory language courses.


\begin{table}[htbp]
\centering
\begin{threeparttable}
\caption{Age and gender of the participants.}
\label{tab-01}
\begin{tabular}{llllll}
\toprule
\multicolumn{6}{c}{The whole questionnaire} \\
\midrule
\multicolumn{3}{c}{Gender}  & \multicolumn{3}{c}{How old are you?}\\
	& N & \% & & N& \% \\
Female & 149 & 70.3\% & From 18 to 23 & 93 & 43.9\% \\
Male & 61 & 28.8\% & From 24 to 35 & 75 &35.4\% \\
Prefer not to answer & 2 & 0.9\% & More than 36 & 44 & 20.8\%\\
\multicolumn{6}{c}{\rule{0pt}{4ex}The video production part of the questionnaire} \\
 & N & \% & & N & \% \\
Female & 26 & 56.5\% & From 18 to 23 & 13 & 28.3\% \\
Male & 20 & 43.5\% &From 24 to 35 & 15 & 32.6\% \\
Prefer not to answer & 0 & 0\% & More than 36 & 18 &39.1\% \\
\bottomrule
\end{tabular}
\source{Own elaboration. Information from the publication of \textcite{shafirova2023} is partly used.}
\end{threeparttable}
\end{table}

\begin{table}[htb]
\centering
\begin{threeparttable}
\caption{Mother tongues of the participants.}
\label{tab-02}
\begin{tabular}{llllll}
\toprule
\multicolumn{3}{>{\raggedright\arraybackslash}p{0.4\textwidth}}{Mother tongue(s) of the whole questionnaire} & 
\multicolumn{3}{>{\raggedright\arraybackslash}p{0.4\textwidth}}{Mother tongue(s) of the video production section of the questionnaire}\\
\midrule
	& \multicolumn{2}{l}{Responses} & & \multicolumn{2}{l}{Responses}\\
	& N & \% & & N & \% \\
Portuguese & 191 & 92.7\% & Portuguese & 39 & 82.6\%\\
Spanish & 8 & 3.9\% & Spanish & 2& 	4.3\% \\
English & 5 & 2.4\% & Russian & 1 & 2.2\% \\
Persian & 4 & 1.9\% & Persian & 1 & 2.2\% \\
Chinese & 2 & 1\% & Chinese & 1 & 2.2\% \\
French & 1 & 0.5\% & Other &3 & 6.6\% \\
Russian & 1 & 0.5\% & Total & 46 & 100.1\% \\
Other & 8 & 3.9\% &&& \\
\hline
Total & 212 & 106.8\%\tnote{1} &&&\\
\bottomrule
\end{tabular}
\begin{tablenotes}
\item[1] More than 100\% due to the overlap of mother tongues.
\end{tablenotes}
\source{Own elaboration. Information from the publication of \textcite{shafirova2023} is partly used.}
\end{threeparttable}
\end{table}

\begin{table}[htb]
\centering
\begin{threeparttable}
\caption{Participants as language learners.}
\label{tab-03}
\begin{tabular}{*6{l}}
\toprule
\multicolumn{6}{c}{Are you learning any language at the moment?}\\
\midrule
\multicolumn{3}{c}{The whole questionnaire} & \multicolumn{3}{c}{The video production part}\\
 & N & \% & & N & \% \\ Yes & 89 & 42\% & Yes & 15 & 32.6\% \\ No & 123 & 58\% & No & 31 & 67.4\%\\
\bottomrule
\end{tabular}
\source{Own elaboration.}
\end{threeparttable}
\end{table}

Following \Cref{tab-01}, we can see that in the video viewing section, we had
more students ranging from 18 to 23 years, than in the video production
section, in which the age of the participants is distributed more
towards the option ``more than 36'' (39.1\%). According to \Cref{tab-02}, most
of the students reported Portuguese as their mother tongue. It is
important to note that this question allowed multiple selections, though
only a few respondents chose multiple languages (hence, the percentages
in the ``total'' column slightly exceed 100\%). Moreover, \Cref{tab-03} shows
that slightly more participants are not currently learning any language,
particularly in the video production section of the questionnaire.

\subsection{Analysis}\label{sub-sec-analysis}

All of the answers provided were first analysed with descriptive
statistics of SPSS including frequencies. We also ran comparison or
cross-tabulation tests with age and gender regarding the platforms the
students chose, and regarding the students' mother tongues when
producing videos. When providing comparisons among age and gender
concerning the platforms the students chose, we considered the Pearson
chi-square test and significance.

We also analysed the open questions with a bottom-up content analysis.
We followed the topics of the responses which arrived from the data;
hence, the majority of the responses were centred on the descriptors of
language proficiency. Consequently, we centred on these descriptors
including pronunciation, vocabulary, grammar, comprehension,
communicative skills, language use in a context, and cultural aspects.
The codebook and examples of analysis are in Annex A.


