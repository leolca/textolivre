\section{Introduction}\label{sec-introduction}

In recent years, communication on social media has undergone major
changes, shifting towards being more multimodal and video-centred \cite{yus2023}. One example of this trend is TikTok, the most popular social
media channel among teenagers and young people, characterized by its
exclusive reliance on videos \cite{aslam2022}. Additionally, in 2017,
Instagram introduced video features, such as stories -- short
videos/images which disappear after 24 hours --, and in 2019, reels,
videos that are posted in the feed, which consist of images, videos and
music and can be edited directly on the platform \cite{instagram2020}.
Thus, video-form has become an important tool for interaction on social
media. In this study, we use the term
\textquotesingle videos\textquotesingle{} broadly, including all
multimodal artefacts that incorporate moving images. Such a broad
definition allows us to encompass a spectrum of social media video
options, such as TikTok videos, Instagram stories and reels, YouTube
videos, as well as streaming across various platforms.

Videos emerging on social media can be considered an equally significant
means of interaction to posts or tweets. This relevance notably extends
to research, particularly in the area of language education. For
instance, from a second language learning perspective, research
indicates that video-making can improve proficiency in speaking a
foreign language \cite{cassany2021}. Simultaneously, online
videos represent multimodal objects that can include different modes
such as writing, images, and video recordings. Through this array of
modes, videos can showcase multiple languages and cultural contexts,
making them a powerful interactive tool on social media. Consequently,
video-making is a promising object of research, particularly in the
analysis of language learning opportunities, linguistic and cultural
diversity, and multimodality.

Take TikTok, for example, where videos can be made in multiple modes and
languages, as users can speak one language, utilize audio from another,
and add text in yet another language \cite{vazquez2022}.
Social media users engage in multilingual expression for various reasons
including language teaching or learning, living in another country,
travelling or speaking in their mother tongue \cite{chang2019}.
Additionally, speakers of minority languages can actively promote their
linguistic heritage by making videos in their mother tongues \cite{stern2017}. Despite these benefits, language education, both at the basic and
higher education levels, still heavily relies on text-based
methodologies \cite{chik2021}. Given this context, we propose
taking a closer look at what is happening in the ``digital wild''.

Rewilding language education is a new and prominent idea based on the
learners' experiences outside the classroom and aimed at transforming
language classrooms into highly engaging and collaborative learning
environments \cite{thorne2021}. Students take on a proactive
role and bring their online and offline language usage into the
classroom, while teachers facilitate opportunities for such language
use. To achieve this rewilding of the classroom, teachers must gain
insight into the students' digital practices. Considerable research on
language learning in the wild describes cases of learners engaging in
sophisticated language practices, such as writing fanfiction \cite{black2006}, translating subtitles \cite{zhang2016}, or dubbing
popular shows \cite{shafirova2019}. These practices are
frequently socially constructed and embedded into the communities of
fans, passionate consumers and producers of popular culture content.

To give an example, \textcite{zhang2022} documented the story
of Miree, a Japanese and Korean popular culture fan engaged in the
practice of fandubbing, who became famous in a particular community by
publishing her videos online. By engaging in this activity, Miree
unintentionally improved her proficiency in several languages such as
Japanese, Korean and English. Her engagement in translation, video
production, search for information and interaction with fans especially
influenced her linguistic growth. The study reveals positive
self-reported language learning results driven by motivation and
feedback from the fan community. It contributes to our understanding of
self-motivation, language use, and autonomous learning while proposing
innovative ideas for the classroom. Nonetheless, a noticeable gap exists
in the literature, with a lack of studies employing a more quantitative
approach. So, it remains uncertain whether such cases of language
learning are commonplace and how many students show similar levels of
proactivity outside the classroom, actively posting videos in multiple
languages.

To address these questions within the Portuguese university context, the
objective of our exploratory study is to explore through a questionnaire
how students create videos in the interactive environment of social
media. This includes examining their use of different languages,
resources, platforms, and objectives, while also assessing
students\textquotesingle{} perceptions of how video practices connect
with language learning. The research questions of the study include: (1)
What social media platforms do students use to post videos and for what
purposes? (2) What languages do students use when making videos? (3) In
what practices do students engage before and after video production? (4)
What are the students' perceptions of the impact of video production in
comparison to video viewing on language learning?

Finally, this article is organized as follows: it begins with a
theoretical framework, focusing on previous studies on video-making and
language learning, along with the framework of multiliteracies. Next is
the methods section, which describes the organization of the
questionnaire, data gathering, participants, and data analysis. The
results section is structured around the research questions, emphasizing
the social media platforms and languages students use for video
production, the practices students engage in before and after making
videos, and students\textquotesingle{} perceptions of the impact of
video production on language learning. Lastly, the discussion section
examines the results in dialogue with previous studies, and the
conclusion section draws broader implications.