\begin{polyabstract}
\begin{abstract}
O estudo trata do problema do uso das Tecnologias de Informação e Comunicação (TIC) no processo de aprendizagem de Inglês para Fins Específicos (IFE) por futuros professores de matemática. O objetivo do estudo é a justificativa teórica, o desenvolvimento metódico e a verificação experimental da tecnologia de aplicação de TIC para aumentar o nível de competência comunicativa em língua inglesa orientada profissionalmente no processo de aprendizagem da disciplina de inglês para fins específicos por futuros professores de matemática em condições de guerra. Com base na análise da literatura científica foi dada uma definição e foram desenvolvidas competências comunicativas em língua inglesa orientada profissionalmente, critérios multinacionais de atividades criativas e seus níveis (baixo, médio, alto), o que permitiu verificar o nível da competência comunicativa em língua inglesa profissionalmente orientada de futuros professores de matemática antes e depois do treinamento experimental. Foi desenvolvida e testada a tecnologia de utilização de ferramentas TIC no processo de estudo do IFE, foram dados exemplos de utilização de ferramentas TIC em aulas de inglês online e analisados os resultados da formação experimental de futuros professores de matemática. Os resultados do estudo confirmaram a hipótese de que o nível de competência comunicativa na língua inglesa profissionalmente orientada de futuros professores de matemática aumentará, se no processo de ensino à distância de IFE e as ferramentas TIC forem usadas de acordo com a tecnologia educacional desenvolvida.
		
\keywords{Uso de TIC \sep Inglês para Fins Específicos (IFE) \sep Futuros professores de matemática \sep Tempo de guerra \sep Competência comunicativa em inglês profissionalmente orientada}
\end{abstract}
	
%\begin{english}
\begin{abstract}
			
The research addresses the use of information and communication technologies (ICT) in English for Specific Purposes (ESP) learning by pre-service mathematics teachers in Ukraine. The study aimed to provide theoretical justification, methodological development, and experimental verification of the application of ICT. It aimed to enhance the pre-service mathematics teachers’ professionally oriented English communicative competency (POECC) in ESP learning during wartime. The definition of POECC was based on a scientific literature analysis. The criteria for POECC were motivational, cognitive, operational, and creative, with three levels each (low, medium, and high). The authors developed and tested the use of virtual resources and digital tools in the process of learning ESP. They also provided examples of how ICT can be used in online English classes. The study evaluated the level of development of pre-service mathematics teachers’ POECC before and after experimental training, and analyzed the results of the training. The study confirmed the hypothesis that the development of pre-service mathematics teachers’ POECC level would increase during ESP online learning, even in wartime conditions, if virtual resources and digital tools were used in accordance with the developed educational technology.
		
\keywords{ICT \sep ESP \sep Pre-service mathematics teachers \sep Wartime \sep Professionally oriented English communicative competency} 
			
\end{abstract}
%\end{english}

\end{polyabstract}
