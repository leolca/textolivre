\section{Results and Discussion}\label{sec-results}

\Cref{tab-03} presents the general results of determining the level of
development of POECC of pre-service mathematics teachers in CG and EG
based on the described criteria, including motivational, operational,
cognitive, and creative aspects.

\begin{table}[htpb]
\centering
\begin{threeparttable}
\caption{Levels of development of POECC of pre-service teachers of mathematics at the pre-and post-stages of experimental training.}
\label{tab-03}
\begin{tabular}{ l l l l l l l l l l}
\toprule		
 &  & \multicolumn{8}{c}{Criteria}\\
\multicolumn{2}{c}{Levels of development} & \multicolumn{2}{c}{Motivational} & \multicolumn{2}{c}{Operational} & \multicolumn{2}{c}{Cognitive} & \multicolumn{2}{c}{Creative} \\
 & & \emph{CG} & \emph{EG} & \emph{CG} & \emph{ЕG} & \emph{CG} & \emph{ЕG} & \emph{CG} & \emph{ЕG} \\
\midrule
\multirow{2}{*}{Low level} & N & 26 & 27 & 21 & 20 & 22 & 20 & 29 & 30 \\
 & \% & 52 & 54 & 42 & 40 & 44 & 42 & 58 & 60 \\
\cmidrule{2-10}
\multirow{2}{*}{Average level} & N & 19 & 19 & 23 & 25 & 21 & 24 & 17 & 16 \\
 & \% & 38 & 38 & 46 & 50 & 42 & 48 & 34 & 32 \\
\cmidrule{2-10}
\multirow{2}{*}{High level} & N & 5 & 4 & 6 & 5 & 7 & 6 & 4 & 4 \\
 & \% & 10 & 8 & 12 & 10 & 14 & 12 & 8 & 8 \\
\cmidrule{2-10}
\multirow{2}{*}{Total} & N & 50 & 50 & 50 & 50 & 50 & 50 & 50 & 50 \\
 & \% & 100 & 100 & 100 & 100 & 100 & 100 & 100 & 100  \\
\bottomrule
\end{tabular}
\source{Own elaboration.}
\end{threeparttable}
\end{table}

The analysis of the results obtained from both the EG and CG showed a
generally low level of students' knowledge of professional English
terminology, as well as their ability to translate professional English
texts and comprehend oral professional speech. The students lacked both
professional foreign language skills and basic listening skills.
However, the results also indicated an average level of the students'
ability to analyze professional English sources. The students were able
to understand the English content, but they struggled to comprehend it
as a whole.

The analysis of the results of establishing the level of development of
POECC of pre-service mathematics teachers in CG and EG showed no
significant statistical difference. 11\% of CG prospective teachers and
9\% of EG future teachers demonstrated a high level of POECC
development, while 40\% (CG) and 42\% (EG) demonstrated an average
level, and 49\% (CG) and 49\% (EG) demonstrated a low level.

The $\chi^2$ criterion was used to determine statistical significance (with 2
degrees of freedom and a significance level of $p < 0.05$). 
The critical value,$\chi^2_{cr}$, was found to be 5.991, while the empirical
value, $\chi^2_{emp}$ , was 0.468. As $\chi^2_{emp} < \chi^2_{cr}$, the null hypothesis
$H_0$ was accepted, indicating that there was
no statistically significant difference in the level of development of
POECC between pre-service teachers of mathematics in the CG and EG and
that any differences observed were likely due to chance. It has been
concluded that the difference in results between the EG and CG at the
beginning of experimental training was statistically insignificant ($\chi^2_{emp} < \chi^2_{cr}$). 
The results obtained indicate that there were no
significant differences in the level of development of POECC between the
pre-service teachers of mathematics in the CG and EG.

Therefore, the assessment of pre-service mathematics teachers' level of
development in terms of their knowledge of basic English professional
terms, translation of professional texts, analysis of the content of
English professional sources, and ability to listen to oral professional
speech indicated a low or average level of proficiency in English for
Specific Purposes (ESP). The study confirmed the necessity of adjusting
the training program by incorporating the use of ICT tools in the
development of POECC for pre-service mathematics teachers. The
insufficiently formed level of POECC among pre-service mathematics
teachers may be related to their low awareness of the potential of using
English information resources in ESP online classes.

During the experimental training, CG students followed the traditional
ESP training program and curriculum, while EG students were taught using
virtual resources and digital tools.

Following the experimental training, a final test of ESP proficiency was
conducted for both groups according to four criteria. The experts
conducted an assessment using a 100-point scale for four components,
graded similarly to the beginning of experimental training. The level of
development of POECC among pre-service mathematics teachers after
experimental training (post-stage) was evaluated based on motivational,
cognitive, operational, and creative criteria. \Cref{tab-04} presents the
results of the dynamics of the development of POECC levels of
pre-service mathematics teachers according to motivational, cognitive,
operational, and creative criteria.

\begin{table}[!htpb]
\centering
\begin{threeparttable}
\caption{Generalized comparative analysis of the levels of POECC development of pre-service teachers of mathematics.}
\label{tab-04}
\begin{tabular}{*{9}{l}}
\toprule
\multicolumn{1}{p{3cm}}{\multirow{3}{=}{Levels of development of POECC}} & \multicolumn{4}{p{4cm}}{Test of ESP proficiency before experimental training} & \multicolumn{4}{p{4cm}}{Test of ESP proficiency after experimental training} \\
 &	\multicolumn{2}{c}{CG} & \multicolumn{2}{c}{ЕG} & \multicolumn{2}{c}{CG} & \multicolumn{2}{c}{ЕG}\\
 & N & \% & N & \% & N & \% & N & \% \\
\midrule
\multicolumn{9}{c}{Motivational criterion} \\
Low & 26 & 52 & 27 & 54 & 24 & 48 & 21 & 42 \\
Average & 19 & 38 & 19 & 38 & 21 & 42 & 22 & 44 \\
High & 5 & 10 & 4 & 8 & 5 & 10 & 7 & 14 \\			
\midrule
\multicolumn{9}{c}{Operational criterion}\\
Low & 21 & 42 & 20 & 40 & 19 & 38 & 16 & 32 \\
Average & 23 & 46 & 25 & 50 & 26 & 52 & 26 & 52 \\
High & 6 & 12 & 5 & 10 & 5 & 10 & 8 & 16 \\
\midrule
\multicolumn{9}{c}{Cognitive criterion}\\
Low & 22 & 44 & 20 & 40 & 23 & 46 & 17 & 34 \\
Average & 21 & 42 & 24 & 48 & 21 & 42 & 25 & 50 \\
High & 7 & 14 & 6 & 12 & 6 & 12 & 8 & 16 \\			
\midrule
\multicolumn{9}{c}{Creative criterion}\\
Low & 29 & 58 & 30 & 60 & 28 & 56 & 22 & 44 \\
Average & 17 & 34 & 16 & 32 & 18 & 36 & 20 & 40 \\
High & 4 & 8 & 4 & 8 & 4 & 8 & 8 & 16 \\
\bottomrule
\end{tabular}
\source{Own elaboration.}
\end{threeparttable}
\end{table}

The results indicate a positive change in the level of POECC development
of pre-service mathematics teachers in the experimental group (EG)
compared to the control group (CG). Specifically, the percentage of
pre-service mathematics teachers with a low level of POECC decreased to
11\% in the EG, while the percentage of those with an average level
increased to 4.5\% and those with a high level increased to 6.5\%.
Although the positive changes were not significant, they were
statistically large enough to be noticed. The results demonstrate that
the use of ICT in ESP learning significantly contributed to the
development of pre-service mathematics teachers' POECC. It is evident
that POECC development is a long process, and only minor positive
changes were observed during the one-semester study.

To ensure the reliability of the results, we formulated the null
hypothesis ($H_0$) that the difference in the
levels of POECC development of pre-service mathematics teachers in the
control group (CG) and experimental group (EG) was statistically
insignificant. The alternative hypothesis
($H_1$) stated that the difference in the
levels of POECC development of prospective mathematics teachers in the
CG and EG was statistically significant and reliable. The statistical
criterion $\chi^2$, as per \cref{eq-01}, was used to evaluate the homogeneity of
groups.



The critical value of the $\chi^{2}$ criterion was 5.991 with 2 degrees of freedom
and a significance level of $p < 0.05$. The empirical
value was calculated to be $\chi^2_{emp} = 9.156$. The criterion $\chi^{2}$ was used to determine
the statistical significance of the difference in the levels of POECC
development of pre-service teachers of mathematics in the CG and EG. As $\chi^2_{emp}$
was greater than $\chi^2_{cr}$, the null hypothesis $H_0$ was
rejected. This indicates that the difference in the levels of POECC
development between the two groups was statistically significant and had
a regular character.



The statistical analysis of the data from EG and CG revealed that the
difference between the values before and after the experimental training
in EG was significant, while the difference between the pre- and
post-training data in CG was insignificant (refer to Table 6).
	
The comparative result analysis of the averaged indicators of the levels of POECC development is shown in \Cref{tab-05}.
	
\begin{table}[!htpb]
\centering
\begin{threeparttable}
\caption{Comparative analysis of the average indicators of the levels of POECC.}
\label{tab-05}
\begin{tabular}{l l l l l}
\toprule
\multicolumn{1}{p{3cm}}{\multirow{3}{=}{Levels of development of POECC}} & \multicolumn{2}{p{3cm}}{before experimental training} & \multicolumn{2}{p{3cm}}{after experimental training} \\
& \emph{CG} & \emph{ЕG} & \emph{CG} & \emph{ЕG}  \\
& \emph{\%} & \emph{\%} & \emph{\%} & \emph{\%}  \\
\midrule
Low & 49 & 49 & 47 & 38 \\
Average & 40 & 42 & 43 & 46.5 \\
High & 11 & 9 & 10 & 15.5 \\
Total & 100 & 100 & 100 & 100 \\
\bottomrule
\end{tabular}
\source{Own elaboration.}
\end{threeparttable}
\end{table}
	
The results indicate a positive change in the level of POECC development
of pre-service mathematics teachers in the experimental group (EG)
compared to the control group (CG). Specifically, the percentage of
pre-service mathematics teachers with a low level of POECC decreased to
11\% in the EG, while the percentage of those with an average level
increased to 4.5\% and those with a high level increased to 6.5\%.
Although the positive changes were not significant, they were
statistically large enough to be noticed. The results demonstrate that
the use of ICT in ESP learning significantly contributed to the
development of pre-service mathematics teachers' POECC. It is evident
that POECC development is a long process, and only minor positive
changes were observed during the one-semester study.
	
To ensure the reliability of the results, we formulated the null
hypothesis ($H_0$) that the difference in the
levels of POECC development of pre-service mathematics teachers in the
control group (CG) and experimental group (EG) was statistically
insignificant. The alternative hypothesis
($H_1$) stated that the difference in the
levels of POECC development of prospective mathematics teachers in the
CG and EG was statistically significant and reliable. The statistical
criterion $x^{2}$, as per \Cref{formula-01}, was used to evaluate the homogeneity of groups.
	
The critical value of the $x^{2}$ criterion was 5.991 with 2 degrees of freedom
and a significance level of $p < 0.05$. The empirical
value was calculated to be $x_{emp}^{2}$ = 9.156. The $x^{2}$ criterion was used to determine
the statistical significance of the difference in the levels of POECC
development of pre-service teachers of mathematics in the CG and EG. As
$x_{emp}^{2}$ was greater than $x_{cr}^{2}$ , the null hypothesis $H_0$ was
rejected. This indicates that the difference in the levels of POECC
development between the two groups was statistically significant and had
a regular character.

The statistical analysis of the data from EG and CG revealed that the
difference between the values before and after the experimental training
in EG was significant, while the difference between the pre- and
post-training data in CG was insignificant (refer to \Cref{tab-06}).
	
\begin{table}[!htpb]
\centering
\begin{threeparttable}
\caption{$x^{2}$ Pearson test values.}
\label{tab-06}
\begin{tabular}{lll}
\toprule
\multirow{2}{*}{Groups} & \multicolumn{1}{p{3cm}}{\multirow{2}{=}{The calculated value of $x_{emp}^{2} $}} & $x_{cr}^{2} $ \\ 
 & & 0,05 \\
\midrule
\multicolumn{3}{c}{Before the experimental training} \\ 
ЕG and CG & 0,468 & 5,991 \\[0.3cm]
\multicolumn{3}{c}{After the experimental training} \\ 
ЕG and CG & 9,156 & 5,991 \\
\bottomrule
\end{tabular}
\source{Own elaboration.}
\end{threeparttable}
\end{table}
	
The data revealed that, following experimental training, students in the
experimental group (EG) demonstrated a higher level of development in
the area of POECC than students in the control group (CG). This was not
due to chance, but rather due to the advantages of implementing ICT in
ESP learning by pre-service mathematics teachers. The study confirmed
the hypothesis that the use of ICT in ESP learning is effective. The
technology was developed and implemented in experimental training.
Following the integration of ICT technology into education, there has
been a noticeable improvement in the development of pre-service
mathematics teachers' POECC. This is due to an
increased focus on the ability to understand and translate professional
English texts, analyze their content, and comprehend professional
speech. It is important to note that these skills require a solid
foundation in basic English professional terminology.

The obtained study results are consistent with previous research
conducted by Sutherland \emph{et al.} (2004), Awada \emph{et al.}
(2020), Røkenes and Krumsvik (2016) that proved the effectiveness of ICT
use in ESP learning and its positive impact on English communicative
competency. Rosa (2016), Gałan and Półtorak (2019) analyzed the digital
English courses on educational platforms, which included all necessary
learning materials and tools for checking completed tasks, and as well
argued that the level of English proficiency increased in the case of
ICT use in the ESP learning.


Due to the small sample size (N=100), the survey results cannot be
generalized to the entire population. Therefore, this study should be
considered as an exploratory investigation aimed at identifying possible
issues and trends for further research.

	
