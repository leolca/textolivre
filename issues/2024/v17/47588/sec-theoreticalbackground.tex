\section{Theoretical Background}\label{sec-theoreticalbackground}
	
POECC refers to the ability to effectively use English language skills in a specific professional context or field. It goes beyond general language proficiency and focuses on the specialized language, terminology, and communication strategies relevant to a particular profession or industry. This type of competency enables individuals to communicate accurately and fluently in English within their professional domain, whether it be business, healthcare, engineering, or any other field. It involves not only linguistic skills but also an understanding of the cultural and contextual aspects of communication within a specific professional setting. POECC requires expertise in English, subject-specific knowledge, and adherence to the norms and rules of the English-speaking professional community. Additionally, it requires the ability to conduct independent educational activities \cite{dmitrenko2020autonomous}.

The goals and learning outcomes of POECC are determined in two areas. Firstly, the motivational and personal-value sphere reflects the social and personal characteristics of pre-service mathematics teachers, including their abilities, interests, awareness, and desire for improvement and development. Secondly, the cognitive sphere contains knowledge of the content essence of POECC and includes the ability to effectively solve various problems in professional English communication. The operational (psychomotor) sphere involves demonstrating competency in various situations and developing a set of skills and abilities. The creative sphere reflects the personal and creative qualities of pre-service mathematics teachers, including their capacity for creative thinking and readiness for professionally oriented English communication \cite{dmitrenko2020autonomousA}.

Integrative relations between the study of mathematics and English for Specific Purposes (ESP) aim to improve knowledge in mathematics and develop the language skills necessary for professional practice. ESP involves the study of basic and specialized terminology, the formation of practical language knowledge and skills, and the preparation of students to apply professional knowledge to solve various tasks in foreign language practice.

\textcite[p. 2]{egloff1997languages} highlighted the importance of considering the individual’s wishes and motivation when learning foreign languages for professional purposes. This includes language needs in both professional training and practice, as well as personal everyday life. It also fosters an interest in foreign language communication in professional practice. Trim (1997) states that learning English for professional purposes is a complex and dynamic process that involves various factors such as goals, content, methods, and the roles of teachers and students. This process requires changes in language learning content and methods, such as a focus on communicative methods, which play a crucial role in personality development and cognitive skill enhancement. Additionally, the use of ICT in the learning process is also important.

The use of ICT tools in foreign language learning has been extensively studied by scholars. \textcite{ebadi2017exploring}, \textcite{hsu2017efl}, \textcite{jalali2014attitudes}, and \textcite{oz2015investigating} have all confirmed the significant role of ICT in foreign language learning, as well as the positive attitude of English as foreign language students towards computer-based language learning (CALL). \textcite{yao2016research}  reported that students had a positive attitude towards learning English through the Internet. \cite{hsu2017efl} found that students had a favorable attitude toward the use of e-textbooks in foreign language lessons. \textcite{alshabeb2018study} demonstrated that ICT played a prominent role in EFL learning. Furthermore, \textcite{sutherland2004transforming}, \textcite{rosa2016experiences}, and \textcite{awada2020effect} have revealed the benefits that teachers could gain from the use of ICT in English classes. \citeauthor{rokenes2016prepared}’s (\citeyear{rokenes2016prepared}) research has also shown that ICT could assist teachers in achieving pedagogical goals and have a positive impact on students’ speaking and foreign language skills. In their \citeyear{lai2018understanding} study, \citeauthor{lai2018understanding} investigated the experiences of students learning foreign languages outside of the classroom, as well as the factors that affect distance learning. Similarly, \textcite{lee2018when} explored the issue of digital foreign language learning, with a focus on form and content and its impact on foreign language communicative competency.

\textcite{janowska2017mediation} described the use of electronic accounts, emails, online forums, and communication chats for foreign language classes. \textcite{budiman2020ict} explored students’ engagement with a foreign language teacher’s educational website. \textcite{huzairin2020technology} studied the use of smartphones in online EFL classes. \textcite{janowska2017mediation} studied the use of podcasts in online foreign language classes, while \textcite{galan2019modern} described methods for working with electronic dictionaries in such classes. \textcite{nocentini2015anti} investigated the possibilities of diversifying online foreign language lessons through the use of interactive computer games and provided recommendations for introducing game-based learning elements. \textcite{carrio2015collaborative} examined the effect of educational games and interactive tasks on the enhancement of students’ listening, speaking, reading, and writing skills in online foreign language lessons.

The analysis of educational and methodological studies has led to the generalization that the use of ICT tools in ESP learning has positive effects. The use of authentic materials in language teaching has several benefits, including motivating learning, enabling teachers to apply an individual approach, promoting the development of students’ independence, increasing awareness of foreign languages and cultures, improving language competency through exposure to various types of texts, offering up-to-date and relevant learning material that meets the interests and needs of students, and providing an opportunity for students to check and self-check their acquired knowledge, skills, and abilities.

The implementation and use of ICT in ESP classes can be effective and helpful as technical visual and auditory aids. It is important to note that these benefits are objective and supported by evidence. They can also serve as auxiliary tools for the educational and cognitive activities of students. Additionally, they can increase students’ motivation for ESP learning and provide a quick and effective means of assessing and controlling their knowledge, abilities, and skills.

Furthermore, the use of ICT in ESP learning can enhance interactive and communicative activities, promote individualized educational activities for students, optimize the assimilation of language structures and grammatical rules, and overcome the monotony of classes in the formation of POECC.
