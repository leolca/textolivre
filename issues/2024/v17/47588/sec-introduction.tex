\section{Introduction}\label{sec-introduction}

The urgency of improving the professional training of modern teachers is determined by new social conditions. Special importance is attached to the preparation of professionals for foreign language communication. It should be noted that foreign language proficiency is a key condition for processing information sources, research, improving professional education, professional interaction with foreign colleagues, as well as student and teacher mobility. The opportunities for modern teachers to improve their professional training are significantly reduced without proper foreign language training. Improper foreign language training limits their access to foreign information sources, and their ability to work with computer technologies that require foreign language programs and editors. Learning foreign languages aims to enhance an individual’s self-awareness and self-identification of an individual through the enrichment of experience, and understanding of cross-cultural and linguistic differences \cite{council2001common} % (Council of Europe, 2003, p. 6) Council of Europe, 2001, p. 6 

Currently, a foreign language is an essential component of the overall teacher training system. Its purpose is to assist pre-service teachers in mastering their specialty as a fundamental aspect of their professional competency. The main objective of English for Specific Purposes (ESP) learning is to cultivate an individual who can coexist with representatives of different languages and cultures, for the purpose of self-development and self-realization. Therefore, it is essential to teach foreign language communication in professionally significant situations, rather than simply as a sign system with a set of typical phrases. In the process of ESP acquisition, pre-service teachers develop the level of professionally oriented English communicative competency (POECC). The concept of POECC refers to the ability and readiness of pre-service mathematics teachers to engage in professional communication in English. It includes the active and practical use of English to express opinions related to their professional activities.

Taking into consideration the war is ongoing in Ukraine, the following peculiarities of teaching a foreign language to prospective teachers shall be highlighted:

 \begin{enumerate*}[label=\arabic*)] 
		\item distance learning, which involves synchronous and asynchronous forms of remote online learning; 
		\item the number of English classes is limited by the program, and at the same time there are high requirements for mastering a foreign language at the end of the course; 
		\item different IQ levels of applicants entering a non-language pedagogical institution of higher education; 
		\item implementation of the specific course “English for professional communication” in the training program of pre-service teachers of mathematics.
\end{enumerate*}

During the war caused by Russia’s aggression against Ukraine, all Ukrainian educational institutions were compelled to switch to\textbf{\textit{ }}remote online learning. While this allowed students to continue their studies regardless of their location, it also gave rise to new educational challenges. Therefore, the development of new approaches to remote online learning became crucial. Online language learning has led to the development and implementation of various virtual resources, digital tools, and technologies for their use. The use of information and communicative technology (ICT) tools has become essential for successfully mastering a foreign language.

Therefore, the implementation of remote online learning due to the state of war in the country compels educators to seek methods, forms, and tools, including ICT, that would facilitate the acquisition of ESP and the development of POECC.

The \textit{aim} of this study is to develop and experimentally verify the technology of ICT use to enhance POECC of pre-service mathematics teachers in remote online learning of ESP in the wartime. The study \textit{hypothesizes} that the level of POECC development of pre-service teachers of mathematics will increase if ICT tools are used in online ESP learning in accordance with the developed educational technology.
