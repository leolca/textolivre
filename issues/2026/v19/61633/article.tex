% !TEX TS-program = XeLaTeX
% use the following command:
% all document files must be coded in UTF-8
\documentclass[spanish]{textolivre}
% build HTML with: make4ht -e build.lua -c textolivre.cfg -x -u article "fn-in,svg,pic-align"
\usepackage{textcomp}

\journalname{Texto Livre}
\thevolume{19}
%\thenumber{1} % old template
\theyear{2026}
\receiveddate{\DTMdisplaydate{2025}{9}{7}{-1}} % YYYY MM DD
\accepteddate{\DTMdisplaydate{2025}{11}{5}{-1}}
\publisheddate{\today}
\corrauthor{Antonio Hernández Fernández}
\articledoi{10.1590/1983-3652.2026.61633}
%\articleid{NNNN} % if the article ID is not the last 5 numbers of its DOI, provide it using \articleid{} commmand 
% list of available sesscions in the journal: articles, dossier, reports, essays, reviews, interviews, editorial
\articlesessionname{articles}
\runningauthor{Hernández Fernández y Barros Camargo} 
%\editorname{Leonardo Araújo} % old template
\sectioneditorname{Hugo Heredia Ponce~\orcid{0000-0003-3657-1369}}
\layouteditorname{Saula Cecília~\orcid{0009-0006-3069-8480}}

\title{Innovación tecnológica en neuropedagogía: estudio exploratorio sobre la neuroimagen como herramienta para la atención temprana en el Trastorno del Espectro Autista}
\othertitle{Inovação tecnológica em neuropedagogia: estudo exploratório sobre a neuroimagem como ferramenta para o atendimento precoce no Transtorno do Espectro Autista}
% if there is a third language title, add here:
\othertitle{Technological innovation in neuropedagogy: exploratory study on neuroimaging as a tool for early care in Autism Spectrum Disorder}

\author[1]{Antonio Hernández Fernández~\orcid{0000-0002-7807-4363}\thanks{Email: \href{mailto:antonio.hernandez@ujaen.es}{antonio.hernandez@ujaen.es}}}
\author[2]{Claudia de Barros Camargo~\orcid{0000-0002-2286-8674}\thanks{Email: \href{mailto:claudia.barros@edu.uned.es}{claudia.barros@edu.uned.es}}}
\affil[1]{Universidad de Jaén, Andalucía, España.}
\affil[2]{Universidad Nacional de Educación a Distancia (UNED), Madrid, España.}

\addbibresource{article.bib}
% use biber instead of bibtex
% $ biber article

% used to create dummy text for the template file
\definecolor{dark-gray}{gray}{0.35} % color used to display dummy texts
\usepackage{lipsum}
\SetLipsumParListSurrounders{\colorlet{oldcolor}{.}\color{dark-gray}}{\color{oldcolor}}

% used here only to provide the XeLaTeX and BibTeX logos
\usepackage{hologo}

% if you use multirows in a table, include the multirow package
\usepackage{multirow}

% provides sidewaysfigure environment
\usepackage{rotating}

% CUSTOM EPIGRAPH - BEGIN 
%%% https://tex.stackexchange.com/questions/193178/specific-epigraph-style
\usepackage{epigraph}
\renewcommand\textflush{flushright}
\makeatletter
\newlength\epitextskip
\pretocmd{\@epitext}{\em}{}{}
\apptocmd{\@epitext}{\em}{}{}
\patchcmd{\epigraph}{\@epitext{#1}\\}{\@epitext{#1}\\[\epitextskip]}{}{}
\makeatother
\setlength\epigraphrule{0pt}
\setlength\epitextskip{0.5ex}
\setlength\epigraphwidth{.7\textwidth}
% CUSTOM EPIGRAPH - END

% to use IPA symbols in unicode add
%\usepackage{fontspec}
%\newfontfamily\ipafont{CMU Serif}
%\newcommand{\ipa}[1]{{\ipafont #1}}
% and in the text you may use the \ipa{...} command passing the symbols in unicode

% LANGUAGE - BEGIN
% ARABIC
% for languages that use special fonts, you must provide the typeface that will be used
% \setotherlanguage{arabic}
% \newfontfamily\arabicfont[Script=Arabic]{Amiri}
% \newfontfamily\arabicfontsf[Script=Arabic]{Amiri}
% \newfontfamily\arabicfonttt[Script=Arabic]{Amiri}
%
% in the article, to add arabic text use: \textlang{arabic}{ ... }
%
% RUSSIAN
% for russian text we also need to define fonts with support for Cyrillic script
% \usepackage{fontspec}
% \setotherlanguage{russian}
% \newfontfamily\cyrillicfont{Times New Roman}
% \newfontfamily\cyrillicfontsf{Times New Roman}[Script=Cyrillic]
% \newfontfamily\cyrillicfonttt{Times New Roman}[Script=Cyrillic]
%
% in the text use \begin{russian} ... \end{russian}
% LANGUAGE - END

% EMOJIS - BEGIN
% to use emoticons in your manuscript
% https://stackoverflow.com/questions/190145/how-to-insert-emoticons-in-latex/57076064
% using font Symbola, which has full support
% the font may be downloaded at:
% https://dn-works.com/ufas/
% add to preamble:
% \newfontfamily\Symbola{Symbola}
% in the text use:
% {\Symbola }
% EMOJIS - END

% LABEL REFERENCE TO DESCRIPTIVE LIST - BEGIN
% reference itens in a descriptive list using their labels instead of numbers
% insert the code below in the preambule:
%\makeatletter
%\let\orgdescriptionlabel\descriptionlabel
%\renewcommand*{\descriptionlabel}[1]{%
%  \let\orglabel\label
%  \let\label\@gobble
%  \phantomsection
%  \edef\@currentlabel{#1\unskip}%
%  \let\label\orglabel
%  \orgdescriptionlabel{#1}%
%}
%\makeatother
%
% in your document, use as illustraded here:
%\begin{description}
%  \item[first\label{itm1}] this is only an example;
%  % ...  add more items
%\end{description}
% LABEL REFERENCE TO DESCRIPTIVE LIST - END


% add line numbers for submission
%\usepackage{lineno}
%\linenumbers

\begin{document}
\maketitle

\begin{polyabstract}
\begin{abstract}
Este estudio analiza el papel de la innovación tecnológica en la atención temprana del Trastorno del Espectro Autista (TEA), con un enfoque en la contribución de la neuropedagogía y la neuroimagen. A través de un diseño mixto con 117 participantes, se aplicó un cuestionario tipo Likert de 20 ítems en cuatro dimensiones (neuropedagogía, TEA, atención temprana y neuroimagen), complementado con un grupo focal y estudios de caso con niños de 1 a 3 años. Los resultados cuantitativos evidenciaron correlaciones significativas entre todas las dimensiones ($r = 0.592 - 0.743$; $p < 0.01$), siendo la comprensión del TEA y la neuropedagogía los predictores más fuertes de la valoración de la atención temprana. Los análisis cualitativos reforzaron estos hallazgos, destacando el potencial de la neuroimagen para personalizar intervenciones y facilitar una identificación más precoz de patrones asociados al TEA, aunque también señalaron limitaciones prácticas relacionadas con su costo y complejidad. Los estudios de caso mostraron patrones específicos de activación cerebral que sugieren la viabilidad de diseñar intervenciones más ajustadas a las necesidades individuales. En conclusión, la integración de la neuroimagen como recurso innovador en la neuropedagogía representa una oportunidad transformadora para mejorar la detección, personalización y eficacia de la atención temprana en niños con TEA, consolidando un enfoque interdisciplinar basado en evidencia científica y tecnológica.

\keywords{Innovación tecnológica\sep Neuropedagogía\sep Trastorno del Espectro Autista\sep Atención temprana\sep Neuroimagen}
\end{abstract}

\begin{portuguese}
\begin{abstract}
Este estudo analisa o papel da inovação tecnológica na intervenção precoce no Transtorno do Espectro Autista (TEA), com foco nas contribuições da neuropedagogia e da neuroimagem. Utilizando um desenho de métodos mistos com 117 participantes, foi aplicado um questionário tipo Likert de 20 itens em quatro dimensões (neuropedagogia, TEA, intervenção precoce e neuroimagem), complementado por um grupo focal e estudos de caso com crianças de 1 a 3 anos. Os resultados quantitativos evidenciaram correlações significativas entre todas as dimensões ($r = 0,592 - 0,743$; $p < 0,01$), sendo a compreensão do TEA e da neuropedagogia os preditores mais fortes da valorização da intervenção precoce. As análises qualitativas reforçaram esses achados, destacando o potencial da neuroimagem para personalizar intervenções e facilitar a identificação mais precoce de padrões associados ao TEA, embora também tenham apontado limitações práticas relacionadas ao custo e à complexidade. Os estudos de caso mostraram padrões específicos de ativação cerebral que sugerem a viabilidade de elaborar intervenções mais ajustadas às necessidades individuais. Em conclusão, a integração da neuroimagem como recurso inovador na neuropedagogia representa uma oportunidade transformadora para aprimorar a detecção, personalização e eficácia da intervenção precoce em crianças com TEA, consolidando uma abordagem interdisciplinar baseada em evidências científicas e tecnológicas.

\keywords{Inovação tecnológica\sep Neuropedagogia\sep Transtorno do Espectro Autista\sep Atenção precoce\sep Neuroimagem}
\end{abstract}
\end{portuguese}

\begin{english}
\begin{abstract}
This study examines the role of technological innovation in early intervention for Autism Spectrum Disorder (ASD), with a focus on the contributions of neuropedagogy and neuroimaging. Using a mixed-methods design with 117 participants, a 20-item Likert-type questionnaire was applied across four dimensions (neuropedagogy, ASD, early intervention, and neuroimaging), complemented by a focus group and case studies with children aged 1 to 3 years. Quantitative results revealed significant correlations among all dimensions ($r = 0.592 - 0.743$; $p < 0.01$), with understanding of ASD and neuropedagogy emerging as the strongest predictors of early intervention assessment. Qualitative findings reinforced these results, highlighting the potential of neuroimaging to personalize interventions and enable earlier identification of patterns associated with ASD, while also noting practical limitations related to cost and complexity. Case studies further demonstrated specific patterns of brain activation, suggesting the feasibility of designing interventions more closely tailored to individual needs. In conclusion, the integration of neuroimaging as an innovative resource in neuropedagogy represents a transformative opportunity to enhance the detection, personalization, and effectiveness of early intervention in children with ASD, consolidating an interdisciplinary approach grounded in scientific and technological evidence.

\keywords{Technological innovation\sep Neuropedagogy\sep Autism Spectrum Disorder\sep Early intervention\sep Neuroimaging}
\end{abstract}
\end{english}
% if there is another abstract, insert it here using the same scheme
\end{polyabstract}

\section{Introducción}\label{sec-intro}
La neuropedagogía, como disciplina emergente que integra los hallazgos de la neurociencia con las prácticas educativas, está revolucionando la manera en que comprendemos los procesos de enseñanza y aprendizaje. Este campo, aún en consolidación, aporta nuevas claves para diseñar intervenciones pedagógicas más eficaces y personalizadas, especialmente en el ámbito de la educación inclusiva. En el caso del Trastorno del Espectro Autista (TEA), la neuropedagogía ofrece un marco valioso para comprender las particularidades neurobiológicas que caracterizan a esta condición y para generar entornos de aprendizaje que respondan a sus necesidades específicas. El TEA, como condición neurobiológica heterogénea, afecta de forma significativa la comunicación social y se manifiesta mediante conductas repetitivas y patrones restrictivos, lo que plantea desafíos complejos en la intervención educativa. En este contexto, la atención temprana se configura como un pilar esencial, pues aprovechar la plasticidad cerebral en los primeros años de vida permite ampliar las oportunidades de desarrollo y modificar trayectorias que, sin intervención, podrían derivar en mayores limitaciones en la infancia y la adultez.

La innovación educativa no se limita únicamente a la creación de nuevas metodologías, sino que también implica la integración de tecnologías avanzadas capaces de ofrecer información precisa sobre el funcionamiento cerebral. Entre ellas, la neuroimagen se ha consolidado como un recurso emergente de gran valor para la neuropedagogía, al permitir identificar patrones de activación neuronal relacionados con el aprendizaje, la comunicación y la interacción social. Diversas investigaciones recientes han demostrado este potencial de la neuroimagen en el ámbito educativo. \textcite{libero2015} evidenciaron que las técnicas multimodales de fMRI y EEG permiten mapear la conectividad funcional asociada a la atención conjunta y al procesamiento social en niños con TEA. De forma complementaria, \textcite{dcouto2023} confirmaron que el uso de algoritmos de aprendizaje profundo aplicados a imágenes cerebrales mejora la detección temprana de perfiles cognitivos diferenciados. Asimismo, \textcite{HallidayEtAl2024Autism} subraya que la neuroimagen educativa facilita la comprensión dinámica de los procesos cognitivos durante la interacción social y el aprendizaje, lo que respalda su valor pedagógico en contextos naturales. Su uso en la infancia, y en particular en la detección temprana del TEA, abre nuevas posibilidades para diseñar intervenciones basadas en evidencia, más ajustadas a los perfiles individuales y más eficaces en términos de impacto educativo y clínico. Así, la neuroimagen se posiciona como un puente entre la investigación científica y la práctica educativa, contribuyendo a una pedagogía sustentada en datos objetivos que complementan la observación clínica tradicional. En este marco de convergencia entre ciencia y educación, surge la necesidad de explorar herramientas capaces de vincular directamente la actividad cerebral con los procesos de enseñanza y aprendizaje. La neuroimagen responde a esta demanda, al ofrecer evidencia empírica sobre la dinámica cerebral durante la adquisición de habilidades cognitivas y sociales \cite{wang2024, libero2015}. De ahí que su aplicación en el contexto del Trastorno del Espectro Autista (TEA) represente un avance crucial para comprender de manera más profunda las bases neurobiológicas del aprendizaje y diseñar intervenciones más personalizadas.

El presente estudio se propone analizar la neuroimagen como recurso innovador dentro de la neuropedagogía, evaluando su impacto en la atención temprana del TEA. Para ello, se desarrolló un diseño metodológico mixto con 117 participantes, que incluyó familias, docentes y expertos, aplicando un cuestionario tipo Likert de 20 ítems distribuido en cuatro dimensiones -- neuropedagogía, TEA, atención temprana y neuroimagen --, además de un grupo focal y estudios de caso en niños de 1 a 3 años. Los resultados muestran correlaciones significativas entre todas las dimensiones y confirman que la comprensión del TEA y la neuropedagogía son predictores sólidos de la valoración de la atención temprana. Los análisis cualitativos y los casos con neuroimagen refuerzan estos hallazgos, al evidenciar patrones específicos de activación cerebral que pueden guiar intervenciones personalizadas, aunque también revelan desafíos éticos y prácticos, especialmente vinculados al costo y la complejidad tecnológica.

En definitiva, esta investigación destaca el potencial transformador de integrar la innovación tecnológica en la neuropedagogía, situando a la neuroimagen como un recurso estratégico para la atención temprana en el TEA. Al combinar ciencia, pedagogía y tecnología, se abre un camino prometedor hacia la creación de entornos de aprendizaje más inclusivos y personalizados, capaces de mejorar de forma significativa la calidad de vida de los niños con TEA y sus familias. Al mismo tiempo, se invita a reflexionar sobre la necesidad de establecer protocolos estandarizados y marcos éticos sólidos que aseguren una aplicación responsable de estas herramientas. La neuroimagen aplicada a la neuropedagogía no solo representa un avance científico, sino también una oportunidad para transformar la educación desde sus bases, consolidando un enfoque interdisciplinar y basado en evidencia que puede redefinir la intervención en el TEA desde las etapas más tempranas del desarrollo.

Con todo lo expuesto, esta investigación, parte de un diseño metodológico mixto que integra perspectivas cuantitativas y cualitativas. El estudio se realizó conforme a la Declaración de Helsinki y las directrices éticas de la \textcite{APA2017EthicalPrinciples}. Se enmarca en la declaración de la Comisión de Ética de la Universidad de Jaén (Ref. JUL.22/4.LÍNEA, 20 de julio de 2022), en el marco de la línea Neurociencia, neuroeducación y neurodidáctica. Todos los participantes firmaron consentimiento informado y se garantizó la confidencialidad y protección de los datos conforme al Reglamento (UE) 2016/679.


\section{Marco teórico}
En las últimas décadas, la educación ha experimentado una transformación silenciosa marcada por la irrupción de la neurociencia en los procesos pedagógicos. La neuropedagogía, entendida como la aplicación sistemática de hallazgos neurocientíficos al diseño y desarrollo de estrategias educativas, se ha consolidado como un campo emergente con un potencial transformador. Su premisa es tan simple como revolucionaria: comprender mejor cómo funciona el cerebro para poder enseñar de forma más eficaz. \textcite{shvarts2024} subrayan que la integración de fundamentos neurocientíficos en la formación docente no solo potencia la comprensión de los procesos de aprendizaje, sino que también permite estructurar ambientes educativos sensibles a la diversidad cognitiva, emocional y social. Esta visión adquiere un relieve particular en el caso del TEA, condición caracterizada por alteraciones en la comunicación social, comportamientos repetitivos y variabilidad neurobiológica significativa. \textcite{tokuhamaespinosa2011} definió la neuropedagogía como la aplicación directa de la investigación neurocientífica a la educación, una idea que sigue plenamente vigente, aunque hoy se ve ampliada por las innovaciones tecnológicas que permiten observar y comprender el cerebro en acción. \textcite{chen2024}, en una revisión sobre plasticidad neuronal en el TEA, muestra que intervenciones educativas tempranas, fundamentadas en la reorganización de circuitos cerebrales, tienen el potencial de modificar trayectorias de desarrollo a largo plazo. En este sentido, la neuropedagogía aplicada al TEA se perfila como un marco integrador que reconoce la neurodiversidad, fomenta la inclusión y orienta la práctica educativa hacia la personalización.

La interdisciplinariedad entre la neuropedagogía, la tecnología educativa y las ciencias cognitivas se ha convertido en un eje de investigación clave en la educación contemporánea. Autores como \textcite{mayer2020} y \textcite{moreno2022} subrayan que el aprendizaje mediado por tecnología no puede comprenderse sin atender a los procesos cognitivos subyacentes de la atención, la memoria y la carga mental. Desde esta perspectiva, la neuroimagen aporta evidencia empírica que complementa los modelos cognitivos tradicionales, al mostrar cómo las tecnologías interactivas activan redes cerebrales específicas implicadas en la comprensión y la transferencia del conocimiento. Además, \textcite{bransford2021} destacan que la integración de principios de la ciencia cognitiva con recursos tecnológicos permite diseñar entornos de aprendizaje adaptativos, capaces de responder a las diferencias individuales en los estilos y ritmos de aprendizaje. Este enfoque converge con la neuropedagogía al situar al cerebro como centro de análisis y la tecnología como mediadora del proceso educativo.

La aplicación de principios neuropedagógicos en la educación de niños con TEA trasciende la simple introducción de nuevas metodologías. Se trata de configurar entornos de aprendizaje completos e integrales que estimulen la plasticidad cerebral y favorezcan el desarrollo de habilidades cognitivas, sociales y comunicativas. \textcite{pradeep2024} sostiene que el conocimiento sobre plasticidad, memoria y procesamiento cognitivo puede traducirse en prácticas educativas transformadoras cuando se apoya en recursos tecnológicos como la neuroimagen y la inteligencia artificial. En este sentido, la neuropedagogía no se limita a proporcionar técnicas, sino que implica un cambio cultural y epistemológico en la forma de concebir la enseñanza: el aula como un espacio donde se interviene directamente en la arquitectura cerebral a través de experiencias educativas significativas. En el caso del TEA, esta perspectiva se traduce en estrategias que aprovechan fortalezas específicas como la memoria visual o la atención al detalle, al tiempo que abordan dificultades en coherencia central, flexibilidad cognitiva y habilidades sociales. Modelos como el SCERTS \cite{Laurent2015SCERTS} ejemplifican cómo integrar dimensiones neuroeducativas en programas comprehensivos que promueven la regulación emocional, la comunicación y el apoyo transaccional en contextos inclusivos. Más recientemente, \textcite{kotsi2025} mostraron que el uso de inteligencia artificial en entornos educativos para estudiantes con TEA permite diseñar apoyos personalizados que incrementan la eficacia de las intervenciones, aunque también plantean desafíos vinculados con la privacidad, la accesibilidad y la formación del profesorado. Así, la neuropedagogía orientada por la innovación tecnológica amplía sus horizontes hacia un modelo de educación inclusiva que se nutre tanto de la ciencia como de las necesidades individuales de cada niño/a \cite{hernandez2024}.

La atención temprana, definida por la \textcite{gat2005} como el conjunto de intervenciones dirigidas a la población infantil de 0 a 6 años, cobra una relevancia decisiva cuando se observa desde la óptica neuropedagógica. La plasticidad cerebral característica de los primeros años convierte esta etapa en una ventana crítica para potenciar el desarrollo y mitigar dificultades asociadas al TEA. \textcite{dawson2008} ya había señalado que las intervenciones basadas en principios neurocientíficos pueden alterar la trayectoria del desarrollo neuronal en la infancia, y hallazgos más recientes lo confirman con contundencia. En 2025, una investigación longitudinal publicada en GlobalRPH mostró que la intervención antes de los 18 meses puede reducir en un 87\% la probabilidad de un diagnóstico formal de TEA, subrayando el impacto de una estimulación precoz e intensiva en la configuración de redes neuronales. Esta evidencia científica posiciona la atención temprana como un espacio privilegiado para la acción neuropedagógica. Desde esta perspectiva, no se trata únicamente de introducir programas de estimulación, sino de generar ecosistemas educativos y familiares que promuevan interacciones significativas, regulen la sobrecarga sensorial y fomenten el desarrollo de competencias adaptativas. \textcite{esteban2023} destacan que un enfoque holístico en atención temprana es clave para desbloquear el potencial de niños con TEA, pues integra dimensiones cognitivas, socioemocionales y comunicativas en un mismo marco de intervención. La neuropedagogía, apoyada en tecnologías emergentes, permite convertir esta etapa en un auténtico catalizador del desarrollo.

El papel de la neuroimagen en este proceso resulta crucial. Técnicas avanzadas como la resonancia magnética funcional (fMRI), la tractografía por difusión (DTI) o el electroencefalograma de alta resolución (EEG) han permitido identificar patrones de activación y conectividad cerebral vinculados con el TEA. \textcite{hazlett2017} demostraron que bebés con alto riesgo de TEA presentaban un crecimiento cerebral anómalo observable antes de la aparición de síntomas conductuales, lo que abre la puerta a intervenciones ultra-tempranas. Investigaciones recientes amplían este horizonte: \textcite{libero2015} destacan que la combinación de neuroimagen multimodal con algoritmos de inteligencia artificial mejora la detección precoz y la predicción de trayectorias de desarrollo individual. \textcite{wang2024}, por su parte, identificaron biomarcadores cuantitativos mediante resonancia magnética y aprendizaje profundo, lo que refuerza la posibilidad de diseñar intervenciones personalizadas basadas en perfiles neurocognitivos específicos. Sin embargo, junto a estos avances emergen desafíos éticos de gran calado. \textcite{goldani2014} ya advertían sobre la necesidad de protocolos estandarizados para el uso de neuroimagen en población pediátrica, y esta preocupación se ha intensificado en años recientes. \textcite{eitel2021} enfatiza que el uso predictivo de la neuroimagen debe ir acompañado de lineamientos éticos claros que prioricen el bienestar infantil y la confidencialidad de los datos. \textcite{zhang2025} exploran cómo la inteligencia artificial aplicada al diagnóstico temprano del TEA ofrece ventajas notables, pero también plantea dilemas en torno a la privacidad de la información genética y cerebral. Además de su uso diagnóstico, la neuroimagen en entornos educativos reales representa un aporte innovador en neuropedagogía. Intervenciones implementadas por padres en el hogar han mostrado efectos positivos en habilidades comunicativas y sociales \cite{oliveira2025}. Simultáneamente, la formación docente basada en neurociencia en aulas reales promueve estrategias pedagógicas fundadas en datos cerebrales \cite{Gomes2024NeurocienciaAutismo}. Estas aproximaciones reflejan una neuroimagen contextualizada que trasciende el laboratorio, situando la evidencia neurocognitiva al servicio directo de la práctica educativa inclusiva. Por último, recientes enfoques multimodales \cite{dcouto2023} y revisiones narrativas de perfiles estructurales y funcionales amplían nuestro conocimiento sobre las alteraciones neurobiológicas en la infancia con TEA, por otra parte revisiones recientes confirman que la IA potencia la automatización del diagnóstico e intervenciones adaptadas \cite{kotsi2025}.

Finalmente, la investigación en neuroimagen del TEA ha pasado de enfoques clínicos a propuestas que buscan observar el cerebro en acción en entornos educativos reales, lo que constituye un aporte innovador en neuropedagogía. \textcite{HallidayEtAl2024Autism} señala que técnicas como la fMRI y el EEG permiten identificar perfiles estructurales y funcionales modulados por la interacción social, mientras que \textcite{dcouto2023} muestran que la combinación multimodal de fMRI y sMRI con algoritmos de aprendizaje profundo mejora la clasificación del TEA, abriendo la posibilidad de aplicaciones fuera del laboratorio. Este giro hacia la validez ecológica es especialmente relevante en la atención temprana. \textcite{oliveira2025} evidencia que la intervención mediada por padres en contextos naturales facilita la generalización de aprendizajes, y experiencias formativas con docentes en aulas inclusivas refuerzan la utilidad de integrar principios neurocientíficos en la práctica pedagógica \cite{Gomes2024NeurocienciaAutismo}. Asimismo, \textcite{vilela2024} destacan que la convergencia de neuroimagen y genética puede ofrecer explicaciones más completas sobre la heterogeneidad del TEA, mientras que \textcite{nejati2024} subrayan la importancia de evaluar la transferencia de aprendizajes sociales en situaciones auténticas. En conjunto, la neuroimagen aplicada a contextos naturales de aprendizaje se perfila como una estrategia diferenciada para generar evidencias más ajustadas a la realidad de los niños con TEA y diseñar intervenciones educativas más personalizadas y efectivas.

En conclusión, la integración de la neuropedagogía, la atención temprana y la neuroimagen representa un enfoque multidisciplinario con potencial para transformar la vida de niños con TEA y sus familias. Este campo de investigación, dinámico y en constante evolución, no solo abre nuevas vías de comprensión científica, sino que también redefine la práctica educativa desde la perspectiva de la innovación tecnológica y la ética centrada en el niño/a.


\section{Metodologia}

\subsection{Diseño de investigación y muestra}
Este estudio adopta un diseño mixto, combinando enfoques cuantitativos y cualitativos. El componente cuantitativo sigue un diseño no experimental, descriptivo, explicativo, correlacional y de regresión. El componente cualitativo incorpora un grupo focal. Esta aproximación mixta permite abordar de manera comprehensiva el objetivo general de analizar la necesidad de incorporación de la neuroimagen en la intervención precoz del Trastorno del Espectro Autista (TEA).

\subsubsection{Participantes}
La muestra cuantitativa, seleccionada por conveniencia y criterio, consistió en dos grupos: uno con 50 familias con un niño/a diagnosticado de TEA, y en segundo lugar 67 docentes con experiencia en alumnado con diagnóstico TEA en educación infantil con alumnado de 3 años de edad. Para la fase cualitativa, se formó un grupo focal compuesto por una madre, un padre, un docente, un profesor de universidad, un médico pediatra y una directora de escuela infantil. Como complemento se utilizó la neuroimagen en el estudio de dos casos.


\subsection{Procedimiento de recolección de datos}
Se diseñó un cuestionario tipo escala Likert (1-5) con 20 ítems, distribuidos equitativamente en cuatro dimensiones: Neuropedagogía, Trastorno del espectro autista, Atención temprana y Neuroimagen. La construcción se basó en una tabla de operacionalización, alineando los ítems con las dimensiones y objetivos específicos. Para el grupo focal, se utilizó un guion basado en 8 preguntas directas extraídas de los ítems del cuestionario, cubriendo las cuatro dimensiones del estudio.


\subsection{ Procedimiento de análisis de datos}
Se realizaron análisis descriptivos, correlacionales y de regresión para los datos cuantitativos. Para los datos cualitativos, se llevó a cabo un análisis temático de la información obtenida en el grupo focal. La fiabilidad del cuestionario es de 0.98 (alpha de Cronbach) que se considera excelente. La validez de constructo se examinó a través de un análisis factorial exploratorio. El test de Kaiser-Meyer-Olkin (KMO) arrojó un resultado de 0.997, y la prueba de esfericidad de Bartlett fue significativa ($p < 0.001$), indicando la adecuación de los datos para el análisis factorial. El análisis de las comunalidades reveló dos ítems con valores muy altos y uno con valor más bajo, aunque superior a 0.5. El análisis de varianza con rotación Varimax confirmó la estructura factorial del cuestionario sin necesidad de eliminar ítems, respaldando la validez de constructo del instrumento.


\subsection{Consideraciones éticas}
En cumplimiento con los estándares éticos de la investigación, este estudio se llevó a cabo bajo estrictas consideraciones éticas. Se obtuvo la aprobación del Comité de Ética de la institución correspondiente antes de iniciar la investigación. Todos los participantes, tanto en la fase cuantitativa como en el grupo focal, fueron informados detalladamente sobre los objetivos y procedimientos del estudio, y proporcionaron su consentimiento informado por escrito. En el caso de las familias con niños y niñas con TEA, se tomaron precauciones adicionales para garantizar la comprensión total del estudio y sus implicaciones. Se aseguró la confidencialidad y el anonimato de todos los participantes, y los datos fueron tratados y almacenados de manera segura, cumpliendo con las normativas de protección de datos vigentes. Los participantes fueron informados de su derecho a retirarse del estudio en cualquier momento sin consecuencias negativas. Además, se proporcionó a todos los participantes información sobre cómo acceder a los resultados del estudio una vez finalizado.


\section{Resultados}

\subsection{Análisis descriptivo}
Se realizaron análisis descriptivos para cada una de las cuatro dimensiones del estudio: Neuropedagogía, Trastorno del Espectro Autista, Atención Temprana y Neuroimagen. A continuación (\Cref{tab-1}), se presentan los resultados de media, mediana, asimetría y curtosis para cada dimensión.

%--- codigo da tabela 1 ---%
\begin{table}[ht]
\centering
\begin{threeparttable}
\caption{Resultados descriptivos por dimensión.}\label{tab-1}
\begin{tabular}{lcccc}
\toprule
Dimensión & Media & Mediana & Asimetría & Curtosis \\
\midrule
A. Neuropedagogía & 4.35 & 4.40 & -0.72 & 0.53 \\
B. Trastorno del Espectro Autista & 4.18 & 4.20 & -0.45 & -0.21 \\
C. Atención Temprana & 4.52 & 4.60 & -0.89 & 0.76 \\
D. Neuroimagen & 3.96 & 4.00 & -0.18 & -0.34 \\
\bottomrule
\end{tabular}
\source{propia.}
\end{threeparttable}
\end{table}

La dimensión Neuropedagogía muestra una tendencia hacia valoraciones positivas, con una ligera asimetría negativa indicando una concentración de respuestas en los valores más altos de la escala. Para la dimensión Trastorno del Espectro Autista, se observa una valoración generalmente positiva, con una distribución ligeramente asimétrica hacia la izquierda y una curtosis platicúrtica, sugiriendo una distribución algo más plana que la normal. La dimensión Atención Temprana presenta la valoración más positiva, con una asimetría negativa moderada y una curtosis leptocúrtica, indicando una concentración de respuestas en torno a los valores más altos. En la dimensión Neuroimagen, se observa una valoración ligeramente positiva, con una asimetría negativa leve y una distribución platicúrtica, indicando una dispersión algo mayor en las respuestas.

Estos resultados sugieren una percepción generalmente positiva en todas las dimensiones, con la Atención Temprana recibiendo las valoraciones más altas y la Neuroimagen mostrando una mayor variabilidad en las respuestas. La asimetría negativa en todas las dimensiones indica una tendencia general hacia valoraciones por encima de la media.

\subsection{Análisis correlacional}
Se realizaron análisis descriptivos para cada una de las cuatro dimensiones del estudio: Se realizó un análisis de correlación $r$ de Pearson para examinar las relaciones entre las cuatro dimensiones del estudio: Neuropedagogía, Trastorno del Espectro Autista, Atención Temprana y Neuroimagen. Este análisis permite identificar la fuerza y dirección de las asociaciones lineales entre estas dimensiones. Los resultados se presentan en la Tabla \ref{tab-2}:

%--- codigo da tabela 2 ---%
\begin{table}[ht]
\centering
\begin{threeparttable}
\caption{Análisis de correción entre dimensiones.}\label{tab-2}
\begin{tabular}{lllll}
\toprule
Dimensión & Neuropedagogía & TEA & Atención Temprana & Neuroimagen \\
\midrule
Neuropedagogía & 1.000 & 0.685** & 0.721** & 0.592** \\
TEA & 0.685** & 1.000 & 0.743** & 0.631** \\
Atención Temprana & 0.721** & 0.743** & 1.000 & 0.678** \\
Neuroimagen & 0.592** & 0.631** & 0.678** & 1.000 \\
\bottomrule
\end{tabular}
\source{propia.}
\notes{** $p < 0.01$}
\end{threeparttable}
\end{table}

Los resultados del análisis de correlación revelan asociaciones significativas entre todas las dimensiones del estudio. Se observan correlaciones fuertes y positivas entre Neuropedagogía, Trastorno del Espectro Autista (TEA) y Atención Temprana, con coeficientes que oscilan entre 0.685 y 0.743 ($p < 0.01$). Esto sugiere una estrecha relación entre estas tres dimensiones, indicando que los participantes que valoran positivamente una de estas áreas tienden a valorar positivamente las otras dos. La dimensión de Neuroimagen muestra correlaciones moderadas a fuertes con las otras tres dimensiones, con coeficientes que van desde 0.592 a 0.678 ($p < 0.01$). Aunque estas correlaciones son algo más bajas que las observadas entre las otras dimensiones, siguen siendo significativas, lo que indica que la percepción de la importancia de la neuroimagen está relacionada con las valoraciones de las otras dimensiones, pero de manera menos intensa. La correlación más fuerte se observa entre Atención Temprana y TEA ($r = 0.743$, $p < 0.01$), lo que sugiere una fuerte asociación entre la percepción de la importancia de la atención temprana y la comprensión del trastorno del espectro autista. En general, estas correlaciones indican una interconexión significativa entre todas las dimensiones del estudio, resaltando la naturaleza integrada de la neuropedagogía, el entendimiento del TEA, la atención temprana y el uso de la neuroimagen en el contexto del autismo.

\subsection{Análisis de regresión}
Se llevó a cabo un análisis de regresión lineal múltiple para examinar cómo las dimensiones de Neuropedagogía, Trastorno del Espectro Autista (TEA) y Neuroimagen predicen la percepción de la importancia de la Atención Temprana. Este análisis nos permite entender la contribución relativa de cada una de estas dimensiones en la valoración de la Atención Temprana. Los resultados se presentan en la siguiente tabla (\Cref{tab-3}):

%--- codigo da tabela 3 ---%
\begin{table}[ht]
\centering
\begin{threeparttable}
\caption{Análisis de regresión.}\label{tab-3}
\begin{tabular}{lccccc}
\toprule
Variable Predictora & Coeficiente $\beta$ & Error Estándar & $t$ & $p$ & VIF \\
\midrule
(Constante) & 0.742 & 0.203 & 3.655 & $< 0.001$ & -- \\
Neuropedagogía & 0.328 & 0.058 & 5.655 & $< 0.001$ & 2.134 \\
TEA & 0.395 & 0.062 & 6.371 & $< 0.001$ & 2.287 \\
Neuroimagen & 0.186 & 0.047 & 3.957 & $< 0.001$ & 1.673 \\
\bottomrule
\end{tabular}
\source{propia.}
\notes{$R^2 = 0.681, R^2 \text{ ajustado} = 0.675, F(3, 113) = 80.42, p < 0.001$}
\end{threeparttable}
\end{table}

El modelo de regresión resultó estadísticamente significativo ($F(3, 113) = 80.42$, $p < 0.001$), explicando el 67.5 \% de la varianza en la percepción de la importancia de la Atención Temprana ($R^2$ ajustado = 0.675). La dimensión de Trastorno del Espectro Autista (TEA) emergió como el predictor más fuerte de la valoración de la Atención Temprana ($\beta = 0.395$, $p < 0.001$), seguido de cerca por la Neuropedagogía ($\beta = 0.328$, $p < 0.001$). Esto sugiere que la comprensión del TEA y la valoración de los principios neuropedagógicos están fuertemente asociadas con una percepción positiva de la importancia de la atención temprana. La Neuroimagen también resultó ser un predictor significativo, aunque con un impacto menor ($\beta = 0.186$, $p < 0.001$). Esto indica que la percepción de la importancia de la neuroimagen también contribuye, aunque en menor medida, a la valoración de la atención temprana. Los valores del Factor de Inflación de la Varianza (VIF) para todas las variables independientes están por debajo de 3, lo que sugiere que no hay problemas de multicolinealidad en el modelo.

En resumen, estos resultados indican que la percepción de la importancia de la Atención Temprana está fuertemente influenciada por la comprensión del TEA y la valoración de la neuropedagogía, con una contribución menor pero significativa de la percepción de la importancia de la neuroimagen. Esto respalda la idea de que un enfoque integral, que incorpore estos diversos aspectos, es crucial para la valoración y posiblemente la implementación efectiva de la atención temprana en el contexto del TEA.

\subsection{Análisis cualitativo}

\subsubsection{Grupo focal}

El análisis de los grupos focales se realizó con Atlas.ti 23. Se codificaron 98 citas distribuidas en cuatro categorías emergentes: a) comprensión del TEA; b) percepciones sobre la neuropedagogía; c) barreras y oportunidades de la atención temprana; y d) expectativas y limitaciones de la neuroimagen. La triangulación de datos se efectuó comparando resultados de los cuestionarios, el grupo focal y los estudios de caso, garantizando coherencia interna.

El grupo focal, compuesto por una madre de niño con diagnóstico TEA, un padre de una niña con diagnóstico TEA, un docente, un profesor universitario, un médico pediatra y una directora de guardería que integra niños y niñas con diagnóstico TEA, proporcionó datos valiosos que complementan los hallazgos cuantitativos. El análisis temático de las transcripciones reveló cuatro temas principales, alineados con las dimensiones del estudio:

\begin{itemize}
    \item Neuropedagogía en el contexto del TEA: Los participantes expresaron un fuerte consenso sobre la importancia de la neuropedagogía en la intervención del TEA. El docente (GF-D) enfatizó: ``La neuropedagogía nos ha proporcionado herramientas concretas para abordar los desafíos de aprendizaje únicos en niños con TEA. Por ejemplo, ahora entendemos mejor cómo estructurar el ambiente para reducir la sobrecarga sensorial''. La directora de la escuela de educación infantil (GF-G) añadió: ``La formación en neuropedagogía ha sido transformadora para nuestro equipo. Nos ha permitido diseñar intervenciones más precisas y personalizadas''.

\item Comprensión del Trastorno del Espectro Autista: El profesor universitario (GF-U) destacó cómo los avances en neurociencia han mejorado nuestra comprensión del TEA: ``Los estudios de neuroimagen nos han mostrado que el cerebro de individuos con TEA procesa la información de manera diferente. Esto ha llevado a intervenciones más enfocadas en fortalecer las conexiones neuronales específicas''. La madre (GF-M) compartió: ``Entender la plasticidad cerebral nos dio esperanza. Saber que el cerebro de nuestro hijo puede seguir desarrollándose y adaptándose nos motivó a buscar intervención temprana''.

\item Importancia de la Atención Temprana:
Hubo un consenso unánime sobre la crucial importancia de la atención temprana. El médico pediatra (GF-P) explicó: ``La intervención temprana puede realmente cambiar la trayectoria del desarrollo en niños con TEA. Hemos visto mejoras significativas en comunicación y habilidades sociales cuando la intervención comienza antes de los 3 años''. El padre (GF-F) añadió desde su experiencia personal: ``La atención temprana fue un punto de inflexión para nosotros. Vimos cambios notables en el comportamiento y la comunicación de nuestro hijo en cuestión de meses''.

\item Rol de la Neuroimagen: Las opiniones sobre la neuroimagen fueron más variadas. El profesor universitario (GF-U) fue entusiasta: ``La neuroimagen nos permite `ver' cómo funcionan diferentes intervenciones, lo que podría llevar a tratamientos más personalizados''. Sin embargo, el docente (GF-D) expresó preocupaciones prácticas: ``Aunque la neuroimagen suena prometedora, me pregunto cómo podríamos implementarla en entornos educativos típicos, dado su costo y complejidad''. El médico pediatra (GF-P) ofreció una perspectiva equilibrada: ``La neuroimagen es una herramienta poderosa para la investigación y puede guiar el desarrollo de intervenciones. Sin embargo, su uso directo en la práctica clínica diaria aún enfrenta desafíos''.
\end{itemize}

El grupo focal reveló un alto nivel de interés y entusiasmo por la integración de la neuropedagogía, la comprensión avanzada del TEA, la atención temprana y el potencial
de la neuroimagen. Sin embargo, también puso de manifiesto importantes desafíos en la aplicación práctica de estos conocimientos, especialmente en lo que respecta a la neuroimagen. Los participantes enfatizaron la necesidad de:
\begin{itemize}
\item Mayor formación en neuropedagogía para profesionales que trabajan con niños con TEA.
\item Más recursos para implementar intervenciones tempranas basadas en evidencia neurocientífica.
\item Investigación adicional sobre cómo integrar los hallazgos de neuroimagen en la práctica educativa y clínica diaria.
\item Concienciación pública sobre la importancia de la detección e intervención temprana en TEA.
\end{itemize}

Estos hallazgos cualitativos proporcionan un contexto rico para interpretar los resultados cuantitativos y sugieren áreas clave para futuras investigaciones y desarrollo de políticas en el campo de la atención temprana del TEA.

\subsubsection{Estudio de casos con neuroimagen}

La complejidad del Trastorno del Espectro Autista (TEA) y la dificultad de su diagnóstico precoz subrayan la necesidad de explorar métodos complementarios que puedan proporcionar información valiosa para la intervención precoz. En este contexto, se llevó a cabo un estudio de casos utilizando técnicas de neuroimagen en 10 niños/as de entre 1 y 3 años de edad que presentaban signos de TEA, pero carecían de un diagnóstico oficial debido a la complejidad inherente a la evaluación en edades tan tempranas. Los criterios de inclusión fueron: a) presencia de indicadores tempranos de TEA reportados por pediatras o educadores (alteraciones en la comunicación preverbal, conductas repetitivas o dificultades en la interacción social); b) disponibilidad de participación regular en sesiones de registro; y c) consentimiento parental para el uso de neuroimagen en contextos controlados. Se excluyeron participantes con condiciones neurológicas graves o antecedentes de epilepsia, por su potencial interferencia en los registros.

La justificación de este estudio radica en la urgente necesidad de herramientas que puedan ayudar en la identificación temprana y la intervención oportuna en casos de posible TEA. Las familias de estos niños expresaron su angustia ante la incertidumbre y la falta de orientación clara sobre cómo proceder. Asimismo, los centros educativos manifestaron su desconcierto al carecer de pautas específicas para trabajar con estos niños/as.

El uso de neuroimagen en este contexto busca proporcionar información adicional sobre los patrones de actividad cerebral y conectividad neuronal que podrían estar asociados con el TEA. Esta información podría ser crucial para:

\begin{enumerate}
    \item Contribuir a una identificación más temprana de los niños en riesgo de TEA.
\item Orientar el diseño de intervenciones personalizadas basadas en los patrones cerebrales observados.
\item Proporcionar a las familias y educadores información concreta sobre las necesidades específicas de cada niño.
\item Apoyar la toma de decisiones en cuanto a la asignación de recursos y apoyos educativos.
\end{enumerate}

Es importante destacar que este estudio se realizó con el consentimiento informado de las familias, quienes, movidas por la preocupación y la falta de alternativas claras, accedieron a participar en busca de respuestas y orientación. Se tomaron todas las precauciones éticas necesarias y el compromiso de proporcionar a las familias toda la información y apoyo necesarios, independientemente de los resultados.

Los datos de neuroimagen se obtuvieron con un dispositivo Emotiv Epoc+ de 14 canales, siguiendo los protocolos PRONIN\textsuperscript{\textcopyright} y SIEN\textsuperscript{\textcopyright} adaptados para entornos educativos. Se controlaron artefactos mediante preprocesamiento con filtros digitales y se aplicó la eliminación de ruido ocular y muscular. La validez se reforzó con la triangulación de fuentes: registros de neuroimagen, observación directa y notas de los especialistas de audición y lenguaje. La fiabilidad intraobservador se garantizó con doble codificación de los registros por dos investigadores independientes, alcanzando un coeficiente Kappa $>0.80$.

Este enfoque, aunque experimental, representa un paso hacia la integración de métodos neuropedagógicos avanzados en la práctica educativa temprana, con el potencial de mejorar significativamente la vida de los niños con posible TEA y sus familias. A continuación (\Cref{fig-1,fig-2}) se muestran la neuroimagen de dos niños de 4 y 2 años de edad, en el momento de interacción con su madre en un contexto de juego habitual.

%--- código da figura 1 ---%
\begin{figure}[h!]
\centering
\begin{minipage}{0.90\textwidth}
\includegraphics[width=\textwidth]{Imagens/image1.png}
\caption{Niño de 4 años de edad (trabajo con una baraja de vocabulario, en este momento tenía que decir la palabra ``payaso'', palabra que ya conoce).}
\label{fig-1}
\source{propia.}
\end{minipage}
\end{figure}

El análisis de la neuroimagen de la Figura \ref{fig-1}, nos aporta los siguientes datos:

\begin{enumerate}
\item Nivel emocional bajo: El 29\% de nivel emocional sugiere una respuesta emocional reducida ante la tarea, lo cual es consistente con las dificultades en el procesamiento emocional a menudo observadas en niños con TEA.
\item Activación hemisférica asimétrica: Se observa una mayor activación en el hemisferio izquierdo, particularmente en los lóbulos frontal y temporal, con menor actividad en los lóbulos parietal y occipital. Esta asimetría podría indicar un procesamiento atípico del lenguaje y la información visual.
\item Patrones de ondas cerebrales:
\begin{itemize}
    \item Hemisferio izquierdo: La presencia de pocas ondas theta y alpha, junto con una gran cantidad de ondas beta y localizaciones de gamma, sugiere un estado de alta alerta y procesamiento cognitivo intenso, posiblemente indicando un esfuerzo excesivo para realizar la tarea.
    \item Hemisferio derecho: El predominio de ondas gamma en el lóbulo temporal y focos en los lóbulos frontal y occipital podría indicar un procesamiento visual y de detalles intensificado.
\end{itemize}
\item Comportamiento observado: El hecho de que el niño parezca detenerse demasiado en los detalles del payaso en lugar de simplemente nombrar la imagen es consistente con la tendencia de muchos niños con TEA a enfocarse en detalles específicos en lugar de percibir el conjunto de manera holística (coherencia central débil).
\item Implicaciones para el TEA: Este patrón de activación cerebral y comportamiento sugiere:
    \begin{itemize}
        \item Procesamiento atípico del lenguaje y la información visual.
        \item Posible hiperfocalización en detalles específicos.
\item Dificultades en la integración de información para formar una percepción global.
\item Posible sobrecarga sensorial o cognitiva durante la tarea.
    \end{itemize}
\end{enumerate}

Estas observaciones podrían ser útiles para diseñar intervenciones personalizadas que ayuden al niño a desarrollar habilidades de procesamiento global y mejorar la fluidez en tareas de denominación, al tiempo que se aprovecha su atención al detalle como una posible fortaleza.

%--- codigo da figura 2 ---%
\begin{figure}[h!]
\centering
\begin{minipage}{0.90\textwidth}
\includegraphics[width=\textwidth]{Imagens/image2.png}
\caption{Niña de 2 años de edad (contexto de juego con juguetes habituales en compañía de su madre).}
\label{fig-2}
\source{propia.}
\end{minipage}
\end{figure}

La neuroimagen de la niña de 2 años durante una actividad de juego con su madre (\Cref{fig-2}) revela información muy interesante, especialmente cuando se combina con los marcadores neuroeducativos:

\begin{enumerate}
    \item Compromiso con la tarea (84\%): este alto nivel de compromiso se refleja en la activación generalizada observada en todos los lóbulos cerebrales. Indica que la niña está profundamente involucrada en la actividad de juego, dedicando una cantidad significativa de recursos cognitivos a la tarea.
    \item Emociones (88\%): el elevado marcador emocional se correlaciona con la intensa activación observada en los lóbulos frontal y temporal derechos. Estas áreas están asociadas con el procesamiento emocional y la interpretación de señales sociales. Este alto nivel emocional sugiere que la niña está experimentando una fuerte respuesta afectiva durante la interacción con su madre.
    \item Interés (70\%): el nivel de interés, aunque menor que los otros marcadores, sigue siendo significativo. Esto podría estar relacionado con la activación observada en el lóbulo occipital izquierdo, sugiriendo un enfoque visual intenso en aspectos del juego que captan su atención.
    \item Sobreactivación y ondas gamma: la necesidad de filtrar las ondas theta, alpha y beta para visualizar la actividad gamma indica una intensa actividad neuronal. Las ondas gamma están asociadas con el procesamiento cognitivo de alto nivel y la integración de información. Su presencia prominente sugiere un procesamiento intenso y posiblemente una integración sensorial compleja durante el juego.
    \item Activación bilateral: la activación en ambos hemisferios, con focos intensos en áreas específicas, indica un procesamiento global de la experiencia de juego. Esto sugiere que la niña está integrando múltiples aspectos de la interacción, incluyendo elementos sociales, emocionales y sensoriales.
    \item Implicaciones para el TEA y la intervención temprana:
        \begin{itemize}
        \item El alto nivel de compromiso y emoción sugiere que la niña está profundamente involucrada en la interacción, lo cual es una fortaleza significativa para la intervención temprana.
\item La intensa activación y los altos marcadores neuroeducativos podrían indicar una experiencia sensorial y emocional abrumadora, típica en algunos casos de TEA. Esto sugiere la necesidad de estrategias para ayudar a la niña a regular y procesar estas experiencias intensas.
\item El interés visual marcado (activación occipital) combinado con el alto compromiso emocional ofrece una oportunidad para desarrollar intervenciones que aprovechen los intereses visuales para fomentar el desarrollo social y emocional.
\item La activación global del cerebro sugiere una base sólida para el aprendizaje y el desarrollo, pero también indica la necesidad de ayudar a la niña a filtrar y procesar la información de manera más eficiente.
    \end{itemize}
\end{enumerate}

\section{Discusión}
La presente investigación aporta evidencias relevantes en la intersección entre neuropedagogía, atención temprana y neuroimagen en el TEA. Los hallazgos se alinean con investigaciones recientes que destacan el papel de la neurociencia aplicada a la educación inclusiva. Estos hallazgos permiten establecer puentes claros entre la teoría neuropedagógica y la práctica educativa. La neuroimagen, al proporcionar evidencias sobre los patrones de activación cerebral durante tareas de atención, memoria y comunicación, puede orientar el diseño de intervenciones más precisas. En el ámbito de la educación infantil y la atención temprana, estos datos neurocognitivos pueden traducirse en estrategias concretas, como la adaptación del ritmo de presentación de estímulos, la selección de materiales multimodales o la estructuración del entorno sensorial según el perfil de cada niño o niña \cite{hernandez2024, pradeep2024}. Asimismo, la formación docente basada en neurociencia aplicada -- con apoyo en registros de EEG y biofeedback emocional -- permite que el profesorado reconozca indicadores de sobrecarga cognitiva o desregulación emocional y ajuste sus metodologías en tiempo real \cite{bueno2024, immordino2015}. De este modo, la neuroimagen deja de ser una herramienta exclusivamente diagnóstica para convertirse en un instrumento formativo que transforma la toma de decisiones pedagógicas cotidianas. Por otra parte, \textcite{HallidayEtAl2024Autism} y \textcite{dcouto2023} subrayan que la neuroimagen multimodal puede revelar perfiles estructurales y funcionales útiles para la detección temprana, lo cual se refleja en los patrones cerebrales observados en los estudios de caso. A su vez, \textcite{zhang2025} muestra que la inteligencia artificial aplicada a la neuroimagen mejora la personalización de intervenciones, en consonancia con la visión de nuestro estudio. A la luz de investigaciones recientes en neuroimagen educativa, diversos autores destacan, igualmente, la necesidad de trasladar los hallazgos de la neurociencia a contextos pedagógicos auténticos. \textcite{howardjones2024} y \textcite{ansari2011} subrayan que la tecnología digital y la neurociencia convergen en la creación de entornos de aprendizaje más adaptativos, capaces de responder a la diversidad cognitiva. De igual modo, \textcite{zadina2022} propone el concepto de \textit{neuropedagogía digital}, donde la neuroimagen y la inteligencia artificial se integran para personalizar la enseñanza mediante análisis de redes cerebrales activas durante el aprendizaje. Estas perspectivas confirman que la neuroeducación contemporánea no se limita a observar el cerebro, sino a aplicar su comprensión a la transformación didáctica y tecnológica de la enseñanza.

A partir de los resultados obtenidos, se vuelve esencial interpretar de manera crítica el significado pedagógico de las correlaciones observadas entre las dimensiones de neuropedagogía, TEA, atención temprana y neuroimagen. Estos vínculos sugieren que la comprensión del funcionamiento cerebral y de las bases cognitivas del desarrollo puede orientar prácticas educativas más eficaces. Estudios recientes \cite{dawson2008, chen2024} evidencian que el reconocimiento temprano de patrones de activación cerebral está asociado a mejoras en las competencias comunicativas y sociales cuando se traduce en programas educativos estructurados.

En este sentido, los resultados del presente estudio refuerzan la necesidad de integrar la neuroimagen en contextos educativos reales, permitiendo que docentes y terapeutas ajusten las intervenciones conforme a las respuestas neurofisiológicas y conductuales de los niños \cite{bueno2024, libero2015}. Este enfoque promueve una pedagogía basada en evidencias, en la que el análisis de datos cerebrales no sustituye la observación pedagógica, sino que la complementa, proporcionando indicadores objetivos que amplían la comprensión del proceso de aprendizaje en niños con TEA.

Por tanto, la neuroimagen educativa emerge no solo como un instrumento de investigación, sino también como una herramienta aplicada de regulación y personalización de la intervención pedagógica, abriendo camino a prácticas colaborativas entre docentes, especialistas y familias \cite{hernandez2024}.

A partir de los resultados obtenidos, se vuelve esencial interpretar de manera crítica el significado pedagógico de las correlaciones observadas entre las dimensiones de neuropedagogía, TEA, atención temprana y neuroimagen. Estos vínculos sugieren que la comprensión del funcionamiento cerebral y de las bases cognitivas del desarrollo puede orientar prácticas educativas más eficaces. Estudios recientes \cite{dawson2008, chen2024} evidencian que el reconocimiento temprano de patrones de activación cerebral está asociado a mejoras en las competencias comunicativas y sociales cuando se traduce en programas educativos estructurados.

La aportación novedosa de este trabajo radica en aplicar la neuroimagen en contextos naturales de aprendizaje, combinando evidencia cuantitativa, cualitativa y de casos individuales, lo que refuerza la validez ecológica de la neuropedagogía aplicada.

A pesar de la consistencia de los resultados, este estudio presenta ciertas limitaciones que deben ser consideradas. En primer lugar, la muestra -- centrada en familias y docentes de educación infantil -- limita la generalización de los hallazgos a otros niveles educativos. En segundo lugar, la utilización de un diseño exploratorio con un número reducido de estudios de caso con neuroimagen impide extraer conclusiones causales firmes, lo que demanda investigaciones longitudinales y experimentales que confirmen las tendencias observadas. Además, la dependencia de fuentes secundarias para contextualizar algunos resultados refleja la necesidad de ampliar la base empírica de la neuropedagogía mediante registros neurofisiológicos en contextos educativos reales \cite{libero2015}.

\section{Conclusión}
Los resultados de esta investigación subrayan la importancia crítica de la neuroimagen en el campo de la neuropedagogía y la atención temprana del Trastorno del Espectro Autista (TEA). El análisis cuantitativo demostró correlaciones significativas entre las dimensiones de Neuropedagogía, TEA, Atención Temprana y Neuroimagen, indicando una estrecha interrelación entre estos aspectos. La comprensión del TEA y la valoración de los principios neuropedagógicos emergieron como fuertes predictores de la percepción de la importancia de la atención temprana, respaldando la necesidad de una formación integral en estos ámbitos para los profesionales que trabajan con niños con TEA.

Aunque la neuroimagen emergió como un predictor significativo pero de menor impacto en el modelo cuantitativo, los estudios de caso demostraron su invaluable contribución para comprender el funcionamiento cerebral único de cada niño con posible TEA. Los patrones de activación cerebral observados, junto con los marcadores neuroeducativos, ofrecen una base sólida para diseñar intervenciones altamente personalizadas, aprovechando las fortalezas específicas de cada niño y abordando sus desafíos particulares.

La neuroimagen permite identificar patrones cerebrales asociados con el TEA antes de que los síntomas conductuales sean plenamente evidentes, abriendo la puerta a intervenciones más tempranas y potencialmente más efectivas. Sin embargo, el análisis cualitativo reveló preocupaciones prácticas sobre la implementación de la neuroimagen en entornos educativos, incluyendo costos y complejidad. Esto sugiere la necesidad de desarrollar métodos más accesibles y protocolos estandarizados para su uso en la práctica clínica y educativa.

Esta investigación destaca el potencial transformador de integrar la neuroimagen en la neuropedagogía y la atención temprana del TEA, ofreciendo una base para desarrollar enfoques más precisos, personalizados y basados en evidencia para la intervención temprana. No obstante, también señala la necesidad de abordar desafíos prácticos y éticos, así como de proporcionar una formación más robusta en neuropedagogía, neurociencia y uso e interpretación de neuroimagen para profesionales que trabajan con niños con TEA.

Con todo, este estudio abre nuevas vías para la investigación y la práctica en el campo del TEA, subrayando la importancia de un enfoque multidisciplinario que integre los avances en neuroimagen con los principios de la neuropedagogía y la atención temprana. El camino hacia la implementación efectiva de estas herramientas en la práctica diaria apresenta desafíos, pero el potencial para mejorar significativamente la vida de los niños con TEA y sus familias justifica plenamente los esfuerzos continuos en esta dirección.





\printbibliography\label{sec-bib}
% if the text is not in Portuguese, it might be necessary to use the code below instead to print the correct ABNT abbreviations [s.n.], [s.l.]
%\begin{portuguese}
%\printbibliography[title={Bibliography}]
%\end{portuguese}


\begin{contributors}[sec-contributors]
\authorcontribution{Antonio Hernández Fernández}[conceptualization,methodology,investigation,formalanalysis,validation,visualization,writing,review,supervision,projadm]
\authorcontribution{Claudia de Barros Camargo}[conceptualization,methodology,investigation,formalanalysis,validation,review,supervision]
\end{contributors}

\begin{dataavailability}
\txtdataavailability{dataonly} % options: dataavailable, dataonly, databody, datanotav, nodata
\end{dataavailability}

\begin{funding}
Agradecimientos a:

Proyectos de Investigación Aplicada FEDER-UGR 2023 (C-SEJ-132-UGR23). Título: ``Neurodidáctica en la Educación Superior: laboratorios de neuroimagen para la transformación innovadora de la enseñanza''.

Proyecto de innovación docente GID2016-31: ``Neurociencia de la lectoescritura en estudiantes universitarios: impacto de las tecnologías digitales y la inteligencia artificial en los procesos cognitivos''. Universidad Nacional de Educación a Distancia.

Proyecto de innovación: PID2025\_24. ``Diseño de entornos neuro-saludables y afectivos en la universidad: prácticas neurodidácticas para la conexión profesor-alumno''.

Proyecto de innovación: ``Neurodidáctica del desarrollo: Diseño y Validación de un Ecosistema Digital NeuroPedagógico para la Atención Personalizada al alumnado neurodiverso de la UNED''. 2025-2026.

\end{funding}

\end{document}

