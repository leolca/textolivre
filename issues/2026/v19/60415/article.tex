% !TEX TS-program = XeLaTeX
% use the following command:
% all document files must be coded in UTF-8
\documentclass[portuguese]{textolivre}
% build HTML with: make4ht -e build.lua -c textolivre.cfg -x -u article "fn-in,svg,pic-align"

\journalname{Texto Livre}
\thevolume{19}
%\thenumber{1} % old template
\theyear{2026}
\receiveddate{\DTMdisplaydate{2025}{7}{19}{-1}} % YYYY MM DD
\accepteddate{\DTMdisplaydate{2025}{10}{24}{-1}}
\publisheddate{\DTMdisplaydate{2025}{12}{10}{-1}}
\corrauthor{Dieila dos Santos Nunes}
\articledoi{10.1590/1983-3652.2026.60415}
%\articleid{NNNN} % if the article ID is not the last 5 numbers of its DOI, provide it using \articleid{} commmand 
% list of available sesscions in the journal: articles, dossier, reports, essays, reviews, interviews, editorial
\articlesessionname{articles}
\runningauthor{Santos e Nunes} 
%\editorname{Leonardo Araújo} % old template
\sectioneditorname{Daniervelin Pereira~\orcid{0000-0003-1861-3609}}
\layouteditorname{Leonardo Araújo~\orcid{0000-0003-3884-2177}}

\title{Salve, linguagem digital! Uma análise tecnodiscursiva da marca Sallve no Instagram}
\othertitle{Hello, digital language! A techno-discursive analysis of the Sallve brand on Instagram}
% if there is a third language title, add here:
%\othertitle{Artikelvorlage zur Einreichung beim Texto Livre Journal}

\author[1]{Maiara Luiza Schönardie dos Santos~\orcid{0009-0008-3338-991X}\thanks{Email: \href{mailto:maiarassantos@sou.faccat.br}{maiarassantos@sou.faccat.br}}}
\author[2]{Dieila dos Santos Nunes~\orcid{0000-0001-5349-5244}\thanks{Email: \href{mailto:dieilanunes@faccat.br}{dieilanunes@faccat.br}}}
\affil[1]{Faculdades Integradas de Taquara, Curso de Letras, Taquara, RS, Brasil.}
\affil[2]{Faculdades Integradas de Taquara, Curso de Letras, Programa de Pós-Graduação em Desenvolvimento Regional, Taquara, RS, Brasil.}

\addbibresource{article.bib}
% use biber instead of bibtex
% $ biber article

% used to create dummy text for the template file
\definecolor{dark-gray}{gray}{0.35} % color used to display dummy texts
\usepackage{lipsum}
\SetLipsumParListSurrounders{\colorlet{oldcolor}{.}\color{dark-gray}}{\color{oldcolor}}

% used here only to provide the XeLaTeX and BibTeX logos
\usepackage{hologo}

% if you use multirows in a table, include the multirow package
\usepackage{multirow}

% provides sidewaysfigure environment
\usepackage{rotating}

% CUSTOM EPIGRAPH - BEGIN 
%%% https://tex.stackexchange.com/questions/193178/specific-epigraph-style
\usepackage{epigraph}
\renewcommand\textflush{flushright}
\makeatletter
\newlength\epitextskip
\pretocmd{\@epitext}{\em}{}{}
\apptocmd{\@epitext}{\em}{}{}
\patchcmd{\epigraph}{\@epitext{#1}\\}{\@epitext{#1}\\[\epitextskip]}{}{}
\makeatother
\setlength\epigraphrule{0pt}
\setlength\epitextskip{0.5ex}
\setlength\epigraphwidth{.7\textwidth}
% CUSTOM EPIGRAPH - END

% to use IPA symbols in unicode add
%\usepackage{fontspec}
%\newfontfamily\ipafont{CMU Serif}
%\newcommand{\ipa}[1]{{\ipafont #1}}
% and in the text you may use the \ipa{...} command passing the symbols in unicode

% LANGUAGE - BEGIN
% ARABIC
% for languages that use special fonts, you must provide the typeface that will be used
% \setotherlanguage{arabic}
% \newfontfamily\arabicfont[Script=Arabic]{Amiri}
% \newfontfamily\arabicfontsf[Script=Arabic]{Amiri}
% \newfontfamily\arabicfonttt[Script=Arabic]{Amiri}
%
% in the article, to add arabic text use: \textlang{arabic}{ ... }
%
% RUSSIAN
% for russian text we also need to define fonts with support for Cyrillic script
% \usepackage{fontspec}
% \setotherlanguage{russian}
% \newfontfamily\cyrillicfont{Times New Roman}
% \newfontfamily\cyrillicfontsf{Times New Roman}[Script=Cyrillic]
% \newfontfamily\cyrillicfonttt{Times New Roman}[Script=Cyrillic]
%
% in the text use \begin{russian} ... \end{russian}
% LANGUAGE - END

% EMOJIS - BEGIN
% to use emoticons in your manuscript
% https://stackoverflow.com/questions/190145/how-to-insert-emoticons-in-latex/57076064
% using font Symbola, which has full support
% the font may be downloaded at:
% https://dn-works.com/ufas/
% add to preamble:
% \newfontfamily\Symbola{Symbola}
% in the text use:
% {\Symbola }
% EMOJIS - END

% LABEL REFERENCE TO DESCRIPTIVE LIST - BEGIN
% reference itens in a descriptive list using their labels instead of numbers
% insert the code below in the preambule:
%\makeatletter
%\let\orgdescriptionlabel\descriptionlabel
%\renewcommand*{\descriptionlabel}[1]{%
%  \let\orglabel\label
%  \let\label\@gobble
%  \phantomsection
%  \edef\@currentlabel{#1\unskip}%
%  \let\label\orglabel
%  \orgdescriptionlabel{#1}%
%}
%\makeatother
%
% in your document, use as illustraded here:
%\begin{description}
%  \item[first\label{itm1}] this is only an example;
%  % ...  add more items
%\end{description}
% LABEL REFERENCE TO DESCRIPTIVE LIST - END


% add line numbers for submission
%\usepackage{lineno}
%\linenumbers

\begin{document}
\maketitle

\begin{polyabstract}
\begin{abstract}
Esta pesquisa tem como objetivo analisar estratégias tecnodiscursivas adotadas pela marca de dermocosméticos nativa digital Sallve, por meio da linguagem publicitária veiculada em seu perfil no Instagram. A abordagem metodológica utilizada foi qualitativa, por intermédio de pesquisa bibliográfica e estudo de caso do perfil oficial da marca @sallve e de dois anúncios selecionados publicados na plataforma em questão. As bases teórico-metodológicas são a Análise do Discurso Digital de \textcite{paveau2021analise}, apresentando as seis características que um discurso digital nativo apresenta (composição, deslinearização, ampliação, relacionalidade, investigabilidade e imprevisibilidade), e o estudo sobre a tipologia do discurso propagandista de \textcite{charaudeau2010discurso}, com enfoque no dispositivo triangular (publicidade-concorrência-público) denominado contrato de semiengodos. Os resultados constataram a necessidade da adequação da comunicabilidade do discurso publicitário no contexto digital, visto que as mudanças tecnológicas digitais implicam uma maior competitividade dentro do mercado de dermocosméticos na internet. Observou-se a importância dos meios discursivos que contemplam as ferramentas disponibilizadas pela plataforma Instagram e a relevância dos mecanismos algorítmicos para a publicidade da marca Sallve. Destaca-se que este estudo pode contribuir para a maior compreensão dos fenômenos dos fenômenos tecnodiscursos e das estratégias discursivas utilizadas na linguagem publicitária digital.

\keywords{Análise do Discurso Digital \sep Instagram \sep Algoritmos \sep Publicidade}
\end{abstract}

\begin{english}
\begin{abstract}
This research aims to analyze the techno-discursive strategies adopted by the digitally native dermocosmetics brand Sallve, through the advertising language conveyed on its Instagram profile. The methodological approach used was qualitative, through bibliographic research and a case study of the brand's official profile @sallve and two selected advertisements published on the platform. The theoretical and methodological bases are \posscite{paveau2021analise} Digital Discourse Analysis, presenting the six characteristics that a native digital discourse presents (composition, delinearization, amplification, relationality, investigability, and unpredictability), and \posscite{charaudeau2010discurso} study on the typology of advertising discourse, focusing on the triangular device (advertising-competition-public) called the semi-deception contract. The results confirmed the need to adapt the communicability of advertising discourse in the digital context, since digital technological changes imply greater competitiveness within the dermocosmetics market on the internet. The importance of the discursive means encompassing the tools provided by the Instagram platform and the relevance of algorithmic mechanisms for Sallve's brand advertising were observed. It is noteworthy that this study can contribute to a greater understanding of techno-discourse phenomena and the discursive strategies used in digital advertising language.

\keywords{Digital Discourse Analysis \sep Instagram \sep Algorithms \sep Advertising}
\end{abstract}
\end{english}
% if there is another abstract, insert it here using the same scheme
\end{polyabstract}

\section{Introdução}\label{sec-intro}
A linguagem é o primeiro poder do ser humano, pois possibilita a convivência social e a construção de laços; sua eficácia comunicativa depende de situações e contratos estabelecidos entre os interlocutores \cite{charaudeau2012linguagem}. Esses atos comunicativos são compreendidos como posicionamentos discursivos que envolvem sentido, estratégia e identidade \cite{charaudeau2005analise} e, no ambiente digital, assumem novas formas, chamadas de tecnodiscursos, adaptando-se às dinâmicas das mídias sociais \cite{paveau2022ressignificacao}.

Dessarte, observa-se que a transformação digital, no âmbito linguístico, ressignificou as maneiras de se comunicar e, consequentemente, manter-se informado, o que afetou diretamente os veículos de comunicação. As novas tecnologias digitais, com destaque para a internet, implicaram sobretudo no modo de fazer publicidade e de propagá-la. As mídias sociais, por exemplo, passaram a ser o grande diferencial, recebendo mais investimento para disseminação dos anúncios publicitários \cite{alves2020publipost}. Todavia, não somente o meio de anunciar aquilo que se busca vender foi transformado, como também a forma de redigir o discurso publicitário \cite{carvalho2010publicidade}.

Sabe-se que as tecnologias digitais continuam a se aprimorar continuamente, trazendo novos aspectos e desafios para o discurso publicitário, o que afeta o âmbito comercial e implica na competitividade entre os diferentes mercados. Por isso, as marcas veem-se na necessidade de atualizar constantemente suas estratégias de comunicação no campo digital para que, dessa forma, sigam atingindo seus públicos-alvo. O estudo feito por \textcite{giering2021discurso} aponta os obstáculos enfrentados no campo linguístico quanto ao discurso digital nativo e a forma como é feita sua interpretação dentro dos padrões de textualidade. Sendo um assunto ainda muito recente, o linguista \textcite{maingueneau2022rasoes} reconhece o aumento de pesquisas no âmbito da comunicação digital, mas aponta que são poucas aquelas que contemplam a análise discursiva nesse meio. Visto isso, evidenciou-se a importância de aprofundar estudos nessa área.

Perante as transformações no campo linguístico dadas na contemporaneidade, levanta-se a problemática: quais são as estratégias tecnodiscursivas utilizadas pela marca Sallve em sua publicidade nas mídias sociais digitais? Nesse contexto, este estudo tem como objetivo analisar as estratégias tecnodiscursivas adotadas pela marca nativa digital de dermocosméticos Sallve em sua linguagem publicitária utilizada na mídia social digital Instagram. A escolha de estudar o perfil da marca supracitada tem a intenção de entender melhor a comunicabilidade com o seu público-alvo e com os clientes que já são fidelizados, como também investigar de que forma ela faz uso dos recursos tecnodiscursivos oferecidos pela plataforma Instagram para influenciar os usuários. Escolheu-se especificamente a Sallve por ser uma marca nativa digital, isto é, originada diretamente na internet.

Este artigo, além das presentes considerações iniciais, compõe-se de outras quatro seguintes seções: a apresentação de premissas teóricas relacionadas às esferas das mídias sociais digitais com foco no Instagram, da Análise do Discurso Digital e do discurso propagandista-publicitário; a metodologia empregada neste estudo; a análise e discussão do perfil e dos anúncios selecionados da marca Sallve; as considerações finais que contemplam os resultados obtidos.

\section{Os algoritmos e sua implicação no campo das mídias sociais digitais}\label{sec-normas}
No auge da era digital, grande parte das pessoas têm acesso à internet, fazendo uso desse mecanismo diariamente para variadas funções, assim, consequentemente, interagindo com os mais diversos algoritmos em seus cotidianos, como aponta \textcite{paveau2021analise}. A autora define algoritmos como “sequências de instruções que permitem a solução de problemas” \cite[p. 39]{paveau2021analise}, esclarecendo que eles atuam conforme sistemas matemáticos, calculando dados de forma que os resultados são filtragens de informações que aparecerão mais para determinado usuário, os locais em que isso ocorrerá, ou até mesmo o inverso: invisibilizando informações. Tendo como ponto de partida as ideias do sociólogo Dominique Cardon, \textcite[p. 39]{paveau2021analise} enfatiza que “ao se encontrarem com a informática, os números tornam-se sinais digitais (listas, botões, contadores, recomendações, linhas do tempo, publicidade personalizada, mapas de GPS, etc.) [...]”, realçando as diferentes formas de reações que os algoritmos têm em contraste às interações provocadas pelos usuários da internet. A autora também descreve os algoritmos como “operadores de coerção discursiva e de instrução semântica” \cite[p. 40]{paveau2021analise}, refletindo que, embora eles não se estendam no campo da linguística, são capazes de influenciar na criação de discursos na internet de forma subentendida, já que, por meio dos cálculos, acabam por ficar “abaixo do radar” da língua.

\textcite{paveau2021analise} considera os algoritmos como uma ferramenta interessante quanto à implicação da análise discursiva digital, porque, mesmo que invisíveis às telas, trazem à tona a classificação heurística do Dominique Cardon desse instrumento de rastreamento.

Acerca da implicação dos hiperlinks no âmbito discursivo, aponta-se sua influência por meio de sua natureza algorítmica, além de também gerarem a ampliação dos locutores que, por meio de uma “inteligência coletiva”, são capazes de transferir o poder e a legitimidade discursiva de uma hierarquia vertical para uma horizontal, na qual o nível de contribuição dos usuários naquele discurso persuade consideravelmente o público \cite{paveau2021analise}.

Quanto aos rastros virtuais deixados pelos usuários, \textcite{paveau2021analise} explica a técnica do \textit{machine learning} de Dominique Cardon, a qual possibilita os algoritmos a compreenderem e preverem o comportamento de determinado usuário por meio de outros com comportamentos semelhantes, pontuando que esse mecanismo tem boa funcionalidade, em especial no campo publicitário, uma vez que tende a produzir discursos que indicam determinados assuntos, sobretudo, anúncios que estão ligados a sites previamente visitados, muitas vezes, refletidos nas mídias sociais. Visto isso, entende-se que os algoritmos têm impacto direto na linguagem publicitária de uma marca, pois são capazes de direcionar e ampliar o alcance do discurso a ser comunicado referente ao público que se almeja alcançar e àquele que se busca manter fidelizado.

No campo publicitário, os algoritmos – especialmente por meio das mídias sociais – têm desempenhado um papel fundamental. \textcite{sao2022algoritmos}, com base nas ideias de \textcite{schuch2020algoritmos}, destaca que a expansão das plataformas on-line impulsionou significativamente a coleta de dados no setor publicitário. Essa tendência transformou o mercado de negócios, uma vez que levou as agências a estruturarem setores especializados, com o objetivo de aprimorar a compreensão e o uso estratégico dos dados coletados.

Dessa forma, frisa-se a praticidade atual das empresas em identificar seus públicos por meio do rastreamento algorítmico do comportamento on-line que tende a diminuir os riscos de erro publicitário. Para \textcite{sao2022algoritmos}, a mudança do “habitus publicitário” exige dos profissionais da área que sejam capazes de dirigir seus trabalhos por intermédio da interpretação dos dados disponibilizados por meio dos usuários on-line via balanço algorítmico.

\subsection{As dimensões do discurso digital e o Instagram}\label{sec-conduta}
Sabe-se que o advento da internet trouxe demasiados recursos que transformaram muitos conhecimentos humanos. \textcite{paveau2021analise}, utilizando o termo “ampliação”, afirma que, no campo digital, a escrita é inegavelmente ampliada por diversos mecanismos disponibilizados aos usuários como comentários, compartilhamentos e até mesmo a possibilidade de que vários autores produzam textos de modo simultâneo sem o risco de confusão enunciativa. Logo, “a escrita digital na ordem da razão computacional é uma escrita ampliada na medida em que suas capacidades expressivas e comunicacionais ultrapassam as da ordem da razão gráfica” \cite[p. 53]{paveau2021analise}, reforçando-se, assim, as novas dimensões estabelecidas pela era digital para a algo tão intrínseco como o ato de escrever.

Os discursos nativos da internet são diretamente criados no espaço digital, sendo compostos por aparatos disponibilizados on-line, diferentes daqueles que adaptam o que está fora da internet para dentro dela \cite{paveau2021analise}, dessa maneira, a autora explica que o discurso digital nativo possui características tão próprias que, fora de tal contexto, as ciências das linguagens não aceitam e nem mesmo dispõem de recursos para ser possível fomentar análises adequadas. Um dos principais traços do discurso digital nativo abordado por \textcite[p. 145]{paveau2021analise}, com base nos estudos de \textcite{bouchardon2011figures}, é a deslinearização, descrita como “[...] intervenção de elementos clicáveis no fio do discurso, que direcionam o leitor-escritor de um fio do discurso-fonte a um fio do discurso-alvo, instaurando uma relação entre dois discursos (por exemplo, uma hashtag ou um hiperlink)”, um vínculo feito a partir das escolhas feitas pelo internauta-leitor.

\textcite{paveau2021analise} define o termo “compósito” como elementos de discurso formados pela mistura do que é linguístico e o que é técnico, citando a \textit{hashtag}, o \textit{hiperlink}, os \textit{usernames} e quaisquer outros botões de ações e/ou reações em mídias sociais como exemplos de compósitos. Sem citar o termo “compósito”, por sua vez, \textcite{salgado2021dimensao} descreve que o engajamento gerado por meio de ações e reações digitais são uma espécie de “moeda de troca” pelo espaço on-line frequentado pelos usuários de maneira que, por intermédio dos algoritmos, mantém o ambiente em constante funcionamento. Dessa forma, nessa explicação, a autora dá enfoque aos modelos de negócio, isto é, empresas que utilizam desses recursos digitais para gerar números a seus engajamentos e, consequentemente, atingir seu público-alvo.

Apresentado ao público em 2010 e desenvolvido pelos engenheiros Kevin Systrom e Mike Kriegerm, o Instagram é hoje uma das mídias sociais mais populares, tendo, em 2017, cerca de 800 milhões de usuários ativos \cite{ramos2018instagram}, total que, em 2023, cresceu para 1,47 bilhão \cite{sampaio2024instagramtiktok}. O Instagram propõe aos seus usuários tamanha facilidade de conexões, pois é necessário apenas acesso à internet, um dispositivo eletrônico e o aplicativo móvel em questão \cite{ramos2018instagram}. Quanto à funcionalidade da rede, o Instagram, hoje, possui diversas funções além do compartilhamento de fotografias como: o \textit{reels}, descrito como “vídeos curtos de entretenimento”; o \textit{story}, definido como o compartilhamento de “momentos diários”; o \textit{messenger} (ou \textit{DMs}), espaços para conversas privativas ou em grupo com outros usuários; a função shopping, espaço dentro da rede social exclusivo para divulgação e venda de produtos, e, por fim, o “\textit{search \& explore}”, descrito pela plataforma como “descubra conteúdos e criadores com base nos seus interesses” \cite{instagram2024verified}.

Nas mídias sociais, os usuários são instruídos a construir um perfil autobiográfico no qual hão de compartilhar mídias, links e/ou textos. Neste espaço digital, a forma de escrita é compartilhada, mesmo que de modo indireto, e delineada no desenvolvimento de múltiplos signos os quais se trespassam “hiper e intertextualmente” dividindo a autoria destas significações com outros usuários, o que ocorre de maneira intermediada por ferramentas das redes como o curtir, compartilhar e comentar \cite[p. 125-126]{ramos2018instagram}. Por sua vez, o Instagram tem seu foco principal voltado para o alcance de seguidores, isto é, pessoas que estão vinculadas umas às outras, via os usuários de suas contas, com a intenção do acompanhamento daquilo que será publicado na plataforma.


\subsection{O discurso propagandista na esfera publicitária e sua expansão nas mídias sociais digitais}\label{sec-fmt-manuscrito}
O ato de linguagem, para \textcite[p. 59, grifo do autor]{charaudeau2010discurso}, “se realiza numa situação de comunicação normatizada, composta pela expectativa da troca e pela presença das restrições de encenação [...]. Essa situação, com suas expectativas, define também a posição de legitimidade dos sujeitos falantes: o \textit{em nome do que se fala}.”. Na Teoria Semiolinguística de Discurso, isso ocorre por meio do que é chamado de contrato e estratégias de comunicação. \textcite{charaudeau2012linguagem} explica o contrato de comunicação como uma dedução previamente feita entre pessoas de um mesmo corpo de práticas sociais e acrescenta que as estratégias de comunicação estão ligadas ao conceito de \textit{mise-en-scène} que corresponde a uma situação de encenação intencional elaborada.

Falar é uma forma de impor sua presença perante o outro, pois, enquanto se fala, quem está “do outro lado” não fala, dessa forma, há a construção de uma legitimidade do ato de falar e, ao mesmo tempo, de uma relação com quem se está a falar, dando a ele um lugar \cite{charaudeau2010discurso}. Nesse sentido, o discurso propagandista é definido por \textcite[p. 62]{charaudeau2010discurso} como um “discurso de incitação a fazer”. No âmbito publicitário, mais especificamente, \textcite{carvalho2010publicidade} chama o contrato estabelecido como “contrato de semi-engodos”, sendo o termo “engodos” definido como algo que foi feito para seduzir alguém \cite[p. 183]{ferreira1977minidicionario}. \textcite{carvalho2010publicidade} explica que o discurso publicitário se estrutura em três instâncias: a publicitária, a da concorrência e a do público, sendo marcado por um contrato de “semiengodo”, no qual o público reconhece o caráter persuasivo desse discurso, mas ainda assim, deseja acreditar nela. Conforme \textcite{carvalho2010publicidade}, a mensagem publicitária funciona como o braço direito da tecnologia moderna. Pelas palavras de \textcite{lagneau1974prolegomenos}, o autor menciona que a publicidade se transformou com o tempo, deixando de ser somente uma maneira de informar a venda de algo, obtendo, assim, uma lógica e linguagem própria em que a sedução e a persuasão tomam o lugar da objetividade e informatividade que se sobressaiam anteriormente.

Quanto à palavra na linguagem publicitária, \textcite{carvalho2010publicidade} afirma que sua função deixa de ser apenas de informar, recebendo uma atmosfera persuasiva, seja de modo claro ou dissimulado, não somente para vender o produto e/ou a marca, mas também para trazer o público para a sociedade do consumo. Sabendo que os recursos linguísticos provêm de uma força a qual pode influenciar e orientar percepções e pensamentos, implicando até mesmo em experiências e ações que podem trazer ou não informações aos indivíduos, a linguagem publicitária utiliza desses recursos estrategicamente como veículo de comunicação. 

\textcite{thompson2002midia} pontua que com as novas tecnologias e as mudanças de aparelhagens passando do sistema analógico para o digital, dessa forma, transformando também os meios de comunicação e seus sistemas de transmissão, flexibilizaram a comunicabilidade fazendo com que o indivíduo receptor deixasse de ser somente um interlocutor dos temas que chegam até ele. \textcite{mangold2009socialmedia} atestam que as mídias sociais ampliaram consideravelmente as interações entre os consumidores, já que existe a possibilidade de transmitir informações de maneira rápida, eficiente e sem censura. Os autores também afirmam que os consumidores se sentem mais atraídos pelas marcas que proporcionam a oportunidade de compartilhamento de \textit{feedbacks}, gerando um senso de comunidade que os aproxima da marca. Toda essa interação entre consumidores tende a passar confiança para novos compradores em potencial, dessa forma, fazendo com que as empresas que utilizam esses recursos posicionem-se em maior destaque perante as concorrentes que não adotam essas práticas na comunicabilidade com os seus clientes.


\section{Metodologia}\label{sec-formato}
Este estudo buscou analisar as estratégias tecnodiscursivas adotadas pela marca de dermocosméticos nativa digital Sallve em sua conta oficial da mídia social Instagram e em dois anúncios escolhidos, sendo um publicado dentro do perfil da marca e o outro patrocinado na plataforma. Pode-se qualificar a abordagem da pesquisa como de cunho qualitativo, uma vez que "tem o ambiente como fonte direta dos dados" \cite[p. 70]{pradanov2013metodologia}. Sendo o ambiente em questão o cibernético, a Internet diretamente, em que se coletou os dados analisados e apresentados, desconsiderando, assim, o uso de estatísticas.

Os meios utilizados para fomentar o estudo presente neste artigo foram o estudo de caso e a pesquisa bibliográfica. Destaca-se também que os estudos de \textcite{charaudeau2010discurso,paveau2021analise} proporcionaram a base teórico-metodológica necessária para a análise da @sallve.

A análise foi realizada em três esferas: 1) o perfil oficial da marca Sallve (@sallve) na rede social Instagram; 2) um anúncio publicado dentro do perfil da marca (inserido em um dos destaques do perfil); 3) um anúncio patrocinado pela marca dentro da plataforma. A escolha de fazer o estudo do perfil surgiu a partir do proposito de conhecer melhor a comunicabilidade da marca para com o público-alvo e os consumidores já fidelizados, além de investigar a forma como ela utiliza os recursos tecnodiscursivos disponibilizados pela plataforma Instagram para influenciar os usuários. A análise da marca em questão pode ser considerada uma amostra que dá possibilidade de outros estudos similares, no campo publicitário, para demais marcas e perfis na rede social Instagram.

Quanto aos anúncios (posteriormente denominados 1 e 2), buscou-se a compreensão mais direta de quais são as estratégias linguísticas presentes no marketing feito pela marca também dentro do Instagram. O anúncio 1 está presente dentro da primeira ferramenta de destaque denominado “descontudo”, e foi escolhido para o estudo por ser um dos primeiros a aparecer nesse destaque dado o seu caráter apelativo. Já o anúncio 2 foi selecionado após o contato direto com o perfil da marca na mídia social digital Instagram. Devido a compreensão algorítmica que é descrita por \textcite{paveau2021analise}, com base nos estudos de \textcite{cardon2013presentation} sobre a técnica \textit{machine learning}, como uma conversão do rastro digital em uma previsão do futuro, o anúncio em forma de \textit{post} patrocinado foi, dias depois do contato mencionado, sugerido no \textit{feed} de postagens de uma das autoras dentro plataforma despertando, assim, o interesse para a realização de seu estudo.


\section{Análise do perfil @sallve no Instagram}\label{sec-modelo}
Utilizando como base teórica a Análise do Discurso Digital \cite{paveau2021analise}, apresenta-se, nesta seção, a análise do perfil oficial da marca de dermocosméticos Sallve (@sallve), via o aplicativo oficial da mídia social, acessado por um dispositivo celular, baseada nas características, pautadas pela autora supracitada, que definem um discurso digital nativo.

\textcite{paveau2021analise} explica as seis características que um discurso digital nativo apresenta, sendo elas:

\begin{enumerate}
    \item Composição: elementos advindos da mistura do linguageiro e tecnológico de natureza informática, todavia, pode também ter relação com elementos visuais, sonoros, animados etc;
    \item Deslinearização: não desenvolvimento do discurso digital nativo em um eixo sintagmático específico do fio do discurso, característica ligada à hipertextualidade;
    \item Ampliação: desenvolvimento de conteúdos com mais de um colaborador por intermédio de recursos tecnológicos que disponibilizem a escrita colaborativa;
    \item Relacionalidade: a relação dos tecnodiscursos entre si, com a máquina, com o enunciador e com quem os lê, e a existência ocorre por meio da subjetividade do usuário on-line;
    \item Investigabilidade: o fato dos tecnodiscursos estarem integrados na memória da internet por meio de códigos e poderem ser pesquisados e redocumentados;
\item Imprevisibilidade: A incapacidade de o enunciador prever a maneira, a propagação ou conteúdo vinculado que o enunciado pode tomar na internet.
\end{enumerate}

Com base nesses conceitos preconizados por \textcite{paveau2021analise}, escolheu-se as cinco primeiras características para efetuar a análise do perfil @sallve no Instagram.

A marca Sallve é voltada para a indústria de dermocosméticos, foi fundada em 2019 de forma nativa, em outras palavras, diretamente na internet, e apresenta-se como uma “combinação perfeita entre conversas sinceras, alta performance e grande investimento em tecnologia e pesquisa”. Quando esta pesquisa estava sendo desenvolvida, seu perfil na mídia social Instagram contabilizou 1,2 milhões de seguidores. Uma característica que chama bastante atenção no perfil (ver \Cref{fig1}) é o emblema que indica que a conta é verificada, tal particularidade, associada ao alto número de seguidores, atribui à marca prestígio na rede em questão. \textcite{alves2020publipost} discorrem que esses aspectos dão poder de influência a quem os têm.

\begin{figure}[h!]
    \centering
    \begin{minipage}{.55\textwidth}
    \includegraphics[width=\linewidth]{Fig1.png}
    \caption{Perfil oficial da Sallve no Instagram.}
    \label{fig1}
    \source{captura de tela realizada pelas autoras em nov. 2024.}
    \end{minipage}
\end{figure}

Observando as primeiras informações além do nome e foto de perfil, verifica-se, no campo biográfico do perfil, o selo de usuário de outra mídia social associada ao Instagram, o Threads, a indicação de Saúde/beleza, e, abaixo, a seguinte frase: “fórmulas poderosas inspiradas por peles reais”. A frase em questão pode ser entendida de duas maneiras, sendo a primeira no meio informativo, fazendo ligação direta com a proposta e história da marca. Já a segunda maneira de interpretação pode ser feita por meio da intertextualidade, explicada por \textcite{koch1997coerencia} como um diálogo entre textos, e os autores também afirmam que a apropriação de provérbios ou ditos populares e sua replicação em textos é uma forma de manifestar esse recurso. Assim, pode-se interpretar tal frase como uma paráfrase de “baseado em fatos reais”, ditado famoso em filmes e documentários, com a intenção de despertar a atenção do público-alvo e aproximá-los do que se é contado. Em seguida, logo abaixo, lê-se a indicação de que a marca é vegana, isto é, os produtos não são feitos com ingredientes de origem animal, além de que se é frisado que não é cometido crueldade animal. Nesse caso, é dado a entender que os testes dos produtos não são feitos em animais. Nota-se o \textit{emoji} de um coelho, animal popularmente conhecido por servir de teste para produtos cosméticos. Por fim, as últimas informações fornecidas neste campo são o endereço principal, a hashtag \#vivasuapele e a frase “compre online e confira endereços nos links da bio”, indicando dois hiperlinks disponibilizados.

\textcite{paveau2021analise} define o \textit{hiperlink} como uma URL clicável que permite ao usuário transitar para outro ecossistema on-line. No âmbito tecnodiscursivo, ele opera como mecanismo de deslinearização: desloca o escritor-leitor do fio do discurso-fonte para um fio de discurso-alvo. No corpus analisado, observa-se esse movimento em dois tipos principais de direcionamento: um \textit{hiperlink} conduz à página inicial do site oficial da marca, enquanto outro remete a uma página interna, também do site oficial, que informa os pontos de venda dos produtos \cite{paveau2021analise}. É importante ressaltar que o \textit{hiperlink} também se caracteriza como um compósito, já que a dupla função: linguística e técnica. É interessante pontuar que o selo Threads e o indicativo de Saúde/beleza, ambos abaixo do nome do perfil, têm as mesmas características da função \textit{hiperlink}. Um redireciona o usuário (caso ele tenha o aplicativo em seu dispositivo) para a rede social \textit{Threads}, propondo mais um laço entre o internauta e a marca, enquanto o outro redireciona o usuário para uma página de explorar da plataforma que abrange outras empresas que também estão no Instagram com o ramo de dermocosméticos.

Quanto à hashtag, para \textcite[p. 223]{paveau2021analise}, esse mecanismo configura-se como “um segmento linguageiro precedido do signo \#”. A \textit{hashtag} também é uma forma de deslinearização \cite{paveau2021analise}, já que ela, assim como o hiperlink, redireciona o discurso quando clicado. No caso da \textit{hashtag} utilizada pela marca Sallve (\#vivasuapele) em seu perfil, esta leva para uma página específica da função explorar do Instagram em que há diversos \textit{publiposts} e \textit{reels} vinculados à marca, postados por diversos usuários, permitindo ao comprador em potencial saber mais dos produtos por meio de outras pessoas da plataforma. Além de uma forma de deslinearizar o discurso, a \textit{hashtag} também apresenta características de compósito, pois “se trata de um segmento ao mesmo tempo linguageiro (siglas, palavras, expressões ou frases inteiras) e técnico, devido a sua natureza clicável (assegurada pelo símbolo cerquilha \#) \cite[p. 120]{paveau2021analise}.

Abaixo do campo biográfico, há os botões seguir, mensagem, loja e opções, todos caracterizam-se como compósitos, já que atribuem em suas formas palavras e operações tecnológicas digitais. O botão seguir permite ao usuário acompanhar a marca diretamente e ver suas publicações no \textit{feed} pessoal, enquanto a mensagem realiza uma função de deslinearização, pois redireciona o internauta à uma página de contato direto com a marca Sallve na plataforma Instagram. Esse contato produz uma ação comunicativa e discursos colaborativos sem que os enunciadores sejam misturados, o que também configura uma característica de ampliação \cite{paveau2021analise}. Os tecnodiscursos produzidos nesse espaço podem dispor de características compósitas, uma vez que há a possibilidade de utilização de diferentes mídias, como imagens, áudios e \textit{hiperlinks}.

Por sua vez, o botão loja, redireciona para uma página de compras dentro do próprio Instagram, na qual é possível conferir produtos da marca, incluindo ver os preços. Caso haja o interesse do internauta, é possível redirecionar-se pelo produto para efetuar a compra dentro do site oficial. Esse botão funciona como o hiperlink do site, porém proporcionando uma espécie de catálogo digital na plataforma, sendo mais simplificado. Já o botão de opções oferece ao usuário duas possibilidades. Via os algoritmos da plataforma, apresenta-se a escolha de se ver contas relacionáveis à Sallve, na sua grande maioria, outras marcas de segmentos parecidos. Essa função permite ao internauta redirecionar-se, de maneira deslinearizada, para esses outros perfis.

Observa-se, na imagem, vários destaques, função que o \textcite{instagram2024homepage} define como sendo compilações de \textit{stories}, com duração de 24h, anteriormente postados. A Sallve, em sua conta, possui muitos destaques, a maioria deles sendo voltados para produtos específicos. Essa função permite ao usuário interagir com vários conteúdos “antigos” de uma só vez. Estes destaques possuem características deslinearizadoras, compósitas, ampliadoras e investigáveis, pois respectivamente: redirecionam o usuário para outros espaços dentro do perfil; apresentam a “mistura”, previamente citada, de palavras e operações tecnológicas, além de dispor da função “curtir” (que também é um compósito); permitem ao usuário responder os \textit{stories}, ampliando, assim, os discursos destes; e sobretudo são capazes de apresentar músicas, \textit{hiperlinks} e localizações, que funcionam como ferramentas que intencionam buscas e relacionam os usuários aos conteúdos.

A respeito do \textit{feed} do perfil @sallve, pode-se observar que este apresenta as cinco características analisadas na pesquisa. Os espaços de publicações, \textit{reels} e marcações apresentam linearidade quando vistas, de modo geral, em suas partes específicas, já que se organizam pela linearidade temporal de postagens, mas no aspecto tecnodiscursivo essa linearidade é rompida, deslinearizando a interação do usuário com os espaços e com as publicações. O internauta é redirecionado ao selecionar o espaço que escolhe visualizar (publicações, \textit{reels} e marcações) e é novamente redirecionado quando seleciona uma postagem específica. Sobretudo, essas postagens apresentam composição, pois há a possibilidade de curtir, enviar a outras pessoas, gerar \textit{link} destas postagens por intermédio de botões que são compósitos. A ampliação acontece pela opção de comentário, o que gera a ampliação discursiva, dentro dessa opção, pode-se até mesmo responder a outros comentários, inserindo-se em discursos dentro do discurso principal da postagem. Contudo, é possível que essa ampliação seja feita, de maneira privada, quando se envia algum post para outro usuário por mensagem. A investigabilidade do \textit{feed} ocorre por meio das hashtags das legendas e nas localizações e/ou músicas colocadas nas publicações, já que essas funções “aproximam” postagens semelhantes que compartilham desses aspectos, gerando a alternativa de pesquisa. Pode-se perceber a investigabilidade no espaço de marcações também, uma vez que esse espaço reúne todas as publicações em que a marca foi marcada por outros usuários.

Quanto à relacionalidade no perfil @sallve, o principal ponto percebido é a busca por uma aproximação com o internauta. As ferramentas dispostas pelo Instagram são utilizadas, além de suas outras características (como previamente analisado) para convidar o usuário a adentrar no universo virtual da Sallve. A relacionalidade independe de estratégias linguísticas, acontecendo de maneira natural e impensada. Compreendendo-se a natureza intrínseca da relacionalidade no meio digital, é perceptível no perfil @sallve como a marca apropria-se desta função para “roubar” a atenção do usuário que interage com o seu perfil. Assim, projetando, em cada ferramenta da plataforma Instagram, mais nuances da Sallve e apresentando mais possibilidades que a marca pode trazer àquele que escolher seus produtos.

Ao que se refere à característica da imprevisibilidade, \textcite{paveau2021analise} atesta que os discursos digitais nativos são, em parte, frutos dos programas tecnológicos e do sistema algorítmico advindo das plataformas digitais, fazendo com que estes sejam imprevisíveis ao público, isto é, o enunciador real (humano) não tem como saber de que forma aquilo que foi enunciado será interpretado por quem o lê. Analisando o perfil @sallve no Instagram, observa-se que a marca intenciona uma comunicação “limpa” e objetiva, estrategicamente, para evitar “ruídos” na mensagem a ser transmitida aos usuários da plataforma. Percebe-se que as características de composição, deslinearização, ampliação e investigabilidade, por intermédio das funções disponibilizadas pelo Instagram, são alinhadas ao perfil da Sallve para gerar uma apresentação coerente e direta parao público visitante mantendo, assim, um “equilíbrio” comunicativo quanto à possibilidade da imprevisibilidade. Desse modo, vê-se que esse movimento discursivo também corrobora a função de relacionalidade e evidencia a busca da marca pela proximidade com o internauta.

\subsection{Análise dos anúncios publicitários da marca Sallve}\label{sec-organizacao}
Foram selecionados dois anúncios, sendo um retirado do primeiro destaque do perfil do Instagram da marca Sallve (ver \Cref{fig2}), apresentado na imagem, nomeado “descontudo”; e o outro postado e patrocinado na plataforma, impulsionado por monetização (ver \Cref{fig3}). Para executar uma análise discursiva quanto à linguagem publicitária da marca, elaborou-se a análise pela teoria quanto à tipologia de discurso propagandista, desenvolvida no artigo “O discurso propagandista: uma tipologia” escrito pelo linguista francês \textcite{charaudeau2005analise}. Focou-se no caráter publicitário elaborado no artigo, que consta com a base de três instâncias, as quais formam um “dispositivo triangular” que contempla um objeto de fala, e o autor denomina como “contrato de semiengodos”.

\begin{figure}[h!]
    \centering
    \begin{minipage}{.65\textwidth}
    \includegraphics[width=\linewidth]{Fig2.png}
    \caption{Anúncio 1: Publicado pela Sallve em um de seus destaques no Instagram.}
    \label{fig2}
    \source{Captura de tela realizada pelas autoras em nov. 2024.}
    \end{minipage}
\end{figure}

Reconhece-se que o ambiente digital está cada vez mais competitivo quando o assunto é publicidade, pois o fluxo de informações cresce a cada instante, o que faz com que as marcas precisem investir cada vez mais na criatividade e buscar estratégias em sua linguagem para cativar o olhar do público \apud{kotler2017marketing}{sao2022algoritmos}. Tendo isso em vista, analisa-se o anúncio 1 selecionado da marca Sallve “pelos olhos” de \textcite{charaudeau2010discurso} que afirma que o discurso publicitário se desenvolve em três instâncias. Denomina-se instância publicitária aquela que projeta a credibilidade da marca em sua posição econômica no mercado, proferindo um discurso que exalte seu produto ao extremo em detrimento do produto da marca concorrente, assim, expressando um discurso superlativo que é projetado na instância de concorrência. Observando o anúncio selecionado, pode-se pontuar, primeiramente, a forma como a imagem dos produtos (protetores solares) são projetadas, trazendo uma visão de baixo, o que faz com que o tamanho deles pareça maior do que realmente são, dando-lhes uma atmosfera de elevação. Além disso, há, abaixo, uma frase entre aspas, especificada em letras miúdas, na extremidade inferior do anúncio, como sendo extraído de um \textit{feedback} de cliente postado no site, a qual diz: “são perfeitos, não uso outra marca de protetor solar”. A frase, junto à superlatividade imagética dos produtos no anúncio, relaciona-se inteiramente a tais instâncias de \textcite{charaudeau2010discurso}, pois expressa uma exaltação da marca Sallve em desfavor de marcas de dermocosméticos concorrentes. Contudo, isso foi aplicado de modo tênue, transferindo a “responsabilidade” do comentário aos consumidores, uma vez que ele foi retirado de um \textit{feedback}, enfatizado como real, presente no site da marca, fazendo com que a afirmação soe mais amigável ao público.

Dessarte, \textcite{charaudeau2005analise} propõe que, no discurso publicitário, o foco no objeto de fala (o produto em questão) apresenta, de maneira dupla, aquilo que está sendo vendido. Primeiro é feito como um “benefício absoluto”, sendo oferecido como uma espécie de sonho e, por conseguinte, como um “bem de consumo”, sendo ele a única maneira de atingir o sonho previamente proposto. A frase “são perfeitos, não uso outra marca de protetor solar”, colocada no anúncio, cumpre ambas as funções, porque dá ao público a ideia de que qualquer um dos três produtos apresentados no anúncio são tão eficientes que chegam a ser um item de necessidade (benefício absoluto). Ao mesmo tempo, essa afirmação de eficiência reforça o fato de que essa necessidade só pode ser suprida por meio dos protetores solares dispostos no anúncio (bem de consumo) e essa eficiência atestada no anúncio publicitário fica mais evidente para o público pelo elemento de complemento da frase representativa de resenha: as cinco estrelas que ficam acima do \textit{feedback}. Algo que também reforça ambas as funções (bem de consumo e benefício absoluto) é o neologismo “descontudo” (desconto + tudo) ligado ao desconto proposto (até 50\% off), uma vez que implica no interlocutor do discurso a intenção de que ele estará em vantagem caso supra a necessidade nele incitada naquele momento, pois os protetores estão em oferta, então, além do benefício oferecido pelos produtos, ele conseguirá beneficiar-se economicamente.

Por fim, a última instância de \textcite{charaudeau2010discurso} é a do público, a qual carrega uma dupla posição, sendo a de “consumidor comprador potencial” e a de “consumidor efetivo da publicidade”. A primeira é um indivíduo em ausência de algo e que deve crer que a busca pelo preenchimento desse vazio está por meio do que lhe é apresentado. A segunda é alguém convidado a contemplar a instância publicitária e sua participação nessa encenação. Contudo, isso pode ser dúbio porque há a possibilidade de uma “mescla” entre posições, pois existe a perspectiva de se apreciar uma publicidade sem o intuito de consumir o que é vendido. No anúncio 1, proposto à análise, é percebido uma posição muito mais forte para o apelo de um consumidor comprador potencial do que para um consumidor efetivo da publicidade, pois, como discorrido na instância anterior, a mensagem em forma de \textit{feedback} traz uma fortíssima intenção de convencer o interlocutor do discurso de que os produtos são tão excelentes que tornam-se uma espécie de necessidade, já que, segundo o que está escrito, somente eles cumprem sua função (a de proteger a pele) de maneira efetiva, e nenhuma outra marca faz daquela forma.

Todavia, quanto ao que foi analisado no parágrafo anterior, é importante também pensar no quesito público-alvo, porque sendo uma marca-nativa digital, a Sallve atinge um número de pessoas, de certa forma, mais restrito, como os usuários da internet (especialmente do Instagram) e, seguindo a lógica algorítmica, pessoas interessadas em dermocosméticos. Alguém de fora do “nicho dermocosmético”, por exemplo, poderia não se sentir tão impactada pelo anúncio, sendo, talvez, muito mais levada a apreciar a mensagem transmitida pelo “descontudo” aliada ao tamanho do desconto (até 50\% off) que o produto em si, dessa forma, tendo maior tendência a tornar-se um apreciador (consumidor efetivo da publicidade) do que inclinado a querer comprar alguma das mercadorias.

Contudo, se o anúncio impactar demasiadamente essa pessoa “de fora” e, eventualmente, ela se tornar alguém interessada por dermocosméticos, há a probabilidade de ela se lembrar da Sallve, ou até mesmo do anúncio em si, e buscar saber mais a respeito, o que pode levá-la a comprar, se não os produtos anunciados, algum outro da marca.

\begin{figure}[h!]
    \centering
    \begin{minipage}{.65\textwidth}
    \includegraphics[width=\linewidth]{Fig3.png}
    \caption{Anúncio 2: Patrocinado pela Sallve por meio de monetização dentro do Instagram.}
    \label{fig3}
    \source{Captura de tela realizada pelas autoras em dez. 2024.}
    \end{minipage}
\end{figure}

A Meta (anteriormente nomeada Facebook), empresa responsável pela mídia social digital Instagram, explica que o patrocínio – ou turbinar, termo utilizado para a ferramenta em questão – de uma publicação por parte de uma conta empresarial pode ocorrer de duas formas: a conta pode escolher uma publicação previamente postada para “turbinar” ou pode “subir” uma publicação nova, específica para anunciar \cite{meta2025anuncios}. Independente de qual seja o meio para se fazer um anúncio, esta ação é paga e, depois que o impulsionamento do anúncio acontece, ele é enviado algoritmicamente para os usuários em seus \textit{feeds} em meio às postagens, dentro da página explorar, entre os \textit{reels} ou até mesmo entre os \textit{stories} assistidos \cite{serasa2023instagram}. Tais anúncios levam uma espécie de “etiqueta” de patrocinado, deixando claro para os usuários o porquê de eles estarem vendo determinada publicação em meio a sua conta. Na postagem patrocinada selecionada para análise, a “etiqueta” fica no topo, abaixo do nome da marca Sallve.

Sob a ótica do contrato de semiengodos proposto por \textcite{charaudeau2010discurso}, observa-se que a instância publicitária projetada no anúncio 2 mantém seu foco apenas na exaltação do produto em pauta: o “kit anti cravo”, dessa vez, não direcionando o discurso publicitário tão diretamente à concorrência, como é feito no anúncio 1. A superlatividade neste anúncio gira em torno da exaltação por meio de texto verbal, com as frases “previne, trata e renova” e “tratamento dia e noite para dar adeus aos cravos”, evidenciando as qualidades que o kit dispõe e as vantagens que o interlocutor do discurso tem a obter a partir desta mercadoria. A instância de concorrência, aqui é projetada pela Sallve sutilmente, sem foco na rivalidade de modo direto, e concentra a credibilidade da marca na grandeza do produto oferecido, visando à atenção do usuário do Instagram. O que faz com que o foco seja inteiramente na própria marca, sem dar espaço para a cogitação de dermocosméticos de concorrentes.

O objeto de fala no anúncio 2 (“kit anti cravo”) coincide com o conceito descrito por \textcite{charaudeau2010discurso}, previamente mencionado nesta seção, mostrando ao interlocutor do discurso uma ideia duplicada daquilo que está à venda. A noção de benefício absoluto é refletida pela marca nas palavras opostas “dia” e “noite” que aparecem tanto em destaque acompanhadas do nome do produto (“kit anti cravo”), quanto na frase “tratamento dia e noite para dar adeus aos cravos”, dando ao usuário do Instagram uma consciência de completude em relação à mercadoria. O “sonho vendido” é o de que os produtos do kit sejam capazes de eliminar os cravos durante um dia inteiro, por conseguinte, de uma vez por todas. A frase “previne, trata e renova” corrobora para essa idealização da completude, pois atesta que os produtos, além de eliminarem, previnem o aparecimento de cravos, tratando e renovando a pele de quem os usar. Por sua vez, a noção de bem de consumo é atestada no anúncio 2 por intermédio dos produtos compostos no “kit anti cravo” da Sallve, sendo eles o limpador enzimático em pó e o tônico renovador. Estes configuram o meio unilateral para a realização do “sonho” anteriormente mencionado. Nesse anúncio, também há uma idealização de privilégio financeiro, oferecida por meio de um desconto monetário, a qual contempla as funções de benefício absoluto e bem de consumo: “Comprando o kit você economiza R\$30”. Vê-se que, estrategicamente, o início da frase está em letras menores e o final está em destaque, pois o foco é prender a atenção do usuário por meio da ideia de que ele estará em vantagem econômica sobre a marca, fazendo-o ignorar momentaneamente o fato de que terá que pagar um valor por isso.

Quanto à instância de público, a qual, segundo \textcite{charaudeau2010discurso}, trata da posição dupla de consumidor comprador potencial e de consumidor efetivo da publicidade, nota-se, no anúncio 2, uma forte inclinação para a primeira delas (consumidor comprador potencial), visto que o discurso publicitário propagado pela Sallve é o de oferecer algo (dar adeus aos cravos) para alguém que está busca disso (pessoas que sofrem com cravos). A partir da descrição de \textcite{charaudeau2010discurso} sobre o consumidor comprador potencial – alguém que está em falta de algo e procura uma maneira de compensar isso com o que lhe é oferecido por meio da publicidade, compreende-se que a intenção da marca com esse anúncio é a de atingir pessoas que estejam precisando daquilo que o seu produto tem a oferecer. Essa ideia é reforçada pelo botão comprar agora, que fica abaixo do post de publicidade, e leva o interlocutor do discurso diretamente para a página do produto (“kit anti cravo”) dentro do site oficial da Sallve.

O fato de o anúncio ser patrocinado, para que haja o seu impulsionamento entre os usuários da plataforma Instagram, também deixa isso evidente, já que a marca, com tal ação, busca um reconhecimento entre o público e, consequentemente, espera vender o seu produto. Porém, essa intenção não anula a posição de um possível consumidor efetivo da publicidade, já que a possibilidade de o interlocutor do discurso somente apreciar a instância publicitária, sem participar dela inteiramente, permanece existindo. Como explicado anteriormente nesta seção, as posições são dúbias, podem coexistir e até implicarem entre si, fazendo com que o consumidor efetivo da publicidade se interesse pelo assunto anunciado (dermocosméticos) e passe a consumi-lo.

Refletindo acerca do que foi analisado até então, com base no contrato de semiengodos de \textcite{charaudeau2010discurso}, é interessante contrastar as estratégias discursivas de ambos os anúncios selecionados. O primeiro, publicado diretamente dentro do perfil da Sallve na mídia social Instagram, tem um discurso mais objetivo, contudo, mais apelativo, uma vez que a superlatividade utilizada na instância publicitária é diretamente em detrimento da instância de concorrência. Em teoria, o público visado ao discurso desta publicação é majoritariamente pessoas interessadas pela marca que buscam ser ou já são clientes: os seguidores. Sendo assim, não há a necessidade de se fazer uma grande apresentação do produto e/ou da marca em si. A frase “são perfeitos, não uso outra marca de protetor solar”, indicada como parte dos \textit{feedbacks} do produto no site, em conjunto à ilustração de cinco estrelas, comunica a credibilidade da Sallve ao público em questão de modo que não o gere dúvida, mas sim relembre o porquê de essas pessoas terem desenvolvido interesse pela marca. Já, pela instância de público e sua dupla posição, pode-se dizer que ambas, a de consumidor comprador potencial e a de consumidor efetivo da publicidade, coexistem nessa situação, já que há tanto interesse real nos produtos quanto conivência à instância publicitária feita pela Sallve.

Por conseguinte, no anúncio 2, o qual é propagado no Instagram por meio de uma ferramenta de patrocínio, entende-se que o propósito discursivo maior da Sallve é o alcance de um novo público. Esse anúncio comunica mais informações sobre seu produto do que o anterior, buscando explicar o que essa mercadoria tem a oferecer e, por meio disso, apresentar-se a esse público, mostrando, assim, a forma como a marca trabalha e com o que está comprometida. É interessante observar que a Sallve faz questão de replicar os ingredientes visíveis nas embalagens dos produtos do “kit anti cravo” no corpo textual da publicidade, reforçando ao seu público-alvo o que é oferecido nessas mercadorias. Para aqueles que são “novos” no mercado de dermocosméticos, essa estratégia pode passar despercebida, mas, para quem já tem conhecimento acerca do assunto, isso pode reforçar a atenção e o interesse pelo produto.

Essa tática não deixa de contribuir, mesmo que sutilmente, para a superlatividade do discurso da instância publicitária em detrimento da instância de concorrência, já que quanto maior for a atenção dedicada à própria Sallve, menor será a chance do consumo de uma marca rival. Como visto previamente nesta seção, a posição que interlocutor do discurso mais tende a ocupar nesse anúncio publicitário, conforme o dispositivo triangular de \textcite{charaudeau2010discurso}, é a de consumidor comprador potencial, pois a Sallve visa a isso quando usa a ferramenta de patrocínio disponibilizada pelo Instagram. E, caso se atinja o que é esperado, há a probabilidade de que esse interlocutor do discurso se torne seguidor da marca na rede social, o que o leva a assumir, simultaneamente, o lugar de consumidor efetivo da publicidade.

\section{Considerações finais}\label{sec-organizacao-latex}
Visto a pertinência do tema e a oportunidade de contribuição, o objetivo deste artigo foi analisar as estratégias tecnodiscursivas que são adotadas na linguagem publicitária da marca nativa digital de dermocosméticos Sallve na plataforma Instagram. Conduzida a análise proposta, e diante das evidências apresentadas, percebeu-se, quanto ao tecnodiscurso do perfil @sallve, o “jogo” de persuasão e influência imposto pela Sallve, por intermédio das ferramentas do Instagram, o qual visa ao desenvolvimento de uma relação próxima entre ambos os sujeitos do discurso (marca e usuário). Já a respeito da linguagem publicitária, notou-se praticidade e aproximação nas estratégias discursivas escolhidas para os anúncios. É verificável a aplicação de abordagem direta, devido à limitação do tempo de atenção nas mídias sociais, porém também é evidenciada a busca por mostrar-se transparente para com o público, tentando cativá-lo e ganhar sua confiança.

Os estudos realizados evidenciam a relevância dos algoritmos no tecnodiscurso, influenciando o consumo de conteúdo e os comportamentos dos usuários, sobretudo na publicidade digital, marcada por anúncios personalizados e discursos mais dinâmicos. As transformações tecnológicas digitais ampliaram as possibilidades discursivas, integrando texto, multimídia e interação. Assim, a comunicação, especialmente a publicitária, tornou-se mais atrativa e adaptável às constantes inovações do meio digital, ressaltando a importância de manter o campo linguístico atualizado e aberto a novas investigações.

\printbibliography\label{sec-bib}
% if the text is not in Portuguese, it might be necessary to use the code below instead to print the correct ABNT abbreviations [s.n.], [s.l.]
%\begin{portuguese}
%\printbibliography[title={Bibliography}]
%\end{portuguese}


%full list: conceptualization,datacuration,formalanalysis,funding,investigation,methodology,projadm,resources,software,supervision,validation,visualization,writing,review
\begin{contributors}[sec-contributors]
\authorcontribution{Maiara Luiza Schönardie dos Santos}[conceptualization,datacuration,methodology,visualization,writing,review]
\authorcontribution{Dieila dos Santos Nunes}[formalanalysis,projadm,supervision,validation,review]
\end{contributors}

\begin{dataavailability}
\txtdataavailability{databody} % options: dataavailable, dataonly, databody, datanotav, nodata
\end{dataavailability}


\end{document}

