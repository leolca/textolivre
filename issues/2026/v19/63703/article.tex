% !TEX TS-program = XeLaTeX
% use the following command:
% all document files must be coded in UTF-8
\documentclass[portuguese]{textolivre}
% build HTML with: make4ht -e build.lua -c textolivre.cfg -x -u article "fn-in,svg,pic-align"

\journalname{Texto Livre}
\thevolume{19}
%\thenumber{1} % old template
\theyear{2026}
\receiveddate{\DTMdisplaydate{2025}{12}{31}{-1}} % YYYY MM DD
\accepteddate{\DTMdisplaydate{2026}{1}{8}{-1}}
\publisheddate{\DTMdisplaydate{2026}{1}{12}{-1}}
\corrauthor{Harlen Cardoso Divino}
\articledoi{10.1590/1983-3652.2026.63703}
%\articleid{NNNN} % if the article ID is not the last 5 numbers of its DOI, provide it using \articleid{} commmand 
% list of available sesscions in the journal: articles, dossier, reports, essays, reviews, interviews, editorial
\articlesessionname{reviews}
\runningauthor{Divino} 
%\editorname{Leonardo Araújo} % old template
\sectioneditorname{Daniervelin Pereira~\orcid{0000-0003-1861-3609}}
\layouteditorname{Leonardo Araújo~\orcid{0000-0003-3884-2177}}

\title{Resenha de Docência aumentada: um guia para o ensino superior na era da inteligência artificial generativa}
\othertitle{Review of Docência aumentada: um guia para o ensino superior na era da inteligência artificial generativa}
% if there is a third language title, add here:
%\othertitle{Artikelvorlage zur Einreichung beim Texto Livre Journal}

\author[1]{Harlen Cardoso Divino~\orcid{0000-0001-8750-9501}\thanks{Email: \href{mailto: harlen.divino@posgrad.ufsc.br}{harlen.divino@posgrad.ufsc.br}}}
\affil[1]{Universidade Federal de Santa Catarina, Programa de
Pós-Graduação em Engenharia e Gestão do Conhecimento, Centro Tecnológico,
Florianópolis, SC, Brasil.}

\addbibresource{article.bib}
% use biber instead of bibtex
% $ biber article

% used to create dummy text for the template file
\definecolor{dark-gray}{gray}{0.35} % color used to display dummy texts
\usepackage{lipsum}
\SetLipsumParListSurrounders{\colorlet{oldcolor}{.}\color{dark-gray}}{\color{oldcolor}}

% used here only to provide the XeLaTeX and BibTeX logos
\usepackage{hologo}

% if you use multirows in a table, include the multirow package
\usepackage{multirow}

% provides sidewaysfigure environment
\usepackage{rotating}

\usepackage{endnotes} % Inclui o pacote endnotes

% CUSTOM EPIGRAPH - BEGIN 
%%% https://tex.stackexchange.com/questions/193178/specific-epigraph-style
\usepackage{epigraph}
\renewcommand\textflush{flushright}
\makeatletter
\newlength\epitextskip
\pretocmd{\@epitext}{\em}{}{}
\apptocmd{\@epitext}{\em}{}{}
\patchcmd{\epigraph}{\@epitext{#1}\\}{\@epitext{#1}\\[\epitextskip]}{}{}
\makeatother
\setlength\epigraphrule{0pt}
\setlength\epitextskip{0.5ex}
\setlength\epigraphwidth{.7\textwidth}
% CUSTOM EPIGRAPH - END

% to use IPA symbols in unicode add
%\usepackage{fontspec}
%\newfontfamily\ipafont{CMU Serif}
%\newcommand{\ipa}[1]{{\ipafont #1}}
% and in the text you may use the \ipa{...} command passing the symbols in unicode

% LANGUAGE - BEGIN
% ARABIC
% for languages that use special fonts, you must provide the typeface that will be used
% \setotherlanguage{arabic}
% \newfontfamily\arabicfont[Script=Arabic]{Amiri}
% \newfontfamily\arabicfontsf[Script=Arabic]{Amiri}
% \newfontfamily\arabicfonttt[Script=Arabic]{Amiri}
%
% in the article, to add arabic text use: \textlang{arabic}{ ... }
%
% RUSSIAN
% for russian text we also need to define fonts with support for Cyrillic script
% \usepackage{fontspec}
% \setotherlanguage{russian}
% \newfontfamily\cyrillicfont{Times New Roman}
% \newfontfamily\cyrillicfontsf{Times New Roman}[Script=Cyrillic]
% \newfontfamily\cyrillicfonttt{Times New Roman}[Script=Cyrillic]
%
% in the text use \begin{russian} ... \end{russian}
% LANGUAGE - END

% EMOJIS - BEGIN
% to use emoticons in your manuscript
% https://stackoverflow.com/questions/190145/how-to-insert-emoticons-in-latex/57076064
% using font Symbola, which has full support
% the font may be downloaded at:
% https://dn-works.com/ufas/
% add to preamble:
% \newfontfamily\Symbola{Symbola}
% in the text use:
% {\Symbola }
% EMOJIS - END

% LABEL REFERENCE TO DESCRIPTIVE LIST - BEGIN
% reference itens in a descriptive list using their labels instead of numbers
% insert the code below in the preambule:
%\makeatletter
%\let\orgdescriptionlabel\descriptionlabel
%\renewcommand*{\descriptionlabel}[1]{%
%  \let\orglabel\label
%  \let\label\@gobble
%  \phantomsection
%  \edef\@currentlabel{#1\unskip}%
%  \let\label\orglabel
%  \orgdescriptionlabel{#1}%
%}
%\makeatother
%
% in your document, use as illustraded here:
%\begin{description}
%  \item[first\label{itm1}] this is only an example;
%  % ...  add more items
%\end{description}
% LABEL REFERENCE TO DESCRIPTIVE LIST - END


% add line numbers for submission
%\usepackage{lineno}
%\linenumbers

\begin{document}
\maketitle

\begin{figure}[htbp]
\centering
\begin{minipage}{.5\textwidth}
\includegraphics[width=\textwidth]{Fig1.png}
 \caption*{\fullcite{pereira_santos2025docencia}}
 %\caption{\fullcite{nascimento_formacao_2019}.}
 \label{fig01}
\end{minipage}
\end{figure}

\begin{polyabstract}
\begin{abstract}
O presente texto é uma resenha da obra Docência Aumentada \cite{pereira_santos2025docencia}, que aborda o impacto da inteligência artificial generativa no ensino superior. O objetivo da obra é avaliar a proposta de inteligência híbrida e a centralidade da mediação docente frente aos processos de automação cognitiva. A resenha considera a organização progressiva da obra e seus referenciais normativos, com destaque para a articulação entre marcos éticos internacionais e \textit{frameworks} operacionais voltados ao contexto das universidades públicas estaduais. Os resultados da avaliação crítica indicam que o livro oferece uma via intermediária consistente para a integração da inteligência artificial às práticas pedagógicas, embora apresente limitações quanto à exploração empírica longitudinal e à problematização da governança algorítmica. Conclui-se que a obra estabelece um marco relevante para a soberania pedagógica e para o redesenho ético dos processos formativos no ensino superior contemporâneo.

\keywords{Docência aumentada \sep Inteligência artificial generativa \sep Ensino superior \sep Mediação pedagógica \sep Soberania pedagógica}
\end{abstract}

\begin{english}
\begin{abstract}
This text is a review of the book Docência Aumentada \cite{pereira_santos2025docencia}, which addresses the impact of generative artificial intelligence on higher education. The objective of this book is to assess the proposal of hybrid intelligence and the centrality of teacher mediation in the face of cognitive automation processes. The review considers the progressive organization of the work and its normative frameworks, highlighting the articulation between international ethical guidelines and operational models applied to the context of public state universities. The results of the critical evaluation indicate that the book offers a consistent middle-ground approach for integrating artificial intelligence into pedagogical practices, although it presents limitations regarding longitudinal empirical exploration and the discussion of algorithmic governance. It is concluded that the work establishes a relevant benchmark for pedagogical sovereignty and for the ethical redesign of formative processes in contemporary higher education.

\keywords{Augmented teaching \sep Generative artificial intelligence \sep Higher education \sep
Pedagogical mediation \sep Pedagogical sovereignty}
\end{abstract}
\end{english}
% if there is another abstract, insert it here using the same scheme
\end{polyabstract}

\section{Introdução}\label{sec-intro}
A incorporação da inteligência artificial generativa (IAGen) ao ensino superior tem provocado um deslocamento profundo nas formas de ensinar, aprender e produzir conhecimento. Conforme aponta \textcite{harari2018licoes}, a humanidade enfrenta o desafio de lidar com tecnologias que podem hackear não apenas sistemas, mas a própria cognição humana. Entre discursos marcados pelo entusiasmo tecnocrático e posições resistentes à automação, a obra "Docência Aumentada", de Ricardo Pereira e Neri dos Santos, propõe uma via intermediária: a centralidade da mediação humana no uso educacional da IA. Os autores aportam sólida expertise acadêmica ao debate: Ricardo Pereira é doutor pela Universidade Federal de Santa Catarina (UFSC) e pesquisador do NAPI Educação do Futuro, com foco em políticas de inovação e competências docentes. Neri dos Santos, também doutor pela UFSC, possui trajetória consolidada nas áreas de engenharia do conhecimento e inteligência artificial, atuando na formulação de frameworks operacionais para a transformação digital nas universidades. O conceito de “docência aumentada” defende que a tecnologia deve operar como extensão das capacidades docentes, ampliando possibilidades pedagógicas e avaliativas em uma relação de inteligência híbrida e complementar.Estruturalmente, a obra organiza-se de forma progressiva: inicia-se com uma "Apresentação" dedicada à fundamentação crítica e genealogia da IAGen. O Capítulo 1 mergulha na reconfiguração das competências digitais de docentes e discentes, explorando os eixos de design e mediação. O Capítulo 2 detalha a aplicação prática de metodologias inovadoras, como PBL, CBL e Sala de Aula Invertida, sob a lógica da inteligência híbrida. O volume é robustecido por cinco apêndices técnicos (A a E), que funcionam como guias práticos, incluindo um Laboratório de Prompts baseado no \textit{framework} C.R.I.A.R. e diretrizes estratégicas voltadas às universidades públicas.

\section{Desenvolvimento e análise crítica}\label{sec-normas}
A obra apresenta uma organização clara, contextualizando o surgimento da IAGen e suas implicações para a autoria acadêmica. A obra aprofunda a transição de paradigma da Era Industrial para a Era Digital, diferenciando com precisão os termos digitização (conversão analógico-digital), digitalização (otimização de processos) e a transformação digital propriamente dita, entendida como uma quebra de paradigma cultural e organizacional. \textcite{pereira_santos2025docencia} argumentam que a universidade corre o risco de automatizar a ineficiência se não compreender que a IAGen é uma alavanca para catalisar uma transformação pedagógica soberana, e não apenas uma ferramenta de produtividade. Dentro desse ecossistema, o professor é reposicionado como um designer e curador de processos, afastando-se de tarefas mecânicas que a máquina executa para focar na mediação de debates complexos e na provocação do pensamento crítico. Os autores realizam uma distinção fundamental entre digitização e transformação digital, alinhando-se à perspectiva de \textcite{stolterman2004goodlife}, ao argumentar que a verdadeira inovação requer uma mudança na essência das práticas humanas e sociais, e não apenas a adoção de novas ferramentas.

A noção de docência aumentada ganha concretude ao ser associada ao planejamento pedagógico assistido por IA e à redefinição dos processos avaliativos. Um dos pontos altos da obra, que a qualifica como uma referência estratégica, é a apresentação do \textit{framework} C.R.I.A.R. (Contexto, Role/Papel, Instrução, Audiência e Refinamento). Este modelo mnemônico sistematiza a Engenharia de Prompt como um diálogo crítico com a máquina, permitindo que o docente desenhe instruções que testem os limites do sistema e evitem respostas genéricas. Ao detalhar cada etapa, desde a atribuição de uma persona à IA (Role) até a definição de restrições de formato e tom (Refinamento), os autores transformam o uso da tecnologia em um ato de design deliberado, mitigando o que chamam de 'plágio de processo'. 

Nesse sentido, o livro dialoga com o DigCompEdu \cite{comissaoeuropeia2017digcompedu}, que estabelece as competências digitais necessárias para que educadores ajam como facilitadores críticos da aprendizagem. Ao evitar soluções simplistas e insistir na formação docente contínua, os autores garantem que a tecnologia não se imponha como instância autônoma de decisão pedagógica. A aplicação prática da docência aumentada é demonstrada através de guias de implementação em metodologias ativas consagradas. Na Sala de Aula Invertida, por exemplo, o livro sugere o uso da IAGen para criar trilhas personalizadas de material pré-aula, otimizando o tempo síncrono para interações humanas profundas. 

Já em abordagens baseadas em desafios (CBL) ou problemas (PBL), a obra descreve a IA como um simulador de cenários complexos e motor de prototipagem rápida. Essa integração, contudo, é sempre acompanhada por uma 'vigilância epistemológica', alertando para o paradoxo da facilidade, no qual o estudante pode delegar o esforço cognitivo ao algoritmo, resultando em um aprendizado superficial e em uma perigosa ilusão de competência.

Assim, a  noção de docência aumentada ganha concretude ao ser associada a \textit{frameworks} operacionais, com destaque para o modelo C.R.I.A.R. (Contexto, Role, Instrução, Audiência e Refinamento). Tal ferramenta orienta o docente no design deliberado de \textit{prompts}, promovendo o que os autores chamam de diálogo crítico com a máquina para evitar a superficialidade cognitiva. Desta forma, o uso de metodologias como a Sala de Aula Invertida \cite{bergmann2016salainvertida} e a Aprendizagem Baseada em Projetos \cite{bender2015projetos} é potencializado na obra por meio do \textit{framework} C.R.I.A.R., que orienta uma engenharia de \textit{prompt} pedagogicamente responsável. 

Um mérito relevante do livro reside em sua ancoragem institucional nas universidades paranaenses, conferindo densidade política ao texto. Ao tratar da ética, a obra ecoa a \textit{Recommendation on the Ethics of Artificial Intelligence} \cite{unesco2022ethicsai}, enfatizando a transparência e a responsabilidade humana sobre os sistemas algorítmicos.

Entretanto, para o pesquisador que busca rigor científico, algumas limitações devem ser apontadas. Observa-se uma menor exploração empírica de dados longitudinais, o que limita a avaliação da efetividade real das propostas no cotidiano universitário. Ademais, o debate sobre a governança algorítmica e a proteção de dados poderia ser mais exaustivo. Como discute \textcite{floridi2014fourth} na "Quarta Revolução", a infoesfera reconfigura nossa realidade de modo que a assimetria entre as plataformas globais e as instituições públicas pode ferir a soberania intelectual se não houver mecanismos de controle institucional robustos. 

Portanto, este cenário é agravado pelas projeções do \textcite{wef2023futurejobs}, que indicam uma reestruturação massiva do mercado de trabalho, exigindo que a universidade repense sua função social além da mera formação técnica. A fundamentação ética da obra é ancorada na Recomendação sobre a Ética da IA da UNESCO \cite{unesco2022ethicsai}, estabelecendo quatro pilares inegociáveis para a prática docente: Direitos Humanos, Conscientização Crítica, Supervisão Humana e Governança Colaborativa.

Segundo os autores, a responsabilidade final pela veracidade e originalidade do conhecimento permanece indelevelmente humana, exigindo que o professor atue como um conector que utiliza a tecnologia para construir comunidades de aprendizagem, e não apenas para entregar conteúdo.

\section{Conclusão}\label{sec-conduta}
Ainda assim, "Docência Aumentada" representa uma contribuição indispensável ao campo da educação. Ao reafirmar o papel do professor como mediador crítico em ambientes digitais complexos, os autores oferecem subsídios para que gestores e docentes integrem a IA de forma ética e reflexiva. 
A obra não apenas descreve uma tendência, mas propõe um horizonte normativo para a universidade contemporânea: uma instituição capaz de incorporar tecnologias avançadas sem abdicar de sua função formativa e crítica, estabelecendo um marco para a soberania pedagógica diante da automação.




\printbibliography\label{sec-bib}
% if the text is not in Portuguese, it might be necessary to use the code below instead to print the correct ABNT abbreviations [s.n.], [s.l.]
%\begin{portuguese}
%\printbibliography[title={Bibliography}]
%\end{portuguese}

\section*{\fontsize{10}{12}\selectfont Nota de transparência metodológica}
A elaboração desta resenha contou com apoio pontual de ferramentas de inteligência artificial generativa, utilizadas na revisão linguística e organização estrutural. A análise crítica e os juízos avaliativos são de inteira responsabilidade do autor, em conformidade com os princípios de ética científica ao qual se exige no âmbito da academia. 

\begin{dataavailability}
\txtdataavailability{nodata} % options: dataavailable, dataonly, databody, datanotav, nodata
\end{dataavailability}

\end{document}

