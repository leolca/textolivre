% !TEX TS-program = XeLaTeX
% use the following command:
% all document files must be coded in UTF-8
\documentclass[portuguese]{textolivre}
% build HTML with: make4ht -e build.lua -c textolivre.cfg -x -u article "fn-in,svg,pic-align"


\journalname{Texto Livre}
\thevolume{19}
%\thenumber{1} % old template
\theyear{2026}
\receiveddate{\DTMdisplaydate{2025}{7}{28}{-1}} % YYYY MM DD
\accepteddate{\DTMdisplaydate{2025}{9}{28}{-1}}
\publisheddate{\DTMdisplaydate{2026}{1}{9}{-1}}
\corrauthor{Patricio Dugnani}
\articledoi{10.1590/1983-3652.2026.60667}
%\articleid{NNNN} % if the article ID is not the last 5 numbers of its DOI, provide it using \articleid{} commmand 
% list of available sesscions in the journal: articles, dossier, reports, essays, reviews, interviews, editorial
\articlesessionname{essays}
\runningauthor{Dugnani} 
%\editorname{Leonardo Araújo} % old template
\sectioneditorname{Daniervelin Pereira~\orcid{0000-0003-1861-3609}}
\layouteditorname{Saula Cecília~\orcid{0009-0006-3069-8480}}

\title{Interculturalidade e meios de comunicação: da extensão dos sentidos à aceleração das trocas culturais}
\othertitle{Interculturality and media: from the extension of the senses to the acceleration of cultural exchanges}
% if there is a third language title, add here:
%\othertitle{Artikelvorlage zur Einreichung beim Texto Livre Journal}

\author[1]{Patricio Dugnani~\orcid{0000-0001-7877-4514}\thanks{Email: \href{mailto:patricio.dugnani@gmail.com}{patricio.dugnani@gmail.com}}}
\affil[1]{Universidade Presbiteriana Mackenzie, São Paulo, SP, Brasil.}


\addbibresource{article.bib}
% use biber instead of bibtex
% $ biber article

% used to create dummy text for the template file
\definecolor{dark-gray}{gray}{0.35} % color used to display dummy texts
\usepackage{lipsum}
\SetLipsumParListSurrounders{\colorlet{oldcolor}{.}\color{dark-gray}}{\color{oldcolor}}

% used here only to provide the XeLaTeX and BibTeX logos
\usepackage{hologo}

% if you use multirows in a table, include the multirow package
\usepackage{multirow}

% provides sidewaysfigure environment
\usepackage{rotating}

% CUSTOM EPIGRAPH - BEGIN 
%%% https://tex.stackexchange.com/questions/193178/specific-epigraph-style
\usepackage{epigraph}
\renewcommand\textflush{flushright}
\makeatletter
\newlength\epitextskip
\pretocmd{\@epitext}{\em}{}{}
\apptocmd{\@epitext}{\em}{}{}
\patchcmd{\epigraph}{\@epitext{#1}\\}{\@epitext{#1}\\[\epitextskip]}{}{}
\makeatother
\setlength\epigraphrule{0pt}
\setlength\epitextskip{0.5ex}
\setlength\epigraphwidth{.7\textwidth}
% CUSTOM EPIGRAPH - END

% to use IPA symbols in unicode add
%\usepackage{fontspec}
%\newfontfamily\ipafont{CMU Serif}
%\newcommand{\ipa}[1]{{\ipafont #1}}
% and in the text you may use the \ipa{...} command passing the symbols in unicode

% LANGUAGE - BEGIN
% ARABIC
% for languages that use special fonts, you must provide the typeface that will be used
% \setotherlanguage{arabic}
% \newfontfamily\arabicfont[Script=Arabic]{Amiri}
% \newfontfamily\arabicfontsf[Script=Arabic]{Amiri}
% \newfontfamily\arabicfonttt[Script=Arabic]{Amiri}
%
% in the article, to add arabic text use: \textlang{arabic}{ ... }
%
% RUSSIAN
% for russian text we also need to define fonts with support for Cyrillic script
% \usepackage{fontspec}
% \setotherlanguage{russian}
% \newfontfamily\cyrillicfont{Times New Roman}
% \newfontfamily\cyrillicfontsf{Times New Roman}[Script=Cyrillic]
% \newfontfamily\cyrillicfonttt{Times New Roman}[Script=Cyrillic]
%
% in the text use \begin{russian} ... \end{russian}
% LANGUAGE - END

% EMOJIS - BEGIN
% to use emoticons in your manuscript
% https://stackoverflow.com/questions/190145/how-to-insert-emoticons-in-latex/57076064
% using font Symbola, which has full support
% the font may be downloaded at:
% https://dn-works.com/ufas/
% add to preamble:
% \newfontfamily\Symbola{Symbola}
% in the text use:
% {\Symbola }
% EMOJIS - END

% LABEL REFERENCE TO DESCRIPTIVE LIST - BEGIN
% reference itens in a descriptive list using their labels instead of numbers
% insert the code below in the preambule:
%\makeatletter
%\let\orgdescriptionlabel\descriptionlabel
%\renewcommand*{\descriptionlabel}[1]{%
%  \let\orglabel\label
%  \let\label\@gobble
%  \phantomsection
%  \edef\@currentlabel{#1\unskip}%
%  \let\label\orglabel
%  \orgdescriptionlabel{#1}%
%}
%\makeatother
%
% in your document, use as illustraded here:
%\begin{description}
%  \item[first\label{itm1}] this is only an example;
%  % ...  add more items
%\end{description}
% LABEL REFERENCE TO DESCRIPTIVE LIST - END


% add line numbers for submission
%\usepackage{lineno}
%\linenumbers

\begin{document}
\maketitle

\begin{polyabstract}
\begin{abstract}
O objetivo principal deste artigo é compreender como se constitui a relação entre o uso dos meios de comunicação, especialmente os digitais, e os processos interculturais na Modernidade Tardia. Partindo de uma revisão bibliográfica, de cunho exploratório, esta reflexão relaciona a teoria dos meios de Marshall McLuhan com a comunicação intercultural para argumentar que a extensão da percepção humana, causada pela evolução técnica, produz uma aceleração nas trocas entre diferentes culturas. Discute-se como esse aumento no fluxo de informações pode superar um modelo multicultural de homogeneização, criticado por seu desequilíbrio, e promover uma interculturalidade baseada na hibridização e na troca mais justa. Os resultados da pesquisa indicam que os meios de comunicação digitais possuem potencial para promover uma interculturalidade mais equilibrada ao democratizar a emissão e recepção de informações, diferentemente dos meios de massa que favorecem a homogeneização. Contudo, esse potencial enfrenta desafios significativos, como o uso comercial e narcisista das plataformas digitais, que transformam relações culturais em relações de consumo. Portanto, se os meios digitais conseguirão de fato promover uma cultura mundial heterogênea é uma questão que permanece em aberto e depende do tempo e de mais investigações.

\keywords{Interculturalidade\sep Meios de comunicação\sep Modernidade tardia}
\end{abstract}

\begin{english}
\begin{abstract}
The main objective of this article is to understand the relationship between the use of communication media, especially digital media, and intercultural processes in Late Modernity. Based on an exploratory literature review, this reflection relates Marshall McLuhan's theory of media to intercultural communication to argue that the expansion of human perception, caused by technological evolution, accelerates exchanges between different cultures. It discusses how this increased flow of information can overcome a multicultural model of homogenization, criticized for its imbalance, and promote an interculturality based on hybridization and fairer exchange. The research results indicate that digital media have the potential to promote a more balanced interculturality by democratizing the transmission and reception of information, unlike mass media, which favor homogenization. However, this potential faces significant challenges, such as the commercial and narcissistic use of digital platforms, which transform cultural relations into consumer relations. Therefore, whether digital media will actually be able to promote a heterogeneous global culture is an open question and depends on time and further research.

\keywords{Interculturality\sep Media\sep Late modernity}
\end{abstract}
\end{english}
% if there is another abstract, insert it here using the same scheme
\end{polyabstract}

\section{Do desequilíbrio multicultural à promessa intercultural}\label{sec-intro}
Esse artigo parte do pressuposto de que a organização social e cultural humana se direcionou para uma intensificação dos processos instaurados historicamente pela Modernidade, e que não se desenvolveu uma nova organização, como indicaria o termo mais usado para denominar o momento contemporâneo: Pós-modernidade.

Dessa forma, entende-se que se vive uma Modernidade Tardia e esse momento histórico se constitui a partir de processos de aceleração, como indica Hartmut Rosa em seu livro \textit{Aceleração: A transformação das estruturas temporais na Modernidade} \citeyear{rosa2019}.

Além disso, observa-se que essa aceleração \cite{rosa2019, rosa2022} se manifesta nos mais diferentes campos da sociedade humana -- cultural, estético, entre outros -- além de se apresentar nas formas de produção e desenvolvimento tecnológico, o que, segundo David Harvey, em \textit{A Condição Pós-Moderna} \citeyear{harvey1996}, produz uma compressão na relação entre espaço e tempo.

Essa compressão do espaço/tempo, tem sido fortemente influenciada, e até mesmo, sustentada pela evolução técnica dos meios de comunicação, pois o uso deles produz mudanças no comportamento e na consciência humana, como indica Marshall McLuhan em seu livro \textit{Os Meios de Comunicação como extensões do Homem} \citeyear{mcluhan2016}. Em acordo com as ideias de \textcite{mcluhan2016}, mas partindo de uma visão mais antropológica e menos funcionalista, \textcite{ferrari2015}, em seus estudos sobre comunicação intercultural, também afirma que se pode concluir que, com cada troca de informações entre culturas, ocorrerão transformações na organização social delas.

Observando essas mudanças, também é possível perceber que o aumento do fluxo de informações, bem como a ampliação em dimensão global das trocas desses conteúdos produzidos pelo ser humano, apoiados pela evolução dos meios de comunicação, principalmente os elétricos e os digitais, tem produzido outro efeito: a troca cada vez mais intensa de ideias entre culturas diferentes, e de diferentes partes do globo terrestre. Esse fenômeno é também conhecido por globalização, como apresenta Stuart Hall no livro \textit{Identidade Cultural na Pós-modernidade} \citeyear{hall2004}.

Contando com esses processos de trocas de informações entre culturas dispersas pelo mundo, e entendendo cultura, de acordo com \textcite{geertz2008}, como um sistema de atribuição de significados, esse aumento de contato e de relações simbólicas, inevitavelmente, acabará por produzir mudanças na organização cultural das diferentes comunidades humanas. Esse processo, que já denominamos como globalização, parece não desenvolver uma troca equilibrada de influências sendo criticado por \textcite{santos2001} como uma globalização injusta, que apenas divide os prejuízos, e não o lucro; as doenças e não a tecnologia, a pobreza e não a igualdade.

Nesse sentido, por causa do desequilíbrio de influências entre as diferentes culturas do mundo, esse processo mundial acaba por se caracterizar por uma visão multicultural, e não intercultural, de acordo com a classificação que Lisete Weissmann, em seu artigo \textit{Multiculturalidade, Transculturalidade, Interculturalidade} \citeyear{weissmann2018}. Isso ocorre, porque a visão multicultural não prevê a mistura de culturas, mas culturas separadas que acabam assumindo uma posição etnocêntrica, e buscam impor seus valores às outras.

\begin{quote}
    A multiculturalidade implica um conjunto de culturas em contato, mas sem se misturar: trata-se de várias culturas no mesmo patamar. As diferenças ficam estanques e separadas em cada cultura, possibilitando pensar no que os antropólogos chamam a lógica do Um, que só tem uma verdade a seguir e uma forma de pensar o mundo \cite[p. 23-24]{weissmann2018}.
\end{quote}

Contrapondo-se a essa visão, torna-se importante entender que é necessário buscar através da comunicação um sistema que possibilite uma troca equilibrada de informações no processo de mistura de culturas. Esse processo pode ser denominado como interculturalidade, de acordo com \textcite{weissmann2018}.

\begin{quote}
   María Laura Méndez (2013, comunicação oral) ressalta que, para pensar a interculturalidade, temos que sair da lógica do Um e nos situar na lógica multívoca, a qual pressupõe multiplicidade e devir, e dentro da qual não podem ser feitas totalizações. Essa multiplicidade acarreta sempre diferença e se conforma dentro da heterogeneidade e suas combinações imprevisíveis. Não pode se fazer uma teoria da interculturalidade, porque isso implicaria uma generalização e universalização, o que é impossível. Define a interculturalidade como ``[...] uma série de gestos, práticas, que supõem sempre uma situação'' \cite[p. 26-27]{weissmann2018}.
\end{quote}

Por causa dessas ideias iniciais, pretende-se, neste artigo, compreender como se constitui a relação entre o uso dos meios de comunicação, especialmente os digitais, e os processos interculturais na Modernidade Tardia. Essa reflexão buscará relacionar os conceitos de \textcite{mcluhan2016} pela teoria dos meios, com a constituição de uma comunicação intercultural apontada por \textcite{ferrari2015}.

Entende-se, a partir dessa relação, que as transformações produzidas pelo efeito de extensão da percepção humana, segundo \textcite{mcluhan2016}, causadas pela evolução técnica dos meios de comunicação, principalmente os meios digitais e elétricos, tem produzido mudanças profundas no ambiente organizacional humano. Essa visão entende que os meios de comunicação digitais são os suportes que possibilitam o aumento entre as trocas de informação em âmbito global, desenvolvendo o que se denomina na atualidade como globalização, mas que poderia auxiliar um processo menos multicultural e mais intercultural.

Tomando essas questões como centrais, e partindo de um levantamento bibliográfico, essa pesquisa teórica e exploratória visa compreender como se constitui essa relação entre o uso dos meios de comunicação, principalmente os digitais, e os processos de misturas culturais na contemporaneidade. Esse artigo dará continuidade a uma pesquisa maior, que vem sendo desenvolvida, e pretende entender a organização social e cultural da Modernidade Tardia a partir do viés dos usos dos meios de comunicação.

Finalmente, o artigo está dividido em quatro partes: a \hyperref[sec-primeira-parte]{primeira parte} dedica-se à definição e ao debate do conceito de interculturalidade. A \hyperref[sec-segunda-parte]{segunda parte}, relaciona as questões sobre a interculturalidade e a comunicação intercultural, com as ideias da Teoria dos Meios de \textcite{mcluhan2016} sobre a extensão dos sentidos produzidas pelo uso dos meios de comunicação. Já a \hyperref[sec-terceira-parte]{terceira parte}, busca aprofundar a questão do uso dos meios de comunicação, observando os efeitos no processo intercultural e comparando quando mediados pelos meios de comunicação de massa e quando mediados pelos meios digitais. A \hyperref[sec-quarta-parte]{quarta parte} se constitui como as considerações finais, e pretende-se fazer a síntese das questões sobre interculturalidade e o uso dos meios de comunicação.\footnote{Este estudo está relacionado ao Grupo de Pesquisa Linguagens e Narrativas Interculturais (LeNI), inscrito no Conselho Nacional de Desenvolvimento Científico e Tecnológico (CNPq) e ligado Mestrado Profissional em Comunicação Intercultural nas Organizações (MPCOM) do Centro de Comunicação e Letras (CCL) da Universidade Presbiteriana Mackenzie (UPM).}

\section{Para além do multiculturalismo: definindo o conceito de interculturalidade}\label{sec-primeira-parte}
Antes de refletirmos sobre a questão da interculturalidade, principalmente em relação ao momento contemporâneo, torna-se fundamental definir um conceito básico para essa área de pesquisa: o conceito de cultura.

Sendo assim, levando-se em consideração que o conceito de cultura ganha historicamente diversas formas, desde aquelas mais tradicionais, até versões mais modernas, nesse artigo pretende aproximar-se da análise de \textcite{geertz2008} sobre o conceito de Cultura. Ele entende a cultura como um fenômeno plural, logo o termo cultura no singular, deveria dar lugar ao termo culturas. Para o antropólogo, cultura é um sistema complexo, uma vez que o indivíduo atribui diferentes significados aos fenômenos culturais conforme o contexto e momento observados. Esse conceito de cultura se aproxima do proposto por \textcite{ferrari2015}, autora ligada aos estudos interculturais, que afirma ser a cultura um fenômeno complexo, mutável e com múltiplas interpretações, produzindo respostas distintas a cada fato, o que se assemelha à ideia de atribuição de significados proposta por \textcite{geertz2008}.

\begin{quote}
    Hoje, tratamos a cultura como um processo em mutação, complexo e criativo, que pode ser abordada de múltiplas maneiras; e, como decorrência de sua peculiaridade, não há consenso entre os estudiosos sobre a sua definição. Justificamos essa nova abordagem pela exposição dos indivíduos aos processos de globalização que os coloca em embates diante das diferenças culturais, de estilo de vida e de pensamento. Os indivíduos e os diferentes grupos diante do cenário a que são expostos produzem respostas distintas ao próprio fato da diferença que, por causa da globalização, parece cada vez mais óbvia. Portanto, hoje as sociedades vão aprender a lidar com as diferenças, mais do que em qualquer outro momento histórico \cite[p. 6]{ferrari2015}.
\end{quote}

Definido um conceito básico de cultura, no cruzamento e na concordância entre a visão de atribuição de significados de \textcite{geertz2008}, com a visão de cultura como um fenômeno complexo, mutável e múltiplo de \textcite{ferrari2015}, agora se torna possível definir o conceito de interculturalidade partindo de uma fundamentação mais sólida.

Entende-se, aqui, de acordo com \textcite{ferrari2015} e concordando com \textcite{weissmann2018}, como interculturalidade, a relação equilibrada de troca de informações e mistura entre culturas. Tomando-se essa ideia, é importante destacar que o fenômeno de mistura de culturas, não é um fenômeno recente, mas acompanha a organização humana desde seu princípio mais remoto.

\begin{quote}
    A interculturalidade significa a relação entre pessoas de distintas culturas e, na verdade, ela se produz desde os inícios da humanidade, à medida que pessoas de culturas diferentes se relacionaram ao longo da história \cite[p. 7]{ferrari2015}.
\end{quote}

O que diferencia a mistura de culturas no momento contemporâneo, em relação ao passado, é que ela é, agora, acelerada pelos meios digitais. Diferente dos processos anteriores, tomando os séculos XX e XXI como referência, que eram regidos, principalmente, por outros meios (meios elétricos, impressões, de comunicação de massa etc.). Essa troca de hegemonia do uso dos meios digitais é que tem produzido uma aceleração da produção e o aumento do alcance das trocas de informação entre as diferentes culturas do mundo \cite{rosa2019, rosa2022}. Esse processo, apoiado, também nos meios de comunicação, é o que \textcite{rosa2019, rosa2022} identifica como sendo um dos fatores responsáveis pelas alterações de comportamento e consciência, além da mudança dos sistemas de mistura de culturas.

A interculturalidade, refletindo sobre esses conceitos, se constitui pela interação, através dos meios de comunicação entre grupos culturais diferentes, e não pela sobreposição de culturas.

\begin{quote}
    Em qualquer situação de interação intercultural, dois ou mais grupos levam consigo repertórios de conhecimento disponíveis e é no contato entre eles que se produz o espaço no qual negociam as interpretações do mundo. Portanto a chave da comunicação intercultural é a interação com o diferente, com tudo aquilo que, de forma objetiva ou subjetiva, se percebe como diferente, seja qual for o motivo da diferença: raça, gênero, classe social, preferência sexual etc \cite[p. 12]{ferrari2015}.
\end{quote}

Com a visão intercultural, se espera que esses grupos troquem informações e tomem como base suas experiências, seus repertórios, todos os conhecimentos desenvolvidos e usados para formular sua concepção de mundo. Ao trocarem informações, interagem partindo de ideias diferentes o que faz com que esses conteúdos culturais se misturem, produzindo alterações de consciência e comportamento nesses grupos, e consequentemente, em suas culturas.

Esse é um modelo de funcionamento da interculturalidade, mediado pelos meios de comunicação, que se espera desenvolver.

\section{O meio é a mensagem: os suportes da comunicação como extensões do homem}\label{sec-segunda-parte}
Buscando o entendimento do conceito e da estrutura que constitui um processo de interculturalidade, se torna fundamental, compreender como esse processo se relaciona com o uso dos meios de comunicação. Além de perceber essa questão, é preciso observar que a interculturalidade só será possível quando mediada pelos meios de comunicação que se tornam suportes da comunicação intercultural. Sendo assim, a observação do funcionamento dos meios de comunicação como agentes transformadores da organização política, social e cultural humana, é fundamental para poder se instaurar relações interculturais.

\begin{quote}
A comunicação intercultural parte das dimensões interativa e relacional do processo de comunicação. É interativa porque concebe o processo comunicativo como mecanismo que permite as ações relacionais, mas é também relacional porque o peso dessas relações condiciona constantemente a direção e o sentido da interação \cite[p. 12]{ferrari2015}.
\end{quote}

Para \textcite{mcluhan2016} os meios de comunicação são mais que simples aparelhos transmissores, pois interferem e influenciam as mudanças de comportamento e de ampliação de consciência humana. 

Sob essa perspectiva, os meios de comunicação são extensões da percepção, dos sentidos, e, para o autor, extensões do próprio sistema nervoso humano. Ou seja, pode-se entender que, quando os meios de comunicação estendem a percepção humana, também aumentam a quantidade de informações e fenômenos que esse ser humano tem contato. Os meios aumentam o alcance dos sentidos, outrora limitados pelos limites biológicos dos mesmos. O rádio estende o alcance da fala e amplia o potencial da audição. A televisão estende a visão e a audição, e assim por diante.

Com esse aumento do alcance da recepção, emissão e transmissão de informações, ocorre também uma ampliação na mistura de conteúdos culturais, e consequentemente, no cruzamento de informações entre diferentes culturas, outrora separadas, muitas vezes, pela dimensão espacial. Esse aumento de contato, essa maior troca de informações foi criando, dinamizando e reformulando as relações entre as diferentes culturas. Ou seja, é através, também, dos meios de comunicação que será possível o desenvolvimento do processo de interculturalidade, entendido por \textcite{ferrari2015}, como sendo a relação de troca equilibrada entre os seres humanos de diferentes culturas. Por isso, os processos de comunicação, incluindo o uso dos meios, estão intimamente ligadas ao desenvolvimento da interculturalidade.

\begin{quote}
    As sociedades e as organizações contemporâneas passam por um dilema intercultural à medida que estão expostas a uma pluralidade de visões sobre diferentes contextos, principalmente decorrentes dos processos de internacionalização que foram facilitados pela tecnologia, pela abertura das economias e pelos processos migratórios. Portanto, o estudo da interculturalidade pode ser comparado a um cenário ou um pano de fundo, que flui e influi no relacionamento das sociedades e organizações dentro e fora de suas fronteiras geográficas. Essa metáfora do pano de fundo, mostra que é necessária a adoção de uma perspectiva sistêmica, em que a cultura e a comunicação são dimensões sinérgicas que não funcionam em separado \cite[p. 1]{ferrari2015}.
\end{quote}

Além da questão da extensão, outro conceito de \textcite{mcluhan2016} que é importante destacar, quando se pretende entender a relação do uso dos meios de comunicação e o processo de interculturalidade, é a ideia que os meios produzem alteração de comportamento e consciência nos seres humanos \cite{dugnani2018, dugnani2022}. Essa afirmação surge de uma das mais polêmicas frases criadas por \textcite{mcluhan2016}: ``o meio é a mensagem''.

O meio é uma mensagem, ou o meio é informação pura. Essas frases reforçam a ideia do autor que os meios não são apenas suportes de informações, mas são agentes transformadores da sociedade.

A princípio essas duas frases estariam equivocadas, partindo de uma visão mais funcionalista dos meios de comunicação, pois na classificação inicial meio é meio, mensagem é mensagem. Ou seja, o meio de comunicação é o suporte material da mensagem que possibilita sua transmissão, de um emissor a um receptor, através de um determinado ambiente. E mensagem seria o conteúdo, a informação criada e organizada por um emissor, a partir de um código e transmitida por um meio para um receptor.

Mas por que \textcite{mcluhan2016} afirma ser o meio uma mensagem?

Essa correlação é feita, na verdade, a partir do conceito de informação.

Entendendo-se informação, concordando com \textcite{teixeira_coelho2012semiotica}, como sendo um conteúdo que promove alguma mudança de comportamento e consciência, e compreendendo que a mensagem é formada de informação: a mensagem (informação) altera comportamento e consciência dos seres humanos.

Como para \textcite{mcluhan2016} os meios de comunicação, ou seja, o advento, ou o simples uso dos meios de comunicação promovem transformações no comportamento e consciência humana, para o autor canadense: ``meio e mensagem cumprem a mesma função''. Por isso ele afirma que meio é mensagem; meio é informação pura, pois, não somente pela mensagem que transporta, mas por seu uso, os meios produzem mudanças na organização e compreensão de mundo dos seres humanos.

Tomando esses princípios para reflexão, pretende-se, no próximo capítulo, ampliar o entendimento de como a comunicação (logo, também, o uso dos meios) está sinergicamente interligada com os processos de interculturalidade. Inclusive em relação ao aumento de contato entre culturas que outrora tinham menos acesso, devido ao espaço e o alcance das mensagens e, na atualidade, graças aos meios digitais, ampliaram sua troca de informações.

\section{A encruzilhada digital: entre a homogeneização da indústria cultural e a hibridização intercultural}\label{sec-terceira-parte}
Embora já tenha sido esboçada a relação entre meios de comunicação e interculturalidade nos capítulos anteriores, pretende-se, na continuidade desse artigo, buscar uma síntese que possa expor melhor como essas questões se apresentam na contemporaneidade.

Em primeiro lugar é preciso entender que essa aceleração nos processos de mistura de culturas, ou seja, o aumento das trocas de informações entre culturas tem uma dupla influência que se destaca: a própria aceleração social e o desenvolvimento dos meios digitais \cite{rosa2019, rosa2022}. Afinal, concordando com \textcite{ferrari2015}, para desenvolver os estudos sobre a interculturalidade é preciso relacionar os processos comunicacionais à organização cultural. De certa forma, buscar os pontos que aproximam antropologia da ciência da comunicação.

\begin{quote}
    Um dos aspectos mais importantes para o estudo da interculturalidade é a identificação dos processos comunicacionais que, ao lado da cultura, estabelecem as bases para o diálogo cultural entre as pessoas e \textit{nas} e \textit{entre} organizações com seus públicos e as demais instituições \cite[p. 1]{ferrari2015}.
\end{quote}

Nesse sentido, é importante entender que, conforme se alteram os processos de comunicação (e nesse artigo, interessa-se, principalmente, o uso dos meios de comunicação), alteram-se, também as relações entre culturas. Esse fenômeno de alteração cria uma necessidade propor novos estatutos nas relações de alteridade entre o eu e o outro, como afirma \textcite[p. 348]{ramos2013}, para se atingir a interculturalidade.

\begin{quote}
    No mundo aberto e plural atual, com a globalização e os novos meios e tecnologias de informação e comunicação, com os média, a internet, as facilidades de deslocação e os meios de transporte rápidos, a diversidade cultural, o Outro, as minorias étnicas têm um outro estatuto e imagem. A diversidade cultural e o Outro não estão longínquos, mas estão mais próximos e presentes no quotidiano, coabitam conosco nos espaços públicos, nas instituições, e reclamam respeito e direitos \cite[p. 348]{ramos2013}.
\end{quote}

Se outrora, quando a sociedade dispunha apenas de meios de comunicação mais limitados em seu alcance, apenas poderiam se organizar em tribos, ou comunidades menores e em espaços mais limitados, hoje se fala de culturas globais: comunidades globais sustentadas pelos meios de comunicação digitais.

Percebe-se essa questão da relação entre espaço e política em \textcite{mcluhan2016}, quando ele discute as interações entre o uso dos meios de comunicação e a organização política. Na visão do autor os meios de comunicação, seu alcance e dinâmica de uso, influenciam a constituição sócio-política humana, além da organização espacial \cite{mcluhan2016}. Essa relação é percebida melhor, quando se observa cronologicamente os momentos em que ocorreram grandes revoluções tecnológicas nos meios de comunicação, e as mudanças na organização social e política.

Em torno do desenvolvimento de uma escrita fonética e mais eficiente e com a possibilidade dos registros se tornarem materiais, a representação humana pode se estender para além da presença física, pois o ser humano agora poderia presentificar seu discurso através de textos. Quando conseguiu estender sua presença e, mesmo, autoridade através do texto, para além do limite biológico do corpo, passou a deixar a organização política das aldeias e tribos, para constituir as grandes civilizações. Podem pensar como exemplo os povos mesopotâmicos e sua escrita cuneiforme, os egípcios e os hieróglifos, e posteriormente, de maneira mais eficiente, a civilização clássica, quando surgem os alfabetos fonéticos similares aos atuais, com em torno de 20 caracteres, os quais foram desenvolvidos pelos fenícios e utilizados pelos gregos posteriormente. Sendo assim, tomando esse exemplo, a organização política humana parece se transformar fortemente, quando ocorre uma revolução tecnológica nos meios de comunicação \cite{dugnani2018}.

De certa forma a fala está para tribo, a escrita para as grandes civilizações, o meio impresso para as nações, a internacionalização de culturas \cite{ortiz2000} está para os meios elétricos/meios de comunicação de massa e a globalização está para os meios digitais \cite{dugnani2018}.

Em cada um desses processos os meios de comunicação estendem a percepção humana \cite{mcluhan2016}, mas ao mesmo tempo, estendem também os processos de homogeneização de culturas, resultado inevitável das trocas de informações entre comunidades. Nenhuma cultura, de nenhuma comunidade, é hermética, basta ocorrer trocas de informações através da comunicação, que as representações, valores de diferentes culturas acabam por contaminar o ambiente simbólico de uma sociedade. Por isso, a cada revolução tecnológica dos meios de comunicação, que aumentam o alcance e a quantidade de pessoas que acessam as informações, maior será, e mais abrangente serão, os processos de homogeneização. E que não se tome essa afirmação como uma crítica, mas sim, como uma afirmação da inevitabilidade desse processo. A homogeneização de culturas é um fato que ocorre desde que o ser humano começou a viver em sociedade e se comunicar através dos meios, ou seja, desde sempre.

Embora não seja uma crítica, mas uma constatação, é preciso, sim, que esse processo de homogeneização seja feito de maneira justa e equilibrada, fato que parece, também, não ter ocorrido com essas características na história da humanidade. Colonização, etnocentrismo e mesmo a atual globalização podem, e devem ser criticadas como processos de homogeneização civilizatórios, os quais não equilibram as trocas de informações, mas impõem modelos culturais a uma comunidade. Segundo \textcite{santos2001}, conforme dito anteriormente, ocorre uma globalização das doenças, da exploração, do desequilíbrio social não uma globalização das riquezas. Dessa forma é preciso tomar cuidado com os processos de homogeneização engatilhados pelos meios de comunicação, pois senão, apenas servirão à interesses econômicos e não à sociedade de modo geral. Esse fato pode dificultar que se substitua uma visão multicultural, por uma visão intercultural \cite{weissmann2018}. Por isso, \textcite{ferrari2015} critica o processo de homogeneização imposto pela globalização e observa a necessidade de haver junto com ele, uma hibridização de culturas.

\begin{quote}
    A visão tradicional de globalização estava relacionada aos processos de homogeneização. Hoje, a visão mais crítica e provocadora trata de conceituar a globalização como um processo impulsor da heterogeneidade. A referida noção de heterogeneidade está vinculada aos processos de hibridização \cite{garciacanclini1999}. Dessa forma, a globalização e a hibridização passam a ser duas dimensões inseparáveis que vão permitir as mesclas culturais \cite[p. 2]{ferrari2015}.
\end{quote}

Embora, também, seja inevitável uma homogeneização, \textcite{garciacanclini1999} identifica nesse processo, de maneira menos pessimista, uma heterogeneização da cultura por processos de hibridização. De qualquer forma, se os meios digitais conseguirão romper esse círculo vicioso de homogeneização da cultura pelos meios, fazendo surgir, de acordo com \textcite{ferrari2015}, um impulso de heterogeneidade na cultura contemporânea, é algo que é preciso observar com o tempo, verificar e investigar. Essa é uma parte do conhecimento que os estudos interculturais estão preparados para analisar.

Esse impulso de heterogeneidade na cultura, não é impossível, pois, se for observado o funcionamento de cada meio de comunicação, também é verdadeiro que seu uso imprime efeitos diferentes na organização da sociedade. Certamente, os meios de comunicação de massa não seriam capazes de impulsionar fortemente a hibridização, mas sim, fortalecer uma homogeneização cultural. Afirma-se isso, porque os meios de comunicação de massa tendem a um fluxo de transmissão de informação mais unilateral. Esse fluxo é mais unilateral, pois a maior quantidade de informação parte dos poucos emissores que têm acesso a decisão do que deve ser transmitido (devido ao seu potencial político ou econômico), para muitos receptores (a massa), que recebem as informações de maneira mais passiva, como descreve \textcite{jenkins2015}. Nos meios digitais ocorre um desequilíbrio no fluxo de informações, pois a recepção é mais democrática do que a emissão \cite{dugnani2020}. A recepção é mais democrática, pois com um baixo investimento, toda informação transmitida por um meio de comunicação de massa fica disponível para quem quiser acessar. Já a emissão é menos democrática, porque o acesso e a decisão das informações que podem, ou devem, ser transmitidas por esses meios fica limitado a grupos que têm poder político ou econômico. Os meios de comunicação de massa, também, precisam de audiência para sobreviver, sendo assim acabam por fabricar informações que agradam o gosto médio da população, o que reforça o processo de homogeneização internacional de culturas, como já indicava os estudos da Escola de Frankfurt, principalmente pela visão de \textcite{adorno2000} e seu conceito clássico de indústria cultural. A informação produzida num modelo industrial, tratada como produto que deve dar lucro, só poderá servir para um processo de homogeneização global de culturas, formando assim uma cultura internacional de massa e reforçando a multiculturalidade, em detrimento da interculturalidade.

Com o advento dos meios digitais de comunicação, onde esse desequilíbrio do fluxo de comunicação entre emissores e receptores se torna mais equilibrado, talvez a hibridização das culturas, como apresenta \textcite{garciacanclini1999}, e o impulso de heterogeneidade como apresenta \textcite{ferrari2015} possam se constituir de uma maneira mais justa e equilibrada. Afinal, o funcionamento dos meios digitais tem potencial para isso.

Afirma-se que os meios digitais têm potencial para isso, pois entende-se que, com um baixo investimento, tanto a emissão, como a recepção possibilitam maior acesso aos indivíduos, ao invés de favorecer apenas as grandes emissoras, como era no meio de comunicação de massa. O indivíduo, com os meios digitais, pode estender sua percepção e transmitir mensagens com alcance global, em um tempo reduzido e com baixo investimento \cite{dugnani2024}.

Num primeiro momento os meios digitais se apresentaram como uma possibilidade de vencer processos industriais de homogeneização da cultura, como ocorria fortemente no século XX, contudo precisarão vencer diversos desafios que se apresentam, que vão desde a ocupação maciça dos espaços virtuais da internet pelas estratégias mercadológicas das grandes instituições; até o individualismo, narcisismo, hedonismo e vaidade dos usuários desses meios de comunicação. Usuários que sentem uma forte compulsão de ocupar o espaço da comunicação com sua imagem apresentada não para reflexão, mas para culto. Nesse sentido, concordando com \textcite{bauman2008} numa sociedade de consumo, o ser humano acaba por querer se tornar uma mercadoria apreciável, por isso nos meios digitais, é comum que o humano se apresente e se ofereça para ser consumido. Nos meios digitais, ou em redes sociais como, por exemplo, o Instagram, a imagem do emissor, constantemente, aparece em situações quando sua experiência parece indiscutivelmente um prazer quase sagrado. Nos meios digitais, e nos ambientes digitais em geral, os emissores se apresentam como uma mercadoria sagrada, a ser profanada e consumida pelos receptores/consumidores, como observa \textcite{agamben2007}.

Essa relação, pode dificultar que os meios digitais rompam com os processos de homogeneização que afetam a interculturalidade, pois o emissor se posta como produtor de informação, e o receptor como consumidor, como afirmou \textcite{toffler2014}, ou seja, as relações sociais e culturais, para esse artigo, são substituídas por relações comerciais nos meios digitais, fazendo com que a mistura de culturas perca seu sentido social, e se promova mais como uma troca comercial.

Sendo assim, os meios digitais apresentam potencial para escapar da homogeneização das culturas, e promover a interculturalidade, mas se conseguirão cumprir essa missão, somente o tempo dirá.

\section{Fronteiras da interculturalidade: síntese, desafios e futuros cruzamentos teóricos}\label{sec-quarta-parte}
A evolução tecnológica dos meios de comunicação, está produzindo uma ampliação global no alcance das mensagens, uma aceleração nas transformações sociais, e um aumento no contato entre culturas, sendo assim: as teorias da comunicação e dos meios, os estudos interculturais e sobre a aceleração social são ideias que trazem reflexões importantes para a compreensão do contexto da Modernidade Tardia.

Finalizando as reflexões feitas sobre essas ideias apresentadas, destaca-se a importância que a interculturalidade assume na Modernidade Tardia, para a busca do desenvolvimento de estratégias para uma comunicação entre culturas, pois esses encontros se tornam cada vez mais frequentes e intensos, com o desenvolvimento dos meios de comunicação digitais e a consequente aceleração dos modos de vida da sociedade. Estratégias que podem envolver e tornar mais equilibrado o contato entre culturas globais, torna-se um tema central do momento contemporâneo. Além de reafirmar a hipótese que a reunião dessas frentes de pesquisa, embora possam, por vezes, parecerem distintas, na verdade, possibilitam o desenvolvimento de estratégias mais amplas para buscar soluções para um problema cada vez mais presente: o aumento da xenofobia, dos discursos preconceituosos e a violência entre culturas que surgem com os movimentos imigratórios intensos e a aceleração do uso dos meios de comunicação.

A aproximação dos estudos interculturais com as ciências da comunicação, são um rico campo de análise para entender a organização sociocultural das comunidades distribuídas pelo globo terrestre. Comunidade essas que estão cada vez mais expostas a processos de misturas culturais, que são potencializados pelo uso e evolução tecnológica dos meios de comunicação.

Por isso, esse artigo também tem a pretensão de dar início ao desenvolvimento de uma visão que possibilite o cruzamento entre diferentes áreas do conhecimento como as ideias de Modernidade Tardia e aceleração de \textcite{rosa2019}, a Hipermodernidade de \textcite{lipovetsky2004}, as teorias dos meios de \textcite{mcluhan2016}, os estudos interculturais de \textcite{weissmann2018}, \textcite{ferrari2015} e \textcite{ramos2013} para que seja possível a criação de uma metodologia que possa analisar de maneira eficiente os processos de trocas entre as culturas, que ocorrem entre os diferentes grupos humanos, de maneira cada vez mais intensa, principalmente, com o advento dos meios digitais e a aceleração dos processos sociais na contemporaneidade.

Esse método terá o desafio de reunir a visão sociológica e antropológica dos estudos culturais, com o entendimento mais funcionalista e filosófico das teorias dos meios e das ciências da comunicação da linha norte-americana. Esse novo método terá que observar que o desenvolvimento tecnológico dos meios de comunicação, bem como seu uso, produz efeitos na organização social, política e cultural humana.

Dessa forma, talvez seja possível, futuramente, através da comunicação intercultural, desenvolver relações mais pautadas no respeito às diferentes culturas que se apresentam no mundo. Talvez com os estudos interculturais e o desenvolvimento dos meios digitais de comunicação, além de suas teorias, seja possível criar uma consciência coletiva de que é preciso respeitar as diversidades culturais, para não mais desenvolver uma sociedade homogeneizada por visões colonialistas e etnocêntricas, mas sim, uma cultura mundial e heterogênea.

Com isso, será deixado aqui uma última reflexão, que se pretende desenvolver num próximo artigo. A ideia de que ao invés da globalização desequilibrada descrita por \textcite{santos2001}, quem sabe, com o apoio dos estudos interculturais e da teoria dos meios, não possamos caminhar para uma aldeia global, como descrita por \textcite{mcluhan2016}, ou pelo caminho da interculturalidade, como descreve \textcite{ferrari2015}. Uma sociedade que através de uma tribalização possa conviver num espaço global, como se fosse uma única comunidade, não por causa da força, mas pelo desejo de conviver de maneira mais próxima e fraterna, como numa tribo, onde todos se respeitam. Nesse sentido, espera-se que as utopias ainda não tenham saído de moda totalmente.




\printbibliography\label{sec-bib}
% if the text is not in Portuguese, it might be necessary to use the code below instead to print the correct ABNT abbreviations [s.n.], [s.l.]
%\begin{portuguese}
%\printbibliography[title={Bibliography}]
%\end{portuguese}


%full list: conceptualization,datacuration,formalanalysis,funding,investigation,methodology,projadm,resources,software,supervision,validation,visualization,writing,review


\begin{dataavailability}
\txtdataavailability{databody} % options: dataavailable, dataonly, databody, datanotav, nodata
\end{dataavailability}



\end{document}

