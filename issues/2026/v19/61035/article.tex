% !TEX TS-program = XeLaTeX
% use the following command:
% all document files must be coded in UTF-8
\documentclass[portuguese]{textolivre}
% build HTML with: make4ht -e build.lua -c textolivre.cfg -x -u article "fn-in,svg,pic-align"

\journalname{Texto Livre}
\thevolume{19}
%\thenumber{1} % old template
\theyear{2026}
\receiveddate{\DTMdisplaydate{2025}{8}{11}{-1}} % YYYY MM DD
\accepteddate{\DTMdisplaydate{2025}{12}{1}{-1}}
\publisheddate{\DTMdisplaydate{2026}{2}{2}{-1}}
\corrauthor{Everton Bedin}
\articledoi{10.1590/1983-3652.2026.61035}
%\articleid{NNNN} % if the article ID is not the last 5 numbers of its DOI, provide it using \articleid{} commmand 
% list of available sesscions in the journal: articles, dossier, reports, essays, reviews, interviews, editorial
\articlesessionname{essays}
\runningauthor{Bedin} 
%\editorname{Leonardo Araújo} % old template
\sectioneditorname{Daniervelin Pereira~\orcid{0000-0003-1861-3609}}
\layouteditorname{Leonardo Araujo~\orcid{0000-0003-3884-2177}}

\title{Competências digitais docentes e políticas públicas na era da informação: por uma formação crítica, intencional e emancipatória}
\othertitle{Teacher digital competencies and public policies in the information age: towards a critical, intentional, and emancipatory training}
% if there is a third language title, add here:
%\othertitle{Artikelvorlage zur Einreichung beim Texto Livre Journal}

\author[1]{Everton Bedin~\orcid{0000-0002-5636-0908}\thanks{Email: \href{mailto:bedin.everton@gmail.com}{bedin.everton@gmail.com}}}
\affil[1]{Universidade Federal do Paraná, Programa de Pós-graduação em Educação em Ciências e em Matemática, Curitiba, PR, Brasil.}

\addbibresource{article.bib}
% use biber instead of bibtex
% $ biber article

% used to create dummy text for the template file
\definecolor{dark-gray}{gray}{0.35} % color used to display dummy texts
\usepackage{lipsum}
\SetLipsumParListSurrounders{\colorlet{oldcolor}{.}\color{dark-gray}}{\color{oldcolor}}

% used here only to provide the XeLaTeX and BibTeX logos
\usepackage{hologo}

% if you use multirows in a table, include the multirow package
\usepackage{multirow}

% provides sidewaysfigure environment
\usepackage{rotating}

% CUSTOM EPIGRAPH - BEGIN 
%%% https://tex.stackexchange.com/questions/193178/specific-epigraph-style
\usepackage{epigraph}
\renewcommand\textflush{flushright}
\makeatletter
\newlength\epitextskip
\pretocmd{\@epitext}{\em}{}{}
\apptocmd{\@epitext}{\em}{}{}
\patchcmd{\epigraph}{\@epitext{#1}\\}{\@epitext{#1}\\[\epitextskip]}{}{}
\makeatother
\setlength\epigraphrule{0pt}
\setlength\epitextskip{0.5ex}
\setlength\epigraphwidth{.7\textwidth}
% CUSTOM EPIGRAPH - END

% to use IPA symbols in unicode add
%\usepackage{fontspec}
%\newfontfamily\ipafont{CMU Serif}
%\newcommand{\ipa}[1]{{\ipafont #1}}
% and in the text you may use the \ipa{...} command passing the symbols in unicode

% LANGUAGE - BEGIN
% ARABIC
% for languages that use special fonts, you must provide the typeface that will be used
% \setotherlanguage{arabic}
% \newfontfamily\arabicfont[Script=Arabic]{Amiri}
% \newfontfamily\arabicfontsf[Script=Arabic]{Amiri}
% \newfontfamily\arabicfonttt[Script=Arabic]{Amiri}
%
% in the article, to add arabic text use: \textlang{arabic}{ ... }
%
% RUSSIAN
% for russian text we also need to define fonts with support for Cyrillic script
% \usepackage{fontspec}
% \setotherlanguage{russian}
% \newfontfamily\cyrillicfont{Times New Roman}
% \newfontfamily\cyrillicfontsf{Times New Roman}[Script=Cyrillic]
% \newfontfamily\cyrillicfonttt{Times New Roman}[Script=Cyrillic]
%
% in the text use \begin{russian} ... \end{russian}
% LANGUAGE - END

% EMOJIS - BEGIN
% to use emoticons in your manuscript
% https://stackoverflow.com/questions/190145/how-to-insert-emoticons-in-latex/57076064
% using font Symbola, which has full support
% the font may be downloaded at:
% https://dn-works.com/ufas/
% add to preamble:
% \newfontfamily\Symbola{Symbola}
% in the text use:
% {\Symbola }
% EMOJIS - END

% LABEL REFERENCE TO DESCRIPTIVE LIST - BEGIN
% reference itens in a descriptive list using their labels instead of numbers
% insert the code below in the preambule:
%\makeatletter
%\let\orgdescriptionlabel\descriptionlabel
%\renewcommand*{\descriptionlabel}[1]{%
%  \let\orglabel\label
%  \let\label\@gobble
%  \phantomsection
%  \edef\@currentlabel{#1\unskip}%
%  \let\label\orglabel
%  \orgdescriptionlabel{#1}%
%}
%\makeatother
%
% in your document, use as illustraded here:
%\begin{description}
%  \item[first\label{itm1}] this is only an example;
%  % ...  add more items
%\end{description}
% LABEL REFERENCE TO DESCRIPTIVE LIST - END


% add line numbers for submission
%\usepackage{lineno}
%\linenumbers

\begin{document}
\maketitle

\begin{polyabstract}
\begin{abstract}
Este ensaio teórico objetiva analisar criticamente a interconexão entre o uso instrumental e a apropriação crítica das Tecnologias Digitais de Informação e Comunicação (TDIC), propondo o Modelo Tridimensional de Apropriação Crítica como referencial para sua incorporação consciente e emancipatória na prática pedagógica. Fundamentado em abordagem qualitativa e perspectiva crítico-reflexiva, o estudo articula referenciais da teoria crítica da tecnologia, pedagogia crítica e epistemologias do Sul, relacionando intencionalidade pedagógica, políticas públicas de formação e condições estruturais que influenciam a apropriação de tecnologias pelos professores. A análise indica que abordagens tecnicistas e fragmentadas ainda predominam nas políticas e práticas formativas, limitando o uso das tecnologias a dimensões instrumentais e descontextualizadas. Como proposição, apresenta-se um modelo estruturado em três eixos interdependentes – condições estruturais, competências críticas e reflexivas e intencionalidade pedagógica – que orientam o uso de tecnologias para além do domínio técnico, valorizando a mediação docente, o protagonismo estudantil e a formação integral.

\keywords{Mediação docente crítica \sep Tecnologias digitais \sep Formação crítica \sep Apropriação tecnológica \sep Intencionalidade pedagógica}
\end{abstract}

\begin{english}
\begin{abstract}
This theoretical essay aims to critically analyze the interconnection between the instrumental use and the critical appropriation of Digital Information and Communication Technologies (DICT), proposing the Three-Dimensional Model of Critical Appropriation as a framework for their conscious and emancipatory incorporation into pedagogical practice. Grounded in a qualitative approach and a critical-reflective perspective, the study articulates references from the critical theory of technology, critical pedagogy, and Epistemologies of the South, relating pedagogical intentionality, public teacher education policies, and structural conditions that influence teachers’ appropriation of technologies. The analysis indicates that technicist and fragmented approaches still predominate in policies and training practices, limiting the use of technologies to instrumental and decontextualized dimensions. As a proposition, the study presents a model structured around three interdependent axes – structural conditions, critical and reflective competencies, and pedagogical intentionality – which guides the use of technologies beyond technical mastery, valuing teacher mediation, student protagonism, and holistic education.

\keywords{Critical teacher mediation \sep Digital technologies \sep Critical training \sep Technological appropriation \sep Pedagogical intentionality}
\end{abstract}
\end{english}
% if there is another abstract, insert it here using the same scheme
\end{polyabstract}

\section{Introdução}\label{sec-intro}
A sociedade contemporânea, caracterizada pela ubiquidade das Tecnologias Digitais de Informação e Comunicação (TDIC), tem imposto transformações profundas em todas as esferas da vida humana, reconfigurando as relações sociais, econômicas, políticas e, inevitavelmente, educacionais. Neste cenário de metamorfoses constantes, a educação encontra-se diante de um paradoxo fundamental: por um lado, é convocada a incorporar as TDIC como ferramentas potencializadoras dos processos de ensino e aprendizagem; por outro, enfrenta o desafio de não sucumbir à racionalidade técnica e instrumental que frequentemente acompanha o discurso tecnológico hegemônico.

Como alertam \textcite{HorkheimerAdorno2006DialeticaEsclarecimento}, a razão instrumental, ao reduzir a tecnologia a mero aparato técnico desprovido de reflexão crítica, pode conduzir a uma educação alienante e reprodutora das desigualdades sociais. Diante deste cenário complexo, emerge a necessidade de repensar as competências docentes para além do domínio operacional das tecnologias, incorporando uma dimensão crítico-reflexiva, que permita ao professor compreender as implicações sociais, políticas e epistemológicas das TDIC no contexto educacional, bem como desenvolver uma intencionalidade pedagógica que oriente seu uso de forma consciente e emancipatória.

A problemática das competências digitais docentes transcende, portanto, a mera instrumentalização técnica, situando-se no campo das tensões entre tecnologia e humanismo, entre racionalidade técnica e racionalidade crítica, entre adaptação e transformação social. Conforme argumenta \textcite{Feenberg2002TransformingTechnology}, a tecnologia não é neutra, mas carrega em si valores e interesses que refletem as relações de poder existentes na sociedade, podendo tanto reforçar estruturas de dominação quanto abrir possibilidades para práticas emancipatórias.

Nesta perspectiva, o desenvolvimento de competências digitais docentes implica, para além de saber utilizar as tecnologias, compreender criticamente o seu papel na configuração das relações sociais e educacionais contemporâneas \cite{SiqueiraBedin2024TPACKLicenciandosFisica}. \textcite{Castells2019SociedadeRede} complementa esta visão ao afirmar que se vive em uma sociedade em rede, na qual o poder está associado ao controle dos fluxos informacionais, o que torna ainda mais urgente a formação de professores capazes de promover uma apropriação crítica e criativa das TDIC. Esta formação deve superar a dicotomia entre tecnofilia acrítica e tecnofobia paralisante, buscando uma abordagem dialética que reconheça tanto as potencialidades quanto os riscos das TDIC no contexto educacional, e que prepare os docentes para atuar como mediadores entre os estudantes e o universo tecnológico-digital.

Como já alertava \textcite[p. 67]{Freire2018PedagogiaAutonomia}, “a educação é um ato de amor, por isso, um ato de coragem. […] Não pode fugir à discussão criadora”. Essa visão reforça a urgência de uma formação docente que incorpore criticamente as TDIC como parte de uma prática educativa dialógica e transformadora. Nesta conjectura, a intencionalidade pedagógica emerge como um elemento central para uma apropriação crítica e transformadora das TDIC na prática docente \cite{QueirozEtAl2024TDICEducacaoQuimica}. Não se trata apenas de incorporar recursos tecnológicos às práticas tradicionais de ensino, mas de repensar os próprios fundamentos epistemológicos e metodológicos do fazer pedagógico à luz das possibilidades e desafios apresentados pelas TDIC, pois, como argumenta \textcite{Levy2010Cibercultura}, elas não são ferramentas; se constituem em verdadeiras tecnologias da inteligência.

Nesta perspectiva, a intencionalidade pedagógica implica a capacidade do professor de selecionar, adaptar e criar estratégias de ensino e de aprendizagem que explorem as potencialidades das TDIC para promover experiências educativas significativas, contextualizadas e emancipatórias. \textcite{Santaella2010PosHumano} complementa esta visão ao destacar que se vive em uma cultura digital caracterizada pela hibridização de linguagens, suportes e espaços, o que exige dos educadores novas competências para navegar e mediar estes ambientes. Portanto, a intencionalidade pedagógica pressupõe uma compreensão profunda dos aspectos técnicos das tecnologias, bem como de suas dimensões culturais, sociais e cognitivas e de suas implicações para os processos de subjetivação e socialização dos estudantes.

O desenvolvimento de competências digitais docentes não pode ser dissociado, contudo, das condições materiais e institucionais que viabilizam ou obstaculizam a incorporação crítica e criativa das TDIC nas práticas educativas. Como alerta \textcite{Selwyn2017EducacaoTecnologiaCritica}, é necessário evitar tanto o determinismo tecnológico quanto o voluntarismo pedagógico, reconhecendo que a apropriação das TDIC pelos professores é mediada por fatores estruturais, como infraestrutura escolar, políticas públicas, condições de trabalho e formação docente \cite{AfonsoSilvaBedin2024Educitec}.

Neste sentido, a discussão sobre competências digitais docentes deve necessariamente incorporar uma análise crítica das políticas públicas de inserção das TDIC nas escolas e nos cursos de formação de professores, bem como das desigualdades no acesso e uso das tecnologias que caracterizam o cenário educacional brasileiro. \textcite{Gatti2010FormacaoProfessores} destaca que a formação de professores no Brasil apresenta fragilidades históricas que se refletem na dificuldade de incorporação crítica das TDIC, evidenciando a necessidade de políticas públicas que articulem de forma coerente e sistemática a formação inicial e continuada, a infraestrutura tecnológica e as condições de trabalho docente.

\textcite{Bauman2001ModernidadeLiquida} complementa esta análise ao apontar que se vive em uma modernidade líquida caracterizada pela fluidez e a instabilidade das relações sociais e institucionais, o que torna ainda mais complexo o desafio de desenvolver políticas educacionais consistentes e duradouras no campo das TDIC. O autor ainda sintetiza essa condição ao afirmar que “vivemos tempos líquidos. Nada foi feito para durar” \cite[p. 8]{Bauman2001ModernidadeLiquida}, o que impõe à educação o desafio de formar sujeitos capazes de lidar com a transitoriedade das tecnologias e dos saberes. Assim, a partir desta proposição, questiona-se: Como as competências digitais docentes podem ser desenvolvidas de forma crítica e emancipatória, superando abordagens tecnicistas e instrumentalizadas, a partir da articulação entre intencionalidade pedagógica, políticas públicas e espaços formativos?

Essa problematização emerge da tensão central deste texto, que é analisar criticamente a interconexão entre o uso instrumental e a apropriação crítica das Tecnologias Digitais de Informação e Comunicação (TDIC), propondo o Modelo Tridimensional de Apropriação Crítica como referencial para sua incorporação consciente e emancipatória na prática pedagógica. A questão dialoga com autores como \textcite{Feenberg2002TransformingTechnology,Freire2018PedagogiaAutonomia,Selwyn2017EducacaoTecnologiaCritica,Morin2011SeteSaberes,AlmeidaEtAl2022Signos}, ao colocar em xeque os modelos de formação que desconsideram as dimensões ética, política e cultural do uso das TDIC.

\section{Metodologia}\label{sec-normas}
Este texto configura-se como um ensaio teórico \cite{Meneghetti2011EnsaioTeorico} de abordagem qualitativa \cite{LudkeAndre1986PesquisaEducacao}, fundamentado em uma perspectiva crítico-reflexiva sobre o desenvolvimento das competências digitais docentes no contexto da sociedade em rede. A escolha por esta abordagem se justifica pela intenção de promover uma análise conceitual e argumentativa que vá além da descrição empírica de fenômenos, buscando compreender as complexas articulações entre intencionalidade pedagógica, políticas públicas de formação docente e apropriações críticas das TDIC na educação a partir de uma articulação entre referenciais teóricos clássicos e contemporâneos.

Diferente de uma revisão sistemática ou integrativa, que busca mapear exaustivamente a produção científica sobre determinado tema, com critérios rígidos de seleção, análise e síntese, este ensaio privilegia a argumentação autoral, a problematização conceitual e a construção de sentidos, baseando-se na liberdade interpretativa e na interdisciplinaridade. Segundo \textcite[p. 322]{Meneghetti2011EnsaioTeorico}, o ensaio teórico é “uma forma de pesquisa que se apoia no aprofundamento e articulação de ideias, na formulação de proposições e no avanço de compreensões teóricas, com base em uma leitura crítica da realidade e da literatura existente”. Logo, o texto assume uma perspectiva ensaística ao tensionar categorias como intencionalidade pedagógica, racionalidade tecnológica e políticas públicas, não para descrevê-las de forma neutra, mas para reconstruí-las criticamente à luz de um projeto formativo emancipador, fundamentado em epistemologias críticas.

O ensaio se sustenta em uma revisão crítica da literatura, articulando autores de diferentes campos do saber, como a teoria crítica da tecnologia \cite{Feenberg2002TransformingTechnology,Marcuse2015HomemUnidimensional,Habermas2014TecnicaCiencia}, a pedagogia crítica \cite{Freire2018PedagogiaAutonomia,Perrenoud2000CompetenciasEnsinar}, as epistemologias do Sul \cite{Santos2010DiscursoCiencias} e os estudos sobre cibercultura e mediação tecnológica \cite{Levy2010Cibercultura,Santaella2010PosHumano,Kenski2012EducacaoTecnologias}. Essa abordagem teórica evidencia a pluralidade de olhares sobre as TDIC na educação e suas implicações para a constituição do ser docente na contemporaneidade. A análise proposta também se ancora em documentos oficiais recentes, como a Matriz de Saberes Digitais Docentes \cite{BrasilMEC2024MatrizSaberes}, com o objetivo de tensionar discursos institucionais que frequentemente oscilam entre o tecnicismo e o praticismo. Esse conjunto de aportes teóricos foi articulado com originalidade para propor uma visão integradora que recusa tanto a tecnofilia ingênua quanto a tecnofobia paralisante, defendendo uma intencionalidade pedagógica ancorada na mediação crítica, na justiça social e na formação ética do professor contemporâneo.

A seleção dos referenciais teóricos seguiu três critérios principais: i) autores reconhecidos nacional e internacionalmente no campo da teoria crítica da tecnologia, pedagogia crítica e cibercultura; ii) produções acadêmicas que abordam explicitamente a articulação entre tecnologia, educação e formação docente de modo crítico e dialógico; e iii) atualidade e relevância para o debate, priorizando publicações dos últimos 15 anos, sem desconsiderar clássicos fundadores da área. A análise dos textos baseou-se em uma estratégia de leitura crítica e reflexiva, focalizando a identificação de categorias-chave, como: intencionalidade pedagógica, racionalidade tecnológica, apropriação crítica das TDIC, mediação docente e políticas públicas de formação.

O percurso metodológico adotado, portanto, privilegiou a construção argumentativa como estratégia de investigação, considerando que, no ensaio teórico, o conhecimento é produzido pela problematização dos conceitos, pela reflexão interdisciplinar e pela elaboração de proposições que contribuam para o avanço do debate acadêmico e político sobre o tema. Nesse sentido, o texto não se limita à crítica, mas propõe caminhos para a superação de modelos reducionistas e para a constituição de uma prática pedagógica que compreenda a tecnologia como mediação formativa, simbólica e cultural.

\section{Dimensões e desafios das competências digitais docentes}\label{sec-conduta}
A conceituação das competências digitais docentes é um campo em constante evolução, influenciado tanto pelas transformações tecnológicas quanto pelas diferentes perspectivas teóricas sobre o papel das TDIC na educação. \textcite{PerinFreitasCoelho2023CompetenciaDigitalDocente} definem essas competências como um conjunto integrado de conhecimentos, habilidades e atitudes que possibilitam ao professor utilizar as TDIC de forma crítica, criativa e pedagogicamente fundamentada. A visão dos autores supera reducionismos centrados no mero domínio técnico, incorporando dimensões pedagógicas, éticas, críticas e socioculturais. A complexidade dessas dimensões evidencia, porém, que ainda faltam diretrizes formativas capazes de integrá-las de modo sistemático. A ausência de um referencial mais amplo faz com que iniciativas de formação permaneçam fragmentadas, dificultando que o professor desenvolva uma apropriação realmente crítica das tecnologias.

O estudo dos autores identifica cinco categorias: i) manejo de ferramentas digitais; ii) habilidades de informação e comunicação; iii) conhecimentos e habilidades para ensinar; iv) competências pedagógicas para o autodesenvolvimento; e v) competências para lidar com questões socioculturais do trabalho docente. Essas categorias evidenciam que a competência digital não se limita ao “saber fazer”, mas envolve também compreender “por que fazer”, “para que fazer” e “em que contexto fazer”. Nessa linha, \textcite{Kenski2012EducacaoTecnologias} lembra que as tecnologias são expressões culturais capazes de transformar modos de pensar, sentir e agir, exigindo dos professores adaptação e compreensão profunda de suas implicações.

Essa concepção contrasta com visões tecnicistas que reduzem as tecnologias educacionais a instrumentos neutros. \textcite{Marcuse2015HomemUnidimensional} alerta que a racionalidade tecnológica, quando guiada apenas por critérios de eficiência e produtividade, tende a empobrecer a experiência educativa e a restringir a formação humana integral. O documento europeu DigCompEdu \cite{Redecker2017DigCompEdu} reforça essa crítica ao afirmar que competências digitais envolvem práticas inovadoras e o fortalecimento da autonomia discente, e não apenas o uso técnico de ferramentas. Essa crítica revela uma lacuna importante na formação docente: a inexistência de modelos que articulem criticamente técnica, cultura digital e projeto pedagógico. Sem essa articulação, as competências digitais tendem a ser tratadas de forma isolada, o que limita sua potência transformadora.

\textcite{Habermas2014TecnicaCiencia} acrescenta que a educação sofre com a colonização do mundo da vida por sistemas técnico-econômicos, levando as práticas pedagógicas a priorizar metas de desempenho e padronização. No mesmo sentido, \textcite{Selwyn2017EducacaoTecnologiaCritica} critica o determinismo tecnológico, que atribui às ferramentas digitais um suposto poder transformador intrínseco, negligenciando que seu uso é moldado por fatores sociais, culturais e institucionais, como interações entre sujeitos, valores comunitários, políticas escolares e infraestrutura disponível. Essas tensões reforçam a necessidade de avançar para referenciais analíticos que articulem criticidade, intencionalidade pedagógica e condições materiais; lacuna ainda pouco explorada nas pesquisas brasileiras sobre TDIC.

Em contraponto, \textcite{Feenberg2002TransformingTechnology} propõe uma teoria crítica da tecnologia, entendendo-a como construção social e política passível de apropriações alternativas que promovam democracia, justiça social e sustentabilidade. Essa perspectiva implica formar professores para ir além do treinamento técnico, articulando teoria e prática, técnica e pedagogia, reflexão e ação. \textcite{Novoa2009Professores} defende que tal formação deve partir da experiência concreta do professor, estimulando a reflexão sobre sua prática e integrando saberes em contextos reais. No entanto, mesmo com essas contribuições, permanece o desafio de organizar essas perspectivas em um quadro coerente que oriente a formação crítica dos professores. Ainda se sente a falta de um modelo que reconheça a docência como prática situada e que permita ao professor analisar a tecnologia como construção social, e não apenas como recurso pedagógico.

Para o campo específico das competências digitais, isso significa criar espaços formativos que combinem experimentação, exemplificação e colaboração, permitindo que o professor construa sentidos pedagógicos para as tecnologias a partir de seu contexto. \textcite{Imbernon2011FormacaoDocente} destaca que essa formação deve ser contínua, acompanhando transformações sociais, culturais e tecnológicas, e preparando o docente para analisar criticamente situações complexas e responder de forma criativa. \textcite{Tardif2014SaberesDocentes} complementa ao lembrar que os saberes docentes são plurais, temporais e heterogêneos, e, portanto, exigem uma formação igualmente plural e contextualizada, capaz de articular diferentes dimensões do conhecimento profissional no cenário das TDIC. Essa constatação reforça a urgência de avançar para estruturas formativas que superem a dispersão conceitual atualmente presente no campo. Há uma necessidade explícita de modelos críticos que orientem o docente a conectar suas escolhas tecnológicas às finalidades educativas, às condições materiais e às implicações sociopolíticas do uso das TDIC.


\section{Políticas públicas, desafios estruturais e realidades brasileiras na formação digital docente}\label{sec-fmt-manuscrito}
As políticas públicas voltadas à inserção das TDIC na educação e na formação docente ainda se desenvolvem em um cenário de tensões, contradições e descontinuidades, refletindo disputas mais amplas sobre o papel da educação na sociedade contemporânea. Para \textcite{Santos2010DiscursoCiencias}, é preciso analisá-las a partir de uma epistemologia do Sul\footnote{A epistemologia do Sul é uma proposta de valorização dos saberes produzidos fora dos centros hegemônicos do conhecimento, especialmente os do sul global. \textcite{Santos2010DiscursoCiencias} critica o monopólio do pensamento científico ocidental moderno e defende a legitimidade de outras formas de conhecimento (como os saberes indígenas, populares e tradicionais) que foram historicamente silenciados.}, que reconheça a pluralidade de saberes e experiências invisibilizadas pelos discursos hegemônicos sobre tecnologia e educação, valorizando práticas pedagógicas enraizadas nas realidades locais e comprometidas com a justiça cognitiva e social.

Sob esse olhar crítico, evidencia-se que políticas de modernização e inclusão digital, muitas vezes, reproduzem lógicas coloniais de dependência tecnológica e epistemológica, impondo modelos prontos que ignoram as especificidades dos contextos educacionais brasileiros \cite{Apple2013EducatingRightWay}. \textcite{Gatti2010FormacaoProfessores} aponta fragilidades estruturais persistentes: currículos fragmentados, dissociação entre teoria e prática, desarticulação entre formação específica e pedagógica, além da precariedade das condições institucionais. Essas lacunas dificultam uma apropriação crítica e criativa das TDIC, afetando diretamente a qualidade da formação docente.

A fragmentação é particularmente visível nas licenciaturas, onde disciplinas sobre TDIC costumam ser oferecidas de forma isolada e desconectada da prática pedagógica. Nesses casos, prevalecem treinamentos técnicos ou introduções instrumentais a ferramentas, distantes da perspectiva de tecnologia como mediação cultural, epistêmica e política defendida por \textcite{Feenberg2002TransformingTechnology}. Essa lógica impede a construção de uma intencionalidade pedagógica integrada e gera um descompasso entre a complexidade da docência na era digital e a formação realmente ofertada \cite{AfonsoSilvaBedin2024Educitec}.

Como destaca \textcite{Pimenta2012SaberesPedagogicos}, as políticas educacionais oscilam entre o tecnicismo, que reduz o professor a executor de pacotes tecnológicos, e o praticismo, que delega a ele toda a responsabilidade pela inovação sem oferecer condições estruturais adequadas. Assim, a infraestrutura escolar, embora indispensável, ainda é insuficiente para garantir a efetiva mobilização das competências digitais docentes. Ações como o PIBID – Programa Institucional de Bolsa de Iniciação à Docência \cite{Brasil2010Decreto7219} –, o PARFOR – Plano Nacional de Formação de Professores da Educação Básica \cite{Brasil2009Decreto6755} – e a UAB – Universidade Aberta do Brasil \cite{Brasil2006Decreto5800} – ampliaram o acesso à formação, especialmente em regiões periféricas, mas carecem de integração a uma política nacional consistente que desenvolva competências digitais de forma crítica. Frequentemente, o uso das TDIC nesses programas restringe-se à mediação remota de conteúdos, sem fomentar reflexão, autoria ou transformação pedagógica.

O PNIEC – Programa Nacional de Inovação da Educação Conectada – \cite{Brasil2017Decreto9204} representa um avanço pontual, mas enfrenta desigualdades regionais, falta de infraestrutura, ausência de formação continuada crítica e escassez de tempo pedagógico. Em muitas escolas, especialmente nas periferias, a exclusão digital é estrutural e política, marcada por equipamentos obsoletos, conectividade precária, ausência de suporte técnico e espaços inadequados. É neste contexto que \textcite{Selwyn2017EducacaoTecnologiaCritica} alerta para o risco do “fetichismo tecnológico”, quando políticas priorizam a compra de equipamentos sem considerar aspectos como manutenção, suporte, tempo de uso e condições de trabalho docente.

À luz dessa discussão, torna-se pertinente problematizar em que medida políticas e programas como o PIBID, o PARFOR e o PNIEC, embora fundamentais para ampliar o acesso à formação docente, também podem reproduzir o que \textcite{Selwyn2017EducacaoTecnologiaCritica} denomina fetichismo tecnológico: a crença de que a simples introdução de recursos digitais é capaz de, por si só, promover inovação pedagógica e transformação educativa. Em muitos contextos, a ênfase recai sobre a disponibilização de equipamentos, plataformas ou ambientes virtuais, enquanto se negligenciam debates estruturais sobre intencionalidade pedagógica, condições de trabalho, epistemologias formativas e autonomia docente. Nessa lógica fetichizada, as tecnologias tornam-se fins em si mesmas, obscurecendo desigualdades materiais e epistemológicas, além de desresponsabilizar o Estado quanto ao dever de garantir políticas integradas, críticas e dialógicas para a formação. Assim, programas voltados à iniciação à docência ou à formação emergencial correm o risco de reforçar usos instrumentais e performáticos das TDIC, em vez de promover uma apropriação consciente, situada e emancipatória, alinhada aos princípios de justiça cognitiva defendidos neste ensaio.

Esse quadro revela uma racionalidade tecnocrática que ignora a centralidade do professor como mediador e sujeito epistêmico, corroborando a crítica de \textcite{Marcuse2015HomemUnidimensional} à lógica funcionalista da técnica. \textcite{Postman2011Technopoly} acrescenta que a tecnologia é frequentemente tratada como inevitável e inquestionável, desconsiderando seus impactos sociais e culturais. Para \textcite{Selwyn2017EducacaoTecnologiaCritica}, a infraestrutura deve ser entendida como um ecossistema sociotécnico que envolve políticas institucionais, culturas escolares e redes de apoio capazes de viabilizar apropriações significativas. Nessa mesma direção, \textcite{Hargreaves2004EnsinoSociedadeConhecimento} defende que a sociedade do conhecimento exige infraestrutura tecnológica, social e organizacional que sustente colaboração, experimentação e reflexão coletiva entre professores.

Contudo, embora as TDIC estejam associadas a metodologias inovadoras – como salas híbridas, gamificação e metodologias ativas –, o cotidiano docente ainda é moldado por políticas de padronização, metas e avaliações externas que limitam a criatividade e a reflexão crítica. No Brasil, isso cria um paradoxo: convoca-se o professor a inovar, mas dentro de molduras rígidas que inibem o risco pedagógico. Nesse contexto, a intencionalidade pedagógica acaba subjugada a uma cultura performática, contrária aos princípios de uma educação emancipatória defendida por \textcite{Freire2018PedagogiaAutonomia,Santos2010DiscursoCiencias}. Para \textcite{McLaren2015PedagogyInsurrection}, o uso crítico das tecnologias é parte fundamental de uma educação voltada à resistência e à justiça social. No entanto, essa insistência na criticidade evidencia a ausência de referenciais mais sólidos para orientar o uso das TDIC. Sem tal orientação, o discurso da inovação tende a ser capturado por agendas performáticas que esvaziam sua potência emancipatória. É justamente essa lacuna que reitera a necessidade de repensar como se estrutura a formação docente frente às tecnologias.

O protagonismo discente é outro elemento central para compreender as competências digitais docentes. \textcite{Freire2018PedagogiaAutonomia} lembra que a educação autêntica se constrói com o educando, em diálogo, reconhecendo e valorizando seus saberes e potencialidades. \textcite{Habermas2014TecnicaCiencia} reforça a importância da ação comunicativa orientada ao entendimento mútuo, papel que o professor exerce como mediador ético e epistêmico das interações pedagógicas mediadas pelas TDIC. \textcite{Santaella2010PosHumano} caracteriza os jovens como leitores ubíquos, que transitam entre diferentes linguagens e plataformas, trazendo experiências que podem enriquecer, mas também desafiar a prática escolar.

A ausência de diretrizes formativas integradas faz com que essa mediação se concretize de forma desigual ou intuitiva, dependendo mais de esforços individuais do que de uma política formativa sólida. Este cenário indica que não basta reconhecer os estudantes como sujeitos ubíquos e conectados; é necessário oferecer ao professor condições epistemológicas e pedagógicas para interpretar criticamente essas experiências digitais e transformá-las em oportunidades educativas reais.

Por isso, o compromisso do estudante com as orientações pedagógicas deve ser entendido como engajamento ativo, não como submissão. \textcite{Morin2011SeteSaberes} defende uma educação que desenvolva autonomia, criticidade e responsabilidade, preparando os jovens para se adaptarem às transformações tecnológicas e contribuírem para futuros mais justos e sustentáveis. Contudo, formar para a autonomia e criticidade em um ambiente tecnologicamente saturado requer mais do que recomendações teóricas; exige a construção de quadros formativos que ajudem o docente a ler, tensionar e reconfigurar as tecnologias de acordo com princípios éticos e finalidades formativas claras.

A intencionalidade pedagógica no uso das TDIC é o núcleo das competências digitais críticas, diferenciando quem apenas utiliza ferramentas de quem as incorpora de forma reflexiva e transformadora. \textcite{Perrenoud2000CompetenciasEnsinar} destaca que ensinar implica agir na urgência e decidir na incerteza, mobilizando saberes e estratégias alinhadas a finalidades conscientes. Nesse sentido, o professor deve planejar, adaptar e criar experiências de aprendizagem significativas, explorando as tecnologias para desenvolver tanto conteúdos específicos quanto metacompetências – como autonomia, criatividade, colaboração, pensamento crítico e resolução de problemas complexos \cite{Levy2010Cibercultura}.

Apesar disso, percebe-se que muitos cursos e programas de formação continuam a enfatizar ferramentas e metodologias, deixando em segundo plano a reflexão sobre condições, implicações e decisões que orientam sua escolha. Essa fragilidade formativa compromete a capacidade docente de articular intencionalidade e criticidade no uso das TDIC, revelando a necessidade de referenciais mais potentes que sustentem essa articulação.

As TDIC, enquanto mediadoras culturais, também se articulam às contribuições de \textcite{Vygotsky2001PensamentoLinguagem}, que enfatiza a mediação semiótica no desenvolvimento das funções psicológicas superiores. No atual cenário de ecologia midiática descrito por \textcite{Santaella2010PosHumano}, o professor precisa transitar entre múltiplas linguagens e suportes, explorando suas complementaridades. \textcite{Levy2010Cibercultura} amplia esse entendimento ao propor a inteligência coletiva como nova forma de cognição distribuída e colaborativa. \textcite{Papert1993ChildrensMachine} reforça que a tecnologia deve ser apropriada ativamente por professores e estudantes, estimulando autoria e criatividade.

Ainda assim, a apropriação ativa das tecnologias permanece limitada quando não acompanhada de processos formativos que permitam ao docente compreender a tecnologia como fenômeno cultural, político e epistemológico. A falta dessa mediação estruturada transforma o uso das TDIC em prática contingente, dependente de iniciativas isoladas em vez de constituir uma orientação pedagógica coletiva e consciente.

Por fim, \textcite{Bauman2001ModernidadeLiquida} lembra que a modernidade líquida impõe instabilidade e aceleração constantes, exigindo do professor mais do que adaptação passiva: é preciso apropriação crítica e criativa das TDIC, considerando riscos e potencialidades. Para \textcite{Hargreaves2004EnsinoSociedadeConhecimento}, isso implica que docentes sejam consumidores e produtores de conhecimento pedagógico, integrados em comunidades profissionais de aprendizagem. Na perspectiva de \textcite{Santos2010DiscursoCiencias}, essa construção deve se apoiar na ecologia de saberes, favorecendo diálogos horizontais e práticas pedagógicas comprometidas com a transformação social.

Diante dessas tensões, torna-se evidente que a formação docente ainda opera sem um horizonte crítico suficientemente claro para orientar o uso das TDIC em direção à justiça cognitiva e à transformação social. A dispersão conceitual e a fragmentação das práticas formativas reforçam a urgência de construir referenciais que ajudem o professor a transitar entre técnica, reflexão e ação pedagógica de forma integrada e situada.


\section{A tecnologia como organismo vivo na formação e nas competências digitais críticas docentes}\label{sec-formato}
Compreender as TDIC como meros instrumentos pedagógicos, desvinculados de intencionalidade e contexto, é negligenciar seu papel constitutivo nos modos contemporâneos de ensinar, aprender e exercer a docência. No campo das competências digitais, é necessário avançar para uma abordagem que reconheça a tecnologia como mediação simbólica, epistêmica e formativa; um organismo vivo que se transforma e transforma, interagindo com professores, estudantes e ambientes de aprendizagem de forma dinâmica. Na prática, isso significa que uma plataforma educacional, por exemplo, não é apenas um recurso técnico; ela molda rotinas, reorganiza tempos e espaços, e altera a própria natureza das interações pedagógicas.

Compreender a tecnologia como organismo vivo implica reconhecer que sua presença nas práticas docentes não é estática e nem linear, mas emergente e relacional. Essa perspectiva evidencia que a formação docente não pode restringir-se a repertórios prontos de habilidades, pois requer ecossistemas formativos capazes de acompanhar, tensionar e ressignificar continuamente as interações sociotécnicas que constituem a prática pedagógica. A ausência dessa ecologia formativa crítica mantém a tecnologia em estado de latência – presente, mas não plenamente apropriada.

Nesse sentido, \textcite{Levy2010Cibercultura} propõe o conceito de “tecnologias da inteligência”, ressaltando que as TDIC apoiam, mas também reconfiguram processos cognitivos, comunicacionais e culturais. Um professor que integra mapas colaborativos online para ensinar geografia não está somente substituindo o mapa físico, mas possibilitando novos fluxos de construção coletiva do conhecimento. Essa visão converge com \textcite{Feenberg2002TransformingTechnology}, que, a partir da teoria crítica da tecnologia, lembra que todo artefato digital está imerso em relações sociais e de poder, e pode ser apropriado tanto para fins emancipadores quanto para reforço de lógicas dominantes. Nesse ponto, conecta-se diretamente a ideia de competências críticas e reflexivas do docente, pois exige dele a capacidade de analisar implicações éticas, políticas e culturais de cada escolha tecnológica.

A formação docente que reconhece essa complexidade precisa ir além de treinamentos técnicos. Como defendem \textcite{BedinDelPino2018Signos}, desenvolver competências digitais implica articular dimensões críticas e reflexivas com a intencionalidade pedagógica, para que decisões tecnológicas estejam alinhadas a propósitos educacionais transformadores. Isso pode se materializar em formações organizadas em ciclos reflexivos, nos quais os docentes experimentam ferramentas (como simuladores de laboratório em química), analisam resultados e repensam usos com base nos contextos concretos de sua prática. A \textcite{OEI2020MarcoCompetenciaDigitalDocente} reforça essa necessidade, ao propor que uma abordagem crítica das TDIC avalie eticamente seus usos e enfrente desigualdades sociais e exclusões digitais.

As competências digitais críticas, nessa perspectiva, não se limitam ao “saber fazer” com as tecnologias, mas ampliam-se para o “saber pensar sobre elas”: seus sentidos, usos, limites e implicações. Isso envolve compreender as TDIC como linguagens e culturas que, ao serem incorporadas à prática pedagógica, influenciam diretamente a identidade profissional docente. \textcite{Santaella2010PosHumano} lembra que se vive em uma ecologia midiática que exige do educador transitar entre códigos, plataformas e sensibilidades distintas, algo que se articula tanto à intencionalidade pedagógica quanto às competências críticas e reflexivas do professor.

Nesse horizonte, o conceito de inteligência coletiva, de \textcite{Levy2010Cibercultura}, e a epistemologia do Sul, de \textcite{Santos2010DiscursoCiencias}, reforçam que as TDIC podem sustentar redes colaborativas, coautoria e diálogo entre saberes historicamente marginalizados. Um exemplo é o uso de rádios comunitárias online por professores em regiões rurais, ampliando o acesso à informação e promovendo aprendizagens contextualizadas. Essa concepção desloca a tecnologia do lugar de fim em si mesma para o de meio de formação integral, permitindo que o professor se forme continuamente como sujeito ético, estético e político.

Essa visão também implica repensar a relação entre professores e estudantes. Reconhecer o protagonismo discente na cultura digital não significa abdicar da mediação docente, mas fortalecê-la com base no diálogo e na problematização. Como lembra \textcite{Freire2018PedagogiaAutonomia}, a educação autêntica é construída com os estudantes, mediada pelo mundo – e, no contexto das TDIC, esse mundo é atravessado por algoritmos, plataformas e fluxos informacionais que precisam ser compreendidos criticamente. \textcite{Morin2011SeteSaberes} complementa ao destacar a importância de um pensamento complexo capaz de articular saberes, superar fragmentações e evitar apropriações superficiais das tecnologias. Assim, o comprometimento dos estudantes com o uso pedagógico das TDIC não se traduz em submissão, mas em engajamento ativo e crítico, em consonância com todos os eixos do modelo proposto.

Diante desse cenário, propõe-se um Modelo Tridimensional de Apropriação Crítica (MTAC) das TDIC (\Cref{fig1}), estruturado em três eixos interdependentes: i) \textit{condições estruturais} – contemplando infraestrutura tecnológica, políticas públicas coerentes e condições dignas de trabalho docente; ii) \textit{competências críticas e reflexivas} – que envolvem compreender implicações sociais, políticas e epistemológicas das tecnologias, alinhando-as a princípios de justiça social e emancipação; e iii) \textit{intencionalidade pedagógica} – capacidade de planejar, selecionar e integrar TDIC a partir de objetivos formativos claros, contextualizados e significativos. A articulação desses eixos busca orientar a prática docente para que o uso das tecnologias seja, além de eficiente, eticamente fundamentado, socialmente relevante e pedagogicamente transformador, respondendo às demandas contemporâneas de uma educação democrática e consciente.

\begin{figure}[h!]
\centering
\begin{minipage}{.9\textwidth}
 \includegraphics[width=\textwidth]{Fig1.png}
 \caption{Modelo Tridimensional de Apropriação Crítica.}
 \label{fig1}
 \source{Elaboração própria.}
\end{minipage}
\end{figure}

O MTAC das TDIC organiza-se em três eixos articulados, cada um composto por uma descrição que delimita seu escopo, e por elementos-chave que explicitam os aspectos essenciais para sua aplicação. O primeiro eixo, condições estruturais, reúne os fatores materiais, institucionais e políticos que sustentam a integração das TDIC, sendo imprescindível para que qualquer proposta formativa possa se efetivar de forma equitativa e consistente. O segundo eixo, \textit{competências críticas e reflexivas}, aborda a capacidade docente de analisar, compreender e intervir criticamente nas implicações sociais, políticas e epistemológicas das tecnologias, orientando seu uso para finalidades éticas e emancipadoras. Já o terceiro eixo, \textit{intencionalidade pedagógica}, enfatiza o planejamento e a integração consciente das TDIC em propostas de ensino contextualizadas e transformadoras.

Embora sejam apresentados de forma estruturada, esses eixos operam de maneira interdependente e encontram pleno sentido quando inseridos dentro de um quadro mais amplo, representado pela \textit{realidade} ou \textit{contexto} em que se desenvolvem. Este nível superior, mais abrangente que o próprio modelo, determina as possibilidades, os limites e as necessidades de adaptação de cada eixo, reforçando que a apropriação crítica das TDIC é sempre situada e relacional.

Ainda, reforça-se que a implementação do modelo na formação docente deve ocorrer por meio de ciclos reflexivos, nos quais teoria, prática e análise crítica se retroalimentam continuamente. Em vez de uma sequência linear que separa momentos de estudo e momentos de aplicação, propõe-se um movimento integrado no qual o diagnóstico das condições estruturais, a construção de competências críticas e a definição da intencionalidade pedagógica sejam revisitados a cada nova experiência formativa. Esse processo cíclico possibilita ajustar estratégias, aprofundar compreensões e ressignificar o uso das tecnologias de acordo com os desafios e as oportunidades que emergem no contexto real, garantindo uma formação situada, participativa e orientada à transformação pedagógica, como propõe a \Cref{tbl1}.

\begin{table}[h!]
\centering
\begin{small}
\begin{threeparttable}
\caption{Protótipo de implementação do MTAC das TDIC.}
\label{tbl1}
\begin{tabular}{>{\raggedright\arraybackslash}p{2cm} >{\raggedright\arraybackslash}p{2cm} >{\raggedright\arraybackslash}p{2.85cm} >{\raggedright\arraybackslash}p{2.85cm} >{\raggedright\arraybackslash}p{2.85cm}}
\toprule 
Etapa do ciclo & Eixo do modelo & Objetivo & Ações formativas & Produto esperado \\ 
\midrule
\emph{Diagnóstico e Sensibilização} & Condições Estruturais & Mapear infraestrutura, políticas e contexto de trabalho & Oficina de mapeamento participativo, entrevistas coletivas & Mapa da realidade digital da escola \\
\emph{Fundamentação Crítica} & Competências Críticas e Reflexivas & Analisar implicações sociais, políticas e epistemológicas das TDIC & Estudos dirigidos, análise de políticas públicas, debates orientados & Síntese crítica sobre TDIC e contexto escolar \\
\emph{Planejamento Pedagógico Intencional} & Intencionalidade Pedagógica & Elaborar propostas de integração crítica das TDIC & Oficina de cocriação de planos de aula e projetos & Plano de aula ou projeto com uso crítico das TDIC \\
\emph{Experimentação em Contexto Real} & Todos os Eixos & Aplicar e observar a proposta em situação real ou simulada & Microensino, prática supervisionada, observação entre pares & Registro de aplicação (vídeo, fotos, diário reflexivo) \\
\emph{Reflexão e Replanejamento} & Competências Críticas e Reflexivas + Intencionalidade & Avaliar e ajustar as práticas a partir da experiência & Rodas de conversa, análise colaborativa de registros & Plano ajustado e estratégias refinadas \\
\emph{Socialização e Comunidade de Prática} & Todos os Eixos & Compartilhar experiências e manter aprendizagem contínua & Encontros periódicos, repositório colaborativo online & Rede ativa de professores e banco de práticas \\ 
\bottomrule
\end{tabular}
\source{Elaboração própria.}
\end{threeparttable}
\end{small}
\end{table}

Ainda, para aprofundar a compreensão das competências digitais docentes e criar uma ponte direta com o MTAC das TDIC, na \Cref{tbl2} expõe-se a categorização proposta por \textcite{PerinFreitasCoelho2023CompetenciaDigitalDocente}, vinculando-as a descrição, exemplos e conexão com o modelo. Nela, cada categoria é descrita de forma objetiva, acompanhada de exemplos práticos que ilustram sua aplicação no cotidiano escolar e de sua relação com os eixos estruturantes do modelo. Essa sistematização torna visível a complexidade dessas competências, bem como antecipa a lógica integradora do modelo, evidenciando como conhecimentos, habilidades e atitudes se articulam com dimensões estruturais, críticas e pedagógicas.

A partir da \Cref{tbl2}, percebe-se que a articulação entre as categorias de competência digital docente e o modelo proposto permite compreender a complexidade dessas competências como dimensões interdependentes que se retroalimentam no processo formativo. Ao integrar diretamente as categorias no modelo, buscou-se ampliar a clareza conceitual e operacional, pois cada eixo é sustentado por competências específicas que, ao mesmo tempo, dialogam com as demais. Essa organização contribui para que a formação docente com TDIC seja concebida como um processo dinâmico e situado, no qual a intencionalidade pedagógica orienta a seleção, a adaptação e a criação de práticas que considerem as demandas do contexto, os objetivos educacionais e as potencialidades das tecnologias.

\begin{table}[h]
\centering
\begin{small}
\begin{threeparttable}
\caption{Categorias de competências digitais docentes e sua relação com o MTAC.}
\label{tbl2}
\begin{tabular}{>{\raggedright\arraybackslash}p{3cm} >{\raggedright\arraybackslash}p{3.35cm} >{\raggedright\arraybackslash}p{3.35cm} >{\raggedright\arraybackslash}p{3.35cm}}
\toprule 
Categoria & Descrição & Exemplos práticos & Conexão com o modelo \\ 
\midrule
\emph{Manejo de ferramentas digitais} & Capacidade de operar recursos e aplicativos de forma eficiente, adaptando-os a diferentes situações de ensino. & Utilizar plataformas de videoconferência, editar vídeos para aulas, criar quizzes online. & Relaciona-se ao eixo condições estruturais, pois depende de infraestrutura e acesso a tecnologias adequadas. \\
\emph{Habilidades de informação e comunicação} & Competência para buscar, selecionar, avaliar e comunicar informações de forma crítica e ética. & Criar atividades de curadoria de conteúdos digitais, ensinar verificação de fake news. & Dialoga com competências críticas e reflexivas, por demandar análise, filtragem e uso responsável da informação. \\
\emph{Conhecimentos e habilidades para ensinar} & Capacidade de integrar as TDIC ao currículo e aos objetivos pedagógicos. & Planejar aulas híbridas, utilizar simulações interativas em Ciências. & Conecta-se à intencionalidade pedagógica, pois envolve alinhar tecnologia e objetivos educacionais. \\
\emph{Competências pedagógicas para autodesenvolver-se} & Habilidade de aprender continuamente e atualizar-se em relação às tecnologias e metodologias. & Participar de cursos online, explorar novas ferramentas em comunidades docentes. & Transita entre competências críticas e reflexivas e intencionalidade pedagógica, já que envolve reflexão e aplicação prática. \\
\emph{Competências para lidar com questões socioculturais} & Capacidade de compreender e intervir nas implicações sociais, culturais e éticas do uso das TDIC. & Discutir inclusão digital, debater privacidade e uso de dados com estudantes. & Vincula-se fortemente a competências críticas e reflexivas, pois considera o contexto e os impactos sociais do uso tecnológico. \\ 
\bottomrule
\end{tabular}
\source{Elaboração própria.}
\end{threeparttable}
\end{small}
\end{table}

A adoção do MTAC das TDIC na formação docente representa um avanço significativo ao integrar, de forma articulada, fatores estruturais, competências críticas e intencionalidade pedagógica. Essa perspectiva contribui para superar as abordagens tecnicistas e fragmentadas, proporcionando aos professores condições concretas e suporte formativo para que possam dominar ferramentas digitais, bem como compreender suas implicações éticas, políticas e sociais. Ao incorporar infraestrutura adequada, políticas públicas coerentes e condições de trabalho dignas, o modelo estabelece as bases materiais indispensáveis para que a prática pedagógica mediada por tecnologias seja efetivamente transformadora, respondendo às demandas contemporâneas por uma educação democrática, inclusiva e comprometida com a justiça social.

Além disso, o modelo estimula uma formação docente que valoriza a reflexão crítica e a intencionalidade pedagógica, favorecendo a criação de experiências de aprendizagem significativas, contextualizadas e emancipadoras. Ao promover o desenvolvimento de competências críticas e reflexivas, os professores tornam-se capazes de selecionar e integrar as TDIC de forma consciente, potencializando a autonomia e o protagonismo estudantil sem abrir mão do papel mediador e ético que lhes cabe. Com isso, o modelo favorece a construção de ecossistemas educacionais mais colaborativos, criativos e adaptáveis, preparados para lidar com a fluidez e a complexidade da sociedade digital, ao mesmo tempo que preservam os princípios humanistas e emancipatórios que orientam uma educação comprometida com a transformação social.


\section{Conclusão}\label{sec-modelo}
Este ensaio buscou problematizar o desenvolvimento das competências digitais docentes sob uma perspectiva crítica, articulando intencionalidade pedagógica, políticas públicas de formação e apropriação das TDIC em contextos educacionais marcados por profundas desigualdades estruturais. Com base em autores como Freire, Feenberg, Selwyn, Lévy e Morin, a reflexão teórica construída rejeita reducionismos tecnicistas e neutralidades ideológicas, questionando os discursos hegemônicos sobre inovação tecnológica na educação.

A principal contribuição deste trabalho consiste na defesa da competência digital como construção complexa, situada e relacional, que se materializa em práticas pedagógicas emancipadoras, mediações culturais e posicionamentos ético-políticos. Ao reconhecer a intencionalidade pedagógica como eixo estruturante da docência no contexto digital, amplia-se o conceito de competência digital para além do domínio técnico, incorporando saberes reflexivos, interdisciplinares e colaborativos. Nesse horizonte, propõe-se repensar programas públicos como PIBID, PARFOR, UAB e PNIEC sob uma perspectiva crítica, capaz de superar a fragmentação e potencializar transformações pedagógicas significativas.

A análise dialoga ainda com marcos internacionais, como o DigCompEdu, a \textcite{UNESCO2022QuadroTICProfessores} e a Rede Ibero-Americana, evidenciando que a formação digital crítica é uma demanda global que requer políticas robustas, infraestrutura adequada e valorização profissional. Essas referências reforçam a urgência de consolidar comunidades profissionais de aprendizagem baseadas na colaboração, na autoria docente e na justiça cognitiva.

Ademais, o MTAC das TDIC apresentado constitui uma ferramenta conceitual e prática para orientar a formação docente tecnológica de maneira crítica, integrada e contextualizada. Ao articular eixos claros, descrições precisas e elementos-chave, ele possibilita compreender as competências digitais docentes como um conjunto dinâmico de saberes, atitudes e práticas que se constroem na relação entre professores, tecnologias e contextos socioculturais. Mais do que prescrever usos ou ferramentas, o modelo oferece um referencial para que a formação docente desenvolva a intencionalidade pedagógica, a reflexão ética e a capacidade de adaptação criativa diante dos desafios e potencialidades das TDIC, contribuindo para a consolidação de uma prática educativa emancipada e socialmente comprometida.

Como limitação, reconhece-se a ausência de investigação empírica sobre a manifestação dessas competências digitais críticas na prática formativa em contextos reais. Estudos futuros podem avançar na análise de experiências concretas de implementação de políticas e práticas de formação digital, bem como na criação de modelos avaliativos que articulem referenciais críticos a indicadores qualitativos de competência digital docente.

Conclui-se que o desenvolvimento de competências digitais críticas exige a articulação entre políticas públicas coerentes, formação docente reflexiva e intencionalidade pedagógica orientada à transformação social. Superar visões instrumentais e reconhecer o professor como sujeito epistêmico e mediador cultural é condição essencial para fortalecer práticas pedagógicas que sustentem uma educação democrática, dialógica e tecnologicamente consciente.


\printbibliography\label{sec-bib}
% if the text is not in Portuguese, it might be necessary to use the code below instead to print the correct ABNT abbreviations [s.n.], [s.l.]
%\begin{portuguese}
%\printbibliography[title={Bibliography}]
%\end{portuguese}

\begin{dataavailability}
\txtdataavailability{databody} % options: dataavailable, dataonly, databody, datanotav, nodata
\end{dataavailability}


\end{document}

