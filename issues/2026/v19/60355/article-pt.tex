% !TEX TS-program = XeLaTeX
% use the following command:
% all document files must be coded in UTF-8
\documentclass[portuguese]{textolivre}
% build HTML with: make4ht -e build.lua -c textolivre.cfg -x -u article "fn-in,svg,pic-align"

\journalname{Texto Livre}
\thevolume{19}
%\thenumber{1} % old template
\theyear{2026}
\receiveddate{\DTMdisplaydate{2025}{7}{16}{-1}} % YYYY MM DD
\accepteddate{\DTMdisplaydate{2025}{10}{13}{-1}}
\publisheddate{\today}
\corrauthor{Jefferson do Carmo Andrade Santos}
\articledoi{10.1590/1983-3652.2026.60355pt}
\articleid{60355pt}
%\articleid{NNNN} % if the article ID is not the last 5 numbers of its DOI, provide it using \articleid{} commmand 
% list of available sesscions in the journal: articles, dossier, reports, essays, reviews, interviews, editorial
\articlesessionname{articles}
\runningauthor{Santos e Boa Sorte} 
%\editorname{Leonardo Araújo} % old template
\sectioneditorname{Daniervelin Pereira~\orcid{0000-0003-1861-3609}}
\layouteditorname{Leonardo Araujo~\orcid{0000-0003-3884-2177}}

\title{Letramentos digitais na educação em língua inglesa}
\othertitle{Digital literacies in English language education}
% if there is a third language title, add here:
%\othertitle{Artikelvorlage zur Einreichung beim Texto Livre Journal}

\author[1]{Jefferson do Carmo Andrade Santos~\orcid{0000-0002-3299-0948}\thanks{Email: \href{mailto:jeffer.leitor@gmail.com}{jeffer.leitor@gmail.com}}}
\author[2]{Paulo Boa Sorte~\orcid{0000-0002-0785-5998}\thanks{Email: \href{mailto:pauloboasorte@academico.ufs.br}{pauloboasorte@academico.ufs.br}}}
\affil[1]{Universidade Estadual de Feira de Santana, Feira de Santana, Bahia, Brasil.}
\affil[2]{Universidade Federal de Sergipe, São Cristóvão, Sergipe, Brasil.}

\addbibresource{article.bib}
% use biber instead of bibtex
% $ biber article

% used to create dummy text for the template file
\definecolor{dark-gray}{gray}{0.35} % color used to display dummy texts
\usepackage{lipsum}
\SetLipsumParListSurrounders{\colorlet{oldcolor}{.}\color{dark-gray}}{\color{oldcolor}}

% used here only to provide the XeLaTeX and BibTeX logos
\usepackage{hologo}

% if you use multirows in a table, include the multirow package
\usepackage{multirow}

% provides sidewaysfigure environment
\usepackage{rotating}

% CUSTOM EPIGRAPH - BEGIN 
%%% https://tex.stackexchange.com/questions/193178/specific-epigraph-style
\usepackage{epigraph}
\renewcommand\textflush{flushright}
\makeatletter
\newlength\epitextskip
\pretocmd{\@epitext}{\em}{}{}
\apptocmd{\@epitext}{\em}{}{}
\patchcmd{\epigraph}{\@epitext{#1}\\}{\@epitext{#1}\\[\epitextskip]}{}{}
\makeatother
\setlength\epigraphrule{0pt}
\setlength\epitextskip{0.5ex}
\setlength\epigraphwidth{.7\textwidth}
% CUSTOM EPIGRAPH - END

% to use IPA symbols in unicode add
%\usepackage{fontspec}
%\newfontfamily\ipafont{CMU Serif}
%\newcommand{\ipa}[1]{{\ipafont #1}}
% and in the text you may use the \ipa{...} command passing the symbols in unicode

% LANGUAGE - BEGIN
% ARABIC
% for languages that use special fonts, you must provide the typeface that will be used
% \setotherlanguage{arabic}
% \newfontfamily\arabicfont[Script=Arabic]{Amiri}
% \newfontfamily\arabicfontsf[Script=Arabic]{Amiri}
% \newfontfamily\arabicfonttt[Script=Arabic]{Amiri}
%
% in the article, to add arabic text use: \textlang{arabic}{ ... }
%
% RUSSIAN
% for russian text we also need to define fonts with support for Cyrillic script
% \usepackage{fontspec}
% \setotherlanguage{russian}
% \newfontfamily\cyrillicfont{Times New Roman}
% \newfontfamily\cyrillicfontsf{Times New Roman}[Script=Cyrillic]
% \newfontfamily\cyrillicfonttt{Times New Roman}[Script=Cyrillic]
%
% in the text use \begin{russian} ... \end{russian}
% LANGUAGE - END

% EMOJIS - BEGIN
% to use emoticons in your manuscript
% https://stackoverflow.com/questions/190145/how-to-insert-emoticons-in-latex/57076064
% using font Symbola, which has full support
% the font may be downloaded at:
% https://dn-works.com/ufas/
% add to preamble:
% \newfontfamily\Symbola{Symbola}
% in the text use:
% {\Symbola }
% EMOJIS - END

% LABEL REFERENCE TO DESCRIPTIVE LIST - BEGIN
% reference itens in a descriptive list using their labels instead of numbers
% insert the code below in the preambule:
%\makeatletter
%\let\orgdescriptionlabel\descriptionlabel
%\renewcommand*{\descriptionlabel}[1]{%
%  \let\orglabel\label
%  \let\label\@gobble
%  \phantomsection
%  \edef\@currentlabel{#1\unskip}%
%  \let\label\orglabel
%  \orgdescriptionlabel{#1}%
%}
%\makeatother
%
% in your document, use as illustraded here:
%\begin{description}
%  \item[first\label{itm1}] this is only an example;
%  % ...  add more items
%\end{description}
% LABEL REFERENCE TO DESCRIPTIVE LIST - END


% add line numbers for submission
%\usepackage{lineno}
%\linenumbers

\begin{document}
\maketitle

\begin{polyabstract}
\begin{abstract}
Este artigo objetiva apresentar e discutir, com base nas teorias dos letramentos digitais, educação e tecnologia, os dados de uma pesquisa desenvolvida a partir da vivência em dois \textit{campi} de um Instituto Federal de Educação, Ciência e Tecnologia da Região Nordeste do Brasil. Neste trabalho, letramentos digitais são entendidos como práticas sociais envolvendo usos de dispositivos digitais, mas, ao mesmo tempo, como a hibridização de performances com tecnologias de suportes diversos, a exemplo dos dispositivos analógicos e eletrônicos que têm emergido desde o século passado. As observações estruturadas e as entrevistas narrativas foram os instrumentos de geração de dados, sendo o método de análise baseado em \textcite{freeman1998teacher_research} e nos mecanismos de codificação de \textcite{auerbach2003qualitative}. Os resultados lançaram luz a temas sociais sobre os usos de tecnologias, a necessidade de pensar e problematizar o termo “tecnologia”, assim como a relevância de implementar práticas sociais que tencionam a hibridização de dispositivos em vez da dissociação de tecnologias.

\keywords{Educação \sep Letramentos digitais \sep Língua inglesa \sep Tecnologias}
\end{abstract}

\begin{english}
\begin{abstract}
This paper aims to display and discuss data from research done through an experience in two campuses of a Federal Institute of Education, Science and Technology in the Northeast of Brazil, based on theories concerning digital literacies, education, and technology. In this study, digital literacies are understood as social practices regarding the use of digital devices as well as hybrid performances with technologies of varied supports, such as analog and electronic devices that have been emerging since the last century. Structured observations and narrative interviews were the instruments adopted for data collection, and the analytical method was based on \textcite{freeman1998teacher_research} along with the coding mechanisms of \textcite{auerbach2003qualitative}. As a result, the emerging topics highlighted the social use of technologies, the relevance of problematizing and thinking about meanings of the term “technology”, as well as the importance of implementing social practices focused on the hybridization of devices instead of on the dissociation of technologies.

\keywords{Education \sep Digital literacies \sep English language \sep Technologies}
\end{abstract}
\end{english}
% if there is another abstract, insert it here using the same scheme
\end{polyabstract}

\section{Para início de conversa…}\label{sec-intro}

\epigraph{Sigo o anúncio e vejo

Em formas de desejo

Um sabonete

Em formas de sorvete

Acordo e durmo

Na televisão [...]

E na lua sou
 
Mais um cosmonauta patrocinador...

\cite{alencar1969comunicacao}}

A epígrafe acima traz a canção “Comunicação”, de autoria de Edson Alencar e Hélio Matheus. A canção ficou amplamente conhecida na voz da cantora Vanusa, em 1969, quando participou do Festival da Música Popular Brasileira da Rede Record e, em gravações posteriores, nas interpretações das cantoras Elis Regina e Dóris Monteiro. Nessa canção, a televisão, então meio de comunicação ainda emergente, era apresentada como um dispositivo que estava influenciando e ditando o cotidiano da sociedade.

No ano de 1969, o mundo estava em meio a feitos significativos, como, por exemplo, a ida do homem à Lua, os primeiros passos dos projetos da internet e os aprimoramentos constantes na indústria fonográfica. Especificamente no Brasil, a década de 1960 representou um salto na área da comunicação, desde a ampliação das redes televisivas iniciadas na década anterior até a criação da Empresa Brasileira de Telecomunicações (Embratel).

Os efeitos sociais dos novos dispositivos tecnológicos em emergência nas décadas de 1950 e 1960 já vinham sendo analisados por estudiosos da área da comunicação, a exemplo de Marshall McLuhan. Na obra \textit{Os meios de comunicação como extensões do homem}, lançada em 1964, foi apontado que os veículos informacionais já não desempenhavam apenas um papel de acesso, uma vez que estavam fazendo parte direta do ser humano. Nas palavras de \textcite[p. 5]{McLuhan1974}, “o meio é a mensagem”, ou seja, as tecnologias não são neutras e podem apresentar a face da humanidade, que, por sua vez, pode utilizar esses dispositivos com objetivos variados, dentre eles, os destrutivos.

Nessa perspectiva, as discussões apresentadas neste texto levam em consideração que as tecnologias, em suas várias dimensões, podem influenciar as ações humanas, fazendo, assim, com que a sociedade seja impactada em diversos níveis. Este artigo\footnote{Este artigo foi desenvolvido a partir de \textcite{SantosJefferson2021,SantosBoaSorte2024}. Apresentamos e discutimos, nesse sentido, os temas de análise evidenciados na dissertação, ao mesmo tempo que retomamos os conceitos sobre tecnologias e letramentos digitais. Esta pesquisa foi aprovada pelo Comitê de Ética em Pesquisa da Universidade Federal de Sergipe (UFS) por meio do parecer nº 3.680.472 após verificação institucional.} caracteriza-se como um estudo qualitativo crítico, com foco no ensino de língua inglesa, que visa problematizar os conceitos sobre tecnologia na sociedade contemporânea à luz das teorias dos letramentos digitais. Para tanto, estabelecemos uma discussão sobre os letramentos digitais na contemporaneidade, analisamos processos de hibridização de tecnologias e examinamos os efeitos da internet na sociedade por meio dos dados gerados em campo.

Levando em consideração as temáticas elencadas acima, este artigo possui os objetivos de apresentar e discutir um panorama dos dados e dos temas gerados a partir da investigação dos usos de tecnologias em práticas de ensino de dois professores de língua inglesa em dois \textit{campi} de um Instituto Federal de Educação, Ciência e Tecnologia localizados na Região Nordeste do Brasil. O problema de pesquisa pondera de que maneira os letramentos digitais manifestam-se nas práticas docentes dos contextos investigados, e de que forma essas práticas contribuem para a construção de sentidos sobre o uso da tecnologia no ensino de língua inglesa. Para tanto, realizamos um recorte que possibilita um olhar panorâmico dos elementos explorados na dissertação \cite{SantosJefferson2021}, produto que serviu de base para a escrita deste artigo. Nesse sentido, este trabalho lança enfoque nos temas de análise que emergiram dos dados.




\section{Conceitos sobre tecnologias e letramentos digitais}\label{sec-normas}
O termo tecnologia tem sido restringido, ao menos no senso comum, à noção de dispositivos digitais e eletrônicos. Essa restrição acaba por encobrir o conceito da palavra, que engloba recursos criados ou adaptados pelo ser humano a fim de mediar práticas sociais e ampliar habilidades sensoriais, musculares e cognitivas \cite{Levy2010}. Nesse sentido, um aspecto flagrante, especialmente na contemporaneidade, é a hibridização de tecnologias, uma vez que o desenvolvimento de novas tecnologias não significa o abandono ou a extinção das anteriores.

Essa percepção ampliada sobre o conceito de tecnologia é explorada, neste artigo, com base na acepção das teorias dos letramentos digitais, que percebem os usos sociais de dispositivos de suportes diferentes como potencializadores da construção de linguagens e hábitos. Em \textcite{SantosJefferson2021}, diante disso, consta a seguinte definição para os letramentos digitais:

\begin{quote}
    Letramentos digitais podem ser entendidos como práticas sociais de leitura, reflexão, análise crítica e construção de sentidos por parte daqueles que, direta ou indiretamente, sofrem as influências do universo digital e dos usos das diversas tecnologias. Em linhas gerais, esses letramentos representam uma tomada de consciência em relação ao universo digital e às tecnologias em geral \cite[p. 79]{SantosJefferson2021}.
\end{quote}

Por mais que pareçam permear apenas o campo da reflexão, os letramentos digitais são efetivados justamente por meio da tomada de decisões a partir dos contextos em que vivemos e das identidades que exercemos na sociedade \cite{gallagher2023hidden_curricula}. Dessa forma, a noção de construção de sentidos e de tomada de consciência apresentada na citação anterior está baseada no entrelace entre reflexão e ação, ou seja, uma consciência viva \cite{cope2024multimodal_ai}. Portanto, os letramentos digitais podem ser entendidos como práticas sociais que envolvem os usos de dispositivos digitais, mas, também, como a hibridização entre essas tecnologias e aquelas de suportes analógicos e eletrônicos.

As perspectivas iniciais sobre letramentos digitais, ainda pautadas em uma acepção singular, costumavam focar nesses letramentos como práticas predominantemente ligadas ao software e ao hardware \cite{LankshearKnobel2015}. No contexto inicial da internet global, da ampliação do acesso à televisão e das comunidades de jogos eletrônicos, o então letramento digital estava fortemente ligado ao domínio técnico de um dispositivo ou sistema digital. Desse modo, aqueles que entendiam sobre o funcionamento de jogos, os códigos computacionais e os sistemas da internet, por exemplo, eram entendidos como letrados digitalmente.

Com o passar dos anos e com o acesso cada vez mais ampliado a uma gama de recursos digitais, porém, a demanda por reflexão sobre os usos de tecnologias digitais se intensificou. Esse é um elemento recorrente na história da humanidade, uma vez que o desenvolvimento e a inclusão de recursos emergentes fazem surgir novas atitudes e práticas na sociedade, o que demanda pensamento crítico sobre tais elementos \cite{SantanaSantanaNascimento2023}. Nesse sentido, as práticas de letramentos digitais, como mencionado anteriormente, têm sido voltadas à hibridização de tecnologias, ao mesmo tempo que atentam para os efeitos inéditos e diferentes da inclusão de cada novo dispositivo ou sistema digital na sociedade.

As mudanças no acesso e no funcionamento da internet global são demonstrações de como as práticas sociais se moldam a partir de recursos emergentes. Na fase inicial da internet, o acesso era restrito, custoso e prioritariamente de banco de dados. Na fase seguinte, conhecida como web 2.0, percebemos o surgimento de um caráter comunicativo ampliado, especialmente por conta da criação de mídias sociais bem próximas daquelas em voga atualmente. Nesta fase em que estamos, uma mistura já diluída entre web 3.0 e 4.0, temos um fenômeno semântico latente, de ação popular, mas fortemente impulsionada por algoritmos \cite{cope2021ai_education}. Cada vez mais a internet tende a personalizar o acesso, e novos conflitos de construção de sentidos e tomada de decisões emergem dia após dia.

Nesse sentido, a inclusão de recursos e dispositivos potencializa o desenvolvimento ou a adaptação de hábitos sociais. Os letramentos digitais, então, sugerem e demandam a percepção de práticas sociais. É seguindo essa trilha que retomamos as discussões sobre práticas de letramentos digitais realizadas durante a pesquisa. Interpretamos que escrever sobre uma pesquisa realizada há algum tempo é um exercício interessante para que possamos destacar percepções que, talvez, não tenham sido tão bem expressas. Em \textcite{SantosJefferson2021}, são apresentadas duas categorias latentes sobre práticas de letramentos digitais, a saber:

\begin{quote}
   (I) práticas de letramentos digitais que se ocupam da análise da organização, tanto em esfera estrutural quanto de lógica de navegação, de ambientes digitais. Essas práticas consistem em identificar como o design de ambientes digitais, a exemplo das mídias sociais, direcionam a produção de conteúdo e a interação dos usuários, o que, no fim das contas, proporciona a construção de sentidos variados; (II) práticas de letramentos digitais essencialmente aplicadas, mas que demandam a noção acerca dos contextos de produção e atuação, a exemplo da produção de memes, fanfiction, remixes e a atuação em videogames \cite[p. 84]{SantosJefferson2021}.
\end{quote}

Percebemos, com a maturidade atual e após um processo de desestabilização dos sentidos então construídos, que tal demarcação pode ilustrar que prática e reflexão não andam juntas. Mesmo propondo uma discussão que simbolizava o entrelace entre prática e reflexão sobre as linguagens por meio de dispositivos e ambientes digitais, percebemos que essa imbricação pode não ter sido evidente para todos os leitores \cite{KalantzisCope2020,SelwynEtAl2023}. Desse modo, aproveitamos este espaço para destacar que as categorias de práticas de letramentos digitais apresentadas na citação anterior são indissociáveis, uma vez que ambas acontecem em formato de rizoma, ou seja, em um imbricamento. 

\section{Trajetos metodológicos}\label{sec-conduta}
Este é um estudo qualitativo crítico com foco no ensino de língua inglesa pelo prisma dos estudos de letramentos \cite{KalantzisCope2020,MonteMor2013}. Acolhemos a noção de pesquisa qualitativa, neste trabalho, em uma perspectiva pós-estruturalista de que não precisamos nos basear em aspectos quantitativos para comprovar o rigor científico. Outrossim, este estudo apresenta, como defendido por \textcite{galeffi2009rigor_pesquisa}, um rigor outro, ou seja, um rigor próprio das pesquisas em ciências humanas pós-estruturalistas. Desse modo, os objetivos não direcionam ao estabelecimento de hipóteses, visto que o foco é não cercear os elementos apresentados pelo campo de pesquisa.
Os dados foram gerados por meio de observações estruturadas e entrevistas narrativas com dois professores de língua inglesa de dois \textit{campi} de um Instituto Federal de Educação, Ciência e Tecnologia da Região Nordeste (IFECTRN). Por questões éticas, nomeamos os professores de Vanusa e Djavan, em referência a dois intérpretes da música brasileira. Como forma de garantir o anonimato das instituições, optamos por denominar os \textit{campi} investigados de Paralelas e Pétala, canções gravadas pelos mesmos intérpretes.

Como mencionado, a pesquisa de campo ocorreu por meio de observação estruturada de aulas e de entrevista narrativa com os professores. A observação estruturada, com base em \textcite{LavilleDionne1999}, serviu para criar um desenho de percepção das práticas de ensino e aprendizagem vivenciadas. Sendo assim, o esquema elaborado para observar as aulas continha espaços para anotação de elementos que ajudariam na fase de organização e análise dos dados, como por exemplo: a turma, o curso, a temática da aula, as perspectivas de ensino adotadas, dispositivos utilizados e outros aspectos que fossem relevantes aos contextos investigados.

A entrevista narrativa, baseada em \textcite{jovchelovitch2002entrevista_narrativa}, por outro lado, foi adotada a fim de criar um cenário de conforto para os docentes relatarem suas experiências de vida e de ensino em relação aos usos de tecnologias. Além disso, o número reduzido de perguntas e a forma como foram redigidas a fim de convidar a narração livre fizeram da entrevista um momento leve e produtivo. Os professores, inclusive, lançaram questionamentos e convites ao diálogo, especialmente por demonstrarem boa recepção ao modo como as perguntas foram elaboradas.  

Após o trabalho de campo, os dados foram organizados e analisados por meio dos mecanismos de codificação de \textcite{auerbach2003qualitative} e da metodologia de análise de \textcite{freeman1998teacher_research}. Em um primeiro momento, os textos brutos das transcrições foram nomeados a partir de uma leitura dinâmica, sendo que fragmentos considerados relevantes foram destacados com marcador de texto. Em seguida, as falas recorrentes em cada transcrição foram agrupadas em temas que possuíam relação com as teorias discutidas ao longo da pesquisa. Por fim, foram estabelecidas relações entre os temas e os dados foram exibidos em formato de mapas, que podem ser sintetizados nos seguintes termos: crítica, tecnologias, hibridização e linguagens.

A metodologia de organização e análise dos dados empregada nesta pesquisa é resultado de diálogo constante em reuniões do Grupo de Estudos e Pesquisas em Tecnologias, Educação e Linguística Aplicada da Universidade Federal de Sergipe (TECLA/UFS). Desse modo, assim como na pesquisa reportada neste artigo, outras investigações, a exemplo de \textcite{SantosAllessandra2020,SilvaLaila2021,barros2022tecnologias,boasorte2023entrelace}, fazem releituras da Pesquisa Docente e dos Mecanismos de Codificação por prismas consistentes teoricamente e, ao mesmo tempo, subjetivos por serem realizados por pesquisadores diferentes.


\section{Práticas docentes, tecnologias e letramentos digitais}\label{sec-fmt-manuscrito}
Na primeira fase da pesquisa de campo, as observações de aulas ocorreram em turmas variadas, ministradas pelos dois professores participantes. Consideramos relevante observar aulas, com base em \textcite{freeman1998teacher_research,LavilleDionne1999}, pois tal atividade nos permitiu explorar o campo por outras lentes além daquelas trazidas pelas entrevistas. Todas as turmas fazem parte do ensino médio integrado, o que significa que os estudantes realizam o ensino médio atrelado a uma formação técnica. Dentre os cursos ofertados no Campus Paralelas, no qual atua a professora Vanusa, estão: Agroindústria, Agropecuária, Aquicultura, e Manutenção e Suporte em Informática. No Campus Pétala, onde atua o professor Djavan, são ofertados os cursos de Edificações, Eletrônica e Rede de Computadores.  

Tanto a professora Vanusa quanto o professor Djavan possuem estudos em nível de pós-graduação lato sensu e stricto sensu na área de língua inglesa. Ambos atuaram na educação básica regular antes de ingressarem na carreira de professor da rede federal de educação profissional e tecnológica. Em relação aos dois \textit{campi}, a estrutura, como ambos relatam em trechos das entrevistas mais à frente, é equipada com ar-condicionado, salas com mobília satisfatória para o conforto dos estudantes e dispositivos digitais para a implementação de práticas de ensino. Como argumenta \textcite{SilvaLaila2021}, a estrutura tecnológica dos \textit{campi} da rede federal de educação profissional justifica-se, dentre outros fatores, pelos cursos ofertados e pela prática constante de pesquisa em áreas que demandam aparato tecnológico. Desse modo, a partir das observações e dos relatos dos professores, consideramos o campo investigado como potencializador de práticas de letramentos digitais \cite{LankshearKnobel2015}.

De forma panorâmica, a observação de aula direciona a nossa atenção à conexão entre práticas digitais e não digitais dos professores. Os docentes fizeram uso de recursos analógicos e digitais em praticamente todas as aulas, sendo que a professora Vanusa utiliza dispositivos eletrônicos, a exemplo das caixas de som, com ênfase na prática de habilidades linguísticas clássicas, como a escuta do idioma. O professor Djavan, por outro lado, tende a usar os dispositivos eletrônicos e digitais principalmente para instigar os estudantes a reportarem os sentidos construídos sobre imagens. Destacamos, com base em \textcite{janks2010reading_critically}, que ambos buscavam basear o ensino dos tópicos gramaticais a partir de uma perspectiva semântica em que as estruturas linguísticas não são neutras e exercem significados diferentes em contextos sociais variados.

Após a observação de algumas semanas de aulas nos dois \textit{campi}, foi necessário encerrar a primeira fase da pesquisa de campo em face da paralisação das atividades educacionais presenciais no Brasil e em várias partes do mundo, em meados de março de 2020, em virtude da pandemia da Covid-19. \textcite{SantosJefferson2021,SilvaLaila2021} argumentaram que a paralisação das atividades educacionais desvelou realidades sociais extremas, uma vez que muitos alunos e professores não tinham estrutura adequada para a implementação de aulas remotas. Em termos de geração de dados, a paralisação não afetou diretamente a fase de observação estruturada, pois, desde dezembro de 2019, as visitas no campo já vinham ocorrendo. Porém, as entrevistas narrativas foram postergadas e, somente quando o ensino remoto foi implantado nos \textit{campi} do instituto investigado, surgiu a possibilidade de realizar as entrevistas por videoconferência.

Diante disso, as entrevistas narrativas com a professora Vanusa e o professor Djavan foram realizadas por meio da plataforma Google Meet, que estava sendo usada oficialmente pelo instituto, e foram gravadas para posterior transcrição e análise. Acolhemos a entrevista narrativa, com base em \textcite{jovchelovitch2002entrevista_narrativa}, por entendermos que aquele momento social desafiador demandava uma perspectiva de entrevista centrada no diálogo e na narração da realidade. Como estávamos em um período de pandemia e ensino remoto, os diálogos das entrevistas fizeram emergir códigos relativos ao cenário em questão, como por exemplo: tecnologias, internet, pandemia, acesso, crítica, dispositivos e conflitos. Esses termos, a priori, estavam bem mais relacionados a um certo desabafo dos professores em face das incertezas do momento de pandemia e, principalmente, à instabilidade gerada pelos diferentes contextos de acesso a tecnologias por parte dos alunos durante o ensino remoto.

Ao longo das entrevistas, porém, os professores reportaram experiências anteriores à pandemia, em que já faziam uso de variados suportes e dispositivos tecnológicos. Esse padrão de narração retoma uma perspectiva de representações sociais sobre tecnologias, como discute \textcite{SantosAllessandra2020}, em que os participantes constantemente leem seus cenários de atuação por meio de suas lentes pessoais, acadêmicas e profissionais. Com isso, códigos mais amplos foram surgindo: Sistema Integrado de Gestão de Atividades Acadêmicas (SIGAA), livro, celular, laboratório, computador, estrutura, música, imagem, comunicação, dentre outros. Esses termos, então, auxiliam a perceber temas emergentes que se entrelaçam em um todo coeso que demonstrava uma perspectiva de hibridização de tecnologias nas práticas educativas da professora Vanusa e do professor Djavan.

Dez temas emergiram das entrevistas narrativas com os dois professores participantes. Conforme os dados foram nomeados e agrupados, percebemos algumas similaridades e divergências em suas falas. Desse modo, as relações de consenso e dissenso geraram temas que, em seguida, foram discutidos à luz das teorias dos letramentos digitais. Na \Cref{fig1}, os dez temas de análise são apresentados. No centro da imagem, figuram os elementos basilares das teorias dos letramentos digitais: crítica, tecnologias, hibridização e linguagens.

\begin{figure}[h!]
    \centering
    \begin{minipage}{0.85\linewidth}
    \includegraphics[width=\linewidth]{Fig1-2.png}
    \caption{Temas de Análise.}
    \label{fig1}
    \source{Adaptação de \textcite{SantosJefferson2021}.}
    \end{minipage}
\end{figure}

Cada tema de análise não é um fator isolado, mas uma maneira de sistematizar os tópicos para discussão com base na metodologia de análise selecionada para explorar os dados gerados. Nos parágrafos a seguir, trazemos recortes de cada tema por meio de trechos de falas dos professores e de uma discussão em conjunto com as teorias abordadas. As falas dos professores, em um panorama, demonstram um aspecto discutido por \textcite{barros2022tecnologias}, em que o uso de tecnologia na educação se tornou uma demanda obrigatória; consequentemente, isso acabou gerando desconforto em face da realidade de atuação, que pode ser precária, ou da preferência de cada profissional por se envolver ou não no uso de tecnologias digitais em sala de aula.  

Alguns sentidos construídos pelos professores acerca do significado da palavra tecnologia nos levaram ao tema \textit{noções sobre tecnologia}. Ao longo das entrevistas, a professora Vanusa e o professor Djavan demonstraram compreender tecnologia por um prisma amplo, ou seja, entendendo que dispositivos originários de diferentes períodos históricos influenciam os processos educacionais. Nas passagens a seguir, os dois professores convergem em pensar que a tecnologia não está restrita ao suporte digital ou eletrônico, ou, ainda, que o fato de uma tecnologia não estar vinculada a uma rede de internet traz um caráter de dispositivo ultrapassado, uma vez que o senso comum tende a focar no atual e extremamente novo.

\begin{quote}
    Professora Vanusa: a tecnologia é relevante. A bola pra mim é uma tecnologia; a partir de uma bola tanta coisa pode ser feita. [...] Então, pra mim, tecnologia não é só o aparelho eletrônico.
    
Professor Djavan: a noção de tecnologia é bem mais ampla do que essa que a gente tem hoje de achar que a tecnologia vai se resumir à internet. [...] Então, não é essa a minha visão de tecnologia. Eu entendo que o quadro é uma tecnologia \cite{interviews2021vanusa_djavan}.
\end{quote}

Os professores trazem a perspectiva de entender tecnologia para além da internet e dos suportes eletrônicos. Tal acepção ecoa em estudos da área da semiótica e da filosofia da tecnologia, que entendem as tecnologias como dispositivos de suportes mecânicos, sensoriais e de inteligência artificial que impactam a humanidade \cite{Levy2010}. Trata-se, portanto, de uma percepção da hibridização das tecnologias na sociedade, o que \textcite{cramer2015postdigital,Striano2019} sinalizam como efeitos do pós-digital. Sendo assim, ao propor que os alunos tenham espaço para refletir sobre as tecnologias por meio de contextualização sócio-histórica em imbricação com os usos que eles fazem, os professores acabam por potencializar práticas de letramentos digitais voltadas à construção de sentidos.

Desse modo, ao acolher os letramentos digitais como práticas sociais, passamos a perceber a relevância de empreender uma atitude sócio-historicamente situada, a fim de que os usos de tecnologias não ocorram por modismos ou por acaso de opinião pessoal negativa ou positiva sobre dispositivos em voga. A educação em língua inglesa que se apropria dos estudos dos letramentos digitais, portanto, busca entender a construção de linguagem como processo reflexivo e crítico, ou seja, é um processo que se abre ao novo para agregar em consonância aos dispositivos já consolidados socialmente. Sobretudo, como apontam \textcite{SelwynEtAl2023}, é pensar que não somos neutros e que as tecnologias propiciam sentidos e ações que, em grande parte, são influenciados pelas atitudes humanas.

As falas dos professores destacam a relevância que as imagens possuem nas aulas de língua inglesa, o que nos levou ao tema \textit{uso de imagens}. Para a professora Vanusa, as imagens são recursos significativos para chamar a atenção e a participação dos estudantes, especialmente quando se trata de auxiliar na compreensão de um texto tipográfico ou no estudo de vocabulário. Para o professor Djavan, porém, as imagens podem tanto funcionar como o ponto de partida de uma aula quanto ser o ponto central da discussão. Desse modo, na acepção do professor Djavan, as imagens são interpretadas como textos independentes e não apenas como andaimes para a codificação de um texto principal.

\begin{quote}
    Professora Vanusa: eu converso muito com os meus alunos, geralmente quando tem um texto, algum tema relacionado à alimentação, aí eu pego aquele momento e faço pre-reading com eles, peço para eles observarem as imagens e vou puxando: gente, o que vocês acham que essa imagem está querendo mostrar?

Professor Djavan: houve situações em que a aula toda ocorreu em cima de uma imagem apenas. Com uma imagem a gente discorreu uma aula inteira. Era uma aula que eu estava explicando aos alunos algumas, digamos assim, algumas técnicas e algumas estratégias que são utilizadas na comunicação social, e a gente estava trabalhando mais em cima da parte discursiva da língua \cite{interviews2021vanusa_djavan}.
\end{quote}

As imagens são elementos recorrentes e essenciais nas culturas digitais \cite{KalantzisCope2020}. Nesse sentido, as práticas de ensino empreendidas pelos participantes contemplam uma perspectiva de acolher a “cifra do presente”. Com isso, geram aprendizagem tanto as práticas em que os textos imagéticos servem de apoio para a decodificação de um texto tipográfico quanto aquelas em que as imagens são os elementos centrais das aulas. Dessa forma, na perspectiva dos letramentos digitais \cite{LankshearKnobel2015}, uma atitude de interação entre suportes é, mais uma vez, latente nas práticas dos participantes.

Ao ler as imagens, portanto, como sinalizam \textcite{KalantzisCope2020}, precisamos questionar o nosso lugar de leitura, a fim de que não fiquemos presos a possíveis sentidos empregados pelo autor original. Na leitura imagética, como os professores mencionaram, estamos lendo as nossas próprias compreensões, pois as percepções sobre qualquer texto, sejam imagéticos, sejam tipográficos, são sempre fruto do nosso lugar de interpretação. O nosso lócus interpretativo, que se insere em um contexto social, histórico e geográfico, não representa necessariamente aquilo que o outro quis representar \cite{janks2010reading_critically}. Portanto, a leitura é fundamentalmente uma atitude subjetiva.

Algumas falas sinalizam as percepções e as atitudes dos professores acerca das tecnologias que utilizam em suas práticas de sala de aula, levando ao tema \textit{tecnologias como recursos didáticos}. Por se tratar de uma seara do conhecimento que, historicamente, acolhe tecnologias emergentes, o ensino de línguas estrangeiras tende a se pautar a partir de uma ênfase no recurso didático. É possível notar, nos comentários da professora Vanusa e do professor Djavan, a presença de elementos clássicos do ensino de línguas, a exemplo da escuta de músicas, exibição de filmes e uso de livros didáticos. Desse modo, recursos como celulares, caixas de som, projetores e computadores foram frequentemente mencionados pelos docentes.

\begin{quote}
    Professora Vanusa: então, a tecnologia aí foi essencial porque eu consegui baixar a música, a minha filha baixou para mim, mas, assim, a tecnologia possibilitou o quê? Que eu baixasse a canção, que eu baixasse o clipe, o videoclipe, que eu tivesse a possibilidade de reservar na escola um computador, levar o projetor, o Datashow, usar a caixinha de som.

Professor Djavan: por exemplo, o celular muitas vezes não é bem-vindo e ponto, sabe? Às vezes a gente nem pensa isso como um recurso e o livro é pensado muitas vezes como um recurso e ponto, e nem sequer é problematizado que às vezes uma aula só com o livro pode se tornar muito maçante, assim como uma aula com o celular pode se tornar uma transferência do que é o livro para o celular \cite{interviews2021vanusa_djavan}.
\end{quote}

Tanto a professora Vanusa quanto o professor Djavan atuaram em cenários educacionais variados antes de serem professores da rede federal técnica. Diante disso, ambos enfatizaram a relevância dos recursos que eles têm à disposição atualmente em suas práticas. A professora Vanusa destacou o quanto agrega utilizar áudios e vídeos em uma aula de língua inglesa, o que se conecta com a perspectiva dos letramentos multimodais, propostos por \textcite{cope2024multimodal_ai}. O professor Djavan mencionou, porém, que os recursos precisam ser usados, em sala, a fim de possibilitar reflexão para os alunos e, sobretudo, ao próprio professor. Ele atentou para a subutilização de dispositivos, visto que é recorrente o uso de recursos digitais apenas como transposição de ações que poderiam ser realizadas em suportes não digitais.

 Plataformas como Twitter, Instagram e TikTok podem ser trabalhadas em sala de aula como ambientes de possibilidade criativa de textos multimodais. Destacamos, porém, que a utilização dessas plataformas não precisa acontecer em um momento separado da aula, mas, especialmente, em meio ao conteúdo programático do componente curricular. Desse modo, a expressão escrita e oral em língua inglesa realizada em sala de aula passa a ser cada vez mais socialmente contextualizada. Acolhemos a perspectiva defendida por Lankshear e Knobel (2015), voltada a uma atitude reflexiva e ativa acerca dos usos de tecnologias dentro e fora de sala de aula. As práticas escolares podem, assim, contemplar os gêneros textuais que os alunos produzem em suas culturas digitais.
 
Os professores relataram como a interação com outras pessoas impacta na relação que eles empreendem com dispositivos tecnológicos, fazendo emergir o tema \textit{influências na noção sobre tecnologias}. Enquanto a professora Vanusa mencionou o quanto as pessoas ao redor solicitam dela uma postura mais aberta ao novo, o professor Djavan citou a relevância dos diálogos com outras pessoas na construção de sentidos sobre o que é tecnologia. Desse modo, ambos destacaram que as posturas adotadas em sala de aula e no cotidiano pessoal em relação aos usos de tecnologias são, em grande parte, resultados das interações interpessoais.

\begin{quote}
    Professora Vanusa: apesar do mundo à minha volta me sugerir que eu evoluísse nesse sentido, eu sempre fui muito resistente porque eu pensava nos... a gente ouve tantos crimes cibernéticos [...]. Com essa história da pandemia, eu me vi obrigada... (risos)... a causar essa revolução na minha vida.

Professor Djavan: uma coisa que eu aprendi com a minha companheira nesses diálogos é que a noção de tecnologia é bem mais ampla do que essa que a gente tem hoje de achar que a tecnologia vai se resumir à internet ou a algum dispositivo que esteja conectado a uma internet das coisas \cite{interviews2021vanusa_djavan}.
\end{quote}

As falas da professora Vanusa enfatizam a necessidade de cuidado e vigilância nas redes, especialmente em face dos crimes comumente relatados por pessoas que utilizam a internet. O professor Djavan, por outro lado, tende a refletir sobre elementos conceituais com mais frequência em suas falas, principalmente para trazer sentidos sobre usos de tecnologias. Esses relatos demonstram, portanto, que os sentidos são construídos a partir dos nossos contextos de vivência e que nossas atitudes frente às tecnologias têm origens sociais, conforme apontam \textcite{LankshearKnobel2015}.

 Em outros momentos das entrevistas, os professores também apontaram o fato de a pandemia e o ensino remoto terem influenciado as posturas que eles tinham em relação ao uso da internet. Em face da então necessidade de distanciamento físico, no início de 2020, principalmente em razão da crescente nos casos de infecção, eles perceberam como o acesso a dispositivos digitais e internet era diverso entre os estudantes. Em muitos casos, os discentes não tinham pacote de dados ou rede roteada suficientes para participar dos encontros síncronos. Em outros casos, a exaustão por não ter um ambiente familiar que favorece a concentração durante as aulas também impactou negativamente a participação e a assiduidade dos estudantes. Nesse sentido, o acesso a dispositivos digitais e à internet tem sido discutido por \textcite{SantosAllessandra2020,SilvaLaila2021,barros2022tecnologias} como um elemento que flagrantemente impacta a educação.
 
Os professores apontaram a relevância de atuarem em uma instituição que dispõe de aparatos digitais e eletrônicos nas salas de aula ou que podem ser facilmente reservados quando precisam implementar alguma prática de ensino, levando ao tema \textit{estrutura e recursos do campus}. Tanto a professora Vanusa quanto o professor Djavan destacaram a importância de recursos do cotidiano, a exemplo de água, iluminação e ventilação, para o andamento adequado das atividades em sala de aula. Ambos também citaram, em momentos das entrevistas, o SIGAA como um facilitador dos registros de atividades, assim como para a postagem de arquivos.

\begin{quote}
    Professora Vanusa: este ano, todas as salas têm ar-condicionado, pelo menos nos prédios mais acima, exceto a Agroindústria, mas é estrutura... Tem setor médico, tem setor odontológico, você está entendendo? Psicóloga... Nem toda escola particular tem isso; então, eu não sei em relação a outras pessoas, mas eu confesso que, para mim, repetindo, dado o meu histórico, dado de onde eu vim, é a escola dos sonhos para qualquer professor.

Professor Djavan: é uma estrutura muito estruturada, não é? (risos)... Apesar da redundância, a gente tem, como eu disse anteriormente, datashows acoplados nas próprias salas que funcionam no sistema “internet das coisas”; então, eu uso um aplicativo no meu computador que já conecta diretamente ao datashow... A gente tem internet no campus, embora muitas vezes falhe, mas ajuda bastante ter internet no campus \cite{interviews2021vanusa_djavan}.
\end{quote}

Os relatos dos professores em relação aos demais recursos tecnológicos que não são necessariamente didáticos nos ajudam a pensar na flagrância da hibridização de tecnologias que vivemos na sociedade. Tem sido discutido, com certa frequência, sobre como a análise dos sentidos construídos acerca das tecnologias se configura como uma atitude sócio-histórica, uma vez que pensar sobre tecnologia é ação social \cite{SantanaSantanaNascimento2023}. Dessa forma, ao apontarem para a relevância do conjunto de tecnologias e de estrutura dos \textit{campi}, os professores demonstram o quão o acesso a bens e serviços é um direito necessário para a efetivação de práticas educacionais pautadas na justiça social.

A estrutura equipada com dispositivos variados e ambientes confortáveis é um elemento comumente apontado pela professora Vanusa, especialmente pelo fato de ela já ter atuado em contextos precários de acesso a recursos. O professor Djavan traz um ponto relevante ao mencionar que a oferta de recursos possibilita que as atividades em sala de aula possam ser mais diversas, assim como permite a cobertura de temas contemporâneos de forma efetiva por meio do acesso à internet. Quanto maior o acesso a bens e serviços nas instituições escolares, portanto, maiores são as chances de implementação de práticas de letramentos digitais.

Os professores relataram como determinadas posturas dos estudantes com as tecnologias digitais podem desfocar dos propósitos estabelecidos para as aulas, o que nos levou ao tema \textit{ética em sala de aula}. Como exemplo, a professora Vanusa mencionou a constante falta de atenção aos conteúdos programáticos por conta do acesso a mídias sociais e jogos em momentos de explicação e discussão em sala. O professor Djavan opinou que costuma ocorrer uma ênfase, por parte dos estudantes, em entender os dispositivos digitais e a internet apenas como entretenimento e, quando tratados como recursos pedagógicos, são limitados à tradução literal de textos em língua inglesa.

\begin{quote}
    Professora Vanusa: então, eles acabam fazendo outra coisa, indo para uma outra tecnologia que aí acaba sendo o celular, porque eles aproveitam que a sala está escura, digamos assim, relativamente escura, baixam a cabeça e vão para uma rede social ou vão jogar um joguinho... Eles saem daquele momento de sala de aula que deveria ser uma integração maior e ele vai se integrar com outras pessoas em outras situações com uma outra tecnologia também.

Professor Djavan: então, tem situações em que os alunos precisam recorrer à tecnologia dentro das aulas e isso acaba se tornando também uma questão mais para uma discussão mais ética, por exemplo, quando eu estou dando aula de inglês instrumental [...] Em vez de dialogar com o texto, já parte para o recurso que traduz o texto \cite{interviews2021vanusa_djavan}.
\end{quote}

Nas passagens selecionadas para representar esse tema, os professores comentam sobre posturas em sala de aula com usos de tecnologias que eles não consideram pautadas em senso de ética. Para a professora Vanusa, a utilização da internet, nos momentos de aula, para fins não didáticos fere os propósitos do ensino. A professora defende, de fato, a justiça social ao considerar que o acesso ao conhecimento pode proporcionar oportunidades de aperfeiçoamento que muitas pessoas não possuem \cite{gallagher2023hidden_curricula}, implicando uma percepção que amplia horizontes e possibilidades. O professor Djavan, por outro lado, enfatizou a importância de saber utilizar os dispositivos como potencializadores do estudo linguístico em vez de subutilizar suas funções apenas para finalizar atividades requeridas.

Os professores destacaram a necessidade de constantemente refletir sobre os usos de tecnologias, levando ao tema \textit{ceticismo em relação a tecnologias e práticas digitais}. A professora Vanusa apontou a ação de perceber os noticiários atuais como forma de evitar posturas que possam afetar negativamente a integridade pessoal e do outro na sociedade. O professor Djavan, por outro lado, lançou destaque ao que ele denomina intelectualidade, ou seja, buscar leituras de mundo em diversas áreas do conhecimento para analisar os usos cotidianos de dispositivos digitais.

\begin{quote}
    Professora Vanusa: esse é o grande perigo da tecnologia: ela permite que a gente faça dez coisas ao mesmo tempo e a gente não consegue realizar uma por inteiro, a gente não consegue se dedicar... Então, em sala de aula, é importante eu acho que de vez em quando.

Professor Djavan: eu não sou o tipo de pessoa que é contra a tecnologia, mas eu também não me vejo como um entusiasta das novas tecnologias ao ponto de pensar, assim, que elas vieram para solucionar tudo. Vieram para abrir muitos caminhos e facilitar muita coisa, mas, ao mesmo tempo, as relações interpessoais ainda são muito determinantes nesse sentido \cite{interviews2021vanusa_djavan}.
\end{quote}

De acordo com os dicionários \textcite{cambridge2023skepticism,MerriamWebsterSkepticism}, a palavra ceticismo pode representar uma postura de dúvida em relação a elementos da vida, sejam eles objetos, doutrinas ou métodos. Adotamos o termo ceticismo como central para nomear esse tema por perceber uma constante inquietação dos professores em balancear benefícios e malefícios dos usos de tecnologias em sala de aula e na vida cotidiana \cite{cope2021ai_education}. Nesse caso, apesar de ambos pensarem em tecnologia em aspecto macro, percebe-se a ênfase nos dispositivos de suporte digital e eletrônico. Nesse sentido, o foco nas tecnologias digitais e na internet deve-se ao fato de serem os elementos que mais nos propõem desafios no tempo presente, o que acaba exigindo, como defendem \textcite{SelwynEtAl2023}, certo ceticismo.

As potencialidades, apresentadas pelos professores, dos dispositivos analógicos e digitais como meios para ter informação e conteúdo levaram ao tema \textit{tecnologias como fontes de informação}. Em suas aulas no Campus Paralelas, a professora Vanusa costuma utilizar dispositivos como computador, projetor, livro didático e caixas de som como veículos para explorar textos em língua inglesa. Com isso, é possível perceber uma postura voltada a entender o uso do dispositivo como intermédio para o estudo de tópicos linguísticos. Em suas aulas no Campus Pétala, o professor Djavan também utiliza recursos digitais para ensinar os conteúdos em língua inglesa, mas o foco de utilização está quase sempre em fazer uma espécie de tempestade de ideias a partir de um vídeo ou uma imagem sem, necessariamente, focar no tópico gramatical de imediato.

\begin{quote}
    Professora Vanusa: levei a turma para o laboratório e eles começaram a pesquisar sobre filatelia. Então, começaram a pesquisar os selos, a história dos selos, muitos ficaram interessados, e alunos, inclusive, na época, chegaram a concorrer a uma, era uma espécie de concurso da agência dos correios.

Professor Djavan: eu trouxe para os alunos estatísticas de pesquisas acadêmicas sobre o conceito de felicidade e como essa ideia de felicidade é assimilada em cada um dos países e quais são os elementos que contribuem para um alto índice de felicidade. Entre eles, está o alto acesso a tecnologias e a informação. Então, era mais nesse sentido \cite{interviews2021vanusa_djavan}.
\end{quote}

Nas falas selecionadas, os professores demonstraram entusiasmo em relação aos atuais cenários de culturas digitais em que os estudantes podem ter acesso a fontes variadas de informação. A professora Vanusa trouxe aspectos que \textcite{LankshearKnobel2015} citam como os primeiros movimentos das práticas de letramentos digitais, a saber, a habilidade de fazer buscas e organizar dados na internet. O professor Djavan, por outro lado, tende a refletir e convidar os estudantes a analisar de quais formas a sociedade tem se apropriado e habitado os ambientes digitais. Desse modo, enquanto a professora Vanusa percebeu o caráter latente da busca por conteúdos relacionados à língua inglesa nas plataformas digitais, o professor Djavan preferiu tratar sobre temas contemporâneos que emergem das culturas digitais.

Os professores apontaram a flagrância no entrelace entre dispositivos de suportes variados na contemporaneidade, fazendo emergir o tema \textit{uso de tecnologias analógicas e digitais}. A professora Vanusa fez uma viagem no tempo para discutir como percebe o problema no imenso descarte tecnológico que vivemos há décadas. Segundo ela, o fato de o universo da tecnologia mudar com frequência faz com que o consumismo cresça e impacte as relações humanas. O professor Djavan, por outro lado, apresentou as tecnologias responsáveis pela iluminação e pela refrigeração das salas de aulas e ambientes escolares como essenciais para a manutenção de boas práticas de ensino.

\begin{quote}
    Professora Vanusa: a palavra certa que você usou foi esta: equilíbrio. A gente não pode ser tão tecnológico, digamos assim, digitalmente falando, e nem pode ser também da idade da pedra! (Risos) Mas tem que misturar.

Professor Djavan: eu tenho que ter uma certa gama de recursividade, e aí assim tanto incluindo essas novas tecnologias como outras tecnologias possíveis. Então, assim, se faltasse energia no campus, eu não teria problema para manter uma aula usando o quadro no improviso [...] A temperatura, por exemplo, a luminosidade, coisas que a gente quando está pensando aula talvez não leve tanto em questão, mas que fazem muita diferença no andamento de uma aula \cite{interviews2021vanusa_djavan}.
\end{quote}

Nas falas selecionadas, os professores apontam a relevância de equilibrar as práticas de ensino a partir do uso de dispositivos de suportes variados. Como mencionado no tema sobre a estrutura dos \textit{campi}, os docentes ressaltaram a importância de aspectos da arquitetura e da engenharia de uma sala de aula como cruciais para a execução de atividades educacionais, elemento também apresentado por \textcite{SilvaLaila2021}. Em determinado momento da entrevista, a professora Vanusa reportou experiências em outra rede de ensino em que ela ministrava aulas em turmas com mais de 40 alunos em salas sem ventilação suficiente. O professor Djavan, ao destacar que não se prende à existência de dispositivos digitais em sala, também mencionou o quão relevantes são os elementos estruturais da instituição.

De forma recorrente, foi apontada a relevância de se manter uma constante postura reflexiva com os usos de tecnologias, levando ao tema \textit{reflexões sobre os impactos das tecnologias na sociedade}. Nesse sentido, os professores indicaram a necessidade de pensar o tempo histórico em que se vive. Nas falas das entrevistas, a professora Vanusa e o professor Djavan relataram a percepção sobre uma ansiedade recorrente percebida em muitos estudantes e professores durante o período de ensino remoto. Além disso, ambos trouxeram à discussão a importância de repensar a expressão nas redes, uma vez que conceitos, a exemplo da noção de felicidade, têm sido revisitados a fim de entender, a partir de referências do presente e do passado, como estamos expressando nossos sentimentos na contemporaneidade.

\begin{quote}
    Professora Vanusa: a tecnologia tem sim seus benefícios, acredito que muito mais, mas, ao mesmo tempo, ela é prejudicial, porque ela controla a nossa vida.

Professor Djavan: essa percepção do espaço digital como uma realidade alternativa do espaço físico ou como uma realidade alternativa, entende? Nem tudo o que a gente está vivenciando dentro do campo virtual é necessariamente condizente com o que acontece fora dele \cite{interviews2021vanusa_djavan}.
\end{quote}

Ao longo de todos os temas, assim como neste especificamente, os professores refletiram sobre as relações que eles estabelecem com as tecnologias dentro e fora de sala de aula. Foi possível inferir que ambos possuem apreço pelo acolhimento de dispositivos digitais em aula, mas, ao mesmo tempo, há uma tentativa de distanciamento para compreender os efeitos desses recursos nas vivências humanas \cite{gallagher2023hidden_curricula,SelwynEtAl2023}. Ambos reportaram, com recorrência, os conflitos das experiências nas mídias sociais, uma vez que nem sempre as representações da vida no universo virtual condizem com os sentimentos vivenciados pelas pessoas em seus contextos sociais presenciais.

Com base nos elementos que emergiram das observações e das entrevistas, apresentamos uma síntese dos dados da pesquisa. Ao tomar um movimento de reagrupamento, em que todos os elementos se entrelaçam, foi exercitada a triangulação entre os dados gerados no campo e as teorias que fundamentaram os objetivos da pesquisa. Nesse sentido, especialmente ao entender que construímos sentidos acerca do campo e dos dados, sempre procuramos um lugar de leitura atenta e reflexão sobre as nossas vivências com os participantes e com os dados, porém, sem nos entendermos como distantes e neutros. Os entrelaces foram constantes entre os dados e as teorias, inclusive na codificação e agrupamento dos temas, com base em \textcite{freeman1998teacher_research,auerbach2003qualitative}. Sendo assim, caracterizam-se como uma contínua tentativa de entender os elementos da pesquisa por um prisma transdisciplinar, conforme defendido por \textcite{galeffi2009rigor_pesquisa}, com foco na discussão dos dados. A \Cref{fig2} traz uma síntese dos aspectos que emergiram da análise dos dados:

\begin{figure}[h!]
    \centering
    \begin{minipage}{0.85\linewidth}
    \includegraphics[width=\linewidth]{Fig2-2.png}
    \caption{Desenho Geral dos Dados.}
    \label{fig2}
    \source{Adaptação de \textcite{SantosJefferson2021}.}
    \end{minipage}
\end{figure}

Os elementos apresentados na \Cref{fig2} representam a prática recorrente de propor uma aprendizagem situada, especialmente em relação aos contextos socioculturais em que a produção de linguagens se insere \cite{janks2010reading_critically}. Desse modo, as práticas de letramentos digitais implementadas nas vivências dos professores e dos alunos sinalizaram o entrelace entre usos de tecnologias sempre vinculados à necessidade de refletir enquanto se exercem práticas digitais \cite{cope2024multimodal_ai}. Como defendido nas bases teóricas dos letramentos digitais, as tecnologias são elementos de ação social que, assim como qualquer outro fator da humanidade, demandam consciência crítica por meio da constante atitude reflexiva pautada na leitura de si e do outro.

\section{Algumas considerações}\label{sec-formato}
Este artigo apresentou uma perspectiva panorâmica dos dados e dos temas gerados por meio da investigação dos usos de tecnologias em práticas de ensino de dois professores de língua inglesa de dois \textit{campi} de um IFECTRN. Em síntese, os temas revelaram que os professores abrem espaço, em suas aulas, para a inclusão e o trabalho com tecnologias analógicas, digitais e eletrônicas. Os usos de tecnologias, em geral, estiveram relacionados ao exercício de habilidades linguísticas, tais como escuta e leitura em língua inglesa, e à abordagem de temáticas sociais contemporâneas.

Concluímos, a partir dos dados apresentados e analisados, que os contextos investigados apresentam elementos flagrantes concernentes à perspectiva da construção de sentidos, elemento basilar das teorias dos letramentos digitais. Por outro lado, tendo em vista os dados gerados e a vivência nos contextos observados, não experienciamos a existência de práticas de letramentos digitais em que os estudantes atuem como construtores de sentidos e linguagens especificamente em contato efetivo com as mídias sociais, em ambientes digitais ou em processos de bricolagem em referência aos elementos da internet, mesmo que em suportes analógicos. Sendo assim, as práticas de letramentos digitais observadas estavam fortemente ligadas ao caráter reflexivo da linguagem. 


\printbibliography\label{sec-bib}
% if the text is not in Portuguese, it might be necessary to use the code below instead to print the correct ABNT abbreviations [s.n.], [s.l.]
%\begin{portuguese}
%\printbibliography[title={Bibliography}]
%\end{portuguese}


%full list: conceptualization,datacuration,formalanalysis,funding,investigation,methodology,projadm,resources,software,supervision,validation,visualization,writing,review
\begin{contributors}[sec-contributors]
\authorcontribution{Jefferson do Carmo Andrade Santos}[conceptualization,datacuration,formalanalysis,investigation,writing,review]
\authorcontribution{Paulo Boa Sorte}[conceptualization,investigation,supervision,writing,review]
\end{contributors}

\begin{dataavailability}
\txtdataavailability{databody} % options: dataavailable, dataonly, databody, datanotav, nodata
\end{dataavailability}

\begin{funding}
Os autores declaram que não há conflito de interesse em relação ao presente artigo. A pesquisa reportada foi financiada, em parte, pela Coordenação de Aperfeiçoamento de Pessoal de Nível Superior (CAPES) por meio de uma bolsa de Mestrado Acadêmico. Agradecemos aos participantes da pesquisa de campo pela disponibilidade e pelo envolvimento nos dados que geraram a dissertação e, consequentemente, este artigo. Agradecemos, ainda, aos integrantes do grupo TECLA e aos membros das bancas de qualificação e defesa, que possibilitaram o amadurecimento da análise empreendida.
\end{funding}


\end{document}

