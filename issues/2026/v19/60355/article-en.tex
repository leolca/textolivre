% !TEX TS-program = XeLaTeX
% use the following command:
% all document files must be coded in UTF-8
\documentclass[english]{textolivre}
% build HTML with: make4ht -e build.lua -c textolivre.cfg -x -u article "fn-in,svg,pic-align"

\journalname{Texto Livre}
\thevolume{19}
%\thenumber{1} % old template
\theyear{2026}
\receiveddate{\DTMdisplaydate{2025}{7}{16}{-1}} % YYYY MM DD
\accepteddate{\DTMdisplaydate{2025}{10}{13}{-1}}
\publisheddate{\DTMdisplaydate{2026}{2}{19}{-1}}
\corrauthor{Jefferson do Carmo Andrade Santos}
\articledoi{10.1590/1983-3652.2026.60355en}
\articleid{60355en}
%\articleid{NNNN} % if the article ID is not the last 5 numbers of its DOI, provide it using \articleid{} commmand 
% list of available sesscions in the journal: articles, dossier, reports, essays, reviews, interviews, editorial
\articlesessionname{articles}
\runningauthor{Santos and Boa Sorte} 
%\editorname{Leonardo Araújo} % old template
\sectioneditorname{Daniervelin Pereira~\orcid{0000-0003-1861-3609}}
\layouteditorname{Leonardo Araujo~\orcid{0000-0003-3884-2177}}

\title{Digital literacies in English language education}
\othertitle{Letramentos digitais na educação em língua inglesa}
% if there is a third language title, add here:
%\othertitle{Artikelvorlage zur Einreichung beim Texto Livre Journal}

\author[1]{Jefferson do Carmo Andrade Santos~\orcid{0000-0002-3299-0948}\thanks{Email: \href{mailto:jeffer.leitor@gmail.com}{jeffer.leitor@gmail.com}}}
\author[2]{Paulo Boa Sorte~\orcid{0000-0002-0785-5998}\thanks{Email: \href{mailto:pauloboasorte@academico.ufs.br}{pauloboasorte@academico.ufs.br}}}
\affil[1]{Universidade Estadual de Feira de Santana, Feira de Santana, Bahia, Brazil.}
\affil[2]{Universidade Federal de Sergipe, São Cristóvão, Sergipe, Brazil.}

\addbibresource{article.bib}
% use biber instead of bibtex
% $ biber article

% used to create dummy text for the template file
\definecolor{dark-gray}{gray}{0.35} % color used to display dummy texts
\usepackage{lipsum}
\SetLipsumParListSurrounders{\colorlet{oldcolor}{.}\color{dark-gray}}{\color{oldcolor}}

% used here only to provide the XeLaTeX and BibTeX logos
\usepackage{hologo}

% if you use multirows in a table, include the multirow package
\usepackage{multirow}

% provides sidewaysfigure environment
\usepackage{rotating}

% CUSTOM EPIGRAPH - BEGIN 
%%% https://tex.stackexchange.com/questions/193178/specific-epigraph-style
\usepackage{epigraph}
\renewcommand\textflush{flushright}
\makeatletter
\newlength\epitextskip
\pretocmd{\@epitext}{\em}{}{}
\apptocmd{\@epitext}{\em}{}{}
\patchcmd{\epigraph}{\@epitext{#1}\\}{\@epitext{#1}\\[\epitextskip]}{}{}
\makeatother
\setlength\epigraphrule{0pt}
\setlength\epitextskip{0.5ex}
\setlength\epigraphwidth{.7\textwidth}
% CUSTOM EPIGRAPH - END

% to use IPA symbols in unicode add
%\usepackage{fontspec}
%\newfontfamily\ipafont{CMU Serif}
%\newcommand{\ipa}[1]{{\ipafont #1}}
% and in the text you may use the \ipa{...} command passing the symbols in unicode

% LANGUAGE - BEGIN
% ARABIC
% for languages that use special fonts, you must provide the typeface that will be used
% \setotherlanguage{arabic}
% \newfontfamily\arabicfont[Script=Arabic]{Amiri}
% \newfontfamily\arabicfontsf[Script=Arabic]{Amiri}
% \newfontfamily\arabicfonttt[Script=Arabic]{Amiri}
%
% in the article, to add arabic text use: \textlang{arabic}{ ... }
%
% RUSSIAN
% for russian text we also need to define fonts with support for Cyrillic script
% \usepackage{fontspec}
% \setotherlanguage{russian}
% \newfontfamily\cyrillicfont{Times New Roman}
% \newfontfamily\cyrillicfontsf{Times New Roman}[Script=Cyrillic]
% \newfontfamily\cyrillicfonttt{Times New Roman}[Script=Cyrillic]
%
% in the text use \begin{russian} ... \end{russian}
% LANGUAGE - END

% EMOJIS - BEGIN
% to use emoticons in your manuscript
% https://stackoverflow.com/questions/190145/how-to-insert-emoticons-in-latex/57076064
% using font Symbola, which has full support
% the font may be downloaded at:
% https://dn-works.com/ufas/
% add to preamble:
% \newfontfamily\Symbola{Symbola}
% in the text use:
% {\Symbola }
% EMOJIS - END

% LABEL REFERENCE TO DESCRIPTIVE LIST - BEGIN
% reference itens in a descriptive list using their labels instead of numbers
% insert the code below in the preambule:
%\makeatletter
%\let\orgdescriptionlabel\descriptionlabel
%\renewcommand*{\descriptionlabel}[1]{%
%  \let\orglabel\label
%  \let\label\@gobble
%  \phantomsection
%  \edef\@currentlabel{#1\unskip}%
%  \let\label\orglabel
%  \orgdescriptionlabel{#1}%
%}
%\makeatother
%
% in your document, use as illustraded here:
%\begin{description}
%  \item[first\label{itm1}] this is only an example;
%  % ...  add more items
%\end{description}
% LABEL REFERENCE TO DESCRIPTIVE LIST - END


% add line numbers for submission
%\usepackage{lineno}
%\linenumbers

\begin{document}
\maketitle

\begin{polyabstract}
\begin{abstract}
This paper aims to display and discuss data from research done through an experience in two campuses of a Federal Institute of Education, Science and Technology in the Northeast of Brazil, based on theories concerning digital literacies, education, and technology. In this study, digital literacies are understood as social practices regarding the use of digital devices as well as hybrid performances with technologies of varied supports, such as analog and electronic devices that have been emerging since the last century. Structured observations and narrative interviews were the instruments adopted for data collection, and the analytical method was based on \textcite{freeman1998teacher_research} along with the coding mechanisms of \textcite{auerbach2003qualitative}. As a result, the emerging topics highlighted the social use of technologies, the relevance of problematizing and thinking about meanings of the term “technology”, as well as the importance of implementing social practices focused on the hybridization of devices instead of on the dissociation of technologies.

\keywords{Education \sep Digital literacies \sep English language \sep Technologies}
\end{abstract}

\begin{portuguese}
\begin{abstract}
Este artigo objetiva apresentar e discutir, com base nas teorias dos letramentos digitais, educação e tecnologia, os dados de uma pesquisa desenvolvida a partir da vivência em dois campi de um Instituto Federal de Educação, Ciência e Tecnologia da Região Nordeste do Brasil. Neste trabalho, letramentos digitais são entendidos como práticas sociais envolvendo usos de dispositivos digitais, mas, ao mesmo tempo, como a hibridização de performances com tecnologias de suportes diversos, a exemplo dos dispositivos analógicos e eletrônicos que têm emergido desde o século passado. As observações estruturadas e as entrevistas narrativas foram os instrumentos de geração de dados, sendo o método de análise baseado em \textcite{freeman1998teacher_research} e nos mecanismos de codificação de \textcite{auerbach2003qualitative}. Os resultados lançaram luz a temas sociais sobre os usos de tecnologias, a necessidade de pensar e problematizar o termo “tecnologia”, assim como a relevância de implementar práticas sociais que tencionam a hibridização de dispositivos em vez da dissociação de tecnologias.

\keywords{Educação \sep Letramentos digitais \sep Língua inglesa \sep Tecnologias}
\end{abstract}
\end{portuguese}
% if there is another abstract, insert it here using the same scheme
\end{polyabstract}

\section{To begin with…}\label{sec-intro}

\epigraph{I watch the advertisement

and I see a soap

in shapes of desire

in shapes of ice cream

I wake up and sleep

on television [...]

And, on the moon,

I am just another sponsored cosmonaut [...]

\cite[our translation]{alencar1969comunicacao}}

The epigraph above brings the song “Comunicação”, by Edson Alencar and Hélio Matheus. It became widely known after being performed by the singer Vanusa in 1969 during the \textit{Festival da Música Popular Brasileira da Rede Record} (Brazilian Popular Music Festival of Record TV). Later, it was also recorded and performed by other singers, such as Elis Regina and Dóris Monteiro. At that time, television was still an emerging means of communication, and it was portrayed across the song as a device that was influencing and dictating society on a daily basis.

In 1969, the world witnessed significant achievements, such as the moon landing, the early stages of Internet projects, and constant advancements in the phonographic industry. In Brazil, especially, the 1960s highlighted a leap in the field of communication, mainly due to the expansion of television networks that had begun in the previous decade, and the establishment of the Brazilian Telecommunication Company (Embratel).

Overall, the social effects of new technological devices that emerged in the 1950s and 1960s have already been analyzed by communication scholars, such as Marshall McLuhan. In \textit{Understanding Media: The Extensions of Man}, first published in 1964, McLuhan pointed out that informational vehicles were no longer merely providing access, but they had become direct extensions of human beings. In \posscite[p. 5]{McLuhan1974} classic quote, “the medium is the message”, as he understands that technologies are not neutral elements and can reflect humans’ intentions, which shows that humans can use such devices for various purposes, including destructive ones.

Similarly, we discuss herein that technologies of varied dimensions and formats can influence human actions, as they diversely impact society. In light of that, this article is a qualitative study focusing on English language teaching, and it problematizes concepts of technology in contemporary society through digital literacy theories. To do so, we consider digital literacies in contemporary times, analyze processes of technology hybridization, and examine the effects of the Internet on society through the results of an empirical study.

In face of that, this article\footnote{This article is based on \textcite{SantosJefferson2021,SantosBoaSorte2024}. We display the analytical themes previously introduced in the thesis to update and discuss concepts of technologies and digital literacies. This research was approved after being institutionally checked by the Federal University of Sergipe Ethical Review Board through the number 3.680.472.} aims to present and discuss an overview of the data collected through the investigation of technology usage in English language teaching practices by two teachers at two campuses of a Federal Institute of Education, Science, and Technology in the Northeast of Brazil. The research problem considers how digital literacies manifest in the teaching practices of the investigated contexts, and how these practices contribute to meaning-making regarding the use of technology in English language teaching. To achieve that, we focused on a panoramic view of the elements explored in the original research \cite{SantosJefferson2021}, which served as the basis for writing this article. Thus, this work displays the research analytical themes by reinterpreting them through updated literature on digital literacies.

\section{Concepts of technologies and digital literacies}\label{sec-normas}
The term technology has been restricted, especially in common sense, to digital and electronic devices. Such a restriction obscures the concept of the term, which should entail a vast scope of devices. Technologies could be defined as resources created or adapted by humans to mediate social practices and enhance sensory, muscular, and cognitive abilities \cite{Levy2010}. Contemporarily, the hybridization of technologies is a striking aspect, as the development of new technologies does not imply the abandonment or extinction of the previous ones.

Such an expanded perception of the concept of technology is explored in this article based on the understanding of digital literacy theories, which perceive the social uses of devices and different supports as a motor for language construction. \textcite{SantosJefferson2021} provides the following definition of digital literacies:

\begin{quote}
    Digital literacies can be understood as social practices of reading, reflection, critical analysis, and meaning-making by those who are directly and indirectly influenced by the digital realm and the uses of diverse technologies. On the whole, such literacies represent a conscious awareness of digital technologies and environments in a macro scope \cite[p. 79, our translation]{SantosJefferson2021}\footnote{“Letramentos digitais podem ser entendidos como práticas sociais de leitura, reflexão, análise crítica e construção de sentidos por parte daqueles que, direta ou indiretamente, sofrem as influências do universo digital e dos usos das diversas tecnologias. Em linhas gerais, esses letramentos representam uma tomada de consciência em relação ao universo digital e às tecnologias em geral” \cite[p. 79]{SantosJefferson2021}.}.
\end{quote}

Even though digital literacy theories may seem to revolve only around reflection, they are effectively accomplished through making decisions based on our contexts and the identities we assume in society \cite{gallagher2023hidden_curricula}. Thus, the concepts of meaning-making and conscious awareness presented in the aforementioned quotation are based on the interweaving of reflection and action, that is, a living consciousness \cite{cope2024multimodal_ai}. Therefore, digital literacies are social practices involving digital devices and the hybridization of such technologies with analog and electronic supports.

The initial perspectives on digital literacies were grounded in a singular understanding and focused on literacies as practices typically encompassing software and hardware \cite{LankshearKnobel2015}. In the early context of the Internet, the expansion of television access, and electronic gaming communities, digital literacy was strongly associated with the technical mastery of digital devices or systems. As a result, those who understood the process of games, computer codes, and Internet systems, for instance, were considered digitally literate.

As time went by, and in the face of increasingly widespread access to a wide range of digital resources, the demand for reflection on the use of digital technologies was intensified. By and large, this is a recurring element in human history, as the development and integration of emerging resources give rise to new attitudes and practices in society, which requires critical thinking about such elements \cite{SantanaSantanaNascimento2023}. In this sense, digital literacy theories have been focusing on the hybridization of technologies while being attentive to the unprecedented and different effects of including each new digital device or system in society.

Changes in Internet access and operation demonstrate how social practices evolve based on emerging resources. In the early stage of the Internet, access was restricted, costly, and primarily limited to databases. Subsequently, we witnessed the emergence of the expanded communicative affordance in Web 2.0, primarily due to the rise of social media platforms that resemble those prevalent today. In our current stage, a diluted mix of Web 3.0 and 4.0, we observe a latent semantic phenomenon shaped by popular action massively driven by algorithms \cite{cope2021ai_education}. Increasingly, the Internet personalizes access, leading to new conflicts in meaning-making and decision-making arising day by day.

In this sense, including resources and devices amplifies the development or adaptation of social practices. Therefore, digital literacy theories suggest and motivate awareness of social practices. In light of that, we summarize discussions on digital literacy practices that underpin the analysis implemented herein. We reinterpret that research report conducted some time ago, which means an exciting exercise of highlighting perceptions that may not have been well expressed. To exercise meaning-making as an evolving phenomenon, we bring two latent categories of digital literacy practices presented in \textcite{SantosJefferson2021}, namely:

\begin{quote}
    (I) Digital literacy practices focus on analyzing the organization of digital environments in terms of structural aspects and navigation logic. These practices involve identifying how the design of digital environments, such as social media platforms, influences content production and user interaction, ultimately leading to varied meaning-making processes. (II) Digital literacy practices are primarily applied but require an understanding of the contexts of production and performance. Some examples include the production of memes, fanfiction, remixes, and engaging in video games \cite[p. 84, our translation]{SantosJefferson2021}.\footnote{“(I) práticas de letramentos digitais que se ocupam da análise da organização, tanto em esfera estrutural quanto de lógica de navegação, de ambientes digitais. Essas práticas consistem em identificar como o design de ambientes digitais, a exemplo das mídias sociais, direcionam a produção de conteúdo e a interação dos usuários, o que, no fim das contas, proporciona a construção de sentidos variados; (II) práticas de letramentos digitais essencialmente aplicadas, mas que demandam a noção acerca dos contextos de produção e atuação, a exemplo da produção de memes, fanfiction, remixes e a atuação em videogames” \cite[p. 84]{SantosJefferson2021}.}
\end{quote}

In the face of current maturity and after destabilizing meanings previously made, we now consider that such demarcation may illustrate that practice and reflection do not always go hand in hand. Despite having previously discussed the interweaving between practice and reflection in language through digital devices and environments, we could have made such an interrelation more evident to all readers \cite{KalantzisCope2020,SelwynEtAl2023}. Therefore, we take this opportunity to emphasize that the categories of digital literacy practices presented in the quotation are not dichotomies, as both congregate reflection and practice, meaning that they are intricately intertwined.

\section{Methodological paths}\label{sec-conduta}
This paper reports on a critical qualitative study focused on English language education in light of literacy perspectives \cite{KalantzisCope2020,MonteMor2013}. We embrace qualitative research herein from a post-structuralist perspective, where we do not need to rely on quantitative aspects to establish scientific rigor. Furthermore, as advocated by \textcite{galeffi2009rigor_pesquisa}, this study encompasses a different sense of rigor, which is the rigor inherent to post-structuralist research in the humanities. Thus, we do not primarily aim to establish hypotheses, as the focus is on not restricting the elements presented by the research field.

The data were collected through structured classroom observation and narrative interviews with two English language teachers from a Federal Institute of Education, Science, and Technology on two campuses in the Northeast of Brazil. We named the teachers Vanusa and Djavan, in reference to two Brazilian singers, for ethical reasons. To ensure the anonymity of the institutions, we have chosen to name the campuses Paralelas and Pétala, songs recorded by those two singers.

As mentioned, the field research was conducted through structured classroom observation and narrative interviews with the teachers. Based on \textcite{LavilleDionne1999}, structured observation helps create a perceptive framework of the teaching and learning practices experienced. Therefore, the scheme developed for observing the classes contained spaces for taking notes of the elements that would assist in the phase of data organization and analysis. Across the elements, we highlighted factors such as lesson topics, teaching perspectives adopted, devices used, and other aspects relevant to the investigated contexts.

On the other hand, the narrative interviews \cite{jovchelovitch2002entrevista_narrativa} were approached to create a comfortable setting for the teachers to share their life and teaching experiences regarding the use of technology. Furthermore, the limited number of questions and the way they were formulated to invite free narration made the interview light and productive. The teachers even posed questions and invited us to dialogue, notably due to a positive response to how the questions were crafted.

After the fieldwork, the data were organized and analyzed using the coding mechanisms of \textcite{auerbach2003qualitative} and \posscite{freeman1998teacher_research} analysis methodology. Initially, the raw text from the transcriptions was labeled through dynamic reading, with relevant fragments being highlighted with text markers. Subsequently, recurring statements in each transcription were grouped into themes related to the theories discussed. Finally, connections were established between such themes, and the data were displayed in maps, mainly referring to critique, technologies, hybridization, and language.

The data organization and analysis methodology employed in the research results from constant dialogue in meetings of the Research and Study Group on Technologies, Education, and Applied Linguistics at the Federal University of Sergipe (TECLA/UFS). Moreover, as in the research reported herein, \textcite{SantosAllessandra2020,SilvaLaila2021,barros2022tecnologias,boasorte2023entrelace} have reinterpreted the Teacher Research and the Coding Mechanisms through theoretically consistent lenses, although each of them has brought forth a subjective perspective.


\section{Teaching practices, technologies, and digital literacies}\label{sec-fmt-manuscrito}
During the first phase of the field research, classroom observations occurred with different groups taught by the two participants. The observation was adopted as we considered it a path to unveiling the field through other lenses beyond the ones brought by the participants in the interview \cite{freeman1998teacher_research,LavilleDionne1999}. Such classes were part of an integrated high school program, meaning that students complete their high school education alongside technical training. The following majors are offered at Campus Paralelas, where teacher Vanusa works: Agribusiness, Agriculture, Aquaculture, and Maintenance and Support in Information Technology. The majors offered at Campus Pétala, where teacher Djavan works, are Civil Engineering, Electronics, and Computer Networking.

Vanusa and Djavan hold lato and stricto sensu postgraduate degrees in English. They also worked in K-12 Education before starting their careers in the federal professional and technological education. Both reported, in the interview excerpts on the infrastructure of the campuses, that all facilities are equipped with air conditioning, furnished and comfortable classrooms, and digital devices for teaching practices. As discussed by \textcite{SilvaLaila2021}, the presence of technological structure on the campuses of the professional educational network is justified by the majors and minors offered as well as the constant practice of research projects that demand cutting-edge technologies. Based on the observations and the participants' reports, we consider the investigated field as a flagrant scenario for digital literacy practices \cite{LankshearKnobel2015}.

By taking a panoramic perspective, classroom observation helped us draw our attention to the connection between digital and non-digital practices. The participants used analog and digital resources in most classes, as Vanusa mostly used electronic devices, such as sound systems, emphasizing classic language skills practice, e.g., language listening. On the other hand, Djavan tended to use electronic and digital devices mainly to encourage students to convey the meanings made from images. As advocated by \textcite{janks2010reading_critically}, it is worth mentioning that both sought to base the teaching of grammatical points on a semantic perspective in which linguistic structures are not neutral, and they provide different meanings in various social contexts.

After observing classes at both campuses for a few weeks, it was necessary to finish the first phase of field research. There was a suspension of in-person educational activities in Brazil and various social spaces of the world in mid-March 2020 due to the COVID-19 pandemic. \textcite{SantosJefferson2021,SilvaLaila2021} argued that the suspension of educational activities unveiled extreme social realities, since many students and teachers did not have the adequate structure at home to implement remote classes. Regarding data collection, that suspension did not directly affect the observation phase, as field visits had been ongoing since December 2019. However, narrative interviews were postponed and conducted via videoconference only when remote teaching was implemented at the campuses.

As a result, narrative interviews with the teachers were conducted through the Google Meet platform, which the institute officially used at that time, and they were recorded for later transcription and analysis. Based on \textcite{jovchelovitch2002entrevista_narrativa}, the narrative interview was adopted as we considered it to provide a meaningful practice where dialogue through the report of the current reality was a central element. Since we lived in a pandemic and remote teaching context, some arguments during the interviews brought forth codes related to that current scenario, such as technologies, Internet, pandemic, access, critique, devices, and conflicts. Initially, those terms were related to the teachers' expressions of uncertainty during the pandemic, especially the instability caused by different levels of access to technology among students during remote education.

Across the interviews, however, both teachers reported experiences before the pandemic where they had already used varied technological supports and devices. This narrative pattern recalls what \textcite{SantosAllessandra2020} discusses as social representations of technologies, since the participants constantly assess their working settings through their personal, professional and academic lenses. As a result, broader codes emerged, including the Integrated System of Management of Academic Activity (SIGAA, in Portuguese), books, cell phones, laboratories, computers, facilities, music, images, and communication. Those codes helped identify emerging themes that intertwined into a cohesive whole, picturing a perspective of technology hybridization in the educational practices.

Ten emerging themes arose from the narrative interviews with the participants. As data were named and grouped, some similarities and differences in the participants’ statements were apparent. Thus, consensus and dissenting relationships generated themes discussed in light of digital literacy theories. \Cref{fig1} showcases ten themes as the foundational elements of digital literacy theories, which are set at the center: critique, technologies, hybridization, and languages.

\begin{figure}[h!]
    \centering
    \includegraphics[width=0.75\linewidth]{Fig1.png}
    \caption{Analytical Themes.}
    \label{fig1}
    \source{Adapted from \textcite{SantosJefferson2021}.}
\end{figure}

The themes above are not taken as isolated elements. However, we consider \Cref{fig1} as a way to systematize the topics for discussion based on the methodology selected for exploring and dealing with the data. In the following paragraphs, we provide excerpts from the participants' statements that represent aspects of each theme as we discuss them in light of the theories approached. In general, the participants argue, as discussed by \textcite{barros2022tecnologias}, that the insertion of technology in education turned into a mandatory demand; consequently, this ended up bringing discomfort in face of the social reality, sometimes precarious, and the teachers’ personal preference as to using or not using digital technologies in their practice.

Some of the definitions of technology mentioned by the teachers led to the theme of \textit{technology awareness}. Throughout the interviews, Vanusa and Djavan demonstrated a broad understanding of technology when they mentioned that devices from different historical periods influence educational processes. In the following excerpts, both converge in their belief that technology is not limited to digital or electronic support, and they also assert that the absence of Internet connectivity does not render a technology outdated, as common perception tends to focus on what is up-to-date and highly new.

\begin{quote}
    Vanusa: Technology is relevant. Even a ball is technology; so many things can be done by using a ball. [...] For me, technology is not only an electronic device.
    
Djavan: The notion of technology is much broader than we think today, as technology is usually limited to the Internet. [...] That is not my view of technology. A blackboard is a technology \cite[our translation]{interviews2021vanusa_djavan}.\footnote{“Professora Vanusa: a tecnologia é relevante. A bola pra mim é uma tecnologia; a partir de uma bola tanta coisa pode ser feita. [...] Então, pra mim, tecnologia não é só o aparelho eletrônico.

Professor Djavan: a noção de tecnologia é bem mais ampla do que essa que a gente tem hoje de achar que a tecnologia vai se resumir à internet. [...] Então, não é essa a minha visão de tecnologia. Eu entendo que o quadro é uma tecnologia” \cite{interviews2021vanusa_djavan}.}
\end{quote}

The participants brought forth the perspective of understanding technology beyond the Internet and electronic supports. This understanding relates to Semiotics and the Philosophy of Technology, where technologies are portrayed as devices encompassing mechanical, sensory, and artificial intelligence frames that impact humanity \cite{Levy2010}. Therefore, this is a perception of the technology hybridization in society, which \textcite{cramer2015postdigital} and \textcite{Striano2019} point out as an effect of the post-digital era. Thus, both participants enhance digital literacy practices aimed at meaning-making by proposing that students have the space to reflect on technologies through socio-historical contextualization intertwined with their social background.

Consequently, by embracing digital literacies as social practices, we realize the relevance of adopting a socio-historically situated attitude so that the use of technology does not occur based on trends, by chance, or only as a result of personal opinions, whether positive or negative, about up-to-date devices. English language education incorporating digital literacy studies aims to understand language experience as a critical and reflective process. It is a process that remains open to the new in order to aggregate in consonance with the socially established devices. As \textcite{SelwynEtAl2023} advocate, it involves recognizing that we are not neutral and that technologies provide meanings and actions primarily influenced by human attitudes.

The participants' statements on the relevance of images in English language classes led to the theme of \textit{using images}. For Vanusa, images are meaningful resources for capturing students' attention and participation, especially when it comes to assisting in the comprehension of typographic texts and vocabulary study. On the other hand, Djavan asserts that images can serve as both the starting point of a lesson and the focus of discussion. He uses images to be interpreted as independent texts, and not only as scaffolds for encoding a typographic text.

\begin{quote}
    Vanusa: I talk a lot with my students; usually, when there is a text or some topic related to food, I take that moment and do pre-reading with them. I ask them to look at the images and ask: folks, what do you think this image is trying to convey?
    
Djavan: There were situations where the lesson was based on just one image. By using one image, we discussed an entire lesson. It was a lesson where I was explaining to the students some, you know, techniques and strategies used in social communication, and we were working more on the discursive part of the language \cite[our translation]{interviews2021vanusa_djavan}.\footnote{“Professora Vanusa: eu converso muito com os meus alunos, geralmente quando tem um texto, algum tema relacionado à alimentação, aí eu pego aquele momento e faço pre-reading com eles, peço para eles observarem as imagens e vou puxando: gente, o que vocês acham que essa imagem está querendo mostrar?

Professor Djavan: houve situações em que a aula toda ocorreu em cima de uma imagem apenas. Com uma imagem a gente discorreu uma aula inteira. Era uma aula que eu estava explicando aos alunos algumas, digamos assim, algumas técnicas e algumas estratégias que são utilizadas na comunicação social, e a gente estava trabalhando mais em cima da parte discursiva da língua” \cite{interviews2021vanusa_djavan}.}
\end{quote}

Images are recurrent and essential elements in digital cultures \cite{cope2024multimodal_ai}. In this sense, the teaching practices employed by the participants encompass a perspective of embracing the code of contemporary times. Consequently, they generate learning both through practices in which imagery serves as support for decoding a typographical text, and practices that make images the central elements of the lessons. Thus, in digital literacies \cite{LankshearKnobel2015}, an attitude of interaction between media is once again evident in the participants' practices.

When reading images, as suggested by \textcite{KalantzisCope2020}, we need to inquire about our position as readers so as not to be confined to possible meanings intended by the original author. As a consequence, we are reading our understandings and perceptions of any text. When reading texts, whether visual or typographic, we are always a product of our interpretive position. Our interpretive locus is situated within social, historical, and geographical contexts, and it does not necessarily represent what the other intended to convey \cite{janks2010reading_critically}. Therefore, reading is fundamentally a subjective act.

The statements about the perceptions and attitudes of both participants regarding the technologies they use in their classroom practices led to the theme of \textit{technologies as learning resources}. As a field of knowledge that has historically embraced emerging technologies, foreign language teaching tends to focus on instructional resources. Considering Vanusa and Djavan's assertions, we can observe classic elements of language teaching, such as listening to music, watching movies, and using textbooks. Therefore, they frequently mentioned resources like cell phones, speakers, projectors, and computers.

\begin{quote}
    Vanusa: So, technology here was essential because I could download the song; my daughter downloaded it for me, but, you know, technology enabled what? I could download the song and the video clip, get a computer at school, bring the projector and the data show, and use the soundbox.
    
Djavan: For example, cell phones are often not welcome, and that is it, you know? Sometimes, we do not even think of it as a resource, and the book is often thought of as a resource, and that is it. Moreover, it is not even questioned that sometimes a lesson with just the book can become very dull, just as a lesson with the mobile phone can become a transfer of what the book is to the phone \cite[our translation]{interviews2021vanusa_djavan}.\footnote{“Professora Vanusa: então, a tecnologia aí foi essencial porque eu consegui baixar a música, a minha filha baixou para mim, mas, assim, a tecnologia possibilitou o quê? Que eu baixasse a canção, que eu baixasse o clipe, o videoclipe, que eu tivesse a possibilidade de reservar na escola um computador, levar o projetor, o Datashow, usar a caixinha de som.

Professor Djavan: por exemplo, o celular muitas vezes não é bem-vindo e ponto, sabe? Às vezes a gente nem pensa isso como um recurso e o livro é pensado muitas vezes como um recurso e ponto, e nem sequer é problematizado que às vezes uma aula só com o livro pode se tornar muito maçante, assim como uma aula com o celular pode se tornar uma transferência do que é o livro para o celular” \cite{interviews2021vanusa_djavan}.}
\end{quote}

Both teachers worked in other educational settings before being part of the federal technical educational system. Therefore, they emphasized the relevance of the resources available for their practices. Vanusa highlighted how beneficial it is to use audio and video in an English language class, which aligns with the perspective of multimodal literacy, as proposed by \textcite{cope2024multimodal_ai}. Conversely, Djavan mentioned that resources must be used in the classroom to enable reflection for students and teachers. He pointed out the underutilization of devices, as he considers that digital resources are usually used as a mere transposition of actions that could be performed with non-digital supports.

In face of that, we bring forth the perspective of \textcite{LankshearKnobel2015}, who advocate for a reflective and active attitude toward using technology inside and outside the classroom. School practices can encompass the textual genres that students produce in their digital cultures. Platforms such as Twitter, Instagram, and TikTok can be used in the classroom as environments for creatively producing multimodal texts. However, it is worth pondering that using those platforms does not have to happen in a separate moment of the lesson; it can be integrated into the curriculum content. Hence, spoken and written English practices become increasingly socially contextualized in the classroom.

The theme of \textit{influence on technology awareness} emerged as the teachers discussed how interactions with other people impact their relationship with technological devices. While Vanusa mentioned how her peers encouraged her to be more open to new technologies, Djavan emphasized the relevance of the dialogue with others in shaping their understanding of what technology is. Both highlighted that the attitudes they adopt in the classroom and personal contexts, especially regarding the use of technology, are primarily influenced by interpersonal interactions.

\begin{quote}
    Vanusa: Despite the world suggesting that I should evolve in this regard, I have always been very resistant because I thought about... since we hear so much about cybercrimes [...]. In the face of this pandemic, I was compelled to... (laugh)... bring about this revolution.
    
Djavan: One thing I learned from my partner in these dialogues is that the notion of technology is much broader than what we tend to think today that technology will be limited to the Internet or some device connected to the Internet of Things \cite[our translation]{interviews2021vanusa_djavan}.\footnote{“Professora Vanusa: apesar do mundo à minha volta me sugerir que eu evoluísse nesse sentido, eu sempre fui muito resistente porque eu pensava nos... a gente ouve tantos crimes cibernéticos [...]. Com essa história da pandemia, eu me vi obrigada... (risos)... a causar essa revolução na minha vida.

Professor Djavan: uma coisa que eu aprendi com a minha companheira nesses diálogos é que a noção de tecnologia é bem mais ampla do que essa que a gente tem hoje de achar que a tecnologia vai se resumir à internet ou a algum dispositivo que esteja conectado a uma internet das coisas” \cite{interviews2021vanusa_djavan}.}
\end{quote}

Vanusa’s statements emphasize the need for caution and vigilance on the Internet, especially in light of commonly reported cybercrimes. On the other hand, Djavan tends to reflect more frequently on conceptual elements in his statements, primarily bringing meaning to the use of technology. Those arguments demonstrate that our meanings are made from our life contexts, as stated by \textcite{LankshearKnobel2015}, since they are socially situated attitudes toward technology.

Across the interviews, the teachers also mentioned how the pandemic and remote teaching influenced their attitudes toward Internet usage. The need for physical distance early in 2020, mainly due to the rising infection rates, made them realize how diverse the access to digital devices and the Internet was among students. Students often needed more data plans or Wi-Fi networks to participate in synchronous meetings. In other cases, exhaustion from not having a home environment conducive to concentration during classes also negatively impacted students’ participation and attendance. Likewise, access to technological devices and the Internet has been discussed by \textcite{SantosAllessandra2020,SilvaLaila2021,barros2022tecnologias} as an element that strongly impacts education.

The teachers pointed out the importance of working in an institution that provides digital and electronic classroom equipment or makes them readily available when they want to implement teaching practices. That led to the theme of \textit{structure and resources at the campuses}. Both Vanusa and Djavan emphasized the relevance of everyday resources such as water, electricity, and ventilation to conduct classroom activities properly. They also mentioned, at crucial interview points, the SIGAA as a facilitator for recording activities and uploading files.

\begin{quote}
    Vanusa: Currently, all classrooms are equipped with air conditioning, at least in the upper buildings, except for Agroindustry, but it is structured… There is a medical department and a dental department. Only some private schools provide a psychologist available, so I do not know about others, but I admit that, for me, just repeating, given my background and where I came from, it is the dream school for any teacher.
    
Djavan: It is a very well-structured facility, isn’t? (laugh)… Despite the redundancy, as I mentioned earlier, we have projectors integrated into the classrooms, which are operated through the ‘Internet of Things’ system. I use an application on my computer that connects directly to the projector… Although it often fails, we have Internet on campus, and it helps a lot to have Internet available on campus \cite[our translation]{interviews2021vanusa_djavan}.\footnote{“Professora Vanusa: este ano, todas as salas têm ar-condicionado, pelo menos nos prédios mais acima, exceto a Agroindústria, mas é estrutura... Tem setor médico, tem setor odontológico, você está entendendo? Psicóloga... Nem toda escola particular tem isso; então, eu não sei em relação a outras pessoas, mas eu confesso que, para mim, repetindo, dado o meu histórico, dado de onde eu vim, é a escola dos sonhos para qualquer professor.

Professor Djavan: é uma estrutura muito estruturada, não é? (risos)... Apesar da redundância, a gente tem, como eu disse anteriormente, datashows acoplados nas próprias salas que funcionam no sistema “internet das coisas”; então, eu uso um aplicativo no meu computador que já conecta diretamente ao datashow... A gente tem internet no campus, embora muitas vezes falhe, mas ajuda bastante ter internet no campus” \cite{interviews2021vanusa_djavan}.}
\end{quote}

The participants' comments regarding other technological resources that are not necessarily instructional help us consider the ubiquity of technology hybridization in our society. It has been frequently discussed how the analysis of meanings made toward technology constitutes a socio-historical attitude, since thinking about technology is a social action \cite{SantanaSantanaNascimento2023}. Therefore, by pointing to the relevance of technology and campus infrastructure, both teachers demonstrate how access to resources and services is necessary for effective educational practices based on social justice.

Vanusa emphasizes the infrastructure with various devices and comfortable environments, primarily because she has previously worked in contexts with limited access to resources. Djavan raises a relevant point that providing resources allows classroom activities to be more diverse and enables adequate coverage of contemporary topics through Internet access. Therefore, the greater the access to materials and services in educational institutions, the greater the chances of implementing digital literacy practices.

As the participants reported how some students' approaches toward digital technologies can deviate from the purposes established for the class, we envisioned the theme of \textit{ethics in the classroom}. In this sense, Vanusa mentioned the constant need for more attention to the curriculum content due to the access to social media and games during explanations and discussions in the classroom. Djavan expressed his opinion that students often emphasize understanding digital devices and the Internet solely as entertainment, and when those devices are taken as educational resources, they are limited to translating texts.

\begin{quote}
    Vanusa: So, they end up doing something else, such as logging into another technology, which happens to be their cell phones, because they take advantage of the fact that the classroom is, let us say, relatively dark, they lower their heads, and access social media or play a game… They leave that classroom moment that should be more integrated and integrate with other people in different situations with another technology.
    
Djavan: So, there are situations where students need to use technology during class, which also becomes more of an ethical discussion. For instance, when I am teaching English for Specific Purposes […] Instead of engaging with the text, they immediately turn to a tool that translates it \cite[our translation]{interviews2021vanusa_djavan}.\footnote{“Professora Vanusa: então, eles acabam fazendo outra coisa, indo para uma outra tecnologia que aí acaba sendo o celular, porque eles aproveitam que a sala está escura, digamos assim, relativamente escura, baixam a cabeça e vão para uma rede social ou vão jogar um joguinho... Eles saem daquele momento de sala de aula que deveria ser uma integração maior e ele vai se integrar com outras pessoas em outras situações com uma outra tecnologia também.

Professor Djavan: então, tem situações em que os alunos precisam recorrer à tecnologia dentro das aulas e isso acaba se tornando também uma questão mais para uma discussão mais ética, por exemplo, quando eu estou dando aula de inglês instrumental [...] Em vez de dialogar com o texto, já parte para o recurso que traduz o texto” \cite{interviews2021vanusa_djavan}.}
\end{quote}

In the excerpts representing this theme, both participants comment on classroom behaviors involving technology use that they do not consider ethically grounded. For Vanusa, using the Internet for non-educational purposes during class time goes against teaching goals. She advocates for social justice by considering that access to knowledge can provide opportunities for improvement that many people do not have \cite{gallagher2023hidden_curricula}. This implies a perception that broadens horizons and possibilities. On the other hand, Djavan emphasizes the importance of knowing how to use devices as enhancers of linguistic study rather than underutilizing their functions to complete required activities.

The participants emphasized the need to reflect on technology use, which led to the theme of \textit{skepticism towards technologies and digital practices}. Vanusa constantly pointed out that staying informed about current news is a way to avoid behaviors that could negatively affect personal integrity in society. On the other hand, Djavan highlighted what he calls intellectualism, which involves seeking a broad range of knowledge in various fields to analyze the everyday use of digital devices.

\begin{quote}
    Vanusa: This is the great danger of technology: it allows us to do ten things at a time, and we cannot complete any of them thoroughly; we cannot dedicate ourselves… So, in the classroom, it is essential occasionally.
    
Djavan: I am not against technology, but I also do not see myself as an enthusiast of new technologies to the point of thinking they came to solve everything… They came to open many paths and make many things more accessible, but at the same time, interpersonal relationships are still very significant in this regard \cite[our translation]{interviews2021vanusa_djavan}.\footnote{“Professora Vanusa: esse é o grande perigo da tecnologia: ela permite que a gente faça dez coisas ao mesmo tempo e a gente não consegue realizar uma por inteiro, a gente não consegue se dedicar... Então, em sala de aula, é importante eu acho que de vez em quando.

Professor Djavan: eu não sou o tipo de pessoa que é contra a tecnologia, mas eu também não me vejo como um entusiasta das novas tecnologias ao ponto de pensar, assim, que elas vieram para solucionar tudo. Vieram para abrir muitos caminhos e facilitar muita coisa, mas, ao mesmo tempo, as relações interpessoais ainda são muito determinantes nesse sentido” \cite{interviews2021vanusa_djavan}.}
\end{quote}

According to the \textcite{cambridge2023skepticism,MerriamWebsterSkepticism} dictionaries, the term skepticism represents an attitude of doubting elements of life, whether they are objects, doctrines, or methods. We adopted the term skepticism as the central label for this theme as we observed a constant concern from the teachers about balancing the perks and drawbacks of using technologies in the classroom and everyday life \cite{cope2021ai_education}. In this case, despite both considering technology on a macro scale, there is an emphasis on digital and electronic devices. In this sense, the focus on digital technologies and the Internet is because they pose the most challenges in the present time, which ultimately requires, as \textcite{SelwynEtAl2023} argue, a certain level of skepticism.

The potential of analog and digital devices to access information and content led to the theme of \textit{technologies as sources of information}. During classes taught at Campus Paralelas, Vanusa often uses devices such as computers, projectors, textbooks, and speakers as tools to explore English language texts. That illustrates an approach focused on understanding the use of these devices as a medium for the study of language aspects. During his classes at Campus Pétala, Djavan also uses digital resources to teach English language content. However, his focus is generally on creating discussions and generating ideas from videos or images, which is not necessarily concentrated on grammar topics.

\begin{quote}
    Vanusa: I took the students to the computer lab, and they started researching philately. So, they began researching stamps and the history of stamps. Many became interested, and some students even competed in a contest promoted by the postal agency at that time.
    
Djavan: I brought statistics from academic research on the concept of happiness, how such an idea of happiness is assimilated in different countries, and what elements contribute to a high level of happiness. Among the reasons is the high access to technology and information. So, it was more in that sense \cite[our translation]{interviews2021vanusa_djavan}.\footnote{“Professora Vanusa: levei a turma para o laboratório e eles começaram a pesquisar sobre filatelia. Então, começaram a pesquisar os selos, a história dos selos, muitos ficaram interessados, e alunos, inclusive, na época, chegaram a concorrer a uma, era uma espécie de concurso da agência dos correios.

Professor Djavan: eu trouxe para os alunos estatísticas de pesquisas acadêmicas sobre o conceito de felicidade e como essa ideia de felicidade é assimilada em cada um dos países e quais são os elementos que contribuem para um alto índice de felicidade. Entre eles, está o alto acesso a tecnologias e a informação. Então, era mais nesse sentido” \cite{interviews2021vanusa_djavan}.}
\end{quote}

In the excerpts above, both participants showed enthusiasm for the current scenarios of digital cultures, in which students have access to various sources of information. Vanusa highlighted aspects that align with the early practices of digital literacy, such as the ability to conduct research and organize data on the Internet, as mentioned by \textcite{LankshearKnobel2015}. On the other hand, Djavan tended to reflect and invite students to analyze how society has appropriated and interacted with digital environments. Thus, while Vanusa observed the latent desire of students to seek content related to the English language on digital platforms, Djavan preferred to explore contemporary themes that emerge from digital cultures.

As the participants pointed out the precise interplay between varied supports in contemporary times, we visualized the theme of \textit{using analog and digital technologies}. Vanusa went back in time to discuss her perception of the massive technological waste we have been experiencing for decades. According to her, as technology changes, it has entirely led to pleased consumerism and impacts on human relationships. On the other hand, Djavan argued that technologies responsible for lighting and cooling classrooms and school environments are essential for maintaining good teaching practices.

\begin{quote}
    Vanusa: The correct word you used was balance. We can’t be too technological, digitally speaking, and we cannot be stuck in the Stone Age either! (Laughs) But we have to mix it up.
    
Djavan: I have a specific range of versatility, including these new and other possible technologies. So, if we run out of power on campus, I would be OK with conducting a class using a blackboard in improvisation […] Things like temperature and lighting, which we might not seriously consider when planning a class, make a big difference in how a class progresses \cite[our translation]{interviews2021vanusa_djavan}.\footnote{“Professora Vanusa: a palavra certa que você usou foi esta: equilíbrio. A gente não pode ser tão tecnológico, digamos assim, digitalmente falando, e nem pode ser também da idade da pedra! (Risos) Mas tem que misturar.

Professor Djavan: eu tenho que ter uma certa gama de recursividade, e aí assim tanto incluindo essas novas tecnologias como outras tecnologias possíveis. Então, assim, se faltasse energia no campus, eu não teria problema para manter uma aula usando o quadro no improviso [...] A temperatura, por exemplo, a luminosidade, coisas que a gente quando está pensando aula talvez não leve tanto em questão, mas que fazem muita diferença no andamento de uma aula” \cite{interviews2021vanusa_djavan}.}
\end{quote}

Both excerpts emphasize the importance of balancing teaching practices using diverse supports. As mentioned in the theme about the infrastructure of the campuses, the participants highlighted the architectural and engineering aspects of a classroom as crucial for educational activities, similar to what \textcite{SilvaLaila2021} discusses. At a certain point in the interview, Vanusa shared experiences from another educational setting where she taught more than 40 students in poorly ventilated rooms. Djavan emphasized that he did not rely solely on digital devices in the classroom and mentioned the importance of structural elements within the institution.

The teachers maintained a constant reflective stance toward using technologies, which led to the theme of \textit{reflection on technology’s impact in society}. In this regard, the participants pointed out the need to consider our current historical time. Vanusa and Djavan shared their perceptions on the recurring anxiety experienced by many students and teachers during remote teaching due to the COVID-19 pandemic. Additionally, both brought up the importance of rethinking expressions on social media, such as the notion of happiness, which has been revisited to understand how we express our feelings in contemporary times, drawing from references from the present and the past.

\begin{quote}
    Vanusa: Technology does indeed have its benefits, but at the same time, it is detrimental because it controls our lives.
    
Djavan: This perception of the digital space as an alternative reality to the physical space or as an alternative reality, you know? Not everything we experience within the virtual realm aligns with what happens outside it \cite[our translation]{interviews2021vanusa_djavan}.\footnote{“Professora Vanusa: a tecnologia tem sim seus benefícios, acredito que muito mais, mas, ao mesmo tempo, ela é prejudicial, porque ela controla a nossa vida.

Professor Djavan: essa percepção do espaço digital como uma realidade alternativa do espaço físico ou como uma realidade alternativa, entende? Nem tudo o que a gente está vivenciando dentro do campo virtual é necessariamente condizente com o que acontece fora dele” \cite{interviews2021vanusa_djavan}.}
\end{quote}

Across the themes, and especially this one, the participants reflected on the relationships they establish with technologies inside and outside the classroom. It was flagrant that both appreciate the integration of digital devices in their teaching. However, they also try to distance themselves from their personal settings to understand the effects of those resources on human experiences as a whole \cite{gallagher2023hidden_curricula,SelwynEtAl2023}. They frequently mentioned the conflicts of experiences on social media, as representations of life in the virtual world do not always align with the feelings experienced by people in their face-to-face contexts.

Based on the elements that emerged from observations and interviews, we present a synthesis of the research data. By rearranging the movement, where all elements intertwine, a triangulation was exercised between the data collected in the field and the theories underpinning the research objectives. Especially considering that we make meanings about the field and the data, we always sought a place of attentive reading and reflection on our experiences with the participants and the data, which was not seen as distant or neutral. There was a constant interplay between data and theories, including in the coding and naming of themes, based on \textcite{freeman1998teacher_research,auerbach2003qualitative}. Such a movement is characterized as a constant attempt to comprehend the research issues through a transdisciplinary prism, as defended by \textcite{galeffi2009rigor_pesquisa}. \Cref{fig2} provides an overview of the emerging elements of data analysis.

\begin{figure}[h!]
    \centering
    \includegraphics[width=0.75\linewidth]{Fig2.png}
    \caption{Data Overview.}
    \label{fig2}
    \source{Adapted from \textcite{SantosJefferson2021}.}
\end{figure}

The elements showcased in \Cref{fig2} represent the recurring practice of both participants in proposing situated learning, especially concerning the sociocultural contexts where language production is embedded \cite{janks2010reading_critically}. After all, the digital literacy practices implemented in the investigated teaching and learning experiences aimed to interweave the use of technologies with the need for reflection while engaging in digital practices \cite{cope2024multimodal_ai}. By taking such an approach, as advocated in the theoretical foundations of digital literacies, technologies are taken as elements of social action. Similar to other social practices, using technology requires a critical stance through a constant reflective attitude based on self-evaluation and social awareness.

\section{Some remarks}\label{sec-formato}
This article presented a panoramic perspective of the data and themes generated by investigating the use of technologies in the teaching practices of two English language teachers from two campuses of a Federal Institute of Education, Science, and Technology in the Northeast of Brazil. The themes revealed that the participants provide classroom moments that include and use analog, digital, and electronic technologies. On the whole, using technologies was related to developing language skills, such as listening and reading in English and addressing contemporary social themes.

Based on the emerging themes, we conclude that the investigated contexts exhibit evident elements related to the perspective of meaning-making, which is a core element in digital literacy theories. On the other hand, considering the experiences observed in the contexts, we did not find evidence of digital literacy practices in which students act as meaning and language creators, particularly concerning social media, digital environments, or processes of bricolage related to Internet elements, even in analog media. Therefore, the digital literacy practices observed were strongly concerned with the reflective nature of language.


\printbibliography\label{sec-bib}
% if the text is not in Portuguese, it might be necessary to use the code below instead to print the correct ABNT abbreviations [s.n.], [s.l.]
%\begin{portuguese}
%\printbibliography[title={Bibliography}]
%\end{portuguese}


%full list: conceptualization,datacuration,formalanalysis,funding,investigation,methodology,projadm,resources,software,supervision,validation,visualization,writing,review
\begin{contributors}[sec-contributors]
\authorcontribution{Jefferson do Carmo Andrade Santos}[conceptualization,datacuration,formalanalysis,investigation,writing,review]
\authorcontribution{Paulo Boa Sorte}[conceptualization,investigation,supervision,writing,review]
\end{contributors}

\begin{dataavailability}
\txtdataavailability{databody} % options: dataavailable, dataonly, databody, datanotav, nodata
\end{dataavailability}

\begin{funding}
The authors attest that there is no conflict of interest regarding this article. The research reported on this article was financed in part by the Coordenação de Aperfeiçoamento de Pessoal de Nível Superior (CAPES) through a Master's Degree scholarship. Special thanks go to the research participants for their commitment and willingness to contribute to the data collection. We also thank our TECLA peers as well as the qualifying and defense committees for their indirect contribution to the analysis implemented herein.
\end{funding}


\end{document}

