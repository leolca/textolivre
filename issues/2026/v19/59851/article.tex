% !TEX TS-program = XeLaTeX
% use the following command:
% all document files must be coded in UTF-8
\documentclass[portuguese]{textolivre}
% build HTML with: make4ht -e build.lua -c textolivre.cfg -x -u article "fn-in,svg,pic-align"

\journalname{Texto Livre}
\thevolume{19}
%\thenumber{1} % old template
\theyear{2026}
\receiveddate{\DTMdisplaydate{2025}{6}{21}{-1}} % YYYY MM DD
\accepteddate{\DTMdisplaydate{2025}{10}{8}{-1}}
\publisheddate{\DTMdisplaydate{2026}{2}{5}{-1}}
\corrauthor{Tacia Rocha}
\articledoi{10.1590/1983-3652.2026.59851}
%\articleid{NNNN} % if the article ID is not the last 5 numbers of its DOI, provide it using \articleid{} commmand 
% list of available sesscions in the journal: articles, dossier, reports, essays, reviews, interviews, editorial
\articlesessionname{articles}
\runningauthor{Rocha e Romualdo} 
%\editorname{Leonardo Araújo} % old template
\sectioneditorname{Daniervelin Pereira~\orcid{0000-0003-1861-3609}}
\layouteditorname{Saula Cecília~\orcid{0009-0006-3069-8480}}

\title{Sociedade do Conhecimento e as TICs na educação: paráfrases do relatório da UNESCO no discurso do \textit{Porvir}}
\othertitle{Knowledge Society and ICTs in education: paraphrases of the UNESCO report in the \textit{Porvir} discourse}
% if there is a third language title, add here:
%\othertitle{Artikelvorlage zur Einreichung beim Texto Livre Journal}

\author[1]{Tacia Rocha~\orcid{0000-0003-1147-0158}\thanks{Email: \href{mailto:tacia.rocha.f@gmail.com}{tacia.rocha.f@gmail.com}}}
\author[2]{Edson Carlos Romualdo~\orcid{0000-0003-0892-7188}\thanks{Email: \href{mailto:ecromualdo@uol.com.br}{ecromualdo@uol.com.br}}}
\affil[1]{Universidade Estadual Paulista, Faculdade de Arquitetura, Artes, Comunicação e Design, Programa de Pós-graduação em Comunicação, Bauru, SP, Brasil.}
\affil[2]{Universidade Estadual de Maringá, Maringá, PR, Brasil.}

\addbibresource{article.bib}
% use biber instead of bibtex
% $ biber article

% used to create dummy text for the template file
\definecolor{dark-gray}{gray}{0.35} % color used to display dummy texts
\usepackage{lipsum}
\SetLipsumParListSurrounders{\colorlet{oldcolor}{.}\color{dark-gray}}{\color{oldcolor}}

% used here only to provide the XeLaTeX and BibTeX logos
\usepackage{hologo}

% if you use multirows in a table, include the multirow package
\usepackage{multirow}

% provides sidewaysfigure environment
\usepackage{rotating}

% CUSTOM EPIGRAPH - BEGIN 
%%% https://tex.stackexchange.com/questions/193178/specific-epigraph-style
\usepackage{epigraph}
\renewcommand\textflush{flushright}
\makeatletter
\newlength\epitextskip
\pretocmd{\@epitext}{\em}{}{}
\apptocmd{\@epitext}{\em}{}{}
\patchcmd{\epigraph}{\@epitext{#1}\\}{\@epitext{#1}\\[\epitextskip]}{}{}
\makeatother
\setlength\epigraphrule{0pt}
\setlength\epitextskip{0.5ex}
\setlength\epigraphwidth{.7\textwidth}
% CUSTOM EPIGRAPH - END

% to use IPA symbols in unicode add
%\usepackage{fontspec}
%\newfontfamily\ipafont{CMU Serif}
%\newcommand{\ipa}[1]{{\ipafont #1}}
% and in the text you may use the \ipa{...} command passing the symbols in unicode

% LANGUAGE - BEGIN
% ARABIC
% for languages that use special fonts, you must provide the typeface that will be used
% \setotherlanguage{arabic}
% \newfontfamily\arabicfont[Script=Arabic]{Amiri}
% \newfontfamily\arabicfontsf[Script=Arabic]{Amiri}
% \newfontfamily\arabicfonttt[Script=Arabic]{Amiri}
%
% in the article, to add arabic text use: \textlang{arabic}{ ... }
%
% RUSSIAN
% for russian text we also need to define fonts with support for Cyrillic script
% \usepackage{fontspec}
% \setotherlanguage{russian}
% \newfontfamily\cyrillicfont{Times New Roman}
% \newfontfamily\cyrillicfontsf{Times New Roman}[Script=Cyrillic]
% \newfontfamily\cyrillicfonttt{Times New Roman}[Script=Cyrillic]
%
% in the text use \begin{russian} ... \end{russian}
% LANGUAGE - END

% EMOJIS - BEGIN
% to use emoticons in your manuscript
% https://stackoverflow.com/questions/190145/how-to-insert-emoticons-in-latex/57076064
% using font Symbola, which has full support
% the font may be downloaded at:
% https://dn-works.com/ufas/
% add to preamble:
% \newfontfamily\Symbola{Symbola}
% in the text use:
% {\Symbola }
% EMOJIS - END

% LABEL REFERENCE TO DESCRIPTIVE LIST - BEGIN
% reference itens in a descriptive list using their labels instead of numbers
% insert the code below in the preambule:
%\makeatletter
%\let\orgdescriptionlabel\descriptionlabel
%\renewcommand*{\descriptionlabel}[1]{%
%  \let\orglabel\label
%  \let\label\@gobble
%  \phantomsection
%  \edef\@currentlabel{#1\unskip}%
%  \let\label\orglabel
%  \orgdescriptionlabel{#1}%
%}
%\makeatother
%
% in your document, use as illustraded here:
%\begin{description}
%  \item[first\label{itm1}] this is only an example;
%  % ...  add more items
%\end{description}
% LABEL REFERENCE TO DESCRIPTIVE LIST - END


% add line numbers for submission
%\usepackage{lineno}
%\linenumbers

\begin{document}
\maketitle

\begin{polyabstract}
\begin{abstract}
O discurso sobre o incremento do aprendizado continuado com a inserção das Tecnologias da Informação e Comunicação (TIC) nos métodos pedagógicos, protagoniza as discussões nesse campo educacional. O objetivo deste artigo é analisar a intertextualidade do discurso oficial da/sobre a sociedade do conhecimento e as TICs na educação, manifestada pela paráfrase linguística presente no vídeo \textit{Especial Tecnologia na educação -- Por que usar tecnologia} \cite{porvir2015a}. A pesquisa, de caráter exploratório e com abordagem qualitativa, ancora-se em uma perspectiva que integra a Linguística Textual e a concepção dialógica da linguagem, examinando a paráfrase como um ato de reformulação e reorientação discursiva. O método mobiliza as técnicas de pesquisa bibliográfica e documental, cotejando o vídeo do \textit{Porvir} com um relatório da Unesco \cite{mansell2015}. Dentre os resultados, verificou-se que o \textit{Porvir} incorpora as vozes da Unesco em um processo dialógico estratégico, reformulando o discurso original para destacar ações concretas alinhadas aos seus interesses institucionais, como a formação de professores e o acesso à tecnologia na educação. O uso de paráfrases (resumidora, explicativa e explicitadora) legitima o conteúdo do \textit{Porvir}. Conclui-se que a paráfrase é um ato discursivo intencional e uma ferramenta poderosa de recontextualização. Contudo, a intertextualidade implícita, ao omitir a autoria da Unesco, desloca a agenda global para a nacional, simplificando-a, o que pode enfraquecer o debate democrático e levar à implementação superficial de políticas públicas.

\keywords{Intertextualidade\sep Tecnologia na educação\sep Porvir\sep Sociedades do conhecimento\sep Educação básica}
\end{abstract}

\begin{english}
\begin{abstract}
The discourse on the increase in continued learning with the insertion of ICTs in pedagogical methods, takes center stage in discussions in this educational field. The objective of this article is to analyze the intertextuality of the official discourse of/about the knowledge society and ICTs in education, manifested through linguistic paraphrase present in the video \textit{Special Technology in education -- Why use technology} \cite{porvir2015a}. The research is classified as exploratory and uses a qualitative approach, anchored in a perspective that integrates Textual Linguistics, which studies language in use and the text as a social construct, and the dialogical conception of language, which views the enunciation as a link in sociocultural communication and the inevitable presence of the other. Paraphrase is thus examined as an act of reformulation and discursive reorientation. The method employs bibliographic research to discuss theories and documentary research, contrasting the \textit{Porvir} video with a Unesco report \cite{mansell2015}. Among the results, it was verified that \textit{Porvir} incorporates Unesco's voices in a strategic dialogical process, reformulating the original discourse (intertext) to highlight concrete actions aligned with its institutional interests, such as teacher training and access to technology in education. The use of paraphrases (summarizing, explanatory, and explicitating) legitimizes \textit{Porvir}'s content. It is concluded that paraphrase is an intentional discursive act and a powerful tool for recontextualization. However, by employing implicit intertextuality, omitting Unesco's authorship, \textit{Porvir} shifts the global agenda to the national one while simplifying it, which may weaken the democratic debate and lead to the superficial implementation of public policies.

\keywords{Intertextuality\sep Technology in education\sep Future\sep Knowledge societies\sep Basic education}
\end{abstract}
\end{english}
% if there is another abstract, insert it here using the same scheme
\end{polyabstract}

\section{Interação social, intertextos, educação e tecnologias: diálogos possíveis}
O discurso sobre o incremento do aprendizado continuado com a inserção das Tecnologias da Informação e Comunicação (TIC) nos métodos pedagógicos, protagoniza as discussões nesse campo educacional. Atualmente, termos como ``era -- ou sociedade -- da informação'' \cite{bell1977, castells1999} e ``sociedade do conhecimento'' \cite{drucker1970} são cunhados para identificar e entender o alcance das mudanças provocadas pela introdução da inteligência artificial e as TIC em nossas vidas. Portanto, o ponto sobre a inovação tecnológica pode ser reduzido para focar imediatamente nos termos ``sociedade da informação'' e ``sociedade do conhecimento''.

Sob tal temática, o objetivo do presente estudo é analisar a intertextualidade do discurso oficial da/sobre a Sociedade do Conhecimento e as TICs na educação, manifestada pela paráfrase linguística presente no vídeo \textit{Especial Tecnologia na educação -- Por que usar tecnologia} \cite{porvir2015a}. Para tanto, os dois objetos colocados em diálogo são: i) o vídeo do \textit{Porvir\footnote{Disponível em: \url{https://porvir.org/sobre-nos/}. Acesso em: 2 jan. 2026.}}, que conta com mais de 347 mil visualizações na plataforma YouTube\footnote{O número de visualizações mencionado foi detectado em 6 de janeiro de 2026.} \cite{rocha2017}\footnote{O mesmo vídeo foi objeto de pesquisa na dissertação defendida no Programa de Pós-graduação em Letras da Universidade Estadual de Maringá (PLE/UEM) por um dos autores do presente artigo \cite{rocha2017}.}, e que foi produzido por esta ``agência de jornalismo e soluções de comunicação, sem fins lucrativos, dedicada a impulsionar inovações na educação'' \cite{porvir2024}; ii) o discurso oficial sobre inovação na educação proveniente do relatório da Unesco (Organização das Nações Unidas para a Educação, a Ciência e a Cultura) -- \textit{Renovando a visão das sociedades do conhecimento para a paz e o desenvolvimento sustentável} \cite{mansell2015} --, que foi elaborado para o Primeiro Encontro de Revisão da Cúpula Mundial sobre a Sociedade da Informação (CMSI), a CMSI+10 em Paris (2013), com foco na transição para o conceito de ``Sociedades do Conhecimento''.

Para atingir o propósito estabelecido, esta pesquisa, de caráter exploratório e com abordagem qualitativa, mobiliza a técnica de pesquisa bibliográfica, na articulação da Linguística Textual (LT) com a concepção dialógica da linguagem, e a técnica documental, para cotejar o vídeo do \textcite{porvir2015a} com relatório da Unesco \cite{mansell2015}. O diálogo teórico-metodológico proposto implica nas premissas de que a interação verbal é a verdadeira substância da língua \cite{volochinov2017} e de que todo texto é polifônico \cite{bakhtin2018}\footnote{Embora o termo polifonia tenha sido utilizado por alguns autores, como \textcite{barros1994}, como sinônimo de dialogismo, é importante notar a distinção no Círculo de Bakhtin. O dialogismo é o princípio constitutivo universal do discurso, enquanto a polifonia é um arranjo estrutural específico (equipolência de vozes) \cite{maciel2016}. Na linguística textual, \textcite{koch2012} consideram que a polifonia é mais ampla que a intertextualidade (no sentido restrito), pois não exige textos efetivamente existentes para sua construção. Portanto, a afirmação se refere a acepção abrangente de \textit{polifonia} para expressar a presença de múltiplas vozes sociais (heterodiscurso) no discurso.}, constituído por várias vozes e mantém diálogo com outros textos (intertextualidade) \cite{koch2011}. Com efeito, os enunciados produzidos acerca da sociedade do conhecimento são tomados como: i) respostas ao ``já dito''; ii) orientados para novas respostas; iii) internamente dialogizados (ponto de partida das múltiplas vozes sociais) \cite{faraco2009}. Assim, considerando que na constituição textual, ao introduzirmos a palavra do outro em nossa fala, inevitavelmente reconstruímos o que foi dito, com nossa compreensão e avaliação, tornando-as bivocais \cite{romualdo2000}, tentamos responder à pergunta: de que modo o vídeo do \textcite{porvir2015a} retoma intertextualmente um relatório oficial da Unesco \cite{mansell2015} e quais efeitos são obtidos com estas operações linguísticas?

Para organizar a discussão e a análise supramencionadas, a estruturamos o texto da seguinte maneira: nesta seção, apresentamos o tema, a perspectiva teórica adotada, o objetivo geral, o problema de pesquisa e o \textit{corpus} de análise. Na \hyperref[sec-dois]{sequência}, discutimos o que é dialogismo, distinguimos texto de discurso e gênero e introduzimos a noção de intertextualidade. Na \hyperref[sec-tres]{terceira seção}, conceituamos paráfrase e detalhamos as três classificações mobilizadas na análise. Na \hyperref[sec-4]{quarta seção}, apresentamos as condições de produção dos textos verbal e audiovisual (o relatório da Unesco e o vídeo do \textit{Porvir}, respectivamente), introduzimos as noções de estudo de texto em vídeo e mobilizamos os conceitos já discutidos sobre paráfrase na análise. Na \hyperref[sec-5]{quinta seção}, tecemos uma análise crítica acerca dos resultados obtidos nas análises. Por fim, na \hyperref[sec-6]{última seção}, retomamos o objetivo principal e os principais resultados do estudo, os limites e riscos do uso estratégico da paráfrase, além de apontarmos caminhos para pesquisas futuras.

\section{Interfaces entre dialogismo e linguística textual}\label{sec-dois}
A análise da paráfrase exige uma perspectiva que integre a dimensão linguística e a social do texto, encontrando na Linguística Aplicada (LA) o campo propício para esse estudo \cite[p. 72]{souza2024}. A LA apoia a visão do sujeito como social e historicamente situado e da língua como objeto social \cite[p. 72]{souza2024}.

\subsection{Convergências: texto, interação e o sujeito sócio-histórico}
A LT e a concepção dialógica da linguagem (dialogismo) convergem na perspectiva sócio-interacionista da linguagem. Ambas defendem que a linguagem é um ``conjunto de práticas sociais e cognitivas historicamente situadas'' \cite[p. 61]{marcuschi2008}, e que o sentido é construído ativamente na interação entre autor, texto e leitor/ouvinte \cite[p. 10]{koch2015}.

Para o Círculo de Bakhtin, a interação verbal é a substância da língua, e a enunciação é um ato bilateral e um elo da comunicação sociocultural, sendo o produto das inter-relações do falante com o ouvinte \cite[p. 205]{volochinov2017}. De forma análoga, a LT adota a perspectiva sociocognitivo-interacionista, definindo o texto como um construto histórico e social \cite[p. 73]{souza2024}. O sentido não está no texto, mas se constrói a partir dele, no curso de uma interação \cite[p. 11]{koch2006}.

Nesse quadro, a intertextualidade emerge como o elo principal entre a LT e o dialogismo. A primeira absorve o princípio dialógico de Bakhtin, utilizando a intertextualidade (em sentido amplo) como uma condição de existência do próprio discurso. \textcite{koch2012} postulam que ``todo texto se constrói como mosaico de citações, todo texto é absorção e transformação de um outro texto''. A paráfrase é, portanto, examinada dentro desse quadro como um ato de reformulação e reorientação discursiva, caracterizando-se como uma ``intertextualidade das semelhanças'' \cite[p. 28]{santanna2003}.


\subsection{Distinções e nuances teóricas}
É crucial diferenciar os conceitos para manter o rigor teórico, especialmente quanto ao escopo do texto/discurso e à polifonia.
\begin{itemize}
    \item a) Texto \textit{vs.} discurso e enunciação: a LT distingue o texto (o fenômeno linguístico empírico, no plano da esquematização) do discurso (a prática linguística codificada, associada a uma prática social, no plano do dizer). O gênero opera como ``a ponte entre o discurso como uma atividade mais universal e o texto, como a peça empírica particularizada e configurada numa determinada composição observável'' \cite[p. 84]{marcuschi2008}. Já o dialogismo foca no enunciado como a unidade real da comunicação, priorizando o tema -- a expressão da situação histórica concreta que origina a enunciação \cite{volochinov2017};
    
    \item b) Dialogismo \textit{vs.} polifonia bakhtiniana: o dialogismo é o princípio constitutivo universal de toda a linguagem, e as ``relações dialógicas'' \cite{maciel2016} se fazem presentes na interação entre quaisquer vozes, independentemente de estarem em conflito ou em consonância \cite[p. 66]{faraco2009}. Em contrapartida, a polifonia bakhtiniana é um arranjo específico dessas relações, usada como metáfora para o romance dostoievskiano, exigindo a equipolência de vozes e um diálogo inconcluso \cite[p. 583]{maciel2016}. A polifonia depende, portanto, da amplitude do diálogo (microdiálogo, diálogo composicionalmente expresso e grande diálogo) \cite[p. 587]{maciel2016}.
\end{itemize}

Apesar dessas distinções, autores como \textcite{koch1997} e \textcite{koch2012} consideram o conceito de polifonia mais amplo que o de intertextualidade (em sentido restrito), visto que não são necessários textos efetivamente existentes para a construção da polifonia.

Mediante esses conceitos, a seguir, discutimos o conceito de paráfrase como um caso de intertextualidade das semelhanças e também nos dedicamos na distinção entre intertextualidade implícita e explícita em função de nosso \textit{corpus} de análise na seção que se segue.

\section{Intertextualidade das semelhanças: a paráfrase}\label{sec-tres}
A intertextualidade pode se manifestar de forma explícita, quando a fonte do intertexto é mencionada (como em citações ou resumos), ou implícita, quando a fonte não é expressa, exigindo que o interlocutor a recupere em sua memória discursiva para a construção de sentidos \cite{koch2004}. A intertextualidade implícita tem seu sentido comprometido se o intertexto não for recuperado, sobretudo quando há subversão de sentidos \cite{koch2004}. Em contrapartida, quando a retomada do texto-fonte segue a mesma orientação argumentativa do intertexto, este fenômeno é classificado como intertextualidade das semelhanças. A paráfrase inclui-se nessa categoria \cite{santanna2003}.

Sob uma perspectiva dialógica, a paráfrase pressupõe uma relação de continuidade de sentidos entre dois enunciados: o parafraseado e o parafraseador. Partindo da concepção de que a enunciação é um ``elo da cadeia dos atos de fala'' \cite{volochinov2017}, o texto original ganha e perde significação, pois o enunciado parafraseador adapta e reorienta os sentidos à nova situação de interação.

Embora o termo paráfrase tenha historicamente apresentado diferentes caracterizações (lógica e gramatical), prevaleceu o \textit{tratamento no âmbito do discurso}, entendido como uma \textit{prática concreta de reformulação}. Nessa perspectiva retórica, a reformulação parafrástica é vista como um ato que repousa sobre uma \textit{interpretação prévia do texto} e envolve o reconhecimento do sentido do texto-fonte, mesmo que de forma provisória, permitindo que a paráfrase oscile ``entre a reprodução pura e simples do conteúdo e a sua deformação'' \cite[p. 134]{fuchs2012}.

Para as análises do fenômeno parafrástico em nosso \textit{corpus}, consideramos as afirmações de \textcite{fuchs2012} apresentadas no parágrafo anterior e elencamos apenas os tipos de paráfrases que ocorrem em nosso \textit{corpus}, tomando por base para esse levantamento o estudo de \textcite{ribeiro2001}:

\begin{itemize}
    \item a) \textbf{Paráfrase resumidora:} destaca, seleciona e apresenta os argumentos que o locutor julga mais relevantes para sustentar a tese que defende \cite[p. 125]{ribeiro2001}. Nos processos de decomposição e recomposição, manifesta-se por meio da \textit{condensação sintático-lexical};

    \item b) \textbf{Paráfrase explicativa:} funciona como um recurso de antecipação que visa a construção da intercompreensão entre os interlocutores. \textcite{hilgert1989} aponta que os processos de decomposição e recomposição se concretizam no texto, respectivamente, por meio de ``expansão e condensação sintático-lexical''. Condensação resume a informação. Já ``expansão parafrástica'' expande, tanto no papel de ``explicação definidora'' de conceitos abstratos quanto na função explicitadora, quando confere maior precisão e detalhamento às informações \cite[p. 129-130]{ribeiro2001};

    \item c) \textbf{Paráfrase explicitadora:} tem a função de definir um termo ou tornar explícito um conteúdo resumido, elaborando enunciados que garantam a eficácia do ``projeto de dizer'' \cite[p. 135-136]{ribeiro2001}.
\end{itemize}

A seleção e o detalhamento desses três tipos de paráfrase foram diretamente determinados pelos dados do nosso \textit{corpus}. Eles serão mobilizados na seção cinco para explicitar como os fragmentos do intertexto da UNESCO foram reformulados no texto parafrástico do \textcite{porvir2015a}, revelando as operações discursivas de condensação, definição e expansão de sentido que se mostraram essenciais para a recontextualização do tema para a audiência do \textit{Porvir}.

Isso posto, na \hyperref[sec-4]{próxima seção} fazemos o necessário preparo para a aplicação dessas ferramentas analíticas, atendendo ao rigor teórico-metodológico. Discutimos o cenário sócio-histórico no qual essa recontextualização discursiva se insere, ao contextualizar o discurso da sociedade do conhecimento (Seção \ref{sec-4-1}), e apresentamos as condições de produção dos textos-fonte (relatório da UNESCO) e parafrástico (vídeo do \textit{Porvir}) (Seção \ref{sec-4-2}), elementos cruciais para a compreensão das operações discursivas analisadas na Seção \ref{sec-5}.

\section{Contextualização discursiva e descrição do \textit{corpus} analítico: UNESCO e \textit{Porvir}}\label{sec-4}
Como já afirmamos, ``o discurso é visto como prática linguística codificada associada a uma prática social'', ``historicamente situada'', esquematizado no fenômeno linguístico empírico, o texto \cite[p. 84]{marcuschi2008}. Esse, por sua vez, é configurado numa determinada composição observável condicionada pelo gênero textual\footnote{A linha teórica do Círculo de Bakhtin adota o termo ``gêneros do discurso'' ou ``gêneros discursivos'' \cite{bakhtin2016} para se referir aos tipos relativamente estáveis de enunciados. Contudo, no presente trabalho, utiliza-se também a terminologia ``gênero textual'' \cite[p. 84]{marcuschi2008} empregada na LT. O uso da noção de \textcite{marcuschi2008} se justifica pela sua importância para a LT, especialmente na distinção entre texto, discurso e gênero, na qual o gênero opera como ``a ponte entre o discurso como uma atividade mais universal e o texto, como a peça empírica particularizada''. A despeito da diferença terminológica, a concepção subjacente do gênero como entidade sócio-histórica e condicionadora da enunciação é convergente e ressoa o postulado dialógico da linguagem.}. Assim, a partir dessa ordem -- discurso dado no plano da enunciação, ``o texto no plano de esquematização'' e o gênero como ``aquele que condiciona a atividade enunciativa'' \cite[p. 81-82]{marcuschi2008} –, propomos na subseção \ref{sec-4-1}, primeiramente, contextualizar o discurso do qual estamos tratando (da sociedade do conhecimento); em seguida, na subseção \ref{sec-4-2} apresentamos as condições de produção e os textos fonte e parafrástico para que, finalmente, efetuemos a análise e a discussão dos resultados.

\subsection{A sociedade do conhecimento: breve apanhado}\label{sec-4-1}
A noção de ``sociedade de informação'', introduzida por \textcite{bell1977}, ressurgiu com força nos anos 1990 devido ao desenvolvimento da Internet e das TICs. Essa agenda global foi incorporada por diversas agências da ONU e culminou na CMSI, programada para 2003 e 2005 \cite{mansell2015}. Nesse encontro, foi aprovada a transição do foco de ``informação'' para ``conhecimento'', partindo da premissa de que o acesso à informação não é suficiente, mas sim que os sujeitos devem ter habilidades para \textit{transformá-la em conhecimento}. Essa mudança prioriza os sujeitos que devem possuir qualificações a partir da \textit{aprendizagem}. No Brasil, a Sociedade da Informação foi formalmente estruturada por iniciativas incorporadas ao Plano Plurianual 2000-2003 e culminou na produção do \textit{Livro Verde} \cite{takahashi2000}. O alinhamento com as diretrizes da ONU resultou na centralidade da aprendizagem no programa brasileiro. O relatório da Unesco \cite{mansell2015} analisado neste estudo é um marco dessa \textit{transição para o conceito de ``Sociedades do Conhecimento''}, foco do Encontro de Revisão CMSI+10 em 2013.

Embora não haja evidências de que o \textit{Livro Verde da Sociedade da Informação no Brasil} (publicado em 2000) tenha sido atualizado, ele se estabeleceu como uma referência histórica sobre as estratégias iniciais do país. Ações subsequentes foram desenvolvidas com base nas discussões resultantes desse documento seminal para o debate público sobre o tema.

\subsection{Condições de produção: texto-matriz e texto-parafrástico}\label{sec-4-2}
Como já informamos, tomamos como \textit{corpus} o vídeo \textit{Especial Tecnologia na educação -- Por que usar tecnologia}, publicado em 24 de agosto de 2015 e idealizado pela plataforma \textit{Porvir} -- um projeto que se autodenomina como ``uma agência de jornalismo e soluções de comunicação, sem fins lucrativos, dedicada a impulsionar inovações na educação'' \cite{porvir2015b}. O vídeo está disponível no ciberespaço (espaço virtual resultante da comunicação mediada por computador), nos canais digitais Youtube (rede social) e minissite (página digital) \cite{porvir2015b}. Neste último, o texto está contido na aba ``Projetos especiais'' do menu principal, na seção ``Guias Temáticos'', cujo \textit{hiperlink} direciona o usuário do site para materiais interativos voltados para a formação de educadores e gestores. ``Os guias trazem conceitos, histórico, casos de sucesso e dicas de aplicação'' \cite{porvir2024}.

O roteiro do vídeo aborda o tema ``impacto da tecnologia na educação'' e adota um estilo de comunicação persuasivo voltado diretamente ao espectador. Trata-se de um testemunhal na modalidade de ``endosso de pessoa típica'' \cite[p. 43]{Barreto2004VendeSe30Segundos}, no qual a mensagem é transmitida por uma pessoa não famosa. No caso analisado, a diretora do Instituto Inspirare, Anna Penido, apresenta argumentos em defesa da inclusão de recursos tecnológicos na educação. Essa abordagem é comum em materiais de comunicação que buscam gerar confiança, estabelecer proximidade com o público ou iniciar a apresentação com uma voz percebida como legítima e relevante.

O vídeo exerce a função de gênero introdutório, equivale a um tipo de prólogo ou apresentação em textos convencionais, como nos livros. A finalidade desse gênero é fornecer uma ``leitura prévia, em termos de orientação, síntese ou convite à leitura dos gêneros que são `introduzidos'" \cite[p. 1]{bezerra2006}. Assim, o minissite poderia ser considerado um tipo de livro ou folheto informativo digital, interativo, e o vídeo, uma introdução de todo conteúdo disponibilizado na página, mediado por um discurso de autoridade \cite{porvir2015b}.

O texto-matriz é o relatório da Unesco, elaborado para o Encontro CMSI+10 em 2013 e publicado no Brasil em 2015. Nele, são avaliados os avanços no uso das TICs para o desenvolvimento sustentável desde a CMSI de 2003. Destaca a promoção de sociedades do conhecimento inclusivas e equitativas, com foco na paz, sustentabilidade, prosperidade econômica, justiça social e na centralidade da aprendizagem. O relatório também propõe diretrizes para a UNESCO com vistas à promoção de sociedades do conhecimento inclusivas e equitativas, em um cenário global em transformação \cite{mansell2015}. Publicado em formato de livro, o documento destaca a necessidade de integrar metas de prosperidade econômica, proteção ambiental, equidade e justiça social. A aprendizagem é apontada como eixo central tanto dessas sociedades quanto das orientações delineadas no relatório.

Dito isso, passamos para a seção \ref{sec-5}, cujo foco se desloca para a aplicação do instrumental teórico-analítico, com o objetivo central de demonstrar como o diálogo estratégico promovido pelas paráfrases (resumidora, explicativa e explicitadora) permite a recontextualização e a reorientação do discurso global da UNESCO \cite{mansell2015} para a agenda específica do \textcite{porvir2015a}, evidenciando, assim, nuances e variações na reformulação do intertexto.

\section{Análise do \textit{corpus} e discussão dos resultados: a paráfrase entre UNESCO e \textit{Porvir}}\label{sec-5}
A análise se baseia na transcrição do vídeo, guião, que é o gênero textual utilizado para registrar a planificação dos elementos visuais e verbais \cite[p. 3]{nogueira2010}. Ele é estruturado em duas colunas: à esquerda, registramos as informações referentes às imagens; à direita, os elementos sonoros (\Cref{quadro-1}). Na coluna da esquerda, as siglas PM e PP se referem ao enquadramento da câmera. A primeira sigla corresponde plano médio (mostra a figura humana até à cintura) e a segunda, primeiro plano ou plano \textit{close-up} (mostra em detalhe a parte do rosto humano); já \textit{lettering} se refere às informações escritas que aparecem na tela, topicalizando a fala da porta-voz. Na \Cref{fig-1} e \Cref{quadro-1}, esses elementos podem ser exemplificados:

%--- codigo da figura 1 ---%
\begin{figure}[h!]
\centering
\begin{minipage}{0.80\textwidth}
\includegraphics[width=\textwidth]{image1.png}
\caption{\textit{Print screen} do vídeo do \textit{Porvir} na plataforma YouTube.}
\label{fig-1}
\source{\textcite{porvir2015a}.}
\end{minipage}
\end{figure}


O critério adotado para a seleção dos enunciados analisados recorta os textos parafrásticos constituídos por paráfrases referenciais, nos quais o texto falado pela porta-voz é \textit{resumido em sentenças menores} e destacado no vídeo por meio do encapsulamento anafórico (novo referente discursivo criado sob informação velha), representado pelo \textit{lettering}. A repetição, sob forma de frase nominal -- o \textit{lettering-}, funciona como uma paráfrase resumidora de uma porção de texto, ``predicando os argumentos que são considerados pelo locutor como de maior peso para a tese que defende'' \cite[p. 125]{ribeiro2001}. Neste caso, o recorte do guião ficou o seguinte:

%--- codigo da tabela 1 ---%
\begin{table}[h!]
\centering
\begin{threeparttable}
\caption{Transcrição dos cinco enunciados parafrásticos do vídeo \textit{(corpus)}.}\label{quadro-1}
\begin{tabular}{p{5cm} p{8.5cm}}
\toprule
IMAGEM & SOM \\ 
\midrule
\begin{minipage}[t]{\linewidth}
(1) Porta-voz em PM.

\medskip

\textit{LETTERING:} \textbf{EQUIDADE}
\end{minipage} & ANNA: O primeiro deles, a \underline{equidade}. Com tecnologia a gente consegue \underline{ampliar o acesso dos alunos, não importa} se eles estão em \underline{regiões} vulneráveis ou até geograficamente dispersas.
\\ \addlinespace[8pt]
\begin{minipage}[t]{\linewidth}
(2) Porta-voz em PP. 

\medskip

\textit{LETTERING:} \textbf{QUALIDADE} \end{minipage} & ANNA: Outro desafio que as tecnologias ajudam a gente a superar é o da \underline{qualidade}.
\\ \addlinespace[8pt]
\begin{minipage}[t]{\linewidth}
(3) Porta-voz em PM. 

\medskip

\textit{LETTERING:} \textbf{É PRECISO EVITAR O AUMENTO DA DESIGUALDADE} \end{minipage} & ANNA: É preciso também \underline{evitar os efeitos prejudiciais} do uso de tecnologia na educação como a dispersão e até mesmo o \underline{aumento da desigualdade} se a gente garantir o \underline{acesso} a esses recursos a \underline{apenas uma parcela estudantes brasileiros}.
\\ \addlinespace[8pt]
\begin{minipage}[t]{\linewidth}
(4) Porta-voz em PM.

\medskip

\textit{LETTERING:} \textbf{FORMAR PROFESSORES} \end{minipage} & ANNA: Outra condicionante importantíssima para o bom uso de tecnologia é a \underline{formação do professor}. É importante que ele seja \underline{capacitado já com o uso desses recursos} para que ele possa ir se familiarizando, entendendo as possibilidades.
\\ \addlinespace[8pt]
\begin{minipage}[t]{\linewidth}
(5) Porta-voz em PP. 

\medskip

\textit{LETTERING:}  \textbf{MOBILIZAR A SOCIEDADE, FAMÍLIA E ALUNOS} \\[0.3\baselineskip]
FADE OUT \\[0.3\baselineskip] Entra assinatura. \end{minipage} & ANNA: Por fim, é importantíssimo mobilizar a sociedade brasileira [...], para garantir uma educação de qualidade para todos os brasileiros, que os preparem para a vida e garanta que eles possam \underline{aprender ao longo de toda sua existência}. \\
\bottomrule
\end{tabular}
\source{\textcite{porvir2015a}.}
\end{threeparttable}
\end{table}

Dessa forma, o guião permitiu a identificação dos elementos multimodais do \textit{corpus} \cite{porvir2015a}, que são confrontados a seguir em quadros de análise com o texto-matriz da Unesco \cite{mansell2015}, seus intertextos.


\subsection{Apresentação dos resultados: o confronto dos fragmentos de intertextos e paráfrases}\label{sec-codigos}
Para o movimento analítico, apresentamos primeiramente os fragmentos do texto da Unesco \cite{mansell2015}, que funcionam como intertextos das paráfrases do \textcite{porvir2015a}. A análise baseia-se na perspectiva retórica de reformulação parafrástica \cite{fuchs2012}, considerando a influência ativa do sujeito e o contexto situacional na reconstrução dos sentidos. Para marcarmos as mudanças na reformulação parafrástica, criamos tabelas para cada paráfrase, negritamos os trechos do texto fonte/intertexto, grifamos sua reformulação no texto parafrástico e incluímos as conclusões analíticas:

%--- codigo da tabela 2 ---%
\begin{table}[htbp]
\centering
\small
\begin{threeparttable}
\caption{Quadro-modelo para análise das paráfrases versus texto original (Unesco).}
\label{quadro-2}
\begin{tabular}{p{1.5cm} p{3.5cm} p{4cm} p{4cm}}
\toprule
Paráfrase [N] & Texto original (MATRIZ -- UNESCO) & Texto parafrástico (\textit{PORVIR} -- VÍDEO) & Análise e classificação \cite{fuchs2012, ribeiro2001} \\ 
\midrule
Fragmentos & Citação da matriz \cite{mansell2015} & Transcrição do \textit{lettering} e da fala (ANNA), extraídos da Tabela \ref{quadro-1}. & \textbf{Tipo de paráfrase:} resumidora, explicitadora e/ou explicativa. \\ \addlinespace
& Destaques: o trecho \textbf{``negritado''} é o conteúdo que será retomado na análise.  & Destaques: o trecho \underline{sublinhado} ou a reformulação são o novo sentido. & \textbf{Operação discursiva:} mecanismo discursivo (condensação ou expansão) e o efeito de recontextualização. \\
\bottomrule
\end{tabular}
\source{Autoria nossa a partir do \textcite{porvir2015a} e \textcite{mansell2015}.}
\end{threeparttable}
\end{table}

A partir do quadro-modelo de análise (\Cref{quadro-2}), passamos ao estudo e à classificação das cinco paráfrases (\Cref{quadro-1}), organizadas em alíneas, para facilitar a visualização do/a leitor/a.
\newline

a) \textbf{Paráfrase 1, \Cref{tab-3}:} o texto matriz, ``[...] acesso universal à informação [...]'' é parte da oração subordinada substantiva objetiva direta. Ela é parafraseada pelo significante ``equidade'', na modalidade de paráfrase resumitiva. De acordo com o dicionário \cite{dicio2026}, ``equidade'' significa ``igualdade; característica de algo ou alguém que revela senso de justiça, imparcialidade, isenção e neutralidade''. ``Igualdade'' é a causa do ``acesso universal à informação'', em outras palavras, funciona como um termo geral, cuja resultante é a universalização da informação (igualdade específica à Sociedade da Informação) ou como a acepção de universalizar diz, ``tornar generalizado; fazer com que atinja o maior número de pessoas''. A substituição de ``acesso universal'' por ``equidade'', termo que apela mais para o caráter emotivo, aproxima o texto de pautas sociais brasileiras. A paráfrase ``ampliar o acesso dos alunos, não importa se eles estão em regiões vulneráveis ou até geograficamente dispersas'' funciona como paráfrase explicitadora tanto da matriz ``acesso universal à informação'' quanto no interior da paráfrase, para explicar como se atinge a ``equidade'', de forma a tangibilizá-la. Para se ter acesso à informação/conhecimento, ele deve ser disponibilizado ou circular de forma universal, isto é, ``em todas as partes do mundo, especialmente nas áreas menos desenvolvidas'' que, como consequência do subdesenvolvimento, tornam-se ``regiões vulneráveis ou até geograficamente dispersas''. Desse modo, ao adicionar ``regiões vulneráveis'', localiza [quem?] o problema do acesso à informação ao contexto brasileiro. Temos, portanto, uma paráfrase explicitadora, resultante de um movimento expansivo. Veja a síntese desta análise na \Cref{tab-3}:

%--- codigo da tabela 3 ---%
\begin{table}[h!]
\centering
\small
\begin{threeparttable}
\caption{Confronto analítico da paráfrase 1 (Tabela \ref{quadro-1}) e texto original (Unesco).}\label{tab-3}
\begin{tabular}{p{1.5cm} p{3.5cm} p{4cm} p{4cm}}
\toprule
Paráfrase 1 &
Texto original (MATRIZ -- UNESCO) &
Texto parafrástico (\textit{PORVIR} -- VÍDEO) &
Análise e classificação \cite{fuchs2012, ribeiro2001} \\
\midrule

Fragmentos &
a) ``A UNESCO compreende que \textbf{acesso universal à informação} é um requisito básico para a criação de sociedades do conhecimento [...]'' \cite[p. ix, grifos nossos]{mansell2015}.

\medskip

b) ``Deve-se priorizar maneiras para facilitar a \textbf{rápida circulação do conhecimento científico em todas as partes do mundo, especialmente nas áreas menos desenvolvidas}''  \cite[p. ix, grifos nossos]{mansell2015}. &

Coluna ``Imagem'': \newline
Porta-voz em PM.

\medskip

\textit{LETTERING:} \newline
\textbf{EQUIDADE}

\medskip

Coluna ``Som'': \newline
ANNA: O primeiro deles, a \textbf{equidade}. Com tecnologia a gente consegue \textbf{ampliar o acesso dos alunos, não importa se eles estão em regiões vulneráveis ou até geograficamente dispersas}. &

\textbf{Tipos de paráfrase:} resumitiva e explicitadora.

\medskip

\textbf{Operação discursiva:} condensação e expansão de sentido.

\medskip

\textbf{Efeito de recontextualização:} a substituição de ``acesso universal'' por um mais condensado, ``equidade'', utiliza um termo com maior apelo emotivo para o público. A paráfrase que menciona ``regiões vulneráveis'' funciona como paráfrase explicitadora, localizando o problema do acesso no contexto brasileiro, tangibilizando a equidade. \\
\bottomrule
\end{tabular}
\source{Autoria nossa a partir de \textcite{porvir2015a} e \textcite{mansell2015}.}
\end{threeparttable}
\end{table}


b) \textbf{Paráfrase 2, \Cref{tab-4}:} atentemo-nos ao vocábulo ``qualidade'' tanto no intertexto quanto na paráfrase. Na matriz, a palavra exerce a função de adjunto adnominal do substantivo ``educação'' (``educação de qualidade'') juntamente aos termos ``para todos, em todos os níveis''. Na paráfrase, ``qualidade'' é o atributo da palavra ``desafio'', esmiuçado na sequência do texto em ações de como pode ser alcançado. Trata-se de uma paráfrase resumitiva, na qual do intertexto para a paráfrase ocorre um deslocamento de sentido que vai do específico para o geral, havendo uma condensação sintático-lexical na paráfrase. O enunciado ``educação de qualidade para todos, em todos os níveis'' é condensado em ``qualidade'', excluindo os níveis de ensino, ou seja, sem considerar as especificidades de cada nível ou modalidade (ensinos fundamental, médio, superior, EJA, EaD), podendo levar a sentidos de que a tecnologia resolveria tudo. Na \Cref{tab-4} organizamos os principais resultados:

%--- codigo da tabela 4 ---%
\begin{table}[h!]
\centering
\small
\begin{threeparttable}
\caption{Confronto analítico da paráfrase 2 (Tabela \ref{quadro-1}) e texto original (Unesco).}\label{tab-4}
\begin{tabular}{p{1.5cm} p{3.5cm} p{4cm} p{4cm}}
\toprule
Paráfrase 2 &
Texto original (MATRIZ -- UNESCO) &
Texto parafrástico (\textit{PORVIR} -- VÍDEO) &
Análise e classificação \cite{fuchs2012, ribeiro2001} \\
\midrule
Fragmentos &
``A \textbf{educação de qualidade para todos, em todos os níveis}, precisa ser um dos principais objetivos das sociedades do conhecimento para a paz e desenvolvimento sustentável'' \cite[p. 7, grifos nossos]{mansell2015}. &

Coluna ``Imagem'': \newline
Porta-voz em PP.

\medskip

\textit{LETTERING:} \newline
\textbf{QUALIDADE}

\medskip

Coluna ``Som'': \newline 
ANNA: Outro desafio que as tecnologias ajudam a gente a superar é o da \underline{qualidade}. &

\textbf{Tipo de paráfrase:} resumitiva.

\medskip

\textbf{Operação discursiva:} condensação sintático-lexical.

\medskip

\textbf{Efeito de recontextualização:} deslocamento do específico para o geral, por meio de simplificação e de omissão de detalhes sobre os níveis de ensino, dando a impressão de que a tecnologia é uma solução universal para a qualidade educacional. \\
\bottomrule
\end{tabular}
\source{Autoria nossa a partir de \textcite{porvir2015a} e \textcite{mansell2015}.}
\end{threeparttable}
\end{table}


c) \textbf{Paráfrase 3, \Cref{quadro-5}:} esse fragmento consiste numa paráfrase explicitadora. A matriz ``superação das novas exclusões digitais'' equivale semanticamente ao enunciado parafrástico (coluna esquerda) ``é preciso evitar o aumento da desigualdade'' \cite{porvir2015a}. Assim, na reformulação parafrástica, o termo ``exclusões digitais'' é substituído por ``desigualdade'', termo mais familiar ao público. Nesse movimento de condensação, o substantivo ``desigualdade'' corresponde efetivamente à matriz a partir da paráfrase explicitadora: o aumento da desigualdade ocorrerá ``se a gente garantir o acesso a esses recursos a apenas uma parcela estudantes brasileiros'', tendo como consequência ``novas exclusões digitais'' ou ``aumento da desigualdade'' do tipo digital \cite{porvir2015a}. Ocorre aqui uma explicitação, num movimento de ampliação de sentidos do intertexto, conforme esquematizado na Tabela \ref{quadro-5}:

%--- codigo da tabela 5 ---%
\begin{table}[h!]
\centering
\small
\begin{threeparttable}
\caption{Confronto analítico da paráfrase 3 (Tabela \ref{quadro-1}) e texto original (Unesco).}\label{quadro-5}
\begin{tabular}{p{1.5cm} p{3.5cm} p{4cm} p{4cm}}
\toprule
Paráfrase 3 &
Texto original (MATRIZ -- UNESCO) &
Texto parafrástico (\textit{PORVIR} -- VÍDEO) &
Análise e classificação \cite{fuchs2012, ribeiro2001} \\
\midrule
Fragmentos &
``Isso demanda políticas focadas na \textbf{superação das novas exclusões digitais} que surgem em todo o mundo, entre diferentes sociedades ou mesmo dentro delas'' \cite[p. vii]{mansell2015}. &

Coluna ``Imagem'': \newline
Porta-voz em PM.

\medskip

\textit{LETTERING:} \newline
\textbf{É PRECISO EVITAR O AUMENTO DA DESIGUALDADE}

\medskip

Coluna ``Som'': \newline
ANNA: É preciso também \underline{evitar os efeitos prejudiciais} do uso de tecnologia na educação como a dispersão e até mesmo o \underline{aumento da desigualdade} se a gente garantir o \underline{acesso} a esses recursos a \underline{apenas uma parcela} \underline{estudantes brasileiros}. &

\textbf{Tipo de paráfrase:} explicitadora.

\medskip

\textbf{Operação discursiva:} expansão de sentidos do intertexto.

\medskip

\textbf{Efeito de recontextualização:} a paráfrase explicita o conteúdo resumido da matriz, contextualizando problemas globais (exclusão digital) na realidade local (desigualdades no Brasil). A reformulação facilita a transposição de discursos técnicos e políticos em linguagem acessível. \\
\bottomrule
\end{tabular}
\source{Autoria nossa a partir de \textcite{porvir2015a} e \textcite{mansell2015}.}
\end{threeparttable}
\end{table}


d) \textbf{Paráfrase 4, \Cref{quadro-6}:} o texto da paráfrase 4 estabelece uma paráfrase explicativa com a matriz. Entre o intertexto, no qual apresenta-se o substantivo ``instrutores'' e o adjunto adnominal ``bem treinados'' \cite[p. x]{mansell2015}, e o texto parafrástico composto pelo predicado nominal ``a formação do professor'', há um movimento de explicação de como o instrutor se torna bem treinado, visto se tratar ainda de algo vago, mais abstrato. Ele deve ser ``capacitado já com o uso desses recursos'' (paráfrase referencial de ``tecnologia'', \Cref{quadro-6}). Para ter ``instrutores bem treinados'' é preciso ``formar professores''. A legenda ``formar professores'' é a ação que precede a existência de ``instrutores bem treinados''. Além disso, é preciso considerar a diferença entre ``instrutores bem treinados'' e ``formar professores'', visto que a diferença entre ``instrutor'' e ``professor'' vai além da sinonímia, pois envolve contexto social de atuação, formação, objetivos pedagógicos e relação com o aprendizado.

%--- codigo da tabela 6 ---%
\begin{table}[h!]
\centering
\begin{threeparttable}
\caption{Confronto analítico da paráfrase 4 (Tabela \ref{quadro-1}) e texto original (Unesco).}\label{quadro-6}
\begin{tabular}{p{1.5cm} p{3.5cm} p{4cm} p{4cm}}
\toprule
Paráfrase 4 &
Texto original (MATRIZ -- UNESCO) &
Texto parafrástico (\textit{PORVIR} -- VÍDEO) &
Análise e classificação \cite{fuchs2012, ribeiro2001} \\
\midrule
Fragmentos &
``Mas esse potencial só pode se materializar se requisitos básicos forem cumpridos: acima de tudo, conteúdo de alta qualidade e \textbf{instrutores bem treinados}'' \cite[p. x, grifos nossos]{mansell2015}. &

Coluna ``Imagem'': \newline
Porta-voz em PM.

\medskip

\textit{LETTERING:} \newline
\textbf{FORMAR PROFESSORES}

\medskip

Coluna ``Som'': \newline
ANNA: Outra condicionante importantíssima para o bom uso de tecnologia é a \underline{formação do professor}. É importante que ele seja \underline{capacitado já com o uso} \underline{desses recursos} para que ele possa ir se familiarizando, entendendo as possibilidades. &

\textbf{Tipo de paráfrase:} explicativa.

\medskip

\textbf{Operação discursiva:} explicação definidora, num movimento de expansão parafrástica. A reformulação ressalta que o requisito abstrato da UNESCO, ``instrutores bem treinados'', é transformado pelo \textit{Porvir} na ação concreta ``formação do professor'', um movimento característico da paráfrase explicativa. A diferença entre ``instrutor'' e ``professor'' vai além da sinonímia, pois envolve contexto social de atuação, formação, objetivos pedagógicos e relação com o aprendizado. \\
\bottomrule
\end{tabular}
\source{Autoria nossa a partir de \textcite{porvir2015a} e \textcite{mansell2015}.}
\end{threeparttable}
\end{table}


e) \textbf{Paráfrase 5, \Cref{quadro-7}:} neste texto, vemos um movimento explicativo, no qual a reformulação parafrástica realizada pelo \textcite{porvir2015a} (\Cref{quadro-7}) provoca inicialmente uma condensação de sentidos, pois enquanto a Unesco \cite{mansell2015} descreve um cenário concreto, com termos como ``expansão da conectividade'', ``sites de educação aberta'', o \textcite{porvir2015a} os elimina, direcionando os sentidos para ``aprender ao longo de toda sua existência''. Dessa forma, produz um texto ligado mais ao simbólico e ao emocional, adequando-o a um público não especializado. Também podemos notar que o \textcite{porvir2015a} substitui o termo técnico por uma expressão de apelo mais universal, ao trocar ``aprendizes ao longo da vida'' \cite[p. 27]{mansell2015} por ``aprender ao longo de toda sua existência'' \cite{porvir2015a}. Essa substituição causa uma mudança nos sentidos, visto que a Unesco \cite[p. 27]{mansell2015} usa um termo político (``aprendizes ao longo da vida'', ligado a políticas públicas globais), enquanto o \textcite{porvir2015a} opta por uma fórmula existencial (``toda sua existência''), que soa mais filosófica e menos institucional. Ao inserir termos como ``sociedade brasileira'' e ``educação de qualidade para todos os brasileiros'', contextualizando o discurso global da Unesco \cite{mansell2015} para uma agenda nacional, o \textcite{porvir2015a} adiciona ao texto valores locais, criando um elo emocional com seu público-alvo -- os educadores brasileiros --, ao passo que intencionalmente omite o aspecto global apresentado pelo intertexto. Confira a síntese na \Cref{quadro-7}:
\newpage
%--- codigo da tabela 7 ---%
\begin{table}[h!]
\centering
\begin{threeparttable}
\caption{Confronto analítico da paráfrase 5 (Tabela \ref{quadro-1}) e texto original (Unesco).}\label{quadro-7}
\begin{tabular}{p{1.5cm} p{3.5cm} p{4cm} p{4cm}}
\toprule
Paráfrase 5 &
Texto original (MATRIZ -- UNESCO) &
Texto parafrástico (\textit{PORVIR} -- VÍDEO) &
Análise e classificação \cite{fuchs2012, ribeiro2001} \\
\midrule
Fragmentos &
``Com a expansão da conectividade em rede, milhares de estudantes e pessoas que são \textbf{aprendizes ao longo da vida} estão usando \textit{sites} de educação aberta'' \cite[p. 27, grifos nossos]{mansell2015}. &

Coluna ``Imagem'': \newline
Porta-voz em PP.

\medskip

\textit{LETTERING:} \newline
\textbf{MOBILIZAR A SOCIEDADE, FAMÍLIA E ALUNOS}

\medskip

FADE OUT. \newline
Entra assinatura.

\medskip

Coluna ``Som'': \newline
ANNA: Por fim, é importantíssimo mobilizar a sociedade brasileira [...], para garantir uma educação de qualidade para todos os brasileiros, que os preparem para a vida e garanta que eles possam \underline{aprender ao longo de toda} \underline{sua existência}. &

\textbf{Tipo de paráfrase:} explicativa.

\medskip

\textbf{Operação discursiva:} condensação (do cenário concreto) e explicação/expansão (da ideia central).

\medskip

\textbf{Efeito de recontextualização:}  a reformulação contextualiza o discurso global da Unesco para uma \textbf{agenda nacional}, inserindo termos como \textbf{``sociedade brasileira''} e \textbf{``todos os brasileiros''} e adicionando \textbf{valores locais}, reforçando um \textbf{elo emocional} com o público-alvo.
Trocando o termo técnico ``aprendizes ao longo da vida'' (Unesco) por uma expressão de maior apelo emocional e existencial ``aprender ao longo de toda sua existência'' (Porvir).
 \\
\bottomrule
\end{tabular}
\source{Autoria nossa a partir de \textcite{porvir2015a} e \textcite{mansell2015}.}
\end{threeparttable}
\end{table}

Isso posto, o confronto detalhado dos cinco fragmentos entre o texto-matriz da Unesco \cite{mansell2015} e o texto parafrástico do \textcite{porvir2015a} demonstrou um padrão consistente no uso da paráfrase como ferramenta de reformulação intencional. Verificou-se a mobilização das paráfrases resumidora, explicativa e explicitadora, as quais operaram condensação e expansão de sentido. Tais operações permitiram ao \textcite{porvir2015a} selecionar argumentos e adaptar conceitos globais (como ``acesso universal à informação'' ou ``superação das novas exclusões digitais'') para termos mais acessíveis e contextualizados na agenda nacional (como ``equidade'' e ``aumento da desigualdade''). A próxima seção aprofunda as implicações desses resultados empíricos, analisando a paráfrase sob a ótica do diálogo estratégico, da intertextualidade das semelhanças implícita e explorando os limites e riscos desse processo de recontextualização.

\subsection{Discussão e implicações dos resultados: recontextualização estratégica e diálogo discursivo}\label{sec-5-2}
A análise das reformulações demonstrou a importância da paráfrase como mecanismo linguístico e discursivo que permite a recontextualização de ideias em diferentes esferas de comunicação. Sob a ótica do Círculo de Bakhtin \cite{volochinov2017}, em que todo enunciado é um elo na cadeia discursiva, o vídeo do \textcite{porvir2015a}, ao retomar o relatório da Unesco \cite{mansell2015}, não apenas reproduz conteúdos, mas os reinterpreta e adapta, reforçando seu papel como mediador entre políticas globais e práticas locais na educação brasileira.

Nesse diálogo estratégico, a paráfrase cumpre um duplo papel essencial, permitindo ao Porvir, como plataforma de jornalismo e mobilização educacional:
\begin{itemize}
    \item a) reinterpretar e adaptar o discurso técnico: O \textit{Porvir} procura traduzir discursos técnicos e políticos em linguagem acessível para seu público. Isso é realizado por meio da condensação de termos complexos (como a substituição de ``acesso universal à informação'' por ``equidade''), tornando o discurso da Unesco mais adaptável e inspirador;
    
    \item b) legitimar e reorientar a agenda: Ao parafrasear um relatório de uma instituição global como a UNESCO, o \textcite{porvir2015a} direciona credibilidade ao seu próprio conteúdo, posicionando-se como porta-voz qualificado das inovações educacionais. Essa reformulação reorienta o discurso original para destacar prioridades institucionais, como a formação docente, e reforça uma narrativa de urgência para a transformação na educação brasileira.
\end{itemize}

Assim, apesar da eficácia da paráfrase como estratégia de recontextualização discursiva, é importante reconhecer seus limites e riscos, especialmente quando aplicada a documentos de políticas públicas complexas, como o relatório da Unesco. A análise do vídeo do \textcite{porvir2015a} revela três desafios principais no uso da paráfrase, quais sejam: i) a simplificação excessiva e perda de nuances de afirmações do intertexto, ao condensar e selecionar elementos do texto-fonte, o que pode levar a uma redução de complexidade; ii) generalizações perigosas, omitindo dimensões críticas apresentadas no texto retomado; e iii) apagamento de contradições, ao simplificar os conteúdos, visto que o texto do \textit{Porvir} foca apenas em benefícios, criando uma narrativa unilateral. Dessa forma, a simplificação pode levar a uma implementação superficial de políticas públicas, na qual a tecnologia é vista como solução mágica, sem enfrentar desafios sistêmicos.

Além disso, ao realizar uma intertextualidade implícita, parafraseando sem citar explicitamente a Unesco \cite{mansell2015}, o \textcite{porvir2015a} apaga a autoria original, fazendo com que seu público, formado por educadores e gestores, não reconheça a fonte das ideias sobre tecnologia educacional, atribuindo-as ao próprio \textit{Porvir} ou ao ``senso comum''. Assim, essa intertextualidade promove uma homogeneização das vozes do relatório, que dialoga com críticos, governos e movimentos sociais, substituindo-a por uma única voz institucional -- a do \textit{Inspirare/Porvir}. Isso pode enfraquecer o debate democrático, pois desvincula as propostas de seu contexto político original, tornando-as menos abertas a questionamentos.

\section{A paráfrase no diálogo (não-inocente): refratando a Unesco no discurso da inovação educacional}\label{sec-6}
As análises realizadas ao longo deste estudo tiveram como objetivo analisar a intertextualidade do discurso oficial da/sobre a Sociedade do Conhecimento e as TICs na educação, manifestada pela paráfrase linguística presente no vídeo \textit{Especial Tecnologia na educação – Por que usar tecnologia} \cite{porvir2015a}. A investigação se ancorou em uma perspectiva que integrou a LT e a concepção dialógica da linguagem. Essa articulação teórica mostrou-se crucial, pois o dialogismo, princípio constitutivo universal de toda a linguagem, permitiu ver a enunciação como um ato bilateral e um elo da comunicação sociocultural, permeado pela presença inevitável do outro. Já a LT, ao tratar a língua em uso e o texto como um construto sócio-histórico, forneceu o instrumental para examinar a paráfrase como um fenômeno de intertextualidade das semelhanças \cite{santanna2003}.

A articulação entre teoria e prática comprovou que a paráfrase é um ato discursivo intencional e uma ferramenta poderosa de recontextualização. O \textcite{porvir2015a} incorpora as vozes da Unesco \cite{mansell2015} em um diálogo estratégico, utilizando a paráfrase para, primeiramente, adaptar e legitimar o discurso, traduzindo discursos técnicos e complexos da macropolítica global da Unesco para uma linguagem mais acessível, o que torna os conceitos mais adaptáveis ao público-alvo de educadores e gestores brasileiros. Em segundo lugar, o uso dos tipos de paráfrase (resumidora, explicativa e explicitadora) demonstrou a seleção e a reconfiguração dos argumentos do intertexto para reforçar prioridades institucionais do \textit{Porvir}; por exemplo, a paráfrase reorienta o discurso original para destacar a formação docente como condição indispensável, um enfoque que era menos central no relatório da Unesco \cite{mansell2015}.

Nesse sentido, esta pesquisa oferece contribuições significativas tanto teóricas quanto empíricas. Do ponto de vista teórico, reforça a perspectiva de que a paráfrase, embora classificada como intertextualidade das semelhanças, é uma ferramenta ``não inocente'' que engendra reformulações histórico-sociais. Essa abordagem crítica permite ir além da mera classificação textual para analisar as implicações ideológicas do processo de reformulação. Quanto à contribuição empírica, a análise detalhada demonstrou como uma Organização da Sociedade Civil (OSC), como o \textit{Porvir}, atua como ponte mediadora entre as demandas globais e as micropráticas escolares no Brasil. O exame da paráfrase como mecanismo linguístico permitiu visualizar as operações discursivas de condensação e expansão de sentido que se mostraram essenciais para essa recontextualização do tema para a audiência local.

Considerando o panorama apresentado, o estudo também revelou os limites e riscos do uso estratégico da paráfrase, como a simplificação e o apagamento tanto da autoria quanto de debates. A condensação e a seleção de elementos do texto-fonte resultaram em uma simplificação excessiva, que pode levar a uma implementação superficial de políticas públicas, nas quais as TICs são vistas como uma solução mágica, sem enfrentar os desafios sistêmicos inerentes. Assim como, ao utilizar a intertextualidade implícita (paráfrase sem citação explícita), o \textit{Porvir} apaga a autoria da Unesco. Isso pode enfraquecer o debate democrático, pois desvincula as propostas de seu contexto político original, promovendo uma homogeneização de vozes e atribuindo as ideias ao próprio \textit{Porvir} ou ao ``senso comum''.

Para pesquisas futuras, sugerimos investigar agências de jornalismo e notícias educacionais, como o \textit{Porvir}, com temas como: i) análise do impacto na recepção; ii) mecanismos de homogeneização; iii) gêneros digitais e intertextualidade. A primeira possibilidade implica investigar o impacto da intertextualidade implícita na memória discursiva do público-alvo (educadores e gestores) do \textit{Porvir}. Como a omissão da fonte afeta o reconhecimento da origem das ideias e a avaliação crítica das propostas de inovação educacional? A outra temática compreende estudos que aprofundem como a paráfrase atua na despolitização de debates e no reforço de hegemonias no discurso educacional digital. E por fim, a exemplo da presente análise, é possível explorar o uso da paráfrase e de outros fenômenos intertextuais em outros gêneros digitais e mídias de mobilização, a fim de comparar as estratégias de recontextualização empregadas por diferentes plataformas de jornalismo educativo ou outras Organizações da Sociedade Civil (OSCs).


\printbibliography\label{sec-bib}
% if the text is not in Portuguese, it might be necessary to use the code below instead to print the correct ABNT abbreviations [s.n.], [s.l.]
%\begin{portuguese}
%\printbibliography[title={Bibliography}]
%\end{portuguese}


%full list: conceptualization,datacuration,formalanalysis,funding,investigation,methodology,projadm,resources,software,supervision,validation,visualization,writing,review
\begin{contributors}[sec-contributors]
\authorcontribution{Tacia Rocha}[conceptualization,methodology,investigation,writing]
\authorcontribution{Edson Carlos Romualdo}[supervision,validation,review]
\end{contributors}

\begin{dataavailability}
\txtdataavailability{databody} % options: dataavailable, dataonly, databody, datanotav, nodata
\end{dataavailability}

\begin{funding}
 Agradecimento ao Centro Universitário Cidade Verde (UniCV), instituição onde uma das autoras leciona, que viabilizou a presente produção acadêmica por meio da ``Bolsa Pesquisa''.
\end{funding}

\end{document}

