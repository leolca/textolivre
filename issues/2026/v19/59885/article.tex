% !TEX TS-program = XeLaTeX
% use the following command:
% all document files must be coded in UTF-8
\documentclass[english]{textolivre}
% build HTML with: make4ht -e build.lua -c textolivre.cfg -x -u article "fn-in,svg,pic-align"

\usepackage{longtable}
\usepackage{threeparttablex}


\journalname{Texto Livre}
\thevolume{19}
%\thenumber{1} % old template
\theyear{2026}
\receiveddate{\DTMdisplaydate{2025}{6}{24}{-1}} % YYYY MM DD
\accepteddate{\DTMdisplaydate{2025}{9}{17}{-1}}
\publisheddate{\DTMdisplaydate{2025}{12}{16}{-1}}
\corrauthor{Amirah Alzahrani}
\articledoi{10.1590/1983-3652.2026.59885}
%\articleid{NNNN} % if the article ID is not the last 5 numbers of its DOI, provide it using \articleid{} commmand 
% list of available sesscions in the journal: articles, dossier, reports, essays, reviews, interviews, editorial
\articlesessionname{articles}
\runningauthor{Alzahrani and Al-Malaji} 
%\editorname{Leonardo Araújo} % old template
\sectioneditorname{Daniervelin Pereira~\orcid{0000-0003-1861-3609}}
\layouteditorname{Saula Cecília~\orcid{0009-0006-3069-8480}}

\title{Mapping of the Metaverse in classrooms: a bibliometric analysis}
\othertitle{Mapeamento do Metaverso na sala de aula: uma análise bibliométrica}
% if there is a third language title, add here:
%\othertitle{Artikelvorlage zur Einreichung beim Texto Livre Journal}

\author[1]{Amirah Alzahrani~\orcid{0000-0002-5967-0813}\thanks{Email: \href{mailto:asalzahrani@ub.edu.sa}{asalzahrani@ub.edu.sa}}}
\author[2]{Tahani Al-Malaji~\orcid{0009-0001-5589-7447}\thanks{Email: \href{mailto:tahaniyasinmm@gmail.com}{tahaniyasinmm@gmail.com}}}
\affil[1]{University of Bisha, College of Education and Human Development, Department of Curriculum and Teaching Methods, Bisha, Saudi Arabia.}
\affil[2]{University of Jordan, Curriculum and Instruction, Educational Technology, Amman, Jordan.}

\addbibresource{article.bib}
% use biber instead of bibtex
% $ biber article

% used to create dummy text for the template file
\definecolor{dark-gray}{gray}{0.35} % color used to display dummy texts
\usepackage{lipsum}
\SetLipsumParListSurrounders{\colorlet{oldcolor}{.}\color{dark-gray}}{\color{oldcolor}}

% used here only to provide the XeLaTeX and BibTeX logos
\usepackage{hologo}

% if you use multirows in a table, include the multirow package
\usepackage{multirow}

% provides sidewaysfigure environment
\usepackage{rotating}

% CUSTOM EPIGRAPH - BEGIN 
%%% https://tex.stackexchange.com/questions/193178/specific-epigraph-style
\usepackage{epigraph}
\renewcommand\textflush{flushright}
\makeatletter
\newlength\epitextskip
\pretocmd{\@epitext}{\em}{}{}
\apptocmd{\@epitext}{\em}{}{}
\patchcmd{\epigraph}{\@epitext{#1}\\}{\@epitext{#1}\\[\epitextskip]}{}{}
\makeatother
\setlength\epigraphrule{0pt}
\setlength\epitextskip{0.5ex}
\setlength\epigraphwidth{.7\textwidth}
% CUSTOM EPIGRAPH - END

% to use IPA symbols in unicode add
%\usepackage{fontspec}
%\newfontfamily\ipafont{CMU Serif}
%\newcommand{\ipa}[1]{{\ipafont #1}}
% and in the text you may use the \ipa{...} command passing the symbols in unicode

% LANGUAGE - BEGIN
% ARABIC
% for languages that use special fonts, you must provide the typeface that will be used
% \setotherlanguage{arabic}
% \newfontfamily\arabicfont[Script=Arabic]{Amiri}
% \newfontfamily\arabicfontsf[Script=Arabic]{Amiri}
% \newfontfamily\arabicfonttt[Script=Arabic]{Amiri}
%
% in the article, to add arabic text use: \textlang{arabic}{ ... }
%
% RUSSIAN
% for russian text we also need to define fonts with support for Cyrillic script
% \usepackage{fontspec}
% \setotherlanguage{russian}
% \newfontfamily\cyrillicfont{Times New Roman}
% \newfontfamily\cyrillicfontsf{Times New Roman}[Script=Cyrillic]
% \newfontfamily\cyrillicfonttt{Times New Roman}[Script=Cyrillic]
%
% in the text use \begin{russian} ... \end{russian}
% LANGUAGE - END

% EMOJIS - BEGIN
% to use emoticons in your manuscript
% https://stackoverflow.com/questions/190145/how-to-insert-emoticons-in-latex/57076064
% using font Symbola, which has full support
% the font may be downloaded at:
% https://dn-works.com/ufas/
% add to preamble:
% \newfontfamily\Symbola{Symbola}
% in the text use:
% {\Symbola }
% EMOJIS - END

% LABEL REFERENCE TO DESCRIPTIVE LIST - BEGIN
% reference itens in a descriptive list using their labels instead of numbers
% insert the code below in the preambule:
%\makeatletter
%\let\orgdescriptionlabel\descriptionlabel
%\renewcommand*{\descriptionlabel}[1]{%
%  \let\orglabel\label
%  \let\label\@gobble
%  \phantomsection
%  \edef\@currentlabel{#1\unskip}%
%  \let\label\orglabel
%  \orgdescriptionlabel{#1}%
%}
%\makeatother
%
% in your document, use as illustraded here:
%\begin{description}
%  \item[first\label{itm1}] this is only an example;
%  % ...  add more items
%\end{description}
% LABEL REFERENCE TO DESCRIPTIVE LIST - END


% add line numbers for submission
%\usepackage{lineno}
%\linenumbers

\begin{document}
\maketitle

\begin{polyabstract}
\begin{abstract}
This study aims to map the research landscape of the Metaverse in classroom settings through a comprehensive bibliometric analysis, identifying trends, key contributors, and citation patterns. A bibliometric analysis was conducted using the Web of Science database, recognized for its extensive scientific literature, on articles related to the Metaverse in education. The search, carried out on September 7th, 2024, covered publications from 2008 to 2024. Data on publication counts, leading research countries, and citations were systematically collected and analyzed. The results revealed a total of 81 publications on Metaverse research in educational contexts from 2008 to 2024, indicating a slow growth trend. China contributed 45 publications, accounting for 23\% of the total, and emerged as the leading contributor. The \textit{IEEE Transactions on Learning Technologies}, a peer-reviewed scientific journal covering advances in learning technologies, was identified as the most productive journal. The cumulative citations of the total publications reached 1,497, with the paper titled ``Extending the Technology Acceptance Model (TAM) to Predict University Students' Intentions to Use Metaverse-Based Learning Platforms'' being the most cited, with 95 citations. China and the United States were recognized as major centers for citation interactions, with China accounting for 501 citations from its 45 articles. The study provides a foundational understanding of the current state of Metaverse research within classroom settings and highlights key contributors and trends to guide future research directions and educational practices.

\keywords{Metaverse\sep Classroom\sep Bibliometric analysis\sep VOS Viewer}
\end{abstract}

\begin{english}
\begin{abstract}
Este estudo tem como objetivo mapear o panorama da pesquisa sobre o Metaverso em contextos de sala de aula por meio de uma análise bibliométrica, identificando tendências, principais contribuintes e padrões de citação. A análise foi realizada na base de dados Web of Science, reconhecida por sua ampla coleção de literatura científica, com a busca sendo executada em 7 de setembro de 2024, considerando publicações de 2008 a 2024. Os dados sobre número de publicações, países líderes e citações foram coletados e analisados sistematicamente. Os resultados revelaram um total de 81 publicações sobre o Metaverso em contextos educacionais, indicando um crescimento lento. A China destacou-se como o principal contribuidor, com 45 publicações, representando 23\% do total. O periódico \textit{IEEE Transactions on Learning Technologies}, uma revista científica revisada por pares que aborda avanços em tecnologias de aprendizagem, foi o mais produtivo. As citações totais atingiram 1.497, sendo o artigo ``Extending the Technology Acceptance Model (TAM) to Predict University Students’ Intentions to Use Metaverse-Based Learning Platforms'' o mais citado, com 95 citações. China e Estados Unidos foram identificados como os principais centros de interação de citações, sendo que a China respondeu por 501 citações provenientes de seus 45 artigos. O estudo fornece uma compreensão fundamental do estado atual das pesquisas sobre o Metaverso em sala de aula e destaca tendências e contribuintes que podem orientar futuras práticas e investigações educacionais.

\keywords{Metaverso\sep Sala de aula\sep Análise bibliométrica\sep VOS Viewer}
\end{abstract}
\end{english}
% if there is another abstract, insert it here using the same scheme
\end{polyabstract}

\section{Introdução}\label{sec-intro}
The concept of Metaverse is composed of two words, ``META'' and ``UNIVERSE'', a blend of the real and virtual worlds \cite{mystakidis2022}. The Metaverse represents a multi-user environment, combining the real and virtual worlds through user interactions. Users select the avatars that represent them, enabling participation in various activities. The Metaverse offers a high degree of freedom for sharing and creating content \cite{yu2022, balat2023}. Additionally, it integrates virtual reality (VR), augmented reality (AR), and mixed reality (MR) \cite{ko2021, asiksoy2023}.

Thus, a lot of gaming companies have developed VR technologies, notably the platform called ``SECOND LIFE'', which was launched in 2003 and allows users to create avatars and engage in social interactions \cite{narin2021metaverse, ning2021, dionisio2013}. Through the avatars, which are virtual, users could interact with others in a three-dimensional (3D) environment, facilitating communication. Also, they can participate in various activities and build their own spaces. Hence, ``SECOND LIFE'' has become an inclusive platform for social interactions and for learning with ease \cite{baker2008}. The Facebook platform transitioned to the Metaverse, emphasizing its goal to expand the services and provide a more engaging interactive experience for users by integrating augmented and virtual reality in a single platform \cite{kraus2022, castronova2021}.

Following the COVID-19 pandemic, there has been a significant increase in the popularity of games that provide a new social space combining augmented and virtual realities, making these activities more common and used daily. Additionally, there was a surge in e-learning, conferences, discussions, and e-commerce, leading to a greater aspiration to leverage the diverse applications of the virtual world \cite{lee2022, kye2021}.

Despite the existence of several studies addressing the Metaverse in education \cite{mystakidis2021svr, ng2022metaverse, williams2022vrchemistry, rozi2023, rahman2023, pangsapa2023, pozosanchez2021}, they have not specified scopes such as classroom settings and have instead discussed education in general. Therefore, the current study aims to conduct a bibliometric analysis to synthesize the literature on the Metaverse in school classrooms. This analysis will involve identifying research trends, journals, countries, institutions, and the most prolific authors. Additionally, it will analyze research based on keywords and other factors to identify hotspots and trends in the Web of Science (WoS) database. The study also aims to enhance research on the Metaverse and understand future research trends, exploring experiences related to the Metaverse with the goal of identifying best practices in classroom settings.

Therefore, the study seeks to answer the following research questions: \newline
RQ1: What are the fundamental bibliometric characteristics of research focusing on Metaverse in the classroom? \newline
RQ2: What are the hotspots and emerging trends related to the application of the Metaverse in classrooms during different periods?

\section{Literature Review}
\subsection{Conceptual Background of the Metaverse in Education}
The word ``Metaverse'' was first introduced in 1992, in a science fiction novel called \textit{Snow Crash}. It emerged as a concept rapidly and evolved using games for purposes of entertainment, presenting challenges to users to enhance their motivation and to engage in various activities depending on synchronous communication with others. Two of the most popular of these games are Minecraft and Roblox \cite{chen2023, hussain2023}. Today, the Metaverse extends beyond entertainment and gaming to involve diverse fields, including health care, trading, and education.

The advancement of technology has led to significant and tangible leaps in the use of the Metaverse in education. The Metaverse is a shared platform that combines augmented and virtual reality, opening new horizons and assisting students in comprehensive learning and acquiring actual learning outcomes \cite{park2022gameful}. Additionally, the Metaverse introduces innovative ideas in teaching and learning, facilitating skill development through interaction and collaboration with others. Thus, it can be said that it enriches and advances education in a sophisticated manner, making it easier for learners \cite{verma2024}. Furthermore, the Metaverse is backed by technologies such as virtual reality (VR), augmented reality (AR), artificial intelligence (AI), robotics, blockchains, the Internet of Things (IoT), and human-computer interaction (HCI) \cite{dwivedi-etal2022, metwally2024, nahi2023}.

The Metaverse has emerged as a tool that fosters learner interaction and engagement in the education process \cite{rospigliosi2022, bizel2023}. The pandemic showed the importance of technological tools in education, resulting in increased utilization of these resources \cite{daniel2020}. Consequently, a digital era emerged, offering new, modern and improved methods and practices for education. The Metaverse offers a variety of digital technologies such as extended reality, augmented reality, virtual reality, cloud computing, big data, and deep learning. These technologies provide an interactive environment for learners and educators, allowing them to use avatars that motivate students and enhance their engagement in the learning process. The Metaverse facilitates access to information through diverse pedagogical methods \cite{guo2022, marquezdiaz2020virtualworld}.

Metaverse as a Community: Although there are no theories or pedagogical frameworks on how Metaverse technologies influence educational fields, the intersection of networked virtual worlds and online learning could create meaningful experiences for students \cite{keskitalo2011secondlife, mystakidis2021svr}. Recent studies have suggested the need to connect the Metaverse to pedagogical theories and to redefine these theories taking into account the characteristics of the Metaverse. Metaverse technologies have been identified as intersecting with a CoI, which deserves further exploration. \textcite{keskitalo2011secondlife} identified the use of Second Life as a process-specific feature of meaningful learning that is released during the learning process in a virtual world. In that study, they discussed the fact that students preferred collaborative activities in Second Life, which could lead to the creation of a CoI.
  
A study conducted by \textcite{mystakidis2021svr} suggested the use of a CoI, where students could build their own personal connections to establish communication, social presence, and collaboration, thus, handing meaningful and deep learning experiences for students in social VR environments. The authors emphasized the importance of social and cognitive support when discussing how to promote student engagement in an aviation virtual laboratory environment with the use of flight simulations and virtual tours \cite{ng2022metaverse}. Similarly, to provide immersive and realistic experiences for students and reduce attendance difficulties and safety concerns, \textcite{williams2022vrchemistry} incorporated meaningful learning into organic chemistry VR laboratories. These studies used the CoI framework as an educational model to investigate students’ knowledge and skills construction in the Metaverse in terms of teachers, social, and cognitive support/presence.

Furthermore, the Metaverse enhances learning outcomes by providing appropriate content and activities tailored to each student, allowing them to learn at their own pace and style. It also facilitates experiential learning through participation in virtual labs, simulations, and interactive case studies, deepening concepts and knowledge in classrooms \cite{rozi2023}.

The Metaverse assists educators and learners in exchanging educational materials, providing access to resources like eBooks and videos, and enabling students to display their academic success and accomplishments \cite{rahman2023}. Additionally, interactive pedagogical methods like flipped classrooms contribute to enhancing student participation and interaction \cite{pangsapa2023}. Research shows that these technologies can enhance student involvement and achievement, particularly in interactive teaching approaches \cite{pozosanchez2021}.

In this context, several studies have addressed the Metaverse in education. For instance, the study by \textcite{chen2023} utilized bibliometric analysis to examine 121 studies in the WoS database. The results indicated a growing interest in the Metaverse as an educational technology that enhances collaboration and interaction among students.

The study by \textcite{anuj2024} focused on the role of the Metaverse and understanding its educational outcomes. This study showed the success of this technology in providing innovative opportunities; yet, its effectiveness is tightly dependent on the instructional design and teaching methods utilized by teachers. Moreover, \textcite{bicen-babatunde2022} examined papers in the Scopus database. The findings underscored the significance of cooperation among educators, researchers, and practitioners to incorporate and advance the Metaverse within educational curriculums.

\textcite{balat2023} examined both bibliometric and content analyses related to the Metaverse in education, emphasizing tools, content, and effects. The results of the study provided a comprehensive understanding of how the Metaverse is utilized in educational environments. The study by \textcite{verma2024} highlights the importance of the Metaverse through its use of bibliometric analysis of a collection of 265 papers from the WoS database. It showcases the Metaverse as a thorough educational platform and a tool for resources that enriches understanding and offers many opportunities for inclusive and hands-on learning. The study by \textcite{bicen-adedoyin2022} analyzed 645 articles from the Scopus database and concluded that the use of the Metaverse has seen significant growth due to technological advancements. The study also highlighted a lack of research in this area, emphasizing the substantial impact of the Metaverse on education, including remote learning and its various applications.

\subsection{Immersive Learning Environments and Student Engagement}
It is in this line that \textcite{mikropoulos2011} pleaded for educational virtual environments to ensure cognitive engagement. Recent technological advancements in extended reality (XR) tools are taken as the building blocks for immersive experiences \cite{ko2021}. Moreover, the change has been proved to facilitate deeper understanding, particularly in such complex domains as science, engineering, and medicine.

Students’ motivation and participation were simply better, since they enjoyed and were more focused while working in 3D virtual worlds than in screen-based traditional teaching, as argued by \textcite{hussain2023}. Avatars and gamification created better competitive presence by the mere fact that there is some form of existence and competition in their learning, and these sustain interests over time. For the classroom learners at war with learning in the classroom, it provides alternative channels of expression and input. In addition, \textcite{guo2022} can confirm that there is increased emotional participation in the true, task-based virtual world, hence better retention of learned material.

Furthermore, the use of interactive Metaverse settings connects the gap between classroom teaching and its application. For example, online science labs can simulate chemical reactions or biological processes that may be very risky, expensive, and difficult to manage in real life \cite{dwivedi-rana2022}. The theory of \textcite{dede2009} is that the interfaces between incredibly designed worlds increase attention and memory by posing the real-world problem-solving task. Immersive Metaverse learning is considered a promising frontier that reforms the classroom by dealing with sensory, cognitive, and emotional dimensions of learning.


\subsection{Accessibility, Equity, and Inclusion in Virtual Education}
One of the most transformative features that metaverse-based education has to offer is the potential for quality learning democratization. According to \textcite{metwally2024}, regional disparities in outcomes and educational results could be minimized through accessibility.

Inclusive learning is more enabled by the flexible nature of avatars, interfaces, and scenarios of learning. According to \textcite{kye2021}, Metaverse platforms work with multiple learning styles that include offering visual, auditory, and kinesthetic inputs for diverse needs.

The Metaverse promises to be the most ultra-place of inclusivity. Yet, it raises a dual important concern about the digital divide. Necessary devices are not equally available to all populations (a virtual reality and high-speed internet, among other devices), which puts pre-existing inequalities at a higher risk of being accentuated. \textcite{gonzalezalcaide2022} forewarned that the overreliance upon the Web of Science and other such Western-centric repositories may result in underrepresentation of research coming from low-income or under-digitized regions. Therefore, educational policymakers must ensure access equity in infrastructure and digital literacy programs from which to take full advantage of the Metaverse as a tool of inclusion rather than exclusion. \textcite{ning2021} proposed flexible, multilingual interfaces and culturally neutral avatars for learners across the globe. In the true spirit of the advance above, international collaborative research can bring diversity of thought into developing applications for the educational Metaverse, and it will not mirror the leading culture but will make the culture truly inclusive of all and create an international educational ecosystem.

\section{Method}
This study utilized bibliometric analysis, recognized as an effective method for examining research trends and identifying gaps in published articles within this domain, aiding in directing future research (2008–2024). Furthermore, this method represents a structured scientific approach that employs statistical elements to analyze earlier literature and offer a thorough summary of the subject \cite{patton2014, alzahraani2024}. By utilizing a bibliometric analysis approach, this study investigates various bibliometric indicators to explore research trends, including the most productive authors, countries, and publication venues, while also identifying key trends and significant keywords for researchers and citations.

\subsection{Database and search}
This study employed the Web of Science database, acknowledged as one of the largest and most significant and available databases. It provides access to a diverse selection of scientific articles and journals, allowing researchers to look up previous studies, examine citations, and identify ongoing research trends \cite{gonzalezalcaide2022}. The search took place on September 7, 2024, and \Cref{tab-1} displays the codes documented in the database. A bibliometric analysis was performed without time limitations, enabling an in-depth examination of research trends.

%--- código da tabela 1 ---%
\begin{table}[h!]
\centering
\caption{Search Queries in Web of Science.}
\label{tab-1}
\begin{threeparttable}
\begin{tabular}{p{0.55\linewidth} p{0.18\linewidth}}
\toprule
(``Metaverse'') AND (``K-12'' OR ``School'' OR ``Students'' OR ``Pupil'' OR ``Learning'' OR ``Teaching'' OR 
``Class'' OR ``Curriculum'' OR ``Assessment'' OR ``Instruction'' OR ``Primary School'' OR 
``Elementary Education'' OR ``Secondary Education'' OR ``Evaluation'' OR ``Classroom'' OR ``Teacher'') & Topic \\[0.5em]

Article & Document type \\[0.5em]

2008--2024 & Time span \\[0.5em]

Education educational Research -- Psychology Multidisciplinary --
Social Sciences Interdisciplinary -- Education Scientific Disciplines -- Humanities
Multidisciplinary -- Social Issues -- Psychology Experimental -- Psychology Social --
Psychology Educational & Categories \\[0.5em]

SSCI \& ESCI & Index \\[0.5em]

English & Language \\
\bottomrule
\end{tabular}
\source{Own elaboration (2024).}
\end{threeparttable}
\end{table}

A bibliometric analysis was conducted on 1,687 articles. After filtering according to \Cref{tab-1}, 192 articles were selected. The research was then reviewed again on September 29, 2024, without any further filtering of the publications.

\subsection{Data Analysis}
The collected data were analyzed using descriptive statistics and the VOS viewer software in CSV format. This software is one of the most commonly used tools in various areas of bibliometric research, providing visual representations of results, facilitating the creation of visual maps, conducting citation analysis, and evaluating research collaboration among researchers and institutions, among other capabilities \cite{vaneck2010}. The resulting images are displayed in the \nameref{sec-results} section.

\subsection{Limitations of the Research Process}
Though the findings are based on up-to-date information, the process of research does have certain limitations:

\begin{itemize}
    \item \textbf{Scope of the Database:} It is most probable that the study used some specific databases like Scopus or Web of Science, which does not cover the important regional studies or gray literature.
    \item \textbf{Bias of Time Frame:} The fact that the analysis is limited to 2008–2024 means that other works, which were more influential, and which provided a base for the applications of Metaverse, were left out.
    \item \textbf{Language Barrier:} There is a potential language bias since most of the datasets are from English publications, which may be laboring under the non-representation of the research efforts put forth by non-English publications, specifically those of developing countries.
    \item \textbf{Quantitative Focus:} Bibliometrics can quantify the research but not the richness and real context of individual studies.
\end{itemize}

\section{Results}\label{sec-results}
\subsection{Publication Distribution by Year}
The study results revealed that the number of articles related to Metaverse research was low in the early years from 2008 to 2015, as shown in \Cref{tab-2}, indicating that Metaverse research was in its infancy during that time. Moreover, certain years showed no publications concerning the Metaverse in classrooms, in contrast to recent years that have witnessed a notable rise. In 2022, publications totaled 21, but they rose to 72 in 2023, continuing this trend into 2024 with a total of 81 publications. These figures indicate a growing interest among researchers in the field of the Metaverse and its application in classroom environments. \Cref{fig-1} illustrates the distribution of years.

%--- código da tabela 2 ---%
\begin{table}[h!]
\centering
\caption{Frequency of published articles.}
\label{tab-2}
\begin{threeparttable}
\begin{tabular}{l r}
\toprule
Number of Publications by Year & Publication Year\\
\midrule
1  & 2008 \\
1  & 2011 \\
2  & 2012 \\
10 & 2015 \\
3  & 2020 \\
1  & 2021 \\
21 & 2022 \\
72 & 2023 \\
81 & 2024 \\
[0.5em]
192 & Total \\
\bottomrule
\end{tabular}
\source{Own elaboration.}
\end{threeparttable}
\end{table}

%--- código da figura 1 ---%
\begin{figure}[h!]
\centering
\begin{minipage}{.80\textwidth}
\includegraphics[width=\textwidth]{Imagens/Fig1.png}
\caption{Distribution of publications by years.}
\label{fig-1}
\source{Own elaboration.}
\end{minipage}
\end{figure}


\subsection{Publication Distribution by Country}
From \Cref{tab-3}, it can be concluded that the foremost country in generating research concerning the Metaverse in educational environments is China, which publishes 45 papers at a rate of 23\%. When compared to other countries, this is considered a prominently elevated percentage, signifying China’s strong enthusiasm for technological advancements. South Korea and the United States follow closely behind, both having 29 publications and a rate of 15\%. This percentage underscores the significant contributions that both nations make in promoting research associated with the Metaverse. Turkey and Brazil are also among the leading producing countries, with Turkey contributing 18 papers at a rate of 9\%, while Brazil produced 12 publications at a rate of 6\%. Overall, there is a growing interest from various countries in the field of the Metaverse, reinforcing the significance of this technology and its applications in classroom settings. \Cref{fig-2} illustrates the distribution of countries based on productivity.

%--- código da tabela 3 ---%
\begin{table}[h!]
\centering
\caption{Top 15 countries.}
\label{tab-3}
\begin{threeparttable}
\begin{tabular}{r r r r}
\toprule
S & Country & Record Count & \% \\
\midrule
1  & China         & 45 & 23.4\% \\
2  & South Korea   & 29 & 15\%   \\
3  & USA           & 29 & 15\%   \\
4  & Turkey        & 18 & 9\%    \\
5  & Brazil        & 12 & 6\%    \\
6  & India         & 11 & 5.6\%  \\
7  & Taiwan        & 11 & 5.6\%  \\
8  & Spain         & 9  & 4.6\%  \\
9  & Australia     & 8  & 4\%    \\
10 & Saudi Arabia  & 7  & 3.5\%  \\
\bottomrule
\end{tabular}
\notes{TP = Total Publications.}
\source{Own elaboration.}
\end{threeparttable}
\end{table}

%--- código da figura 2 ---%
\begin{figure}[h!]
\centering
\begin{minipage}{.80\textwidth}
\includegraphics[width=\textwidth]{Imagens/Fig2.png}
\caption{Distribution of Publications by Country (2020-2024).}
\label{fig-2}
\source{Own elaboration.}
\end{minipage}
\end{figure}

\subsection{The most productive journals}
\Cref{tab-4} illustrates the list of the most productive journals in Metaverse research in classroom settings, with \textit{IEEE Transactions on Learning Technologies} at the forefront with 11\% and a total of 22 pu\-bli\-ca\-tions. This journal features a variety of fields, encompassing computer science and educational research, highlighting the convergence of disciplines and notable interest in Metaverse research across multiple fields. The journal \textit{Education and Information Technologies} ranks second, with 16 publications at a rate of 8\%. This journal emphasizes the educational aspects, highlighting the significance of utilizing technology to improve the education process. In third place is \textit{Advances in Educational Technologies and Instructional Design Book Series}, with 10 publications at 5\%. This journal covers fields such as computer science, engineering, and multidisciplinary applications, showcasing the diversity of topics it addresses. Other journals, such as \textit{Learning in Metaverses: Co-existing in Real Virtuality and Interactive Learning Environments}, appear in fourth and fifth places, reflecting the growing interest in the technical aspects of Metaverse research.

We can also conclude that most of the journals are published in the United States, which underscores the significant interest of the U.S. in scientific research and development in the fields of technology and education \cite{narin2021metaverse}.

%--- código da tabela 4 ---%
\begin{table}[h!]
\centering
\small
\caption{List of the 10 most productive journals.}
\label{tab-4}
\begin{threeparttable}
\begin{tabular}{p{0.10\linewidth} p{0.10\linewidth} p{0.28\linewidth} r r p{0.20\linewidth} r}
\toprule
Country & Publisher & JCR Subject Categories & P & C & Journal & R \\
\midrule
USA & IEEE Computer Soc & Computer Science, Interdisciplinary Applications | Education \& Educational Research & 11\% & 22 & IEEE Transactions on Learning Technologies & 1 \\[0.3em]

USA & Springer & Education \& Educational Research & 8\% & 16 & Education and Information Technologies & 2 \\[0.3em]

South Korea & IGI Global & Computer Science, Interdisciplinary Applications | Engineering, Multidisciplinary & 5\% & 10 & 
Advances in Educational Technologies and Instructional Design Book Series & 3 \\[0.3em]

USA & IGI Global & Psychology, Applied; Psychology, Social & 4\% & 8 &
Learning in Metaverses: Co-existing in Real Virtuality & 4 \\[0.3em]

United Kingdom & Taylor \& Francis & Education \& Educational Research & 4\% & 8 &
Interactive Learning Environments & 5 \\[0.3em]

Switzerland & Frontiers Media SA & Psychology, Multidisciplinary & 3\% & 6 &
Frontiers in Psychology & 6 \\[0.3em]

USA & Mary Ann Liebert & Psychology, Social & 2.6\% & 5 &
Cyberpsychology Behavior and Social Networking & 7 \\[0.3em]

Switzerland & Frontiers Media SA & Education \& Educational Research & 2\% & 4 &
Frontiers in Education & 8 \\[0.3em]

USA & Springer & Humanities, Multidisciplinary | Social Sciences, Interdisciplinary & 2\% & 4 &
Humanities Social Sciences Communications & 9 \\[0.3em]

USA & Amer Chemical Soc & Chemistry, Multidisciplinary | Education, Scientific Disciplines & 2\% & 4 &
Journal of Chemical Education & 10 \\
\bottomrule
\end{tabular}
\source{Own elaboration.}
\end{threeparttable}
\end{table}

\subsection{Co-Authorship Analysis by Country}
%--- código da figura 3 ---%
\begin{figure}[h!]
\centering
\begin{minipage}{.95\textwidth}
\includegraphics[width=\textwidth]{Imagens/Fig3.png}
\caption{Visualization of country performance.}
\label{fig-3}
\source{Own elaboration using VOS viewer.}
\end{minipage}
\end{figure}

It can be concluded that there are 12 countries involved in research related to the Metaverse in the classroom, with noticeable interactions among a group of nations. The \Cref{fig-3} illustrates the results of collaboration among authors from different countries, highlighting connections between nations such as the USA, China, and South Korea, reflecting research networks and international cooperation in the field of the Metaverse.

\subsection{Citation}
Citation analysis assists researchers in identifying the most impactful publications within a field and uncovers the cognitive framework of the research area by finding similar publications and recurring citations \cite{donthu2021}. The findings show that the overall citation count reached 1,497, demonstrating researchers’ engagement with Metaverse research in educational settings within the classroom. Reviewing the research, as shown in Table \ref{tab-5}, we find that the article ``Extending the Technology Acceptance Model (TAM) to Predict University Students' Intentions to Use Metaverse-Based Learning Platforms'' topped the list with 95 citations, highlighting the significance of this model in enhancing students’ motivation to use Metaverse platforms.

Following in second place is the article ``Virtual World as a Resource for Hybrid Education'', which received 92 citations, illustrating the role of virtual worlds in promoting hybrid learning. In third place, the article ``Utilizing Metaverse for Learner-Centered Constructivist Education in the Post-Pandemic Era: An Analysis of Elementary School Students'' received 87 citations, while the other articles had different citation counts. This underscores the growing interest in the Metaverse and how it can be applied and leveraged in education.

%--- código da tabela 5 ---%
\begin{longtable}{p{0.62\linewidth} r r}
\caption{Top 10 cited articles.}
\label{tab-5} \\

\toprule
Article & Citation & R \\
\midrule
\endfirsthead

% --- NÃO REPETE CABEÇALHO NA SEGUNDA PÁGINA ---
\endhead

% ---------------- Página 1 ---------------- 
Extending the Technology Acceptance Model (TAM) to Predict University Students’ Intentions to Use Metaverse-Based Learning Platforms & 95 & 1 \\[0.3em]

Virtual World as a Resource for Hybrid Education & 92 & 2 \\[0.3em]

Utilizing the Metaverse for Learner-Centered Constructivist Education in the Post-Pandemic Era: An Analysis of Elementary School Students & 87 & 3 \\[0.3em]

Benefits of Taking a Virtual Field Trip in Immersive Virtual Reality: Evidence for the Immersion Principle in Multimedia Learning & 73 & 4 \\[0.3em]

Constructing an Edu-Metaverse Ecosystem: A New and Innovative Framework & 62 & 5 \\[0.3em]

% ---------------- Página 2 ----------------

Using Augmented Reality to Stimulate Students and Diffuse Escape Game Activities to Larger Audiences & 62 & 6 \\[0.3em]

Metaverse-Powered Experiential Situational English-Teaching Design: An Emotion-Based Analysis Method & 43 & 7 \\[0.3em]

Strategic use of immersive media and narrative message in virtual marketing: Understanding the roles of telepresence and transportation & 43 & 8 \\[0.3em]

Virtual is so real! Consumers’ evaluation of product packaging in virtual reality & 38 & 9 \\[0.3em]

When makers meet the metaverse: Effects of creating NFT Metaverse exhibition in maker education & 36 & 10 \\
\bottomrule
\source{Own elaboration.}
\end{longtable}


\subsection{Citation Analysis by Article}
\Cref{fig-4} illustrates the citations among articles, revealing a notable interaction and collaboration among researchers. This indicates an active engagement in citations. For instance, the article by \textcite{wang2022edumetaverse} recorded 62 citations, while \textcite{suh2022learnercentered} achieved engagement with 87 citations, and \textcite{marquezdiaz2020virtualworld} garnered 92 citations. These results highlight the significance of their work and its impact on other research.

%--- código da figura 4 ---%
\begin{figure}[h!]
\centering
\begin{minipage}{.95\textwidth}
\includegraphics[width=\textwidth]{Imagens/Fig4.png}
\caption{Visualization of Citation by article.}
\label{fig-4}
\source{Own elaboration using VOS viewer.}
\end{minipage}
\end{figure}

\subsection{Citation Analysis by Countries}
In terms of citations across various countries within the field, both China and the United States are crucial centers for citation interactions in research. China had 501 citations across 45 articles, whereas the United States recorded 326 citations from 29 articles. This highlights the significance of these two countries in Metaverse studies in classroom environments. South Korea emerges as a significant contributor, with 249 citations from 29 articles, indicating its significant collaboration with other countries. Conversely, countries like Turkey, Australia, England, Saudi Arabia, Spain, and the UAE exhibit less citations when compared to top countries, suggesting a necessity to improve collaboration between researchers in Metaverse research. \Cref{fig-5} displays the citations across various countries.

%--- código da figura 5 ---%
\begin{figure}[h!]
\centering
\begin{minipage}{.95\textwidth}
\includegraphics[width=\textwidth]{Imagens/Fig5.png}
\caption{Visualization of Citation by Countries.}
\label{fig-5}
\source{Own elaboration using VOS viewer.}
\end{minipage}
\end{figure}


\subsection{Co-occurrence Analysis by words}

%--- código da figura 6 ---%
\begin{figure}[h!]
\centering
\begin{minipage}{.95\textwidth}
\includegraphics[width=\textwidth]{Imagens/Fig6.png}
\caption{Visualization of Word Co-occurrence.}
\label{fig-6}
\source{Own elaboration using VOS viewer.}
\end{minipage}
\end{figure}

Through co-occurrence and frequency analysis of keywords, we can identify the topics most used and researched by scholars by setting a minimum threshold of 50 occurrences, as shown in Figure \ref{fig-6}. Results show a web of related keywords, classified into three categories according to their usage frequency. This research identified a collection of common keywords and categorized them into three groups based on their usage frequency, as illustrated in Figure \ref{fig-7}.

The first group encompasses fundamental Metaverse technologies, including artificial intelligence, virtual reality, augmented reality, and innovative technologies. Virtual reality (VR) is highlighted as the most often mentioned technology, underscoring its significance in improving educational experiences \cite{mikropoulos2011}. The second group focuses on Metaverse applications in education, which include immersive learning, virtual environments, and gamification. Findings indicate that immersive learning is the most frequent keyword, highlighting the importance of engagement in the educational process and the creation of an interactive environment through active understanding and participation from students \cite{dede2009}. The third group focuses on infrastructure and support, which includes blockchain and information technology, indicating the necessity of support for Metaverse applications. Blockchain is notable for its transparency and security in virtual environments, making it an essential element in this context \cite{catalini2016}, as shown in Figure \ref{fig-7}.

%--- código da figura 7 ---%
\begin{figure}[h!]
\centering
\begin{minipage}{.95\textwidth}
\includegraphics[width=\textwidth]{Imagens/Fig7.png}
\caption{Co-Occurrence Analysis of Key Terms in Metaverse Research.}
\label{fig-7}
\source{Own elaboration using VOS viewer.}
\end{minipage}
\end{figure}

\section{Discussion}

The wave of interest in the Metaverse at schools marks a big shift in ethnicity. As digital tools change quickly, the Metaverse opens doors to make tools to have immersive, inclusive, and effective learning experiences that are way outside the box of traditional methods.

The combination of Metaverse technologies, with special emphasis on virtual reality (VR) and augmented reality (AR), makes room for a model of active learning where students interact with the content directly. Such involvement has proven to make comprehension more profound, particularly in scientific disciplines such as chemistry or physics, which are complex—an observation made by \textcite{dede2009}, as well as \textcite{guo2022}, who highlight VR technology in enhancing engagement and emotional connections in virtual learning.

This is opposed to the passive nature of traditional teaching. This is also in line with \textcite{hussain2023}, who argues that the metaverse-based method of teaching fosters cognitive and emotional stimulation, both factors that make a difference to 21st-century learning.

The novel coronavirus was a push factor for the adoption of Metaverse platforms. These, therefore, became immersive alternatives to dull old online learning. The Metaverse platform enabled students to take part in a lesson as they would in a physical classroom, where they could interact with peers and get real-time feedback. This is what is currently being witnessed following the findings of \textcite{daniel2020} and \textcite{colomomagana2021} that called for the need for social presence and interaction platforms during emergency shifts in education.

Mapping the Metaverse in the classroom requires a comprehensive strategy to incorporate virtual environments into the learning experience. This strategic approach starts with a comprehensive understanding of learning objectives, curriculum requirements, and the particular needs of the learners. Educators should carefully choose suitable Metaverse platforms and tools that match these objectives, taking into consideration factors like accessibility, usability, and the presence of pertinent content \cite{guo2022}. The subsequent step includes designing or producing captivating and interactive learning experiences, such as virtual explorations of historical locations, simulations of scientific experiments, or team projects that promote collaboration and problem-solving. Moreover, successful mapping entails setting defined guidelines for student involvement, ensuring a secure and inclusive online environment \cite{suh2022learnercentered}. This includes giving students clear instructions, supplying continuous assistance and feedback, and also observing their interactions to promote better engagement. The mapping process also includes creating \cite{chen2023} assessment strategies that evaluate student learning accurately, combining virtual and real-life activities to assess comprehension.

This strategic approach ensures that the virtual learning environment is not only engaging but also aligned with educational goals, ultimately maximizing the benefits of this innovative technology. Through careful planning of the Metaverse experience, educators can develop a vibrant and impactful learning atmosphere that boosts student engagement, encourages critical thinking abilities, and cultivates a more profound comprehension of the subject. To conclude, in order to effectively integrate the Metaverse in education, it necessitates a well-defined mapping process that includes learning goals, selection of a platform, content development, student engagement guidelines, and assessment methods. This strategic method guarantees that the online learning environment is both engaging and in line with educational objectives, eventually optimizing the benefits of this cutting-edge technology. Interestingly, what \textcite{marquezdiaz2020virtualworld} said earlier, that hybrid learning could be better facilitated through the virtual world platforms, is what we are saying now, indicating a convergence in findings. One major advantage of the Metaverse is its ability to bridge geographic divides, allowing access to learners in remote or underserved regions. This inclusivity supports \textcite{dwivedi-etal2022}, who say that the Metaverse fosters digital equity by enabling borderless learning and the development of digital competencies.

While most researchers have tried to answer the question of what an ideal Metaverse platform should be in the field of education, some researchers have evaluated the existing leading platforms in the Metaverse. For example, it has been argued that transforming the Second Life platform into an educational platform with features such as graphic richness, high level of interaction, creative possibilities and networking can contribute to students' learning levels \cite{marquezdiaz2011virtualworld}. In another study, the innovative design and security advantage of the Vortex platform could host its use in the education field \cite{jovanovic2022vortex}. User evaluations of existing platforms are very important for the development and further enrichment of Metaverse platforms that will carry out virtual education activities. In this context, testing of existing platforms will also guide future studies. The Metaverse is also thought to help students in terms of communication and teamwork \cite{gomes2013secondlife}. Metaverse technologies will contribute to the development of interpersonal social relations by providing people with opportunities such as planning, acting, and interacting. The Metaverse will lead to social and cultural interaction experiences, and it will go beyond being just a game or social media platform: it will exist in a complementary way with the real world \cite{park2022gameful}.

The Metaverse also helps creativity and self-expression, as learners use Metaverse tools to build their content, encouraging critical thinking and innovation. These observations are similar to those in \textcite{asiksoy2023}, who emphasized the constructivist potential of immersive educational technologies.

The strength of China in the exploration of the Metaverse is due to very strong support from the government, advanced digital infrastructure, and prioritization of future technologies within education. These advantages, both institutional and infrastructural, have led to China producing the highest volume of research in this area, as shown in the findings of \textcite{chen2023} and \textcite{bizel2023} from their bibliometric analysis.

Some other countries that invest in such technologies are the U.S. and South Korea. They also are at earlier stages in the incorporation and implementation of such technologies. The interactive model between these countries, as evidenced in \textcite{anuj2024} and \textcite{balat2023}, reflects a new global circuit of knowledge flow and creativity in Metaverse education.

A few factors that can alter the prominence and effect of Metaverse research in education are:

\begin{itemize}
    \item General and fundamental theories, like versions of the Technology Acceptance Model, attract more citations (see \textcite{donthu2021}).
\item Niche or derivative works have lower referencing except for being more explicit in explaining new ideas.
\item Publication time, especially in periods of worldwide changes like the pandemic, increases citation effects, as evidenced by the results of \textcite{gonzalezalcaide2022}.
\item Interdisciplinary studies linking the use of technology tools with teaching also provide insights into effective educational practices.
\end{itemize}

The significant rise in publications related to the Metaverse subject from 2008 to 2024 can be linked to several factors. Firstly, the post-pandemic era expedited investments in immersive technologies as educational institutions and governments aimed for sustainable digital models for education. Secondly, the increase of support from governments, such as in China, alongside global collaborations, generated momentum for extensive research programs. Lastly, the priority of international policy reports and funding programs is digital equity and immersive learning, encouraging researchers to align their work with these objectives Furthermore, patterns of keyword co-occurrence emphasize the thematic focus of Metaverse research. Terms like ``classroom'', ``learning'', ``teaching'', and ``curriculum'' are commonly used, showcasing a significant focus on pedagogical innovation. Simultaneously, terms such as ``interaction'', ``equity'', and ``assessment'' indicate a growing emphasis on social presence, inclusive access, and innovative evaluation methods. This shows that research in this field is not only assessing the feasibility of technology but also tackling educational effectiveness and equity in virtual settings.

This thematic orientation also varies across countries, where differences in infrastructure, funding, and international collaborations shape the visibility and citation impact of Metaverse research.

Countries with robust infrastructure and international linkages, such as China and the U.S., enjoy higher visibility and citation. Whereas it is a process of countries like Turkey and the UAE to bring their ecosystems and collaborations up to the mark of the world.

\section{Conclusion}
Utilizing the Metaverse in teaching is crucial, as it can change conventional educational settings into engaging and effective experiences via technologies like virtual reality (VR) and augmented reality (AR). Incorporating these technologies in classrooms allows students to engage with educational material more actively, improving their understanding of the material and making learning more engaging and authentic. This study addressed significant inquiries concerning the utilization of the Metaverse in education and the trends noted in academic publications. The bibliometric analysis showed an increase in research production from 2008 to 2024, concentrating on interactive and experiential learning, emphasizing the countries and institutions that are most engaged in this field. Applying the Metaverse in classroom settings has numerous benefits, including fostering creativity, facilitating effective interactions between students and teachers in virtual settings, and enhancing remote learning, particularly during emergencies like the COVID-19 pandemic. The Metaverse can enhance remote educational accessibility in under-resourced regions, promoting inclusivity in the education process. The study, however, also pointed out various challenges, such as the significant initial expenses for infrastructure, hardware, and software, which could worsen digital equity concerns if not managed carefully. Teachers need thorough training to incorporate Metaverse tools into their lessons efficiently and address the specific educational factors to take into consideration in virtual environments. Ongoing research is essential to create educational resources, maintain digital security, and secure personal data. The analysis indicates that researchers can find opportunities to investigate less-studied fields, including interdisciplinary uses, remote learning settings, and online evaluations. Teachers are urged to incorporate metaverse-based approaches to enhance critical thinking, problem-solving, and engagement. Policymakers should allocate resources to infrastructure, guarantee fair and equitable access to technology, and establish policies that promote safe, secure and efficient online education. To summarize, the Metaverse holds great possibilities for changing education into an engaging, interactive, inclusive, and thorough experience. To harness this potential, it is essential to engage in meticulous planning, concentrated teacher training, and sustained support from institutions and policies, to ensure that every student benefits from metaverse-based learning.

\section{Limitations}
This study emphasizes various limitations and offers recommendations for upcoming studies regarding the integration of the Metaverse into educational environments. First of all, there is a significant lack of experimental research exploring practical applications and the effects of the Metaverse in classrooms. Although the possibility of the Metaverse to improve educational experiences is broadly recognized, further research is required to explore its practical application and effectiveness across various learning environments.

Study-specific limitations: This research depended exclusively on the Web of Science (WoS) database, which could have excluded relevant publications indexed elsewhere in other databases, possibly limiting the overall comprehensiveness of the results. Moreover, the selection of keywords might have caused bias, influencing the studies to be retrieved. Additionally, the analysis didn’t include qualitative coding, which might have yielded more profound insights into the thematic and conceptual trends in the literature.

Limitations associated with implementing the Metaverse: Challenges encompass technical difficulties like low resolution and accessibility issues, high expenses, and educational concerns about successful teaching methods. Privacy, security, and ethical concerns also necessitate thoughtful consideration.

Future research recommendations: Upcoming studies should aim to expand the database sources beyond WoS, utilize wider and neatly designed keywords, and incorporate qualitative analysis to gain deeper insights. Researchers must also concentrate on enhancing technical infrastructure, improving teacher training, and developing inclusive solutions. It is essential to explore effective teaching strategies, alongside considerations for privacy and security, as well as the social and ethical implications of employing the Metaverse in education. For instance, interactive simulations can enable students to explore sophisticated concepts in biology, chemistry, and physics, which can benefit Science education.

To fill the current gaps, several studies should explore the realities and possibilities of implementing the Metaverse in various educational stages and settings. Policymakers, including Ministries of Education, ought to promote the integration of Metaverse technologies in teaching practices by developing conducive policies and frameworks. Involvement of educators and decision-makers in seminars and conferences centered on the Metaverse is crucial to raise awareness and reinforce best practices.

Finally, the bibliometric analysis showed that studies on the Metaverse in education have been present since 2008. Although the publication count was low at first, it has risen significantly in recent years. China became the top contributor with 45 publications, representing about 23\% of the total research, underscoring the need for enhanced international research endeavors to thoroughly explore the possibilities of the Metaverse in education.
\printbibliography\label{sec-bib}
% if the text is not in Portuguese, it might be necessary to use the code below instead to print the correct ABNT abbreviations [s.n.], [s.l.]
%\begin{portuguese}
%\printbibliography[title={Bibliography}]
%\end{portuguese}


%full list: conceptualization,datacuration,formalanalysis,funding,investigation,methodology,projadm,resources,software,supervision,validation,visualization,writing,review
\begin{contributors}[sec-contributors]
\authorcontribution{Amirah Alzahrani}
[conceptualization,methodology,formalanalysis, writing,visualization]
\authorcontribution{Tahani Al-Malaji}[datacuration,review]
\end{contributors}

\begin{dataavailability}
\txtdataavailability{nodata} % options: dataavailable, dataonly, databody, datanotav, nodata
\end{dataavailability}



\end{document}

