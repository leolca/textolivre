% !TEX TS-program = XeLaTeX
% use the following command:
% all document files must be coded in UTF-8
\documentclass[portuguese]{textolivre}
% build HTML with: make4ht -e build.lua -c textolivre.cfg -x -u article "fn-in,svg,pic-align"

\journalname{Texto Livre}
\thevolume{19}
%\thenumber{1} % old template
\theyear{2026}
\receiveddate{\DTMdisplaydate{2025}{5}{27}{-1}} % YYYY MM DD
\accepteddate{\DTMdisplaydate{2025}{9}{3}{-1}}
\publisheddate{\DTMdisplaydate{2025}{11}{25}{-1}}
\corrauthor{Suellen Cristina Rodrigues Kotz}
\articledoi{10.1590/1983-3652.2026.59368}
%\articleid{NNNN} % if the article ID is not the last 5 numbers of its DOI, provide it using \articleid{} commmand 
% list of available sesscions in the journal: articles, dossier, reports, essays, reviews, interviews, editorial
\articlesessionname{articles}
\runningauthor{Kotz, Sobrinho e Rodrigues} 
%\editorname{Leonardo Araújo} % old template
\sectioneditorname{Daniervelin Pereira~\orcid{0000-0003-1861-3609}}
\layouteditorname{Saula Cecília~\orcid{0009-0006-3069-8480}}

\title{Comunicação científica e engajamento \textit{online}: análise da revista Science no Instagram e Facebook}
\othertitle{Science communication and online engagement: analysis of the journal Science on Instagram and Facebook}
% if there is a third language title, add here:
%\othertitle{Artikelvorlage zur Einreichung beim Texto Livre Journal}

\author[1]{Suellen Cristina Rodrigues Kotz~\orcid{0000-0002-8987-7065}\thanks{Email: \href{mailto:suellencrp5@gmail.com}{suellencrp5@gmail.com}}}
\author[1]{Asdrúbal Borges Formiga Sobrinho~\orcid{0000-0002-3213-4498}\thanks{Email: \href{mailto:asdru_bal@uol.com.br}{asdru\_bal@uol.com.br}}}
\author[1]{Marina Silva Bicalho Rodrigues~\orcid{0000-0001-7636-2479}\thanks{Email: \href{mailto:maribicalho@gmail.com}{maribicalho@gmail.com}}}
\affil[1]{Universidade de Brasília, Departamento de Psicologia, Brasília, DF, Brasil.}

\addbibresource{article.bib}
% use biber instead of bibtex
% $ biber article

% used to create dummy text for the template file
\definecolor{dark-gray}{gray}{0.35} % color used to display dummy texts
\usepackage{lipsum}
\SetLipsumParListSurrounders{\colorlet{oldcolor}{.}\color{dark-gray}}{\color{oldcolor}}

% used here only to provide the XeLaTeX and BibTeX logos
\usepackage{hologo}

% if you use multirows in a table, include the multirow package
\usepackage{multirow}

% provides sidewaysfigure environment
\usepackage{rotating}

% CUSTOM EPIGRAPH - BEGIN 
%%% https://tex.stackexchange.com/questions/193178/specific-epigraph-style
\usepackage{epigraph}
\renewcommand\textflush{flushright}
\makeatletter
\newlength\epitextskip
\pretocmd{\@epitext}{\em}{}{}
\apptocmd{\@epitext}{\em}{}{}
\patchcmd{\epigraph}{\@epitext{#1}\\}{\@epitext{#1}\\[\epitextskip]}{}{}
\makeatother
\setlength\epigraphrule{0pt}
\setlength\epitextskip{0.5ex}
\setlength\epigraphwidth{.7\textwidth}
% CUSTOM EPIGRAPH - END

% to use IPA symbols in unicode add
%\usepackage{fontspec}
%\newfontfamily\ipafont{CMU Serif}
%\newcommand{\ipa}[1]{{\ipafont #1}}
% and in the text you may use the \ipa{...} command passing the symbols in unicode

% LANGUAGE - BEGIN
% ARABIC
% for languages that use special fonts, you must provide the typeface that will be used
% \setotherlanguage{arabic}
% \newfontfamily\arabicfont[Script=Arabic]{Amiri}
% \newfontfamily\arabicfontsf[Script=Arabic]{Amiri}
% \newfontfamily\arabicfonttt[Script=Arabic]{Amiri}
%
% in the article, to add arabic text use: \textlang{arabic}{ ... }
%
% RUSSIAN
% for russian text we also need to define fonts with support for Cyrillic script
% \usepackage{fontspec}
% \setotherlanguage{russian}
% \newfontfamily\cyrillicfont{Times New Roman}
% \newfontfamily\cyrillicfontsf{Times New Roman}[Script=Cyrillic]
% \newfontfamily\cyrillicfonttt{Times New Roman}[Script=Cyrillic]
%
% in the text use \begin{russian} ... \end{russian}
% LANGUAGE - END

% EMOJIS - BEGIN
% to use emoticons in your manuscript
% https://stackoverflow.com/questions/190145/how-to-insert-emoticons-in-latex/57076064
% using font Symbola, which has full support
% the font may be downloaded at:
% https://dn-works.com/ufas/
% add to preamble:
% \newfontfamily\Symbola{Symbola}
% in the text use:
% {\Symbola }
% EMOJIS - END

% LABEL REFERENCE TO DESCRIPTIVE LIST - BEGIN
% reference itens in a descriptive list using their labels instead of numbers
% insert the code below in the preambule:
%\makeatletter
%\let\orgdescriptionlabel\descriptionlabel
%\renewcommand*{\descriptionlabel}[1]{%
%  \let\orglabel\label
%  \let\label\@gobble
%  \phantomsection
%  \edef\@currentlabel{#1\unskip}%
%  \let\label\orglabel
%  \orgdescriptionlabel{#1}%
%}
%\makeatother
%
% in your document, use as illustraded here:
%\begin{description}
%  \item[first\label{itm1}] this is only an example;
%  % ...  add more items
%\end{description}
% LABEL REFERENCE TO DESCRIPTIVE LIST - END


% add line numbers for submission
%\usepackage{lineno}
%\linenumbers

\begin{document}
\maketitle

\begin{polyabstract}
\begin{abstract}
Nas últimas décadas, as mídias sociais transformaram o acesso à informação e a maneira como ela é compartilhada. Com isso, tornaram-se ferramentas essenciais para a comunicação científica, ao ampliar o alcance de conteúdos e promover interações. Este estudo teve como objetivo analisar as estratégias de comunicação da revista Science em suas páginas no Instagram e no Facebook, comparar métricas de curtidas, comentários e compartilhamentos, e investigar se os formatos de postagem e os temas influenciam a interação. A pesquisa, quantitativa, descritiva e exploratória, analisou postagens publicadas por um período de 60 dias. Os dados foram coletados manualmente e submetidos a análises descritivas e inferenciais (Mann-Whitney $U$ e Kruskal-Wallis). Os resultados apontaram diferenças entre as plataformas, com maior engajamento no Instagram, mas sem influência significativa de formatos ou temas. Conclui-se que estratégias de comunicação devem considerar as especificidades de engajamento entre mídias sociais.

\keywords{Comunicação científica\sep Mídias sociais\sep Engajamento online\sep Ciência e tecnologia}
\end{abstract}

\begin{english}
\begin{abstract}
Over the past decades, social media has transformed access to information and the ways in which it is shared. As a result, these platforms have become essential tools for science communication, expanding the reach of content and fostering interaction. This study aimed to analyze the communication strategies employed by Science magazine on its official Instagram and Facebook pages, compare engagement metrics (likes, comments, and shares), and investigate whether post formats and themes influenced audience interaction. This quantitative, descriptive, and exploratory research examined posts published over a 60-day period. Data were collected manually and subjected to descriptive and inferential statistical analyses (Mann-Whitney $U$ and Kruskal-Wallis tests). The results indicated differences between the platforms, with higher engagement on Instagram, but no significant influence of post formats or themes. It is concluded that communication strategies should take into account the specificities of engagement across social media platforms.

\keywords{Science communication\sep Social media\sep Online engagement\sep Science and technology}
\end{abstract}
\end{english}
% if there is another abstract, insert it here using the same scheme
\end{polyabstract}

\section{Introdução}\label{sec-intro}
Nos últimos anos, o avanço das tecnologias tem transformado a maneira como o conhecimento é produzido, disseminado e acessado. Nesse cenário, a educação digital surge como uma parceira essencial da comunicação científica, ao fornecer recursos que ampliam as possibilidades de interação, engajamento e compreensão do público \cite{bates2015, dong2020, kirkwood2014}. Essa sinergia permite superar barreiras tradicionais e, assim, promover um acesso mais inclusivo e equitativo ao conhecimento. Permite também fortalecer a cultura científica, uma vez que a participação pública na ciência pode contribuir para a compreensão dos métodos científicos \cite{bonney2009, collins2022}.

A comunicação científica exerce um papel fundamental na disseminação do saber, ao promover o pensamento crítico e o desenvolvimento em diferentes contextos sociais, educacionais e culturais \cite{bauer2011, hungaro2024}. Neste artigo, a expressão ``comunicação científica'' é utilizada para se referir à produção, circulação e apropriação de informações em ambientes digitais. Tal abordagem contempla práticas institucionais de difusão do conhecimento e se distingue, por exemplo, do jornalismo científico.

No entanto, os conceitos de comunicação e divulgação científica são frequentemente confundidos, o que torna necessário diferenciá-los. Enquanto a primeira está mais associada à circulação do conhecimento no âmbito acadêmico, a segunda busca traduzir esse saber para o público geral, com o objetivo de ampliar o acesso e, ao mesmo tempo, incentivar a curiosidade pública, utilizando uma linguagem acessível \cite{ferreira2023, gregory1998}. Apesar das diferenças, há um aspecto comum, no qual certas práticas podem incorporar elementos de ambas. Por exemplo, periódicos científicos, ao utilizar as plataformas de mídias sociais, adaptam o conteúdo para formatos curtos e visuais. Essa abordagem combina características da comunicação (precisão e linguagem técnica) com elementos da divulgação (simplicidade e apelo visual). Essa convergência é especialmente comum entre entidades que pretendem alcançar tanto públicos especializados, quanto leigos \cite{carvalho2023, trench2008}. Nesse caso, ambas se fazem necessárias \cite{bucchi2021}.

A prática da comunicação científica envolve um conjunto de habilidades, como a capacidade de simplificar termos técnicos, elaborar narrativas atrativas, adaptar conteúdos a diferentes formatos e dialogar com diferentes audiências. Porém, muitas dessas competências nem sempre estão desenvolvidas ou disponíveis quando se busca comunicar ciência em ambientes digitais \cite{baramtsabari2017, bucchi2021}. Esses desafios são intensificados pela necessidade de competir por atenção em um ambiente midiático saturado, dinâmico e cada vez mais complexo, dificultando que conteúdos científicos se destaquem e alcancem o público de forma eficaz \cite{scheufele2019}.

Nesse contexto, pesquisas são necessárias para identificar estruturas e processos que permitam engajar o público em maior escala. O potencial de engajamento online, em particular, merece atenção \cite{davies2009, franciscojunior2024}, principalmente, porque a demanda por estratégias criativas se torna cada vez mais urgente diante de questões de grande impacto social, econômico e global \cite{corner2010}. As mídias sociais oferecem oportunidades para expandir e complementar os métodos tradicionais de comunicação da ciência e permitem que o conhecimento seja compartilhado de forma acessível e criativa.

Diante dessas transformações, a revista Science tem aliado prestígio acadêmico à presença digital. É reconhecida mundialmente por sua rigorosa revisão por pares e pela difusão de descobertas de alto impacto em múltiplas áreas do conhecimento. Além de sua atuação no meio acadêmico tradicional, tem ampliado sua visibilidade em plataformas digitais, por meio das quais divulga de forma acessível os resultados de pesquisas.

Desse modo, este estudo teve como objetivo analisar as estratégias de comunicação da revista Science, em suas páginas no Instagram e no Facebook, bem como comparar métricas de curtidas, comentários e compartilhamentos, e investigar se os formatos de postagem e os temas influenciam a interação do público. Para tanto, desenvolveu-se uma pesquisa quantitativa, de caráter descritivo e exploratório, com base na análise de postagens coletadas e submetidas a procedimentos estatísticos descritivos e inferenciais. O texto organiza-se a partir da revisão teórica, seguida da descrição dos procedimentos metodológicos e, por fim, da análise e discussão dos resultados, que sintetizam os principais achados e suas contribuições para o campo.


\section{Comunicação científica e criatividade}
O rigor na circulação do conhecimento científico exige formas específicas de transmissão, geralmente restritas a contextos formais e especializados. Nesse sentido, fala-se em comunicação científica, entendida como a troca de informações entre pesquisadores, acadêmicos e outros profissionais, sustentada por uma linguagem técnica e estruturada \cite{bucchi2021}. Exemplos comuns incluem artigos acadêmicos, conferências e relatórios técnicos, nos quais o objetivo principal é promover o avanço do conhecimento por meio do compartilhamento de resultados de pesquisas, possibilitando o desenvolvimento de novas teorias e práticas \cite{falk2005}. Essa abordagem geralmente ocorre em canais especializados, como periódicos, que operam sob padrões rigorosos de revisão por pares.

Contudo, os métodos tradicionais frequentemente apresentam barreiras para o público leigo ou menos especializado. Nesse sentido, o estilo de apresentação constitui-se um desafio, sugerindo a necessidade de uma abordagem inovadora e adaptada \cite{bucchi2009, oliveira2024}. Além disso, a comunicação científica no contexto contemporâneo é bidirecional, dinâmica e ocorre em ambientes complexos. Ela envolve uma rede de indivíduos, grupos e organizações que atuam simultaneamente como emissores e receptores. Esse caráter multidirecional amplia as possibilidades de diálogo e participação social, mas, conforme destacam \textcite{brossard2013}, também impõe desafios para a eficácia da comunicação. Sobretudo no que se refere à credibilidade das fontes, à fragmentação dos públicos e à circulação de informações contraditórias.

Complementando essa perspectiva, \textcite{fischhoff2013} enfatiza a necessidade de uma abordagem interdisciplinar, considerando não apenas o conteúdo, mas também o formato, o emissor e o contexto. Ele ressalta a influência da comunicação científica nas decisões políticas, sociais e individuais e destaca sua relevância em um mundo onde a compreensão pública da ciência é cada vez mais necessária. Autores como \textcite{burns2003} e \textcite{ferreira2023} sinalizam que, para ser eficaz, esse tipo de prática vai além da transmissão de fatos, requerendo inovação e uma abordagem centrada no público. Os autores criticam o modelo linear, em que os cientistas são os emissores e o público é apenas receptor, e propõem um modelo mais dinâmico e interativo, no qual o público é um participante ativo no processo, ou seja, cocriador do conhecimento. Portanto, o sucesso da comunicação científica depende de estratégias que equilibrem informações precisas com engajamento e relevância contextual \cite{mckinnon2015}.

Nesse contexto, a criatividade torna-se fundamental para a construção de abordagens comunicativas eficazes, capazes de despertar interesse e promover conexões com diversos públicos. \textcite{lubart2007} e \textcite{lubart2019} definem a criatividade como um fenômeno multidimensional, resultante da interação entre fatores cognitivos, conativos, afetivos e ambientais. De forma complementar, \textcite{glaveanu2013} enfatiza uma abordagem sociocultural, ao apontar a criatividade como um processo relacional, no qual indivíduos, objetos e contextos interagem, de maneira dinâmica. Essa perspectiva é especialmente relevante quando se considera a comunicação científica nas mídias sociais. Ambientes nos quais o potencial de comunicabilidade da ciência se beneficia de estratégias colaborativas, que envolvem instituições, tecnologia, o público e cientistas que produzem conhecimentos com base em suas trajetórias, crenças, vivências profissionais e criatividade \cite{kotz2023a, kotz2024}. Trata-se de uma troca bidirecional, em que a criatividade se manifesta como uma construção coletiva e cultural. Assim, o uso criativo das mídias sociais não se limita à estética ou ao formato, mas visa tornar a linguagem científica mais acessível, fortalecer o vínculo com audiências mais amplas e promover o engajamento de forma significativa e democrática.

\section{O impacto das mídias sociais na comunicação científica}
A utilização de recursos tecnológicos tem se mostrado cada vez mais relevante para a aprendizagem e para a criatividade na ciência, especialmente em contextos acadêmicos \cite{kotz2023b}. Quando integrada às mídias sociais, pode potencializar a comunicação de conteúdos científicos e o engajamento de públicos diversos. Estudos como os de \textcite{dwivedi2021}, por exemplo, destacam as oportunidades que a Inteligência Artificial (IA) oferece para personalizar conteúdos e aumentar o engajamento, baseando-se em interesses ou histórico de interação.

Plataformas como Twitter (atualmente X),\footnote{O termo Twitter é mantido conforme aparece no texto consultado, ainda que a plataforma tenha sido posteriormente renomeada para X.} Instagram, YouTube e TikTok possibilitam que o conteúdo científico seja adaptado a diferentes formatos, como vídeos curtos, infográficos, \textit{podcasts} e transmissões ao vivo \cite{bik2013, priem-costello2010,welmer2024}. O uso de mídias sociais, embora ainda enfrente certo estigma no meio acadêmico, vem sendo reconhecido como ferramenta legítima para comunicar ciência \cite{sugimoto2017}. Quando utilizadas de forma estratégica, não apenas aproximam a ciência da sociedade, mas podem fortalecer a carreira de pesquisadores ao facilitar colaborações e o engajamento com diferentes audiências.

O conceito de engajamento \textit{online}, discutido em áreas como \textit{marketing digital}, tem se consolidado como um dos principais operadores analíticos para compreender a ocorrência dessas interações \cite{brodie2011}. De forma geral, refere-se ao grau de envolvimento, participação e interação dos indivíduos com conteúdos, plataformas ou comunidades, abrangendo dimensões cognitivas, emocionais e comportamentais. Do ponto de vista operacional, o engajamento costuma ser associado a métricas de interação, como curtidas, comentários, compartilhamentos e visualizações. Estes, geralmente, funcionam como indicadores do nível de resposta do público em relação a determinada mensagem \cite{pavelle2020}.

No campo da comunicação científica, o engajamento \textit{online} representa não apenas a responsividade imediata do público às postagens, mas também o potencial de aproximar ciência e sociedade \cite{priem-costello2010, bik2013}. No entanto, esse processo enfrenta lacunas relevantes, como as diferenças nas métricas de engajamento entre plataformas e a própria transitoriedade dos conteúdos digitais. Essas questões tornam-se evidentes na pesquisa de \textcite{rezende2023} que analisou o engajamento de postagens realizadas no Instagram, Facebook e Twitter, por periódicos científicos com avaliação Qualis A1 e A2. Os resultados mostram que, embora o Facebook tenha maior número de seguidores, o Instagram e o Twitter apresentaram percentuais mais elevados de interações.

A pesquisa de \textcite{oliveira2019} investigou a presença e circulação \textit{online} de revistas da área de Comunicação e Informação, mapeando sua atuação em redes sociais como Facebook, Twitter, YouTube e Instagram. Os resultados apontaram para uma baixa presença dessas revistas nas plataformas, com circulação da produção científica mais relacionada à divulgação por pesquisadores individuais do que por perfis institucionais dos periódicos. Por outro lado, \textcite{ferreira2023} realizaram uma pesquisa com o objetivo de discutir a importância da popularização do conhecimento científico através das redes sociais e o conteúdo veiculado nesses ambientes. Os resultados revelaram que as redes sociais simbolizam uma nova dimensão da democratização da ciência, regida pelas tecnologias digitais, em que o diálogo entre cientistas e a popularização de conteúdo podem ser praticados simultaneamente, no mesmo espaço.

Da mesma forma, \textcite{delbianco2021} investigaram como os avanços tecnológicos e o uso de mídias sociais influenciaram os estudos métricos da informação. Os resultados mostraram que os avanços tecnológicos podem ser utilizados tanto de forma positiva, quanto negativa. Mas, de um modo geral, ajudam no desenvolvimento da ciência e no compartilhamento da informação, porque os canais foram significativamente ampliados.

Esses estudos apresentam pesquisas focadas na análise de estratégias de comunicação científica em mídias sociais, cada uma com perspectivas, objetivos e contextos específicos. Porém, o presente estudo busca contribuir ao analisar as postagens da revista Science em suas páginas oficiais do Instagram e Facebook, de forma a compreender como distintas estratégias podem impactar a interação com as audiências.


\section{Metodologia}
O estudo é de natureza quantitativa, descritiva e exploratória, com abordagem não experimental e coleta de dados observacionais. A amostra foi composta por postagens da revista Science, extraídas de suas páginas oficiais no Instagram e no Facebook, por um período de 60 dias. Foram considerados critérios de inclusão postagens dos últimos seis meses, relacionadas às pesquisas desenvolvidas, com a exclusão de conteúdos como entrevistas, divulgação de eventos e ou outros que não estivessem de acordo com os critérios de inclusão. As postagens analisadas deveriam conter dados completos sobre número de curtidas, comentários e compartilhamentos. A amostra final foi composta por aproximadamente 116 postagens.

As variáveis analisadas incluíram métricas de engajamento (número de curtidas, comentários e compartilhamentos), características das postagens, como formato (posts de imagens estáticas e Reels de vídeos curtos) e temas abordados (classificados em categorias como biologia, medicina, tecnologia, entre outros) e a rede social utilizada (Instagram ou Facebook). A coleta de dados foi realizada manualmente, devido às limitações técnicas das plataformas. Esse procedimento permitiu a conferência individual das postagens, minimizando perdas de informação e assegurando a fidelidade ao conteúdo original. Os dados foram inicialmente organizados em uma planilha Excel e, em seguida, transferidos para o software estatístico Statistical Package for the Social Sciences (SPSS), versão 26, para análise. Os dados utilizados neste estudo foram depositados e estão disponíveis publicamente no repositório Figshare.

A análise de dados incluiu estatística descritiva para calcular média, desvio padrão e amplitude das métricas de engajamento por rede social, formato de postagem e tema. Testes de normalidade, como Shapiro-Wilk ou Kolmogorov-Smirnov, foram realizados para verificar a distribuição das variáveis. Para as comparações entre redes sociais, foi aplicado o teste $U$ de Mann-Whitney, a fim de avaliar diferenças nas métricas de engajamento entre Instagram e Facebook. O Kruskal-Wallis foi utilizado para avaliar as categorias de áreas de conhecimento em relação ao engajamento.

\section{Resultados e discussão}
A análise teve como objetivo avaliar as métricas de engajamento em diferentes categorias, verificando normalidade dos dados e comparações estatísticas entre os grupos. A Tabela \ref{tab-1} apresenta a análise descritiva entre Instagram e Facebook. 

%--- codigo da tabela 1 ---%
\begin{table}[h!]
\centering
\small
\begin{threeparttable}
\caption{Estatística descritiva.}\label{tab-1}
\begin{tabular}{p{1.5cm} p{5.5cm} p{5.5cm}}
\toprule
Métrica & Instagram & Facebook \\
\midrule
Média & Curtidas: 3132,05 $(\pm1427,72)$ & Curtidas: 449,73 $(\pm45,04)$ \\
      & Comentários: 35,02 $(\pm23,46)$ & Comentários: 40,90 $(\pm8,60)$ \\
      & Compartilhamentos: 674,66 $(\pm445,29)$ & Compartilhamentos: 171,32 $(\pm24,29)$ \\ 
[4pt]
Mediana & Curtidas: 1279,00 & Curtidas: 320,50 \\
        & Comentários: 7,00 & Comentários: 15,50 \\
        & Compartilhamentos: 92,00 & Compartilhamentos: 95,00 \\
[4pt]
Máximo & Curtidas: 80711,00 & Curtidas: 2000,00 \\
       & Comentários: 1320,00 & Comentários: 350,00 \\
       & Compartilhamentos: 25000,00 & Compartilhamentos: 875,00 \\
\bottomrule
\end{tabular}
\source{Análise realizada no SPSS com dados do Instagram e Facebook – Revista Science.}
\end{threeparttable}
\end{table}


Para o número de curtidas, o Instagram apresentou uma média maior $(3132,05 \pm 1427,72)$ do que o Facebook $(449,73 \pm 45,04)$, assim como uma mediana mais elevada (1279,00 contra 320,50) e um número máximo de curtidas (80711,00 no Instagram e 2000,00 no Facebook), indicando maior engajamento médio e picos mais altos na plataforma. Quanto ao número de comentários, o Facebook teve uma média ligeiramente superior $(40,90 \pm 8,60)$ em comparação ao Instagram $(35,02 \pm 23,46)$, além de uma mediana maior (15,50 contra 7,00). Para os compartilhamentos, o Instagram apresentou uma média mais alta $(674,66 \pm 445,29$ \textit{versus} $171,32 \pm 24,29)$, enquanto as medianas foram semelhantes entre as plataformas (92,00 no Instagram e 95,00 no Facebook). No entanto, o número máximo de compartilhamentos foi muito superior no Instagram (25,000 contra 875,00 no Facebook). Esses resultados indicam que o Instagram se destaca em curtidas e compartilhamentos, enquanto o Facebook apresenta maior consistência em comentários.

Antes de realizar as comparações entre os grupos investigados, foi avaliada a normalidade dos dados para as variáveis número de curtidas, número de comentários e número de compartilhamentos, utilizando os testes de Kolmogorov-Smirnov e Shapiro-Wilk. Para todas as categorias analisadas (formato de postagem, áreas de conhecimento e rede social), os valores de significância $(p)$ foram inferiores a .05, indicando que os dados não seguem uma distribuição normal.

Diante disso, foram utilizados testes não paramétricos para as comparações: o teste de Mann-Whitney $U$ foi aplicado para comparar dois grupos (formato de postagem: \textit{post} e Reels; rede social: Instagram e Facebook), enquanto o teste de Kruskal-Wallis foi utilizado para categorias com três ou mais grupos (áreas de conhecimento). Além disso, a análise de postos médios foi conduzida para identificar tendências entre os grupos investigados.

Com relação às redes sociais analisadas, a Tabela \ref{tab-2} apresenta os dados relacionados ao engajamento de cada rede.

%--- código da tabela 2 ---%
\begin{table}[h!]
\centering
\small
\begin{threeparttable}
\caption{Teste de hipótese.}\label{tab-2}
\begin{tabular}{p{4cm} p{4cm} p{2cm} p{3cm}}
\toprule
Hipótese Nula & Teste & Valor de $p$. & Decisão \\
\midrule
A distribuição de n.º de curtidas é a mesma entre as categorias de redes sociais. & Teste $U$ de Mann-Whitney de amostras independentes. & .000 & Rejeitar a hipótese nula. \\
    A distribuição de n.º de comentários é a mesma entre as categorias de redes sociais. & Teste $U$ de Mann-Whitney de amostras independentes. & .002 & Rejeitar a hipótese nula. \\
    A distribuição de n.º de compartilhamentos é a mesma entre as categorias de redes sociais. & Teste $U$ de Mann-Whitney de amostras independentes. & .719 & Reter a hipótese nula. \\
\bottomrule
\end{tabular}
\source{Análise realizada no SPSS.}
\end{threeparttable}
\end{table}


Os resultados indicaram diferenças significativas para o número de curtidas ($U$ = 404,000, $p$ = .000) e o número de comentários ($U$ = 1047,500, $p$ = .002) entre as redes sociais. O Instagram apresentou maior engajamento médio em curtidas, enquanto o Facebook se destacou nos comentários. Para o número de compartilhamentos, não houve diferença significativa entre as redes ($U$ = 1615,000, $p$ = .719), indicando que essa métrica foi semelhante em ambas as plataformas. Esses resultados destacam variações específicas para curtidas e comentários, mas não para compartilhamentos, conforme visualizado na Tabela \ref{tab-3}.

%--- código da tabela 3 ---%
\begin{table}[h!]
\centering
\begin{threeparttable}
\caption{Postos Médios.}\label{tab-3}
\begin{tabular}{llll}
\toprule
Engajamento & Rede social & Nº & Posto Médio \\
\midrule
Curtidas & Instagram & 56,00 & 81,29 \\
         & Facebook  & 60,00 & 37,23 \\
         & Total     & 116,00 & -- \\
[4pt]
Comentários & Instagram & 56,00 & 47,21 \\
            & Facebook  & 60,00 & 69,04 \\
            & Total     & 116,00 & -- \\
[4pt]
Compartilhamentos & Instagram & 56,00 & 57,34 \\
                  & Facebook  & 60,00 & 59,58 \\
                  & Total     & 116,00 & -- \\
\bottomrule
\end{tabular}
\source{Análise realizada no SPSS.}
\end{threeparttable}
\end{table}


Os resultados sinalizaram diferenças no desempenho entre Instagram e Facebook. O Instagram apresentou um posto médio significativamente maior em curtidas (Mdn = 81,29) do que o Facebook (Mdn = 37,23), enquanto o Facebook teve um desempenho superior em comentários (Mdn = 69,04), em comparação ao Instagram (Mdn = 47,21). Quanto aos compartilhamentos, os postos médios foram similares entre as redes sociais (Instagram: Mdn = 57,34; Facebook: Mdn = 59,58), indicando um desempenho equilibrado.

Diante dos resultados, estudos sugerem que as redes sociais têm se consolidado como plataformas essenciais para a comunicação científica, permitindo que conteúdos acadêmicos alcancem públicos amplos, de maneira dinâmica e interativa \cite{rezende2023}. Plataformas como Instagram e Facebook oferecem ferramentas para a disseminação do conhecimento científico, mas operam de maneiras distintas. Enquanto o Instagram é conhecido por sua orientação visual, ao promover interações rápidas, o Facebook se destaca pelo foco em textos e interações discursivas \cite{bik2013, franciscojunior2024}. Essas diferenças são fundamentais para compreender como a escolha da rede social impacta o alcance e a interação do público com conteúdos científicos \cite{kietzmann2011}.

Além disso, as métricas de engajamento são amplamente reconhecidas como indicadores-chave para avaliar o impacto de postagens em redes sociais \cite{pavelle2020}. Segundo \textcite{napoli2011}, enquanto as curtidas refletem interações rápidas e de baixo envolvimento, os comentários indicam maior engajamento discursivo e troca de ideias. Por outro lado, compartilhamentos são vistos como uma métrica qualitativa de disseminação, representando a intenção do público de ampliar o alcance de um conteúdo. No que se refere aos resultados do tipo de postagem (\textit{post} e vídeos curtos -- Reels), a Tabela \ref{tab-4} apresenta resultados de testes de hipótese utilizando o Teste $U$ de Mann-Whitney, avaliando se há diferenças significativas entre categorias de tipo de postagem em relação a três métricas de engajamento.

%--- código da tabela 4 ---%
\begin{table}[h!]
\centering
\begin{threeparttable}
\caption{Teste de hipótese.}\label{tab-4}
\begin{tabular}{p{4cm} p{4cm} p{2cm} p{3cm}}
\toprule
Hipótese Nula & Teste & Valor de $p$. & Decisão \\
\midrule
A distribuição de n.º de curtidas é a mesma entre as categorias de tipo de postagem. & Teste $U$ de Mann-Whitney de amostras independentes. & .133 & Reter a hipótese nula. \\
A distribuição de n.º de comentários é a mesma entre as categorias de tipo de postagem. & Teste $U$ de Mann-Whitney de amostras independentes. & .189 & Reter a hipótese nula. \\
A distribuição de n.º de compartilhamentos é a mesma entre as categorias de tipo de postagem. & Teste $U$ de Mann-Whitney de amostras independentes. & .148 & Reter a hipótese nula. \\
\bottomrule
\end{tabular}
\source{Análise realizada no SPSS.}
\end{threeparttable}
\end{table}


Os resultados mostraram que não houve diferenças estatisticamente significativas entre os formatos de postagem (imagens estáticas e vídeos curtos) para nenhuma das métricas de engajamento analisadas: curtidas ($U$ = 790,500, $p$ = .133), comentários ($U$ = 852,500, $p$ = .189) e compartilhamentos ($U$ = 868,500, $p$ = .148). Esses resultados sugerem que o formato das postagens não exerce influência significativa sobre o engajamento, indicando que o público interage de forma semelhante com os diferentes tipos de postagens, conforme apresentado na Tabela \ref{tab-5}.

%--- código da tabela 5 ---%
\begin{table}[h!]
\centering
\begin{threeparttable}
\caption{Postos Médios.}\label{tab-5}
\begin{tabular}{lllll}
%\begin{tabular}{p{3cm} p{4cm} p{2cm} p{3cm} p{3cm}}
\toprule
Métricas & Formato de Postagem & Nº & Posto Médio & Soma de Classificações \\
\midrule
Curtidas & Post  & 96,00 & 56,73 & 5446,50 \\
         & Vídeo & 20,00 & 66,98 & 1339,50 \\
         & Total & 116,00 &  & -- \\
[4pt]
Comentários & Post  & 96,00 & 57,38 & 5508,50 \\
            & Vídeo & 20,00 & 63,88 & 1277,50 \\
            & Total & 116,00 &  & -- \\
[4pt]
Compartilhamentos & Post  & 96,00 & 57,55 & 5524,50 \\
                  & Vídeo & 20,00 & 63,08 & 1261,50 \\
                  & Total & 116,00 &  & -- \\
\bottomrule
\end{tabular}
\source{Análise realizada no SPSS.}
\end{threeparttable}
\end{table}


Os resultados revelaram que os Reels (formato de vídeo) apresentaram postos médios ligeiramente superiores em curtidas (66,98 vs. 56,73), comentários (63,88 vs. 57,38) e compartilhamentos (63,08 vs. 57,55) em comparação às postagens estáticas \textit{(posts)}. No entanto, essas diferenças não foram estatisticamente significativas. Esses resultados indicam que, embora os vídeos apresentem resultados um pouco mais altos, o formato das postagens não influencia de forma significativa o engajamento do público, que interage de maneira semelhante com ambos os formatos. Enquanto o formato vídeo tem maior engajamento em curtidas e comentários, o número de compartilhamentos é similar entre os dois formatos.

Vídeos curtos, como os Reels, são especialmente eficazes para capturar a atenção do público, principalmente em plataformas voltadas a elementos visuais. Esses formatos, por sua natureza dinâmica, em geral, não apenas atraem mais visualizações, mas também promovem maior retenção de informações e engajamento emocional, superando conteúdos textuais ou estáticos \cite{bik2013, priem-costello2010, welmer2024}. Essa eficácia ajuda a explicar por que esse tipo de postagem frequentemente apresenta melhores desempenhos, enquanto postagens estáticas ou predominantemente textuais tendem a gerar níveis mais moderados de engajamento.

No que se refere à área de conhecimento, a Tabela \ref{tab-6} apresenta os resultados do teste de Kruskal-Wallis, usado para comparar mais de duas amostras independentes. Neste caso, o objetivo foi avaliar se havia diferenças entre categorias de área de conhecimento em relação a três métricas de engajamento.

%--- código da tabela 6 ---%
\begin{table}[h!]
\centering
\begin{threeparttable}
\caption{Teste de hipótese.}\label{tab-6}
\begin{tabular}{p{4cm} p{4cm} p{2cm} p{3cm}}
\toprule
Hipótese Nula & Teste & Valor de $p$. & Decisão \\
\midrule
A distribuição de n.º de curtidas é a mesma entre as categorias de área de conhecimento. & Teste Kruskal-Wallis de amostras independentes. & .785 & Reter a hipótese nula.\\
A distribuição de n.º de comentários é a mesma entre as categorias de área de conhecimento. & Teste Kruskal-Wallis de amostras independentes. & .388 & Reter a hipótese nula.\\
A distribuição de n.º de compartilhamentos é a mesma entre as categorias de área de conhecimento. & Teste Kruskal-Wallis de amostras independentes. & .872 & Reter a hipótese nula.\\
\bottomrule
\end{tabular}
\source{Análise realizada no SPSS.}
\end{threeparttable}
\end{table}


Os resultados indicaram que não há diferenças estatisticamente significativas entre as categorias de conhecimento nas variáveis analisadas. Para a hipótese 1, que investigava a distribuição do número de curtidas, o valor de significância foi .785, indicando que não houve diferença significativa. O mesmo ocorreu para a hipótese 2, sobre o número de comentários, com um valor de significância de .388. Por fim, para a hipótese 3, que tratava do número de compartilhamentos, o valor de significância foi .872, sugerindo que não há diferenças estatísticas entre as categorias em relação a essa variável. Isso sinaliza que as categorias, conforme definidas, não impactam significativamente o engajamento do público, conforme visualizado na Tabela \ref{tab-7}.

%--- código da tabela 7 ---%
\begin{table}[h!]
\centering
\caption{Postos Médios.}\label{tab-7}
\begin{threeparttable}
\begin{tabular}{llll}
\toprule
Métricas & Área de Conhecimento & N.º & Posto Médio \\
\midrule
Curtidas & Ciências Biológicas & 47,00 & 59,82 \\
         & Ciências da Saúde & 20,00 & 60,28 \\
         & Tecnologia Interdisciplinar & 29,00 & 58,16 \\
         & Ciências Exatas & 5,00 & 39,50 \\
         & Ciências Humanas & 15,00 & 59,00 \\
         & Total & 116,00 & -- \\
\midrule
Comentários & Ciências Biológicas & 47,00 & 53,79 \\
            & Ciências da Saúde & 20,00 & 55,65 \\
            & Tecnologia Interdisciplinar & 29,00 & 59,74 \\
            & Ciências Exatas & 5,00 & 79,00 \\
            & Ciências Humanas & 15,00 & 67,83 \\
            & Total & 116,00 & -- \\
\midrule
Compartilhamentos & Ciências Biológicas & 47,00 & 59,43 \\
                  & Ciências da Saúde & 20,00 & 59,65 \\
                  & Tecnologia Interdisciplinar & 29,00 & 61,28 \\
                  & Ciências Exatas & 5,00 & 53,30 \\
                  & Ciências Humanas & 15,00 & 50,43 \\
                  & Total & 116,00 & -- \\
\bottomrule
\end{tabular}
\source{Análise realizada no SPSS.}
\end{threeparttable}
\end{table}


Os postos médios revelam diferenças no engajamento entre as áreas de conhecimento. Ciências da Saúde (60,28) e Ciências Biológicas (59,82) apresentaram maior engajamento em curtidas, enquanto Ciências Exatas teve o menor (39,50). Em comentários, Ciências Exatas liderou (79,00), seguida por Ciências Humanas (67,83) e Tecnologia Interdisciplinar (59,74). Para compartilhamentos, Tecnologia Interdisciplinar destacou-se (61,28), enquanto Ciências Humanas teve o menor engajamento (50,43). Esses resultados indicam que as áreas de conhecimento influenciam de forma distinta o engajamento em métricas específicas.

Os resultados do estudo corroboram achados da literatura que destacam a influência das áreas de conhecimento no engajamento do público nas redes sociais. \textcite{stieglitz2013} argumentam que temas que apresentam um apelo emocional e prático podem favorecer maior engajamento, o que ajuda a explicar os resultados observados para as áreas de Ciências da Saúde e Ciências Biológicas. No entanto, o engajamento discursivo, como os comentários, foi mais evidente em áreas como Ciências Exatas e Ciências Humanas, o que está alinhado à discussão de \textcite{brossard2013}, que aponta para a tendência de que tópicos que requerem reflexão crítica ou abordam controvérsias científicas tendem a suscitar debates e discussões mais profundas.

Esse engajamento também reflete a análise de \textcite{priem-hemminger2010}, que destacam como áreas técnicas incentivam interações discursivas pela necessidade de contextualizar informações complexas. Por outro lado, o maior número de compartilhamentos na Tecnologia Interdisciplinar se conecta às observações de \textcite{kaplan2010}, segundo os quais conteúdos percebidos como inovadores têm maior probabilidade de serem disseminados em redes sociais. De acordo com \textcite{berger2012}, conteúdos com alta carga emocional, que despertam admiração ou oferecem informações úteis, tendem a ser mais compartilhados nas plataformas. Assim, publicações que apresentam avanços tecnológicos com aplicações práticas e impacto social, podem ativar mecanismos emocionais e cognitivos que favorecem o engajamento e a disseminação espontânea nas redes.

Nesse contexto, a análise reforça que as áreas de conhecimento também impactam o tipo de interação, corroborando a ideia de \textcite{rowe2005} e \textcite{ataidemalcher2025} de que os mecanismos de engajamento devem ser adaptados às características tanto do conteúdo quanto do público. Essa constatação ajuda a compreender por que campos distintos, como as Ciências Exatas e as Ciências Humanas, demandam abordagens comunicativas diferenciadas. Cada domínio apresenta especificidades próprias, que influenciam o estilo mais apropriado a ser empregado nos meios da comunicação científica.

Desse modo, as mídias sociais, ao atuarem como mediadoras entre o conhecimento e o público, oferecem oportunidades únicas para aplicar modelos de comunicação que vão além do simples fornecimento de informações \cite{burns2003}. Os achados de \textcite{mckinnon2015} sugerem que o engajamento deve ser compreendido como um conceito limiar tanto para a educação quanto para a comunicação científica. Sua compreensão exige não apenas o domínio técnico da informação, mas também sensibilidade às necessidades e expectativas do público. Nesse sentido, a criatividade contribui como um elemento primordial para desenvolver táticas comunicativas eficazes que promovam interações significativas. Essa perspectiva se reflete na capacidade de cientistas e instituições adotarem abordagens que incorporem elementos bidirecionais. A criatividade, entendida como resultado de interações entre indivíduos e contextos \cite{lubart2007, lubart2019, glaveanu2013}, assume nas mídias sociais um caráter coletivo, marcado pelo encontro entre cientistas, público e tecnologia.

Portanto, os resultados deste estudo, aliados à literatura, sugerem que a criatividade na comunicação científica deve ser entendida como uma construção coletiva entre cientistas que, a partir de suas crenças, evidências e vivências em sua área \cite{kotz2023a, kotz2024}, comunicam resultados de pesquisas por meio de diferentes canais -- sejam eles tradicionais ou midiáticos --, conectando saberes, contextos e público em geral, de maneira integrada. Nas mídias sociais, essa criatividade se manifesta por meio de narrativas que não apenas informam, mas engajam e transformam.

\section{Conclusão}
Os resultados do estudo identificam que o engajamento em redes sociais varia de acordo com as métricas analisadas (curtidas, comentários e compartilhamentos), o formato de postagem (imagens estáticas \textit{versus} vídeos curtos) e as áreas de conhecimento. O Instagram destacou-se em curtidas e compartilhamentos, enquanto o Facebook apresentou maior engajamento em comentários, demonstrando que cada plataforma possui características específicas que influenciam a interação do público.

No que diz respeito ao formato das postagens, embora vídeos curtos tenham mostrado resultados ligeiramente superiores, as diferenças não foram estatisticamente significativas, indicando que o público interage de forma semelhante em ambos os formatos. Quanto às áreas de conhecimento, Ciências da Saúde e Ciências Biológicas apresentaram maior engajamento em curtidas, possivelmente devido ao apelo emocional e prático de seus temas. Em contrapartida, Ciências Exatas e Ciências Humanas se destacaram em comentários, sugerindo um maior engajamento discursivo e reflexivo, enquanto Tecnologia Interdisciplinar liderou em compartilhamentos, reforçando a percepção de utilidade e inovação associada a essa área.

Esses achados corroboram a literatura existente, que destaca a importância das características do conteúdo e das plataformas para adaptar as interações do público. Com base nessa fundamentação, os resultados do presente estudo reforçam a necessidade de desenvolver estratégias personalizadas de comunicação científica nas mídias sociais, considerando as especificidades de cada ambiente digital e os perfis dos públicos envolvidos. Além disso, sugere-se que instituições científicas adotem métricas de engajamento como ferramentas para ajustar suas práticas comunicacionais, promover mensagens acessíveis a públicos diversos e fortalecer ações de enfrentamento à desinformação, contribuindo para ampliar o impacto social da ciência.

Embora este estudo tenha utilizado uma abordagem quantitativa, análises futuras poderiam integrar os resultados a perspectivas interpretativas, mobilizando teorias e categorias da análise. Tal articulação permitiria compreender não apenas a forma como a ciência é consumida nas redes sociais, mas também os efeitos que o discurso científico veiculado produz. Esse diálogo entre métricas de impacto e análise discursiva representaria uma contribuição inovadora para os estudos de comunicação científica.



\printbibliography\label{sec-bib}
% if the text is not in Portuguese, it might be necessary to use the code below instead to print the correct ABNT abbreviations [s.n.], [s.l.]
%\begin{portuguese}
%\printbibliography[title={Bibliography}]
%\end{portuguese}


%full list: conceptualization,datacuration,formalanalysis,funding,investigation,methodology,projadm,resources,software,supervision,validation,visualization,writing,review
\begin{contributors}[sec-contributors]
\authorcontribution{Suellen Cristina Rodrigues Kotz}[conceptualization,methodology,projadm,writing,review,formalanalysis]
\authorcontribution{Asdrúbal Borges Formiga Sobrinho}[review,formalanalysis]
\authorcontribution{Marina Silva Bicalho Rodrigues}[review]
\end{contributors}

\begin{dataavailability}
\txtdataavailability{dataonly} % options: dataavailable, dataonly, databody, datanotav, nodata
\end{dataavailability}


\end{document}

