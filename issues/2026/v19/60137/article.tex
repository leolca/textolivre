% !TEX TS-program = XeLaTeX
% use the following command:
% all document files must be coded in UTF-8
\documentclass[portuguese]{textolivre}
% build HTML with: make4ht -e build.lua -c textolivre.cfg -x -u article "fn-in,svg,pic-align"

\journalname{Texto Livre}
\thevolume{19}
%\thenumber{1} % old template
\theyear{2026}
\receiveddate{\DTMdisplaydate{2025}{7}{5}{-1}} % YYYY MM DD
\accepteddate{\DTMdisplaydate{2025}{10}{24}{-1}}
\publisheddate{\DTMdisplaydate{2025}{12}{2}{-1}}
\corrauthor{Marceli Aquino}
\articledoi{10.1590/1983-3652.2026.60137}
%\articleid{NNNN} % if the article ID is not the last 5 numbers of its DOI, provide it using \articleid{} commmand 
% list of available sesscions in the journal: articles, dossier, reports, essays, reviews, interviews, editorial
\articlesessionname{articles}
\runningauthor{Aquino e Alcoforado} 
%\editorname{Leonardo Araújo} % old template
\sectioneditorname{Daniervelin Pereira~\orcid{0000-0003-1861-3609}}
\layouteditorname{Saula Cecília~\orcid{0009-0006-3069-8480}}

\title{Práticas multimodais da oralidade: explorando \textit{reality shows} no ensino de alemão como língua adicional}
\othertitle{Multimodal oral practices: exploring reality shows in teaching German as an additional language}
% if there is a third language title, add here:
%\othertitle{Artikelvorlage zur Einreichung beim Texto Livre Journal}

\author[1]{Marceli Cherchiglia Aquino~\orcid{0000-0003-0518-7639}\thanks{Email: \href{mailto:marceli.c.aquino@usp.br}{marceli.c.aquino@usp.br}}}
\author[1]{Sofia Leria Alcoforado~\orcid{0009-0007-3680-1247}\thanks{Email: \href{mailto:sofialeria@usp.br}{sofialeria@usp.br}}}
\affil[1]{Universidade de São Paulo, Faculdade de Letras, Departamento de Letras Modernas, SP, Brasil.}

\addbibresource{article.bib}
% use biber instead of bibtex
% $ biber article

% used to create dummy text for the template file
\definecolor{dark-gray}{gray}{0.35} % color used to display dummy texts
\usepackage{lipsum}
\SetLipsumParListSurrounders{\colorlet{oldcolor}{.}\color{dark-gray}}{\color{oldcolor}}

% used here only to provide the XeLaTeX and BibTeX logos
\usepackage{hologo}

% if you use multirows in a table, include the multirow package
\usepackage{multirow}

% provides sidewaysfigure environment
\usepackage{rotating}

% CUSTOM EPIGRAPH - BEGIN 
%%% https://tex.stackexchange.com/questions/193178/specific-epigraph-style
\usepackage{epigraph}
\renewcommand\textflush{flushright}
\makeatletter
\newlength\epitextskip
\pretocmd{\@epitext}{\em}{}{}
\apptocmd{\@epitext}{\em}{}{}
\patchcmd{\epigraph}{\@epitext{#1}\\}{\@epitext{#1}\\[\epitextskip]}{}{}
\makeatother
\setlength\epigraphrule{0pt}
\setlength\epitextskip{0.5ex}
\setlength\epigraphwidth{.7\textwidth}
% CUSTOM EPIGRAPH - END

% to use IPA symbols in unicode add
%\usepackage{fontspec}
%\newfontfamily\ipafont{CMU Serif}
%\newcommand{\ipa}[1]{{\ipafont #1}}
% and in the text you may use the \ipa{...} command passing the symbols in unicode

% LANGUAGE - BEGIN
% ARABIC
% for languages that use special fonts, you must provide the typeface that will be used
% \setotherlanguage{arabic}
% \newfontfamily\arabicfont[Script=Arabic]{Amiri}
% \newfontfamily\arabicfontsf[Script=Arabic]{Amiri}
% \newfontfamily\arabicfonttt[Script=Arabic]{Amiri}
%
% in the article, to add arabic text use: \textlang{arabic}{ ... }
%
% RUSSIAN
% for russian text we also need to define fonts with support for Cyrillic script
% \usepackage{fontspec}
% \setotherlanguage{russian}
% \newfontfamily\cyrillicfont{Times New Roman}
% \newfontfamily\cyrillicfontsf{Times New Roman}[Script=Cyrillic]
% \newfontfamily\cyrillicfonttt{Times New Roman}[Script=Cyrillic]
%
% in the text use \begin{russian} ... \end{russian}
% LANGUAGE - END

% EMOJIS - BEGIN
% to use emoticons in your manuscript
% https://stackoverflow.com/questions/190145/how-to-insert-emoticons-in-latex/57076064
% using font Symbola, which has full support
% the font may be downloaded at:
% https://dn-works.com/ufas/
% add to preamble:
% \newfontfamily\Symbola{Symbola}
% in the text use:
% {\Symbola }
% EMOJIS - END

% LABEL REFERENCE TO DESCRIPTIVE LIST - BEGIN
% reference itens in a descriptive list using their labels instead of numbers
% insert the code below in the preambule:
%\makeatletter
%\let\orgdescriptionlabel\descriptionlabel
%\renewcommand*{\descriptionlabel}[1]{%
%  \let\orglabel\label
%  \let\label\@gobble
%  \phantomsection
%  \edef\@currentlabel{#1\unskip}%
%  \let\label\orglabel
%  \orgdescriptionlabel{#1}%
%}
%\makeatother
%
% in your document, use as illustraded here:
%\begin{description}
%  \item[first\label{itm1}] this is only an example;
%  % ...  add more items
%\end{description}
% LABEL REFERENCE TO DESCRIPTIVE LIST - END


% add line numbers for submission
%\usepackage{lineno}
%\linenumbers

\begin{document}
\maketitle

\begin{polyabstract}
\begin{abstract}
Este artigo discute a inclusão de aspectos da oralidade e de discursos coloquiais no ensino de Alemão como Língua Adicional (ALA) em contexto universitário, com foco na formação docente. Apesar do avanço de tecnologias e materiais didáticos, ainda se observa uma lacuna no trabalho com materiais autênticos e situados, o que limita a compreensão da língua como prática social. Argumenta-se que a incorporação de gêneros orais espontâneos aproxima estudantes de Letras de práticas linguísticas e culturais reais, enriquecendo a dimensão sociocultural do ensino. A discussão fundamenta-se, portanto, em estudos sobre pragmática, multimodalidade e autenticidade no ensino de línguas. A proposta didática apresentada utiliza episódios das séries \textit{Queer Eye Germany} e \textit{Too Hot to Handle} como materiais audiovisuais para explorar marcadores pragmáticos e construções identitárias em contextos autênticos de uso da língua. Conclui-se que o uso de gêneros multimodais de oralidade contribui não apenas para o desenvolvimento do repertório linguístico-pragmático, mas também para a formação docente crítica, criativa e sensível à diversidade linguística e cultural.

\keywords{Oralidade\sep Multimodalidade\sep Ensino de alemão como língua adicional\sep Gêneros orais}
\end{abstract}

\begin{english}
\begin{abstract}
This article discusses the inclusion of orality and colloquial discourse features in the teaching of German as an Additional Language (GAL) in university contexts, with a focus on teacher education. Despite advances in technologies and teaching materials, there remains a gap in the use of authentic and situated materials, which limits the understanding of language as a social practice. It is argued that incorporating spontaneous spoken genres brings language students closer to real linguistic and cultural practices, enriching the sociocultural dimension of language teaching. The discussion is grounded in studies on pragmatics, multimodality, and authenticity in language education. The proposed didactic approach uses episodes from the series \textit{Queer Eye Germany} and \textit{Too Hot to Handle} as audiovisual materials to explore pragmatic markers and identity constructions in authentic contexts of language use. The study concludes that employing multimodal oral genres contributes not only to the development of linguistic-pragmatic repertoires but also to critical, creative, and culturally sensitive teacher education.

\keywords{Orality\sep Multimodality\sep Teaching of German as an additional language\sep Oral genres}
\end{abstract}
\end{english}
% if there is another abstract, insert it here using the same scheme
\end{polyabstract}

\section{Introdução}\label{sec-intro}
Com o objetivo de aproximar estudantes de Letras, futuras docentes de Alemão como Língua Adicional (ALA)\footnote{Em diferentes contextos de ensino de línguas no Brasil, o termo adicional tem gradualmente substituído estrangeira, por refletir uma concepção mais inclusiva e cidadã do uso dos idiomas -- como meio de participação em comunidades de prática diversas, dentro e fora do país.}, do uso autêntico e informal da língua, propomos nesta seção um conjunto de atividades didáticas baseadas em trechos de \textit{reality shows} enquanto materiais audiovisuais multimodais. A presente iniciativa busca desenvolver conhecimentos pragmáticos, interculturais e multimodais por meio do contato com situações comunicativas autênticas, marcadas por espontaneidade, interação e diversidade cultural. Ao mesmo tempo, pretende-se fomentar a reflexão crítica sobre os modos de representação e construção de sentido em produtos midiáticos contemporâneos.

O ensino ALA no Brasil, especialmente no contexto dos cursos de Letras, enfrenta desafios significativos em relação à autenticidade e representatividade dos materiais utilizados em sala de aula. Embora o idioma seja parte do currículo de diversas universidades do país, ainda observamos a predominância do uso de livros didáticos internacionais, produzidos por grandes editoras europeias, voltados a um público global, isto é, com uma concepção homogênea de língua, cultura e sociedade \cite{marquesschafer2016, voerkel2020, Aquino2023, uphoff2024}. Esses materiais, apesar de didaticamente estruturados, tendem a apresentar uma visão limitada da língua em uso, frequentemente desvinculada dos contextos reais de comunicação, assim como das vivências e interesses das estudantes\footnote{Empregamos o artigo feminino em referências genéricas, compreendendo-o como forma inclusiva que abrange diferentes gêneros e funções, que pode ser substituído por ``pessoas que aprendem'' e ``pessoas que ensinam''.} \cite{schmidt2016}.

Assim, a predominância de livros didáticos globais nos cursos de ALA causa uma lacuna importante no processo de aprendizagem: o distanciamento em relação à oralidade, à linguagem coloquial e aos aspectos socioculturais contemporâneos da língua-alvo. Nesse sentido, compreendemos autenticidade como a presença de características linguísticas e discursivas espontâneas, não roteirizadas e situadas em contextos reais de uso da língua, isto é, não elaboradas para o objetivo de ensino, nas quais se evidenciam variação, imprevisibilidade e marcas de interação social \cite{breen1985, gilmore2007}.

Embora o acesso aos conteúdos autênticos em língua alemã seja atualmente cada vez mais amplo, com a oferta de filmes, \textit{podcasts}, canais de \textit{streaming} e plataformas educacionais como a \textit{Deutsche Welle}, a questão central não reside apenas na disponibilidade desses recursos, mas na mediação pedagógica que orienta seu uso. Tanto o levantamento de materiais, como a elaboração de formas de trabalho em sala de aula pode representar um desafio às docentes, restringindo o desenvolvimento de habilidades comunicativas em suas dimensões pragmática, intercultural e multimodal \cite{freitaghild2010}.

A habilidade pragmática refere-se à capacidade de empregar a língua de modo situado em relação às intenções comunicativas, ao contexto e às normas de interação social, envolvendo aspectos como cortesia, ironia e outras estratégias discursivas \cite{barron2016}. Já a sensibilidade intercultural diz respeito à habilidade de reconhecer, interpretar e negociar valores culturais, isto é, conjuntos de normas sociais, práticas comunicativas, crenças e representações identitárias que orientam modos de agir, pensar e interagir em diferentes comunidades discursivas \cite{byram1997, kramsch1998}. Portanto, compreende-se cultura, não como um sistema fixo de tradições ou produtos simbólicos, mas como um processo dinâmico e relacional de construção de significados constantemente negociados nas interações mediadas pela linguagem \cite{kramsch1998, holliday2011}. Por sua vez, a consciência multimodal envolve o reconhecimento de que a comunicação contemporânea se realiza por meio da relação entre múltiplos modos semióticos, como gestos, imagens, sons e expressões corporais, o que requer da aprendiz uma leitura crítica e sensível dessas camadas de sentido \cite{kress2001, aguado2021}.

Particularmente relevantes nesse contexto são os gêneros audiovisuais oriundos da cultura midiática atual, como os \textit{reality shows} de \textit{streamings} como a Netflix, que oferecem uma rica variedade de situações espontâneas de comunicação. Programas desse tipo tendem a retratar interações informais, linguagem coloquial, negociações interpessoais e elementos expressivos como gestos, pausas, entonações e reações emocionais, que são fundamentais para o trabalho com autenticidade em ALA. Ao incorporar trechos de \textit{reality shows} ao ensino universitário de alemão, é possível não apenas expor estudantes às formas genuínas de fala, mas também promover a análise crítica de práticas sociais, construções identitárias e valores culturais.

Com base nesse panorama, este artigo tem como objetivo discutir estratégias para a ampliação da presença da oralidade e da linguagem informal no ensino universitário de ALA, com foco na formação docente em cursos de Letras. Considerando os limites impostos pelo uso predominante de livros didáticos internacionais, propõe-se o uso complementar de materiais audiovisuais autênticos como forma de expor aprendizes à língua em uso. Reconhecemos que a inclusão desse tipo de material didático exige um trabalho criterioso por parte das docentes -- identificar trechos relevantes, adaptar o nível de dificuldade, planejar atividades contextualizadas e conduzir discussões significativas -- representa uma tarefa complexa. Neste sentido, as reflexões reunidas neste artigo visam contribuir com subsídios teóricos e sugestões práticas, mas sem esgotar a temática. A proposta é oferecer um ponto de partida para que (futuras) docentes possam repensar suas abordagens e ampliar os repertórios de condutas pedagógicas voltadas à fala autêntica, de modo a articular oralidade, multimodalidade e reflexão intercultural adaptáveis aos seus ambientes de ensino.

Neste artigo, propomos uma abordagem pedagógica que integra práticas multimodais de oralidade ao ensino de ALA a partir de episódios de dois \textit{reality shows} da plataforma Netflix: \textit{Queer Eye Germany} e \textit{Too Hot to Handle: Germany}. Esperamos que essa proposição permita o contato com formas comunicativas que não se limitam apenas à fala ou à escrita, e que envolvam uma coordenação de recursos expressivos que refletem também as práticas sociais. Assim, consideramos que a linguagem midiática dos \textit{reality shows}, fortemente multimodal e espontânea, constitui um ponto de partida fértil para o trabalho com a recepção oral e a análise de aspectos pragmáticos, discursivos e identitários em alemão.

Apesar do crescente uso de filmes e séries em contextos educacionais, os \textit{reality shows} permanecem pouco explorados, em parte por seu caráter híbrido e informal que desafia as convenções tradicionais do material didático e convida a uma mediação crítica da linguagem televisiva \cite{FerreiraAquino2024}. Logo, a proposta distingue-se da literatura existente por explorar um gênero ainda pouco abordado em pesquisas sobre ensino de línguas, o que, certamente, implica não apenas desafios, mas também oportunidades, que buscamos explorar ao longo deste artigo.

O texto está estruturado em quatro seções: a primeira discute os conceitos de oralidade, multimodalidade e autenticidade no ensino de línguas, com base em contribuições da didática do alemão e da pragmática intercultural. A segunda aborda o papel da experiência pragmática na aprendizagem de línguas adicionais, com foco nas dimensões interacional e sociocultural da fala. A terceira apresenta a proposta didática, com exemplos de atividades elaboradas a partir de dois \textit{reality shows} e análise de seu potencial pedagógico. Por fim, a conclusão retoma os principais pontos e destaca os desafios e possibilidades de implementação dessa abordagem.


\section{Oralidade, modalidade e autenticidade}
A oralidade envolve não apenas a produção, mas também a recepção da linguagem falada, já que ambas constituem dimensões interdependentes do processo comunicativo. Conforme \textcite{aguado2021}, o principal objetivo da comunicação oral é construir um discurso coerente, fluente e pertinente ao contexto. Trata-se, contudo, de uma atividade complexa, já que em uma língua adicional, a formulação de enunciados conforme as normas do idioma já exigem elevado esforço cognitivo, ao passo que a coerência e a adequação interacional demandam prática contínua que depende de múltiplos fatores, como psicológicos, cognitivos, socioculturais e linguísticos \cite{ellis2003}.

O desenvolvimento das habilidades de oralidade desempenha um papel essencial nesse continuum por serem intrinsecamente dialógicas, isto é, marcadas pela dinamicidade da alternância de turnos, tempo de fala, interdependência entre quem fala e quem ouve e pela co-construção do sentido na interação \cite{bakhtin1986}. Nessas condições, o trabalho didático com a oralidade apresenta desafios próprios que tornam o processo de compreensão e produção em ALA especialmente complexo, o que pode resultar em frustração e desmotivação.

Contudo, as estratégias de aprendizagem, como a compreensão parcial, são parte integrante desta tarefa. De acordo com a hipótese do \textit{input compreensível (i+1)}, a exposição a mensagens ligeiramente acima do nível atual de proficiência estimula o desenvolvimento do repertório comunicativo \cite{krashen1982, gass2008}. Da mesma forma, em abordagens baseadas em tarefas, as lacunas de entendimento são vistas como oportunidades para inferência, negociação de sentido e ampliação do repertório linguístico \cite{ellis2003}. Para favorecer esse processo, \textcite{aguado2021} recomenda o uso de materiais multimodais, que ampliem o acesso ao significado por meio de gestos, expressões faciais e outros recursos visuais e sonoros, aproximando a experiência de escuta das condições autênticas.

Considerando essas perspectivas, o ensino da oralidade deve contemplar tanto a compreensão quanto a produção por meio de discursos autênticos, integrando habilidades linguísticas, pragmáticas e multimodais. Nessa direção, \textcite{solmecke2010} propõe um ensino com dois focos complementares: o domínio de estruturas linguísticas (vocabulário, gramática, pronúncia) e as condições de lidar com situações de comunicação espontâneas. O autor defende a combinação de abordagens tradicionais e modernas, como a aprendizagem baseada em tarefas, que estimula o uso ativo da língua de forma contextualizada. Experiências relatadas por \textcite{sousa2024}, \textcite{acevedodelapena2024} exemplificam essa perspectiva: no primeiro estudo, estudantes produziram propagandas em inglês, desenvolvendo habilidades críticas, criativas e comunicativas; no segundo, criaram \textit{podcasts}, o que favoreceu a expressão individual e a prática situada da língua. Ambas as propostas exploram materiais midiáticos populares e demonstram caminhos possíveis para integrar a comunicação oral, multimodal e autêntica além dos limites do livro didático.

Essas experiências reforçam a importância de ampliar o conceito de material didático, incluindo não apenas gêneros tradicionais, mas também produtos discursivos contemporâneos. Nesse sentido, o trabalho com gêneros digitais autênticos amplia o ensino de línguas, inserindo práticas de linguagem em contextos sociais contemporâneos \textcite{Rojo2009}. Plataformas como Instagram, TikTok, Threads, X (antigo Twitter) e YouTube constituem exemplos de espaços legítimos de produção discursiva, onde circulam gêneros multimodais que combinam linguagem verbal, visual e sonora. A formação docente, nesse contexto, precisa contemplar o fomento de letramentos críticos e multimodais \cite{cope2009}, capazes de preparar professoras e alunas para interpretar e produzir conteúdo que faça parte do cotidiano digital.

Neste trabalho defendemos que a autenticidade dos materiais selecionados para o ensino é também essencial para propiciar o contato com situações de fala diversas e, com isso, favorecer o desenvolvimento de habilidades pragmáticas, isto é, capacidade de empregar a língua de modo socialmente situado, levando em conta gestos, expressões e conhecimentos socioculturais \cite{barron2016, kramsch1998}. O trabalho com temas socioculturais atuais e localmente contextualizados torna-se, assim, indispensável, tanto por possibilitar experiências comunicativas mais situadas quanto por promover identificação e representatividade, aspectos frequentemente negligenciados no ensino convencional. Quanto maior a conexão entre o tema proposto e o repertório das aprendizes, maior tende a ser o seu engajamento nas discussões e tarefas propostas \cite{Rojo2009}. Conforme destaca \textcite{solmecke2010}, cabe à docente atuar como mediadora e impulsionadora de processos de aprendizagem, concebendo a construção da autonomia e de habilidades estratégicas como elementos centrais do ensino de línguas.

Nessa perspectiva, propomos o uso de \textit{reality shows} por seu potencial para o ensino de oralidade e pragmática. Essa escolha se fundamenta na natureza multimodal e interacional desse gênero midiático que combina linguagem verbal, gestual, visual e sonora em situações comunicativas próximas às da vida cotidiana \textcite{aguado2021, gilmore2007}. Embora tais programas possuam algum grau de roteirização e edição, preservam elementos de espontaneidade, como sobreposição de falas, interrupções, hesitações e negociações de sentido, traços característicos da fala autêntica. Além disso, por apresentarem uma estrutura episódica e não linear, os \textit{reality shows} permitem a seleção de trechos independentes, o que facilita sua adaptação ao contexto pedagógico sem perda significativa de coerência temática, aspectos esses que serão tratados na próxima seção.

\section{O trabalho com a oralidade na formação docente}
No contexto da formação docente, é fundamental que futuras professoras tenham contato com estratégias e materiais que integrem diferentes dimensões da oralidade, possibilitando experiências contextualizadas de uso da língua \cite{freeman2002}. O trabalho com textos orais autênticos é especialmente relevante, pois oferece uma aproximação mais direta em relação à realidade comunicativa do idioma e favorece o desenvolvimento de um repertório linguístico reflexivo e intercultural. Ao situar-se no contexto sociocultural e histórico de uma comunidade discursiva, a fala espontânea permite compreender a língua como prática social e culturalmente situada, promovendo uma aprendizagem mais significativa e engajada.

Entendemos oralidade aqui não apenas como a produção ou recepção de fala, mas como uma prática discursiva multimodal e interacional, que envolve a construção conjunta de sentidos por meio da linguagem verbal, gestos, entonação, olhar e ritmo \cite{marcuschi2007, koch2011}. Nessa perspectiva, a produção e recepção oral ultrapassa o plano fonológico, configurando-se como evento comunicativo permeado por fatores socioculturais, cognitivos e contextuais. Essa concepção dialoga com abordagens que compreendem a linguagem como ação social e prática cultural \cite{bauman1990, kramsch1998}, permitindo que o ensino vá além da aprendizagem técnica da fala para incluir também a reflexão crítica sobre os usos e significados da língua em contextos próximos a realidade de uso do idioma.

Observa-se, no entanto, que os materiais didáticos regulares de ALA apresentam conteúdos orais predominantemente em formato de áudio linear, sem incorporar elementos multimodais (visuais, gestuais e contextuais) fundamentais para a compreensão do idioma em situações comunicativas diversas. Além disso, esses áudios costumam ter uma função meramente ilustrativa, servindo de suporte à explicação de regras gramaticais, o que resulta em diálogos artificiais, previsíveis e distantes da fala autêntica. Um exemplo ilustrativo aparece no livro \textit{Akademie Deutsch A1+} \cite[p. 75]{bleiner2020a}, em um exercício intitulado \textit{``Menschen im Herbst''} (``Pessoas no outono''), na \Cref{fig-1}.

%--- código da figura 1 ---%
\begin{figure}[h!]
\centering
\begin{minipage}{.90\textwidth}
\includegraphics[width=\textwidth]{image1.png}
\caption{Enunciado do exercício.}
\label{fig-1}
\source{Akademie Deutsch A1+ \citeyear{bleiner2020a}.}
\notes{Tradução: ``3.2 Pessoas no outono.  a) Veja as pessoas nas fotos. O que elas fazem no outono? Fale no curso. \textit{O homem na imagem 1 trabalha no outono … / A mulher na imagem 3 anda no outono.} b) Ouça os textos e escolha qual texto é relacionado a imagem. Qual imagem não tem um texto relacionado (-)?''}
\end{minipage}
\end{figure}

Neste exercício, a letra b) pede que sejam ouvidos quatro áudios -- nomeados de A a D -- para associá-los a uma das imagens. O site da própria editora do livro oferece a seguinte transcrição para o áudio A\footnote{\textit{Link} para o material: \url{https://www.hueber.de/akademie-deutsch/unterrichten/materialien}.}: \emph{``Ich mag den Herbst. Okay, es wird schneller dunkel, das mögen ja viele nicht so besonders. Aber dafür wird das Leben auch ruhiger, und die Leute bleiben öfter mal zu Hause. Das ist doch total gemütlich! Ich lege mich dann gern mit einem guten Buch aufs Sofa und trinke eine Tasse warmen Tee – dazu noch ein paar Kerzen und eine Tafel Schokolade, das ist für mich Entspannung pur!''}\footnote{Tradução: ``Eu gosto do outono. Ok, escurece mais rápido, muitos não gostam muito disso. Mas a vida também fica mais calma e as pessoas ficam mais em casa. Mas isso é muito confortável! Eu gosto de deitar com um bom livro no sofá e bebo uma xícara de chá quente -- isso acompanhado de algumas velas e uma barra de chocolate é puro relaxamento para mim!''}.

O exercício aborda o tema do outono europeu, que embora culturalmente distante da realidade da maioria das estudantes brasileiras pode desempenhar um papel formativo relevante no desenvolvimento do repertório intercultural, ao introduzir elementos da vida cotidiana e das práticas simbólicas da língua-alvo. De acordo com \textcite{byram1997, kramsch1998, liddicoat2013}, o contato com diferenças culturais amplia a percepção das aprendizes sobre outras formas de ser e agir no mundo, favorecendo uma postura reflexiva e crítica diante da própria cultura. Ainda assim observa-se que muitos materiais didáticos tendem a privilegiar temas eurocêntricos padronizados, raramente articulando comparações, problematizações ou espaços para que as aprendizes relacionem esses conteúdos às suas próprias experiências socioculturais – o que limita o potencial de diálogo intercultural efetivo.

Quanto ao áudio do exercício, percebe-se o uso de uma linguagem informal, com marcas de oralidade e partículas modais (PMs) típicas da fala cotidiana – por exemplo: ``das mögen \emph{ja} viele nicht so besonders''\footnote{Muitos não gostam muito disso.} e ``das ist \emph{doch} total gemütlich''\footnote{Mas isso é muito confortável.}. Apesar dessas características, o diálogo soa ensaiado e excessivamente controlado, com pausas marcadas e entonação pouco natural. Reconhecemos que em materiais voltados a níveis iniciais tal controle é uma escolha pedagógica, deliberada e necessária para garantir a compreensão auditiva e o desenvolvimento gradual de conhecimentos linguístico, assim, mais do que uma ``falta de autenticidade'' no sentido proposto por \textcite{breen1985}, trata-se de uma autenticidade pedagógica, ajustada às necessidades de quem está começando a aprender. O problema, no entanto, não reside na existência deste tipo de abordagem e contato com a oralidade, mas na sua dependência exclusiva, sem o complemento de práticas que exponham as aprendizes, progressivamente, a interações mais espontâneas e complexas, ainda que desafiadoras ou parcialmente compreensíveis, como discutido anteriormente. Esse movimento é fundamental para que o processo de aprendizagem não se restrinja à previsibilidade do livro didático, mas incorpore situações reais de comunicação nas quais o ``não entender totalmente'' também se torne oportunidade de construção de sentido e conhecimentos linguísticos \cite{SchmidtAquino2025}.

Estudos sobre o ensino da oralidade em línguas adicionais têm destacado as tensões entre autenticidade, compreensibilidade e intencionalidade pedagógica \cite{richards2006, thornbury2005, goh2012}. Em análise semelhante à de \textcite{SchmidtAquino2025}, trabalhos como os de \textcite{aguado2021, barron2016, freitaghild2010} apontam que, mesmo quando os livros didáticos globais incluem marcas de oralidade – como hesitações, repetições e PMs – o resultado ainda tende a soar artificial, pois os textos e áudios são cuidadosamente planejados para fins instrucionais. Essa tensão reflete a dificuldade de conciliar a clareza didática com a imprevisibilidade e a fluidez da fala. Assim, mesmo que o conteúdo contenha elementos típicos da fala, a forma controlada de sua apresentação reduz a naturalidade e distancia aprendizes das dinâmicas da comunicação.

Considerando essas limitações, é necessário refletir sobre como incorporar práticas discursivas mais autênticas ao ensino, especialmente em contextos universitários de formação docente. Diante dessas constatações, torna-se necessário repensar como a interação oral pode ser trabalhada de modo mais situado e significativo, articulando o desenvolvimento linguístico a fomento à consciência intercultural e multimodal. Nesse sentido, o gênero \textit{reality show} se caracteriza por uma linguagem mais espontânea e interacional, permitindo às estudantes observar marcas reais da fala, como sobreposições, hesitações e negociações de sentido \cite{gilmore2007}\footnote{De acordo com o Quadro Europeu Comum de Referência \cite{conselho2020}, a compreensão de gêneros orais espontâneos costuma exigir nível intermediário. No entanto, consideramos que o seu uso em níveis iniciais no curso de Letras é viável com mediação e adaptação adequadas.}.

Além disso, defendemos que o papel da docente é justamente negociar os limites entre desafio e acessibilidade, oferecendo \textit{input compreensível}, mas que também provoque a ampliação de repertórios linguísticos e culturais \cite{krashen1982, ellis2003}. Nesse sentido, não se trata de substituir os livros didáticos tradicionais, mas de complementá-los com experiências autênticas e desafiadoras, que permitam aprimorar não apenas o conhecimento linguístico, mas também a consciência intercultural e multimodal das aprendizes. A proposta, portanto, parte da premissa de que não há abordagens infalíveis nem prescrições universais -- o próprio quadro comum deve ser compreendido de forma contextualizada e flexível. Logo, mais do que seguir níveis de proficiência pré-definidos, importa considerar o contexto formativo acadêmico brasileiro e envolver aprendizes nas decisões pedagógicas, estimulando a autonomia e reflexão crítica sobre o ensino e aprendizagem de ALA.

Em síntese, o trabalho com a oralidade na formação docente deve ser compreendido não apenas como o desenvolvimento de habilidades comunicativas, mas como uma prática formativa e crítica, capaz de integrar linguagem, cultura e reflexão pedagógica em um processo de ensino-aprendizagem situado e transformador. Na próxima seção apresentamos a nossa sugestão de ensino de oralidade por meio de \textit{reality shows}.

\section{Proposta didática}
O uso de gêneros audiovisuais, como filmes, séries e \textit{reality shows}, no ensino de línguas adicionais têm sido amplamente reconhecido como um meio eficaz de aproximar aprendizes de práticas comunicativas reais e de múltiplas formas de representação cultural \cite{kaiser2014, canningwilson2000, king2002}. Esses materiais combinam linguagem verbal, gestual, visual e sonora, constituindo, portanto, textos multimodais que reproduzem, de modo mais dinâmico, as condições de interação da vida cotidiana. Além de contribuírem para o desenvolvimento da compreensão auditiva e do repertório pragmático, eles favorecem o engajamento e a motivação, por apresentarem contextos socioculturais significativos e próximos dos interesses das estudantes \cite{gilmore2007, aguado2021}.

No contexto do ensino de ALA, entretanto, o potencial desses recursos ainda é pouco explorado, especialmente em cursos de formação docente, onde a abordagem audiovisual tende a ser limitada a trechos de filmes ou vídeos com função ilustrativa. Defendemos que a integração planejada e crítica desses gêneros pode ampliar o repertório didático, articulando oralidade autêntica, multimodalidade e reflexão intercultural. Isso requer, contudo, uma concepção metodológica que vá além da mera exposição ao material e que considere a complexidade cognitiva e linguística envolvida em sua recepção.

Nesse sentido, a proposta aqui apresentada fundamenta-se em princípios consolidados da linguística aplicada à aprendizagem de línguas adicionais, com base nas noções de insumo compreensível \textit{(comprehensible input)} e de apoio gradual \textit{(scaffolding)} no processo de aquisição \cite{krashen1982, vygotsky1978, ellis2003}. Defendemos, assim, que trabalhar com materiais autênticos não significa expor aprendizes a conteúdos inalcançáveis, mas criar situações desafiadoras e mediadas, nas quais o contato com a língua em uso se converta em experiência de aprendizagem significativa. O papel da docente, portanto, é o de mediadora que oferece suporte linguístico, cultural e interpretativo, transformando o contato inicial com o texto audiovisual em oportunidade para desenvolver habilidades comunicativas, pragmáticas e interculturais.

A proposta aqui apresentada foi elaborada a partir de análise qualitativa e exploratória de materiais audiovisuais disponíveis em plataformas de \textit{streaming}, priorizando trechos curtos (2 a 10 minutos) com potencial didático para o ensino de ALA. As atividades sugeridas têm caráter ilustrativo e investigativo, buscando evidenciar possibilidades metodológicas – e não um modelo prescritivo de ensino. Essa natureza exploratória reforça o objetivo de compreender como produtos midiáticos podem ser integrados de modo crítico, contextualizado e adaptável a diferentes níveis de proficiência, sempre considerando as condições reais de ensino e o papel ativo das aprendizes no processo de construção de sentido.

Entre os gêneros audiovisuais disponíveis, os \textit{reality shows} se destacam por aliarem espontaneidade, diversidade sociocultural e relevância temática. Eles apresentam interações cotidianas que envolvem negociação de sentidos, gestão de conflitos, expressão de emoções e construção de identidades – dimensões essenciais da pragmática e da interculturalidade. Nesta proposta, foram selecionados dois programas da Netflix, \textit{Queer Eye Germany} e \textit{Too Hot to Handle: Germany}, que permitem observar tais fenômenos em situações de fala naturalizadas e socialmente significativas. Diferentemente dos roteiros ficcionais ou dos diálogos didatizados dos livros, esses programas oferecem exemplos ricos de linguagem espontânea, com hesitações, interrupções, sobreposição de vozes, gestos e expressões corporais. Para a seleção dos trechos, recomenda-se priorizar cenas curtas, entre dois até dez minutos (realizando pausas sempre que necessário), com boa qualidade de áudio e imagem, relevância temática e presença de elementos linguísticos e socioculturais contextualizados. A escolha deve ser feita de maneira colaborativa entre docente e estudantes, o que contribui para o engajamento e evita a imposição de temas demasiadamente sensíveis ou distantes da realidade do grupo. Consideramos neste trabalho que, com as devidas adaptações, o uso dos \textit{reality shows} pode ser flexível e adaptável a diferentes níveis de proficiência:

\begin{itemize}
    \item Turmas iniciais: trechos curtos e acompanhados de legendas em português ou alemão; foco na compreensão global; tarefas centradas na compreensão de situações e expressões; uso do português como língua de apoio;

    \item Turmas intermediárias: ampliar o tempo de exposição com legenda em alemão; através do uso de expressões previamente abordadas, discutir sobre aspectos pragmáticos e culturais; realizar produções escritas e orais sobre temas selecionados e trabalhados em sala;

\item Níveis avançados: trabalho com trechos mais longos, com ou sem legendas em alemão; debates sobre aspectos linguísticos e culturais mais complexos; realizar produções escritas e orais sobre temas selecionados.
\end{itemize}

O trabalho pode se desenvolver em cinco momentos complementares, articulando recepção, reflexão e produção.

\begin{enumerate}
    \item Seleção e contextualização: a professora apresenta o gênero \textit{reality show}, discute suas características e convida as estudantes a refletirem sobre a sua experiência com o gênero e suas características;
    \item Pré-escuta: ativação de conhecimentos prévios, introdução de vocabulário e formulação de perguntas: ``O que esperam do \textit{reality} que vamos assistir?'', ``Como se expressa em alemão desacordo ou empatia?'', entre outros;
    \item Escuta com propósito: exibição do trecho com legendas adequadas ao nível da turma, estimulando a identificação da situação comunicativa por meio de expressões, tons, gestos e marcadores pragmáticos;
    \item Análise e discussão: em plenária, debate-se o resultado do ponto 3, observando o uso de linguagem coloquial, marcadores discursivos e aspectos culturais, problematizando possíveis desentendimentos e dúvidas;
    \item Produção e prática: realização de tarefas de produção oral e escrita a partir dos pontos anteriores. Em níveis iniciais, podem descrever cenas, reconstruir falas ou criar legendas alternativas; em níveis mais avançados, criar diálogos, \textit{podcasts} ou entrevistas inspiradas nos episódios. O importante é que o foco recaia sobre o uso significativo da língua a partir do interesse das estudantes.
\end{enumerate}

O papel da docente ao longo de todas as etapas é o de fornecer suporte linguístico, cognitivo e emocional, ajustando o nível de desafio de acordo com a proficiência da turma. Esse apoio gradual \cite{vygotsky1978, wood1976} é o que torna possível trabalhar com materiais autênticos mesmo em níveis iniciais, desde que haja planejamento cuidadoso, mediação constante e abertura para o uso do português como ferramenta de reflexão e análise.

Por fim, é importante reconhecer que o uso de \textit{reality shows} apresenta limites e desafios pedagógicos, como a complexidade linguística, a velocidade da fala e a presença de representações socioculturais sensíveis. No entanto, tais desafios podem ser transformados em oportunidades formativas quando o material é abordado de forma crítica, mediada e reflexiva. Assim, o trabalho com gêneros multimodais e orais autênticos não pretende substituir o livro didático, mas complementá-lo oferecendo às futuras docentes experiências de linguagem diversificadas que articulem oralidade, cultura e pensamento crítico.

\section{\textit{Queer Eye Germany}: Empatia, identidade e partículas modais}

\textit{Queer Eye Germany} é uma versão do \textit{reality show} norte-americano em que cinco especialistas ajudam pessoas a transformarem aspectos de suas vidas (visual, estilo, casa, saúde emocional e autoestima). Consideramos o programa adequado para abordar estratégias de compreensão da língua, assim como abordar temas voltados ao apoio emocional, empatia e identidade, além do uso de PMs e formas de tratamento informais que são frequentemente utilizadas na série \cite{Aquino2023}.

\textbf{Trecho sugerido:} Início do primeiro episódio (cerca de um minuto de extensão), no qual os ``Fab 5'' se apresentam. A seguir apresentamos a transcrição das falas das personagens com destaque para os aspectos linguísticos e socioculturais que podem ser abordados em sala de aula (\Cref{tab-1}):

%--- código do quadro 1 ---%
\begin{table}[h!]
\centering
\begin{threeparttable}
\caption{Categorização do \textit{reality Queer Eye Germany}.}
\label{tab-1}
\begin{tabular}{p{0.50\linewidth} p{0.45\linewidth}}
\toprule
Tempo e Fala & Aspectos linguísticos e socioculturais \\
\midrule
49:58 -- Ayan: Queer eye kommt endlich nach Deutschland! Wie \textbf{geil} ist das \textbf{denn}? & - Gíria e expressões \newline- PM: denn \\ [1.5em]
49:53 -- David: Ich bin David, \textbf{Influencer, Hair and Makeup Artist}. & - Internacionalismos \newline - Profissão \\ [1.5em]
49:49 -- Jan: \textbf{Kein Plan} von Mode? Hier bin ich. Ich bin Jan-Henrik und ein bisschen \textbf{dandy}. & - Gíria e expressões \\ [1.5em]
49:42 -- Leni: Ich bin Leni und ich hab richtig \textbf{Bock}, Leuten zu helfen, \textbf{einfach} glücklich zu werden. & - Gíria e expressões \newline - PM: einfach \\ [1.5em]
49:37 -- \textbf{Aljoscha}: Ich bin Aljoscha, \textbf{Ernährungsberater}, Arzt und \textbf{Youtuber}. & - Profissão \newline - Nomes dos personagens e influência turca \\ [1.5em]
49:32 -- Ayan: Mein Name ist Ayan, ich bin \textbf{halt ‘n} Strahlemann und habe eine Agentur für \textbf{Interior Design}. & - PM: halt \newline - Abreviações \newline - Profissão \\ [1.5em]
49:24 -- David: Das ist \textbf{mehr als 'ne Make-over show}. Wir kommen zu dir nach Hause und helfen dir dabei, dein Leben aus \textbf{'ner} neuen Perspektiven zu betrachten. & - Comparação \newline - Abreviação \newline - Internacionalismos \\ [1.5em]
49:17 -- Aljoscha: Ich bin \textbf{so} stolz, Teil der Fab 5 zu sein und ich kann's kaum erwarten, endlich \textbf{loszulegen}. & - Advérbio de intensidade \newline - Derivação com zu \\ [1.5em]
49:17 -- Ayan: Mit Queer eye kann ich das Vorbild sein, das ich selbst als \textbf{schwuler Türke} nicht hatte. & - Discussão: ser gay em diferentes comunidades \\ [1.5em]
49:17 -- Leni: Ich bin \textbf{non-binary}. Es ist an der Zeit, dass Deutschland das versteht. & - Discussão: diferentes formas de identidade de gênero \\ [1.5em]
49:06 -- Danid: Wir wollen \textbf{ja} Diversität, oder? Ich musste noch \textbf{'n} Hairflip. Ich \textbf{hab} noch keinen Hairflip gebracht im ganzen Ding. & - PM: ja \newline - Abreviação \newline - Discussão: maior inclusão da diversidade na sociedade \\ 
\bottomrule
\end{tabular}
\source{As autoras.}
\end{threeparttable}
\end{table}

\textbf{Atividades propostas:}

\begin{itemize}
    \item \textbf{Pré-escuta:} Discutir o objetivo do programa a partir de conhecimentos anteriores e levantar hipóteses do que esperar a partir de uma imagem de divulgação da série. Também podem ser apresentados trailers da série original em inglês para contextualização;

    \item \textbf{Observação guiada:} Identificar vocabulário novo e indicar dificuldades, como velocidade de fala, assim como aspectos que ajudaram a interpretação como contexto, gestos, entre outros. Nesta fase, a professora pode apresentar perguntas guia, a depender do objetivo da atividade, como por exemplo: sobre o que as personagens estão falando? Quais são as suas primeiras impressões da série? O que mais te chamou atenção?;

    \item \textbf{Discussão:} A depender do tempo e objetivo da atividade, a docente pode selecionar temas específicos, por exemplo, gírias e expressões, o ensino das PMs, discussão sobre representatividade e diversidade ou mesmo aspectos linguísticos do idioma;

    \item \textbf{Produção:} Nesse caso seria interessante que a docente apresentasse alternativas para a produção escrita e oral baseadas nos objetivos da atividade. De qualquer forma, as estudantes podem ser convidadas a criar um novo personagem, adaptar as falas para o contexto brasileiro, escrever uma carta para participar do programa, desenvolver uma cena com uma possível participante que será ajudada pelos ``Fab 5'', entre outras.
\end{itemize}

As informações apresentadas na tabela, associadas às atividades propostas, revelam o potencial do episódio inicial de \textit{Queer Eye Germany} para fomentar discussões relevantes sobre identidade, diversidade e modos de interação afetiva na língua alemã. A presença de PMs como \emph{halt, denn} e \emph{einfach}, bem como de expressões coloquiais e vocativos, possibilitam trabalhar aspectos fundamentais do conhecimento pragmático: a suavização da fala, o reforço emocional e a construção de empatia em contextos sociais específicos \cite{schmidt2016, barron2016}. Além disso, a heterogeneidade dos apresentadores e suas formas de autodefinição promovem um espaço discursivo potente para refletir sobre pertencimento, representação e posicionamentos sociais, dimensões apontadas por \textcite{freitaghild2010} como centrais para o fomento de habilidades interculturais. Além disso, essas atividades podem incentivar as estudantes a assistirem os outros episódios do reality, ampliando as possibilidades de aprendizagem autônoma e motivadora.

Finalmente, ao elaborar produções orais e escritas inspiradas no episódio e nas suas experiências, como a criação de novos personagens ou adaptações para o contexto brasileiro, as estudantes não apenas praticam estruturas linguísticas, mas também exercitam sua criatividade e agência discursiva, fortalecendo sua formação como futuras docentes sensíveis às dimensões sociolinguísticas e culturais do ensino. A depender do contexto e interesse, as aprendizes podem ser incentivadas a preparar novas atividades didáticas com a série para as colegas, ampliando ainda mais os espaços de formação docente crítica.

\section{\textit{Too Hot to Handle:} Relacionamentos, gíria e negociação}
O \textit{reality Too Hot to Handle: Germany}, acompanha jovens adultos em um retiro de férias que incentiva o autocontrole para construir relações emocionais mais estáveis. Para tanto, proíbe-se, por meio de punição financeira, que as participantes mantenham contato físico íntimo. O formato gera conflitos interpessoais, alianças e conversas emocionais espontâneas sobre sexo, amor e relacionamentos, que apresentam linguagem coloquial, gírias e marcadores discursivos.

\textbf{Trecho sugerido:} Trecho de três minutos do segundo episódio, a primeira vez que as participantes são confrontadas com as punições pelos descumprimentos das regras. Na \Cref{tab-2}, apresentamos a transcrição de cenas com diversas falas das personagens com destaque para aspectos linguísticos e socioculturais que podem ser abordados em sala de aula.

%--- código do quadro 2 ---%
\begin{table}[h!]
\centering
\begin{threeparttable}
\caption{Categorização do reality \textit{Too Hot to Handle: Germany}.}
\label{tab-2}

\renewcommand{\arraystretch}{1.50} % AUMENTA O ESPAÇAMENTO ENTRE AS LINHAS

\begin{tabular}{p{0.45\linewidth} p{0.45\linewidth}}
\toprule
Cena & Aspectos linguísticos e socioculturais \\
\midrule
26:52 -- Emely: \textbf{What's happening?} \newline Laura: Wir kriegen \textbf{eine rechts und links} von Lana \newline Furkan: Irgendjemand war unartig. \textbf{Boa}, Lanna, bitte nicht. Ich zitter schon. \newline Tobias: \textbf{Vielleicht} erklärt sie uns nur die Preisliste. \newline Kevin: \textbf{Vielleicht} gibt es auch mal gute Neuigkeiten &
- Internacionalismos \newline - Gírias e expressões \newline - Marcadores discursivos \newline - Formas de expressar possibilidades e hipóteses \\[1.2em]

26:33 -- Stella: Ich \textbf{bin gespannt}, was wir alles Böses angestellt haben \newline Kevin: Bei uns war alles \textbf{grünen Bereich}. \newline Anna: \textbf{Des glaube ich euch nit} \newline Laura: Wir haben \textbf{schon} zugepackt. Aber ich weiß \textbf{ja} nicht, was wir ausgegeben haben im Endeffekt. \newline Stella: Ich weiß \textbf{ja}, dass ich nicht \textbf{ganz brav} war und \textbf{bin gespannt}, wie die anderen das auffassen. \newline Emely: Lana sees everything. Und \textbf{ich glaube}, ein Paar Leute werden noch \textbf{sauer werden}. &
- Expressões \newline - Variedade alemão da Áustria \newline - PM: schon, ja \newline - Formas de expressar ansiedade, opinião e concordância \newline - Internacionalismos \\[1.2em]

25:48 - Lana: Teilnehmer \textbf{meines Programms}. Willkommen in Palapa. \textbf{Wir müssen reden}. Ihr seid erst seit 16 Stunden im Retreat und habt schon sehr viele \textbf{Regeln gebrochen}. Kaum war das \textbf{Sexverbot} in Kraft, kam es zum ersten \textbf{Regelverstoß}, nach 32 min. \newline Alle: \textbf{Oh mein Gott. Was?} \newline Furkan: \textbf{He}, 32 min ist \textbf{halt schon} heftig, \textbf{ne?} \newline Onyi: Wie kann man nach 32 min so \textbf{horny} sein? \newline Stella: Da haben sich gleich zwei Leute irgendwo hin verzogen. \newline Kevin: Das ist \textbf{schon} respektlos. \newline Stella: Ich kann \textbf{ja} vielleicht mal auflösen, dass es nur ein Kuss war, aber Tobi war es nicht. \newline Anna: Ich \textbf{hab’s total} gut verheimlicht. Und dann kommt Stella e diz es einfach. \newline Lana: Dieser Regelverstoß führt zu einer Reduzierung textbf{des Preisgeldes} um 6.000 Euro. \newline Emely: Ey, textbf{ich schwör}, so teuer. \newline Onyi: Damit hätte man so textbf{geilen} Urlaub machen können. &
- Formalidade e informalidade com uso do genitivo \newline - Vocabulário de regra e quebra de regras \newline - Reações de surpresa \newline - Marcadores do discurso \newline - PM: halt, schon \newline - Internacionalismo \newline - Abreviatura \newline - Variedade da Áustria \newline - Expressões e gírias \\
\bottomrule
\end{tabular}
\source{As autoras.}
\end{threeparttable}
\end{table}


\textbf{Atividades propostas:}

\begin{itemize}
    \item \textbf{Pré-escuta:} Discutir sobre a premissa do programa a partir de conhecimentos anteriores e levantar hipóteses do que esperar a partir do título da série, uma imagem de divulgação ou mesmo um trailer. Também pode ser interessante contextualizar o tema das cenas a serem apresentadas para auxiliar na compreensão;
    \item \textbf{Observação guiada:} Identificar vocabulário novo e indicar dificuldades, como velocidade de fala, assim como aspectos que ajudaram a interpretação, gestos, sons, expressões faciais, entre outros. Da mesma forma do outro \textit{reality}, a professora pode apresentar algumas expressões base para a discussão, assim como perguntas norteadoras, para a compreensão e levantamento de aspectos a serem discutidos posteriormente;
    \item \textbf{Discussão:} A depender do tempo e objetivo da atividade, a docente pode selecionar temas específicos (ou uma relação entre eles). Sugere-se que, sempre que possível, sejam incluídos aspectos da recepção auditiva de textos informais e autênticos, assim como de variedades linguísticas e representações socioculturais diversas;
    \item \textbf{Produção:} A conversa com o robô Lana continua após o minuto vinte e quatro, assim as estudantes podem, em pequenos grupos e depois em plenária, apresentar suposições sobre o desenrolar da cena. Para tanto, pode-se mostrar a cena até o final (até o minuto 19:22) e discutir o resultado.
\end{itemize}

A análise do trecho selecionado de \textit{Too Hot to Handle: Germany} evidencia um contexto de interação marcado por fortes emoções, além de conflito e usos informais, representando um ótimo cenário para explorar elementos como gírias, variações linguísticas regionais e marcadores pragmáticos de surpresa, dúvida e indignação \emph{(ey, schon, halt, ne)}. As falas espontâneas e sobrepostas, somadas às reações visuais e expressões faciais das personagens, favorecem o trabalho com a interação oral enquanto prática multimodal, conforme discutido por \textcite{aguado2021}. Além disso, o formato do programa convida à problematização de normas de comportamento, negociação de regras e reações afetivas, o que abre espaço para reflexões interculturais sobre formas de expressar (ou reprimir) emoções no idioma – sendo esses aspectos essenciais do conhecimento pragmático \cite{kramsch1998, ishihara2010}. Ao propor simulações e discussões baseadas nas consequências das infrações e no uso da linguagem para lidar com conflitos, a docente estimula a criatividade linguística das estudantes, ao mesmo tempo que amplia seu repertório expressivo para lidar com situações de tensão ou confronto, recorrentes em contextos interacionais autênticos.

As atividades sugeridas a partir de \textit{Queer Eye Germany} e \textit{Too Hot to Handle: Germany} ilustram como materiais audiovisuais situados podem ser explorados pedagogicamente para o desenvolvimento da oralidade em contextos diversos de aprendizagem de ALA. As cenas selecionadas permitem o trabalho com gírias, marcadores pragmáticos e expressões culturais, promovendo uma escuta atenta e reflexiva, sensível aos múltiplos modos que compõem a interação oral. No entanto, é necessário reconhecer que o uso de gírias e expressões coloquiais traz também desafios de representatividade e compreensão, pois muitas dessas formas pertencem a grupos sociais ou comunidades específicas e não são necessariamente compartilhadas por todas as pessoas que falam o idioma. Esse aspecto pode ser discutido criticamente com as estudantes, de modo a evidenciar que, assim como ocorre com repertórios locais como o \emph{pajubá} no Brasil, as variedades linguísticas também expressam identidades, pertencimentos e exclusões. Assim, a abordagem dessas expressões deve ir além da simples aprendizagem lexical, promovendo reflexão sobre normas, variação e poder simbólico da linguagem \cite{bucholtz2005, kramsch1998}.

Do ponto de vista do conhecimento intercultural, é importante reconhecer que seu desenvolvimento não ocorre de forma linear, nem uniforme, entre os diferentes níveis de proficiência. Em níveis iniciais, o trabalho intercultural pode concentrar-se em atividades de sensibilização e comparação simples, que aproximem as estudantes de aspectos cotidianos e culturais da língua-alvo, permitindo o reconhecimento de semelhanças e diferenças básicas. Já em níveis mais avançados, é possível propor reflexões críticas mais complexas – envolvendo a análise do discurso, representações e práticas socioculturais \cite{byram1997, liddicoat2013}. Essa diferenciação evita expectativas irreais sobre o alcance da proposta e reforça a necessidade de adequar os objetivos pedagógicos à proficiência linguística e cognitiva das aprendizes.

Assim, embora os \textit{reality shows} ofereçam um terreno fértil para a observação da língua em uso real e da diversidade de interações sociais, seu potencial pedagógico precisa ser equilibrado com uma análise cuidadosa de suas limitações. É preciso considerar o nível de complexidade linguística e pragmática, que pode gerar sobrecarga cognitiva em aprendizes iniciantes, e as questões éticas e representacionais envolvidas no gênero, como a exposição a conflitos, a construção de identidades e a espetacularização de relacionamentos. Defendemos que o trabalho com \textit{reality shows}, deve ser visto não como solução única ou desafio demasiadamente complexo, mas como uma das possibilidades de ampliar o repertório didático, complementando outras práticas e gêneros textuais.

Tais propostas respondem à lacuna identificada neste estudo entre o ensino baseado em livros didáticos globais e a vivência da língua em uso situado. Essa lacuna, contudo, não deve ser compreendida como um problema absoluto, mas como um ponto de reflexão sobre as condições de produção e circulação dos materiais e sobre o papel docente como mediadora crítica desses recursos \cite{freitaghild2010, schmidt2016, aguado2021}. O uso de \textit{reality shows} é, nesse sentido, uma alternativa exploratória, que evidencia o potencial de práticas mais contextualizadas, mas que também depende da formação e da autonomia pedagógica das professoras para selecionar, adaptar e mediar conteúdos de forma sensível às especificidades do grupo. Por fim, com as sugestões didáticas apresentadas neste capítulo esperamos ter contribuído para a formação de professoras capazes de articular oralidade, multimodalidade e consciência intercultural de modo contextualizado, criativo e crítico, promovendo um ensino de alemão mais situado e sensível às práticas sociais reais da língua-alvo.

\section{Conclusão}
O presente artigo discute o ensino de ALA a partir da centralidade da oralidade, da multimodalidade e da autenticidade no desenvolvimento de habilidades comunicativas e interculturais. Partindo de uma análise crítica das limitações dos livros didáticos globais e da predominância de abordagens estruturalistas, argumentamos que a integração de gêneros audiovisuais espontâneos e contemporâneos como os \textit{reality show}, pode aproximar o ensino das práticas reais de uso da língua, ampliando a participação e o engajamento das aprendizes.

Contudo, reconhecemos que o uso de materiais autênticos, especialmente de gêneros midiáticos como os \textit{reality shows}, envolve desafios significativos -- questões como o nível de proficiência das turmas, a complexidade linguística, o risco de sobrecarga cognitiva e a presença de estereótipos culturais -- o que exige uma mediação pedagógica cuidadosa e contextualizada. O trabalho docente, nesse cenário, é o de equilibrar o desafio e a acessibilidade, oferecendo apoio gradual e \textit{input compreensível} \cite{krashen1982, vygotsky1978} para que as atividades mantenham o caráter espontâneo, mas continuem acessíveis e significativas para aprendizes de diferentes níveis. Logo, a proposta apresentada não pretende oferecer um modelo fechado, mas propor caminhos possíveis para integrar a oralidade autêntica e os multiletramentos na formação docente. Os \textit{reality shows} e gêneros digitais são, deste modo, compreendidos como recursos complementares e não substitutos do livro didático, capazes de fomentar reflexão crítica sobre linguagem, identidade e cultura. Sua aplicação deve, então, levar em conta o contexto de ensino, o perfil das estudantes e os objetivos de aprendizagem de cada turma, ajustando o nível de complexidade e as estratégias de mediação.

À luz dos resultados apresentados neste artigo, em especial ao potencial de maior engajamento das aprendizes, de ampliação do repertório linguístico-pragmático e da viabilidade do uso de gêneros multimodais com mediação docente, concluímos que a integração planejada de reality shows e outros gêneros digitais é um caminho promissor para o ensino e aprendizagem de ALA. Persistindo limites relacionados ao tempo de preparação, à seleção criteriosa de trechos e à necessidade de apoio gradual, ainda assim os achados indicam potencial para práticas mais situadas, críticas e inclusivas. Sugerimos, por fim, que investigações futuras aprofundem esses resultados em diferentes níveis de proficiência e contextos formativos, de modo a consolidar e refinar o desenho pedagógico aqui proposto.

\printbibliography\label{sec-bib}
% if the text is not in Portuguese, it might be necessary to use the code below instead to print the correct ABNT abbreviations [s.n.], [s.l.]
%\begin{portuguese}
%\printbibliography[title={Bibliography}]
%\end{portuguese}


%full list: conceptualization,datacuration,formalanalysis,funding,investigation,methodology,projadm,resources,software,supervision,validation,visualization,writing,review
\begin{contributors}[sec-contributors]
\authorcontribution{Marceli Cherchiglia Aquino}[conceptualization,datacuration,formalanalysis,investigation,methodology,projadm,resources,supervision,validation,visualization,writing,review]
\authorcontribution{Sofia Leria Alcoforado}[datacuration,formalanalysis,investigation,methodology,visualization,writing]
\end{contributors}

\begin{dataavailability}
\txtdataavailability{databody} % options: dataavailable, dataonly, databody, datanotav, nodata
\end{dataavailability}


\end{document}

