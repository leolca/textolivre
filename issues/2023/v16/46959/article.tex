% !TEX TS-program = XeLaTeX
% use the following command:
% all document files must be coded in UTF-8
\documentclass[portuguese]{textolivre}
% build HTML with: make4ht -e build.lua -c textolivre.cfg -x -u article "fn-in,svg,pic-align"

\journalname{Texto Livre}
\thevolume{16}
%\thenumber{1} % old template
\theyear{2023}
\receiveddate{\DTMdisplaydate{2023}{7}{24}{-1}} % YYYY MM DD
\accepteddate{\DTMdisplaydate{2023}{9}{20}{-1}}
\publisheddate{\DTMdisplaydate{2023}{11}{7}{-1}}
\corrauthor{Ana Larissa Adorno Marciotto Oliveira}
\articledoi{10.1590/1983-3652.2023.46959}
%\articleid{NNNN} % if the article ID is not the last 5 numbers of its DOI, provide it using \articleid{} commmand 
% list of available sesscions in the journal: articles, dossier, reports, essays, reviews, interviews, editorial
\articlesessionname{articles}
\runningauthor{Oliveira e Marques} 
%\editorname{Leonardo Araújo} % old template
\sectioneditorname{Daniervelin Pereira}
\layouteditorname{João Mesquita}

\title{Faticidade e polidez em vereditos de vídeos tutoriais do YouTube}
\othertitle{Phaticity and politeness in of YouTube video tutorials}
% if there is a third language title, add here:
%\othertitle{Artikelvorlage zur Einreichung beim Texto Livre Journal}

\author[1]{Ana Larissa Adorno Marciotto Oliveira~\orcid{0000-0003-1857-0207}\thanks{Email: \href{mailto:adornomarciotto@gmail.com}{adornomarciotto@gmail.com}}}
\author[1]{Joao Pedro Marques~\orcid{0000-0001-7312-9516}\thanks{Email: \href{mailto:joaop.marquess@gmail.com}{joaop.marquess@gmail.com}}}
\affil[1]{Universidade Federal de Minas Gerais, Faculdade de Letras, Belo Horizonte, MG, Brasil.}

\addbibresource{article.bib}
% use biber instead of bibtex
% $ biber article

% used to create dummy text for the template file
\definecolor{dark-gray}{gray}{0.35} % color used to display dummy texts
\usepackage{lipsum}
\SetLipsumParListSurrounders{\colorlet{oldcolor}{.}\color{dark-gray}}{\color{oldcolor}}

% used here only to provide the XeLaTeX and BibTeX logos
\usepackage{hologo}

% if you use multirows in a table, include the multirow package
\usepackage{multirow}

% provides sidewaysfigure environment
\usepackage{rotating}

% CUSTOM EPIGRAPH - BEGIN 
%%% https://tex.stackexchange.com/questions/193178/specific-epigraph-style
\usepackage{epigraph}
\renewcommand\textflush{flushright}
\makeatletter
\newlength\epitextskip
\pretocmd{\@epitext}{\em}{}{}
\apptocmd{\@epitext}{\em}{}{}
\patchcmd{\epigraph}{\@epitext{#1}\\}{\@epitext{#1}\\[\epitextskip]}{}{}
\makeatother
\setlength\epigraphrule{0pt}
\setlength\epitextskip{0.5ex}
\setlength\epigraphwidth{.7\textwidth}
% CUSTOM EPIGRAPH - END

% LANGUAGE - BEGIN
% ARABIC
% for languages that use special fonts, you must provide the typeface that will be used
% \setotherlanguage{arabic}
% \newfontfamily\arabicfont[Script=Arabic]{Amiri}
% \newfontfamily\arabicfontsf[Script=Arabic]{Amiri}
% \newfontfamily\arabicfonttt[Script=Arabic]{Amiri}
%
% in the article, to add arabic text use: \textlang{arabic}{ ... }
%
% RUSSIAN
% for russian text we also need to define fonts with support for Cyrillic script
% \usepackage{fontspec}
% \setotherlanguage{russian}
% \newfontfamily\cyrillicfont{Times New Roman}
% \newfontfamily\cyrillicfontsf{Times New Roman}[Script=Cyrillic]
% \newfontfamily\cyrillicfonttt{Times New Roman}[Script=Cyrillic]
%
% in the text use \begin{russian} ... \end{russian}
% LANGUAGE - END

% EMOJIS - BEGIN
% to use emoticons in your manuscript
% https://stackoverflow.com/questions/190145/how-to-insert-emoticons-in-latex/57076064
% using font Symbola, which has full support
% the font may be downloaded at:
% https://dn-works.com/ufas/
% add to preamble:
% \newfontfamily\Symbola{Symbola}
% in the text use:
% {\Symbola }
% EMOJIS - END

% LABEL REFERENCE TO DESCRIPTIVE LIST - BEGIN
% reference itens in a descriptive list using their labels instead of numbers
% insert the code below in the preambule:
%\makeatletter
%\let\orgdescriptionlabel\descriptionlabel
%\renewcommand*{\descriptionlabel}[1]{%
%  \let\orglabel\label
%  \let\label\@gobble
%  \phantomsection
%  \edef\@currentlabel{#1\unskip}%
%  \let\label\orglabel
%  \orgdescriptionlabel{#1}%
%}
%\makeatother
%
% in your document, use as illustraded here:
%\begin{description}
%  \item[first\label{itm1}] this is only an example;
%  % ...  add more items
%\end{description}
% LABEL REFERENCE TO DESCRIPTIVE LIST - END


% add line numbers for submission
%\usepackage{lineno}
%\linenumbers

\begin{document}
\maketitle





\begin{polyabstract}
\begin{abstract}
O objetivo deste estudo é investigar como/se a comunicação fática \cite{zuckerman2016}, bem como o exercício da polidez \cite{locher2005,miller2008}, atuam na emissão de vereditos não-oficiais \cite{austin1975, labinza2021}, emitidos em vídeos tutoriais do YouTube. Os dados foram coletados em abril e maio de 2020, por meio da lista dos vídeos mais visualizados na área de informática básica. A análise foi feita seguindo uma metodologia qualitativa e visou à identificação de recursos ligados à polidez relacional e à linguagem fática empregados nos vereditos. Os resultados revelaram que os vereditos foram construídos nesses vídeos com o objetivo de legitimar a informação comunicada, o que foi feito juntamente com o emprego de elementos ligados à polidez e ao trabalho relacional. O papel da comunicação fática foi também atestado como um traço significativo da comunicação não-oficial na esfera digital.    

\keywords{Faticidade \sep Polidez \sep Vereditos \sep Vídeos tutoriais}
\end{abstract}

\begin{english}
\begin{abstract}
The aim of this study is to investigate how/if phatic communication \cite{zuckerman2016} serves to foster the production of non-official verdicts \cite{austin1975, labinza2021, locher2005,miller2008} in YouTube tutorial videos. The data were collected in April and May 2020 among the most watched videos in the field of basic informational technology. The analysis followed the qualitative method, and aimed at identifying the resources linked to relational politeness and phatic language in the veredicts. Our results revealed that the verdicts were performed in these videos with the aim of legitimizing the information imparted, which was also done together with the resources associated with politeness and relational work. The role of phatic communication  was also asserted in the data as a significant feature of non-official communication in the digital sphere.

\keywords{Phaticity \sep Politeness \sep Verdicts \sep Tutorial videos}
\end{abstract}
\end{english}
% if there is another abstract, insert it here using the same scheme
\end{polyabstract}

\section{Introdução}\label{sec-intro}

Quando o conhecimento pronto para uso (ready-made knowledge) é o objetivo principal da comunicação, caso dos vídeos tutoriais do YouTube, tende a ser depositada uma maior confiança na autonomia do público espectador que, em geral, é composto por estranhos, representando, tipicamente, uma audiência abstrata. Os autores de vídeos tutoriais tendem, então, a produzir atos de fala do tipo vereditivos e a empregar recursos fáticos, associados ao engajamento social, com a finalidade de sugerir o pertencimento da audiência. Uma das maneiras de atingir esse propósito, é pelo exercício da polidez positiva \cite{brown1987, leech2014,miller2008}, ligada ao sentimento de pertencimento de grupo. Na proposta deste artigo, hipotetizamos que a comunicação de conhecimento pronto para uso, que circula na esfera digital, pode ser investigada da perspectiva da pragmática linguística, especificamente no que diz respeito ao fenômeno da polidez em interface com a faticidade.   

Ao tomar emprestado o termo de \textcite{malinowski1923}, \textcite{jakobson1960} introduziu a função fática como uma das seis funções da linguagem, associada à orientação para o contato, ou seja, para o canal físico de comunicação. A faticidade pode indicar, também, a preocupação em produzir efeitos gregários, para criar vínculos sociais pela linguagem. Para fazer isso, são tipicamente empregados enunciados considerados semanticamente 'vazios', utilizados a fim de enriquecer o pertencimento social, tal como ocorre com as saudações formulaicas em conversas casuais. Por essa razão, o conceito de faticidade combina o conteúdo semântico-referencial com os aspectos socialmente orientados da comunicação humana \cite{meltzer2003, hazaparu2015, zuckerman2016}.

De um ponto de vista mais amplo, os aspectos fáticos da interação atuam, de acordo com \textcite[p. 255]{simmel1949}, para gerar afiliação e para mitigar a “solidão do indivíduo”. Esses elementos levaram \textcite{ogden1946} a caracterizarem a comunicação fática como “um tipo de discurso em que os laços sociais são criados por uma mera troca de palavras” \cite[p. 315]{ogden1946}. Para além do aspecto gregário, \cite{nunez1983} afirma que a faticidade apresenta um componente dual, já que, ao reforçar os laços sociais, ela também pode enriquecer a comunicação de conteúdo informacional.

Na proposta deste estudo, o conceito de faticidade é combinado à análise dos elementos emergentes da situação de comunicação, particularmente aqueles ligados à polidez e ao trabalho relacional \cite{locher2005,miller2008}. Nessa perspectiva, a comunicação fática pode ser vista como um recurso cumulativo, forjado pelos interlocutores por meio da participação em interações sociais cotidianas, por exemplo, uma reunião de trabalho, um jantar entre amigos, ou nos casos em que a audiência pode ser ampla e abstrata, como em um vídeo tutorial. 

Diante desse panorama, as seguintes perguntas de pesquisa guiam este estudo: os vídeos tutorais do YouTube podem apresentar características ligadas aos atos de fala vereditivos? Se sim, qual o papel da faticidade e da polidez na produção desses vereditos? Mais especificamente, nossa hipótese é de que os vídeos tutoriais, produzidos por amadores para o YouTube, possam apresentar atos de fala vereditivos \cite{austin1975,labinza2021}, já que emitem um julgamento, ou uma avaliação, sobre um tópico de interesse do público, bem como possam conter elementos associados à comunicação fática \cite{yus2019,zuckerman2016} e à polidez \cite{locher2006,miller2008}, que servem para garantir engajamento com a audiência. 
 Feito esse panorama inicial, nas seções seguintes, discutimos o embasamento teórico do estudo.


\section{A faticidade e a polidez}\label{sec-conduta}

A comunicação fática é caracterizada, por um lado, pela criação de laços sociais em situações em que os objetivos gregários prevalecem sobre os demais, caso dos bate-papos informais \cite{coupland2010other, mccarthy2009}. Nessas situações, conforme \cite{laver1975}, o emprego de recursos fáticos é também uma forma de exercer a competência sociopragmática, na medida em que os interlocutores selecionam os recursos disponíveis de modo a “afirmar laços de união ou de solidariedade social, mas simultaneamente também limitando a força destes mesmos laços” \cite[p. 225]{laver1975}.

 Por outro lado, a faticidade também é registrada em contextos em que a comunicação de conteúdo informacional é a tônica, por exemplo, no caso de textos jornalísticos, como mostrado por \textcite{hazaparu2015}. De forma similar, na visão de \cite{laver1975}, o emprego de recursos fáticos é também um modo de exercer a competência socio-pragmática, na medida em que os recursos fáticos disponíveis são selecionados de modo a “afirmar laços de união ou de solidariedade social, mas simultaneamente também limitando a força destes mesmos laços” \cite[p. 225]{laver1975}.
 
Nos cenários em que tradicionalmente a comunicação referencial (ou informacional) é considerada hegemônica, os recursos fáticos são, em geral, empregados de duas formas principais: por meio da modalidade participativa e da modalidade discursiva, de acordo com \textcite{hazaparu2015}. Na modalidade participativa, a proximidade entre os interlocutores ocupa papel central, sugerindo cumplicidade e favorecendo a criação de uma cognição conjunta (common ground), que convida à coparticipação, principalmente por meio de trocas verbais, que visam ao engajamento social da audiência. 

Na modalidade discursiva, a faticidade é forjada no nível da estrutura interna do discurso. Essa modalidade pode ser atingida, segundo \textcite{hazaparu2015}, por meio da organização lógica do discurso, ainda que a interação com a audiência não seja simultânea. Em suma, e ainda mais importante para este estudo, em todas as modalidades apresentadas por \textcite{hazaparu2015}, os componentes interacionais são considerados fáticos quando visam a formar vínculos de participação com o interlocutor (real ou presumido), o que é feito pela combinação dos aspectos sociais da linguagem com o conteúdo informacional comunicado.

Por estar relacionada a expressões convencionalizadas, por exemplo, cumprimentos, perguntas retóricas e sinalizações de entendimento (back-channel clues), a faticidade permeia múltiplos contextos da comunicação humana. No âmbito jornalístico, por exemplo, esse aspecto corresponde ao conceito de “informação social”. Na perspectiva deste estudo, especula-se que o caráter fático da linguagem possa estar presente também mediante a emissão de vereditos não-oficiais, divulgados em ambiente digital, como ocorre com os vídeos tutorais produzidos por não-especialistas, no YouTube. 

No contexto digital, a faticidade ganha proeminência pois, nos termos de \textcite[p. 395]{miller2008}, cria-se uma “cultura fática”, na qual manter e enriquecer vínculos sociais pode ser tão ou mais importante do que comunicar ideias ou conteúdo informacional. Dessa forma, nos ambientes cibernéticos, a comunicação fática ocorre, por exemplo, por meio do emprego de variados elementos multimodais, tais como imagens, sons, emojis, stickers, além de alterações tipográficas \cite{oliveira2018,cunha2020teorias,carneiro2020}, que correspondem ao que \textcite{yus2019} caracteriza como “internet fática". 

A esse respeito, \textcite{yus2019} também postula que, embora a comunicação fática seja tradicionalmente considerada como insubstancial, ou seja, apresentando conteúdo proposicional pouco relevante, ela também pode exercer um papel gregário importante, ao veicular o que o autor chama de "conteúdo social", ligado à necessidade de permanente conexão entre os usuários de redes sociais.

Do ponto de vista do trabalho relacional, variados recursos de (im)polidez são empregados nas interações sociais, visando à manutenção e à projeção das imagens públicas dos interlocutores e, portanto, também fortalecendo os vínculos sociais temporários entre os participantes da interação. O trabalho de face, portanto, na proposta de \textcite{brown1987}, é feito por meio da projeção da face positiva dos interlocutores, ligada ao pertencimento de grupo e à solidariedade, bem como da preservação da face negativa, direcionada ao respeito pelo espaço físico e psicológico do outro, ou seja, ao direito à não-imposição.

Na visão de \textcite{miller2008}, a polidez é também condicionada por variáveis sociológicas.  Essas variáveis podem ser de dois tipos principais: 

\begin{itemize}
    \item Distância social entre falante e ouvinte (relação simétrica/horizontal);
    \item Poder relativo de falante e ouvinte (relação assimétrica/vertical).
\end{itemize}

Vista dessa forma, a distância social é uma função do grau de intimidade ou de proximidade entre falante e ouvinte. Por essa razão, ela é maior entre desconhecidos e menor entre colegas, familiares e amigos, por exemplo. Ao empregar estratégias de polidez, os falantes, em geral, reconhecem a distância social como um elemento importante, ligado ao encadeamento discursivo, e necessário para que a interação possa prosseguir, atingindo os objetivos comunicativos propostos.

Além disso, ao postular o Princípio de Polidez, \cite[p. 87]{leech2014} argumenta que a polidez “é uma restrição observada no comportamento comunicativo humano, que nos influencia a evitar a discordância ou a ofensa comunicativa e a manter ou a aumentar concordância ou a cortesia comunicativa”. Além desse aspecto, para os estudos da polidez, é também de primeira importância considerar que a noção de trabalho de face é um fenômeno dinâmico e emergente da interação, além de mutuamente construído \cite{culpeper2003,spencer2007,grainger2018}.

A noção de polidez é também consoante com a proposta teórica iniciada por \textcite{goffman1967}, em que o trabalho de face se refere às ações linguísticas (e não linguísticas), realizadas pelos participantes para “reivindicar seus valores sociais, ou para manter sua autoimagem, de forma considerada satisfatória para a interação” \cite[p. 65]{haugh2013}. Dessa forma, ao empreenderem o trabalho de face, os interlocutores lançam mão de estratégias de polidez que visam a reduzir antagonismos e objeções, colaborando para uma desejada harmonia interacional \cite{brown1987,kerbrat1992}.

A polidez atua, ainda, como um recurso de coesão social, diminuindo a agressividade e a possibilidade de confronto direto. Essa concepção de polidez reside na base da teoria clássica, que preconiza o falante racional e estratégico, hábil em empregar as estratégias de polidez de forma instrumental, com vistas à consecução de objetivos interacionais variados, bem como com a intenção de evitar ameaças de face \cite{brown1987,leech2014}.

Nas abordagens mais recentes, contudo, os atos de fala não são vistos como inerentemente ameaçadores de face, isso porque, ao lado dos atos ameaçadores, há também aqueles que enaltecem a face do interlocutor (face-enhancing acts ou face-maintaining acts), conforme observado por \textcite{kerbrat1992} e por \textcite{leech2014}. Assim sendo, outro aspecto importante nos estudos da (im)polidez é a ideia de que o comportamento harmônico dos falantes nem sempre representa uma centralidade. Nesses casos, a impolidez e a falsa polidez também obtêm destaque \cite{culpeper2016}, já que a linguagem pode ser usada para debochar, atacar, provocar e difamar o outro \cite{Cunha2019,carneiro2020,cunha2020teorias}.

De relevância para o trabalho de face e a polidez, é também a ideia de que diferentes objetivos comunicativos possam produzir diferentes tipos de avaliação sobre “a adequação discursiva e a conformidade social” (Spencer-Oatey, 2007, p.107-108) dos enunciados produzidos. Por essa razão, uma classificação amplamente aceita dos objetivos comunicativos engloba dois tipos principais de propósitos: 
\begin{enumerate}
    \item Os transacionais (por exemplo, realizar uma determinada tarefa);
    \item Os relacionais (por exemplo, enriquecer a solidariedade dentro do grupo).
\end{enumerate}

Nos dois tipos de objetivos, a noção de orientação fática é um componente subjacente, já que o aspecto gregário da comunicação é reconhecido. 

\section{A comunicação de informação e os vereditos}\label{sec-fmt-manuscrito}
Em sua teoria, \textcite{austin1975} apresenta os "vereditos", que geralmente expressam a avaliação de fatos e de valores com base em evidências ou em crenças. Os vereditos podem ser oficiais ou não-oficiais. A principal diferença reside no fato que os vereditos não-oficiais prescindem da apresentação de evidências. Os vereditos oficiais, por outro lado, expõem evidências e/ou fatos que são geralmente aceitos como verdades científicas, ou como conhecimento válido e autorizado. Ao atualizarem o conceito de vereditos, \textcite[p. 80]{labinza2021} postulam que, ao desempenhar esses atos de fala, os interactantes:  

\begin{enumerate}
    \item Pressupõem que o locutor esteja em posição de proferir o veredito, ou seja, que é competente para emitir julgamentos sobre o tema em questão.
    \item Comprometem o locutor com a verdade, a justiça ou a correção do seu veredito.
    \item Comprometem o locutor com o fornecimento de evidências ou de razões usadas para embasar os vereditos comunicados, caso esses elementos sejam solicitados pela audiência.
    \item Autorizam a audiência a confiar no julgamento do locutor e em seu comportamento verbal e não verbal subsequente: ou seja, a audiência sente-se licenciada para emitir o mesmo tipo de julgamento, ou utilizá-lo como premissa, ou como base, para tomar uma decisão futura \cite[p. 80]{labinza2021}.
\end{enumerate}

Conforme \textcite{labinza2021} salientam, o sucesso da comunicação de conteúdo informacional está ligado ao reconhecimento, pelo público, de que a informação comunicada é verdadeira e que deve, portanto, ser legitimada. Por isso, uma estratégia importante para alcançar o êxito da comunicação de informação reside na própria autoridade do locutor, que deve ser visto como portador de verdade e de conhecimento, seja ele científico, técnico, prático ou outro.

De todo modo, como o contexto imediato afeta a forma pela qual a informação é comunicada, podem ser necessários alguns ajustes para que esta seja bem-aceita pela audiência.  Esses ajustes envolvem, entre outros aspectos, o grau de familiaridade que a audiência tem com respeito ao tema tratado, bem como os tipos de laços sociais que a comunicação informacional é capaz de forjar \cite{miller2008, xie2021}. Por essa razão, na visão deste estudo, vídeos tutoriais produzidos por amadores para o YouTube podem conter elementos pragmáticos ligados aos atos de fala vereditivos, já que emitem um julgamento, ou uma avaliação, sobre um tópico de interesse do público, como se verá a seguir.


\section{Os vídeos tutoriais e a comunicação de informação {\textquotedbl}pronta para o uso{\textquotedbl}}\label{sec-formato}

O termo “tutorial” tem origem em “tutoria”, expressão usada em referência a um tipo de instrução, geralmente oferecida por um professor em formação a um grupo reduzido de alunos. Etimologicamente ligado à ideia de “indivíduo mais maduro ou mais treinado, designado para orientar outros em algum ponto de seus estudos” \cite{tarquini2019}, o termo “tutor” tornou-se bastante pervasivo na educação formal. Também muito comum nos ambientes digitais de aprendizagem remota, em vários níveis, é a figura do tutor, que exerce um papel específico, sendo responsável pela mediação entre os alunos e o professor/instrutor.

Mais recentemente, o público em geral tem recorrido aos chamados “vídeos tutoriais”, principalmente no YouTube, em busca de explicações ou de instruções sobre como resolver problemas do cotidiano, tais como, instalar um programa no computador, montar um móvel ou fazer uma maquiagem. Na perspectiva de \textcite[p. 148]{tarquini2019}, os vídeos tutoriais podem ser definidos “como vídeos de instrução assíncronos, que fornecem orientação passo a passo para atividades especializadas”. De forma similar, para \textcite{mogos2015}, o sucesso dos vídeos tutoriais do YouTube reside, fundamentalmente, em sua utilidade, determinada pelos próprios espectadores (ou consumidores) do canal. Ao fator utilidade, ainda segundo \textcite{mogos2015}, soma-se o nível de credibilidade e de confiança que os vídeos tutoriais podem despertar na audiência.

Do ponto de vista comunicativo, os vídeos tutoriais do YouTube apresentam um padrão estruturante para a comunicação de informação não-oficial. Isso ocorre porque, na chamada “modernidade líquida”, conforme apontou \textcite[p. 139]{knorr1997}, os indivíduos encontram-se em permanente transformação de suas relações sociais, que são substancialmente mediadas por um objeto (o computador ou o telefone celular, por exemplo). Nos referidos tutoriais, o objetivo da comunicação tende, então, a ser o de estabelecer/fortalecer conexões por meio de recursos digitais variados (imagem, som, entre outros), além de elementos linguísticos, que atuam para superar a ausência da comunicação face-a-face e em tempo real.

Além disso, os vídeos tutoriais, em geral, comunicam conhecimento “pronto-para uso” \cite{xie2021}, ou seja, são moldados frente às expectativas utilitárias e imediatistas da audiência. A esse respeito, de acordo com a página do YouTube, “os vídeos tutoriais são utilizados não só para partilhar conhecimentos, mas também para ajudar as empresas a comercializarem as suas marcas e a fornecerem mais documentação visual aos clientes” (disponível em \href{https://creatoracademy.youtube.com/page}{ }\url{https://creatoracademy.youtube.com/page}). No caso deste estudo, no entanto, os tutoriais apresentavam um caráter não-oficial, eram abertos e destinados à resolução de problemas cotidianos dos usuários, na área de instalação de sistemas operacionais. Eles podiam atingir, por essa razão, uma audiência bastante diversificada e ampla. 

Diante dos aspectos apontados nesta seção, passamos, a seguir, para a descrição dos procedimentos de coleta e de análise de dados empregados neste estudo.



\section{Metodologia de coleta e de análise de dados}\label{sec-modelo}

Os dados coletados foram retirados de canais públicos do YouTube, falados em língua portuguesa brasileira, contendo 20 mil inscritos ou mais, na época da coleta de dados (julho/21). Destinados à disseminação de conhecimento não-oficial no campo da informática, esses canais são produzidos por "amadores'', empregando-se aqui a terminologia proposta por \textcite{tarquini2019}. Neles, são oferecidos vídeos tutoriais acerca de tópicos ligados ao uso de dispositivos como notebooks, desktops, celulares e seus respectivos sistemas operacionais. Postados em abril e em maio de 2020, os quatro vídeos selecionados para a análise apresentavam entre 20 e 30 min de duração e tinham mais de 2 milhões de visualizações cada um. Esses números sugerem que os referidos tutoriais eram representativos desse tipo de artefato digital. Além disso, os vídeos figuravam entre os mais vistos sobre esse tema no YouTube naquele ano, o que nos autorizou a selecioná-los para análise como pertencendo a um padrão relevante desse tipo de material. 

Nossa análise centrou-se, exclusivamente, no conteúdo verbal dos vídeos e não se estendeu aos comentários da audiência sobre eles. Já que nosso objetivo era examinar a comunicação de conhecimento não-oficial pronto para uso, feita pela via digital, não nos concentramos na recepção dos vídeos e, por essa razão, os comentários a eles não foram analisados. Nosso indicativo de engajamento se restringiu ao número de visualizações e à lista dos vídeos mais assistidos, como detalhado anteriormente. 

Mediante os scripts completos dos tutoriais, os procedimentos de análise de dados envolveram a identificação e a classificação qualitativa dos seguintes elementos, adaptados de \textcite{hazaparu2015}: 

\begin{enumerate}
    \item Movimentos de início, de continuidade e de fechamento dos vídeos ou de suas seções.
    \item Estratégias de engajamento social, empregadas com vistas à criação de comunicação fática.
\end{enumerate}

Na amostra apresentada na análise, serão discutidos quatro extratos, considerados prototípicos e selecionados aleatoriamente.
Do ponto de vista ético, embora os vídeos tutoriais sejam públicos e com acesso aberto, optamos por não mencionar nenhuma informação que identifique seus autores, já que esses dados não são de interesse direto da pesquisa, que se centra em aspectos linguísticos recorrentes nos vídeos tutoriais, especificamente no que concerne ao elemento fático da comunicação de informação, à polidez e à ocorrência de atos de fala vereditivos. 


\section{Análise de dados}\label{sec-organizacao}
Nesta seção, por razões de espaço, será discutida uma amostra de cinco extratos, selecionados aleatoriamente dos dados, com o objetivo de identificar se/como os seguintes componentes estão presentes na comunicação de informação nos vídeos tutoriais analisados: 

\begin{enumerate}
    \item Movimentos de início, de continuidade e fechamento dos vídeos ou de suas seções.
    \item Estratégias de engajamento social, empregadas com vistas à criação de efeitos fáticos, conforme será demonstrado no Extrato 1, a seguir:
\end{enumerate}

\vspace{1ex}
Extrato 1
\begin{quote}
    \textit{Fala, conectados! Estamos aqui para um rápido tutorial para explicar para vocês como que faz a atualização do Windows 7 para o Windows 10 gratuitamente.} 
\end{quote}


	O Extrato 1 representa a abertura do vídeo e pode ser segmentado em duas etapas: 
 
 \begin{enumerate}
    \item O chamamento da audiência, feito pelo bordão "Fala, conectados!"
    \item A apresentação do objetivo central do tutorial: Estamos aqui para um rápido tutorial para explicar para vocês como fazer a atualização do Windows 7 para o Windows 10 gratuitamente, que serve para comprometer o locutor com o fornecimento de evidências utilizadas para embasar o veredito comunicado \cite{labinza2021}.

\end{enumerate}

	Nesse Extrato, o tutorial é também descrito como “rápido e gratuito”, duas características que podem ser consideradas importantes para garantir o êxito perante o público que, em geral, recorre aos vídeos tutoriais do YouTube em busca de soluções ágeis e simples para problemas do cotidiano \cite{mogos2015}, que também comprometem o locutor do vídeo com a correção do conteúdo por ele comunicado em seu veredito \cite{labinza2021}. 
 
Note-se, ainda, o emprego do nós inclusivo (estamos). Sobre esse uso, como um tutorial contém aspectos injuntivos, ligados ao "como fazer", a ocorrência de estamos, além de encorajar a filiação da audiência, caracteriza a modalidade participativa, como descrita por \textcite{hazaparu2015}. Utilizada para sugerir cumplicidade e favorecer a formação de uma cognição conjunta (common ground), a modalidade participativa é também essencial para a promover a orientação fática da comunicação (oral ou escrita) e pode ser identificada nesse e em outros pontos do vídeo. 

Além disso, considerando que a distância social é uma função do grau de intimidade ou de proximidade entre falante e ouvinte \cite{brown1987}, ao empregar estratégias de aproximação, o locutor reconhece a distância social como um elemento decisivo, principalmente porque a comunicação da informação do vídeo é feita para uma audiência em abstrato. Do ponto de vista da polidez, o emprego de “estamos” atua para mitigar imposições, maximizando a liberdade de ação da audiência e também forjando o sentimento de grupo \cite{leech2014, miller2008}. 

\vspace{1ex}
Extrato 2
\begin{quote}
    \textit{Mais operacional aí. O pessoal que acompanha tecnologia deve ter sentido uma estranheza quando eu falei isso porque faz tempo porque o Windows 10 tá disponível. Então parece que esse tutorial já sai meio velho. Mas o fato é que muita gente ainda usa o Windows 7 só que ele vai ter o suporte dele vai ser encerrado em 14 de janeiro de 2021.}
\end{quote}

Nesse trecho, o status deôntico do locutor do vídeo é afirmado, pois ele apresenta-se como alguém capaz de comunicar uma informação válida, ou seja, alguém que “entende de tecnologia” e, portanto, habilitado a emitir um veredito, ainda que não-oficial \cite{austin1975, labinza2021}. O público é, então, encorajado a confiar no julgamento do autor do vídeo para uma tomada de decisão futura (atualizar ou não o sistema operacional), legitimando-o, assim, como uma autoridade não-oficial acerca do tema abordado no vídeo. 

Observa-se, ainda, no trecho, que possíveis objeções quanto à legitimidade da informação comunicada no vídeo são antecipadas e/ou previamente refutadas pelo locutor (pessoal que acompanha tecnologia deve ter sentido uma estranheza quando eu falei isso. Mas o fato é muita gente ainda usa o Windows 7, ele vai ter o suporte dele vai ser encerrado em 14 de janeiro de 2021). Na mesma direção, a utilidade prática do tutorial é ressaltada, já que o argumento acerca da necessidade de atualização do Windows 10, tema central do tutorial, é fortalecido (o suporte dele vai ser encerrado em 14 de janeiro de 2021). Esse aspecto também ressalta o caráter “pronto para uso” do tutorial, reforçando a ideia de que os vídeos atuam para ajudar a audiência a resolver problemas de forma rápida e eficaz, ou seja, são centrados no caráter instrumental de seu conteúdo.  

\vspace{1ex}
Extrato 3
\begin{quote}
    \textit{A gente agora vai mostrar rapidinho como se faz porque realmente é muito simples. Primeiramente, vamos começar falando dos ingredientes que você precisa para fazer essa atualização.}
\end{quote}

Como pode ser observado, o Extrato 3 é construído para reforçar a centralidade temática do tutorial. Para fazer isso, o objetivo (ou a utilidade) do vídeo é novamente destacado: A gente agora vai mostrar rapidinho como se faz. Em seguida, o caráter procedimental do tutorial é ratificado por meio da analogia a uma receita, sugerida pelo emprego do termo “ingredientes”, bem como pela apresentação de seus passos sequenciais, indicados, principalmente, pelo uso de “primeiramente”. Esses elementos também contribuem para que a audiência reconheça a credibilidade do locutor do vídeo como alguém que comunica uma informação válida, útil e acessível, fatores importantes em tutoriais em geral \cite{tarquini2019}.

Dois atributos importantes para caracterizar um vídeo tutorial são ainda empregados nesse trecho: “muito simples” e “rapidinho”. Ambos remetem à eficácia desse tipo de comunicação, associada ao benefício direto esperado pela audiência, que deve ser atingido sem muito dispêndio de tempo ou de energia \cite{xie2021}. De forma similar, o emprego de “a gente” e “rapidinho” são também dignos de nota por assinalarem um tom informal e íntimo à interação, correspondendo ao que é classificado como “informação social”, ou seja, ligada ao caráter fático, presente na comunicação de informação \cite{zuckerman2016}. Esse aspecto também pode ser observado em:

\vspace{1ex}
Extrato 4
\begin{quote}
    \textit{A gente vai dar uma mencionada nisso tudo aqui daqui a pouco. Detalhe: se seu Windows é pirata e é craqueado, é provável que vai aparecer isso aqui igual e na hora da atualização a Microsoft vai pegar.}
\end{quote}


Nesse ponto, a afirmação A gente vai dar uma mencionada nisso tudo aqui (relativa às configurações do sistema operacional), representa uma interrupção do tópico central do tutorial (a atualização do sistema operacional), bem como o anúncio de sua retomada em breve (daqui a pouco). Empregado para potencialmente garantir que o espectador identifique a conexão entre temas diferentes e possa, portanto, mais facilmente acompanhar o vídeo, esse trecho também incentiva a afiliação da audiência, que é encorajada a permanecer conectada até o final do tutorial, ou, ao menos, até um ponto mais adiante de sua transmissão.

Em seguida, o trecho Detalhe: se seu Windows é pirata e é craqueado, é provável que vai aparecer isso aqui igual e na hora da atualização a Microsoft vai pegar realça a relevância do caráter injuntivo do vídeo tutorial, ou seja, uma vez mais, destaca a sua utilidade prática, ligada ao "como fazer". Ao mesmo tempo, o locutor imprime a si mesmo o status de alguém apto para aconselhamentos e para emissões de alertas sobre como seus seguidores devem (ou não) agir, destacando, novamente, sua autoridade não-oficial sobre o tema do tutorial. 

Também é digno de nota o fato de que o termo “detalhe” remete ao próprio discurso do locutor, (um detalhe acerca do que eu vou dizer nesse vídeo), ou seja, consubstancia o emprego de estratégias meta-discursivas e pragmáticas ligadas à validade do que é dito e, principalmente, do como é dito \cite{austin1975}. Ademais, ao lado da expressão “detalhe”, o uso de se seu Windows é craqueado aponta para elementos que, segundo \textcite{hazaparu2015}, indicam atenção destinada à audiência (seu Windows), bem como familiaridade com os termos informais da área (craqueado). Em combinação, esses componentes também atuam para promover faticidade, potencialmente permitindo que laços com a audiência sejam forjados com vistas à consecução de um objetivo transacional específico \cite{spencer2007}, nesse caso, instalar a atualização de um sistema operacional.

\vspace{1ex}
Extrato 5
\begin{quote}
    \textit{Agora vamos finalmente realmente pôr a mão na massa para ver a instalação como se faz. Primeiro passo, vou acessar aqui de novo porque já tem um cadinho.}
\end{quote}



No Extrato 5, os adverbiais “agora”, “finalmente” e “cadinho” situam a audiência tanto na temporalidade do próprio vídeo, como também no percurso da fala de seu autor. Nesse sentido, eles contribuem para tornar o vídeo tutorial mais acessível, mediante a apresentação de uma linha temporal facilmente seguida. Ressalta-se ainda a tentativa de reforçar a face positiva dos interlocutores, promovendo engajamento e afiliação por meio do trabalho relacional, o que fica também saliente no uso de termos indicativos de proximidade social e/ou de informalidade, tais como “pôr a mão na massa” e “cadinho”, além do emprego do intensificador “realmente”, que serve para reforçar a validade do conteúdo comunicado.

Como observado até aqui, é importante salientar que, mediante os extratos analisados nesta seção, a comunicação informacional apresenta um componente fático, que incide no alinhamento relacional dos interlocutores, mesmo que eles estejam apenas virtualmente conectados, ou sejam presumidos. Dito de outro modo, ao promover as condições necessárias para a comunicação de conteúdo informacional, ou seja, ao perseguir os objetivos transacionais da interação virtual, o componente fático é destacado pelos locutores dos vídeos, ainda que a audiência não esteja fisicamente presente, ou esteja apenas abstratamente definida.
Na próxima seção, faremos uma breve discussão dos dados apresentados, antes de nos encaminharmos para as considerações finais do estudo.


\subsection{Discussão}\label{sec-organizacao-latex}
Embora o conceito de faticidade e de pertencimento a uma comunidade digital estejam essencialmente correlacionados com a noção de convencionalização, na medida em que “quanto mais convencional for o significado de uma expressão particular, menos diretamente relacionado ele estará com a polidez individualizada” \cite[p. 176]{house1989}, nossos dados sugerem que houve espaço para algum grau de criatividade nos vídeos que constituem o corpus deste estudo. Conforme atestado em nossa análise, ao emitir vereditos não-oficiais, os locutores dos vídeos empregaram recursos linguísticos idiossincráticos, ligados à busca pela facticidade, tais como o uso de intensificadores e as marcas de postura avaliativa. Ademais, as variadas expressões de postura positiva, por eles manifestas, também sugerem que a criatividade verbal é uma função do esforço acentuado dos locutores dos vídeos para simbolicamente retribuírem à audiência pelo engajamento social nos vídeos, atestado pelo grande número de visualizações que eles receberam.

Quanto aos elementos ligados aos atos de fala vereditivos, em específico, foi possível identificar as estratégias linguísticas empregadas para reafirmar o status deôntico do locutor, que visavam afirmá-lo como um detentor de conhecimento válido, autorizando a emissão de vereditos, ainda que não-oficiais. Em conjunto, esses aspectos podem estar subjacentes ao êxito dos tutoriais aqui examinados, listados entre os mais visualizados do YouTube, conforme mencionado anteriormente neste texto. A seguir, serão tecidas as considerações finais deste estudo.


\section{Considerações finais}\label{sec-titulo}
Este texto objetivou lançar luz sobre o componente fático e relacional, presente na comunicação de informação, especificamente por meio de vídeos tutoriais. Na tentativa de investigar esse tema, foram analisados quatro vídeos, listados entre os mais visualizados no YouTube em 2020. Destinados à resolução de problemas práticos na área de informática básica, que afetam o usuário comum, tais como a atualização de sistemas operacionais, esses vídeos foram produzidos por amadores, que emitiram vereditos não-oficiais. 

Ao fazerem julgamentos e produzirem avaliações sobre o tema dos tutoriais, os locutores procuraram garantir a credibilidade do conhecimento comunicado, comprometendo-se explicitamente com a validade das informações veiculadas. Eles também lançaram mão de elementos ligados à faticidade e à polidez, empregados para evitar imposições e ameaças diretas de face, além de também servirem para antecipar possíveis objeções e encorajarem a filiação e o pertencimento de grupo. Diante desses elementos, nossos resultados indicam que a faticidade, em combinação com as noções de polidez, contribui para o engajamento social da audiência, ainda que esta seja constituída em abstrato. Nesse sentido, uma implicação importante deste estudo é que a criação e a manutenção de laços sociais são importantes para o êxito da comunicação de informação não-oficial, haja vista o alto número de visualizações dos referidos vídeos.

Outra implicação importante é que, nos extratos apresentados na análise, foi possível identificar pelo menos três elementos ligados à comunicação fática e à polidez. O primeiro, de caráter instrumental, é associado à explicitação da utilidade e da praticidade do tutorial. Afirmados em vários trechos dos vídeos, esses aspectos incentivam a audiência a engajar-se, ao anteciparem um benefício imediato, ligado à solução rápida de problemas específicos. O segundo elemento, associado à explicitação da conexão entre os tópicos apresentados no vídeo, assinala a tentativa de encorajar o público a permanecer conectado e a seguir os passos indicados, evitando o abandono da audiência e a consequente diminuição nos índices de visualização. O terceiro componente é relativo ao emprego de linguagem informal e criativa, e de termos de filiação, em que o público é diretamente evocado, o que potencialmente objetiva gerar proximidade e pertencimento social, valorizando o trabalho relacional. Em conjunto, esses elementos confirmam a nossa hipótese inicial de que, nos vídeos tutoriais do YouTube analisados, o objetivo da comunicação de informação também está ligado ao estabelecimento e ao fortalecimento de conexões sociais, necessárias para que o conteúdo informacional possa ser comunicado, mas também atingindo engajamento social e, como consequência, o maior número de visualizações possível. 

Finalmente, é importante salientar que, tendo em vista que os vídeos tutorias permitem não apenas comunicar conhecimento não-oficial, por meio da emissão de vereditos, mas também estreitar a comunicação com uma audiência em abstrato, formada por estranhos, a criação de laços fáticos parece ser fator substancial. Nesse sentido, acreditamos que nossos resultados possam abrir caminhos para que novas pesquisas surjam em áreas interfaciais à faticidade e à comunicação de conhecimento, dentro e fora do ambiente dos vídeos tutoriais do YouTube. Consideramos serem esses temas importantes para o pesquisador interessado em melhor entender a comunicação de conteúdo informacional nas mais variadas formas em que ela se apresenta na contemporaneidade.

\section{Agradecimentos}
Conselho Nacional de Desenvolvimento Científico e Tecnológico - CNPq. Processo número: 309492/2020-3.

\printbibliography\label{sec-bib}
% if the text is not in Portuguese, it might be necessary to use the code below instead to print the correct ABNT abbreviations [s.n.], [s.l.]
%\begin{portuguese}
%\printbibliography[title={Bibliography}]
%\end{portuguese}


%full list: conceptualization,datacuration,formalanalysis,funding,investigation,methodology,projadm,resources,software,supervision,validation,visualization,writing,review
\begin{contributors}[sec-contributors]
\authorcontribution{Ana Larissa Adorno Marciotto Oliveira}[conceptualization,formalanalysis,writing]
\authorcontribution{João Pedro Marques}[investigation,writing,review]
\end{contributors}



\end{document}

