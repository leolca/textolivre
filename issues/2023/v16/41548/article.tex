% !TEX TS-program = XeLaTeX
% use the following command:
% all document files must be coded in UTF-8
\documentclass[english]{textolivre}
% build HTML with: make4ht -e build.lua -c textolivre.cfg -x -u article "fn-in,svg,pic-align"

\usepackage{float}

\journalname{Texto Livre}
\thevolume{16}
%\thenumber{1} % old template
\theyear{2023}
\receiveddate{\DTMdisplaydate{2022}{10}{21}{-1}} % YYYY MM DD
\accepteddate{\DTMdisplaydate{2022}{12}{21}{-1}}
\publisheddate{\DTMdisplaydate{2023}{1}{31}{-1}}
\corrauthor{Jesús López-Belmonte}
\articledoi{10.1590/1983-3652.2023.41548}
%\articleid{NNNN} % if the article ID is not the last 5 numbers of its DOI, provide it using \articleid{} commmand 
% list of available sesscions in the journal: articles, dossier, reports, essays, reviews, interviews, editorial
\articlesessionname{articles}
\runningauthor{Marín-Marín et al.} 
%\editorname{Leonardo Araújo} % old template
\sectioneditorname{Hugo Heredia Ponce}
\layouteditorname{Thaís Coutinho}

\title{Influence of the application of a reading plan on motivation, emotional intelligence, fluency and reading comprehension in Spanish primary school students}
\othertitle{Influência da aplicação de um plano de leitura na motivação, inteligência emocional, fluência e compreensão leitora em alunos espanhóis do ensino fundamental}

% if there is a third language title, add here:
%\othertitle{Artikelvorlage zur Einreichung beim Texto Livre Journal}

\author[1]{José-Antonio Marín-Marín~\orcid{0000-0001-8623-4796}\thanks{Email: \href{mailto:jmarin@ugr.es}{jmarin@ugr.es}}}
\author[2]{Jesús López-Belmonte~\orcid{0000-0003-0823-3370}\thanks{Email: \href{jesuslopez@ugr.es}{jesuslopez@ugr.es}}}
\author[3]{Georgios Lampropoulos~\orcid{0000-0002-5719-2125}}
\author[2]{Antonio-José Moreno-Guerrero~\orcid{0000-0003-3191-2048}}
\affil[1]{Universidad de Granada, Facultad de Ciencias de la Educación, Departamento de Didáctica y Organización Escolar, Granada, España.}
\affil[2]{Universidad de Granada. Facultad de Educación, Economía y Tecnología. Departamento de Didáctica y Organización Escolar, Ceuta, España.}
\affil[3]{International Hellenic University, Department of Information and Electronic Engineering, Thessaloniki, Greece.}

\addbibresource{article.bib}
% use biber instead of bibtex
% $ biber article

% used to create dummy text for the template file
\definecolor{dark-gray}{gray}{0.35} % color used to display dummy texts
\usepackage{lipsum}
\SetLipsumParListSurrounders{\colorlet{oldcolor}{.}\color{dark-gray}}{\color{oldcolor}}

% used here only to provide the XeLaTeX and BibTeX logos
\usepackage{hologo}

% if you use multirows in a table, include the multirow package
\usepackage{multirow}

% provides sidewaysfigure environment
\usepackage{rotating}

% CUSTOM EPIGRAPH - BEGIN 
%%% https://tex.stackexchange.com/questions/193178/specific-epigraph-style
\usepackage{epigraph}
\renewcommand\textflush{flushright}
\makeatletter
\newlength\epitextskip
\pretocmd{\@epitext}{\em}{}{}
\apptocmd{\@epitext}{\em}{}{}
\patchcmd{\epigraph}{\@epitext{#1}\\}{\@epitext{#1}\\[\epitextskip]}{}{}
\makeatother
\setlength\epigraphrule{0pt}
\setlength\epitextskip{0.5ex}
\setlength\epigraphwidth{.7\textwidth}
% CUSTOM EPIGRAPH - END

% LANGUAGE - BEGIN
% ARABIC
% for languages that use special fonts, you must provide the typeface that will be used
% \setotherlanguage{arabic}
% \newfontfamily\arabicfont[Script=Arabic]{Amiri}
% \newfontfamily\arabicfontsf[Script=Arabic]{Amiri}
% \newfontfamily\arabicfonttt[Script=Arabic]{Amiri}
%
% in the article, to add arabic text use: \textlang{arabic}{ ... }
%
% RUSSIAN
% for russian text we also need to define fonts with support for Cyrillic script
% \usepackage{fontspec}
% \setotherlanguage{russian}
% \newfontfamily\cyrillicfont{Times New Roman}
% \newfontfamily\cyrillicfontsf{Times New Roman}[Script=Cyrillic]
% \newfontfamily\cyrillicfonttt{Times New Roman}[Script=Cyrillic]
%
% in the text use \begin{russian} ... \end{russian}
% LANGUAGE - END

% EMOJIS - BEGIN
% to use emoticons in your manuscript
% https://stackoverflow.com/questions/190145/how-to-insert-emoticons-in-latex/57076064
% using font Symbola, which has full support
% the font may be downloaded at:
% https://dn-works.com/ufas/
% add to preamble:
% \newfontfamily\Symbola{Symbola}
% in the text use:
% {\Symbola }
% EMOJIS - END

% LABEL REFERENCE TO DESCRIPTIVE LIST - BEGIN
% reference itens in a descriptive list using their labels instead of numbers
% insert the code below in the preambule:
%\makeatletter
%\let\orgdescriptionlabel\descriptionlabel
%\renewcommand*{\descriptionlabel}[1]{%
%  \let\orglabel\label
%  \let\label\@gobble
%  \phantomsection
%  \edef\@currentlabel{#1\unskip}%
%  \let\label\orglabel
%  \orgdescriptionlabel{#1}%
%}
%\makeatother
%
% in your document, use as illustraded here:
%\begin{description}
%  \item[first\label{itm1}] this is only an example;
%  % ...  add more items
%\end{description}
% LABEL REFERENCE TO DESCRIPTIVE LIST - END


% add line numbers for submission
%\usepackage{lineno}
%\linenumbers

\usepackage{siunitx}
\newcolumntype{C}{>{$}c<{$}}


\begin{document}
\maketitle

\begin{polyabstract}
\begin{abstract}
Reading is positioned as an important skill in people's daily lives. That is why work is done and time is dedicated to its development in educational centers through didactic proposals that are compiled in the Reading Plan. This research focuses on knowing the impact of the Reading Plan on various dimensions such as motivation, emotional intelligence, fluency and reading comprehension in primary school students from different educational centers in Spain. For this, a quantitative methodology has been followed with a pre-post quasi-experimental design with a control and an experimental group. 331 students enrolled between 5$^{th}$ and 6$^{th}$ grade of said educational stage participated in the study. The data has been collected through many validated instruments (Questionnaire of Motivated Strategies for Learning, BarOn Emotional Intelligence Inventory and the EMLE-TALE 2000) for the objectives pursued. After the statistical analysis, the results obtained reveal how the increase in emotional intelligence of the experimental group has improved the students’ motivation, both intrinsically and extrinsically, and this has favoured the acquisition of greater fluency and reading comprehension concerning the control group. It is concluded that motivation, emotional intelligence, fluency and reading comprehension are higher in experimental group than those achieved by students in the control group. This shows that the Reading Plan developed by the experimental group positively impacts the improvement of these dimensions in the students.

\keywords{Reading comprehension \sep Reading fluency \sep Motivation \sep Emotional intelligence}
\end{abstract}

\begin{portuguese}
\begin{abstract}
A leitura se posiciona como uma habilidade importante no cotidiano das pessoas. Por isso, trabalha-se e dedica-se tempo ao seu desenvolvimento nos centros educativos através de propostas didáticas que constam do Plano de Leitura. Esta pesquisa se concentra em conhecer o impacto do Plano de Leitura em várias dimensões, como motivação, inteligência emocional, fluência e compreensão de leitura em alunos do ensino fundamental de diferentes centros educacionais na Espanha. Para isso, seguiu-se uma metodologia quantitativa com um desenho pré-pós quase-experimental com um grupo controle e um experimental. Participaram do estudo 331 alunos matriculados entre o 5º e o 6º ano da referida etapa de ensino. Os dados foram recolhidos através de vários instrumentos validados (Questionário de Estratégias Motivadas para a Aprendizagem, Inventário de Inteligência Emocional BarOn e o EMLE-TALE 2000) para os objetivos perseguidos. Após a análise estatística, os resultados obtidos revelam como o aumento da inteligência emocional no grupo experimental melhorou a motivação dos alunos, tanto intrínseca quanto extrinsecamente, e isso favoreceu a aquisição de maior fluência e compreensão de leitura em relação ao grupo controle. Conclui-se que motivação, inteligência emocional, fluência e compreensão de leitura são maiores no grupo experimental do que aquelas alcançadas pelos alunos do grupo controle. Isso mostra que o Plano de Leitura desenvolvido pelo grupo experimental tem um impacto positivo na melhora dessas dimensões nos alunos.

\keywords{Compreensão de leitura \sep Fluência de leitura \sep Motivação \sep Inteligência emocional}
\end{abstract}
\end{portuguese}
% if there is another abstract, insert it here using the same scheme
\end{polyabstract}

\section{Introduction}\label{sec-intro}
The digital age has drastically affected the way people consume content, gain knowledge and communicate. Information can be acquired through various means using multimedia such as texts, images, videos and audio. Despite the rapid technological advancements, reading remains one of the main ways to get informed and educated and thus, it is deemed as an essential skill for students to develop and master as it not only leads to academic success but also helps maintain a decent quality of life in an ever-increasing literate society \cite{tan_reading_2020}. Within this context, virtual images, illustrations and information as well as electronic books and digital material can affect young students’ reading motivation and reading comprehension \cite{therole_2017}. Hence, in order to support reading specifically and literacy education, in general, several instructional tools that capitalize on digital technologies have been created. As e-content based learning is becoming more popular, the way students learn to read, write and access information is changing and simultaneously teaching and learning activities are leaning towards personalized learning and utilizing tools such as digital devices, e-books etc. in order to assist the educational process \cite{baron_reading_2017}.

\subsection{Reading comprehension}
Reading constitutes a fundamental cultural skill which can greatly impact one’s social life and is often correlated with their academic performance as most school subjects depend on their reading skills. Consequently, it is essential to cultivate, foster and promote the skill and will to read from early education as difficulties in learning to read can have severe consequences in one’s life \cite{hulme_learning_2013}.

More specifically, reading is multidimensional and consists of too many components to be characterized by a singular theory and therefore, various methods, theories and tools have been examined to assess and evaluate an individual’s reading ability. In the reading process, meaning that is based on text information, reading context as well as readers’ existing knowledge and experiences is created. Within this context, the role of reading comprehension is a fundamental element of education the impact of which, although only indirectly observed, affects students’ reading competence, performance and accomplishments \cite{pearson2017}.

Comprehension is accomplished when meaning is elicited through the interaction and involvement of the reader with the text information based on their current viewpoints, knowledge and experiences \cite{mcnamara_chapter_2009}. Reading comprehension can be described as the result of printed word identification and listening comprehension. Additionally, comprehension occurs as a mental representation of the text and its meaning and in order for comprehension to be achieved, several cognitive processes in the word, sentence and text levels are materialized. Thus, students’ cognitive abilities that support reading comprehension, such as verbal abilities and decoding skills are crucial.

As reading comprehension entails the integration of information and the reasoning and correlation of the related material, the way students read can diversely impact their language learning, such as reading rate and comprehension as well as their vocabulary and knowledge acquisition. Specifically, students gain more benefits in their learning when following an extensive reading schedule rather than an intensive reading one \cite{blachowicz_reading_2017}. When students start to learn to read, there is a slight correlation between spoken language and reading but as they grow up and develop their skills, this correlation increases and levels out and as a result, the spoken language comprehension is becoming the limiting factor in reading comprehension instead of word recognition. Hence, language comprehension and decoding are major determinants of reading comprehension, particularly, for students of young age \cite{hjetland_pathways_2019}. As readers become more mature, they acquire a wider vocabulary and a deeper understanding of learning strategies and they become better at summarizing, drawing conclusions, monitoring, adjusting as well as using background knowledge and the text structure. Therefore, in order for students to develop and improve their reading comprehension skills and proficiency, various reading strategies should be applied \cite{brookbank1999improving}. These strategies should i) involve making inferences, monitoring, summarizing, visualizing, predicting, generating and asking questions as well as activating and using background knowledge, ii) adopt reading comprehension theories (e.g. cognitive processes, mental representations, content literacy) and iii) follow a bottom-up, top-down or interactive model \cite{pourhosein_gilakjani_how_2016}.

\subsection{Reading fluency}
Difficulties in reading fluency bring about problems in reading comprehension of written texts which result in children’s school failure \cite{alvarez-canizo_reading_2020}. Reading fluency can be regarded as a skill which contributes towards the effective and efficient recognition or decoding of words that enables readers to construct the text meaning during silent reading comprehension. It is difficult to comprehend the meaning of the text when word recognition is slow and labored. In order to facilitate reading comprehension, reading fluency is deemed essential as it enables readers to focus on the text meaning by releasing their cognitive resources. Using multicomponent interventions, audiobooks and repeated reading is an efficient way to enhance reading fluency and comprehension \cite{stevens_effects_2017}. This is particularly true when it comes to students with learning disabilities in which case the early identification of reading disabilities and the provision of assistance in cultivating linguistic skills are crucial \cite{catts_early_2016}. The main three overlapping and interdependent factors of reading fluency are speed, accuracy and expression \cite{national2000teaching}. Students that manage to improve on these three areas will have better reading fluency, linguistic comprehension and word recognition which lead towards improved reading comprehension \cite{kim_text_2015}.

In addition to these factors, automaticity and prosodic components of reading affect students’ reading fluency and comprehension differently based on their age \cite{calet_cross-sectional_2015}. Moreover, difficulty in reading, expressing thoughts and distraction lead to poor reading skills in the early years of learning mostly due to inadequate attention control \cite{arrington_contribution_2014}. Furthermore, attention is a determinant in transforming written language into spoken language which subsequently improves reading fluency. Hence, focused, sustainable, divided and selective attention are prerequisites for effective learning. Using attention-enhancing activities can assist in improving good readers’ reading speed, comprehension, prosody and word recognition accuracy which are determining factors of reading fluency \cite{yildiz_relationship_2017}.

Reading fluency has a bidirectional relationship with reading comprehension as an increase in fluency most commonly associates with an increase in comprehension. When it comes to primary education students, reading fluency constitutes an indicator of reading comprehension while prosody is a significant predictor to fluent reading skills. Thus, by improving their reading skills, students simultaneously enhance their reading comprehension abilities \cite{basaran_okudugunu_2013}. As students move from primary to secondary education, their reading fluency continues to develop but as the narrative and expository texts are becoming more difficult to comprehend, thus more emphasis should be given on prosody and on adopting repeated reading methods in order to enhance students’ fluency \cite{alvarez-canizo_reading_2020}. Throughout the educational years, decoding skills are vital in both oral and silent reading fluency which are strongly related to list and text reading fluency as well as listening and reading comprehension. The relationships among all these differ predictably with development \cite{kim_developmental_2012}.

\subsection{Reading motivation}
Complex cognitive skills are required in order for students to develop their reading fluency and reading comprehension. In addition, enhanced reading motivation and engagement are conducive to increasing learning outcomes and reading achievements for students of all levels and different cultures. As cognitive skills and intrinsic reading motivation are important contributors in students’ reading comprehension performance and growth, they should be promoted throughout their education \cite{tan_reading_2020}.

Consequently, it is important to embed reading motivation programs within the curriculum to improve students’ reading skills and motivation and to prevent a decline in the motivation of young students from occurring as it might predict their future involvement in reading \cite{nevo_enhancing_2020}. In order for this to happen, teachers and school administrators should learn and apply the principles of reading motivation and engagement while devoting the necessary time to planning related learning activities and implement stimulating tasks which can enhance situational interest and create long term intrinsic motivation \cite{guthrie_influences_2006}. When supporting classrooms adopt learning strategies which utilize motivational and engagement practices, such as allowing students’ autonomy, choice and input into instruction, emphasizing the importance, benefits and usefulness of reading, collaborative literacy activities and competence support on text comprehension, better learning outcomes in terms of reading motivation and reading comprehension can be yielded. Furthermore, teachers’ behavior, attitude and involvement can drastically affect students’ reading motivation. Teachers should also be able to select and use suitable for the learning context and students’ age texts based on readability formulas in order to match students’ interests and capacities and trigger their reading motivation \cite{pearson2017}.

There is a bidirectional relationship between linguistic skills and reading motivation and this is particularly true for children of young age, as those that start reading young and on a regular basis, cultivate their word recognition, vocabulary, reading comprehension and enrich their general knowledge, becoming, thus, skillful readers and proficient learners in the future \cite{morgan_is_2007}. Regarding primary education students, motivation greatly contributes towards increased reading comprehension both concurrently and longitudinally as well as improved decoding skills, reading and verbal abilities. Nonetheless, based on students’ underlying personality traits, their ethnicity, their class time devoted to reading, their characteristics (e.g. age, gender, reading skills, socioeconomic status) and the different types of reading activities and contexts they are involved in \cite{ramos-navas-parejo_validation_2022}, their engagement level differs.

As motivation is essential in reading activities, reading motivation positively affects students’ reading comprehension and encourages them to perform better, take educated guesses, solve reading difficulties or problems and reduce comprehending anxiety. Students that are intrinsically motivated to read do so because they enjoy it and are interested in the reading process, hence there is a positive association and significant indirect effects between intrinsic motivation and reading comprehension when taking metacognitive knowledge and the reading amount into consideration. On the other hand, students who are extrinsically motivated read due to instrumental reasons. As a result, extrinsic motivation is not significantly related to their reading comprehension. In addition to reading motivation, situated motivation also acts as a predictor for general reading motivation growth \cite{guthrie_influences_2006}. Moreover, when taking into consideration students both with good and poor reading abilities as a whole, in contrast to extrinsic reading motivation which acts as a negative predictor for students’ reading literacy, intrinsic motivation and intrinsic reading motivation can have reciprocal impact on reading achievement. But when examining only students with excellent reading scores and abilities, extrinsic motivation is associated with variation in reading skills \cite{hebbecker_reciprocal_2019}.

Students’ active involvement, curious and competitive nature, grades, compliance, recognition and work avoidance are the main dimensions which explain the significant relationship among reading motivation, reading enjoyment, reading competence and reading behavior. Although reading motivation that encourages students to be actively involved and engaged in reading activities are positively correlated with students’ reading comprehension, competition-oriented reading motivation has a negative impact on and association with their overall reading comprehension. Students’ intrinsic motivation is essential in order for them to read willingly and frequently. Their intrinsic reading motivation is reflected in their enjoyment, belief in their competence as well as their eagerness and persistence in reading \cite{nevo_enhancing_2020}. As these aspects increase, so do their overall reading and writing skills as well as their positive attitudes towards language learning and reading comprehension in both first language and foreign language learning. Thus, fostering and promoting the intrinsic motivation of low ability readers is particularly important in order to encourage and support their efforts in improving their skills. Students’ reading motivation is the key factor in assisting their linguistics development from a young age as it affects their reading competence development, their reading fluency and their reading comprehension, performance and behavior positively and significantly \cite{nevo_enhancing_2020}.

\subsection{Emotional intelligence and reading comprehension}
Emotional intelligence constitutes a determining factor to one’s reading comprehension, reading fluency and reading motivation. Emotional intelligence can be regarded as a social intelligence type which contributes to one’s well-being as it refers to one’s verbal and nonverbal appraisal and expression of emotions, the accurate identification and efficient regulation of affect within oneself and others and the use of emotional information to improve one’s thinking, problem-solving and actions. In particular, emotional intelligence can be described as the ability to identify, comprehend, express, regulate and use emotions to accurately reason with emotions and to utilize emotions and emotional knowledge to improve thought \textcite{kotsou_improving_2019}. Emotional intelligence utilizes skills that combine intelligence with emotions, such as perceiving emotions, assessing emotions, managing emotions, comprehending emotions and emotional knowledge as well as utilizing emotions to facilitate one’s thought process \cite{mayer2004emotional}. Therefore, emotional intelligence greatly impacts one’s physical and psychological health, social and intimate relationships as well as work and academic achievements \cite{nelson_emotional_2011}.

When it comes to students’ reading achievement, there is a direct and positive relationship of emotional intelligence with reading comprehension, reading fluency, reading habits, vocabulary mastery and reading skills in general as students with higher emotional intelligence tend to perform better at reading tasks \cite{resmisari_correlation_2022}. Although intelligence quotient (IQ) is a determinative factor regarding students’ reading comprehension, emotional intelligence subscales such as intrapersonal and interpersonal abilities as well as stress management can also affect their reading proficiency \cite{ghabanchi2014correlation}.

As far as students’ emotional intelligence and reading comprehension are concerned, learning strategies and programs which are of medium length and use personality traits as an assessment means yield better outcomes and their impact is more evident in primary education students than those of secondary education \cite{puertas_molero_inteligencia_2019}. Furthermore, learning styles significantly influence the cultivation of one’s emotional intelligence. Specifically for primary education students, constructivist learning strategies and conducive learning environments should be adopted. Social perception can also contribute towards a student’s listening and reading comprehension \cite{froiland_emotional_2020} while emotional intelligence factors such as adaptability and inter-personality influence their linguistic competences. Additionally, emotional intelligence can be related to students’ Big Five personality traits and predict their social and academic performance \cite{perpina_marti_does_2020}.

Although the facilitation of emotional intelligence has the potential to improve students’ academic performance and well-being while reducing delinquency and mental health related problems, it still remains uncertain if emotional intelligence is not just the outcome of combining personality variables with general cognitive abilities. Nonetheless, it is important to note that emotional intelligence can be objectively evaluated and that the skills that comprise its construct can be taught and learned \cite{brackett_emotional_2011}.

\section{Justification and objectives}
As has been reflected, the promotion and development of reading habits is fundamental for the formative and maturing processes of the student. However, this has been affected in recent years due to the pandemic caused by COVID-19, which has strongly conditioned instructional actions \cite{corell-almuzara_covid-19_2021}. To alleviate the effects of the pandemic at the educational level, numerous remote resources have proliferated to be able to carry out teaching and learning processes in a ubiquitous and safe manner \cite{lopez-belmonte_co-word_2021}. In this sense, the pandemic at the educational level has led to a great appearance of didactic resources mediated by technology \cite{moreno-guerrero_flipped_2021}, to continue with the educational practice from various means and motivating and autonomous work environments \cite{marin-marin_steam_2021}, facilitating the task and adapting to the different capacities and needs of the students \cite{carmona-serrano_evolution_2021} through interactive and immersive practices \cite{lopez-belmonte_impact_2022}. As a result of this, this study presents the main findings after the design, implementation and development of a reading plan made by an educational center from an innovative point of view, taking into account the benefits of the exposed ICT.

The objectives that lead to the realization of this work are the following: a) To know the influence of the reading plan in the motivation of the students; b) Discover the determination of the reading plan in the emotional intelligence of the students; c) Demonstrate the impact of the reading plan on students' fluency and reading comprehension.

Likewise, to guide the study, the following research questions (RQ) are presented:
\begin{description}
\item[RQ1] Does the reading plan carried out influence the motivation of the students? 
\item[RQ2] Does the reading plan carried out influence the emotional intelligence of the students?
\item[RQ3] Does the reading plan carried out influence the reading fluency of the students?
\item[RQ4] Does the reading plan carried out influence the reading comprehension of the students?
\end{description}

\section{Methodology}

\subsection{Research design}

This study has been based on a quantitative research methodology. Specifically, a quasi-experimental research design of a pre-post nature has been deployed, with a control and experimental group.


\subsection{Participants}

In total, 331 primary school students from educational centers in Spain participated in the study. Of the group of students, 175 were boys and the rest are girls, aged between 9 and 11, enrolled in the 5th and 6th years of said educational stage. These students were divided into a control group (n = 203) and an experimental group ($n = 128$). These participants were selected through a purposive sampling technique.

\subsection{Instruments}

The data collection has been produced through various validated instruments that have adequate psychometric properties:
\begin{itemize}
    \item Motivated Strategies for Learning Questionnaire (MSQL) that analyzes the intrinsic and extrinsic motivation of students in the application of pedagogical strategies. This instrument is composed of two dimensions (intrinsic and extrinsic), of four items each \cite{bonanomi2020psychometric}.
    \item BarOn Emotional Intelligence Inventory: Youth Version (EQ-i:YV) that analyzes the emotional intelligence of students. The instrument is made up of 60 items, grouped into four dimensions (intrapersonal, interpersonal, stress management, and adaptability) \cite{serrano2017inteligencia}.
    \item Magellan Reading and Writing Scales (EMLE-TALE 2000), which is made up of four sub-tests: reading aloud (grapheme-phoneme conversion and fluency), reading comprehension, copying and dictation \cite{toro2000emle}.
\end{itemize}

\subsection{Procedure and data analysis}
This research has been developed in several phases. Firstly, it began with the request for the necessary permits to carry out the study in the different educational centers, as well as the request of the Ethics Committee by the competent University. Once the permissions and approval of said Committee were obtained, the configuration of groups (control and experimental) of the participating educational centers was carried out. Once the groups were articulated, the first measurement (pre-test) was carried out to find out the starting point of the students in the different dimensions analyzed. This measure was carried out in the month of September 2021. The different innovative training proposals included in the reading plan of said educational center were then carried out (\url{https://cutt.ly/8L8qoee}). Finally, the second measurement (post-test) was carried out, in order to contrast the impact that said plan has had. In this case, the measure was carried out in March 2022.

The data analysis has been carried out using the Statistical Package for the Social Sciences (SPSS) program in its version 28. Descriptive tests have been carried out, such as the mean, standard deviation, the standard error of the mean, asymmetry, kurtosis and the coefficient of variance to reveal the trend of the distribution. In addition, the Student's t-test was used to compare the means between the measurement moments.

\section{Results}
As shown in \Cref{tab01}, the measures achieved in motivation, emotional intelligence, reading fluency and reading comprehension are above the mean, both in the pretest and posttest measures of the control and experimental groups. An upward trend is observed in motivation and emotional intelligence in the experimental group between pretest and posttest measures, while in the control group, a downward trend is observed. On the other hand, in reading fluency and comprehension, an increasing trend is observed in both the control group and the experimental group. 

% \parbox[t]{2mm}{\multirow{3}{*}{\rotatebox[origin=c]{90}{rota}}}
\begin{table}[h!]
\centering
\small
\begin{threeparttable}
\caption{Descriptive analysis of the control group and the experimental group in the pretest and posttest measures.} 
\label{tab01}
\setlength{\tabcolsep}{3pt}
\begin{tabular}{ll*{8}{S}}
\toprule
 & & \multicolumn{8}{c}{Parameters} \\
 & \multirow{2}{*}{Dimensions} & \multicolumn{4}{c}{Pretest} & \multicolumn{4}{c}{Posttest} \\
 & & {M} & {SD} & {$S_{kw}$} & {$K_{me}$} & {M} & {SD} & {$S_{kw}$} & {$K_{me}$} \\
 \cmidrule{2-2} \cmidrule(lr){3-6} \cmidrule(lr){7-10}
 \parbox[t]{2mm}{\multirow{4}{*}{\rotatebox[origin=c]{90}{CON}}}
 & MOTIVATION & 4.25 & .526 & -1.19 & 1.43 & 3.67 & 1.03 & -.709 & -1.01 \\
 & INTELLIGENCE & 3.00 & .240 & -.581 & 1.60 & 2.73 & .472 & -.446 & -.899 \\
 & FLUIDITY & 107.7 & 23.3 & -.393 & 1.25 & 126.4 & 29.1 & -.234 & .637 \\
 & COMPRENSION & 1.71 & .959 & 1.428 & 1.53 & 2.07 & 1.40 & .895 & -.685 \\
 \midrule
 \parbox[t]{2mm}{\multirow{4}{*}{\rotatebox[origin=c]{90}{EXP}}}
 & MOTIVATION & 4.16 & .668 & -1.57 & 1.01 & 4.26 & .553 & -1.44 & 1.62 \\
 & INTELLIGENCE & 2.97 & .292 & -.091 & 1.50 & 3.03 & .365 & .632 & .980 \\
 & FLUIDITY & 103.4 & 33.8 & .218 & .405 & 137.8 & 41.2 & .187 & .300 \\
 & COMPRENSION & 1.88 & .660 & 2.304 & 10.85 & 3.33 & 1.40 & -.309 & -1.15 \\
\bottomrule
\end{tabular}
\notes{M=Mean; SD=Standard deviation; Skw=Skewness; Kme=Kurtosis.}
\end{threeparttable}
\end{table}


The degree of independence of the results obtained was measured with Student's t-statistic, although from two different perspectives. On the one hand, independent samples were analyzed, i.e., a comparison was made between the control and experimental groups, both in the pretest and posttest tests. The data show \Cref{tab02} that in the post-test measurements statistically significant differences were found between the control group and the experimental group. In this case, the measures achieved by the experimental group are higher than those of the control group.

\begin{table}[h]
\centering
\begin{threeparttable}
\caption{Study of the value of independence between independent samples with pretest and posttest. Student's t test for independent samples.}
\label{tab02}
\begin{tabular}{llC*{3}{S}}
\toprule
Dimensions &  & \mu(X_1 - X_2) & {$t_{n1+n2-2}$} & {gl} \\
\midrule
MOTIVATION & pre & .086 (4.25-4.16) & 1.315 & 329 \\
 & pos & -.589 (3.67-4.26) & -5.912** & 329 \\
INTELLIGENCE & pre & .033 (3.00-2.97) & 1.143 & 329 \\
 & pos & -.299 (2.73-3.03) & -6.102** & 329 \\
FLUIDITY & pre & 4.357 (107.76-103.41) & 1.386 & 329 \\
 & pos & -11.337 (126.47-137.80) & -.173 & 329 \\
COMPRENSION & pre & -.173 (1.71-1.88) & -11.33* & 329 \\
\bottomrule
\end{tabular}
\notes{µ(X1-X2); µ=mean difference; X1=control group; X2=experimental group; **. Significant correlation less than 0.01; *. Significant correlation between 0.05 and 0.01; n.s. Correlation not significant.}
\end{threeparttable}
\end{table}


On the other hand, related samples have been analyzed, i.e., between pretest and posttest of the control and experimental groups (\Cref{tab03}). These reveal statistically significant differences between the pretest and posttest measures of the control group in motivation and emotional intelligence. In this case, the means of the posttest measures are lower than the pretest measures. In addition, statistically significant differences are observed between the pretest and posttest measures of the experimental group and the control group in reading fluency and reading comprehension. In this case, the means of the posttest measures are higher than the pretest measures. In addition, the mean differences are higher in the experimental group than in the control group. 



\begin{table}[h]
\centering
\begin{threeparttable}
\caption{Study of the value of independence between dependent samples between the control group and the experimental group. Student's t test for related samples.}
\label{tab03}
\setlength{\tabcolsep}{3pt}
\begin{tabular}{llCSSSS}
\toprule
Dimensions &  & \mu (Y_1-Y_2) & {$t_{n1+n2-2}$} & {gl} & {SD} & {SEA} \\
\midrule
MOTIVATION & con & .578(4.25-3.67) & 7.484** & 202 & 1.10 & .077 \\
 & exp & -.098(4.16-4.26) & -1.397 & 127 & .798 & .070 \\
INTELLIGENCE & con & .271(3.00-2.73) & 7.378** & 202 & .524 & .036 \\
 & exp & -.061(2.97-3.03) & -1.855 & 127 & .373 & .032 \\
FLUIDITY & con & -18.704(107-126.47) & -9.352** & 202 & 28.49 & 2.000 \\
 & exp & -34.398(103.41-137.80) & -11.44** & 127 & 34.01 & 3.006 \\
COMPRENSION & con & -.360(1.71-2.07) & -4.041** & 202 & 1.268 & .089 \\
 & Exp & -1.445(1.88-3.33) & -10.79** & 127 & 1.515 & .134 \\
 \bottomrule
\end{tabular}
\notes{Note: µ(Y1-Y2); µ=mean difference; Y1=control group; Y2=experimental group; **. Significant correlation less than 0.01; *. Significant correlation between 0.05 and 0.01; n.s. Correlation not significant.}
\end{threeparttable}
\end{table}


As it is shown in \Cref{tab04}, the measures achieved in the pretest and posttest measures of the control and experimental groups in intrinsic and extrinsic motivation, it can be observed that in the control group the tendency is downward, while in the experimental group it is upward. In both cases, extrinsic motivation is higher than intrinsic motivation in the pretest and posttest measures.

The standard deviation is well below one point in all the study dimensions, both in the pretest and posttest measures, except in the posttest measures of the control group, which are slightly above 1. The students tend to agree on their answer. Most of the asymmetry and kurtosis values are between ±~1.96, considered within a normal distribution.



\begin{table}[h]
\centering
\begin{threeparttable}
\caption{Descriptive analysis of the control group and the experimental group in the pretest and posttest measures of motivation.}
\label{tab04}
\setlength{\tabcolsep}{3pt}
\begin{tabular}{ll*{8}{S}}
\toprule
 & & \multicolumn{8}{c}{Parameters} \\
 & \multirow{2}{*}{Dimensions} & \multicolumn{4}{c}{Pretest} & \multicolumn{4}{c}{Posttest} \\
 & & {M} & {SD} & {$S_{kw}$} & {$K_{me}$} & {M} & {SD} & {$S_{kw}$} & {$K_{me}$} \\
 \cmidrule{2-2} \cmidrule(lr){3-6} \cmidrule(lr){7-10}
 \parbox[t]{2mm}{\multirow{2}{*}{\rotatebox[origin=c]{90}{CON}}}
 & INTRINSIC & 4.18 & .642 & -.676 & .277 & 3.52 & 1.03 & -.385 & -1.16 \\
 & EXTRINSEC & 4.32 & .682 & -1.21 & 1.73 & 3.83 & 1.16 & -.676 & -1.18 \\
 \midrule
 \parbox[t]{2mm}{\multirow{2}{*}{\rotatebox[origin=c]{90}{EXP}}}
 & INTRINSIC & 3.98 & .792 & -1.19 & 1.77 & 4.06 & .793 & -1.29 & 1.90 \\
 & EXTRINSEC & 4.35 & .708 & -1.63 & 1.83 & 4.47 & .574 & -1.56 & 1.79 \\
\bottomrule
\end{tabular}
\notes{Note: M=Mean; SD=Standard deviation; Skw=Skewness; Kme=Kurtosis.}
\end{threeparttable}
\end{table}


The degree of independence of the results obtained was measured with Student's t-statistic, although from two different perspectives. On the one hand, independent samples were analyzed, i.e., a comparison was made between the control and experimental groups, both in the pretest and posttest tests. The data reveal \Cref{tab05} that there are statistically significant differences in the posttest measures between the control group and the experimental group, both in extrinsic motivation and intrinsic motivation. In both cases, the mean of the experimental group is higher than that of the control group.

\begin{table}[h]
\centering
\begin{threeparttable}
\caption{Study of the value of independence between independent samples with pretest and posttest. Student's t test for independent samples on motivation.}
\begin{tabular}{llCSS}
\toprule
Dimensions &  & \mu (X_1-X_2) & {$t_{n1+n2-2}$} & {gl} \\
\midrule
INTRINSIC & pre & .198(4.18-3.98) & 2.497* & 329 \\
 & pos & -.539(3.52-4.06) & -5.045** & 329 \\
EXTRINSEC & pre & -.024(4.32-4.35) & -.316 & 329 \\
 & pos & -.640(3.83-4.47) & -5.794** & 329 \\
\bottomrule
\end{tabular}
\notes{$\mu=$mean difference; X1=control group; X2=experimental group; **. Significant correlation less than 0.01; *. Significant correlation between 0.05 and 0.01; n.s. Correlation not significant.}
\label{tab05}
\end{threeparttable}
\end{table}

On the other hand, related samples were analyzed between the pretest and posttest of the control and experimental groups (\Cref{tab06}). In this analysis, statistically significant differences were observed in the control group, both in intrinsic and extrinsic motivation. That is, the mean differences between pretest and posttest measures in the control group are significant. In this case, the pretest means are higher than the posttest measures.

\begin{table}[h]
\centering
\begin{threeparttable}
\caption{Study of the value of independence between dependent samples between the control group and the experimental group. Student's t-test for related samples on motivation.}
\label{tab06}
\begin{tabular}{llCSSSS}
\toprule
Dimensions &  & \mu(Y_1-Y_2) & {$t_{n1+n2-2}$} & {gl} & {SD} & {SEA} \\
\midrule
INTRINSIC & con & .660(4.18-3.52) & 8.326** & 202 & 1.14 & .080 \\
 & exp & -.078(3.98-4.06) & -.849 & 128 & 1.04 & .092 \\
EXTRINSEC & con & .496(4.32-3.83) & 5.626** & 202 & 1.25 & .088 \\
 & Exp & -.119(4.36-4.47) & -1.737 & 128 & .776 & .068 \\
\bottomrule
\end{tabular}
\notes{µ=mean difference; Y1=control group; Y2=experimental group; **. Significant correlation less than 0.01; *. Significant correlation between 0.05 and 0.01; n.s. Correlation not significant.}
\end{threeparttable}
\end{table}

In this dimensional analysis of emotional intelligence \Cref{tab07}, the means reached in the pretest and posttest measures of the control group and the experimental group show that in the control group the tendency is downward, while in the experimental group it is upward. The analysis of each of the dimensions of emotional intelligence shows that in the pretest measures of the control group, the Mood dimension is the one with the highest mean, while the Stress dimension is the one with the lowest mean. In the pretest measures of the experimental group, the dimension with the highest mean is FEELING, while the dimension with the lowest mean is STRESS. In the posttest measures of the control group, the dimension with the highest mean is INTER, while the dimension with the lowest mean is INTRA. In the posttest measures of the experimental group, the dimension with the highest mean is INTER. On the other hand, the dimension with the lowest mean is INTRA.

The standard deviation is well below one point in all the study dimensions, both in the pretest and posttest measures. In this case, it can be considered that there is no response dispersion. The student body tends to coincide in its response. Most of the values of skewness and kurtosis are between $\pm 1.96$.

\begin{table}[h]
\centering
\begin{threeparttable}
\caption{ Descriptive analysis of the control group and the experimental group in the pretest and posttest measures of emotional intelligence.}
\label{tab07}
\setlength{\tabcolsep}{3pt}
\begin{tabular}{ll*{8}{S}}
\toprule
 & & \multicolumn{8}{c}{Parameters} \\
 & \multirow{2}{*}{Dimensions} & \multicolumn{4}{c}{Pretest} & \multicolumn{4}{c}{Posttest} \\
 & & {M} & {SD} & {$S_{kw}$} & {$K_{me}$} & {M} & {SD} & {$S_{kw}$} & {$K_{me}$} \\
 \cmidrule{2-2} \cmidrule(lr){3-6} \cmidrule(lr){7-10}
 \parbox[t]{2mm}{\multirow{5}{*}{\rotatebox[origin=c]{90}{CON}}} & MOOD STATUS & 3.36 & .355 & -.955 & .663 & 2.98 & .649 & -.570 & -1.15 \\
 & ADAPTA & 3.04 & .430 & -.816 & 1.105 & 2.80 & .642 & -.073 & -1.25 \\
 & STRESS & 2.56 & .465 & .028 & .117 & 2.36 & .498 & .751 & .071 \\
 & INTER & 3.26 & .400 & -.330 & -.112 & 3.04 & .691 & -.477 & -1.19 \\
 & INTRA & 2.66 & .509 & -.208 & .769 & 2.34 & .551 & .269 & .079 \\
 \midrule
 \parbox[t]{2mm}{\multirow{5}{*}{\rotatebox[origin=c]{90}{CON}}} & MOOD STATUS &
 3.38 & .416 & -1.37 & 1.23 & 3.36 & .454 & -1.35 & 1.60 \\
 & ADAPTA & 3.08 & .498 & -.466 & .124 & 3.18 & .523 & -.216 & -.603 \\
 & STRESS & 2.37 & .536 & .588 & .216 & 2.54 & .603 & .750 & .067 \\
 & INTER & 3.31 & .401 & -.396 & -.152 & 3.38 & .486 & -.981 & .697 \\
 & INTRA & 2.51 & .655 & .013 & -.022 & 2.52 & .744 & .139 & -.358 \\
 \bottomrule
\end{tabular}
\notes{M=Mean; SD=Standard deviation; Skw=Skewness; Kme=Kurtosis.}
\end{threeparttable}
\end{table}



The degree of independence of the results obtained was measured with Student's t-statistic, although from two different perspectives. On the one hand, independent samples were analyzed, that is, a comparison was made between the control and experimental groups, both in the pretest and posttest. The data show (\Cref{tab08}) statistically significant differences in all dimensions of emotional intelligence in the posttest measures. In all cases, the means of the experimental group are higher than those of the control group.

\begin{table}[h]
\centering
\begin{threeparttable}
\caption{Study of the value of independence between independent samples with pretest and posttest. Student's t-test for independent samples on emotional intelligence.}
\label{tab08}
\begin{tabular}{llCSS}
\toprule
Dimensions &  & \mu(X_1-X_2) & {$t_{n1+n2-2}$} & {gl} \\
\midrule
MOOD STATUS & pre & -.020(3.36-3.38) & -.481 & 329 \\
 & pos & -.379(2.98-3.36) & -5.778** & 329 \\
ADAPTA & pre & -.043(3.04-3.08) & -.840 & 329 \\
 & pos & -.383(2.80-3.18) & -5.669** & 329 \\
STRESS & pre & .186(2.56-2.37) & 3.347* & 329 \\
 & pos & -.174(2.36-2.54) & -2.858* & 329 \\
INTER & pre & -.047(3.26-3.31) & -1.045 & 329 \\
 & pos & -.338(3.04-3.38) & -4.828** & 329 \\
INTRA & pre & .152(2.66-2.51) & 2.363* & 329 \\
 & pos & -.186(2.34-2.52) & -2.607* & 329 \\
\bottomrule
\end{tabular}
 \notes{$\mu$=mean difference; $X_1$=control group; $X_2$=experimental group; **. Significant correlation less than 0.01; *. Significant correlation between 0.05 and 0.01; n.s. Correlation not significant.} 
\end{threeparttable}
\end{table}

On the other hand, related samples were analyzed, that is, between the pretest and posttest of the control and experimental groups (\Cref{tab09}). Statistically significant differences are observed in all dimensions of the control group. The means achieved in the pretest measures are higher than those achieved in the posttest measures. In the ADAPTA and STRESS dimensions of the experimental group, statistically significant differences are observed. The measures reached in the post-test measures are higher than those reached in the pre-test measures.

\begin{table}[h]
\centering
\begin{threeparttable}
\caption{Study of the value of independence between dependent samples between the control group and the experimental group. Student's t-test for related samples on emotional intelligence.}
\label{tab09}
\begin{tabular}{llCSSSS}
\toprule
Dimensions &  & \mu(Y_1-Y_2) & {$t_{n1+n2-2}$} & {gl} & {SD} & {SEA} \\
\midrule
MOOD  STATUS & con & .382(3.36-2.98) & 8.052** & 202 & .676 & .047 \\
 & exp & .023(3.38-3.36) & .586 & 128 & .452 & .040 \\
ADAPTA & con & .237(3.04-2.80) & 4.881** & 202 & .694 & .048 \\
 & exp & -.102(3.08-3.18) & -2.087* & 128 & .554 & .049 \\
STRESS & con & .197(2.54-2.36) & 4.574** & 202 & .616 & .043 \\
 & exp & -.163(2.37-2.54) & -3.066* & 128 & .603 & .053 \\
INTER & con & .222(3.26-3.04) & 4.347** & 202 & .729 & .051 \\
 & exp & -.068(3.31-3.38) & -1.588 & 128 & .485 & .042 \\
INTRA & con & .320(2.66-2.34) & 6.594** & 202 & .691 & .048 \\
 & exp & -.018(2.51-2.52) & -.236 & 128 & .873 & .077 \\ 
\bottomrule
\end{tabular}
\notes{µ=mean difference; Y1=control group; Y2=experimental group; **. Significant correlation less than 0.01; *. Significant correlation between 0.05 and 0.01; n.s. Correlation not significant.}
\end{threeparttable}
\end{table}


\section{Discussion and conclusions}

As in any learning, for the results to be as desired, a series of factors must come together. In the specific case that concerns us, reading, there are several elements that influence students to become highly motivated avid readers. Among other factors, reading fluency, comprehension or motivation, both intrinsic and extrinsic, are essential. In the case of this study, in addition to those mentioned, emotional intelligence is included as a global variable that acts on the other elements, since the personal perception that an individual has about himself generates a great impact on physical and psychological health, in social and intimate relationships, as well as work and academic achievements \cite{nelson_emotional_2011}. In this sense, according to the results of this study, the development of the activities programmed in the Reading Plan of the experimental center has considerably improved the dimensions that encompass the emotional intelligence construct. This fact can be related to the ascending results that have been obtained in fluency, reading comprehension or motivation in the experimental center.

As \textcite{perpina_marti_does_2020}, emotional intelligence factors, such as adaptability and interpersonal qualities, influence the language skills of individuals in addition to predicting their social and academic performance. The data obtained is in line with other recent studies \cite{ramos-navas-parejo_uso_2020}, since a higher value in emotional intelligence in the group of students in the experimental group has directly influenced the improvement of all the measured reading processes. On the other hand, in the control group, where emotional intelligence has decreased with respect to the pretest, the other dimensions evaluated, although they rise slightly, are in no case higher than those of the experimental group, where there is a significant increase with respect to the pretest.

With regard to the motivation variable, the execution of the Reading Plan has allowed students to generate greater intrinsic and extrinsic motivation. This fact favors the reading process and the promotion of readers with constant reading habits. A review of the activities proposed in the Reading Plan makes it possible to identify how teachers have designed them to increase situational interest and create long-term intrinsic motivation, in line with what \textcite{guthrie_influences_2006}. From this perspective, the design and implementation of activities to promote the reading habit is shown as a crucial element for the student to develop a personal interest in reading that transcends the academic to the recreational sphere. The latter is necessary to forge reading habits that continue into adulthood. However, a high extrinsic motivation, as can be verified by the results obtained, is important for educational stages such as primary education where the study has been carried out \cite{pearson2017}, but for higher stages where this motivation is decreasing, it is essential that intrinsic motivation be strong. For this reason, cultivating intrinsic motivation is a factor to take into account to create long-term reading habits. 

Another of the elements studied was fluency and reading comprehension. The interrelationship between these variables has been made clear in the study by verifying how both rise both in the control group and in the experimental group. According to the results of recent studies \cite{alvarez-canizo_reading_2020}, a greater reading fluency (speed, precision and expression) will favor the person to improve linguistic comprehension and word recognition that lead to better reading comprehension. This will result in higher academic performance and less school failure. In this sense, although in both groups there is an increase in these variables compared to the pretest study, it is in the experimental group that there is a much more significant increase. This implies that the activities developed by the Reading Plan of this center are highly positive for the development and improvement of reading fluency and comprehension.

The results of the study support the coherence in the programming and in the activities developed for the promotion of reading in the students. In this way, the improvement in motivation in students is evident, both intrinsically and extrinsically, as well as in the main aspects related to emotional intelligence and reading fluency and comprehension.

It is concluded that motivation, emotional intelligence, fluency and reading comprehension are higher in the experimental group than those achieved by the students in the control group. This shows that the Reading Plan developed by the experimental group has a positive impact on the improvement of these dimensions in the students. Specifically, in the students of the experimental group, the means of motivation and emotional intelligence increased slightly, while in the students of the control group they decreased from the pre-test to the post-test. The significant increase occurs in fluency and reading comprehension in the experimental group, being much higher in the post-test measures than in the pre-test measures. This fact also occurs in the control group, although the increase is not as high as that which occurs in the experimental group.

Similarly, extrinsic motivation and intrinsic motivation are higher in the experimental group than in the control group. That is, the actions carried out in the Reading Plan developed in the experimental group produce an improvement in extrinsic and intrinsic motivation. It is significant how motivation, both intrinsic and extrinsic, rises slightly in the students of the experimental group from the pre-test to the post-test, while it decreases in the students of the control group. In all cases, extrinsic motivation is superior to intrinsic motivation.

Regarding the dimensions of emotional intelligence, they are also higher in the experimental group than in the control group. That is to say, the actions deployed in the Reading Plan of the experimental group cause an improvement in mood, adaptability, stress management, interpersonal competence and intrapersonal competence. It is significant how the dimensions of emotional intelligence rise slightly or remain the same in the students of the experimental group from the pre-test to the post-test, while it decreases in the students of the control group.

This study presents different limitations that will be taken into consideration to determine future lines of action. On the one hand, only the influence of a specific Reading Plan has been analysed. That is, the findings are conditioned to the proposals included in such plan, so what is exposed in this work should be taken with caution, since a variation of the didactic proposals can vary the results obtained. On the other hand, the focus of study, since it is only aimed at the population of Spanish primary school students. After these limitations, future lines of research are proposed to design various training proposals to promote and work on reading in order to verify which is the most effective. Likewise, it is intended to increase the population range, extrapolating the sample to other educational stages such as secondary education.

This research reflects a number of implications, both theoretical and practical. At a theoretical level, this work has led to increased interest in delving into the state of the art, reflecting the current state of the art and showing readers interested in this thematic focus what are the new knowledge bases with which to support future research related to the promotion of reading in primary school students. On a practical level, the findings have shown how the different proposals of the Reading Plan have influenced the dimensions taken into account in this work, marking a first step and leaving the door open for members of the scientific community to continue along the path begun in this study, with the aim of discussing and contrasting the results presented here, as well as incorporating other dimensions that may be relevant for assessing the impact of the Reading Plan.

\section{Funding}
This study is part of the project with reference OTRI-4995 entitled Services related to the pilot phase of the evaluation of educational programs, financed by the “La Caixa” Foundation and sponsored by the University of Granada and the EducaTech Research Group (SEJ-666).

\section{Acknowledgments}
Mare Nostrum School of Ceuta (Ceuta, Spain). Ministry of Education and Vocational Training of the Government of Spain.


\printbibliography\label{sec-bib}
% if the text is not in Portuguese, it might be necessary to use the code below instead to print the correct ABNT abbreviations [s.n.], [s.l.]
%\begin{portuguese}
%\printbibliography[title={Bibliography}]
%\end{portuguese}


%full list: conceptualization,datacuration,formalanalysis,funding,investigation,methodology,projadm,resources,software,supervision,validation,visualization,writing,review
\begin{contributors}[sec-contributors]
\authorcontribution{José-Antonio Marín-Marín}[conceptualization,formalanalysis,investigation,projadm,validation,visualization,writing,review]
\authorcontribution{Jesús López-Belmonte}[datacuration,formalanalysis,investigation,methodology,resources,supervision,visualization,writing,review]
\authorcontribution{Georgios Lampropoulos}[conceptualization,formalanalysis,investigation,validation,visualization,writing,review]
\authorcontribution{Antonio-José Moreno-Guerrero}[datacuration,formalanalysis,funding,investigation,resources,software,supervision,visualization,writing,review]
\end{contributors}


\end{document}

