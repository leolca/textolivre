% !TEX TS-program = XeLaTeX
% use the following command:
% all document files must be coded in UTF-8
\documentclass[spanish]{textolivre}
% build HTML with: make4ht -e build.lua -c textolivre.cfg -x -u article "fn-in,svg,pic-align"

\journalname{Texto Livre}
\thevolume{16}
%\thenumber{1} % old template
\theyear{2023}
\receiveddate{\DTMdisplaydate{2023}{2}{26}{-1}} % YYYY MM DD
\accepteddate{\DTMdisplaydate{2023}{3}{31}{-1}}
\publisheddate{\DTMdisplaydate{2023}{6}{6}{-1}}
\corrauthor{Verónica Mas García}
\articledoi{10.1590/1983-3652.2023.44851}
%\articleid{NNNN} % if the article ID is not the last 5 numbers of its DOI, provide it using \articleid{} commmand 
% list of available sesscions in the journal: articles, dossier, reports, essays, reviews, interviews, editorial
\articlesessionname{essays}
\runningauthor{Mas García et al.} 
%\editorname{Leonardo Araújo} % old template
\sectioneditorname{Hugo Heredia Ponce}
\layouteditorname{Thaís Coutinho}

\title{Formación y competencia digital del profesorado de Educación Secundaria en España}
\othertitle{Formação e competência digital de professores do Ensino Médio na Espanha}
\othertitle{Training and digital competence of secondary education teachers in Spain}
% if there is a third language title, add here:
%\othertitle{Artikelvorlage zur Einreichung beim Texto Livre Journal}

\author[1,2]{Verónica Mas García~\orcid{0000-0002-8615-9949}\thanks{Email: \href{mailto:vemasgar@alumni.uv.es}{vemasgar@alumni.uv.es}}}
\author[2]{Vicente Gabarda Méndez~\orcid{0000-0001-6159-5173}\thanks{Email: \href{mailto:vicente.Gabarda@uv.es}{vicente.Gabarda@uv.es}}}
\author[2]{José Peirats Chacón~\orcid{0000-0002-6580-2712}\thanks{Email: \href{mailto:jose.peirats@uv.es}{jose.peirats@uv.es}}}
\affil[1]{Universidad Internacional de Valencia (VIU), Facultad de Educación, Valencia, España.}
\affil[2]{Universitat de Valencia (UV), Facultad de Filosofía y Ciencias de la Educación, Valencia, España.}

\addbibresource{article.bib}
% use biber instead of bibtex
% $ biber article

% used to create dummy text for the template file
\definecolor{dark-gray}{gray}{0.35} % color used to display dummy texts
\usepackage{lipsum}
\SetLipsumParListSurrounders{\colorlet{oldcolor}{.}\color{dark-gray}}{\color{oldcolor}}

% used here only to provide the XeLaTeX and BibTeX logos
\usepackage{hologo}

% if you use multirows in a table, include the multirow package
\usepackage{multirow}

% provides sidewaysfigure environment
\usepackage{rotating}

% CUSTOM EPIGRAPH - BEGIN 
%%% https://tex.stackexchange.com/questions/193178/specific-epigraph-style
\usepackage{epigraph}
\renewcommand\textflush{flushright}
\makeatletter
\newlength\epitextskip
\pretocmd{\@epitext}{\em}{}{}
\apptocmd{\@epitext}{\em}{}{}
\patchcmd{\epigraph}{\@epitext{#1}\\}{\@epitext{#1}\\[\epitextskip]}{}{}
\makeatother
\setlength\epigraphrule{0pt}
\setlength\epitextskip{0.5ex}
\setlength\epigraphwidth{.7\textwidth}
% CUSTOM EPIGRAPH - END

% LANGUAGE - BEGIN
% ARABIC
% for languages that use special fonts, you must provide the typeface that will be used
% \setotherlanguage{arabic}
% \newfontfamily\arabicfont[Script=Arabic]{Amiri}
% \newfontfamily\arabicfontsf[Script=Arabic]{Amiri}
% \newfontfamily\arabicfonttt[Script=Arabic]{Amiri}
%
% in the article, to add arabic text use: \textlang{arabic}{ ... }
%
% RUSSIAN
% for russian text we also need to define fonts with support for Cyrillic script
% \usepackage{fontspec}
% \setotherlanguage{russian}
% \newfontfamily\cyrillicfont{Times New Roman}
% \newfontfamily\cyrillicfontsf{Times New Roman}[Script=Cyrillic]
% \newfontfamily\cyrillicfonttt{Times New Roman}[Script=Cyrillic]
%
% in the text use \begin{russian} ... \end{russian}
% LANGUAGE - END

% EMOJIS - BEGIN
% to use emoticons in your manuscript
% https://stackoverflow.com/questions/190145/how-to-insert-emoticons-in-latex/57076064
% using font Symbola, which has full support
% the font may be downloaded at:
% https://dn-works.com/ufas/
% add to preamble:
% \newfontfamily\Symbola{Symbola}
% in the text use:
% {\Symbola }
% EMOJIS - END

% LABEL REFERENCE TO DESCRIPTIVE LIST - BEGIN
% reference itens in a descriptive list using their labels instead of numbers
% insert the code below in the preambule:
%\makeatletter
%\let\orgdescriptionlabel\descriptionlabel
%\renewcommand*{\descriptionlabel}[1]{%
%  \let\orglabel\label
%  \let\label\@gobble
%  \phantomsection
%  \edef\@currentlabel{#1\unskip}%
%  \let\label\orglabel
%  \orgdescriptionlabel{#1}%
%}
%\makeatother
%
% in your document, use as illustraded here:
%\begin{description}
%  \item[first\label{itm1}] this is only an example;
%  % ...  add more items
%\end{description}
% LABEL REFERENCE TO DESCRIPTIVE LIST - END


% add line numbers for submission
%\usepackage{lineno}
%\linenumbers

\begin{document}
\maketitle

\begin{polyabstract}
\begin{abstract}
La competencia digital del profesorado es un elemento clave de la calidad de los sistemas educativos en la actualidad. Bajo este planteamiento, este texto revisa la integración de los contenidos tecnológicos en la formación inicial del profesorado como punto de partida de su capacitación tecnológica. Asimismo, se exploran los marcos de referencia de la competencia digital docente que identifican las destrezas con que debe contar el profesorado para desarrollar su labor y se cruzan ambas cuestiones con la situación actual que desvela la literatura científica sobre las habilidades digitales de los docentes en formación y en ejercicio. Los resultados ponen de manifiesto que hay una escasa implementación de la tecnología en los planes formativos que conducen hacia el desempeño de la docencia y que los planteamientos que imponen los marcos institucionales son demasiado amplios en comparación con las destrezas tecnológicas del profesorado, identificándose claramente áreas de mejora que requieren el establecimiento de la formación continua.

\keywords{Educación secundaria \sep Formación del profesorado \sep Tecnologías \sep Competencia digital}
\end{abstract}

\begin{portuguese}
\begin{abstract}
A competência digital dos professores é um elemento chave da qualidade dos sistemas educativos atuais. Sob essa abordagem, este texto revê a integração de conteúdos tecnológicos na formação inicial de professores como ponto de partida para a sua formação tecnológica. Explora também os quadros de referência da competência digital no ensino que identificam as competências que os professores precisam ter para realizar o seu trabalho e cruza ambas as questões com a situação atual revelada pela literatura científica sobre as competências digitais dos professores estagiários e praticantes. Os resultados mostram que existe uma implementação escassa de tecnologia nos planos de formação que conduzem ao desempenho pedagógico e que as abordagens impostas pelos quadros institucionais são demasiado amplas em comparação com as competências tecnológicas dos professores, identificando claramente as áreas de melhoria que requerem o estabelecimento de formação contínua.

\keywords{Ensino médio \sep Formação de professores \sep Tecnologias \sep Competência digital}
\end{abstract}
\end{portuguese}

\begin{english}
\begin{abstract}
The digital competence of teachers is a key element of the quality of education systems today. Under this approach, this text reviews the integration of technological content in initial teacher training as a starting point for their technological training. It also explores the frameworks of reference for digital competence in teaching that identify the skills that teachers need to have in order to carry out their work and crosses both issues with the current situation revealed by the scientific literature on the digital skills of both trainee and practising teachers. The results show that there is a scarce implementation of technology in the training plans leading to teaching performance and that the approaches imposed by institutional frameworks are too broad compared to the technological skills of teachers, clearly identifying areas for improvement that require the establishment of continuous training.

\keywords{Secondary education \sep Teacher training \sep Technologies \sep Digital competence}
\end{abstract}
\end{english}

% if there is another abstract, insert it here using the same scheme
\end{polyabstract}

\section{Introducción}

La tecnología, en sus múltiples formatos y usos, se ha convertido en un elemento indispensable en el entorno laboral, social, económico, artístico, científico, etc. Igualmente, se ha convertido en parte de nuestras vidas y ha transformado, entre otras cosas, el modo en que enseñamos y aprendemos.

Este panorama tiene implicaciones en el modo en que la tecnología se implementa en el ámbito educativo, siendo esencial tener en cuenta aspectos como el equipamiento necesitado, el enfoque curricular planificado o las competencias docentes deseadas para el desempeño del rol profesional en el seno de este nuevo escenario.

En esta línea, este texto trata de analizar el planteamiento institucional que se realiza desde la Administración educativa española acerca de la capacitación tecnológica del profesorado, focalizándolo en la Educación Secundaria. Con ese objetivo, en primer lugar, se explora qué papel asume la competencia digital del docente en el desarrollo de su formación inicial, analizando la normativa que regula los planes de formación universitarios. Posteriormente, se realiza un estudio acerca de la competencia digital del profesorado ya en ejercicio a través de la literatura, análisis que sirve, entre otras cuestiones, para valorar el ajuste entre la formación inicial y las competencias requeridas en el desempeño del rol docente.


\section{Integración de la competencia digital en el sistema educativo español}

Para poder entender cuál es el planteamiento actual de la formación inicial del profesorado de Educación Secundaria, hemos de remontarnos alrededor de dos décadas atrás. En 1999, diversos países de la Unión Europea, incluido España, se suscribieron a la Declaración de Bolonia. Se trataba de una estrategia conjunta que suponía la transformación del Espacio Europeo de Educación Superior (EEES), estructurando las titulaciones que la componían y ofreciendo un nuevo panorama de estudios de Grado (4 años) y Máster (1-2 años) que sustituían las antiguas diplomaturas, licenciaturas e ingenierías.

En esta misma estrategia, comenzaba a hacerse patente la necesidad de mejorar e integrar las tecnologías en las aulas con tal de dar respuesta a las demandas de una nueva sociedad incipientemente digitalizada, algo contradictorio teniendo en cuenta que la implementación del EEES supuso la desaparición de parte de la formación docente troncal en materia tecnológica \cite{herrada_valverde_adaptacion_2011}.

Fue, sin embargo, el desarrollo de las Competencias Clave para el Aprendizaje Permanente \cite{comision2006} revisadas por el Consejo de la Unión Europea años más tarde en 2018 la que permitió identificar la competencia digital docente como un elemento clave del desarrollo del ciudadano. Su reconocimiento, por tanto, tuvo un fuerte impacto a nivel económico y social, convirtiéndose el ámbito educativo en un espacio de enorme relevancia para poder impulsar la capacitación tecnológica de la sociedad. Aunque en el caso del sistema educativo español ya había habido algunas aproximaciones para la integración de la tecnología tanto a nivel estatal \cite{colas-bravo_impact_2018} como autonómico \cite{paredes2013politicas}, este documento de la Comisión Europea sirvió de pretexto para que las nuevas regulaciones normativas incorporaran de manera más explícita contenidos y competencias relacionadas con la competencia digital.

La \textcite{de2006ley} incorporó estas destrezas como una de las competencias básicas de la escolaridad, considerando que eran fundamentales para la calidad del sistema educativo. Concretamente, consideró las tecnologías como parte de los principios pedagógicos, las promocionó como medio didáctico y reconoció, de manera directa, la necesidad de que el profesorado mejorara el desarrollo de aptitudes, conocimientos y destrezas en su uso. Estos planteamientos se consolidaron a través de las dos leyes generales posteriores \cite{de2013ley,de2020ley}, consolidando la presencia de la tecnología en el ámbito formativo y reconociendo la capacitación digital del profesorado como una línea estratégica de la Administración educativa.

Abordar esta cuestión supone plantearse a nivel curricular qué tipo de interacción necesita el proceso educativo con la tecnología, ya que esta debe ser entendida como un mecanismo de cohesión que permita nutrir la parte pedagógica creando nuevos escenarios de aprendizaje que permitan dar respuesta a necesidades de la sociedad actual \cite{luengo_horcajo_fortalezas_2021}. En esta misma línea, la reflexión acerca de las nuevas modalidades de enseñanza y aprendizaje \cite{lopez_belmonte_alisis_2019}, la recurrente aparición de nuevas metodologías mediadas por la tecnología \cite{gamez_guillen_efectos_2020} o la necesidad de analizar el impacto de todos estos procesos sobre la inclusión o la equidad \cite{basham2020opportunity}. Resulta igualmente una necesidad indagar sobre la relación de los diferentes agentes con la tecnología y analizar cómo se desarrolla su competencia digital en las diferentes etapas \cite{guitert_digital_2021}.

Esta necesidad es especialmente destacable en el caso del profesorado, siendo preciso plantearse cuáles son las necesidades digitales y el nivel de competencia que son necesarios para desempeñar el rol docente \cite{arillorente_desarrollo_2018}, así como la incidencia de variables como el nivel de estudios, la formación previa o la edad, aspectos que han sido subrayados por otros estudios \cite{garzon_artacho_competencia_2021}.

Se puede afirmar, por tanto, que este fenómeno ha sido, en los últimos años, una fuente de preocupación de las diferentes administraciones, que han tratado de establecer marcos de competencia digital para los docentes, una cuestión que se analiza con mayor profundidad a continuación.

\section{Marcos de referencia de competencia digital}

Como ya hemos apuntado, las diversas organizaciones e instituciones nacionales e internacionales comenzaron a preocuparse por la formación del profesorado en el ámbito de las Tecnologías de la Información y la Comunicación (TIC), considerándola un aspecto esencial en su desarrollo profesional \cite{redecker_european_2017}. Es por ello que, desde hace más de una década, se han venido desarrollando diferentes marcos de competencias que han ido regulando y ofreciendo de manera unificada las Competencias Digitales (CD) necesarias para que los docentes desarrollen su carrera personal y profesional \cite{martin2022analysis} de un modo ajustado a las demandas de la sociedad actual.

Concretamente, ya en 2008 organismos internacionales como la Organización de Naciones Unidas para la Educación, la Ciencia y la Cultura (UNESCO) o la International Society for Technology in Education (ISTE) \cite{international_society_for_technology_in_education_national_2008} comenzaron a diseñar estrategias para la identificación de aquellas destrezas tecnológicas que eran necesarias para la función docente y que dieron lugar a lo que hoy en día conocemos como Marcos de Referencia de Competencia Digital Docente \cite{ferrando-rodriguez_crea_2023}. 

Estas primeras aproximaciones, no obstante, tuvieron un carácter parcial y diverso \cite{ramirez_martinell_marcos_2015}, bien porque se contextualizaron en un territorio geográfico en concreto (en el caso del marco de la ISTE, en el estadounidense) o porque desarrollaron una orientación específica (en el caso de la UNESCO, por su visión educativa y eminentemente económica).

De este modo, la \textcite{noauthor_unesco_2008} propuso un marco general denominado ICT competency standards for teachers: competency standards modules (ICT-CFT) para el desarrollo de estándares en competencias TIC de los docentes. La propuesta combinaba tres cuestiones: la alfabetización tecnológica, el desarrollo y profundización del conocimiento y su creación. Estos tres factores se relacionaban con seis componentes esenciales del sistema educativo que sirvieron para avanzar en una serie de reflexiones y propuestas en torno a ámbitos como la política, los planes de estudio y programas, la pedagogía, la organización y gestión de la tecnología o su uso para el desarrollo docente \cite{esteve-mon_competencia_2016}.

Años más tarde, ese mismo organismo actualizó la propuesta \cite{butcher_marco_2019} con la finalidad de integrar de manera eficiente las TIC en los centros educativos transformando así la pedagogía y el empoderamiento del estudiantado. De esta maner, su pretensión se dirigía hacia la definición de las políticas educativas y la formación de los docentes, identificándose seis áreas: comprensión del papel de las TIC en las políticas educativas, currículo y evaluación, pedagogía, aplicación de competencias digitales, organización y administración y aprendizaje profesional de los docentes. Esta propuesta, más amplia y ambiciosa, planteaba igualmente tres niveles de competencia: adquisición de conocimientos, profundización de conocimientos y creación de conocimientos, ofreciendo un marco no solo de dimensiones, sino también de certificación de grado de habilidades.

En el caso de la \emph{International Society for Technology in Education} (ISTE), publicaron en 2008 el documento de Estándares de Tecnologías de la Información y la Comunicación para docentes (NETS-T). Esta propuesta fue un primer intento para describir la Competencia Digital Docente (CDD) en base a la inclusión de matrices de evaluación para el profesorado, ofreciendo igualmente una aplicación adaptada a diferentes niveles de logro: principiante, medio, experto y transformador \cite{gabarda_mendez_profesorado_2021}. En cuanto a las dimensiones, se contemplaban cinco áreas de desarrollo competencial: Competencias que faciliten el aprendizaje y la creatividad de los estudiantes, desarrollar experiencias de aprendizaje y evaluaciones, modelar el trabajo y el aprendizaje que caracteriza la “Era digital”, promover la responsabilidad en la ciudadanía digital y comprometerse con el crecimiento profesional y liderazgo.

Esta primera aproximación fue retomada años más tarde por el mismo organismo dando lugar al ISTE, Standards for Educators: A Guide for Teachers and Other Professionals \cite{crompton_iste_2017}. En este caso, proponían profundizar en la práctica de la enseñanza fomentando la colaboración del alumnado repensando los enfoques tradicionales \cite{caberoet-almenara_marcos_2020}. En su desarrollo se definen siete perfiles que el profesorado debe desarrollar a lo largo de su carrera docente mediada por las tecnologías (aprendices, líderes, ciudadanos, colaboradores, diseñadores, facilitadores y analistas), ofreciendo una perspectiva progresiva de la capacitación docente en materia tecnológica vinculada a un rol activo e innovador \cite{gutierrez2017diseno}.

En el contexto comunitario europeo, la primera directriz para la identificación de habilidades tecnológicas es el Marco Europeo de Competencias Digitales para la Ciudadanía desarrollado por la propia Comisión Europea \cite{ferrari_digcomp:_2013} y que sirvió de base para el desarrollo posterior de propuestas centradas de manera explícita en la capacitación docente. Supuso, de este modo, una declaración de intenciones en lo relativo a la planificación de iniciativas en materia de CD a nivel europeo con los Estados miembros, además de convertirse en el anteproyecto que guio el Marco europeo para la competencia digital de los educadores \cite{redecker_european_2017}, reconocido como DigCompEdu.

Este marco, traducido al español en 2020, tiene por objetivo principal dar respuesta a la necesidad cada vez más imperante de establecer una propuesta común sobre las CDD concretas del profesorado de todos los niveles educativos. De este modo, se especifican elementos específicos desde la primera infancia hasta la educación superior, pasando por el alumnado con necesidades de apoyo e incluso para los contextos de aprendizaje no formal, ofreciendo así un enfoque complementario a los promovidos anteriormente. Su estructura considera seis áreas (compromiso profesional, contenidos digitales, evaluación y retroalimentación, enseñanza y aprendizaje y empoderamiento de los estudiantes, y competencia digital de los estudiantes) y un total de 22 competencias, estableciéndose niveles de desarrollo en una escala del A1 (novato) al C2 (pionero).

En el ámbito nacional, el Instituto Nacional de Tecnologías Educativas y Formación del Profesorado (INTEF) lleva también varios años (concretamente desde 2013) \cite{intef2013marco} perfilando un Marco de Competencia Digital Docente (MCDD). La versión completa, que data de 2017 \cite{intef2017}, incluye cinco áreas competenciales (Información y alfabetización informacional, Comunicación y colaboración, Creación de contenidos digitales, Seguridad y Resolución de problemas) para el desarrollo profesional digital y un total de 21 competencias. Todas ellas se estructuran en seis niveles competenciales progresivos de manejo: niveles A1 y A2 (nivel básico), B1 y B2 (nivel intermedio) y C1 y C2 (nivel avanzado). En 2022, una nueva actualización ha generado el Marco de Referencia de la Competencia Digital Docente \cite{intef2022marco}, más cercano a los postulados del DigCompedu (2017), pero con la peculiaridad de que se adapta al contexto educativo español, aparte de proponer fases de desarrollo profesional docente desde el inicio de su formación hasta su incorporación en el ámbito profesional \cite{mas2022competencia}. Además, se incorporan alusiones a referencias legales vinculadas a leyes de protección de datos y de protección a la infancia y la adolescencia, se crea una nueva competencia vinculada a la protección de datos personales, privacidad y seguridad y se presentan nuevos criterios de desarrollo profesional docente, quedando definidos indicadores de logro en cada uno de los niveles de las competencias para clarificar el grado de desarrollo que se pretende alcanzar.

Este panorama ofrece, por tanto, un conjunto de propuestas promovidas por organismos diversos para la identificación de las competencias digitales del profesorado que han tratado de ir adaptando su concreción ajustándolo a los cambios de la sociedad y el avance de las propias tecnologías \cite{gabarda_mendez_competencias_2022}. Ahora bien, aunque tengan una misma finalidad, pueden encontrarse matices diferentes en todas ellas, tanto por el enfoque como por el alcance. Por eso, y con objeto de definir una propuesta más global, \textcite{cabero-almenara_marco_2020} han conseguido compatibilizar el DigCompEdu (2017) y el Marco Común de Competencia Digital Docente (2017) dando lugar a una herramienta de autorreflexión para docentes denominada DigCompEdu Check-In, traducido y adaptado al territorio español. Entre sus finalidades se encuentra la de ayudar a los educadores a comprender los marcos establecidos y proporcionar una herramienta que les permita evaluar cuáles son sus fortalezas y debilidades en materia tecnológica, convirtiéndose en los últimos años en un referente para el diagnóstico y evaluación de la competencia digital.

Este análisis histórico nos ha servido para poner de relieve los esfuerzos institucionales por parte de los organismos nacionales e internacionales para la promoción de la competencia digital y la identificación de las destrezas básicas que se le asocian. Cabe preguntarse, ahora, qué impacto han tenido estas estrategias institucionales en formación de la profesión docente \cite{cuartero_duran_certificacion_2019}, un aspecto que se explora a continuación.

\section{La formación del profesorado de educación secundaria en España. Normativa y realidad}
A tenor de lo expuesto, es patente la preocupación de los países miembros por la formación del profesorado en materia tecnológica, poniendo de relieve la necesidad de generar propuestas y reformas legislativas consecuentes con las directrices internacionales \cite{gabarda_mendez_profesorado_2021} y con el proceso de digitalización de las propias instituciones educativas \cite{tondeur2016}.

La identificación de los marcos analizados anteriormente invita a reflexionar sobre la función docente, trazando estrategias que puedan velar porque el profesorado cuente con las destrezas necesarias para desarrollar su labor de un modo más eficaz. Es importante destacar en este punto que no solamente se concibe que la competencia digital es necesaria para el docente, en tanto que ciudadano, sino también como parte responsable de que el alumnado desarrolle las habilidades y los conocimientos básicos para su alfabetización digital \cite{prendes_competencia_2018}.

Partiendo de esta premisa, y centrando en primer lugar la atención en la formación inicial del profesorado, la identificación de la competencia digital como una competencia clave no solo sirvió, como hemos visto anteriormente, para poder integrarla en el currículum de las diferentes etapas educativas, sino también para definir líneas de trabajo para la formación del profesorado.

De este modo, y más allá de los cambios que ya introdujo el Espacio Europeo de Educación Superior, la Orden ECI/3858/2007, de 27 de diciembre, por la que se establecen los requisitos para la verificación de los títulos universitarios oficiales que habiliten para el ejercicio de las profesiones de Profesor de Educación Secundaria Obligatoria y Bachillerato, Formación Profesional y Enseñanzas de Idiomas planteó como objetivo “Buscar, obtener, procesar y comunicar información (oral, impresa, audiovisual, digital o multimedia), transformarla en conocimiento y aplicarla en los procesos de enseñanza y aprendizaje en las materias propias de la especialización cursada” (art. 3). Si nos fijamos en la especialidad de Orientación Educativa, cuenta con algunas especificaciones concretas en el módulo “Los procesos de la orientación educativa y el asesoramiento psicopedagógico”, donde se explicita que el futuro profesional de esta área debe conocer y utilizar herramientas digitales para la orientación y tutoría.

Este planteamiento pone de relieve varias cuestiones: la primera es que la falta de actualización de la normativa (recordemos que data de 2007) supone una carencia manifiesta a nivel cuantitativo de referencias a la tecnología y la digitalización de los procesos formativos. Además, la alusión general que se realiza se vincula únicamente al área de Alfabetización mediática y, de manera indirecta, con la creación de contenidos. Es reseñable, para terminar, considerar necesarias matizaciones para la especialidad de Orientación Educativa, aunque podrían ser aplicables a cualquier especialidad por tratarse de habilidades básicas.

Si bien la habilidad en el uso e implementación de las tecnologías en el proceso de enseñanza aprendizaje debería ser un requisito de calidad para los docentes \cite{castaneda_por_2018}, este tema dista mucho de la realidad actual que se ofrece en las aulas \cite{rodriguez2017formacion}. Existen lagunas en la formación que reciben los estudiantes por parte del profesorado y esto se debe en parte a la separación que ofrecen los estudios de formación inicial del máster del profesorado (datan de 2007) con la realidad actual y la asociación histórica de la CD a la presencia instrumentalizada de tecnologías sin establecer una base pedagógica y tranversal \cite{esteve-mon_competencia_2016}. 

Se puede afirmar, por tanto, que la perspectiva que ofrece la normativa que sirve de base para el diseño de los planes de estudio del Máster de Formación del Profesorado de Educación Secundaria está desconectada de la realidad que ofrece el contexto actual tanto a nivel de equipamiento como a nivel de uso instrumental y de implicaciones éticas sobre dicha utilización.

Dicho esto, no se debe olvidar que cada universidad tiene cierta autonomía para la configuración de los planes de estudio, existiendo la posibilidad de que la tecnología pueda integrarse en mayor o menor medida en función de los intereses de cada institución lo que muestra, por tanto, un panorama diverso respecto a este fenómeno \cite{peirats_chacon_competencia_2018}.

La incertidumbre que nos deja este punto de partida de la formación inicial ha tratado de ser solventado por la Administración educativa a través de la formación permanente, habiendo promovido, por ejemplo, la actualización del Marco de Referencia de la Competencia Digital Docente y vinculando la oferta formativa pública para el profesorado (vertebrada por el INTEF) en torno a ella. Al margen de haber trabajado en los últimos años en la definición de un marco, como hemos apuntado anteriormente, se le atribuyen funciones en la elaboración y difusión de materiales en soporte digital y audiovisual de todas las áreas de conocimiento, la promoción de programas de formación del profesorado específicos para el desarrollo de la competencia digital o el desarrollo y mantenimiento evolutivo de aplicaciones, plataformas y portales para el ámbito educativo, siempre bajo una perspectiva colaborativa y de intercambio de experiencias y recursos entre los docentes.

Esta estrategia a nivel nacional se complementa, igualmente, con los planes de formación permanente que las diferentes Comunidades Autónomas van definiendo y que, de manera generalizada, definen la competencia digital como una de las líneas estratégicas \cite{ruizlazaro2019analisis}.

Toda esta reflexión hace conveniente una revisión de cuál es la realidad de la competencia digital del profesorado de Educación Secundaria en ejercicio o en formación. Asimismo, es necesario identificar qué variables inciden de un modo directo en las destrezas tecnológicas de los docentes de esta etapa. 

En primer lugar, cabe destacar que la literatura científica avala que los futuros docentes de Educación Secundaria (y por tanto aquellos que están cursando la formación que les habilitará para el ejercicio de la docencia en esta etapa), perciben la utilidad de la tecnología para la función docente \cite{alvarez_uria_propuesta_2022}.

En cuanto al nivel de competencia digital, el profesorado en formación cuenta con unas destrezas intermedias \cite{rodriguezmartin2016estudio}, teniendo mayores habilidades en cuestiones de tipo instrumental y las de nivel más básico \cite{napal_fraile_development_2018}.

Asimismo, ciertos estudios avalan diferencias en la competencia en función del área. De este modo, mientras \textcite{marin_suelves_alisis_2022} apuntan que cuentan con un nivel de competencia mayor en la dimensión de Alfabetización mediática e informacional, siendo la de creación de contenidos la que muestra mayores carencias. Estos mismos resultados pueden contrastarse en el estudio de \textcite{torres2020competencia}, quienes identificaban estas dos mismas dimensiones donde eran más y menos hábiles los futuros docentes de Educación Secundaria.

Es reseñable que se detectan diferencias en el seno de una misma dimensión. Por ejemplo, tal y como apuntan \textcite{moreno_guerrero_competencia_2020}, aunque los futuros docentes tienen un nivel destacable en el área de Información y alfabetización informacional, muestran mayores habilidades para el almacenamiento y recuperación de la información, que en destrezas para su almacenamiento o recuperación o para la evaluación de la información.

Por otro lado, investigaciones como la de \textcite{nieto-isidro_competencia_2022} o de \textcite{gende2021variacion} apuntan a que hay una diferencia acusada entre la competencia autopercibida de los futuros docentes y sus destrezas reales, siendo más elevadas las primeras y poniendo así de manifiesto una visión poco realista de las propias habilidades.

Si analizamos, por otro lado, posibles variables influyentes en el nivel de competencia digital, la capacitación específica para el desarrollo de habilidades digitales juega un papel relevante. De este modo, estudios como el de \textcite{jimenez-hernandez_mejora_2021} o el de \textcite{moreno_martinez_experiencia_2016} muestran el impacto de la formación en el desarrollo de las destrezas digitales.

Otros textos ponen también de relieve la importancia de factores como la edad. De este modo, \textcite{moreno-guerrero_area_2020} y \textcite{perez-navio_university_2021} concluyen que los participantes mayores de 30 años presentan un mayor nivel competencial que los más jóvenes en el área de Información y alfabetización informacional. Sin embargo, no hay consenso en relación a este aspecto, dado que el estudio de \textcite{jimenez-hernandez_mejora_2021} no concluyó relación entre esta variable y la competencia digital.

Sobre la variable género, sí parece haber evidencia de una tendencia positiva hacia el género masculino, mostrando mayores habilidades tecnológicas los hombres que las mujeres \cite{ortega-sanchez_self-perception_2020}. Sin embargo, otros estudios ponen de relieve una actitud más positiva hacia el uso de tecnología en el ámbito educativo por parte de las mujeres \cite{gomez_carrasco_estrategias_2020}.

En el caso del profesorado en ejercicio la situación no dista demasiado respecto a la que ofrece el profesorado en formación. De este modo, \textcite{anes2018pizarra} evidenciaron la percepción positiva de los docentes en el uso de la tecnología y su impacto en la motivación y el compromiso, a pesar de que estudios como el de \textcite{fuentes_alisis_2019} demostraron que cuentan con un menor nivel de competencia digital que el profesorado de etapas inferiores.

Por otro lado, igual que sucedía con los futuros docentes, muestran más habilidades en la dimensión de información y alfabetización mediática \cite{lopez_belmonte_pedagogical_2020}, aunque se identifican destrezas en otras áreas como la colaboración mediante canales digitales y el uso de la tecnología de manera creativa \cite{moreno_guerrero_competencia_2021}. Parecen tener, sin embargo, dificultades para la evaluación digital \cite{fernandez_miravete_evaluacion_2021} y, según el estudio de \textcite{miguel-revilla_assessing_2020}, para integrar pedagógicamente la tecnología.

\textcite{aznar_diaz_alfabetizacion_2019} también evidenciaron diferencias en base a variables como la edad, un hallazgo que comparten estudios como el de \textcite{lopez_belmonte_pedagogical_2020} a favor de los más jóvenes. También parece haber consenso, en este caso, en la influencia del género en el desarrollo de la competencia digital, siendo mayor en el caso de los hombres que de las mujeres \cite{lopez_belmonte_alisis_2019, portillo_self-perception_2020}.

Este panorama, variado en propuestas y en resultados, podría tener mayor coherencia en caso de poder disponer de una conceptualización clara del propio concepto de competencia digital, así como de un único marco que, a nivel nacional permitiera contar con un instrumento común sobre el que evaluar la competencia digital del profesorado \cite{herrero2023reflexiones}.


\section{Conclusión}

El análisis realizado permite reflexionar sobre la formación tecnológica del profesorado de Educación Secundaria, los planteamientos institucionales en torno a la competencia digital docente y el nivel de habilidades con que el profesorado cuenta en la actualidad.

Respecto a la primera cuestión, se ha podido ver que en la normativa que da amparo al diseño de planes de estudio conducentes a la titulación que habilita para el ejercicio de la docencia en la etapa de Educación Secundaria \cite{ordeneci2007}, la presencia de la tecnología es más bien escasa \cite{pinto-santos_development_2022}. Esto supone que, en el plano de la formación inicial, queda en manos de cada institución de Educación Superior la integración de contenidos y competencias que favorezcan el desarrollo de habilidades tecnológicas en los futuros docentes. Esta falta de concreción condena, por tanto, a la competencia digital a un segundo plano y a la voluntad de vincularla a las estrategias y políticas en cada una de las instituciones \cite{sanchez-caballe_integrating_2021}.

Se ha podido constatar, por lo demás, que el nivel de competencia digital tanto de los docentes en formación como de los que se encuentran ya en ejercicio es intermedio y que sus habilidades se vinculan a un uso básico e instrumental de la tecnología \cite{garzon_artacho_competencia_2021, marin_suelves_alisis_2022}. De este modo, de manera generalizada, cuentan con mayores destrezas en el área de información y alfabetización, así como en la de colaboración y comunicación, siendo mayores sus dificultades en las dimensiones de creación de contenidos, seguridad y resolución de problemas \cite{mas2022competencia}.

En base a lo expuesto, se urge reforzar tanto los planes de formación inicial como de formación permanente del profesorado \cite{rodriguez-muniz_secondary_2021} a fin de que puedan responder, no solamente a la necesidad de que puedan mejorar sus tareas docentes, sino también acompañar el desarrollo de la competencia digital del alumnado.





\printbibliography\label{sec-bib}
% if the text is not in Portuguese, it might be necessary to use the code below instead to print the correct ABNT abbreviations [s.n.], [s.l.]
%\begin{portuguese}
%\printbibliography[title={Bibliography}]
%\end{portuguese}


%full list: conceptualization,datacuration,formalanalysis,funding,investigation,methodology,projadm,resources,software,supervision,validation,visualization,writing,review
\begin{contributors}[sec-contributors]
\authorcontribution{Verónica Mas García}[conceptualization,writing,review]
\authorcontribution{Vicente Gabarda Méndez}[conceptualization,writing,review,supervision]
\authorcontribution{José Peirats Chacón}[conceptualization,writing,review,supervision]
\end{contributors}



\end{document}

