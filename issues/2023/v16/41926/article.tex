% !TEX TS-program = XeLaTeX
% use the following command:
% all document files must be coded in UTF-8
\documentclass[spanish]{textolivre}
% build HTML with: make4ht -e build.lua -c textolivre.cfg -x -u article "fn-in,svg,pic-align"

\journalname{Texto Livre}
\thevolume{16}
%\thenumber{1} % old template
\theyear{2023}
\receiveddate{\DTMdisplaydate{2022}{11}{26}{-1}} % YYYY MM DD
\accepteddate{\DTMdisplaydate{2022}{12}{25}{-1}}
\publisheddate{\DTMdisplaydate{2023}{2}{13}{-1}}
\corrauthor{Rosa Tabernero Sala}
\articledoi{10.1590/1983-3652.2023.41926}
%\articleid{NNNN} % if the article ID is not the last 5 numbers of its DOI, provide it using \articleid{} commmand 
% list of available sesscions in the journal: articles, dossier, reports, essays, reviews, interviews, editorial
\articlesessionname{dossier}
\runningauthor{Tabernero Sala y Laliena} 
%\editorname{Leonardo Araújo} % old template
\sectioneditorname{Hugo Heredia Ponce}
\layouteditorname{Daniervelin Pereira}

\title{El libro ilustrado de no ficción en la formación de lectores: análisis de las claves discursivas y culturales para leer en la sociedad digital}
\othertitle{O livro ilustrado de não-ficção na formação dos leitores: análise das chaves discursivas e culturais da leitura na sociedade digital}
\othertitle{The non-fiction picturebook in reader training: analysis of the discursive and cultural keys to reading in the digital society}
% if there is a third language title, add here:
%\othertitle{Artikelvorlage zur Einreichung beim Texto Livre Journal}

\author[1]{Rosa Tabernero Sala~\orcid{0000-0002-2332-5807}\thanks{Email: \href{mailto:rostab@unizar.es}{rostab@unizar.es}}}
\author[1]{Daniel Laliena~\orcid{0000-0001-9476-8472}\thanks{Email: \href{mailto:dlaliena@unizar.es}{dlaliena@unizar.es}}}
\affil[1]{Universidad de Zaragoza, Facultad de Ciencias Humanas y de la Educación, Departamento Didácticas Específicas (Área de Didáctica de la Lengua y la Literatura), Huesca, España.}

\addbibresource{article.bib}
% use biber instead of bibtex
% $ biber article

% used to create dummy text for the template file
\definecolor{dark-gray}{gray}{0.35} % color used to display dummy texts
\usepackage{lipsum}
\SetLipsumParListSurrounders{\colorlet{oldcolor}{.}\color{dark-gray}}{\color{oldcolor}}

% used here only to provide the XeLaTeX and BibTeX logos
\usepackage{hologo}

% if you use multirows in a table, include the multirow package
\usepackage{multirow}

% provides sidewaysfigure environment
\usepackage{rotating}

% CUSTOM EPIGRAPH - BEGIN 
%%% https://tex.stackexchange.com/questions/193178/specific-epigraph-style
\usepackage{epigraph}
\renewcommand\textflush{flushright}
\makeatletter
\newlength\epitextskip
\pretocmd{\@epitext}{\em}{}{}
\apptocmd{\@epitext}{\em}{}{}
\patchcmd{\epigraph}{\@epitext{#1}\\}{\@epitext{#1}\\[\epitextskip]}{}{}
\makeatother
\setlength\epigraphrule{0pt}
\setlength\epitextskip{0.5ex}
\setlength\epigraphwidth{.7\textwidth}
% CUSTOM EPIGRAPH - END

% LANGUAGE - BEGIN
% ARABIC
% for languages that use special fonts, you must provide the typeface that will be used
% \setotherlanguage{arabic}
% \newfontfamily\arabicfont[Script=Arabic]{Amiri}
% \newfontfamily\arabicfontsf[Script=Arabic]{Amiri}
% \newfontfamily\arabicfonttt[Script=Arabic]{Amiri}
%
% in the article, to add arabic text use: \textlang{arabic}{ ... }
%
% RUSSIAN
% for russian text we also need to define fonts with support for Cyrillic script
% \usepackage{fontspec}
% \setotherlanguage{russian}
% \newfontfamily\cyrillicfont{Times New Roman}
% \newfontfamily\cyrillicfontsf{Times New Roman}[Script=Cyrillic]
% \newfontfamily\cyrillicfonttt{Times New Roman}[Script=Cyrillic]
%
% in the text use \begin{russian} ... \end{russian}
% LANGUAGE - END

% EMOJIS - BEGIN
% to use emoticons in your manuscript
% https://stackoverflow.com/questions/190145/how-to-insert-emoticons-in-latex/57076064
% using font Symbola, which has full support
% the font may be downloaded at:
% https://dn-works.com/ufas/
% add to preamble:
% \newfontfamily\Symbola{Symbola}
% in the text use:
% {\Symbola }
% EMOJIS - END

% LABEL REFERENCE TO DESCRIPTIVE LIST - BEGIN
% reference itens in a descriptive list using their labels instead of numbers
% insert the code below in the preambule:
%\makeatletter
%\let\orgdescriptionlabel\descriptionlabel
%\renewcommand*{\descriptionlabel}[1]{%
%  \let\orglabel\label
%  \let\label\@gobble
%  \phantomsection
%  \edef\@currentlabel{#1\unskip}%
%  \let\label\orglabel
%  \orgdescriptionlabel{#1}%
%}
%\makeatother
%
% in your document, use as illustraded here:
%\begin{description}
%  \item[first\label{itm1}] this is only an example;
%  % ...  add more items
%\end{description}
% LABEL REFERENCE TO DESCRIPTIVE LIST - END


% add line numbers for submission
%\usepackage{lineno}
%\linenumbers

\begin{document}
\maketitle

\begin{polyabstract}
\begin{abstract}
En el marco teórico que definen las investigaciones de \textcite{wolf_lector_2020,delgado_dont_2018,gil_lectura_2020}, entre otros, se impone, en lo que a formación de lectores se refiere, la necesidad de combinar paradigmas digitales y analógicos con el fin de desarrollar el "cerebro bialfabetizado" \cite{wolf_lector_2020} que la sociedad digital reclama. Se propone, en este artículo, el análisis de un corpus de treinta libros y álbumes ilustrados de no ficción publicados en los últimos quince años con el objetivo de identificar las principales claves de lectura de estas obras. Así la superación de las fronteras entre ficción y no ficción, la información sesgada, la materialidad como eje de construcción del discurso, la lectura fragmentada y no lineal, la importancia de los peritextos, la hibridación de paradigmas y la dimensión cultural son algunos de los rasgos que se desprenden del análisis efectuado. Estos resultados proponen un lector modelo que se explica desde los parámetros de una nueva práctica letrada.

\keywords{Libro ilustrado de no ficción \sep Lector crítico \sep Sociedad de la información \sep Dimensión cultural}
\end{abstract}

\begin{portuguese}
\begin{abstract}
No marco teórico definido pelas pesquisas de \textcite{wolf_lector_2020,delgado_dont_2018,gil_lectura_2020}, entre outros, a necessidade de combinar paradigmas digitais e analógicos para desenvolver o "biliterate brain" \cite{wolf_lector_2020} que a sociedade digital exige é essencial na formação dos leitores. Este artigo propõe a análise de um corpus de trinta livros e álbuns de não-ficção publicados nos últimos quinze anos com o objetivo de identificar as principais chaves para a leitura destas obras. Assim, superar as fronteiras entre ficção e não ficção, informação tendenciosa, materialidade como eixo de construção do discurso, leitura fragmentada e não linear, importância dos peritexos, hibridização de paradigmas e a dimensão cultural são algumas das características que emergem da análise realizada. Estes resultados propõem um modelo de leitor que pode ser explicado a partir dos parâmetros de uma nova prática literária.

\keywords{Livro de imagens de não-ficção \sep Leitor crítico \sep Sociedade da informação \sep Dimensão cultural}
\end{abstract}
\end{portuguese}

\begin{english}
\begin{abstract}
In the theoretical framework defined by the research of \textcite{wolf_lector_2020,delgado_dont_2018,gil_lectura_2020}, among others, the need to combine digital and analogue paradigms in order to develop the "biliterate brain" \cite{wolf_lector_2020} that the digital society demands is essential in the training of readers. This article proposes the analysis of a corpus of thirty non-fiction picturebooks published in the last fifteen years with the aim of identifying the main keys to reading these works. Thus, overcoming the boundaries between fiction and non-fiction, biased information, materiality as the axis of discourse construction, fragmented and non-linear reading, the importance of peritexts, the hybridisation of paradigms and the cultural dimension are some of the features that emerge from the analysis carried out. These results propose a model reader that can be understood from the parameters of a new literary practice.

\keywords{Non-fiction picturebook \sep Critical reader \sep Information society \sep Cultural dimension}
\end{abstract}
\end{english}
% if there is another abstract, insert it here using the same scheme
\end{polyabstract}

\section{La lectura y el libro ilustrado de no ficción en la sociedad digital}\label{sec-intro}
La proliferación del libro de no ficción en el panorama editorial infantil contemporáneo invita a considerar este fenómeno en el marco de la sociedad digital en la que se inserta, una sociedad donde la hibridación de paradigmas digitales y analógicos propone derroteros específicos en lo que se refiere a la formación de lectores. Esta nueva configuración del ecosistema de lectura ha sido abordada por diversos informes y estudios \cite{baron_how_2021,cordon-garcia_combates_2018,cordon-garcia_lectura_2021,delgado_dont_2018,gil_lectura_2020,kovac_lectura_2020,ocde_21st-_2021}, que han confirmado que la lectura en papel se asocia a una mejor comprensión lectora y que insisten en la necesidad de favorecer procesos de lectura en profundidad que faciliten el pensamiento crítico \cite{wolf_lector_2020}.

Así, el modo en el que los formatos condicionan nuestra manera de leer y pensar \cite{littau_teorias_2008}, junto al actual requerimiento de lectores capaces de seleccionar, filtrar y organizar la información para convertirla en conocimiento \cite{european_commission__2018,millan_lectura_2017}, plantean un marco de acción en el que el libro ilustrado de no ficción se erige como un elemento del ecosistema de lectura sobre el que conviene reflexionar.

Pese a la novedad que supone la llamativa vitalidad del género en el mercado actual, conviene recordar que fue Ioannes Amos Comenius quien en 1658 publicó el primer libro ilustrado dirigido a la infancia como una enciclopedia para el aprendizaje del latín. En el prólogo de este \textit{Orbis Pictus} se reconocen principios pedagógicos sobre la lectura para las primeras edades que, por su énfasis en el aprendizaje como algo experiencial y en la relevancia de la observación como clave para el acceso al deleite y al conocimiento, resultan perfectamente válidos en la actualidad. Desde entonces, el libro infantil ha evolucionado y ha incorporado aspectos como la pedagogía activa del S. XX y su defensa de la conjugación del conocimiento, el arte y el juego. Sin embargo, el libro de no ficción contemporáneo ha debido integrar también las particularidades de la sociedad digital del S. XXI. Esto es algo que ha conseguido proponiendo al lector claves de interpretación consecuentes con uno de los principales retos de la formación de lectores en la actualidad: el desarrollo de la competencia informacional.

De este modo, el entorno digital se presenta como un mar de información en el que el lector ha de aprender a navegar y jerarquizar con el fin de generar conocimiento. En este contexto, el libro de no ficción se ajusta perfectamente a los nuevos modos de lectura impuestos por la sociedad por su manera de fomentar la curiosidad del lector infantil \cite{patte_jenlos_2008} como vía de aproximación al descubrimiento y al aprendizaje. El cambio de paradigma en el acceso a la información modifica, por tanto, la propuesta de los tradicionales libros de conocimiento desde la renovación del diseño y la imagen \cite{baro-llambias_libros_1996,baro-llambias_libros_2022,garralon_leer_2013} en una sociedad digital que, paradójicamente, refuerza la concepción del libro como objeto y la materialidad como clave para la construcción de sentidos \cite{cordon-garcia_combates_2018}.

Pese a que ha existido un debate acerca de la definición y la denominación de la noción de no ficción \cite{von-merveldt_informational_2018}, la pluralidad de opciones existentes (libro informativo, de conocimiento, documental, divulgativo, etc.) parece haber dejado paso a la etiqueta “libro de no ficción” como término preferente para designar un grupo de obras heterogéneas que obedecen a la transformación de la información en conocimiento \cite{duke_3.6_2000,duke_reading_2003}. Este concepto ha sido definido tradicionalmente desde las nociones de realidad, concienciación, curiosidad o lectura crítica \cite{carter_libros_2001,gonzalez-yunis_libros_2011,garralon_leer_2013} aunque el actual cambio de paradigma en el acceso a la información requiere trazar nuevas líneas en la definición del libro de no ficción contemporáneo. En este sentido, el volumen coordinado por \textcite{grilli_non-fiction_2020a} plantea una aproximación desde enfoques variados que, en su conjunto, sugiere la superación de la acepción enciclopédica para dar paso a otro modo de comprender el conocimiento en la formación de lectores. Así, la definición del libro de no ficción contemporáneo ha de atender a las palabras de \textcite{soriano_literatura_1995} sobre la necesidad de considerar las obras en relación con las condiciones que establecen su contexto de producción.

A partir de la relevancia que ha adquirido el libro de no ficción en los entornos de enseñanza \cite{malloy_assessing_2017,vaughn_examining_2022}, la investigación sobre el mismo y la búsqueda de una comprensión concreta del concepto resultan especialmente pertinentes. Esta pertinencia es aún más evidente a la luz del desfase entre las disposiciones legislativas y las buenas prácticas que se viene señalando desde hace tiempo \cite{duke_3.6_2000}, un desfase que indica cómo la incorporación del libro de no ficción en los programas curriculares puede no traducirse en una mayor presencia del mismo en la actividad de las aulas.

\section{Objetivo y método del estudio}\label{sec-normas}
Partiendo del marco teórico expuesto, el presente estudio se planteó las siguientes preguntas de investigación: ¿cuáles son las claves de lectura que propone el libro ilustrado de no ficción contemporáneo? ¿Cómo se relacionan dichas claves con la sociedad en la que este libro está inserto? A partir de dichas preguntas, se propuso el siguiente objetivo para el estudio:

\begin{itemize}
    \item Identificar las principales claves de lectura del libro ilustrado de no ficción contemporáneo en el marco de la sociedad digital.
\end{itemize}

Dentro de una previsión inicial de 150 libros ilustrados de no ficción, se escogió como corpus para el análisis una muestra final de 30 libros (ver \Cref{tab01}). Esta selección estuvo condicionada por un criterio cronológico, puesto que se requirió que las obras hubiesen sido publicadas en los últimos 15 años; un criterio editorial, ya que se privilegió la elección de obras publicadas por editoriales que estuvieran concediendo un espacio al concepto que ha sido conocido habitualmente como libro de conocimiento o libro informativo; y, finalmente, por un criterio de construcción del discurso optando por la selección de libros y álbumes ilustrados.

En lo que respecta a este último criterio, el libro ilustrado de no ficción contemporáneo supone en sí mismo un discurso multimodal \cite{kress_multimodality:_2010,canamares-torrijos_estudio_2019,moya_guijarro_libros_2020,unsworth_multiliteracies_2008} que combina palabra e imagen. Por tanto, la ilustración en esta clase de obras ha superado el papel de simple ornamento que pudiera haber desempeñado tradicionalmente, en línea con lo que definía \textcite{bader_american_1976}, como una parte esencial del contenido. Las relaciones entre texto e imagen se definen desde lo que \textcite{nikolajeva_images_2001} identifican como ampliación y complementariedad, llegando en la actualidad a casos habituales en los que el libro se articula desde la imagen y el formato, mientras el texto desempeña exclusivamente una función de anclaje \cite{swartz_choosing_2020}. Sirva como ejemplo un título como \textit{Las mil y una formas de la naturaleza} (Véase \Cref{tab01}).

Esta interdependencia entre texto e ilustración invita a reflexionar sobre la esencia del libro ilustrado de no ficción como género muy próximo al álbum ilustrado desde los parámetros de \textcite[p. 1]{bader_american_1976}:

\begin{quote}
    A picturebook is text, illustrations, total design; an item of manufacture and a commercial product; a social, cultural, historical document; and, foremost, an experience for a child.  As an art form it hinges on the interdependence of pictures and words, on the simultaneous display of two facing pages, and on the drama of the turning page. On its own terms its possibilities are limitless.
\end{quote}

Sin embargo, conviene reparar en dos elementos diferenciadores de relevancia que surgen de la especificidad del libro ilustrado de no ficción. La relación de dependencia mutua entre texto e imagen se da únicamente desde la ampliación y la complementariedad en un proceso en el que la imagen adquiere una dimensión informativa más acusada que la que pudiera haber tenido en los álbumes ilustrados ficcionales. Asimismo, el ritmo de lectura y el “drama de la vuelta de página” mencionados por Bader se entienden de distinto modo ya que la observación del lector, como estrategia fundamental en su interacción con el libro, imprime un ritmo de lectura diferente al del álbum ficcional. La imagen, por otra parte, favorece en el lector el desarrollo de respuestas emocionales \cite{nikolajeva_emotions_2018} que resultan fundamentales para producir el efecto de asombro mencionado por Rachel Carson en su ensayo \textit{El sentido del asombro} \cite{carson_sentido_2021}. El libro de no ficción se aproxima, en este sentido, al mensaje artístico cuando plantea interpretaciones alejadas de los contenidos objetivos que requieren la participación personal y emocional del lector, lo que sugiere asimismo una defensa de lo propuesto por Umberto Eco en \textit{Obra abierta} \cite{eco_obra_1993} y del lector crítico dibujado por \textcite{sanders_literature_2017}. Así una obra como \textit{Semillas}, de J. R. Alonso e ilustrada por Marco Paschetta, combina información científica sobre las semillas con dobles páginas en las que estas adquieren voz propia y narran sus aventuras. En la misma línea, un título como Estaciones de Blexbolex plantea la asociación de cada una de las estaciones con sensaciones y acciones.

\begin{table}[h!]
\centering \small
\begin{threeparttable}
\caption{Corpus del análisis.}
\label{tab01}
\begin{tabular}{p{0.9\textwidth}}
\toprule
\textbf{Obras seleccionadas para el análisis} \\
 \midrule
ALONSO, José Ramón; PASCHETTA, Marco. \textit{Semillas. Un pequeño gran viaje.} Barcelona: A Buen Paso, 2018. ISBN: 9788417555085. \\
ALZIAL, Sylvain; RAJCAK, Hélène. \textit{Panthera tigris}. Buenos Aires: Iamiqué, 2020. ISBN: 9789874444295. \\
BÁRCENA, Halil; CABASSA, Mariona. \textit{Historias de Nasrudín}. Barcelona: Akiara, 2021. ISBN: 9788417440930. \\
BLEXBOLEX. \textit{Estaciones}. Madrid: Kókinos, 2010. ISBN: 9788492750054. \\
CASTEL-BRANCO, Inês; ELLA, María. \textit{Tu canción}. Barcelona: Akiara, 2016. ISBN: 9788415518310. \\
COHEN, Irène; PALMARUCCI, Claudia. \textit{Marie Curie en el país de la ciencia}. Barcelona: Ekaré, 2020. ISBN: 9788412163674. \\
DA COLL, Ivan. \textit{Supongamos}. Bogotá: Babel, 2015. ISBN: 9789588841953. \\
DUPRAT, Guillaume. \textit{Zoóptica}. Madrid: SM, 2014. ISBN: 9788467562538. \\
DUTHIE, Ellen; MARTAGÓN, Daniela. \textit{Mundo Cruel}. Madrid: Wonder Ponder, 2014. ISBN: 9788494316708. \\
EQUIPO PLANTEL; Pina, Marta. \textit{Cómo puede ser la democracia}. Valencia: Media Vaca, 2015. ISBN: 9788494362507. \\
GERVAIS, Bernadette. \textit{La mariquita}. Barcelona: Juventud, 2017. ISBN: 9788426143860. \\
GOES, Peter. \textit{La línea del tiempo. Un viaje ilustrado por la historia}. Madrid: Maeva, 2016. ISBN: 9788419110084. \\
GÓMEZ, Nuria; SOLÍS, Santiago. \textit{El berrinche de Moctezuma}. Barcelona: Ekaré, 2022. ISBN: 9788412416688. \\
GRAVETT, Emily. \textit{El gran libro de los miedos del Ratoncito}. Barcelona: Picarona, 2015. ISBN: 9788416117444. \\
GRAVETT, Emily.\textit{ Meerkat Mail}. Nueva York: Simon \& Schuster, 2007. ISBN: 9781416934738. \\
GRECCO, Hernán; PICYK, Pablo. \textit{Física hasta en la sopa}. Buenos Aires: Iamiqué, 2018. ISBN: 9789874444172. \\
GRUNDMANN, Emmanuelle; GUIRAUD, Florence. \textit{Las mil y una formas de la naturaleza}. Barcelona: Libros del zorro rojo, 2018. ISBN: 9788494884856. \\
JENKINS, Steve. \textit{Nunca sonrías a un mono}. Barcelona: Juventud, 2015. ISBN: 9788426142177. \\
LUJÁN, Jorg; SADAT, Mandana. \textit{Volcancito nevado}. Madrid: Kókinos, 2022. ISBN: 9788417742706. \\
MENA, Pato. \textit{Onsen ¿Qué hacen los monos?} Barcelona: A buen paso, 2021. ISBN: 9788417555542. \\
MILLÁ, Marta; LUCIANI, Rebeca. Jatakas. \textit{Seis cuentos budistas}. Barcelona: Akiara, 2017. ISBN: 978-8415518662. \\
ODRIOZOLA, Elena. \textit{Ya sé cultivar el huerto}. Donostia: Ediciones Modernas El Embudo, 2022. ISBN: 9788412247565. \\
NOGUÉS, Alex; CASTAÑO, Samuel.\textit{ Mil tomates y una rana}. Historia de un huerto mínimo. Barcelona: A buen paso, 2020. ISBN: 9788417555351. \\
PÊG, Ana; CARVALHO, Bernardo; MINHÓS, Isael.\textit{ Plasticus maritimus}. Un especie invasora. Pontevedra: Kalandraka, 2021. ISBN: 9788413430171. \\
PEIXE, Maria; TEIXEIRA, Inés; CARVALHO, Bernardo. \textit{Ahí fuera. Guía para descubrir la naturaleza}. Barcelona: Planeta, 2016. ISBN: 9788408152279. \\
RAYOS, Sonia; SILVANA, Andrés; BERRIO, Juan. \textit{En construcción}. Albuixech: Litera, 2018. ISBN: 9788494843914. \\
SASEK, Miroslav. \textit{Esto es París}. Madrid: Nórdica Infantil, 2018. ISBN: 9788417281960. \\
SÍS, Peter. \textit{El árbol de la vida}. Charles Darwin. Barcelona: Ekaré, 2021. ISBN: 9788412372878. \\
VAUGEDALE, Anaïs. \textit{Cómo construir un hermano mayor. Un libro de anatomía y bricolaje}. Barcelona: Juventud, 2018. ISBN: 9788426145215. \\
WILLIAMS, Rachel; CARNOVSKY. \textit{Iluminaturaleza}. Madrid: SM, 2016. ISBN: 9788467589955. \\
\bottomrule
\end{tabular}
\source{Elaboración propia.}
\end{threeparttable}
\end{table}

Este corpus de 30 títulos fue examinado mediante un análisis de contenido fundamentado en los parámetros de la Estética de la Recepción y de la Pragmática literaria \cite{eco_role_1979,iser_acto_2022,jauss_pour_1987}. Dicho análisis atendió asimismo a los estudios desarrollados en esta línea por los autores del presente artículo \cite{tabernero__2022}, así como a las claves de análisis del libro de no ficción propuestas por \textcite{sanders_literature_2017}, entre las que destacan la presencia del sesgo del autor y el modo en el que la construcción del discurso invita al lector a participar críticamente en la generación de sentidos.

\section{Claves de lectura del libro ilustrado de no ficción}\label{sec-conduta}
\subsection{La hibridación como eje de construcción}
En el marco de la actual modernidad líquida \cite{bauman_modernidad_2017}, los límites entre el arte, el aprendizaje, la ficción o la información no se encuentran claramente diferenciados. El libro de no ficción contemporáneo presenta, por tanto, estrategias de ficción y de no ficción y reclama un lector que se mueva con facilidad entre los procesos de lectura eferente y estética \cite{rosenblatt_literatura_2002}.

En este sentido, la investigación de \textcite{alexander_pleasures_2018} sobre los prejuicios existentes acerca de la lectura de no ficción revela que esta última tiende a asociarse al ámbito escolar y a lo no placentero. Sin embargo, los libros analizados en el presente estudio evidencian una superación de estas concepciones y muestran, tal como lo sugirió \textcite{bader_american_1976}, que los libros de no ficción proponen otro modo de leer tanto en lo referente a la definición del género como a la del lector modelo. Así, no solo se vinculan las nociones de verdad, realidad y pensamiento crítico a la no ficción \cite{yenika-agbaw_why_2018}, sino que podemos reconocer cómo en estos libros el pensamiento lógico-matemático y el narrativo \cite{bruner_realidad_2010} se entremezclan y se confunden. Valga como ejemplo de esta convivencia de lo ficcional y lo no ficcional la obra \textit{Cómo construir un hermano mayor}, de Anaïs Vaugelade, en la que la historia ficcional propuesta ejerce una clara función de hilo argumental que permite introducir las informaciones sobre el cuerpo humano de manera oportuna y ordenada.

El libro de no ficción se inserta, de este modo, en lo que se ha denominado carácter híbrido, no ficción narrativa o textos de doble propósito \cite{graff_revisiting_2020,grilli_non-fiction_2020a,pappas_information_2006,romero_oliva_validacion_2021,smith_navigating_2019,von-merveldt_informational_2018}. Estos libros se explican desde la hibridación de lenguajes, contenidos y lectores y, consecuentemente, se enmarcan en paradigmas mixtos que integran la perspectiva cognitiva, emocional y experiencial en la aprehensión del conocimiento. De nuevo, se rompen las fronteras que vinculan arte y conocimiento y se plantea, desde la curiosidad y el asombro, una imagen de lector crítico y creativo.

Como ejemplo prototípico de la hibridación en el libro de no ficción, se puede citar \textit{Mil tomates y una rana. Historia de un huerto mínimo}, de Alex Nogués e ilustrado por Samuel Castaño. Se trata de un diario que repasa semana a semana la creación de un huerto y que termina de la siguiente manera:

\begin{quote}
    A veces me pregunto: ¿qué hicimos para que el huerto fuese tan generoso? Y no doy con mejor respuesta: tan solo lo fundamos. Le dimos un lugar. Un buen lugar y respeto. La naturaleza está ahí, en cada metro cuadrado, y es, sencillamente, asombrosa.
\end{quote}

Ese tono poético, que confiere su sentido artístico a una narración precisa desde el punto de vista conceptual, encuentra su reflejo en las sutiles ilustraciones de Samuel Castaño. Otro ejemplo representativo de lo que se acaba de exponer es \textit{Panthera Tigris}, de Sylvain Alzial y Hélène Rajcak. En esta obra, el relato ficcional que narra la incursión en la selva por parte de un presuntuoso sabio naturalista y su conversación con un joven guía explorador se combina de un modo totalmente natural con la aportación de informaciones no ficcionales sobre el tigre de bengala. Más en la línea de los libros abecedario, \textit{Volcancito nevado} ilustra a la perfección cómo la hibridación en el libro de no ficción puede darse a través de la configuración de un discurso no ficcional desde lo poético. Finalmente, es posible que el ejemplo más representativo del nuevo libro de no ficción sea \textit{Marie Curie en el país de la ciencia}, de Irène Cohen e ilustrado por Claudia Palmarucci \cite{tabernero__2022}.

Esta biografía de la conocida científica entrelaza las historias de la ciencia y del arte al combinar una narración de estilo íntimo y personal, por un lado, y un epígrafe final en el que se exponen las fuentes documentales que han inspirado el trabajo de la ilustradora, por otro lado.

Como apunta \textcite{von-merveldt_informational_2018}, la ficción ha recibido un trato más favorable que la no ficción por parte de la crítica en lo que se refiere a la construcción del lector. El sesgo negativo imprimido en la no ficción ha llevado a numerosos estudiosos a denostar la finalidad instructiva en la comunicación artística y ha desterrado el instruir deleitando horaciano mientras, paradójicamente, se defendía la necesidad del deleite en la formación del lector. Este es el contexto en el que los libros de no ficción de los últimos años han venido a recuperar la finalidad que se confería al lenguaje poético --\textit{prodesse et delectare}-- y a proponer una combinación de arte y conocimiento en el diseño de un lector crítico que busca posicionarse en el mundo desde la curiosidad y el asombro que el texto le provoca.

Por tanto, los libros ilustrados de no ficción recobran el espíritu de Amos Comenius y actualizan las necesidades de un lector en formación que, para navegar con criterio en el entorno virtual, debe prepararse desde los presupuestos analógicos de gran utilidad en la red --fragmentariedad, multimodalidad, hipertextualidad-- que el libro de no ficción le proporciona.

Esta hibridación que se erige como rasgo propio del libro de no ficción de los últimos años apunta a la posibilidad de desplazar los conceptos de ficción y no ficción de los libros a la lectura de tal modo que tal vez haya que pensar en que, en última instancia, sea el lector el que proponga una lectura ficcional o no ficcional. Valga como ejemplo de esto la lectura que sugiere \textit{Onsen ¿Qué hacen los monos?} Se trata de una obra que refleja una anulación de los límites entre la ficción y la no ficción y da paso a una historia ficcional con referentes no ficcionales que siembra en el lector la curiosidad por conocer más allá de los límites del libro. Este es el lector y, por ende, el modo de lectura a la que invitan muchos de los nuevos libros infantiles en el marco de la sociedad digital.

Tal como muestra el corpus analizado, el concepto de hibridación se extiende a la integración de un discurso que se define entre la razón y la emoción. Los libros ilustrados de no ficción contemporáneos se alejan de forma clara del tono enciclopédico y objetivo que ha caracterizado en el pasado esta clase de obras. Así, la pretensión de verdad absoluta deja paso a una interpretación subjetiva de la realidad que plantea distintos mundos posibles e interpela a un lector que debe colaborar en la creación de sentidos ante un libro que no le ofrece conocimientos puramente objetivos ni cerrados. En esta propuesta de lectura lo racional no se separa de lo emocional, sino que, como señala \textcite{grilli_beauty_2020b} ambos hemisferios cerebrales son requeridos en un proceso de generación de conocimiento que combina tanto la comprensión intelectual como la experiencia sensorial facilitada por el componente estético. Esta misma autora señala cómo el asombro viene ligado al conocimiento estético, un asombro que, en palabras de \textcite{montes_buscar_2018} llega al lector gracias a la combinación del arte y el conocimiento.

De esta manera, los libros analizados parecen evitar la búsqueda de control propia de los enfoques científicos y ofrecen una representación artística del mundo que, a través del asombro, promueve en el lector la adquisición de conocimientos. Esta relevancia del asombro y de la percepción sensorial en el aprendizaje, tan acorde al pensamiento de Montessori, es evidente en obras como \textit{Esto es París o en La línea del tiempo. Un viaje ilustrado por la historia}, en las que el lector accede a una representación artística, personal y subjetiva de la realidad que lo invita a maravillarse y a implicarse en la generación de sentidos.

\subsection{El lector crítico y un nuevo concepto de autoría}\label{sec-fmt-manuscrito}
En lugar de la literatura de respuestas que tradicionalmente ha identificado a la no ficción, la sociedad digital favorece la emergencia de una literatura de preguntas. \textcite{sanders_literature_2017} destaca cómo el libro de no ficción está caracterizado por el sesgo de un autor que selecciona la información que desea transmitir (en detrimento de otras) y que utiliza una serie de estrategias que invitan al lector a participar en el texto. Dicha participación se da por medio de un proceso que supera la mera absorción de información y que requiere que la persona que lee se involucre y desarrolle una investigación propia motivada por el asombro y la curiosidad producidos por la obra. Esto implica que los libros de no ficción transmiten una visión personal del universo que puede o no ser compartida por quien está leyendo \cite{grilli_beauty_2020b}, lo que proyecta un lector implícito crítico que se posiciona ante lo que se le expone, detecta fisuras en el discurso y cuestiona la credibilidad del autor.

Este alejamiento del concepto tradicional de autoridad se percibe, como sugiere \textcite{sanders_literature_2017}, en el modo en el que la voz, el personaje y los peritextos de las obras plantean fisuras que dibujan un lector crítico e implicado. Ejemplo de ello serían títulos como \textit{Supongamos o Mundo Cruel}, que interpelan a quien está leyendo con preguntas abiertas y sin respuestas, lo que, a su vez, reclama un lector adulto que acompañe la lectura en un proceso compartido de asombro y cuestionamiento. Del mismo modo, \textit{El árbol de la vida. Charles Darwin} ilustra cómo el género biográfico puede alejarse de la linealidad y ofrecer un relato fragmentario que invita al lector a construir su comprensión de la realidad a partir de lo que le exponen diversas perspectivas, para lo que debe vincular voces y tipografías y ser capaz de jerarquizar las informaciones que se le presentan.

Asimismo, la consolidación de esta noción de lector crítico en el libro de no ficción acaba dirigiéndose en ocasiones hacia un mensaje de concienciación en la línea de los ODS \cite{goga_ecocritical_2018}, por el que lo ecocrítico adquiere relevancia en obras que generan inquietudes desde la sostenibilidad medioambiental, como es el caso de \textit{Plasticus Maritimus}.

En consonancia con el nuevo tipo de lector implícito de los libros ilustrados de no ficción, se propone también una nueva concepción de autoría. Si atendemos a proyectos editoriales como el de Iamiqué, es posible identificar la ingente labor por parte de las editoras a la hora de seleccionar temas, autores, tono e información de los textos, ilustradores o formatos. Libros como \textit{Física hasta en la sopa} sugieren, por tanto, una noción de autoría coral en la que se requiere tanto la intervención del editor como la del autor científico para combinar el rigor de lo expuesto con la adecuación del tono del mensaje ideado por palabras e imágenes. Solo así es posible plantear un discurso científico que se entienda como una conversación entre una persona adulta y un lector infantil y que consiga atraer la atención de la infancia y de la edad adulta a través del reto y el interés de la información.


\subsection{Ruptura de la linealidad del discurso}\label{sec-formato}
Propone el corpus seleccionado un sentido renovado de la noción de paratexto definida por \textcite{genette_palimpsestos:_1989}. La fragmentariedad y la ruptura de la linealidad del discurso propias de estas obras implican que el paratexto no solo orienta la lectura del texto \cite{lluch_epitextos_2015}, sino que las notas del autor, la biografía, las anotaciones o los epígrafes pasan a formar parte del mensaje transmitido y contribuyen a la creación de significados. Valga como muestra la obra \textit{En construcción} y su anexo dedicado a la exposición comentada de los monumentos arquitectónicos emblemáticos en el siglo XX. Otro buen ejemplo de esto sería \textit{Nunca sonrías a un mono}, que termina con una advertencia de peligro al lector a través de la explicación de curiosidades sobre diferentes animales y presenta tras esto una exposición enciclopédica de sus estrategias de supervivencia.

Esta misma fragmentariedad es la que anima al lector curioso a realizar una lectura hipertextual que lo conduce a buscar información más allá del libro, algo que hace a través de los términos de búsqueda que se entiende que la obra destaca. La sección “Si te apetece saber más” de la obra \textit{Ahí fuera, guía para descubrir la naturaleza} es un ejemplo representativo por su facilitación de enlaces web de instituciones dedicadas al cuidado de la naturaleza. Sin embargo, no es necesario que los libros expliciten sus hipervínculos virtuales, ya que la mera presentación fragmentaria de su información invita al lector a acudir a otros libros y a fuentes digitales para ampliar y completar los contenidos.

Desde la ruptura de la linealidad, adquiere entidad propia la dimensión objetual en la aprehensión del conocimiento. En los títulos analizados se pueden reconocer las ideas de \textcite{munari_como_2016} sobre la naturaleza objetual del libro y sobre sus implicaciones en la interacción del niño con el mismo. Estas ideas encuentran su reflejo en la relevancia que la materialidad ha adquirido en la aprehensión de conocimiento en el libro ilustrado de no ficción. De este modo, la manipulación de las obras se presenta como una de las vías por las que el lector interacciona con formatos, textura, solapas y acetatos en un proceso metaléptico que lo invita a intervenir en la lectura y en la generación de conocimiento.

Así, obras como \textit{Zoóptica o Iluminaturaleza} requieren un lector que colabore manipulativamente para construir sentidos, descubrir cómo ven los animales o lo que la naturaleza esconde. El juego con desplegables y solapas, por otro lado, resulta imprescindible para que el lector se encuentre los utensilios pertinentes para las labores del huerto en \textit{Ya sé cultivar el huerto}, para que explore las características de \textit{La mariquita} de Bernadette Gervais o para que profundice en los miedos expuestos por el protagonista de \textit{El gran libro de los miedos del ratoncito}.

Una de las razones de la importancia que lo material está adquiriendo en el mercado del libro contemporáneo es la saturación actual de lecturas en pantalla y de formatos electrónicos que, como indica \textcite{salisbury_true_2020}, ha potenciado el redescubrimiento de la lectura como experiencia física y la explotación de lo material en la creación de sentidos. Esta paradoja por la que la sociedad digital ha fortalecido los aspectos físicos del libro \cite{cordon-garcia_combates_2018} se encuentra en consonancia, de hecho, con la defensa que \textcite{bonnafe_libros_2008} realizaba de la necesidad de jugar y manipular con los libros de los lectores infantiles.

\section{La dimensión cultural en el libro ilustrado de no ficción}\label{sec-modelo}
El alejamiento del tono enciclopédico que define al libro ilustrado de no ficción contemporáneo implica que la voz del autor se hace presente en la transmisión de informaciones y conocimientos en las obras. Como ya se ha expuesto, esta presencia reconocible del sesgo autoral requiere del lector una implicación crítica y cautelosa en la lectura y el aprendizaje. Así, una de las maneras a través de las cuales los libros ilustrados de no ficción se presentan como determinados por la perspectiva subjetiva de quien los escribe es la presencia de marcadores culturales \cite{schapers_especificidad_2016} que evidencian un modo de comprender el mundo y de transmitir información que se encuentra inserto en un contexto cultural concreto.

De este modo, se presentan obras como \textit{El gran libro de los miedos del Ratoncito}, álbum en el que Emily Gravett expone un tratado sobre fobias que se explican, en gran medida, a través de referencias culturales y literarias específicamente británicas. La misma autora parece querer aplicar en \textit{Meerkat Mail} un filtro conocido a una realidad ajena a ella, la fauna africana, lo que consigue principalmente a través de guiños a la cultura estadounidense en las ilustraciones. Esto permite a Gravett compaginar la presencia de informaciones sobre diferentes tipos de mangostas africanas y sus hábitats con una propuesta visual que hace referencia explícita a elementos culturales norteamericanos del siglo pasado, como el largometraje \textit{Cantando bajo la lluvia}, la campaña “I Love New York” de Milton Glaser, los \textit{diners} americanos o el estilo \textit{googie}. En la misma línea, las ilustraciones de Marta Pina en la reedición de \textit{Cómo puede ser la democracia} de la editorial Media Vaca parecen apelar a un lector adulto nostálgico que comprende la democracia desde el recuerdo concreto de la transición española, lo que es corroborado además en dos epígrafes al inicio y al final del libro.

Así, los libros de no ficción pueden sugerir al lector que está accediendo a una visión de la realidad mediatizada por la cultura en la que la obra se ha producido (o bien por culturas dominantes en el panorama global, como es la estadounidense), de tal manera que la presencia de marcadores culturales podría añadirse a las marcas por las que \textcite{sanders_literature_2017} plantea que el autor puede hacerse visible en las obras de no ficción. Sin embargo, esta influencia del contexto del autor sobre lo que se propone en la obra no siempre es evidente, por lo que el lector puede no llegar a reconocer que el texto se inserta en un contexto en particular y que, en consecuencia, es pertinente una aproximación cautelosa o atenta al mismo. Se entendería como tal aproximación una que prestase atención al hecho de que el autor se está comunicando en el marco de un sistema semiótico cultural determinado, así como a los posibles sesgos y a los elementos culturalmente específicos que determinan dicha comunicación \cite{osullivan_comparative_2005}.

En este sentido, en los últimos tiempos ha proliferado un tipo específico de libro de no ficción que, precisamente, explicita el modo en el que los contextos condicionan la construcción del discurso. Conviene hacer referencia, a este respecto, a títulos como \textit{El berrinche de Moctezuma}, \textit{Tu canción}, \textit{Historias de Nasrudín} o \textit{Jatakas}. \textit{Seis cuentos budistas}. Lo que las editoriales Ekaré y Akiara Books presentan en estas obras son historias ficcionales seguidas de un epígrafe no ficcional que, a modo de glosario o de guía de lectura, proporcionan al lector información que le permite realizar una segunda lectura más compleja de la historia desde la comprensión del texto en el marco de un contexto y una tradición cultural determinados.

En el caso de \textit{El berrinche de Moctezuma}, por ejemplo, el lector comienza accediendo en primer lugar a esta canción que se presenta sobre las páginas como una historia ficcional en verso acerca del emperador azteca y del origen del chocolate. Durante la lectura de la misma, es posible que dicho lector no repare en la trascendencia de ciertos elementos culturales que aparecen en el texto y cuyo conocimiento condiciona la recepción. La serpiente emplumada, la alusión a “la conquista” o la noción del tlatoani son ejemplos representativos en esta línea. Sin embargo, al llegar al final de la obra, el lector encontrará un glosario que le proporcionará información sobre estos elementos. Este mismo glosario le ofrecerá la posibilidad de acudir a otros libros o a la red para ampliar sus conocimientos sobre ello, permitiéndole asimismo realizar una segunda lectura de la obra en la que podrá aplicar un conocimiento cultural ignorado hasta ese momento. Esta obra en particular, por otro lado, refleja la labor que la editorial Ekaré está realizando en relación con la recuperación de la riqueza cultural y lingüística propia de los pueblos latinoamericanos.

Otro ejemplo representativo de lo comentado sería \textit{Historias de Nasrudín}, de la editorial Akiara Books. Su epígrafe final, que funciona como guía de lectura, es anunciado de manera explícita en la contracubierta a un lector adulto que sabrá reconocer el valor de la presencia de este apartado en la obra. Dicho epígrafe permite al lector comprender que las breves historias ficcionales presentadas en el libro provienen de una tradición cultural concreta, contexto en el cual adquieren su pleno sentido.

De esta manera, en las obras analizadas se presenta al lector un discurso fragmentado y no lineal que, desde la convergencia entre la ficción y la no ficción, llama la atención sobre la necesidad de contar con un cierto bagaje intelectual y cultural \cite{wolf_lector_2020} para poder realizar lecturas críticas y contextualizadas. Los paratextos, en la forma de epígrafes no ficcionales, son, por tanto, una de las claves mediante las que el libro de no ficción contemporáneo defiende la importancia de la dimensión cultural de la lectura y provee al mediador y al lector curioso de herramientas para dotar de sentido ubicándose en las fisuras que el texto plantea \cite{sanders_literature_2017}.

\section{Conclusión}\label{sec-organizacao}
Partiendo de investigaciones previas sobre el análisis del discurso de los libros ilustrados de no ficción\cite{grilli_non-fiction_2020a,pappas_information_2006,sanders_literature_2017,tabernero__2022,von-merveldt_informational_2018}, se ha planteado en este estudio la identificación de las claves de lectura de este tipo de obras sobre la selección de 30 títulos publicados en los últimos quince años.

Del análisis efectuado sobre este corpus, se desprende la consolidación de tres aspectos que aglutinan las claves ya determinadas en estudios previos \cite{grilli_non-fiction_2020a,tabernero__2022} y la emergencia de la dimensión cultural como una marca identificatoria con entidad propia.

Por una parte, se extiende a todas las instancias discursivas el concepto de hibridación de tal modo que se impone como eje articular de las obras analizadas desde la combinación de paradigmas analógicos y digitales, hasta la presencia de elementos ficcionales y no ficcionales pasando por la multimodalidad propia del libro ilustrado que resulta de la fusión de la naturaleza diversa de los distintos lenguajes que lo componen --gráfico, verbal y matérico--. Obras como \textit{Mil tomates y una rana. Historia de un huerto mínimo}, de Alex Nogués e ilustrado por Samuel Castaño resultan representativas en la evolución del modelo de lectura basado en el desarrollo de la curiosidad que se propone en el libro ilustrado de no ficción en los últimos años.

Por otra parte, se confirma el concepto de literatura de preguntas con el que \textcite{sanders_literature_2017} calificó el libro ilustrado de no ficción. Se trata de libros que superan el saber enciclopédico tradicional y ofrecen fisuras en su exposición tanto en lo que respecta a la fiabilidad del narrador como en lo que implica la inserción de una autoría coral compartida entre escritor, ilustrador, supervisores de conocimiento y editor responsable de los paratextos en todas sus dimensiones. Se instaura, vinculado a la autoría coral, el concepto de un lector crítico que deberá construir sentidos desde las diferentes propuestas y que asume, de esta manera, claves que le invitan a desenvolverse de manera crítica en la actual sociedad digital y sus modos de ofrecer información. Obras como \textit{Supongamos} de Ivar Da Coll o \textit{Plasticus maritimus}. \textit{Una especie invasora} de Pêg, Carvalho y Minhós constituirían ejemplos representativos del camino por el que transita el libro ilustrado de no ficción. Como consecuencia de lo expuesto, la ruptura de la linealidad del discurso identificada asimismo en tratados anteriores \cite{tabernero__2022} y anunciada en títulos asentados como \textit{Zoóptica} de Duprat se convierte en factor inherente a este tipo de obras tal como se refleja en títulos recientes como \textit{Ahí fuera, guía para descubrir la naturaleza} de Peixe, Teixeira y Carvalho.

Por último, se dibuja como emergente en el panorama editorial actual la importancia de la dimensión cultural \cite{schapers_especificidad_2016} en el libro ilustrado de no ficción como así lo sugieren obras publicadas en los últimos meses que recuperan un concepto de identidad cultural con el que responder a la globalización en la que el lector se inserta. En este sentido, títulos como \textit{El berrinche de Moctezuma} de Gómez y Solís o \textit{Historias de Nasrudín} de Bárcena y Cabassa pueden constituir ejemplos del asentamiento de una tendencia que parece acentuarse en el libro de no ficción de la actualidad con el fin de explicar, desde las características de este género, la posibilidad de anotar culturalmente el discurso desde la pluralidad de los diferentes contextos. Conviene, por tanto, profundizar en esta dimensión con el fin de consolidar los cimientos de una ciudadanía crítica que se explica en la diversidad.


\section{Agencia financiadora}\label{sec-conclusao}
El Ministerio de Universidades del Gobierno de España y su apoyo a través de una de las Ayudas para la formación de profesorado universitario (FPU).

El grupo de investigación de referencia ECOLIJ (Educación Comunicativa y Literaria en la Sociedad de la Información. Literatura Infantil y Juvenil en la construcción de identidades) a través del proyecto I D i “Lecturas no ficcionales para la integración de ciudadanas y ciudadanos críticos en el nuevo ecosistema cultural (PID2021-126392OB-100)” financiado por el Ministerio de Ciencia e Innovación del Gobierno de España.


\printbibliography\label{sec-bib}
% if the text is not in Portuguese, it might be necessary to use the code below instead to print the correct ABNT abbreviations [s.n.], [s.l.]
%\begin{portuguese}
%\printbibliography[title={Bibliography}]
%\end{portuguese}


%full list: conceptualization,datacuration,formalanalysis,funding,investigation,methodology,projadm,resources,software,supervision,validation,visualization,writing,review
\begin{contributors}[sec-contributors]
\authorcontribution{Rosa Tabernero Sala}[conceptualization,investigation,methodology,writing,review]
\authorcontribution{Daniel Laliena}[conceptualization,investigation,methodology,writing,review]
\end{contributors}


\end{document}

