% !TEX TS-program = XeLaTeX
% use the following command:
% all document files must be coded in UTF-8
\documentclass[spanish]{textolivre}
% build HTML with: make4ht -e build.lua -c textolivre.cfg -x -u article "fn-in,svg,pic-align"

\journalname{Texto Livre}
\thevolume{16}
%\thenumber{1} % old template
\theyear{2023}
\receiveddate{\DTMdisplaydate{2023}{1}{28}{-1}} % YYYY MM DD
\accepteddate{\DTMdisplaydate{2023}{3}{29}{-1}}
\publisheddate{\DTMdisplaydate{2023}{5}{11}{-1}}
\corrauthor{Angélica Janeth Cortez Soto}
\articledoi{10.1590/1983-3652.2023.42793}
%\articleid{NNNN} % if the article ID is not the last 5 numbers of its DOI, provide it using \articleid{} commmand 
% list of available sesscions in the journal: articles, dossier, reports, essays, reviews, interviews, editorial
\articlesessionname{articles}
\runningauthor{Cortez Soto et al.} 
%\editorname{Leonardo Araújo} % old template
\sectioneditorname{Hugo Heredia Ponce}
\layouteditorname{Thaís Coutinho}

\title{Profesores universitarios: condiciones de teletrabajo y uso de tecnologías en el marco de la enseñanza remota de emergencia}
\othertitle{Professores universitários: condições de teletrabalho e uso de tecnologias no contexto do ensino remoto de emergência}
\othertitle{University professors: teleworking conditions and use of technologies in the emergency remote teaching framework}
% if there is a third language title, add here:
%\othertitle{Artikelvorlage zur Einreichung beim Texto Livre Journal}

\author[1]{Angélica Janeth Cortez Soto~\orcid{0000-0002-8540-2417}\thanks{Email: \href{mailto:angelica.janet@gmail.com}{angelica.janet@gmail.com}}}
\author[2]{Sara Guadalupe Unda Rojas~\orcid{0000-0002-6113-055X}\thanks{Email: \href{mailto:saraunda@unam.mx}{saraunda@unam.mx}}}
\author[3]{Pedro Gil-LaOrden~\orcid{0000-0001-7541-43888}\thanks{Email: \href{mailto:pedro.gil-monte@uv.es}{pedro.gil-monte@uv.es}}}
\author[2]{Marlene Rodríguez Martínez~\orcid{0000-0002-9912-8500}\thanks{Email: \href{mailto:psicología.trabajo@hotmail.com}{psicología.trabajo@hotmail.com}}}
\author[2]{José Horacio Tovalín Ahumada~\orcid{0000-0003-4419-9392}\thanks{Email: \href{mailto:htovalin@gmail.com}{htovalin@gmail.com}}}
\affil[1]{Tecnológico de Monterrey, Escuela de Humanidades y Educación, Monterrey, Nuevo León, México.}
\affil[2]{Universidad Nacional Autónoma de México, Facultad de Estudios Superiores Zaragoza, Ciudad de México, México.}
\affil[3]{Universitat de València, Unidad de Investigación Psicosocial de la Conducta Organizacional (UNIPSICO), España.}

\addbibresource{article.bib}
% use biber instead of bibtex
% $ biber article

% used to create dummy text for the template file
\definecolor{dark-gray}{gray}{0.35} % color used to display dummy texts
\usepackage{lipsum}
\SetLipsumParListSurrounders{\colorlet{oldcolor}{.}\color{dark-gray}}{\color{oldcolor}}

% used here only to provide the XeLaTeX and BibTeX logos
\usepackage{hologo}

% if you use multirows in a table, include the multirow package
\usepackage{multirow}

% provides sidewaysfigure environment
\usepackage{rotating}

% CUSTOM EPIGRAPH - BEGIN 
%%% https://tex.stackexchange.com/questions/193178/specific-epigraph-style
\usepackage{epigraph}
\renewcommand\textflush{flushright}
\makeatletter
\newlength\epitextskip
\pretocmd{\@epitext}{\em}{}{}
\apptocmd{\@epitext}{\em}{}{}
\patchcmd{\epigraph}{\@epitext{#1}\\}{\@epitext{#1}\\[\epitextskip]}{}{}
\makeatother
\setlength\epigraphrule{0pt}
\setlength\epitextskip{0.5ex}
\setlength\epigraphwidth{.7\textwidth}
% CUSTOM EPIGRAPH - END

% LANGUAGE - BEGIN
% ARABIC
% for languages that use special fonts, you must provide the typeface that will be used
% \setotherlanguage{arabic}
% \newfontfamily\arabicfont[Script=Arabic]{Amiri}
% \newfontfamily\arabicfontsf[Script=Arabic]{Amiri}
% \newfontfamily\arabicfonttt[Script=Arabic]{Amiri}
%
% in the article, to add arabic text use: \textlang{arabic}{ ... }
%
% RUSSIAN
% for russian text we also need to define fonts with support for Cyrillic script
% \usepackage{fontspec}
% \setotherlanguage{russian}
% \newfontfamily\cyrillicfont{Times New Roman}
% \newfontfamily\cyrillicfontsf{Times New Roman}[Script=Cyrillic]
% \newfontfamily\cyrillicfonttt{Times New Roman}[Script=Cyrillic]
%
% in the text use \begin{russian} ... \end{russian}
% LANGUAGE - END

% EMOJIS - BEGIN
% to use emoticons in your manuscript
% https://stackoverflow.com/questions/190145/how-to-insert-emoticons-in-latex/57076064
% using font Symbola, which has full support
% the font may be downloaded at:
% https://dn-works.com/ufas/
% add to preamble:
% \newfontfamily\Symbola{Symbola}
% in the text use:
% {\Symbola }
% EMOJIS - END

% LABEL REFERENCE TO DESCRIPTIVE LIST - BEGIN
% reference itens in a descriptive list using their labels instead of numbers
% insert the code below in the preambule:
%\makeatletter
%\let\orgdescriptionlabel\descriptionlabel
%\renewcommand*{\descriptionlabel}[1]{%
%  \let\orglabel\label
%  \let\label\@gobble
%  \phantomsection
%  \edef\@currentlabel{#1\unskip}%
%  \let\label\orglabel
%  \orgdescriptionlabel{#1}%
%}
%\makeatother
%
% in your document, use as illustraded here:
%\begin{description}
%  \item[first\label{itm1}] this is only an example;
%  % ...  add more items
%\end{description}
% LABEL REFERENCE TO DESCRIPTIVE LIST - END


% add line numbers for submission
%\usepackage{lineno}
%\linenumbers

\begin{document}
\maketitle

\begin{polyabstract}
\begin{abstract}
El objetivo de este estudio es conocer las experiencias de los profesores universitarios durante y después de la enseñanza remota de emergencia (ERT), que surgió en el marco del aislamiento por la pandemia de Covid-19. Su metodología tiene un diseño cualitativo, no experimental, transversal y descriptivo con un enfoque fenomenológico. Como resultados, los profesores universitarios consultados consideran que, durante el ERT, el tiempo dedicado al trabajo aumentó, debido tanto a reuniones fuera del horario como a la creación de materiales educativos. Expresan tener que usar sus propios recursos para enseñar sus materias en línea y también detallan que hay una falta de capacitación tecnológica. Los profesores asimismo comentan que sufrieron afectaciones a la salud, como debilidad visual, dolores corporales, estrés, ansiedad y depresión. Como conclusión, los profesores enfrentan la situación actualizándose en tecnologías de la información y educativas, siendo flexibles y adaptándose. Una limitación de este estudio es haber obtenido la muestra con un método no probabilístico por conveniencia, en 35 Instituciones de educación superior mexicanas (IESM), por lo que es difícil generalizar estos resultados a todo México.

\keywords{Profesores \sep Tecnoestrés \sep Formación tecnológica \sep Tecnologias de la información \sep Universidades}
\end{abstract}


\begin{portuguese}
\begin{abstract}
O objetivo deste estudo é conhecer as experiências dos professores universitários durante e após o ensino remoto de emergência (ERT), que surgiram no contexto do isolamento devido à pandemia de Covid-19. Sua metodologia tem um desenho qualitativo, não experimental, transversal e descritivo com uma abordagem fenomenológica. Como resultado, os professores universitários consultados consideram que durante o ERT, o tempo dedicado ao trabalho aumentou, devido tanto às reuniões fora do horário quanto à criação de materiais educativos. Eles expressam ter que usar seus próprios recursos para ensinar suas matérias \textit{online} e também detalham que há falta de treinamento tecnológico. Os professores relatam ainda que sofrem de problemas de saúde como deficiência visual, dores no corpo, estresse, ansiedade e depressão. Em conclusão, os professores lidam com a situação mantendo-se atualizados com as tecnologias de informação e educação, sendo flexíveis e adaptando-se. Uma limitação deste estudo é que a amostra foi obtida com um método não-probabilístico por conveniência, em 35 Instituições Mexicanas de Ensino Superior (IESM). Por isso, é difícil generalizar esses resultados para todo o México.
\keywords{Professores \sep Tecnoestresse \sep Formação tecnológica \sep Tecnologias de informação \sep Universidades}
\end{abstract}
\end{portuguese}

\begin{english}
\begin{abstract}
The aim of this study is to learn about the experiences of university teachers during and after emergency remote teaching (ERT), which emerged in the context of isolation due to the Covid-19 pandemic. Its methodology has a qualitative, non-experimental, cross-sectional, and descriptive design with a phenomenological approach. As result, the university professors consulted consider that during the ERT, the time dedicated to work increased, both due to meetings outside the timetable and to the creation of educational materials. They express having to use their own resources to teach their subjects online and also detail that there is a lack of technological training. The teachers also commented that they suffered health problems, such as visual weakness, body aches, stress, anxiety and depression. In conclusion, teachers face the situation by updating themselves with information and educational technologies, being flexible and adapting. A limitation of this study is that the sample was obtained with a non-probabilistic method by convenience, in 35 Mexican Institutions of Higher Education (IESM), so it is difficult to generalize these results to all of Mexico.

\keywords{Professors \sep Technostress \sep Technological training \sep Information technologies \sep Universities}
\end{abstract}
\end{english}
% if there is another abstract, insert it here using the same scheme
\end{polyabstract}

\section{Introducción}\label{sec-intro}
En años recientes, la educación ha cambiado y evolucionado acorde con las transformaciones sociales, culturales y económicas, las cuales han afectado a los paradigmas educativos. Estos cambios han sido derivados principalmente de la invención y utilización de las tecnologías de la información y comunicación (TIC), lo cual ha traído problemáticas, investigaciones, adaptaciones, nuevas metodologías y retos \cite{tejadafernandez2000docente}. Hasta antes de la pandemia esas modificaciones habían sido paulatinas y acumulativas, sin embargo, este evento vino a traer cambios inesperados y acelerados en la educación.

A partir de que la Organización Mundial de la Salud declaró al COVID-19 como emergencia de salud pública de alcance internacional el 30 de enero del 2020, y el 11 de marzo la clasificó como pandemia, \cite{escudero_pandemia_nodate} en México a partir del 16 de marzo del 2020 se suspendieron las labores presenciales, incluyendo las educativas, para reducir la transmisión de la enfermedad \cite{secretariaacuerdo20}.

Debido al incremento de los casos de COVID-19 en México, las clases continuaron llevándose a cabo en diferentes modalidades decididas por cada nivel y centro educativo. Esto fue obligatorio hasta el 23 de agosto del 2021, cuando el “Diario Oficial de la Federación” publicó diversas disposiciones para reanudar las actividades del servicio educativo de forma presencial, responsable y ordenada, y dar cumplimiento a los planes y programas de estudio de educación básica, normal y demás para la formación de maestros de educación básica aplicables a toda la república, al igual que aquellos planes y programas de estudio de los tipos medio superior y superior \cite{secretariaacuerdo21}.

A nivel mundial, durante el periodo de educación a distancia obligatorio e incluso después, los docentes de más de 160 millones de estudiantes en Latinoamérica \cite{unesco} de los diferentes niveles educativos tuvieron que elaborar estrategias para continuar con el proceso de enseñanza-aprendizaje, generalmente a través de tecnología como medida compensatoria del sistema presencial \cite{almazan2020covid}.

A esta modalidad educativa derivada de la pandemia se le llamó “educación remota de emergencia” o ERT, ya que, en contraste con las metodologías de enseñanza-aprendizaje que se diseñan y planifican para llevarse a cabo a distancia, esta modalidad de enseñanza es una modificación provisional a un modelo diferente debido a circunstancias de crisis \cite{hodges2020difference}. La ERT implica la enseñanza a distancia acondicionada a las circunstancias de cada institución y curso, no obstante, estos cambios afectaron a los docentes, ya sea positiva o negativamente \cite{rocha_estrada_docentes_2022}. 

\textcite{garcia_aretio_covid-19_2021} señala que dos modalidades son posibles cuando la enseñanza es totalmente remota: a) sincronía en remoto al 100\%, replicando la presencialidad completamente y b) sincronía y asincronía, llevando en formato síncrono las enseñanzas del profesor, para realizar el resto del tiempo de trabajo también en línea, pero en formato asíncrono. 

La enseñanza a distancia implica necesariamente el uso de las TIC por lo tanto, su uso se volvió un aspecto primordial. Estas tecnologías dependen de las circunstancias espaciales de las instituciones formativas, nivel socioeconómico del alumnado y sus familias; y nivel de dominio digital \cite{gomez-arteta_educacion_2021}. En el caso de la educación media superior y superior, debido a la complejidad de las actividades de enseñanza-aprendizaje, las tecnologías elegidas fueron las computadoras de escritorio o laptops, ya que, los profesores utilizaron correo electrónico, chats, videoconferencias y videos pregrabados \cite{hidalgo_cajo_adopcion_2021}.

Durante este periodo, diferentes estudios señalan afectaciones diversas por el uso excesivo de la tecnología en el trabajo vía remota de académicos, particularmente de educación superior. \textcite{murgu_modern_2021} reporta en los docentes sobrecarga laboral, tecnoestrés, inseguridad profesional y síndrome del impostor, además de agotamiento, ansiedad y sentimientos de inadecuación; \textcite{casali_impacto_nodate} en un estudio con docentes argentinos reporta en el 51.2\% dolores musculares y corporales y en el 44.5\% ansiedad y un mayor grado de empeoramiento en esta situación de trabajo, debido a la falta de formación y capacitación de los docentes de una metodología didáctica y pedagógica en el ámbito del trabajo virtual. \textcite{estrada-munoz_technostress_nodate} señalan en una muestra de profesores chilenos, que durante la pandemia del COVID 19 los profesores presentaron, tecno-fatiga y tecno-ansiedad, escepticismo y síntomas de ineficacia.

Dentro de los aspectos positivos de la ERT, destaca que este proceso logró que los docentes reflexionaran sobre sus prácticas pedagógicas, buscando la mejora y la innovación \cite{rodriguez2021ensenanza}. Además, en el estudio de caso realizado por \textcite{gonzalez_fernandez_capacitacion_2021} donde participaron 337 profesores del Bachillerato Tecnológico del Estado de Jalisco, se obtuvo como resultado que los participantes consideraron que mejoraron su habilidad en el uso de Google Drive y Google Classroom; y ampliaron sus conocimientos de la evaluación en línea, creación de cursos y recursos.

\section{Método}


Debido a la pandemia y a la ERT, son necesarias investigaciones que recojan los relatos de vivencias, opiniones y percepciones de los profesores durante los años que duró el confinamiento de emergencia por la pandemia de COVID-19 y en los años de adaptación hacia la nueva normalidad, con la finalidad de mejorar estas condiciones en las modalidades que aún se mantienen y en las que puedan surgir en el futuro.

Ante este panorama, consideramos que los profesores universitarios mexicanos pudieron vivir condiciones de dificultad similares a las de otros docentes en Latinoamérica y a nivel mundial, por lo que el objetivo del estudio fue conocer las experiencias, algunos problemas y afectaciones de los profesores durante y después de la ERT, que surgió debido al aislamiento por la pandemia Covid-19.

El presente estudio tiene un alcance exploratorio, con un diseño cualitativo, no experimental, transversal, descriptivo y con un enfoque fenomenológico, en el que, según \textcite{fuster_guillen_investigacion_2019}, se pretende comprender la naturaleza de los hechos e incluso transformarlos, a través del estudio de las experiencias de vida respecto de un suceso, desde la perspectiva del sujeto. Los resultados de este estudio son parte del proyecto PAPIIT UNAM IN302922.

\section{Participantes}

La muestra fue no probabilística de conveniencia. Se invitó a profesores de 35 instituciones de educación superior públicas y privadas de diferentes estados de México a participar en un estudio sobre teletrabajo y tecnoestrés, derivados de la educación remota de emergencia. El estudio contó con la aprobación del comité de ética de la Facultad de Estudios Superiores Zaragoza en el contexto del proyecto.

La muestra total fue de 669 profesores, sin embargo, la pregunta cualitativa fue contestada por 100 profesores. De estos 100 profesores, con un promedio de edad de 51.7 años y 19.2 años de experiencia docente; 34.5\% pertenecen a escuelas privadas y 65.5\% a escuelas públicas; 16.5\% cuenta con licenciatura, 43\% con maestría y 40.5\% con doctorado.

Los criterios de inclusión para participar en el estudio fueron: a) vivir la transición del formato presencial a la modalidad en línea, b) impartir clases en el nivel pregrado o posgrados universitarios, y c) impartir clases en México.


\section{Procedimiento}

Se estableció contacto con los docentes vía electrónica por medio de las autoridades educativas de sus universidades y/o su correo electrónico personal publicado en las páginas de las universidades. Se les envió un mensaje donde se explicaron las características del estudio y se envió a los participantes la liga del formulario de \emph{google forms} con la carta de consentimiento y un cuestionario sobre condiciones de teletrabajo y tecnoestrés. En el cuestionario se pedían datos demográficos y laborales de los docentes, la sección cualitativa consistió en una pregunta de texto abierto donde se solicitaban comentarios sobre sus condiciones de teletrabajo y uso de tecnologías durante la pandemia.

\section{Análisis de datos}

Para codificar la información se utilizó el software MaxQDA, se cargaron las respuestas y se realizó una codificación obteniendo 8 códigos y 195 estados significativos. Durante el análisis de la información se utilizó el procedimiento de \textcite{perez_serrano_investigacion_1994}, en el cual se explica que el análisis de datos en la investigación cualitativa consiste en reducir, categorizar, sintetizar y comparar la información, para así obtener una visión completa de la posible realidad objeto de estudio.


\section{Resultados}

Después del análisis de los datos surgieron dos dimensiones principales que incluyen categorías y subcategorías: a) retos profesionales laborales; b) retos personales como trabajadores. A continuación, se describen a detalle los campos, resaltando que no se incluyeron todos los comentarios, solo se eligieron los más representativos.

\subsection{Dimensión 1: Retos profesionales laborales}
Esta dimensión recopila los retos profesionales que enfrentaron los profesores debido a los cambios en los procesos de enseñanza y administrativos; y los retos derivados de las condiciones laborales, como los horarios de trabajo, juntas y los equipos tecnológicos disponibles para el uso de los profesores.

\begin{itemize}
 \item Categoría uno: Exceso de trabajo
\end{itemize}

El 20\% de los docentes consideraron que la planeación y el diseño de materiales educativos aumentó considerablemente el tiempo que dedicaron a su trabajo, el15\% de los docentes manifestaron que las reuniones, asesorías y comunicaciones, al ser a distancia, no respetaron el horario laboral.
\begin{quote}
 “... Una materia de dos horas a la semana se convirtió en mínimo 8 horas de manera virtual”. “Al inicio fue mucho más difícil (...) fue horrible, difícil, cansado (...) ahora ya no es en línea, pero muchas cosas son en línea”. 
 
“... Durante la pandemia el horario de trabajo aumentó por la cantidad de actividades didácticas que había que diseñar para garantizar el aprendizaje, y por los horarios de asesoría interminables para los alumnos”. “La demanda de tiempo por reuniones de trabajo se duplicó y se espera que trabaje no solo a distancia o presencial (hibridez), sino simultáneamente estar en dos reuniones al mismo tiempo”. 

“Convocan a muchas reuniones que a veces solo hacen perder el tiempo”. “El teletrabajo ha invadido mi vida familiar y generado disturbios graves en mi paz mental” “Lo más pesado es que es más trabajo y te envían mensajes y correos a la hora que quieren”. 

“Otro problema ha sido que hay algunos directivos que no respetan los tiempos en las reuniones, pues si regularmente estas duraban dos horas, en ocasiones las han alargado hasta cuatro horas, sin llegar a un punto en concreto, lo cual me parece una total pérdida de tiempo”. “Al estar en la universidad en transición tenemos actividades presenciales y muchas en línea con actividades académicas, lo cual implica trabajar tanto en la institución como en casa y a mí y a mis colegas les pasa es que nunca terminamos y cada día nos piden más trabajo que no se ajusta al tiempo contratado, pero eso no importa, hay que sacar el trabajo porque está en riesgo tu contratación”. 

“En pandemia enfermé, perdí familiares y perdí mis estímulos. Ni modo esas son las reglas del juego, no hay ninguna consideración humanística”. “Las autoridades implican que al estar en casa trabajando queda uno a su disposición a cualquier hora y día. Se vuelve obligatorio usar el propio celular en cuestiones como recibir información de reuniones administrativas y académicas, participar en reuniones de trabajo, pertenecer a grupos de docentes y de alumnos, entre otros puntos, dejando de lado las vías oficiales de comunicación que se han vuelto receptores de publicidad. Todo lo anterior sin importar el día y la hora”. 
\end{quote}

\begin{itemize}
 \item Categoría dos: Falta de apoyo Institucional en capacitación y recursos digitales.
\end{itemize}


El 61\% de los profesores emitió opiniones negativas, de los cuales el 30\% señalan la necesidad de utilizar sus propios recursos para impartir sus materias en línea, tanto antes de la pandemia como durante y después. El 15\% reporta falta de capacitación tecnológica proporcionada por la institución educativa, el 9\% las fallas en el internet que ocasionan problemas para impartir la clase tanto dentro como fuera de la institución. Un 2\% menciona la falta de apoyo psicológico que pudiera haber ofrecido la institución, el 3\% los bajos salarios de los docentes y por último el 2\% las fallas en los exámenes.
\begin{quote}
 “Desde 2015 que comencé a impartir clases, yo he utilizado mi propia computadora y proyector porque la institución no cuenta con proyectores suficientes”. “Se han incrementado mis gastos en equipo y accesorios para la computadora, pero los tomo como inversión en material didáctico, hay temas que no se pueden tratar a distancia con los estudiantes”. 
 
“El costo económico que ha implicado dar las clases vía remota ha sido financiado por parte de los y las docentes, sin que por parte de la institución haya alguna iniciativa para apoyar al respecto”. 

“No tengo computadora en mi trabajo desde antes de la pandemia, se descompuso y por eso, sigo trabajando desde mi casa en línea, con mis propios recursos”. “En mi institución se le da poca importancia a la capacitación de nuevas tecnologías, se enfocan más en las materias de la carrera. Tenemos que buscar por fuera tal capacitación”. “La conexión a internet desde mi institución es muy inestable y no en todos los espacios hay la misma calidad de la red”. 

“No hay apoyo psicológico ni pedagógico, mucho menos político para superar el estado de anomia de los estudiantes y de los académicos que viven haciendo planes de estudio y la realidad se cae a pedazos a nuestro alrededor, eso es muy molesto y depresivo”. “Me preocupa que los docentes seamos considerados poco importantes y los salarios no mejoren”. “Los exámenes en línea a veces no los recibo a tiempo para checarlos”. 
\end{quote}


\begin{itemize}
 \item Categoría tres: Apoyo institucional y capacitación
\end{itemize}

Solo el 18\% presentaron comentarios positivos sobre sus instituciones, de los cuales el 8\% señalaron que les ofrecieron todo el equipo necesario para dar sus clases, el 4\% recibieron capacitaciones necesarias para afrontar las clases virtuales y el 6\% recibió apoyo para superar los problemas, cabe resaltar que la mayoría de los profesores que respondieron que contaron con apoyo se encuentran en instituciones privadas.
\begin{quote}
 “Mi universidad es pionera en clases virtuales, la pandemia no nos hizo retroceder, al contrario, nos apoyó mucho y como todos estamos acostumbrados a usar plataformas, dispositivos electrónicos y aplicaciones para la impartición de clases, el migrar a clases en línea fue rápido y no sentimos estrés alguno”. “He tenido ayuda por parte de mi universidad en cuanto a computadora, micrófono, cámara y un monitor extra de la laptop, además de silla adecuada”.

 “Aunque el trabajo es altamente demandante, la institución educativa ofrece los apoyos humanos y materiales con los que cuenta, en todo momento”.

 “Siempre tengo apoyo por parte de mi Institución, mis alumnos son muy participativos y cumplidos con tareas, comentarios, sugerencias y participaciones en clase, me gusta trabajar en línea, con la plataforma de la Universidad. En (…) “el salón donde desempeño mi trabajo, tengo todo como: computadora, cañón, programas (...) por lo tanto, hago mi trabajo bien, utilizando la tecnología con la que cuento para mejorar el aprovechamiento de mis alumnos”.
\end{quote}

\begin{itemize}
 \item Categoría cuatro: Dificultades en la interacción y comunicación humana con estudiantes
\end{itemize}

Un aspecto negativo en la interacción y comunicación humana con estudiantes presente en las respuestas de los profesores fue que el 22\% consideró como problema en el aprendizaje en línea que muchos alumnos no prendieron la cámara, el micrófono o no participaron de la misma manera en que lo hacían en clases presenciales, y por lo tanto, no se consiguieron los aprendizajes esperados o no era posible conocer su avance, provocando angustia y frustración en los docentes. Otro aspecto relevante fue que el 7\% señaló que hay materias que no es posible ejecutar adecuadamente en línea, en este estudio se mencionaron música, medicina, física, química y mecánica. Además otra preocupación es que el 2\% de los docentes consideran el posible plagio por parte de los estudiantes.
\begin{quote}
 “He tenido que marcar límites a mis alumnos, llegaron a llamarme a las 12:00 horas de la noche del domingo para preguntarme algo que estaba en la plataforma”. “Nunca abren sus cámaras, con diferentes excusas, no sirve la cámara, problemas de internet o que están desde su teléfono”. “Algunos no asisten a clase y a pesar de estrategias que implemento para asegurarme que están atendiendo, simplemente dejan abierta la sesión y no contestan y dejan de participar”.
 
“En muchas ocasiones el alumno te manda información o escribe por el chat de la plataforma de la escuela fuera de todo horario escolar”. “Noto que los alumnos al regreso a clases presenciales están apáticos, poco participativos, no tienen conocimientos adecuados que debieron haber adquirido en los semestres que tuvieron clases en línea”. 

“Las clases presenciales son fundamentales, ya que, el alumno debe realizar experimentos y desarrollar las habilidades y destrezas en el uso y manejo de las técnicas experimentales, es más fácil para los alumnos entender la resolución de problemas en un pizarrón, puesto que en las plataformas no hay esta interacción efectiva”. “En lo personal me preocupa mucho el plagio en las tareas o trabajos, considero importante la implantación de softwares que detectan cualquier tipo de plagio”.
\end{quote}

\begin{itemize}
 \item Categoría cinco: Clases en línea vs. clases híbridas o presenciales
\end{itemize}

Después del confinamiento y la necesidad del regreso a clases híbridas o presenciales, algunos profesores manifestaron los aspectos positivos que lograron en su adaptación a las clases en línea y preferir mantenerse de esa manera. El 22\% señaló que el gasto de tiempo y dinero en traslados es menor, el 8\% de profesores señaló al trabajo híbrido como trabajo doble y no lo considera conveniente, el 5\% consideran que mantener días de clases en línea y días presenciales es la propuesta más adecuada, y el 3\% consideran necesarias las clases presenciales.
\begin{quote}
 “Me gusta trabajar en línea, me he formado una disciplina como si estuviera en el instituto, y estoy contenta porque no me estreso por la ida todos los días porque me queda lejos y el tráfico de autos no lo vivo”. “Me es muy cómodo siempre y cuando no sea híbrido. Al ser híbrido el trabajo se multiplica exponencialmente”.
 
“En nuestra facultad hemos regresado a actividades de "forma voluntaria" en un modelo híbrido, donde nos proponen que nuestro propio celular sirva como micrófono para la clase en línea sin descuidar a las personas que están en presencial. El regreso voluntario ha resultado contraproducente porque ni las ni los estudiantes ni la mayoría de las y los docentes lo toman en serio, lo cual repercute en un bajo compromiso del estudiantado y más desgaste para nosotros como docentes”. “Considero que es importante la incorporación total de horas presenciales”.

“Me preocupa que los alumnos falten mucho a las clases en línea, por referir deficiencia o falta de equipos, así como red o internet y por problemas personales o familiares, como tener que trabajar o cuidar familiares enfermos. La materia que imparto es principalmente de práctica, atender a pacientes, lo cual realizan desde el lugar de origen, ya que, la mayoría de los alumnos son foráneos, por lo tanto, me envían vídeos o audios de cada consulta, claro respetando la identidad del paciente, pero es muy cansado ver cada video de un grupo grande de 32 alumnos, cuando deberían ser máximo 20, me agobia en verdad”.
\end{quote}


\subsection{Dimensión 2: Retos personales como trabajador}
Esta dimensión recopila los retos personales que vivieron los profesores debido al uso prolongado de las tecnologías computacionales, destacando afectaciones a la salud física y mental, la necesidad de adaptarse a las nuevas condiciones de trabajo y la necesidad de socialización y retroalimentación que experimentaron durante la ERT.

\begin{itemize}
 \item Categoría seis: Afectaciones a la salud
\end{itemize}

Las afectaciones para los profesores fueron diversas, pero una de las más preocupantes fueron los impactos a su salud física y mental, de acuerdo a lo reportado, podemos considerar tres tipos de afectaciones a la salud: primeramente la fatiga visual ocasionada por el uso de dispositivos electrónicos (6\%), segundo, el reporte de estrés, ansiedad y depresión causados por los cambios laborales y los problemas derivados de la pandemia (5\%) y por último, las dolencias musculoesqueléticas por tener que mantenerse en la misma posición frente al computador la mayor parte del día (5\%).
\begin{quote}
 “El trabajo en línea ha propiciado el desgaste visual a mi persona por el número de horas que paso usando la computadora”. 
 
“En general, las clases en línea se desarrollan bien y los alumnos son comprometidos, solo he notado que mi vista ha desmejorado después del uso constante de la computadora debido a la pandemia”.

“Necesito un espacio o ayuda para enfrentar todo el estrés, angustia, ansiedad que me han provocado todos estos cambios de modalidad de enseñanza”. “Mis niveles de estrés han aumentado considerablemente; estando a veces al borde del burnout”.

“Debido a que pasaba más de siete horas diarias frente a la computadora se me lastimó la espalda, ahorita tengo un problema de síndrome piriforme o sea ciática lumbar, ya estoy en tratamiento, pero es muy doloroso”. 

“El teletrabajo deteriora la salud de espalda, piernas y cuello, se necesitan lentes de mayor graduación”

“Definitivamente la contingencia por la COVID-19 ha cambiado la forma de gestionar la práctica docente, los efectos derivados y sus repercusiones en los académicos han ocasionado inclusive la generación de una patología”.
\end{quote}

\begin{itemize}
 \item Categoría siete: Adaptación a los cambios
\end{itemize}

Los profesores expresaron haber tenido que adaptarse a los cambios en la educación durante la pandemia y después de ella, lo cual lograron por medio del esfuerzo y por el compromiso con sus estudiantes (15\%), sin embargo, esta adaptación tuvo consecuencias importantes en el bienestar de los docentes.
\begin{quote}
 “Es muy importante adecuarse a las situaciones y cambios”. “Es un proceso con una curva de aprendizaje grande, pero una vez dominada puede ser fácilmente adaptable”. 
 
“La sociedad y la educación evolucionan todo el tiempo, el docente debe evolucionar también, con vocación, compromiso y respeto por sus estudiantes”. “Es un gran cambio, el trabajo en la pandemia modificó la forma tradicional del salón de clases. Y nos tuvimos que adaptar de un día al otro”. 

“Si bien el teletrabajo requiere de un cambio total de percepción y de una buena dosis de autodisciplina”. 

“Los procesos enfrentados ante los cambios en los modos de trabajar, ya sea en presencial y en línea o virtual, a todos nos afectan. Las instituciones no están totalmente preparadas-actualizadas con las tecnologías y por ende, quienes laboramos en ellas dependemos en gran parte de sus programas, sus actualizaciones y políticas. La actitud como académico siempre ha sido y será de avanzar de manera positiva”.
\end{quote}

\begin{itemize}
 \item Categoría ocho: Aislamiento vs. Socialización
\end{itemize}

Los profesores comentaron que algo muy importante que se pierde con la ERT es la interacción y la socialización con los alumnos, además de la convivencia y retroalimentación por parte de compañeros (12\%). El aislamiento en el teletrabajo es una de las mayores fuentes de estrés tanto en docentes como en los estudiantes. Estos profesores resaltan la necesidad de mayor interacción humana y el motor de la motivación y la afectividad.
\begin{quote}
 “La tecnología favorece y facilita los procesos, pero limita la capacidad de interactuar y crecer como individuos”. “Es bueno el uso de TIC, pero con la pandemia sufrimos un proceso de adecuación muy complicado, y se perdió mucho los vínculos afectivos y de motivación al trabajo con los estudiantes, para mí es mucho mejor el sistema presencial, que permita la interacción humana, y se fortalezcan los vínculos socio afectivos”.
 
“Prefiero las clases presenciales, por tener interacción con los alumnos”.

“El trabajo en línea no permite tener interacciones cercanas y objetivas de los alumnos, además existe una disparidad entre los estudiantes en cuanto al acceso de equipo y conexión a internet, por tanto, resulta complicado hacer evaluaciones justas a su desempeño”.
\end{quote}

\section{Discusión}

Los resultados que se obtuvieron sugieren que la actividad de los docentes durante la pandemia representa grandes retos, tanto laborales como individuales, por las exigencias que tuvieron que enfrentar y que, actualmente en el regreso de manera presencial hay un balance tanto positivo como negativo por las posibles afectaciones presentadas.

En los relatos podemos observar cómo los trabajadores docentes en un primer momento se ven avasallados por la falta de recursos, de formación, de manejo de las TIC, de la falta de apoyo institucional, el aislamiento, la ausencia de límites en sus actividades académicas. A lo largo del proceso, estas problemáticas se van superando, ya que, las condiciones ponen en juego los recursos disponibles, tanto individuales como institucionales, para enfrentar las problemáticas y vicisitudes del momento.

Un primer reto señalado es la sobrecarga laboral, debido a tareas que se tuvieron que enfrentar y a la falta de límites que dieron como resultado una jornada de trabajo amplia e intensiva, lo que concuerda con estudios realizados por \textcite{godoy2022carga}, que señala que el trabajo del docente aumentó tanto por la necesidad de elaborar material didáctico, la atención a alumnos y padres de familia, lo cual afecta su desempeño debido a que incrementó la presencia de estrés. También el uso intensivo de las TIC \cite{mendoza_castillo_lo_2020}, al señalar que las tecnologías de la información pueden tener una alta curva de aprendizaje, lo cual dificulta la creación de recursos o materiales interactivos para mejorar la calidad de la clase. Al igual que en la investigación realizada por \textcite{galvis_lopez_tensiones_2021}, los profesores manifiestan un incremento sustancial en la carga laboral, esto derivado del diseño de actividades en formato digital, lo cual implica el uso de herramientas desconocidas por algunos docentes, además de la desorganización y exigencias de las universidades de atender reuniones, grupos en aplicaciones y ofrecer atención personalizada a los alumnos en cualquier momento del día \cite{cortes2021teacher}, lo cual ocasionó una invasión a la vida personal y familiar de los profesores.

Además, la mayoría de los profesores participantes han manifestado su molestia por el hecho de tener que invertir su salario en la compra de equipos de cómputo y allegarse de sus propios recursos para impartir sus clases, aunado a esto, el gasto de sus propios recursos para mejorar sus servicios de internet dentro de sus hogares, aspecto mencionado en el estudio de \textcite{sanchez_mendiola_retos_2020} con profesores universitarios, el cual señala esta dificultad, aunado al uso de diversas plataformas educativas y a la preocupación de los profesores al saber que sus estudiantes también podrían carecer del recurso y limitar entonces su aprovechamiento y su aprendizaje.

Un siguiente hallazgo es el apoyo institucional heterogéneo, ya que, los resultados que se obtuvieron son desiguales, pues en algunos casos de profesores afiliados a instituciones privadas, estos ya contaban con la capacitación y la formación necesaria en el manejo de TIC, además recibían los recursos materiales para llevar a cabo su labor, pero no así en algunas universidades públicas, donde los profesores debieron conseguir la capacitación a través de sus propios recursos. Lo que es cierto es que la formación y capacitación fue disímil y llegó en diversos momentos de la pandemia, y se empezó a generalizar hasta finales del segundo año y ahora en el regreso de clases híbridas y presenciales.

Algunos profesores participantes señalaron que no reciben capacitación por parte de la institución y consideran que este componente hace falta. Los que sí la recibieron, la agradecen y valoran; sin embargo, en otros estudios con profesores como en el de \textcite{rocha_estrada_docentes_2022}, los profesores consideraron a la capacitación como horas de trabajo extra no remunerado, señalando que no debe ser obligatoria y debe ajustarse a las necesidades específicas de cada profesor.

Como se señaló en los resultados, los profesores consideraron a las dificultades de interacción con sus estudiantes como un aspecto central, pues constituye un elemento esencial para cumplir con el objetivo del proceso de formación a nivel universitario, además de ser una de las actividades sustanciales de las universidades. Sin embargo, los docentes han expresado incomodidad con los alumnos en las clases virtuales por no prender su cámara y desconocer si ellos están presentes y activos en las clases o no, sin embargo, reconocen que deben ser comprensivos ante la situación personal de cada uno \cite{monetti_ensenar_2021}. Esta falta de interacción y socialización, al parecer, ha sido una situación problema para la educación a nivel mundial, ya que, el excesivo uso de las tecnologías ha sido asociado a una forma deshumanizante del aprendizaje, contrario a la gran importancia que se da al aprendizaje interactivo, social y colaborativo, como señala \textcite{covarrubias_hernandez_educacion_2021}, uno de los principales retos es humanizar el aprendizaje a distancia. 

Adicional a estas molestias, los profesores han exteriorizado que algunos de ellos han sufrido afectaciones en su estado de salud por el uso prolongado de dispositivos electrónicos, como estrés, debilidad visual y dolores musculares, los cuales son efectos comunes del denominado tecnoestrés \cite{tarafdar_impact_2007, ragu-nathan_consequences_2008} resultado de las circunstancias estresantes asociadas al manejo de las tecnologías.

\section{Conclusiones}

Este estudio indaga de forma exploratoria el impacto de la ERT en los docentes, mediante una perspectiva fenomenológica, desde las opiniones, vivencias y experiencias de profesores de pregrado y posgrado, tanto de instituciones públicas como privadas, de diversos estados de la república mexicana.

Al concluir este estudio se logra identificar que la falta de capacitación y de recursos materiales en tecnologías de la información es uno de los problemas que más afecta a los profesores, lo que hace considerar que en algunas de las universidades mexicanas se espera que el docente utilice sus recursos personales para trabajar desde casa y en ocasiones incluso para el trabajo dentro de las instituciones, lo que es contrario a las reformas al Artículo 311 de la Ley Federal del Trabajo en materia de Teletrabajo \cite{secretariaacuerdo21}.

Los docentes explicaron sus opiniones respecto a los modelos de aprendizaje en línea, presencial e híbrido, de acuerdo con la materia que imparten y sus circunstancias personales; por lo cual es necesario facilitar que, tanto docentes como alumnos puedan elegir la modalidad que se adapte mejor a sus necesidades. Además, ya que el aprendizaje a distancia durante la ERT limitó los espacios para socializar y convivir, afectó la capacidad de compartir el conocimiento, comunicar y retroalimentar adecuadamente. Queda claro que es necesario mejorar este aspecto, buscando actividades lúdicas que puedan adaptarse al formato en línea e integrarlas a los calendarios escolares.

Es importante que todos los actores del sistema educativo se preparen para afrontar los cambios que conlleva el uso prolongado de las nuevas tecnologías, las diferentes metodologías de enseñanza derivadas de estas (como el aula invertida) y el uso de plataformas como Canvas, Blackboard, Moodle, o herramientas de videoconferencias como Google Classroom, Zoom y Teams, además de que elaboren planes para actualizarse constantemente, preferentemente por medio de las facilidades que deben ofrecer las Universidades.

Esta investigación refleja algunos de los retos a que se enfrentaron los profesores, los cuales fueron superados actualizándose, siendo flexibles y adaptándose. Destaca que los retos que tuvieron que sobrellevar los docentes no fueron exclusivos de algún rango de edad, sexo, preparación académica, o si el profesor trabaja en una universidad privada o pública. Por lo mismo es importante resaltar que estas implicaciones parecen estar determinadas por las facilidades, los recursos y las políticas educativas de las organizaciones.

El presente estudio no está libre de limitaciones. Una limitación es haber obtenido la muestra con un método no probabilístico por conveniencia, en 35 Instituciones de educación superior mexicanas (IESM), por lo que es difícil generalizar estos resultados a todo México y de los participantes en el estudio una minoría presentó sus opiniones. Otro aspecto a considerar es el tipo de universidad donde laboran los docentes, puesto que el 65.5\% de la muestra eran profesores de universidades públicas. Finalmente, la ausencia de datos longitudinales no permite conocer si las condiciones de los profesores han cambiado, pues actualmente las clases en formato presencial han aumentado y las clases a distancia han disminuido, esto debido a la reducción de la severidad y los casos de la pandemia COVID-19.

Los estudios futuros deberían considerar estas limitaciones y extender el muestreo a otras universidades de México, así como replicar este estudio y que pueda ser útil para otras áreas profesionales. Asimismo, contribuiría la triangulación y confiabilidad de la información, contar con otro tipo de medidas objetivas o provenientes de observadores externos. A pesar de todo ello, el estudio es de especial interés teniendo en cuenta el tamaño de la muestra objeto de estudio, así como el importante número de universidades participantes, tanto públicas como privadas.

Los resultados del estudio podrían ayudar a diseñar programas de intervención que promuevan la salud y el bienestar emocional de los docentes. Además, es importante que las universidades tomen en cuenta los factores ergonómicos que contribuyen al deterioro de la salud de los docentes, pues a mediano y largo plazo podrían aumentar los costos en incapacidades y disminuir la calidad del proceso enseñanza-aprendizaje.

\section{Agradecimiento}
El presente trabajo deriva del Proyecto PAPIIT IN302922 “Programa de Intervención psicosocial para prevenir y atender Tecnoestrés, Síndrome de Quemarse por el Trabajo y síntomas asociados en profesores universitarios” de la F.E.S. Zaragoza, Universidad Nacional Autónoma de México.




\printbibliography\label{sec-bib}
% if the text is not in Portuguese, it might be necessary to use the code below instead to print the correct ABNT abbreviations [s.n.], [s.l.]
%\begin{portuguese}
%\printbibliography[title={Bibliography}]
%\end{portuguese}


%full list: conceptualization,datacuration,formalanalysis,funding,investigation,methodology,projadm,resources,software,supervision,validation,visualization,writing,review
\begin{contributors}[sec-contributors]
\authorcontribution{Angélica Janeth Cortez Soto}[review,investigation]
\authorcontribution{Sara Guadalupe Unda Rojas}[projadm,formalanalysis,investigation,review]
\authorcontribution{Pedro Gil-LaOrden}[investigation,formalanalysis,review]
\authorcontribution{Marlene Rodríguez Martínez}[formalanalysis,investigation,review]
\authorcontribution{José Horacio Tovalín Ahumada}[investigation,formalanalysis,review,supervision]
\end{contributors}



\end{document}

