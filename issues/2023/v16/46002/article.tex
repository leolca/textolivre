% !TEX TS-program = XeLaTeX
% use the following command:
% all document files must be coded in UTF-8
\documentclass[portuguese]{textolivre}
% build HTML with: make4ht -e build.lua -c textolivre.cfg -x -u article "fn-in,svg,pic-align"

\journalname{Texto Livre}
\thevolume{16}
%\thenumber{1} % old template
\theyear{2023}
\receiveddate{\DTMdisplaydate{2023}{4}{28}{-1}} % YYYY MM DD
\accepteddate{\DTMdisplaydate{2023}{7}{12}{-1}}
\publisheddate{\DTMdisplaydate{2023}{9}{14}{-1}}
\corrauthor{Ana Luísa Freitas}
\articledoi{10.1590/1983-3652.2023.46002}
%\articleid{NNNN} % if the article ID is not the last 5 numbers of its DOI, provide it using \articleid{} commmand 
% list of available sesscions in the journal: articles, dossier, reports, essays, reviews, interviews, editorial
\articlesessionname{articles}
\runningauthor{Freitas et al.} 
%\editorname{Leonardo Araújo} % old template
\sectioneditorname{Daniervelin Pereira}
\layouteditorname{Thaís Coutinho}

\title{Bases sociocognitivas do discurso de ódio \emph{online} no Brasil: uma revisão narrativa interdisciplinar}
\othertitle{Sociocognitive underpinnings of online hate speech in Brazil: an interdisciplinary narrative review}
% if there is a third language title, add here:
%\othertitle{Artikelvorlage zur Einreichung beim Texto Livre Journal}

\author[1,2]{Ana Luísa Freitas~\orcid{0000-0002-9383-2679}\thanks{Email: \href{mailto:analuisadefreitas@gmail.com}{analuisadefreitas@gmail.com}}}
\author[1,2]{Ruth Lyra Romero~\orcid{0000-0002-5364-6193}\thanks{Email: \href{mailto:ruthlyraespinosa@hotmail.com}{ruthlyraespinosa@hotmail.com}}}
\author[1,2]{Fernanda Naomi Pantaleão~\orcid{0000-0003-1038-4370}\thanks{Email: \href{mailto:fernandanpantaleao@outlook.com}{fernandanpantaleao@outlook.com}}}
\author[1,2]{Paulo Sérgio Boggio~\orcid{0000-0002-6109-0447}\thanks{Email: \href{mailto:psboggio@gmail.com}{psboggio@gmail.com}}}
\affil[1]{Universidade Presbiteriana Mackenzie, Laboratório de Neurociência Cognitiva e Social, São Paulo, SP, Brasil.}
\affil[2]{Instituto Nacional de Ciência e Tecnologia em Neurociência Social e Afetiva, São Paulo, SP, Brasil.}

\addbibresource{article.bib}
% use biber instead of bibtex
% $ biber article

% used to create dummy text for the template file
\definecolor{dark-gray}{gray}{0.35} % color used to display dummy texts
\usepackage{lipsum}
\SetLipsumParListSurrounders{\colorlet{oldcolor}{.}\color{dark-gray}}{\color{oldcolor}}

% used here only to provide the XeLaTeX and BibTeX logos
\usepackage{hologo}

% if you use multirows in a table, include the multirow package
\usepackage{multirow}

% provides sidewaysfigure environment
\usepackage{rotating}

% CUSTOM EPIGRAPH - BEGIN 
%%% https://tex.stackexchange.com/questions/193178/specific-epigraph-style
\usepackage{epigraph}
\renewcommand\textflush{flushright}
\makeatletter
\newlength\epitextskip
\pretocmd{\@epitext}{\em}{}{}
\apptocmd{\@epitext}{\em}{}{}
\patchcmd{\epigraph}{\@epitext{#1}\\}{\@epitext{#1}\\[\epitextskip]}{}{}
\makeatother
\setlength\epigraphrule{0pt}
\setlength\epitextskip{0.5ex}
\setlength\epigraphwidth{.7\textwidth}
% CUSTOM EPIGRAPH - END

% LANGUAGE - BEGIN
% ARABIC
% for languages that use special fonts, you must provide the typeface that will be used
% \setotherlanguage{arabic}
% \newfontfamily\arabicfont[Script=Arabic]{Amiri}
% \newfontfamily\arabicfontsf[Script=Arabic]{Amiri}
% \newfontfamily\arabicfonttt[Script=Arabic]{Amiri}
%
% in the article, to add arabic text use: \textlang{arabic}{ ... }
%
% RUSSIAN
% for russian text we also need to define fonts with support for Cyrillic script
% \usepackage{fontspec}
% \setotherlanguage{russian}
% \newfontfamily\cyrillicfont{Times New Roman}
% \newfontfamily\cyrillicfontsf{Times New Roman}[Script=Cyrillic]
% \newfontfamily\cyrillicfonttt{Times New Roman}[Script=Cyrillic]
%
% in the text use \begin{russian} ... \end{russian}
% LANGUAGE - END

% EMOJIS - BEGIN
% to use emoticons in your manuscript
% https://stackoverflow.com/questions/190145/how-to-insert-emoticons-in-latex/57076064
% using font Symbola, which has full support
% the font may be downloaded at:
% https://dn-works.com/ufas/
% add to preamble:
% \newfontfamily\Symbola{Symbola}
% in the text use:
% {\Symbola }
% EMOJIS - END

% LABEL REFERENCE TO DESCRIPTIVE LIST - BEGIN
% reference itens in a descriptive list using their labels instead of numbers
% insert the code below in the preambule:
%\makeatletter
%\let\orgdescriptionlabel\descriptionlabel
%\renewcommand*{\descriptionlabel}[1]{%
%  \let\orglabel\label
%  \let\label\@gobble
%  \phantomsection
%  \edef\@currentlabel{#1\unskip}%
%  \let\label\orglabel
%  \orgdescriptionlabel{#1}%
%}
%\makeatother
%
% in your document, use as illustraded here:
%\begin{description}
%  \item[first\label{itm1}] this is only an example;
%  % ...  add more items
%\end{description}
% LABEL REFERENCE TO DESCRIPTIVE LIST - END


% add line numbers for submission
%\usepackage{lineno}
%\linenumbers

\begin{document}
\maketitle

\begin{polyabstract}
\begin{abstract}
O crescimento das redes sociais deu força sem precedentes aos discursos de ódio, que têm causado danos globalmente. Este artigo objetivou discutir os substratos sociocognitivos do discurso de ódio e o papel das redes sociais no agravamento do problema, integrando conhecimentos das neurociências, da Psicologia Social, Análise Crítica do Discurso, entre outras, propondo uma breve revisão narrativa para auxiliar a compreensão e o combate ao discurso de ódio no contexto brasileiro. Por meio da articulação dessas áreas, foram abordados temas centrais ao discurso de ódio: sua natureza como prática social e os processos sociocognitivos subjacentes a ele, como a categorização social e formação de estereótipos, preconceitos e identidade social, fenômenos que podem mediar conflitos interpessoais e intergrupais. A partir de conceitos já bastante consolidados, buscou-se literatura atualizada para compreender e ilustrar a dimensão da problemática dos discursos de ódio. Este trabalho aponta direções estratégicas para combater e mitigar efeitos negativos dos discursos de ódio, para promover sociedades mais justas e cooperativas, com adoção de medidas socioeducativas dentro e fora da Internet.

\keywords{Discurso de ódio \sep Redes sociais \sep Liberdade de expressão \sep Conflito entre grupos}
\end{abstract}


\begin{english}
\begin{abstract}
The growth of social networks has given unprecedented strength to hate speech, which has caused damage globally. This article aimed to discuss the sociocognitive substrates of hate speech and the role of social networks in exacerbating the problem, integrating knowledge from neurosciences, Social Psychology, Critical Discourse Analysis, among others, proposing a brief narrative review to aid understanding and combat hate speech in the Brazilian context. Through the articulation of these areas, central themes were addressed to hate speech: its nature as a social practice and the sociocognitive processes underlying it, such as social categorization and the formation of stereotypes, prejudices, and social identity, phenomena that can mediate interpersonal and intergroup conflicts. From already well-established concepts, up-to-date literature was sought to understand and illustrate the dimension of the problem of hate speech. This work points out strategic directions to combat and mitigate the negative effects of hate speech, to promote more just and cooperative societies, with the adoption of socio-educational measures both on and off the Internet.

\keywords{Hate speech \sep Social media \sep Freedom of speech \sep Intergroup conflict}
\end{abstract}
\end{english}
% if there is another abstract, insert it here using the same scheme
\end{polyabstract}

\section{Introdução}\label{sec-intro}
Discursos são ações que constroem e são construídas por três dimensões interativas de sentido: identidade social, relações sociais e sistemas individuais de conhecimento e crença \cite{fairclough2001}. Pelos discursos, indivíduos constroem e reproduzem percepções da sociedade e transformam-na pela interação dessas três dimensões \cite{fairclough2001}. Comumente, discursos estruturam-se pela linguagem, que permite imaginação e criação poéticas, mas que também expõe sujeitos linguísticos à vulnerabilidade, pois na linguagem se reconhece uma agência ferina que converte sujeitos linguísticos em alvo de ataques injuriosos \cite{butler2021discurso}, os discursos de ódio.

Não há unanimidade na definição de discurso de ódio, que transita entre os contextos jurídico, linguístico, científico e prático, este referindo-se a plataformas como as redes sociais \cite{papcunova2023hate}. Aqui, por discurso de ódio entendem-se manifestações segregacionistas “que visam inferiorizar, desacreditar e humilhar uma pessoa ou um grupo social, em função de características como gênero, orientação sexual, filiação religiosa, raça, lugar de origem ou classe” \cite{trindade2022discurso}, um mecanismo de subordinação - verbal, física ou simbólica - para amedrontar, intimidar, assediar e discriminar \cite{lederer1995} grupos tipicamente desprivilegiados, para manter hierarquias sociais \cite{harff2021discurso}.

Momentos de conflito ou crise comumente marcam ações linguageiras com hostilidade, porque relações sociais são naturalmente construídas e negociadas pela linguagem \cite{kopytowska2017stereotypes}. Discursos de ódio, porém, parecem extrapolar os limites dos insultos verbais naturais - e transitórios - em discussões acaloradas e impõem consequências graves aos seus alvos. Linguagem é ação \cite{austin2009things} e, quando o ódio é expresso por ela, torna-se também ação passível de consequências violentas e discriminatórias, pois há, nele, uma violência simbólica baseada em dicotomias de poder (dominante-dominado; superior-inferior) que reforça, naturaliza e perpetua desigualdades históricas e subordinantes \cite{souzade2020discurso}.

O crescimento e a consolidação do ciberespaço como esfera pública ilimitada delinearam novos modelos interativos e construtores de comunidade e identidade grupal \cite{kopytowska2017stereotypes}, fazendo crescerem também consequências negativas dessas interações, como o discurso de ódio, transponível à esfera não digital sob a forma de comportamentos violentos \cite{williams2020hate}, incluindo genocídios \cite{united2019}. Em 2022, por exemplo, desde o início do processo eleitoral brasileiro (agosto), Facebook e Instagram apagaram mais de 1 milhão de publicações que propagavam discurso de ódio \cite{matoso2023}. Ainda no segundo semestre de 2022, denúncias de crimes de ódio reportados na Internet, no Brasil, cresceram quase 70\% \cite{pinheiro2022}, crescimento também verificado em outros países, como nos Estados Unidos da América e no Reino Unido \cite{pretus2023psychology}, impondo desafios relevantes para unificar a definição de discurso de ódio, estabelecer limites com a liberdade de expressão e, enfim, sancioná-lo e regulá-lo \cite{luccas2020}, nacional e internacionalmente.


Compreender esse fenômeno social requer esforço interdisciplinar, porque, particularmente na Internet, esse comportamento proporciona riscos graves à segurança e à saúde mental dos cidadãos em diversas sociedades modernas \cite{solovev2023moralized}, ao ponto de estudos sobre o ódio migrarem para o domínio da saúde pública, em face de sua inerente relação com a violência \cite{krug2002world}. Combater discursos de ódio é urgente para proteger direitos humanos, democracias, inclusão e respeito à diversidade, com especial atenção aos movimentos radicais e extremistas violentos, que têm, entre outros fatores, o ódio em suas raízes \cite{sternberg2005}.

O olhar especial às manifestações de ódio \emph{online} deve-se à facilidade de acesso, que, somada à possibilidade de anonimato, magnitude da audiência e, mormente, à instantaneidade e ao dinamismo da heterogeneidade com que se apresenta o ódio no âmbito virtual, tende a ampliar a ocorrência de comportamentos odientos, em proporções desconhecidas no universo \emph{offline} \cite{brown2017hate}. As redes sociais são, como afirmou António Guterres, “megafone global para o ódio” \cite{united20192020} e, por isso, requerem atenção especial.

Considerando-se, assim, a propagação desses discursos no mundo, substancialmente pelo avanço tecnológico, o presente trabalho objetivou discutir os principais substratos sociocognitivos dessa prática social negativa e o papel das redes sociais. Trata-se, aqui, de uma contribuição teórico-conceitual que integra conhecimentos das neurociências, da Psicologia Social, da Análise Crítica do Discurso, da Sociologia, entre outras, para proporcionar, em língua portuguesa e de acesso aberto, uma breve revisão narrativa da literatura para auxiliar a compreensão atual e o combate ao discurso de ódio no contexto brasileiro.



\section{O ódio e o discurso de ódio como prática social}

O ódio é um fenômeno complexo de extrema antipatia e/ou impulsos agressivos direcionados a um indivíduo ou grupo, que envolve componentes cognitivos, emocionais e comportamentais \cite{sternberg2005}. O ódio se estrutura, segundo \textcite{sternberg2005}, sobre três componentes: (i) negação de intimidade ou distanciamento social, associados a nojo e repulsa; (ii) paixão, associada a medo e raiva; e (iii) compromisso, componente cognitivo associado à desvalorização, diminuição ou desumanização do grupo-alvo. Comumente comparado à raiva, em termos de avaliações e objetivos motivacionais \cite{allport1979nature}, o ódio pressupõe a avaliação negativa do outro como um todo, na crença de esse outro ser mau por essência e, portanto, incapaz de mudar; a raiva, em contrapartida, pressupõe avaliar negativamente ações específicas de outrem, não a pessoa em si \cite{dijk2016schadenfreude}.

Experiências discriminatórias, percepção de ameaça, falta de contato com pessoas diferentes, influências midiáticas ou culturais são causas possíveis do ódio, que pode perpetuar-se pela desumanização do outro, pelos estereótipos negativos e pela manutenção de mitos e rumores, especialmente sustentados pelas narrativas de ódio \cite{sternberg2005}. Essas narrativas tipicamente contam histórias organizadas sobre dois papéis interdependentes: o perpetrador, alvo do ódio; e a vítima, que direciona ódio ao perpetrador \cite{baumeister2001evil}. Para \textcite{baumeister2001evil}, perpetradores dificilmente se percebem como agentes maus e creditam suas ações à percepção distorcida e reação desproporcional de suas vítimas, isso quando não se autodenominam vítimas daqueles que eles mesmos atacam\footnote{Como ilustração, tomam-se aqui as ideias do “racismo reverso” e da “ideologia de gênero”. No caso do “racismo reverso”, acredita-se, segundo \textcite{almeida2019racismo}, que a população negra se volta contra a população branca para lhe impor desvantagens sociais, a fim de demovê-la do “topo da pirâmide social” \cite[p.30]{schwarcz2019}, posição que lhe seria naturalmente assegurada pela superioridade natural da própria raça. \textcite{almeida2019racismo} chama atenção para o fato de o termo utilizar “reverso” em sua construção, assumindo-se existir um racismo natural que encontraria, hoje, um movimento deturpador da sua deslegitimação. Já no caso da “ideologia de gênero”, os personagens envolvidos são, de um lado, as mulheres, mais especificamente os movimentos feministas, e a comunidade LGBTQIA+, e, do outro lado, os movimentos cristãos conservadores da extrema direita \cite{machado2018christian}. Essa segunda ideia, qualificada por \textcite{schwarcz2019} como “depreciativa”, prega a narrativa falaciosa de que feministas e LGBTQIA+ organizam-se para institucionalizar o “homossexualismo”, uma espécie de doutrinação homossexual \cite{machado2018christian}, que ameaçaria a manutenção da “família tradicional”, heteronormativa. Dada a proposta desta revisão, não cabe maior discussão sobre esses dois exemplos. Pontua-se, entretanto, que, academicamente, o que existe são estudos sobre a “perspectiva de gênero”, que investigam as iniquidades entre os direitos dos homens, das mulheres e da população LGBTQIA+, além de propostas de políticas públicas que assegurem às minorias sexuais suas autonomias \cite{machado2018christian}.}.

Narrativas de ódio podem ser propagadas e sustentadas entre gerações familiares, grupos de amigos e outras fontes de informação socialmente disponíveis, como as redes sociais \cite{kaczmarczyk2019online}. Geralmente, estruturam-se na experiência do objeto odiado como um inimigo que é, entre outras qualidades: estranho; impuro, contaminado; inimigo de Deus; amoral; criminoso; ávido por poder; destruidor do destino dos membros do “nosso grupo” \cite{sternberg2005}. A veiculação dessas histórias se dá por quaisquer formas de expressão, incluindo as não verbais e as multimodais \cite{kress2001multimodal}, como os memes \cite{united20192020}, gênero textual de rápida difusão \textit{online}, geralmente para fins humorísticos e satíricos, mas que, por refletir visões de mundo e experiências pessoais, também pode constituir ameaça a minorias, como nos casos de “racismo recreativo”, forma de racismo escondida sob o véu do humor \cite{moreira2019}.

Em maio de 2023, por exemplo, o Tribunal de Justiça de São Paulo determinou ao YouTube a retirada do vídeo do \emph{show} de um comediante brasileiro cuja \emph{performance} atacava variadas minorias, como idosos, mulheres, negros escravizados e pessoas com deficiência \cite{ramos2023}. A determinação judicial acendeu, mais uma vez, o debate sobre os limites entre o exercício do direito à liberdade de expressão, em especial a liberdade do fazer humorístico, e o cerceamento de comportamentos de natureza preconceituosa e discriminatória, que contribuem, muitas vezes, para a naturalização do riso ao se falar de fatos históricos não risíveis, como a escravidão.

No ambiente da comédia, por exemplo, a constância com que pessoas negras são retratadas com características morais inferiores, associadas à criminalidade e a animais não humanos, chama a atenção de \textcite{moreira2019}. Não raro, casos de discriminação são julgados como improcedentes, porque a naturalização das práticas sociais discriminatórias relativizam ou não reconhecem a intenção lesiva dessas ações, muitas vezes interpretadas como interação social amistosa e bem humorada \cite{moreira2019}. O raciocínio aplicado a episódios racistas aplica-se também a quaisquer outras formas de ataques amistosamente velados, independentemente do grupo minoritário ao qual se pretende endereçar.

Em parte, a vasta gama de gêneros textuais sob os quais o ódio se manifesta impõe desafios à sociedade, dificultando a determinação do que se constitui como ódio e a sua detecção em formatos de texto cada vez mais complexos, em especial os amplamente multiplicados nas esferas digitais, como os memes ou os \emph{shows} de \emph{stand up}. Discursos, todavia, independentemente de suas materializações, são processos sociais em que indivíduos compartilham conhecimentos socialmente construídos sobre (parte de) uma realidade observada \cite{kress2001multimodal} e, como práticas sociais, podem implicar-se em orientações econômicas, políticas, culturais e ideológicas deles inextricáveis \cite{fairclough2001}. As orientações políticas constitutivas dos discursos relacionam-se diretamente ao estabelecimento, à manutenção e à transformação das relações de poder e da coletividade, enquanto orientações ideológicas relacionam-se à capacidade de constituir, naturalizar, manter e transformar visões de mundo a partir das relações de poder \cite{fairclough2001}.

Especialmente das relações políticas e ideológicas surgem os discursos de ódio, instrumentos que justificam e legitimam a discriminação e a exclusão sociais, sob a premissa de minorias serem moralmente inferiores, indignas de consideração ou respeito \cite{brown2017hate}. \textcite{cattani2020}, não se referindo ao ódio em si, mas à malignidade, pontua que, sociologicamente, o mal extrapola o nível da intimidade da opinião pessoal (pensamento) e deve, por isso, ser considerado como fato social, uma vez que nele se articulam e verificam as ideologias, materializadas em ações intencionais de intolerância e violência devidamente endereçadas. Esses processos de discriminação e exclusão relacionam-se intimamente à infra-humanização, percepção do outro com menos atributos exclusivamente humanos, quando comparados aos membros do grupo aos quais os infra-humanizadores sentem-se pertencentes \cite{haslam2014dehumanization}\footnote{Essa infra-humanização aplica-se também às tentativas de se justificarem as ações intencionalmente maléficas, quando os indivíduos maldosos são rotulados como monstruosos ou demoníacos e, por isso, não humanos \cite{cattani2020}. Uma reflexão importante posta por \textcite{cattani2020} é a de que as pessoas não são unanimemente percebidas como más. Trata-se de pessoas a que o autor chama de “normais”, tão comuns quanto o vizinho simpático que faz caridade à porta da igreja, mas que critica programas sociais de distribuição de renda e que compartilha, em suas mídias sociais, publicações contra os direitos das minorias.}.

Como observou \textcite[p. 46]{smith2015teoria}, o ódio é “paixão insociável” transformadora do outro em “objeto de horror e desgosto universais”, “animal selvagem”, não humano, a ser extirpado de convívio social. O ódio pode, assim, associar-se à “intolerância selvagem”, intolerância essencialmente acrítica que, caso se torne doutrina, torna-se também incapaz de ser sobrepujada \cite{eco2020migraccao}. Dessa forma, compreender os processos sociocognitivos subjacentes ao desenvolvimento do ódio que as pessoas nutrem umas pelas outras e que culminam, amiúde, em consequências graves para diversos grupos vulneráveis é, pois, essencial à elaboração e execução de planos de ações eficazes para o seu combate.


\section{Substratos do discurso de ódio}

A cultura influencia processos biológicos básicos e diferentes ambientes em que seres humanos atuam moldam visões de mundo e formas de relacionar-se socialmente \cite{marsh2020fear}. Para dar conta da complexidade dos discursos de ódio, é importante entender os processos cognitivos subjacentes a ele, como os envolvidos na construção das identidades individuais e coletivas, auxiliares das experiências sociais, mas também fomentadores de conflitos.


\subsection{ Categorização social}

Categorizações são processos fundamentais para a construção de pensamentos, percepções e práticas sociodiscursivas \cite{lakoff2008women}, processos pelos quais se organizam informações sobre pessoas à sua volta de acordo com características semelhantes, como gênero, idade e raça, facilitando a navegação pelo mundo social \cite{rhodes2019development}. Por essa organização, indivíduos classificam uns aos outros como sendo seus semelhantes, percepção originadora da identificação de um endogrupo - grupo social com que o indivíduo se identifica como membro e com quem compartilha valores e crenças -, e de um exogrupo - grupo social percebido como diferente e distinto do endogrupo \cite{tropp2001ingroup}. Similaridade gera, assim, conexão, e a homofilia proporciona às espécies relacionamentos diversos, como amizades e casamentos \cite{mcpherson2001birds}.

Reconhecer endogrupos requer sentimento de pertencimento e proximidade, mas a fluidez do trânsito social humano torna a identificação grupal menos simples do que parece a uma primeira vista. Diferenciar “nós” e “eles” pode considerar gênero, raça e idade, mas critérios arbitrários, como a mera alocação em equipes com diferentes cores de uniforme, também podem determinar esses reconhecimentos \cite{van2008neural}. Essa fluidez, explicada pelo “paradigma dos grupos mínimos” \cite{tajfel1974social}, demonstra que a mera categorização das pessoas em grupos aleatórios é suficiente para levá-las a discriminarem outros grupos. Esse paradigma auxiliou a entender a arbitrariedade dos preconceitos, demonstrando a diversidade dos aspectos influenciadores da percepção de (não) pertencimento ao grupo, dependentes de contexto e de características que podem ser socialmente valorizadas \cite{cikara2022hate}.

Sentindo-se pertencentes a um grupo, indivíduos tendem a apresentar viés de favoritismo pelo endogrupo, qualquer que seja ele; membros do endogrupo são tipicamente vistos mais positivamente, o que reflete em percepção, emoções, preferências e comportamentos sociais \cite{liberman2017origins}. A positividade dirigida ao endogrupo pode, entretanto, gerar viés negativo para exogrupos \cite{brewer2016intergroup}, percebidos como ameaças. \textcite[p.43]{eco2020migraccao} sinaliza que seres humanos são natural e constantemente “expostos ao trauma da diferença”, o aprendizado diário, desde a infância, de perceber o outro por aquilo em que ele difere do “eu”, em vez de percebê-lo pelas semelhanças. Como consequência, reconhecer o diferente pode levar ao reconhecimento de um inimigo cuja aproximação ameaça, especialmente, as identidades, sejam individuais ou coletivas \cite{eco2021construir, eco2022}. Nesses casos, acredita-se que esse inimigo precisa, então, ser combatido.

Durante a evolução e socialização humanas, diversas estratégias foram adaptadas para promover segurança e sobrevivência, como cooperação com membros do endogrupo e monitoramento de ameaças \cite{schaller2010}. Algumas pessoas, porém, são mais sensíveis para perceber ameaças inexistentes, como em cenários de xenofobia, em que uma população se vê ameaçada pelo estrangeiro, percebido como alguém que deseja deturpar a identidade comunitária de quem o recebe \cite{eco2020migraccao}. Essa sensibilidade torna algumas pessoas mais suscetíveis a estereótipos e preconceitos \cite{neuberg2015}, conforme se discute a seguir.


\subsection{Estereótipo, preconceito e discriminação}

Por envolverem generalizações, categorizações sociais permitem inferências sobre desconhecidos a partir de experiências prévias \cite{rhodes2019development}. Apesar dos benefícios cotidianos, como ergonomia cognitiva, categorizações podem trazer consequências negativas. Generalizações facilitam estereótipos, esquemas cognitivos auxiliares do processamento de informações sobre outrem \cite{hilton1996stereotypes}, que caracterizam grupos por traços ou circunstâncias \cite{amodio2014neuroscience} e que, aplicados aos membros típicos de um grupo, constroem uma imagem do que se acredita inerente e representativo daquele grupo \cite{dovidio2010}.

Estereótipos não são negativos \textit{per se}, mas, ao experienciar situação negativa com uma pessoa ou grupo específico, estender a avaliação negativa aos demais indivíduos daquele grupo pode implicar erros associativos. É o que acontece quando objetos aleatórios em mãos de pessoas negras, população culturalmente mais associada a comportamentos criminosos e violentos, são percebidos como armas \cite{payne2001prejudice}. Associar rapidamente pessoas negras à criminalidade traz consequências sociais graves, como mortes de inocentes, a exemplo dos brasileiros que portavam um guarda-chuva e uma furadeira e foram mortos por policiais, ao terem seus objetos confundidos com armas de fogo \cite{merola2010, moura2018}. 

A mera estereotipação já pode influenciar comportamentos dos indivíduos e das pessoas que a eles se dirigem, de modo que a autopercepção de alguém pode afetar-se pelos estereótipos que lhe são atribuídos \cite{rhodes2019development}. Por exemplo, acreditar que homens são intelectualmente mais habilidosos torna meninas menos propensas a acreditar na inteligência feminina e, por isso, engajar menos em atividades consideradas de “crianças inteligentes” \cite{bian2017gender}; na vida adulta, isso afeta o interesse, o envolvimento e a sensação de pertencimento de mulheres nos campos de Ciência e Tecnologia \cite{cheryan2009ambient, nosek2009national}. 

As tratativas dadas a diferentes grupos com base em seus estereótipos diferem dependendo das características a eles atribuídas, porque diferentes percepções de ameaça eliciam diferentes emoções e comportamentos \cite{neuberg2015}. Ameaças de agressão física, por exemplo, podem eliciar medo e respostas de luta ou fuga, enquanto cheiros ruins podem eliciar nojo e afastamento \cite{neuberg2015}. Essas respostas, comumente associadas aos sem-teto, incentivam o apoio a políticas contraproducentes de combate à miséria, que afastam da vista os marginalizados, mas não resolvem o problema \cite{clifford2017}\footnote{Em 2021, por exemplo, uma ação da Prefeitura de São Paulo (SP) instalou paralelepípedos em áreas que ficam sob viadutos paulistanos. A medida foi interpretada, à época, como higienista, pois tirava da vista dos transeuntes os moradores de rua que se instalavam embaixo daqueles viadutos, forçando-os a migrarem para outras áreas, sem que lhes fossem oferecidas condições dignas de moradia \cite{g1sp}.}.

Intimamente ligado a estereótipos, o preconceito é um fenômeno que engloba avaliações e atitudes baseadas em crenças sobre um grupo-alvo e que estão, simultaneamente, associadas a afetos negativos, como antipatia, e a uma predisposição comportamental negativa (discriminação) em relação ao alvo \cite{dovidio2010}. Funciona, no nível individual, para alimentar a autoestima do preconceituoso, que se sente superior \cite{fein1997prejudice}, e, no coletivo, para perpetuar hierarquias sociais sustentadas pela estratificação socioeconômica \cite{bobo1999prejudice}. Nos dois casos, o preconceito afeta relações intergrupais, podendo interferir, por exemplo, na quantidade e na qualidade da oferta de emprego a pessoas negras, no nível educacional, na representação política e na taxa de mortalidade dessa população \cite{ibge2019}.

Distinguem-se, aqui, ainda, o endosso dos estereótipos e o preconceito explícito: no primeiro, o preconceito restringe-se às representações cognitivas das crenças culturais sobre um grupo; no segundo, apresenta-se afeto negativo pelo exogrupo \cite{liberman2017origins}. Quando o preconceito extrapola o âmbito privado do afeto e se torna ação, ocorre a discriminação: tratamento injusto direcionado a indivíduos devido à sua identidade grupal \cite{dovidio2010}\footnote{Em agosto de 2023, em sessão para a descriminalização do porte de maconha para uso pessoal, o Ministro Alexandre de Moraes, do Supremo Tribunal Federal (STF), citou, em seu voto, a discrepância do tratamento policial e jurídico dedicado às populações brancas, pardas e pretas no Brasil. Se for branco, para que um portador de maconha seja considerado traficante, é necessário que ele esteja portando 80\% a mais de maconha que um preto ou um pardo, por exemplo \cite{pauxis2023}. Essa disparidade é fruto de uma retroalimentação: o estereótipo e o preconceito facilitam a abordagem policial e a condenação de populações pretas e pardas, fazendo aumentar seu encarceramento e, por conseguinte, os dados demográficos carcerários são fornecidos como prova para reforçar o perfil socialmente desviante de pretos e pardos.}. Comportamentos discriminatórios podem aumentar por gatilhos psicológicos e eventos traumáticos, quando as pessoas estão mais vulneráveis a reagir de maneira não ponderada, mas também por incentivos discursivos que incitam violência \cite{takano2023dynamics}. Se estereótipos e preconceitos facilitam essas incitações, compreender suas consequências na sociedade implica, também, compreender processos de construção de identidades.

\subsection{Identidade social}

A definição que indivíduos têm de si mesmos constitui-se também pelas comparações e identificações com valores e características da sociedade em que estão inseridos, porque “nossas identidades são formadas pelos valores que nos são emprestados pelos grupos que chamamos de nossos” \cite[p.ix]{lieberman2013social}. Algumas teorias buscam compreender como se forma a autoimagem dentro das sociedades e sua relação com a formação de grupos, como as teorias da identidade social \cite{tajfel1971social} e da autocategorização \cite{turner1987}. Para \textcite{tajfel1971social}, categorização social é o sistema de orientação definidor do lugar do indivíduo em uma sociedade, percepção acompanhada de significado emocional e valores compartilhados entre os membros do mesmo grupo. Identidade social proporciona pertencimento e reconhecimento da identidade individual. \textcite{turner1987} reforçou essa perspectiva com a teoria da autocategorização, pela qual o autoconceito do indivíduo dinamicamente se constitui pelas relações entre sua identidade humana (reconhecer-se humano), social (reconhecer-se membro de um grupo social) e pessoal (reconhecer-se igual a alguns pares, em contraposição a outros - autocategorização).

Não se pode, pois, ignorar o peso da influência social sobre as ações dos indivíduos, uma vez que a influência é um processo pelo qual comportamentos, atitudes ou crenças individuais alteram-se pelo grupo social com que se interage \cite{fiske2014social}, e pelo qual são validadas as percepções individuais e formadas as visões de mundo \cite{spears2021social}. Diferentemente da persuasão, em que o agente influenciador intenciona modificar atitudes alheias, sem muito contato social, na influência social a intenção não é pré-requisito, mas as interações sociais, sim \cite{fiske2014social}. Interagir com o outro modula a construção do indivíduo e, em situações de incerteza quanto ao que acreditar, por exemplo, o peso do grupo é potencializado, pois busca-se, no grupo, a validação de percepções pessoais e a formação de visões de mundo, inspiradas por membros dos espaços aos quais se pertence ou se deseja pertencer \cite{spears2021social}.

Aspectos contextuais também podem afetar o impacto do grupo sobre comportamentos individuais. Considerando-se o papel de controle do grupo sobre indivíduos, poder-se-ia esperar que relações virtuais, caracterizadas por distanciamento físico e possibilidade de anonimato, diminuíssem o impacto do poder do grupo sobre as pessoas, já que responsabilidade e confiança estão diretamente ligadas ao conhecimento dos nomes, isto é, das identificações \cite{han2018no}. No contexto \emph{online}, todavia, ampliam-se as conexões sociais (virtuais), reúnem-se grupos dificilmente formados presencialmente e amplia-se também a influência social e o seu alcance \cite{spears2021social}. A despersonalização proporcionada pelo anonimato\footnote{Para o filósofo sul-coreano Byung-Chul Han, o que sustenta o respeito é o conhecimento do nome do outro. O anonimato exclui, assim, qualquer possibilidade de respeito e, se este emerge do reconhecimento de valores pessoais e morais, é esperado que inexista em contextos cuja saliência é para a falta de virtude alheia \cite{han2018no}.} da multidão \textit{online} fortalece, assim, identidades grupais já relevantes, ajudando a explicar a polarização comumente observada nas redes sociais \cite{spears2021social}.

Se grupos, então, influenciam identidades individuais, identidades grupais também influenciam a dinâmica das relações intra e intergrupos. Compreender a formação dessas identidades, por conseguinte, auxilia a explicar por que grupos com forte identidade social tendem, internamente, a ser mais coesos e cooperativos, embora mais propensos a excluírem outros grupos e gerar conflitos entre eles \cite{dovidio2020}.

\subsection{Conflitos entre grupos}

Conflitos podem relacionar-se ao fato de a sensação de pertencimento frequentemente basear-se no reconhecimento de exogrupos como ameaças ou inimigos comuns\footnote{Há, na literatura, a defesa de que a existência de um inimigo proporciona a construção de identidades: sem um inimigo, seja ele real ou imaginário, diminuem-se as chances de se provar o próprio valor, individual ou coletivo, e, consequentemente, diminuem-se as chances de se criar um herói virtuoso \cite{eco2021construir, eco2022}. Havendo um inimigo, há também a possibilidade de combatê-lo, derrotá-lo e criar-se um herói.} \cite{tajfel1974social}, correlacionando a identificação positiva com o endogrupo à hostilidade intergrupal \cite{dovidio2020}. Divergências ideológicas, por exemplo, podem levar à percepção da inferioridade moral do outro e, como resultado, conflitos ideológicos podem levar a hostilidade e agressividade, como nos casos em que a linguagem serve de instrumento de ataque marcador da polarização política \cite{bilewicz2020hate}.

Dois ou mais grupos podem conflitar ao se verem competindo por recursos limitados \cite{sherif1988robbers}, mas conflitos também surgem por processos cognitivos e afetivos, como estereótipos e preconceitos. Conforme observaram \textcite{cikara2022hate}, nem sempre estereótipos explicam a complexidade das disputas intergrupos, que devem considerar a sensibilidade para detectar ameaças. Preferir o endogrupo pode predizer, todavia, discriminação de exogrupos \cite{hewstone2002intergroup} e, detectada ameaça ou competição, é possível surgir antipatia pelos grupos externos \cite{brewer2016intergroup}. Uma vez comprometida a empatia, percebe-se maior tendência a \textit{Schadenfreude}, o prazer no sofrimento alheio que pode motivar conflitos \cite{cikara2014their, cikara2011us}.

Assim como outros aspectos da vida humana, os conflitos intergrupais se estenderam à esfera \textit{online}, manifestados, especialmente, em discursos de ódio nas redes sociais \cite{kaczmarczyk2019online}. De início, o ódio nas redes era visto como uma ameaça fictícia, como observou \textcite{cattani2020}, e, por isso, parecia intransponível ao mundo não virtual, fazendo os contemporâneos do Século XXI (fingirem-se) surpresos com tanto ódio se aproximando mais e mais daqueles que, antes, sentiam-se dele distantes ou mesmo imunes \cite{glucksmann2007}.

A constância e a adesão de grandes e influentes grupos e pessoas a essas manifestações discursivas nos ambientes digitais, todavia, contribuíram para a “materialização do horror”, para usar os termos de \textcite[p.10]{cattani2020}, uma vez que os ambientes virtuais se tornaram um catalisador de comportamentos das massas. Por essas razões, compreender o papel do ódio no engajamento \emph{online} é, junto à compreensão de suas bases sociocognitivas, fundamental também à compreensão dos efeitos nefastos por ele causados às mais diversas esferas sociais, públicas ou privadas, como se discute a seguir.

\section{Engajamento \emph{online}}

A necessidade de pertencimento faz indivíduos utilizarem espaços \textit{online} para melhorarem a autoestima \cite{santoso2018} e construírem redes de apoio e senso comunitário \cite{garg2023handling}. Em contrapartida, essas comunidades podem erguer-se do “\textit{ethos} de violência” \cite{silvada2018ethos}, motivação pela qual algumas pessoas reúnem-se em grupos coesos para atacar aqueles por quem nutrem afetos negativos. Sobre esse \textit{ethos}, embora seu conceito emerja da Retórica aristotélica para designar a imagem que o locutor imprime de si no discurso que constrói, a fim de influenciar percepções e comportamentos de seus alocutários \cite{amossycharaudeua2008}, \textcite[p. 71]{silvada2018ethos} destaca que, na Internet, para além da mera construção de uma imagem, muitos indivíduos buscam “a adesão de outros sujeitos ao seu discurso” e, por conseguinte, às suas ações violentas.

Em alguns contextos, como o da dramaturgia, violência e crueldade servem como meios de provocar o espectador ao questionamento e à recusa de certas ordens sociais que, na visão dos dramaturgos, mantêm as sociedades estagnadas por valores arbitrariamente impostos \cite{glucksmann2007}. Essa reflexão pode ser estendida à discussão de \textcite{silvada2018ethos}, que põe a violência como ferramenta de sedução por meio da qual se cria a imagem de um enunciador corajoso por violar normas sociais, impulsionando a adesão de outros enunciadores igualmente simpáticos a comportamentos de intimidação.

Em grande parte, o agrupamento dos “ciberintimidadores”, para usar o termo de \textcite[p.71]{silvada2018ethos}, encoraja-se pelo anonimato possível na Internet, facilitador de ações tipicamente desencorajadas fora dela \cite{suler2004online} e que podem ser reforçadas pelo \emph{feedback} do grupo ao qual se deseja pertencer \cite{brady2021social}. Vários estudos hoje debruçam-se sobre a compreensão de fenômenos psicossociais motivadores de engajamento e difusão de conteúdos online, que contribuem para a polarização política \cite{amira2021group}, para a disseminação de desinformação \cite{pennycook2021shifting} e para o incentivo à violência \emph{offline} \cite{mooijman2018moralization}.

\textcite{tosi2020grandstanding} atentaram para as redes sociais serem um “palanque” para a moralidade, onde indivíduos comportam-se ostensivamente buscando reconhecimento e aprovação, frequentemente sem objetivo legítimo ou sincero. Mais do que expor ideias para discussões sociais relevantes, o foco é expor opiniões desejáveis para membros de um determinado grupo ao qual se pertence ou se deseja pertencer. A imagem que se quer construir no discurso baseia-se, como observou \textcite{amossycharaudeua2008}, nos estereótipos nos quais se deseja encaixar e que são eficazes no ambiente em que o enunciador se apresenta, mesmo que essa imagem discursiva não seja, necessariamente, representativa do sujeito fora desse contexto em que ele se apresenta.

Alinhando-se ao que propôs \textcite{silvada2018ethos} sobre a violência, aspectos morais e emocionais têm sido, assim, apontados como preditores de engajamento \emph{online} \cite{brady2017emotion}, porque os traços morais, mais do que as próprias experiências pessoais (memória autobiográfica), são apontados, na literatura, como a parte mais essencial da identidade de uma pessoa, aquilo que a configura como um “eu” \cite{strohminger2014essential}. O ódio, como uma emoção moral \cite{smith2015teoria}, cumpre papel fundamental no engajamento das pessoas em publicações \textit{online} \cite{solovev2023moralized}, carecendo, por isso, de atenção especial, uma vez que as redes sociais mobilizam comportamentos agressivos tanto dentro quanto fora do espaço virtual, sob o risco de transformar tais comportamentos em um padrão socialmente aceito.

Sabe-se que, quando amplamente difundida, uma prática social pode se tornar a norma dominante \cite{bilewicz2020hate} e, se normas sociais são fortes preditores de comportamentos preconceituosos \cite{crandall2002social}, podem normalizar discursos de ódio ao ponto de tornarem-nos regra. No ambiente virtual, tais conteúdos contam com apoio dos algoritmos, que criam “bolhas” limitadoras do contato apenas entre pessoas com ideias semelhantes \cite{pariser2012}, e dos \emph{bots}, que reforçam crenças discriminatórias \cite{uyheng2022bots} e polarização política - que, por sua vez, utiliza discursos de ódio como tática populista persuasiva \cite{mercuri2020hate}. No Brasil, por exemplo, a polarização culminou em mortes politicamente motivadas\footnote{Em julho de 2022, em Foz do Iguaçu, no Paraná, um policial penal federal apoiador de Jair Bolsonaro atirou contra um militante petista que comemorava o próprio aniversário no salão de festas do condomínio em que morava \cite{azevedo2023}; em setembro de 2022, no município de Rio do Sul, em Santa Catarina, um petista assassinou a facadas um apoiador de Bolsonaro em uma briga em um bar, motivada por desafetos políticos e familiares, segundo matéria divulgada no portal de notícias do UOL \cite{estadao2022}; em maio de 2023, em São João do Rio do Peixe, na Paraíba, um bolsonarista matou um primo petista a tiros após discussão sobre política \cite{marzullo2023}.} \cite{pinho2022} e, mais recentemente, a mobilização de extremistas nas redes sociais levou à depredação, física e simbólica, dos Três Poderes \cite{guerra2023}.

A preocupação hoje latente no que diz respeito ao funcionamento dos algoritmos das redes sociais e à noção de agência a eles atribuída não é tão nova quanto se pode acreditar, em 2023, quando o mundo experiencia um crescimento maciço de inteligências artificiais. Sistemas inteligentes são modelados para, também, influenciar outros agentes, como observou Conte (2000), ao discorrer sobre a natureza não apenas \emph{vetorial}, mas \emph{agentiva} dos memes\footnote{Aqui, a noção de “meme” à qual se refere \textcite{conte2000} não é a dos memes humorísticos da Internet, mas ideias ou comportamentos passíveis de transmissão entre gerações, comumente por imitação (mimese). A discussão aqui proposta não foge, porém, à discussão anterior sobre o gênero textual “meme”, porque esse gênero, imbuído de ideologias e crenças, existe no espaço da cultura digital, que proporciona interações dentro e fora das redes e é ocupado por agentes humanos que, recursivamente, influenciam e são influenciados por essa cultura \cite{wiggins2019discursive}} na transmissão cultural. 

Os algoritmos, ao funcionarem como reforçadores das “bolhas” de opiniões e informações, facilitam a emergência de vários discursos, como o \emph{cyberbullying} \cite{wiggins2019discursive} e outras formas de agressão, especialmente porque, nas redes sociais, publicações sobre um exogrupo e publicações com linguagem marcada por hostilidade têm chances significativamente maiores de alcançar maior engajamento \cite{rathje2021out}.

Engajar com discursos de ódio \textit{online} pode, assim, ampliar a segregação e a animosidade entre diferentes grupos sociais, além de restringir a liberdade de expressão de quem deseja ocupar esses espaços. Com os ambientes virtuais se tornando cada vez mais intimidadores, inibe-se a ampla participação de indivíduos com visões de mundo diversificadas e facilita-se a proliferação de desinformação, com limites significativos à pluralidade dos debates públicos. Dessas relações, surge, então, o questionamento sobre os limites entre a livre expressão do pensamento, garantia constitucional, e a violação da dignidade do outro.


\section{Discurso de ódio ou liberdade de expressão?}

Liberdade de expressão é direito fundamental consagrado pela Constituição Federal e, por tratar de princípio central à democracia e à dignidade humana, recebe atenção jurídica especial \cite{luccas2020}. Essa liberdade, todavia, não é ilimitada e respeitá-la não significa que todas as ideias são justificáveis, corretas ou igualmente boas \cite{moshman2020}\footnote{\textcite{eco2022} ressalta que certas ideias, hábitos e comportamentos são intoleráveis e assim devem permanecer. É o que \textcite{popper1974} chamou de “paradoxo da intolerância”, quando tolerar ilimitadamente conduz ao desaparecimento da própria tolerância.}. Discussões sobre os limites entre punir discursos de ódio ou cercear a liberdade de expressão decorrem, assim, da subjetividade desses discursos \cite{garg2023handling}, desprovidos de critérios bem demarcados para distinguir entre linguagem depreciativa e outras formas não depreciativas de crítica \cite{bilewicz2020hate}.

Impasses como esses suscitam discussões fundamentais à garantia da democracia também no âmbito dos Três Poderes, quando os limites da imunidade parlamentar são questionados \cite{costa2020discurso} em face de comportamentos discriminatórios direcionados a minorias advindos de autoridades políticas, como os ataques à comunidade LGBTQIA+ \cite{dalmolin2015, pimentel2023, pinho2022}. Parte significativa da defesa dos limites à imunidade parlamentar apoia-se sobre o fato de que, quando pessoas públicas engajam em discursos de ódio, elas avalizam e naturalizam essa prática como norma social aceitável \cite{crandall2002social}, estimulando mais hostilidade intergrupos.

Quando a discriminação vai contra normas sociais, para se sustentar, ela precisa buscar validação em justificativas plausíveis dentro do contexto social \cite{modesto2018racismo}. No contexto brasileiro, em que algumas discriminações não são legalmente permitidas, surgem as justificativas forçosamente construídas em discursos parlamentares. A autorização, expressa ou não, de autoridades públicas para a população geral discriminar os próprios pares torna impessoais essas práticas de agressão. Se a sociedade se comporta como uma mera cumpridora de ordens, exime-se da responsabilidade sobre os próprios atos, embora seja, sim, por eles responsável \cite{cattani2020}.

As implicações das diferenças de valores orientadores das políticas públicas no Brasil também são percebidas no âmbito global, quando diferentes culturas e sistemas de governo dificultam soluções universais de combate e sanção a discursos discriminatórios \cite{pereira2020regulaccao}. Muito se cobra, então, das autoridades e plataformas das redes sociais, mas é importante lembrar que a responsabilidade do combate às práticas segregacionistas recai também sobre a sociedade civil \cite{kaczmarczyk2019online}.


\section{Prevenção e combate ao discurso de ódio}

Em 2023, o Governo Federal criou um grupo interdisciplinar vinculado ao Ministério dos Direitos Humanos e da Cidadania, reunindo estudiosos e integrantes de outros ministérios, para analisarem o problema dos discursos de ódio e, conjuntamente, traçar diretrizes eficazes ao seu combate \cite{rodrigues2023}. Grande parte dos estudos sobre o tema concentra-se na área jurídica \cite{brown2017hate}, que busca compreensão e diretrizes para possível limitação, detecção e sanção de tais conteúdos. Outra área é a computação, que desenvolve modelos automatizados para lidar com as publicações \emph{online} e que, no Brasil, particularmente, carece de dados lexicais anotados para aperfeiçoar a automação da classificação desses conteúdos \cite{santos2022}, especialmente quanto à granularidade emocional \cite{cortiz2021weakly}, esta fundamental à diferenciação entre o ódio e outras emoções.

Para efeitos positivos duradouros, prevenir práticas sociais indesejáveis deve, entretanto, começar pela Educação, e a Organização das Nações Unidas aponta para a potencialidade de promover igualdade, tolerância e compreensão entre grupos a partir de iniciativas educacionais de diversidade e inclusão \cite{united20192020b}. Ensinar adolescentes sobre discurso de ódio e cooperação pode promover empatia pelas vítimas e aumentar a autoeficácia para responder proativamente contra tais discursos \cite{wachs2023effects}; adicionalmente, comunicar aos pares opiniões pró-diversidade também contribui para promover atitudes positivas \cite{murrar2020exposure}.

Para enfrentar casos de discurso de ódio online já existentes, pode-se cobrar das redes sociais mecanismos de denúncia transparentes e acessíveis que garantam a rápida remoção de conteúdos inconformes \cite{getahun2023countering}, além de estimular as testemunhas dos discursos de ódio \emph{online} para rebaterem os violadores, demonstrando-lhes reprovação \cite{obermaier2023too}. Essas abordagens demonstram ser possível combater o discurso de ódio, dentro e fora das redes, construindo-se normas sociais saudáveis.



\section{Conclusão}

Apesar dos múltiplos impasses para caracterizar os discursos de ódio e instaurar medidas práticas eficientes para combatê-lo, é notório que minimizá-lo como mera expressão de ideias impopulares ou consequência exclusiva da percepção distorcida das vítimas é ignorar o caráter performativo das ações linguageiras. Conforme postulou \textcite{austin2009things}, palavras têm poder de ação e servem à consecução de diversos objetivos, que, no caso dos discursos de ódio, parecem alinhar-se não apenas à necessidade de validação individual, mas à ativação de estereótipos negativos e negação da identidade humana do outro. Se “hábitos linguísticos são sintomas importantes de sentimentos não expressos” \cite[p.23]{eco2022}, indicadores textuais de ódio não podem ser descartados como manifestações linguageiras irrelevantes, porque, como se tem demonstrado, a violência verbal pode trazer consequências psicológicas e físicas àqueles que a experienciam.

Alinhada a uma legislação contundente para regular e sancionar os discursos de ódio, em especial sua manifestação \emph{online}, a educação é meio fundamental à consecução de efeitos positivos. Salienta-se, entretanto, que a Educação não é uma panaceia e, por isso, não pode, sozinha, resolver o problema dos discursos de ódio. Embora haja críticas sobre as práticas sociais de ódio comumente se justificarem por privações socioeconômicas, culturais, educacionais ou psíquicas \cite{glucksmann2007}, como uma emoção, o ódio faz parte do repertório humano e pode, por isso, passar por um aprendizado regulatório. Como emoção, o ódio não deixará de existir, inclusive porque emoções, sejam elas positivas ou negativas, cumprem um papel evolutivo fundamental, desde a evitação de contextos que põem em risco a vida da espécie até a construção de grandes grupos sociais \cite{sebastian2018}.

Se o ódio apenas aguarda um fator desencadeador para fazê-lo emergir, ele não depende estritamente do indivíduo detentor do ódio, mas também do contexto, que pode ser interno (psicológico) ou externo (ambiental) ao próprio sujeito. É possível, por isso, que os indivíduos atuem ativamente para a construção de ambientes mais plurais, de modo que se priorize o aprendizado da tolerância \cite{eco2020migraccao}. Como emerge em um sujeito para outro(s) sujeito(s), acredita-se, aqui, ser possível lançar mão de estratégias educacionais para dirimir os seus efeitos nefastos, porque educar é, primeiramente, intervir sobre indivíduos, que, por sua vez, intervêm no mundo, como propunha o educador Paulo Freire.

Estudos em neurociência e em psicologia positiva\footnote{Sugere-se, aqui, a leitura de trabalhos relacionados à teoria \emph{Broaden-and-Build}, de Barbara L. Fredrickson (ver \textcite{fredrickson1998good, fredrickson2013positive, fredrickson2003good}.}, ademais, têm trazido contribuições significativas à compreensão dos processos emocionais e de regulação emocional, demonstrando, inclusive, que a indução de emoções positivas, a exemplo da emoção moral da gratidão, melhora os efeitos da regulação emocional, fazendo com que estímulos de valência negativa sejam menos negativamente avaliados após tarefa de indução dessa emoção positiva \cite{boggio2020writing}. Achados como esse demonstram o poder de estratégias que podem ser implementadas desde a mais tenra infância, de modo que os agentes sociais se construam mais positivos, tolerantes e, eventualmente, mais resilientes.

Com isso em vista, este trabalho objetivou explorar os principais substratos para compreensão do fenômeno dos discursos de ódio, articulando distintas áreas do conhecimento para complementar a compreensão dos aspectos sociais e do papel das redes sociais virtuais no agravamento do problema no contexto brasileiro. Apesar da magnitude dos desafios para mitigar efeitos negativos dessa prática social, é possível aliar aos processos de legislação e regulação das redes sociais investimentos em treinamento para a população identificar e combater o discurso de ódio em suas raízes. Acredita-se que, especialmente pelas práticas socioeducativas, é possível reduzir preconceito, discriminação e conflitos, porque, ao ampliar o repertório de exemplos sociais positivos, com o estímulo constante ao desenvolvimento da empatia, contribui-se para a internalização de normas de respeito à diversidade e inclusão, consolidadoras de sociedades mais justas e saudáveis.

\section{Agências financiadoras}


Os autores deste trabalho contaram com financiamento da Coordenação de Aperfeiçoamento de Pessoal de Nível Superior - CAPES (ALF: 88887.374936/2019-00; PSB: 88887.310255/2018–00; RLR: 88887.833137/2023-00), da Fundação de Amparo à Pesquisa do Estado de São Paulo - FAPESP (FNP: 2022/05313-6) e do Conselho Nacional de Desenvolvimento Científico e Tecnológico - CNPq (ALF: 371611/2023-7; PSB: 309905/2019-2; 406463/2022-0).



\printbibliography\label{sec-bib}
% if the text is not in Portuguese, it might be necessary to use the code below instead to print the correct ABNT abbreviations [s.n.], [s.l.]
%\begin{portuguese}
%\printbibliography[title={Bibliography}]
%\end{portuguese}


%full list: conceptualization,datacuration,formalanalysis,funding,investigation,methodology,projadm,resources,software,supervision,validation,visualization,writing,review
\begin{contributors}[sec-contributors]
\authorcontribution{Ana Luísa Freitas}[conceptualization,writing,review]
\authorcontribution{Ruth Lyra Romero}[conceptualization,writing]
\authorcontribution{Fernanda Naomi Pantaleão}[conceptualization,writing]
\authorcontribution{Paulo Sérgio Boggio}[conceptualization,writing,supervision,projadm]
\end{contributors}




\end{document}

