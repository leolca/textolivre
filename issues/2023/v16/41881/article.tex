% !TEX TS-program = XeLaTeX
% use the following command:
% all document files must be coded in UTF-8
\documentclass[portuguese]{textolivre}
% build HTML with: make4ht -e build.lua -c textolivre.cfg -x -u article "fn-in,svg,pic-align"

\journalname{Texto Livre}
\thevolume{16}
%\thenumber{1} % old template
\theyear{2023}
\receiveddate{\DTMdisplaydate{2022}{11}{22}{-1}} % YYYY MM DD
\accepteddate{\DTMdisplaydate{2023}{2}{13}{-1}}
\publisheddate{\DTMdisplaydate{2023}{4}{2}{-1}}
\corrauthor{Luísa Torre}
\articledoi{10.1590/1983-3652.2023.41881}
%\articleid{NNNN} % if the article ID is not the last 5 numbers of its DOI, provide it using \articleid{} commmand 
% list of available sesscions in the journal: articles, dossier, reports, essays, reviews, interviews, editorial
\articlesessionname{articles}
\runningauthor{Torre e Jerónimo} 
%\editorname{Leonardo Araújo} % old template
\sectioneditorname{Daniervelin Pereira}
\layouteditorname{Thaís Coutinho}

\title{Esfera pública e desinformação em contexto local}
\othertitle{Public sphere and disinformation in a local context}
% if there is a third language title, add here:
%\othertitle{Artikelvorlage zur Einreichung beim Texto Livre Journal}

\author[1]{Luísa Torre~\orcid{0000-0002-5948-106X}\thanks{Email: \href{mailto: luisa.torre@ubi.pt}{ luisa.torre@ubi.pt}}}
\author[1]{Pedro Jerónimo~\orcid{0000-0003-1900-5031}\thanks{Email: \href{mailto:pj@ubi.pt}{pj@ubi.pt}}}
\affil[1]{Universidade da Beira Interior, Faculdade de Artes e Letras, Departamento de Comunicação, Filosofia e Política, LabCom - Comunicação e Artes, Covilhã, Portugal.}

\addbibresource{article.bib}
% use biber instead of bibtex
% $ biber article

% used to create dummy text for the template file
\definecolor{dark-gray}{gray}{0.35} % color used to display dummy texts
\usepackage{lipsum}
\SetLipsumParListSurrounders{\colorlet{oldcolor}{.}\color{dark-gray}}{\color{oldcolor}}

% used here only to provide the XeLaTeX and BibTeX logos
\usepackage{hologo}

% if you use multirows in a table, include the multirow package
\usepackage{multirow}

% provides sidewaysfigure environment
\usepackage{rotating}

% CUSTOM EPIGRAPH - BEGIN 
%%% https://tex.stackexchange.com/questions/193178/specific-epigraph-style
\usepackage{epigraph}
\renewcommand\textflush{flushright}
\makeatletter
\newlength\epitextskip
\pretocmd{\@epitext}{\em}{}{}
\apptocmd{\@epitext}{\em}{}{}
\patchcmd{\epigraph}{\@epitext{#1}\\}{\@epitext{#1}\\[\epitextskip]}{}{}
\makeatother
\setlength\epigraphrule{0pt}
\setlength\epitextskip{0.5ex}
\setlength\epigraphwidth{.7\textwidth}
% CUSTOM EPIGRAPH - END

% LANGUAGE - BEGIN
% ARABIC
% for languages that use special fonts, you must provide the typeface that will be used
% \setotherlanguage{arabic}
% \newfontfamily\arabicfont[Script=Arabic]{Amiri}
% \newfontfamily\arabicfontsf[Script=Arabic]{Amiri}
% \newfontfamily\arabicfonttt[Script=Arabic]{Amiri}
%
% in the article, to add arabic text use: \textlang{arabic}{ ... }
%
% RUSSIAN
% for russian text we also need to define fonts with support for Cyrillic script
% \usepackage{fontspec}
% \setotherlanguage{russian}
% \newfontfamily\cyrillicfont{Times New Roman}
% \newfontfamily\cyrillicfontsf{Times New Roman}[Script=Cyrillic]
% \newfontfamily\cyrillicfonttt{Times New Roman}[Script=Cyrillic]
%
% in the text use \begin{russian} ... \end{russian}
% LANGUAGE - END

% EMOJIS - BEGIN
% to use emoticons in your manuscript
% https://stackoverflow.com/questions/190145/how-to-insert-emoticons-in-latex/57076064
% using font Symbola, which has full support
% the font may be downloaded at:
% https://dn-works.com/ufas/
% add to preamble:
% \newfontfamily\Symbola{Symbola}
% in the text use:
% {\Symbola }
% EMOJIS - END

% LABEL REFERENCE TO DESCRIPTIVE LIST - BEGIN
% reference itens in a descriptive list using their labels instead of numbers
% insert the code below in the preambule:
%\makeatletter
%\let\orgdescriptionlabel\descriptionlabel
%\renewcommand*{\descriptionlabel}[1]{%
%  \let\orglabel\label
%  \let\label\@gobble
%  \phantomsection
%  \edef\@currentlabel{#1\unskip}%
%  \let\label\orglabel
%  \orgdescriptionlabel{#1}%
%}
%\makeatother
%
% in your document, use as illustraded here:
%\begin{description}
%  \item[first\label{itm1}] this is only an example;
%  % ...  add more items
%\end{description}
% LABEL REFERENCE TO DESCRIPTIVE LIST - END


% add line numbers for submission
%\usepackage{lineno}
%\linenumbers

\begin{document}
\maketitle

\begin{polyabstract}
\begin{abstract}
A desinformação não é um fenômeno novo. Ainda assim, nos últimos anos, a vitória dos apoiantes do Brexit (Britan exit – Saída do Reino Unido da União Europeia) no Reino Unido ou a eleição de Donald Trump nos Estados Unidos mostraram sua relevância na agenda pública. O interesse acadêmico corre paralelamente à consideração da desinformação como prioridade crescente para governos e organizações internacionais, pela sua relevância geoestratégica e para a segurança nacional. Por outro lado, o aparecimento da pandemia de Covid-19 veio acelerar o declínio dos meios regionais, já afetados pela transformação digital e o modelo de negócios, agora desafiado pelas plataformas, que se tornaram mediadores essenciais no mercado publicitário. O declínio dos meios regionais deixa as comunidades em estado de grave vulnerabilidade, já que a informação é cada vez mais consumida através das redes sociais e nelas a desinformação facilmente prolifera. Tal como no contexto pandêmico, também a desinformação é um vírus que se propaga rapidamente e com elevado potencial de dano à democracia, designadamente em nível mais local. É precisamente aí que pretendemos centrar o debate, particularmente onde tem estado pouco presente. Entendemos que é a partir da esfera pública local que podem emergir respostas à problemática da desinformação, especificamente numa relação colaborativa entre jornalistas e (outros) membros ativos da comunidade.

\keywords{Esfera pública \sep Jornalismo de proximidade \sep Imprensa regional \sep Desinformação \sep Checagem de fatos}
\end{abstract}

\begin{english}
\begin{abstract}
Disinformation is not a new phenomenon. Even so, in recent years its relevance on the public agenda has increased, as the victory of Brexit supporters in the UK or the election of Donald Trump for US president have shown. Academic interest runs in parallel to the consideration of disinformation as a growing priority for governments and international organizations. On the other hand, the emergence of the COVID-19 pandemic accelerated the decline of regional media, already affected by the digital transformation and declining business models, now challenged by the platforms, which became essential mediators in the advertising market. The decline of regional media leaves communities in a state of serious vulnerability as information is increasingly consumed through social media, where disinformation easily proliferates. As in the pandemic context, disinformation is also a virus that spreads quickly and has a high potential for damage to democracy, namely at a local level. It is precisely where we intend to focus the debate, curiously where it has been little present. It is precisely from the local public sphere that responses to disinformation can emerge,  especially in a collaborative relationship between journalists and (other) active members of the community.

\keywords{Public sphere \sep Local journalism \sep Regional media \sep Disinformation \sep Fact-checking.}
\end{abstract}
\end{english}
% if there is another abstract, insert it here using the same scheme
\end{polyabstract}

\section{Introdução}
A desinformação não é um fenômeno novo, é algo que acompanha a história da humanidade \cite{ireton2019}, mas é fácil perceber que sua relevância na agenda pública aumentou à medida que cresceu a velocidade de disseminação de informação proporcionada pela ascensão das plataformas de redes sociais. A desinformação passa a alcançar espaço na agenda pública desde o anúncio do Brexit e da eleição de Donald Trump, em 2016, uma vez que muitos analistas atribuíram a esse fenômeno uma forte influência nesses resultados. Nos dois casos, o uso das mais recentes tecnologias de Big Data é fundamental para que as mensagens cheguem mais a um público maior, porém mais direcionado, por vezes estando mais propenso a aceitar as mensagens divulgadas. Algoritmos se tornaram atores relevantes na esfera pública, e decisões da vida cotidiana se tornam cada vez mais mediadas por essas tecnologias, resultando em novos desafios para a democracia, em que se inclui a polarização, o fenômeno da desinformação e a formação das câmaras de eco e filtros de bolhas \cite{garcia-orosa_algorithms_2023}. 

Combater a desinformação tornou-se uma tarefa relevante para governos e organizações internacionais. Relativamente à Comissão Europeia, os esforços no combate à desinformação datam de 2015, com o lançamento mais recentemente do "Code of Practice on Disinformation"\footnote{Disponível em \url{https://digital-strategy.ec.europa.eu/en/policies/code-practice-disinformation}}, em junho de 2022. Parte desse documento baseia-se na experiência acumulada durante a pandemia de Covid-19, quando, segundo a Organização Mundial da Saúde (OMS), viveu-se uma “infodemia”, ou seja, uma “pandemia” de informação – e também de desinformação\footnote{Ver \url{https://www.who.int/health-topics/infodemic}}.

Se por um lado a pandemia de Covid-19 fortaleceu temporariamente alguns meios de comunicação, já que os cidadãos confinados em casa passaram a buscar mais informação sobre um vírus até então desconhecido, ela também acelerou o declínio dos meios de comunicação social regionais, aqueles que realizam a cobertura noticiosa das realidades mais locais. O jornalismo de proximidade enfrenta dificuldades financeiras e dependência do apoio estatal, o que resulta em um número reduzido de jornalistas a trabalhar nessas redações. Assim como acontece na prática jornalística como um todo, no caso dos media regionais, a transição para o digital acaba também por restringi-los aos muros das redações de jornal já que precisam lidar com um fluxo avassalador de informação vinda do público e das redes sociais, reduzindo o tempo para o trabalho de campo \cite{jeronimocorreia_are_2022}. Ao mesmo tempo, conteúdos produzidos por membros das comunidades são facilmente acessados pelo telefone celular por esse mesmo público. No entanto, a falta de verificação desses conteúdos pode abrir espaço para a disseminação de desinformação ou de informação incorreta no espaço público em um nível local \cite{jeronimoespartaza022}.

As dificuldades financeiras evidenciadas pela pandemia de Covid-19 vão configurar um cenário de grave vulnerabilidade para os meios regionais, que têm desaparecido de alguns territórios. Nesses territórios onde não há cobertura local de notícias, atribui-se o conceito de “deserto de notícias”, definido por \textcite{abernathy_expanding_2018} como comunidades, rurais ou urbanas, com acesso limitado a notícias e informações confiáveis que virão a alimentar a democracia em um nível mais local.

É nesse cenário que as plataformas de mídia social vão atuar, tornando-se intermediárias poderosas entre produtores e consumidores de informação. Baseadas em um modelo de negócio sustentado pela recolha massiva de dados, as plataformas funcionam a partir da recomendação de conteúdos e da retenção dos usuários, bombardeando-os com informação que vai ao encontro de suas preferências, em um processo sem mediação humana, todo feito por algoritmos \cite{rieder2018examinando}. Reter o usuário nas plataformas e fazê-lo se engajar nas publicações, comentando, compartilhando e interagindo é a base da relevância delas também no mercado publicitário. A disseminação de desinformação, normalmente feita à medida para viralizar, e em regra alcançando muito mais gente que as notícias verdadeiras \cite{baptista_understanding_2020, vosoughi_spread_2018}, acaba por alimentar o lucro financeiro dessas empresas, que, ao mesmo tempo em que se comprometem com a limitação desses conteúdos, o fazem por vezes de uma forma lenta e insuficiente.

A amplificação da informação na esfera pública é tão relevante quanto o acesso à informação, e pode se dar de diversas formas, segundo \textcite{jungherr_disinformation_2021}. A partir da vigilância generalizada e a recolha massiva de dados, plataformas como \emph{Facebook}, \emph{Twitter}, \emph{YouTube}, e \emph{WhatsApp}, agora parte fundamental da esfera pública, oferecem aos seus usuários a oportunidade de distribuir qualquer tipo de informação sem qualquer verificação profissional, incluindo conteúdo político que vão complementar, por vezes de forma enganosa, o noticiário midiático \cite{jungherr_disinformation_2021}. Os temas debatidos muitas vezes são impulsionados de forma ilegítima à ordem do dia, arraigados no uso de bots e inteligência artificial \cite{jeronimoespartaza022}. Essas ações podem criar uma percepção falsa da esfera pública, já que há uma distorção ao nível da importância atribuída aos temas e assuntos debatidos e ao nível da opinião pública.

Neste artigo, pretendemos lançar o debate sobre o papel da desinformação na esfera pública local. Tal como no contexto pandêmico, também a desinformação é um vírus que se propaga rapidamente e com elevado potencial de dano à democracia, designadamente ao nível mais local. É precisamente aí que pretendemos centrar o debate, curiosamente onde tem estado pouco presente. Entendemos que é precisamente a partir da esfera pública local que podem emergir possíveis respostas à problemática da desinformação, nomeadamente numa relação colaborativa entre jornalistas e (outros) membros ativos da comunidade.

Muitas iniciativas de \emph{fact-checking} têm surgido nos últimos anos em Portugal, notadamente a nível nacional. Em nível local, as redações jornalísticas têm menor disponibilidade de jornalistas para realizar checagens de fatos. Ao mesmo tempo, o \emph{fact-checking} realizado por profissionais, de forma geral, têm problemas tanto de escalabilidade quanto de confiança nos meios que o estão a realizar \cite{allen_scaling_2021}. A partir dessa observação, muitos estudos, nomeadamente nos Estados Unidos, voltaram-se para o aproveitamento da “sabedoria das multidões” \cite{surowiecki_wisdom_2005} nos processos de \emph{fact-checking} nas redes sociais. Os resultados têm sido promissores, no entanto, há diversas limitações que devem ser levadas em consideração e tratadas em novos estudos.

Nesta proposta, queremos entender: como o público poderia se envolver e colaborar com o processo de \emph{fact-checking} para gerar informação credível que vai contribuir para um debate público são? Essa questão merece ser explorada como parte dos esforços de combater a desinformação também nos contextos de proximidade. Propõe-se, neste artigo, que o envolvimento das audiências dos jornais regionais pode trazer potencialidades para alavancar a checagem de factos e ampliar o combate à desinformação em contextos de proximidade.


\section{Desinformação, um fenômeno antigo amplificado pelos meios digitais}
O fenômeno da desinformação, embora não seja novo, é amplificado pela velocidade das redes sociais, tendo sua capacidade de enganar e manipular a opinião pública aumentada. Em alguns casos, conteúdos desinformativos acabam por se tornar “destaque na agenda pública, condicionando resultados eleitorais e gerando episódios de desestabilização política” \cite[p.1]{jeronimoespartaza022}. Há registo de que, na antiguidade, o imperador romano Júlio César divulgava informações enganosas a fim de criar uma imagem negativa a povos estrangeiros \cite{kapferer1993boatos}, enquanto na era da internet um dos episódios falsos mais conhecidos é o endosso do Papa à candidatura de Donald Trump às eleições estadunidenses \cite{baptista_understanding_2020}. A circulação de informações deliberadamente falsas ou gravemente distorcidas nas redes sociais recebeu a alcunha de “\emph{fake news}”, uma terminologia forjada na prática jornalística, mas contestada na literatura acadêmica por carregar em seu nome a palavra notícia ("\emph{news}"), além de ter conotação política que visa deslegitimar os meios tradicionais \cite{rieder2018examinando, baptista_elections_2022}. Por conta dessa dificuldade, a Comissão Europeia optou por utilizar em seus relatórios a terminologia “\emph{disinformation}” (“desinformação”), para cobrir uma gama maior de conteúdos enganosos ou fraudulentos que circulam na esfera pública.

\textcite{wardle_reflexao_2019} chamam o fenômeno de circulação de informações falsas na esfera pública de "desordem da informação" (\emph{information disorder}), um problema sistêmico e abrangente, em que se incluem as ideias de desinformação (disinformation), má-informação (\emph{misinformation}) e informação incorreta (\emph{malinformation}). Esse fenômeno vai abranger uma variedade de conteúdos falsos, que incluem elementos deliberadamente enganosos em seu conteúdo ou em seu contexto \cite{bakir_fake_2018}, incluindo desinformação compartilhada com a intenção de enganar \cite{wardle_reflexao_2019} produzida com um objetivo econômico ou ideológico \cite{tandoc_defining_2018}. A desinformação frequentemente se torna má-informação quando é compartilhada, pois acaba sendo disseminada por indivíduos que não sabem sobre sua falsidade e a divulgam com a intenção de informar seus pares \cite{wardle_reflexao_2019}. Ou ainda fatos reais que não mereciam receber a quantidade de atenção que receberam; a propaganda política que mistura discursos verdadeiros, falsos e enganadores; ou informações que têm como objetivo poluir o ecossistema de notícias \cite{zuckerman2017fake}. O risco da desinformação é que os indivíduos não acreditem mais em todo o conteúdo com que têm contato, incluindo o jornalismo, em um cenário em que as pessoas provavelmente vão acreditar mais em conteúdos emocionais oferecidos pelas redes sociais e deixarão de lado o envolvimento racional com as informações. Dessa forma, esse fenômeno vai turvar o ecossistema informacional e acabar por enfraquecer os fatores de racionalidade que levam os indivíduos, por exemplo, a fazer escolhas eleitorais \cite{ireton2019}, além de minar a confiança nas instituições públicas \cite{rivas-de-roca_comunicacion_2022}, contribuindo para o enfraquecimento do debate público. Esse é um fenômeno fundamentalmente ligado ao poder \cite{kuo_critical_2021} e também um fenômeno de legitimação das “afinidades eletivas”, ou seja, da relação entre como as ideologias ressoam com as predisposições psicológicas das pessoas \cite{zmigrod_cognitive_2021}.

A disseminação da desinformação enquanto fenômeno deve ser vista não como a problemática em si, mas como um sintoma de uma mudança estrutural da esfera pública a partir da introdução dos meios digitais, defendem \textcite{jungherr_disinformation_2021}. Os autores utilizam o conceito de “\emph{public arena}” (arena pública) ao invés de esfera pública para indicar que nessa nova esfera pública mediada não há só o consenso, mas também o conflito.

\begin{quote}
    Mudanças estruturais na arena pública primariamente dizem respeito aos fluxos de informação e a alocação da atenção e incluem o grau com que \emph{gatekeepers} conseguem introduzir ou excluir informação. Os media digitais se tornaram poderosos canais de informação política para grandes audiências independentemente da sua fonte e a atenção não é mais apenas moldada pela autoridade ou popularidade de uma fonte, mas pode ser influenciada por compras de anúncios direcionados e de padrões otimizados para interação e compartilhamento. \cite[p. 2, tradução nossa]{jungherr_disinformation_2021}\footnote{Texto original: “Structural changes in the public arena primarily concern information flows and the allocation of attention and include the degree to which gatekeepers are able to introduce and exclude information. Digital media have become powerful conduits of political information to large audiences irrespective of their source and attention is no longer only shaped by the authority or popularity of a source but can be influenced through targeted ad buys and patterns optimized for interaction and sharing” \cite[p. 2]{jungherr_disinformation_2021}}
\end{quote}

Isso ocorre porque, nas plataformas de mídia social, em que boa parte do conteúdo de desinformação é partilhado \cite{baptista_understanding_2020}, os processos de disseminação das informações vão resultar de uma combinação de processos de filtragem. A própria rede à qual o usuário está vinculado vai dar visibilidade àquilo que considera importante, ao mesmo tempo em que despreza aquilo que não o é, onde cada usuário da rede observa o fluxo de informações e define quais merecem receber visibilidade, de forma análoga ao \emph{gatekeeping}, o que \textcite{bruns_gatekeeping_2011} chama de \emph{gatewatching}: a curadoria ("\emph{curation}") colaborativa das notícias pelas comunidades de usuários. Esse processo tem lugar, no entanto, em um contexto da economia da atenção: enquanto nessa economia, o capital, o trabalho, a informação e o conhecimento existem em oferta suficiente e, em muitos casos, em excesso, a atenção humana a todos esses estímulos é um recurso extremamente escasso. A atenção humana é também alvo das métricas das organizações não apenas de mídia, mas também de tecnologia, neste caso, e determinante de seu sucesso ou fracasso \cite{davenport_attention_2001}.

Para além disso, a filtragem resulta não só da ação dos indivíduos na difusão de informações para sua rede de conexões, mas também da ação dos algoritmos, códigos matemáticos que decidem o que será visto primeiro e o que não será visto, a partir de critérios totalmente diversos daqueles do jornalismo, influenciando de forma profunda o debate na esfera pública \cite{recuero2017midia}.

\section{A crescente supremacia da classificação algorítmica}

Diante de uma produção avassaladora de \emph{posts} e de compartilhamento de \emph{links}, fotos, vídeos e artigos, não só os conteúdos dos amadores são classificados por esses critérios nas mídias sociais, como também as notícias produzidas por organizações de comunicação social. Com isso, várias formas de "\emph{gatekeeping} algorítmico" tornam-se parte da indústria midiática e o \emph{gatekeeping} na era digital muda de uma lógica de relevância para a de popularidade \cite{heinderyckx_reformed_2016}. Atualidade, relevância, verdade, fatos, objetividade e outros critérios que asseguram a credibilidade da produção jornalística são substituídos, nas redes sociais, por outros como popularidade, produtividade e interação, todos ligados à atenção humana, que se torna “moeda” na equação do negócio das plataformas de mídia social \cite{davenport_attention_2001}.  Além disso, em especial os jornais regionais têm grande dificuldade em construir sites que têm grande adesão das audiências: em geral são pouco interativos, pouco atualizados e pouco atrativos visualmente \cite{hidman2015}.

Dessa forma, a unidade primária de notoriedade é cada vez mais a popularidade de uma história entre os usuários digitais \cite{heinderyckx_reformed_2016}. A atenção que um \emph{post} ou um artigo vai receber, portanto, é medida pelas interações que recebe, métrica que não está relacionada com sua veracidade. Nesta lógica, \emph{likes}, reações, comentários e compartilhamentos vão servir como critério de credibilidade nas redes sociais tanto para artigos de desinformação como para notícias \cite{delmazo_fake_2018}. O consumo crescente de informações nas plataformas de mídia social vai acabar por estar dependente dessas lógicas da determinação algorítmica, em que as notícias são filtradas e priorizadas de uma forma desconhecida para o usuário \cite{garcia-orosa_algorithms_2023}.

Nas plataformas de mídias sociais, indivíduos constroem perfis públicos ou semipúblicos para participar de grupos onde realizam trocas sociais através de uma estrutura em rede, em uma nova geração de “espaços públicos mediados” \cite{boyd_social_2007}. São ambientes onde as pessoas podem se reunir publicamente através da mediação da tecnologia, em que as estruturas sociais pré-existentes no mundo \emph{offline} se repetem \cite{recuero2009redes}. As relações estabelecidas em grupos sociais, constituídas a partir de suas relações nos mais variados ambientes vão conferir aos atores determinadas posições nas suas redes sociais, que são fundamentais para compreender seu comportamento. As conexões entre atores estão também associadas à ideia de capital social, ou seja, a rede de relacionamentos de conhecimento mútuo e reconhecimento ou a pertença a grupos sociais. O tamanho da rede de conexões que ele consegue mobilizar e o volume do capital daqueles a quem está ligado vai influenciar a ação dos atores de uma rede no compartilhamento de informação e também na disseminação de desinformação \cite{bourdieu1986forms, recuero2017introduccao}. O usuário que compartilha conteúdos, sejam eles verdadeiros ou falsos, sente sua reputação social reforçada ao se colocar diante de sua rede como alguém informado, que detém informação relevante para o grupo \cite{baptista_understanding_2020}.

O objetivo de quem produz artigos de desinformação é viralizar para que seja cumprido seu objetivo econômico ou ideológico. Artigos de desinformação vão carregar, especialmente em seu título – uma vez que muitos usuários leem apenas o título –, conteúdos que incorporam emoções e evocam sentimentos fortes como raiva ou indignação; que são exagerados; que imitam o formato ou a linguagem jornalística para criar uma falsa sensação de credibilidade; e que são sensacionalistas ou \emph{clickbait} \cite{baptista_understanding_2020}. Isso mostra o quanto a ecologia contemporânea das mídias digitais é crescentemente permeada pelas emoções \cite{bakir_fake_2018}. Baseadas em polêmicas, fatos curiosos ou absurdos, \emph{fake news} têm 70\% mais chances de viralizar que as notícias verdadeiras e alcançam muito mais gente \cite{vosoughi_spread_2018}. Mais engajamento nas redes sociais significa mais tempo gasto nas plataformas, o que reverte em lucro para seus donos.

Os disseminadores de desinformação se aproveitam de outra questão para atingir a desejada “viralização”. Uma rede reúne, antes de tudo, pessoas com pontos de vista em comum, ou seja, uma rede é, antes de tudo, um agrupamento do mesmo, o que favorece a criação de bolhas de informação, onde há menos espaço para novas percepções e aprendizados e a tendência é de que os indivíduos se fechem em um mundo construído a partir do que é familiar \cite{pariser2012filtro, dominique2010informar}. As bolhas serão resultado da classificação algorítmica e das dinâmicas próprias das afinidades eletivas, em um cenário em que a tendência de personalizar informações oferece o risco de o indivíduo se restringir a informações que confirmam sua visão de mundo \cite{reis_onde_2021}, o que, na ponta, irá aumentar a probabilidade de os indivíduos compartilharem aqueles conteúdos enviesados \cite{baptista_understanding_2020}.


\section{Ressignificação da esfera pública em um contexto de rede}
A publicação descentralizada e a possibilidade de os cidadãos engajarem-se em uma autocomunicação de massas \cite{castells_sociedade_2012} fazem com que os indivíduos encontrem informação que vai ao encontro de seus pontos de vista e torna o debate aberto à desinformação. Ao mesmo tempo, os meios digitais, em particular as plataformas de redes sociais, a partir da vigilância generalizada e a recolha massiva de dados, permite aos anunciantes atingir públicos específicos à exposição de seus conteúdos impulsionados por financiamento às plataformas sem a mediação dos \emph{gatekeepers} tradicionais, podendo divulgar desinformação feita à medida para o usuário visado \cite{jungherr_disinformation_2021}. Assim, atores que querem manipular tópicos ou comportamentos na esfera pública introduzem facilmente informações no debate público e dependem apenas de sua rede para que a disseminação alcance um número suficientemente grande de indivíduos. 

A amplificação da informação na esfera pública é tão relevante quanto o acesso à informação, e se dá pela interação do público, pela seleção e hierarquização dos algoritmos, e pelo impulsionamento pago de conteúdo. A amplificação social ocorre de forma mais expressiva quando a informação é construída a partir do humor ou das conexões de grupo, o que pode ser utilizado por atores com intenções maliciosas \cite{jungherr_disinformation_2021}. Nesse contexto, uma outra questão relevante é o uso de \emph{bots} e inteligência artificial, que exponencialmente aumenta o impacto da disseminação de um conteúdo \cite{jeronimoespartaza022}, podendo resultar em uma falsa ideia de que aquela discussão merece grande atenção pública. Essas ações podem criar uma percepção falsa da esfera pública.

\textcite{recuero2017midia} também lembram que os processos participativos das ferramentas que constituem as redes de mídia social dependem de ações e percepções individuais. O que vemos, porém, na internet e nas redes sociais, é a interconexão entre diversos públicos e mídias, em que mídias de massa e indivíduos dialogam. Não é possível, então, considerar a internet e seus espaços de conexão como uma esfera pública isolada ou independente, "mas que pode ser utilizada pelos seus usuários para incrementar as discussões na esfera pública, podendo, em determinadas situações, desencadear demandas sobre o sistema político formal" \cite[p. 180]{sampaio2011internet}.

Para \textcite{aurelio_as_2019}, há uma mudança estrutural no espaço público nas últimas duas décadas. O que ocorre atualmente é uma multiplicação dos polos de enunciação \cite{pinto_fontes_2000} e a perda da centralidade dos meios tradicionais no debate público \cite{bentes_midia-multidao:_2015}. Com as novas tecnologias e novas formas de comunicação, o que continua a ser designado por espaço público já dá lugar a uma outra realidade, com novos instrumentos e novos protagonistas: "Longe de corresponderem a um simples alargamento do raio de ação da imprensa tradicional, as novas tecnologias da comunicação representam uma alteração estrutural da sociedade em que hoje em dia trabalhamos e vivemos" \cite[p. 124]{allen_scaling_2021}.

O jornalismo profissional passou a ser uma voz em meio a uma enorme multidão, e, nas redes sociais digitais, a fronteira entre a argumentação e o ruído pode se tornar imperceptível \cite{pinto_fontes_2000, dominique2010informar}. Não há mais distinção entre leitor e jornalista e qualquer um divulga imagens, opiniões e informações que julga relevantes, mesmo sem verificação.  


\section{A Covid-19 e a ampliação dos “desertos de notícias”}
A pandemia da Covid-19, ao mesmo tempo que amplificou os processos sociais mediados por plataformas, traduziu-se também no declínio dos meios regionais tradicionais. Nos Estados Unidos, por exemplo, se por um lado as versões digitais de jornais regionais alcançaram picos de audiência durante a pandemia, entre o final de 2019 e maio de 2022, mais de 360 impressos fecharam \cite{jolley202, abernathy2022}. Dificilmente uma comunidade que perde um jornal impresso vê uma alternativa digital nascer em seu lugar \cite{abernathy2022}. A existência de uma identidade local definida conectada a um território específico e o compromisso comunicacional do jornalismo em um nível mais local com os pequenos territórios é denominado por alguns autores como jornalismo de proximidade \cite{camponez2002jornalismo, coelho_tv_2003, lopez_garcia_ciberperiodismo_2008, jeronimo2015ciberjornalismo}. Enquanto sua identidade está ligada às comunidades, os meios regionais fornecem aos membros da comunidade informação e conhecimento necessários para que os indivíduos possam exercer seu papel de cidadãos. A emergência dos meios digitais, porém, tem resultado em uma perda significativa da atenção dos indivíduos nas notícias locais, o que ameaça a sobrevivência dos jornais regionais em um cenário em que conteúdos produzidos por membros das comunidades são facilmente acessados pelo celular \cite{jeronimo2013jornalismo, jeronimoespartaza022}.

Além disso, também em um contexto regional, muitos meios regionais ainda se fundam num modelo de negócio baseado em publicidade e assinaturas, que foi abalado pela introdução das redes sociais que, a partir da recolha massiva de dados comportamentais de seus utilizadores, oferecem um modelo de publicidade mais atrativo e mais acessível aos negócios locais que o modelo tradicional do jornalismo. Como resultado, os meios regionais testemunham uma quebra brutal de seus modelos de negócios tradicionais \cite{jeronimoespartaza022}.

No entanto, como resultado da pandemia, viu-se a ampliação dos chamados “desertos de notícias”. No caso dos Estados Unidos, estudos sistematizados revelaram um novo cenário bastante negativo em que o declínio dos meios regionais deixa as comunidades em estado de grave vulnerabilidade à desinformação. O relatório "\emph{The State of Local News}" \cite{abernathy2022} revelou que entre o final de 2019 e maio de 2022, mais de 360 jornais locais foram encerrados. Desde 2005, os EUA perderam mais de um quarto de seus jornais ou um total de 2.500 jornais que serviam suas comunidades locais. Uma boa parte das localidades que perderam jornais não vão ter um substituto, seja ele digital ou impresso \cite{abernathy2022}. Desta forma, cidades, vilas e aldeias vivem uma contradição enquanto possuem acesso à informação confiável sobre acontecimentos de outros estados e países, através dos media nacionais e da internet, mas não possuem uma fonte de informação credível sobre sua própria comunidade.

No caso de Portugal, em 2020, um estudo mostrou que em 57 dos 308 concelhos do país não haviam sido encontrados nem publicações impressas, nem digitais, nem rádios, nem operadores de televisão \cite{ramos2021deserto}. Já em 2022, foram 54 concelhos classificados como desertos de notícias, onde não há nem rádios nem meios impressos ou digitais \cite{jeronimo2022}. Este é um fenômeno bastante observado no interior do país, onde estão os distritos com menores populações.

Embora quase um quinto dos concelhos não tenham cobertura de notícias locais, a mais recente edição do \emph{Reuters Digital News Report} sobre Portugal \cite{cardoso__reuters_2022} mostra que o interesse dos portugueses por notícias locais permanece elevado. O gênero internacional lidera o ranking de interesse, indicado por 55,6\% dos entrevistados, seguido por notícias sobre a pandemia da Covid-19, indicados por 55,2\%, e as notícias locais, por 53,9\%.

As plataformas nesse vazio dos desertos de notícias trazem potencial de polarização maior ao espaço público \cite{abernathy2022}, um debate sempre com alguma toxicidade, enquanto se o espaço for partilhado por informação jornalística isso ajuda a equilibrar a balança. Enquanto a desinformação propaga-se com elevado potencial de dano à democracia, designadamente ao nível mais local, dada a relação de proximidade e os recursos humanos escassos que muitos meios regionais vivenciam, o engajamento das comunidades locais pode ser aproveitado também para combater a desinformação em nível local, uma vez que gozam de uma relação especial de confiança e conexão emocional com o público por supostamente defenderem os mesmos interesses e veicularem informações de interesse daquela comunidade específica em que também os jornalistas estão inseridos. Segundo \textcite{jeronimoespartaza022}, o papel dos meios regionais no combate à desinformação é central, e pode incluir rotinas de \emph{fact-checking} e reportagens em profundidade.

Dessa forma, a participação dos cidadãos e comunidades locais também pode ser uma forma de combater a desinformação em uma escala local. A partir da esfera pública local podem emergir respostas à problemática da desinformação, nomeadamente numa relação colaborativa entre jornalistas e outros membros ativos da comunidade.

\section{Ampliar o \emph{fact-checking} para o público no contexto local}
Muitas iniciativas de \emph{fact-checking} têm surgido nos últimos anos em Portugal, notadamente a nível nacional. A verificação de fatos, ou \emph{fact-checking}, tem o potencial de reduzir a proliferação de desinformação na esfera pública. Primeiro, por informar uma parte dos usuários sobre a falta de precisão daquele conteúdo, reduzindo tanto a crença no artigo quanto a intenção de compartilhar o conteúdo. E, segundo, por influenciar os algoritmos de recomendação das mídias sociais a mostrá-los menos aos usuários \cite{allen_scaling_2021}. No entanto, o \emph{fact-checking} realizado por profissionais têm problemas tanto de escalabilidade quanto de confiança nos meios que o estão a realizar.

Isso porque a quantidade de checagens realizadas por essas organizações profissionais é muito limitada comparada à quantidade de conteúdos enganosos espalhados nas mídias sociais, enquanto a informação falsa normalmente irá circular até receber tanta atenção que acabe por ser checada \cite{prollochs_community-based_nodate}. Além disso, muitos indivíduos não confiam nos verificadores de informação por acreditarem que eles não são isentos \cite{allen_scaling_2021}, ainda que existam estudos a demonstrar o contrário \cite{baptista_elections_2022}, e ainda por causa das afinidades eletivas e seus efeitos a nível das crenças e ideologias \cite{zmigrod_cognitive_2021}. Esses fatores podem limitar a eficácia dos \emph{fact-checking} realizados por agências jornalísticas profissionais. Uma outra questão que se coloca é a crucialidade da amostra que é selecionada a ser checada e, posteriormente, desmentida, em seu impacto final na esfera pública \cite{drolsbach2022diffusion}.

A partir dessa observação, há estudos que apontam para o aproveitamento das multidões enquanto participantes ativos dos processos de \emph{fact-checking} nas redes sociais. A justificativa é que checagens poderiam ser escaladas a um custo muito mais baixo que a ampliação de equipes de jornalistas checadores profissionais \cite{allen_scaling_2021, pennycook_fighting_2019}. A confiança nas checagens, por outro lado, teria o potencial de aumentar dada a natureza coletiva das classificações \cite{prollochs_community-based_nodate}. Aliás, em 2021, o Twitter lançou o projeto piloto “\emph{Birdwatch}”, um projeto que aproveita as multidões para adicionar “contexto” aos \emph{tweets}, como a própria plataforma define, e que em novembro de 2022 passou a ser chamado “\emph{Community Notes}” \cite{coleman2021, twitter_birdwatch_2022}. Usuários podem marcar tweets como errôneos e justificar a avaliação. Esse projeto foi ampliado em 2022, mas ainda se mantém restrito aos usuários dos Estados Unidos. A Meta, dona do Facebook e do Instagram, mantém sua abordagem ligada aos checadores profissionais, baseando-se nas avaliações feitas por o que a empresa chama de “checadores independentes”, que adotam a metodologia da Rede Internacional de Verificação de Fatos (IFCN). Em 2019, o CEO da empresa, na época ainda nomeada Facebook, ventilou a ideia de aproveitar as multidões para impulsionar as checagens na rede \cite{bell2019}.

A literatura mostra que, embora os indivíduos possam realizar classificações sem rigor e exatidão individualmente, quando em grupo, a média das avaliações realizadas por todos os indivíduos apresenta uma precisão surpreendente. A esse fenômeno, \textcite{surowiecki_wisdom_2005} chamou de “\emph{wisdom of the crowds}” (“sabedoria das multidões”).

Nas redes sociais, dada sua organização em rede e seu caráter colaborativo, a sabedoria das multidões introduz uma promissora possibilidade quando aplicada à checagem de conteúdos virais e nem sempre exatos que circulam nas redes. Para que a precisão seja alcançada, no entanto, é preciso que o grupo satisfaça quatro requisitos, diz \textcite{surowiecki_wisdom_2005}: diversidade de opiniões, independência, descentralização e agregação.

\begin{quote}
    Se você perguntar a um grupo suficientemente grande de pessoas diversas, independentes para fazer uma predição ou estimar uma probabilidade, e tirar uma média dessas estimativas, os erros que cada um deles fazem quando formulam uma resposta irão se cancelar. \cite[p. 10, tradução nossa]{surowiecki_wisdom_2005}\footnote{Texto original: “If you ask a large enough group of diverse, independent people to make a prediction or estimate a probability, and then average those estimates, the errors each of them makes with coming up with an answer will cancel themselves out”. \cite[p. 10]{surowiecki_wisdom_2005}}
\end{quote}

Alguns estudos sugerem que as multidões podem ser utilizadas para marcar certos conteúdos partilhados \emph{online} como duvidosos ou falsos, e a partir desta classificação, aumentar a eficiência e a agilidade dos processos de checagem de conteúdos que circulam em redes como o Facebook, o Twitter ou o Weibo, utilizando algoritmos que, alimentados pela avaliação dos usuários, determinem a partir de quantas sinalizações um \emph{post} merece ser checado \cite{kim_leveraging_2018}. Dessa forma, processos algorítmicos podem ser programados para reduzir a exposição de conteúdos falsos nas redes. Além disso, enquanto muitos \emph{fact-checkers} são acusados de terem vieses em suas checagens, um grupo grande de pessoas leigas de todos os espectros políticos podem realizar checagens de forma equilibrada e confiável graças à sabedoria das multidões \cite{allen_scaling_2021}.

Outros estudos sugerem que a contribuição da multidão pode ir além de apenas rotular conteúdos. Dois estudos realizados nos Estados Unidos mostraram que multidões podem realizar checagens de conteúdos a partir dos títulos e lides de notícias e de artigos de desinformação com um nível de precisão bastante satisfatório se comparado aos de \emph{fact-checkers} profissionais, assim como conseguem classificar a confiabilidade dos editores de notícias de forma análoga aos verificadores profissionais \cite{allen_scaling_2021, pennycook_fighting_2019}. Os resultados sugerem, portanto, que um número relativamente pequeno de leigos consegue produzir um julgamento agregado bastante próximo ao de checadores profissionais observando apenas o título e o lide de um artigo. A proximidade dos resultados entre leigos e profissionais é particularmente maior quando se fala de informação política \cite[p. 5, tradução nossa]{allen_scaling_2021}:\footnote{Texto original: Here, we have shown that crowdsourcing can help identify misinformation at scale. We find that, after judging merely the headline and lede of an article, a small, politically balanced crowd of laypeople can match the performance of fact-checkers researching the full article. (...) Together, our findings suggest that crowdsourcing could be a powerful tool for scaling fact-checking on social media. (p. 5).}

\begin{quote}
    Aqui, mostramos que o crowdsourcing pode ajudar a identificar desinformação em escala. Acreditamos que, após o mero julgamento do título e do lide de um artigo, um grupo de leigos pequeno e politicamente balanceado consegue ter uma performance semelhante a de \emph{fact-checkers} investigando o artigo complete. (…) Juntas, as nossas descobertas sugerem que \emph{crowdsourcing} pode ser uma ferramenta poderosa para dar escala ao \emph{fact-checking} nas redes sociais.
\end{quote}


\section{Desafios e limitações a serem observados}
Por outro lado, algumas limitações precisam ser observadas. Um estudo sobre as checagens realizadas por não profissionais na plataforma \emph{Birdwatch} sugere que informação divulgada por líderes de opinião tendem a trazer desafios para as checagens não profissionais \cite[p. 803]{prollochs_community-based_nodate}. \emph{Tweets} de políticos influentes, como, por exemplo, a congressista norte-americana democrata Alexandria Ocasio-Cortez ou a republicana Marjorie Taylor Greene despertam um nível de consenso significativamente menor no \emph{Birdwatch} se o conteúdo é rotulado como desinformação. Ao mesmo tempo, corresponde a uma parcela significativa de \emph{tweets} checados pela audiência, o que indica que os usuários têm um forte interesse em desmentir conteúdo político ou que os usuários estão mais propensos a relatarem como falsos os conteúdos que não se alinham a sua ideologia \cite{prollochs_community-based_nodate}.

A familiaridade com o editor também é bastante relevante para a percepção de confiabilidade de um site, ou seja, as pessoas desconfiam predominantemente de editores de notícias com os quais não estão familiarizados, o que pode gerar distorções \cite{allen_scaling_2021}. Uma classificação que leva em conta a fonte onde o conteúdo é encontrado ainda pode falhar no caso de sites e portais noticiosos novos \cite{godel_moderating_2021}.

Em relação à avaliação dos conteúdos pelos usuários, o grande risco está ligado à manipulação por \emph{bots} ou por grupos organizados. \textcite{allen_scaling_2021} sugerem que utilizar uma escala de classificação em que o usuário é convidado a dar opiniões sobre os conteúdos ou contratar pessoas leigas para realizar as classificações pode eliminar esses riscos. Apesar de ser promissor, é preciso considerar ainda a falta de envolvimento no raciocínio cognitivo e as avaliações politicamente motivadas em um cenário sociopolítico muito polarizado pode trazer impactos negativos relevantes ao processo de \emph{fact-checking} em um contexto de multidões \cite{prollochs_community-based_nodate}. \textcite{godel_moderating_2021} também avalia que grupos compostos de pessoas leigas aleatórias podem ser pouco efetivos em distinguir conteúdos enganosos, em especial quando há vieses, o que é o caso de muitos artigos de desinformação. Em vez de multidões aleatórias, neste caso, grupos selecionados de indivíduos com características específicas – ideologias balanceadas ou com alto conhecimento político – podem produzir uma performance melhor nas checagens.

Por fim, embora a ideia de checagem coletiva seja promissora, há dúvidas se será possível realizar checagens de maneira confiável em larga escala e em tempo real. \textcite{godel_moderating_2021} observa que uma das potencialidades do \emph{fact-checking} feito pelas multidões é a velocidade, uma vez que a maior parte das informações enganosas que circulam nas redes têm seu ciclo completo de nascimento e morte em poucos dias \cite{vosoughi_spread_2018}, o que também impacta negativamente o trabalho dos \emph{fact-checkers} e dos jornalistas profissionais. Em um ambiente de checagem em tempo real, \textcite{godel_moderating_2021} observa que uma parte grande dos checadores não profissionais acaba por rotular conteúdos verdadeiros como falsos e vice-versa, criando falsos positivos, o que poderia resultar em censura nas plataformas de mídias sociais.

Embora vejam diversas restrições na abordagem da checagem coletiva de conteúdos, \textcite{godel_moderating_2021} considera que há potencialidades se essa abordagem for parte de um conjunto de maior de iniciativas para restringir a disseminação de informação falsa nas redes sociais.


\section{Conclusão}

A crise dos meios de comunicação regionais em Portugal e no mundo, aprofundada pela pandemia de Covid-19, traz imensos desafios às populações que estão em locais mais afastados dos grandes centros, em especial poder lidar com a desinformação de forma a não passarem a acreditar essencialmente em conteúdos manipulados e enganosos. No entanto, os poucos recursos e a escassez cada vez maior de financiamento para esses meios tornam essa tarefa especialmente desafiadora para os jornalistas dessas regiões. Dedicar-se à checagem de fatos é uma tarefa que consome tempo e recursos humanos, em geral pouco abundantes em redações pequenas como a dos meios regionais portugueses. Nos desertos de notícias, a situação torna-se ainda mais dramática.

Nesse contexto, aproveitar a sabedoria das multidões, que já provou ter potencial no campo da informação digital, pode ser uma importante saída, em especial por causa da relação de confiança construída ao longo das décadas entre a imprensa tradicional e os habitantes da região em cuja cobertura os jornalistas centram-se. Outras iniciativas de colaboração em ambientes \emph{online} já revelaram que a verificação de informação externa ao círculo jornalístico, em que cidadãos podem colaborar com objetivo último de ter informação credível, pode resultar em um ecossistema informacional local confiável.

A participação do público no processo jornalístico já foi explorada em projetos anteriores no contexto regional português, no âmbito da seleção dos temas a serem cobertos. O projeto “Agenda dos Cidadãos”, por exemplo, experimentou práticas jornalísticas que objetivaram reforçar o compromisso dos cidadãos com a comunidade e analisou as possibilidades do jornalismo cívico e da colaboração entre sociedade civil e órgãos de comunicação social regionais. Observaram-se nessa experiência potencialidades de participação cívica nas práticas jornalísticas no contexto regional, ao mesmo tempo em que foram identificadas algumas dificuldades relacionadas a elementos de natureza política e cultural, como o défice de participação política, em conjunto com assimetrias regionais, sociais, etárias e culturais \cite{correia2014agenda}.

Além desses, outros desafios já comentados emergem dessas dinâmicas, o que leva à necessidade de realizar mais estudos. A dificuldade de classificar informação disseminada por políticos influentes, os vieses ideológicos, a existência de meios de comunicação social novos e desconhecidos do público e o desafio de classificar conteúdos nunca antes debatidos como verdadeiros ou falsos são algumas das principais limitações já comentadas. No entanto, há formas de mitigar os efeitos dessas limitações e realizar um estudo confiável a partir das premissas estabelecidas por estudos anteriores.

Do ponto de vista do jornalismo de proximidade, outra questão se coloca. Mobilizar o público pode ser uma tarefa complicada uma vez que, embora os jornalistas dos meios regionais partilhem territórios com suas audiências, com a digitalização a permear todas as instâncias de produção e as redações a diminuir, eles tendem a ir menos a campo, perdendo o contato direto com a comunidade. Em segundo lugar, permitir – e até incentivar – as multidões a realizarem um trabalho que é do domínio dos jornalistas pode resultar na perda do prestígio ou reputação, assim como o seu poder social enquanto autoridade da informação confiável nessas comunidades \cite{jeronimocorreia_are_2022}.

Todas essas questões se colocam e precisam ser levadas em consideração quando se propõe envolver o público em um processo essencialmente do domínio da práxis jornalística. No entanto, combater a desinformação já se provou ser imperativo também dos atores dos contextos mais locais, que sofrem há anos com sucessivas crises financeiras e escassez tanto de recursos humanos quanto materiais. Nesse caso, envolver o público pode trazer mais benefícios que malefícios ao jornalismo de proximidade, reforçando laços de confiança e fortalecendo a democracia e a esfera pública em um nível mais local.


\section{Financiamento}
O presente estudo foi desenvolvido no âmbito do MediaTrust.Lab, projeto financiado pela Fundação para a Ciência e Tecnologia (PTDC/COM-JOR/3866/2020), Portugal.



\printbibliography\label{sec-bib}
% if the text is not in Portuguese, it might be necessary to use the code below instead to print the correct ABNT abbreviations [s.n.], [s.l.]
%\begin{portuguese}
%\printbibliography[title={Bibliography}]
%\end{portuguese}


%full list: conceptualization,datacuration,formalanalysis,funding,investigation,methodology,projadm,resources,software,supervision,validation,visualization,writing,review
\begin{contributors}[sec-contributors]
\authorcontribution{Luísa Torre}[investigation,methodology,writing]
\authorcontribution{Pedro Jerónimo}[conceptualization,funding,projadm,supervision,methodology,review]
\end{contributors}



\end{document}

