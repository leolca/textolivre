\section{Conclusão}\label{sec-conclusao}

Buscamos investigar o potencial de um ambiente imersivo na aprendizagem de
vocabulário de inglês como LE, observando, em particular, sua capacidade de
promover atenção e motivação nos alunos. Para responder às perguntas de
pesquisa, foram realizados três experimentos com alunos de nível elementar de
proficiência em língua inglesa de um colégio de aplicação em Minas Gerais.

A partir dos resultados dos testes de vocabulário, verificou-se que houve ganho
de aprendizagem nas três condições de testagem, respondendo positivamente à
primeira pergunta de pesquisa: ``\emph{O ambiente imersivo em 360º contribui para a
aprendizagem de vocabulário de inglês como língua estrangeira?}''. Além disso,
verificou-se que o ambiente imersivo em 360º, ao proporcionar uma experiência
nova aos alunos através da imersão e da multimodalidade, é capaz de fazer com
que se sintam mais motivados e, assim, direcionem melhor sua atenção ao objeto
de aprendizagem \cite{gardner2010}, o que resulta em mais \emph{noticing} \cite{schmidit1990}
e possibilita o aprendizado do vocabulário, respondendo também afirmativamente
à segunda pergunta: ``\emph{O ambiente imersivo em 360º contribui para promoção da
motivação e da atenção durante o aprendizado do vocabulário de inglês como
LE?}''.

Ainda, verificou-se que o ganho de aprendizagem foi mais expressivo em uma
condição que em outras. Assim, calculou-se o ganho médio de aprendizagem
lexical para cada grupo. Os grupos expostos ao ambiente imersivo em 360º, ao
ambiente de leitura com glossário multimodal e ao ambiente imersivo em 360º +
ambiente de leitura com glossário multimodal apresentaram ganho médio de
aprendizagem de, respectivamente, 18\%, 28\% e 62\%. Logo, responde-se à terceira
pergunta de pesquisa (``\emph{Comparativamente, qual das três condições testadas –
ambiente imersivo em 360º, ambiente hipermodal de leitura com glossário ou
ambiente imersivo em 360º + ambiente hipermodal de leitura com glossário – é
mais eficiente para a aprendizagem de vocabulário em LE?}''), pois a condição
mais eficiente para a aprendizagem de vocabulário foi a terceira, com 62\%. Isso
evidencia a importância do verbal e da apresentação simultânea das diferentes
modalidades aos alunos de nível elementar \cite{procopio2016}, pois os grupos que
contaram com um suporte da escrita (grupos 2 e 3), simultaneamente às imagens,
obtiveram maior ganho de aprendizagem que o grupo que contava apenas com as
modalidades sonora e visual (grupo 1).

Conclui-se que o ambiente imersivo em 360º tem um grande potencial pedagógico,
pois, ao apresentar o estímulo de forma multimodal, contribui
significativamente para o aprendizado e a retenção do vocabulário da LE, como
mostra a literatura \cite{chun1996,saito2015,procopio2016,mayer2001,monteiro2021}.
Ademais, ele é uma alternativa mais acessível para a
inserção da RV em salas de aula de LE, pois é mais econômica e propicia a
interação do aprendiz com elementos virtuais de forma dinâmica, fazendo com que
ele engaje mais nas atividades propostas.

Contudo, algumas limitações precisam ser apontadas, como a impossibilidade de
reproduzir as pronúncias das palavras no glossário devido à falta de
dispositivos de reprodução de som nos computadores da escola, impedindo que os
alunos acessassem uma das modalidades presentes do ambiente (a sonora). Outra
limitação que precisa ser destacada é a impossibilidade de inserir um glossário
multimodal no ambiente imersivo em 360º, o que permitiria a consulta das
palavras alvo de forma simultânea à sua exploração, oferecendo um suporte
verbal durante a atividade.

Ainda, mais estudos precisam ser desenvolvidos para que o uso ambientes de RV
seja cada vez mais popularizado e integrado ao ensino de LE, estando cada vez
mais presente nas salas de aula brasileiras. Há também necessidade de modelos e
teorias de ASL que considerem em suas abordagens atenção e motivação de forma
integrada e como elementos centrais do processo de aprendizagem de línguas,
visto que esses construtos, quando abordados, o são de forma periférica e não
integrados em estudos encontrados na literatura \cite{krashen1985,ellis1997,gass1997,hede2002,mayer2002,moreno2007,saito2015,monteiro2021}.
