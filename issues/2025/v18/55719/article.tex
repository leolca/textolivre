\documentclass[portuguese]{textolivre}

% metadata
\journalname{Texto Livre}
\thevolume{18}
%\thenumber{1} % old template
\theyear{2025}
\receiveddate{\DTMdisplaydate{2024}{11}{5}{-1}}
\accepteddate{\DTMdisplaydate{2025}{1}{15}{-1}}
\publisheddate{\today}
\corrauthor{Rafaela Lemos Sales}
\articledoi{10.1590/1983-3652.2025.55719}
%\articleid{NNNN} % if the article ID is not the last 5 numbers of its DOI, provide it using \articleid{} commmand 
% list of available sesscions in the journal: articles, dossier, reports, essays, reviews, interviews, editorial
\articlesessionname{dossier}
\runningauthor{Sales}
%\editorname{Leonardo Araújo} % old template
\sectioneditorname{Hugo Heredia Ponce}
\layouteditorname{João Mesquita}

\title{O uso de um ambiente imersivo para promover aprendizagem de vocabulário de inglês como língua estrangeira}
\othertitle{The use of an immersive environment to promote English as a foreign language vocabulary learning}

\author[1]{Rafaela Lemos Sales~\orcid{0000-0003-2272-8847}\thanks{Email: \href{rafaelalemos834@yahoo.com.br}{rafaelalemos834@yahoo.com.br}}}
\author[1]{Patrícia Nora de Souza Ribeiro~\orcid{0000-0003-1713-0445}\thanks{Email: \href{mailto:patnora.souza@gmail.com}{patnora.souza@gmail.com}}}

\affil[1]{Universidade Federal de Juiz de Fora (UFJF), Faculdade de Letras, Departamento de Línguas Estrangeiras Modernas, Juiz de Fora, MG, Brasil.}

\addbibresource{article.bib}

%\usepackage{calc,easyReview,enumitem,threeparttable}

\begin{document}
\maketitle
\begin{polyabstract}
\begin{abstract}
Dispositivos tecnológicos estão cada vez mais presentes no ensino,
culminando, mais recentemente, no uso de ambientes imersivos como
ferramentas pedagógicas. Nesse contexto, esta pesquisa objetiva
investigar o uso de um ambiente imersivo em 360º para promover atenção e
motivação nos alunos durante a aprendizagem de vocabulário de inglês
como língua estrangeira (LE). No plano teórico, o estudo é fundamentado
pela \emph{Noticing Hypothesis} \cite{schmidit1990}, pela Teoria
Motivacional de \textcite{gardner1979,gardner2010} e pela Teoria Cognitiva de
Aprendizagem Multimídia \cite{mayer2001,moreno2007}. Metodologicamente, esta
pesquisa é quase experimental, buscando controlar o uso de um ambiente
imersivo em 360º para a promoção da atenção e da motivação durante a
aprendizagem de vocabulário de inglês como LE. Para tanto, foram
realizados três experimentos com 81 alunos do Ensino Médio de uma escola
pública de Juiz de Fora (MG). Esses alunos foram divididos igualmente em
três grupos e cada grupo foi submetido a uma condição de testagem:
ambiente imersivo em 360º, ambiente de leitura com glossário multimodal
e ambiente imersivo em 360º + ambiente de leitura com glossário
multimodal. Os resultados mostram que houve ganho de aprendizagem nas
três condições de testagem, mas a aquisição lexical dos alunos expostos
à terceira condição foi superior. Conclui-se que isso se deve à
multimodalidade presente no ambiente, que contribuiu para a promoção da
atenção e da motivação, resultando no aprendizado.
  
   
\keywords{Aprendizagem de vocabulário \sep Realidade Virtual \sep
Multimodalidade \sep Atenção \sep Motivação}
\end{abstract}

\begin{english}
\begin{abstract}
  Technological devices have become increasingly present in education,
  recently leading to the use of immersive environments as pedagogical
  tools. In this context, this research aims to investigate the use of a
  360º immersive environment to promote attention and motivation during
  English vocabulary learning. Theoretically, the study is grounded in
  Schmidt\textquotesingle s Noticing Hypothesis \citeyear{schmidit1990},
  Gardner\textquotesingle s Motivational Theory \citeyear{gardner1979,gardner2010}, and Mayer's
  Cognitive Theory of Multimedia Learning \cite{mayer2001,moreno2007}. Methodologically,
  this is a quasi-experimental study designed to control the use of a 360º
  immersive environment to promote attention and motivation during English
  vocabulary acquisition. Three experiments were conducted with 81 high
  school students from a public school in Juiz de Fora, MG. These students
  were equally divided into three groups and each group was exposed to a
  different condition: a 360º immersive environment, a reading environment
  with a multimodal glossary, and a 360º immersive environment + reading
  environment with a multimodal glossary. The results indicate learning
  gains across all three conditions, but students exposed to the third
  condition demonstrated superior lexical acquisition. This improvement is
  attributed to the multimodality of the environment, which helped enhance
  attention and motivation, ultimately leading to better learning
  outcomes.
  


\keywords{Vocabulary learning \sep Virtual Reality \sep Multimodality \sep
Attention \sep Motivation}
\end{abstract}
\end{english}
\end{polyabstract}

\section{Introdução}\label{sec-intro}

O processo de escrita é uma das atividades mais complexas que o ser
humano é capaz de realizar, em razão de vários fatores, a exemplo das
exigências feitas à memória e ao raciocínio durante o momento de
produção \cite{garcez2020}. São inúmeros os conhecimentos e habilidades que
precisam ser articulados e harmonizados para que o texto tome forma.
Tendo em vista esse seu caráter complexo, ainda são recorrentes falsas
crenças sobre a produção textual, que levam pessoas a acreditarem que
podem dominá-la a partir de ``dicas'' desvinculadas de seu contexto de
produção.

A ideia de que fórmulas pré-fabricadas e ``dicas'' isoladas são métodos
cabíveis no ensino de produção textual, apenas negligencia as etapas
necessárias que caracterizam um texto adequado conforme seu contexto de
produção \cite{garcez2020}. O processo de escrita é uma atividade que
carece de idas e vindas, pois deve admitir três grandes momentos que se
intercalam e devem ser compreendidos de modo indissociável: o do
planejamento, o da escrita propriamente dita e o da revisão \cite{antunes2005}. Enquadrar esse processo em uma perspectiva prescrita e linear
pode resultar em estudantes frustrados pela construção de textos
truncados e artificiais.

Essa realidade se agrava quando observamos o cenário acadêmico, em que
as exigências com relação a produções textuais se intensificam. As
expectativas quanto a essas produções não se limitam à utilização
adequada da norma-padrão ou a vocabulários específicos; expandem-se para
aspectos implícitos de produção que precisam ser considerados, como o
que pode ser dito, por quem, de que forma, sob que ponto de vista e
fundamentado em qual autor \cite{oliveira2024}.

Pensando na comunidade discursiva acadêmica, uma das produções textuais
mais demandadas em cursos de graduação da área de humanas é o artigo
acadêmico \cite{motta-roth2010}. Por ser um dos principais
veículos de divulgação científica, a circulação desse gênero na academia
é incontornável, sendo bastante exigido o seu consumo e produção por
parte de professores, estudantes e pesquisadores. Embora seja uma
produção essencialmente ligada ao meio universitário, sua feitura é
quase sempre exigida sem antes ser ensinada. Esse fato pode levar os
alunos a somarem suas dificuldades com o processo de escrita à
dificuldade de produzir um texto do qual desconhecem seu contexto de
produção, estrutura composicional e outras ``dimensões escondidas''
\cite{street2010} que perpassam a construção de um artigo.

Ao exigir do autor capacidade de síntese, descrição, análise e
argumentação, utilizando-se das convenções próprias à determinada área,
o artigo contempla informações geradas em pesquisas a serem submetidas a
apreciações públicas \cite{motta-roth2010}. Sua relevância remonta
à popularização da ciência que, por sua vez, possui a potencialidade de
descrever fenômenos sociais e até mesmo gerar algum impacto benéfico ao
público em geral.

A partir dessas pontuações, torna-se clara a importância de produzir
artigos acadêmicos e a responsabilidade do seu produtor de popularizar
os conhecimentos produzidos na esfera acadêmica. Quando essa tarefa de
produção precisa ser desenvolvida por graduandos e estes normalmente não
recebem orientação para tal, muitas vezes, recorrem a materiais digitais
sobre esse assunto, pois lhes propiciam as mais variadas estratégias de
ensino de acordo com o ritmo e as preferências do estudante \cite{falkembach2005}. Um fator que pode justificar essa recorrência é a facilidade de
acesso a plataformas digitais, que disponibilizam, na maioria das vezes
de forma gratuita, conteúdos digitais educacionais. Antigamente, os
estudantes consultavam manuais impressos que ensinavam a como produzir
textos acadêmicos, hoje, frente aos recursos tecnológicos, os locais de
aprendizagem se ampliam para a cibercultura. Como cibercultura,
compreendem-se vários ambientes da esfera digital que abrigam
informações, até mesmo os que simulam uma sala de aula a partir de
vídeos \cite{martins2018,rocha2005}.

Tendo a cibercultura se tornado uma potencializadora de novas abordagens
educativas, deve-se averiguar sua eficiência enquanto ferramenta de
ensino, a forma como se ensina determinados conteúdos, a exemplo da
produção textual de artigo acadêmico e seus aspectos constitutivos, foco
do presente estudo. Nesse sentido, traçamos dois objetivos para este
trabalho: identificar e analisar objetos de ensino explorados em
videoaulas sobre artigo acadêmico publicadas na plataforma YouTube.

Para tanto, organizamos este artigo em 5 seções, a saber: esta
introdução, contendo uma contextualização inicial sobre o objeto de
investigação da pesquisa, a problemática que o envolve e os objetivos
delineados; o embasamento teórico, no qual apresentamos os pressupostos
que fundamentam o estudo --- as práticas de ensino de Língua Portuguesa
em contexto didático-digital \cite{laurentino2023}, o artigo
acadêmico \cite{motta-roth2010}, as etapas de produção textual
(Antunes, 2003) e os objetos de ensino \cite{linodearaujo2014}; a
metodologia, na qual explicitamos a abordagem e o tipo de pesquisa, bem
como os procedimentos de coleta e análise de dados; os resultados,
contendo a exploração dos objetos de ensino contemplados nas videoaulas
sobre ensino de produção de artigo acadêmico; as considerações finais,
nas quais sinalizamos algumas implicações advindas dos resultados
alcançados.
\section{Aporte Teórico}\label{sec-aporteteo}

\subsection{Atenção}\label{sec-atencao}

Quando se fala de aprendizagem, há construtos que são essenciais e precisam ser
considerados, pois, sem eles, ela não ocorre. Um desses construtos é a atenção.
Os mecanismos atencionais estão presentes em diversas espécies e resolvem um
problema muito comum: o congestionamento de informações. A todo momento, nosso
cérebro é bombardeado por estímulos de natureza diversa, mas é impossível que
todas elas sejam registradas devido aos recursos limitados de nosso cérebro
\cite{dehaene2022,james1890principles,lima2005,posner1990}. Dessa forma, a atenção é um
mecanismo de extrema importância, pois atua como um filtro e executa a triagem
seletiva dessa ampla gama de estímulos. Logo, em cada estágio do processamento
de informação, nosso cérebro decide quanta importância deve dar a cada estímulo
recebido e aloca recursos somente às informações consideradas mais essenciais.

A atenção é amplamente estudada por diversas áreas do conhecimento e pode ser
mensurada em diferentes níveis, como o anatômico e o cognitivo. Na
Neurociência, ela é um campo de estudo abrangente e pode ser definida de
diferentes formas, sobretudo, como um processo cognitivo complexo que envolve
várias redes neurais e funções cerebrais. \textcite{posner1971,posner1990}
define atenção como um sistema integrado de controle do processamento mental
dividido em três grandes subsistemas: alerta, orientação e atenção executiva. O
alerta nos indica o momento em que devemos permanecer vigilantes, ou seja,
quando é necessário direcionar nossa atenção a um estímulo específico. O de
orientação está associado à escolha dos objetos aos quais devemos dedicar nossa
atenção. O da atenção executiva determina como a informação em questão será
processada, estando ligado à limitação da capacidade de processamento no
sistema de controle executivo, o qual guia, direciona e supervisiona nossos
processos mentais.

Na Linguística, a atenção é mais enfatizada em descrições cognitivas de
desenvolvimento da LE, sobretudo, nas abordagens psicolinguísticas,
sendo a atenção ao \emph{input} vista como essencial para o
armazenamento da informação e para a formação e a testagem de hipóteses
sobre a língua. Nos estudos sobre a ASL, ela é considerada um construto
necessário para que o \emph{input} seja convertido em \emph{intake} e se
torne disponível para posteriores processamentos mentais.

\textcite{schmidit1990,schmidt1993awareness,schmidit2001} investiga a relação entre atenção, processos
controlados (como a consciência) e aprendizagem. Fundamentado nos
princípios da psicologia cognitiva e da neurociência, ele argumenta que
a atenção exerce controle sobre o acesso à consciência. Assim, quando
uma pessoa direciona sua atenção para um estímulo, ela se torna
consciente dele. Para o autor, o grau de consciência do aluno se divide
em três níveis cognitivos fundamentais: 1) percepção
(\emph{perception}), relacionada à representação e à organização mental
dos eventos externos; 2) registro cognitivo (\emph{noticing}), isto é,
notar conscientemente os aspectos formais da língua alvo; e 3)
compreensão (\emph{understanding}), relacionada à reflexão acerca dos
aspectos linguísticos notados conscientemente e à tentativa de
compreendê-los.

A partir disso, ele postula a \emph{Noticing Hypothesis}, segundo a
qual, para que o aprendizado ocorra, o aprendiz deve alocar,
conscientemente, sua atenção aos aspectos linguísticos presentes no
\emph{input} da LE para registrá-los cognitivamente (\emph{noticing}) e
compreendê-los. Logo, percebe-se que a atenção é indispensável para a
transformação do \emph{input} em \emph{intake} durante a aprendizagem de
uma LE, visto que sem ela, não há \emph{noticing} e, consequentemente,
não há aprendizado. Assim, o que acontece no espaço atencional determina
significativamente o curso da aprendizagem e do desenvolvimento da
língua. Contudo, um fator determinante para que o aluno se envolva com a
atividade e, desse modo, direcione melhor seu foco atencional para o
objeto de aprendizagem é a motivação, melhor discutida a seguir.

\subsection{Motivação}\label{sec-motivacao}

A motivação é um componente essencial no emprego dos recursos
individuais para alcançar determinado objetivo. Assim, um indivíduo
motivado apresenta um comportamento ativo e empenhado no processo de
aprendizagem, sentindo a necessidade de aprender e atribuir significado
àquilo que foi aprendido, o que reforça o papel central da motivação no
processo de ensino-aprendizagem, despertando o interesse de diversos
pesquisadores e professores pelo tema, sobretudo na área de ASL.
Pesquisas indicam que a motivação tem um impacto direto na frequência
com que os alunos utilizam estratégias de aprendizagem em LE, na
intensidade de interação com falantes nativos, na quantidade de
\emph{input} recebido na língua-alvo, no desempenho em testes, no nível
de proficiência alcançado e na duração das habilidades linguísticas após
a conclusão dos estudos (Christopher, 1986; Gardner, 1992;Robin; Oxford, 1992 \emph{apud} \textcite{oxford1994language}).

A Teoria Motivacional de \textcite{gardner1979} tem sido um alicerce para o
desenvolvimento de estudos sobre a motivação no processo de aprendizagem
de LE. Ele postula que a motivação é responsável, entre outras coisas,
pela atenção ao conteúdo ministrado em sala de aula e pela prática do
conteúdo aprendido em contextos informais (fora da sala de aula).
Segundo essa teoria, o sucesso na aprendizagem de LE se baseia na
cognição individual, que envolve inteligência e aptidão linguística, e
em variáveis motivacionais. \textcite{gardner2010} argumenta que a motivação é
essencial para a aprendizagem da LE, visto que alunos motivados aprendem
mais, pois, estando motivados, prestam mais atenção, o que resulta em
mais \emph{noticing}, e isso faz com que níveis mais altos de
consciência e resultados mais significativos de aprendizagem sejam
alcançados. O autor também destaca que a motivação resulta da combinação
de quatro elementos: objetivo, desejo de alcançar o objetivo, atitudes
positivas diante da aprendizagem da língua e esforço.

Além da atenção e da motivação, fatores essenciais para que a
aprendizagem ocorra, deve-se também considerar recursos tecnológicos
enquanto potenciais ferramentas pedagógicas, visto que podem promover
atenção e motivação nos alunos e auxiliar a aprendizagem. Nesse sentido,
um recurso importante é a multimodalidade, melhor discutida a seguir.

\subsection{Multimodalidade}\label{sec-multimodalidade}

Com a evolução das tecnologias digitais e o advento e a relativa
democratização de dispositivos móveis, o uso de aparatos tecnológicos
com fins pedagógicos se torna cada vez mais viável e diversificado, o
que pode favorecer o ensino e a aprendizagem através da multimodalidade
presente em ambientes digitais. Dessa forma, é pertinente entendermos
melhor como ela pode ser integrada ao ensino de LE, mais especificamente
de vocabulário.

A multimodalidade é um dos componentes da hipermodalidade, que combina
hipertextos com recursos multimodais. Essa combinação permite apresentar
informações de várias formas e organizar o conteúdo de maneira não
linear, integrando diferentes modalidades e promovendo a interação entre
o aprendiz e o material pedagógico \textcite{braga2004}. Nesse sentido, neste
estudo, enfocaremos a multimodalidade para investigar o potencial de
ambientes imersivos como mediadores do processo de ensino e
aprendizagem, entendendo que ela está presente também em ambientes
hipermodais. \textcite{leeuwen2015} define a multimodalidade como um
fenômeno, segundo o qual quase todo discurso é multimodal. Ele
compreende a multimodalidade como a combinação e a integração de
diferentes modos semióticos em determinada instância ou tipo de
discurso.

A multimodalidade, ao permitir que o aluno faça a seleção, a conexão e a
relação entre as informações multimodais apresentadas, partindo de seus
interesses, necessidades e estilos cognitivo e de aprendizagem, pode
potencializar a aprendizagem de LE. Logo, ambientes multimodais de
aprendizagem podem motivar os alunos e fazer com que engajem mais nas
atividades, tornando esse processo mais prazeroso e promovendo uma
aprendizagem mais significativa. Nessa direção, surgem modelos e teorias
de aprendizagem multimodal, sendo os cognitivistas mais comumente
utilizados na investigação desse processo, principalmente aqueles
mediados por computadores e dispositivos móveis.

Entre as teorias e os modelos cognitivistas de processamento da
informação na literatura, destaca-se a Teoria Cognitiva de Aprendizagem
Multimídia\footnote{Em sua teoria, \textcite{mayer2001} utiliza o termo
\emph{multimídia} em vez de \emph{multimodalidade} para se referir ao
uso simultâneo de palavras e imagens. Porém, é comum vermos na
literatura esses termos sendo utilizados como sinônimos. O próprio
Mayer, em trabalho posterior \cite{moreno2007}, ao passar a usar
o termo \emph{multimodalidade} em vez de \emph{multimídia}, sugere que
esses termos podem ser utilizados de forma intercambiável. No entanto,
neste trabalho, priorizaremos o uso do termo \emph{multimodalidade}
por entendermos mídias como ``simplesmente meios, isto é, suportes
materiais, canais físicos, nos quais as linguagens se corporificam e
pelos quais transitam.'' \cite[p.~77]{santaella2007linguagens}, enquanto entendemos
modalidades como subjacentes à multiplicidade dos sistemas sígnicos, a
partir das quais, por processos de combinações e misturas de
diferentes modos (verbais e não verbais), originam-se diferentes
formas de linguagem e processos de comunicação.} \cite{mayer2001}, a qual
sugere que o processamento da informação ocorre em dois sistemas
separados: o visual e o verbal. Porém, essa teoria, apesar de possuir
orientação cognitivista e ter sido inicialmente desenvolvida para
facilitar a compreensão de informações científicas, tem sido amplamente
aplicada, por conta de sua natureza abrangente, na compreensão e na
elaboração de materiais instrucionais em outras disciplinas e áreas do
conhecimento, como a ASL, mais especificamente, na aprendizagem de
vocabulário de LE mediada por tecnologias digitais \cite{souza2004,procopio2016,monteiro2021}.

Segundo o modelo de aprendizagem proposto por \textcite{mayer2001}, palavras
entram no sistema cognitivo por meio dos ouvidos (em caso de palavras
faladas) e imagens entram por meio dos olhos. Na seleção de palavras e
imagens, o aprendiz presta atenção a determinadas palavras e aspectos da
imagem, o que desencadeia a construção do som da palavra e da imagem na
memória de trabalho. Na organização, o aprendiz mentalmente organiza as
palavras e as imagens selecionadas em representações coerentes na
memória de trabalho, o que o autor chama, respectivamente, de modo
verbal e modo imagético. Por fim, na integração, o aprendiz mentalmente
integra esses dois modos ao conhecimento prévio na memória de longo
prazo. Assim, ocorre o aprendizado da informação multimodal à qual o
aprendiz foi exposto. Logo, a teoria propõe que se aprende melhor com
palavras e imagens que apenas com palavras, pois utilizar dois canais
para apresentar a informação (verbal e visual) é como apresentar o
material duas vezes, expondo o aluno duas vezes mais ao conteúdo alvo.

Considerando o papel central da atenção e da motivação no ensino e
aprendizagem de LE, além do uso de tecnologias digitais como mediadoras
desse processo, nossa hipótese é que a multimodalidade presente em
ambientes digitais é um recurso com grande potencial para promover e
facilitar a aprendizagem. Ela pode tornar o processo de aprendizagem
mais agradável e eficiente ao motivar os alunos durante as atividades e
ajudá-los a direcionar melhor sua atenção ao objeto de aprendizagem para
registrá-lo cognitivamente (\emph{noticing}) e, consequentemente,
aprendê-lo. Porém, não são encontrados na literatura modelos e teorias
de ASL que considerem em suas abordagens a atenção e a motivação de
forma integrada e como elementos centrais do processo de aprendizagem de
LE, pois esses construtos, quando abordados, o são de forma periférica e
não integrada \cite{krashen1985,ellis1997,gass1997,hede2002,mayer2002,moreno2007,saito2015,monteiro2021}. Logo,
surge a necessidade de modelos e teorias que integrem esses construtos
caros à aprendizagem de línguas e os considere como centrais nesse
processo.

% !TeX root = main.tex

\section{Metodología}\label{sec-metodología}

La metodología de la investigación la entendemos como el conjunto de
procedimientos y técnicas que el equipo investigador ha utilizado en el
diseño, desarrollo y análisis del estudio. En este caso concreto, el
método utilizado ha sido de corte mixto, utilizando técnicas
cualitativas y cuantitativas. Las cualitativas se han basado en la
etnografía virtual de los datos generados en los sNOOC y las
conclusiones del juicio del equipo de expertos. Las cuantitativas
provienen de los cuestionarios de satisfacción del alumnado y de los
datos de interacción del alumnado en la plataforma de aprendizaje.


\subsection{Objetivos e hipótesis}\label{sub-sec-objetivosehipotesis}

El objetivo general de este estudio es analizar el proceso de creación
de redes comunicativas de estudiantes para la implementación de sNOOC
como método de evaluación continua en la UAD y su repercusión en la capa
social como modelo de formación mediática en personas de la tercera
edad. Con base en este objetivo general, los objetivos específicos hacen
referencia a:

\begin{itemize}
\item
Objetivo Específico 1 (OE1): Investigar las percepciones y opiniones
de las redes comunicativas de estudiantes respecto a la utilidad y
efectividad de un sNOOC como método de evaluación continua y su
impacto en la motivación hacia el aprendizaje.
\item
Objetivo Específico 2 (OE2): Examinar el proceso de desarrollo de las
redes comunicativas de estudiantes para la creación de contenidos de
los sNOOC, centrándose en el impacto del uso de pedagogías inclusivas,
IA y Metaverso en EAD.
\item
Objetivo Específico 3 (OE3): Evaluar el nivel de implicación activa de
las redes comunicativas de estudiantes en la plataforma de la UNED y
en la creación colaborativa del sNOOC en tmooc.es.
\end{itemize}

A continuación, se formulan las hipótesis para dar respuesta a las
relaciones causales:

\begin{itemize}
\item
Hipótesis 1 (H1-OE1): Si las redes comunicativas de estudiantes
perciben el sNOOC como una herramienta efectiva y útil para la
evaluación continua, aumentará su motivación intrínseca hacia el
aprendizaje y su participación en las actividades y recursos del
itinerario de aprendizaje propuesto.
\item
Hipótesis 2 (H2-0E2): Si el modelo sNOOC es diseñado y aplicado
considerando criterios pedagógicos inclusivos y herramientas
tecnológicas adecuadas, mejorará la comprensión de los contenidos por
parte de las redes comunicativas de estudiantes, incrementando su
satisfacción general con la experiencia de EAD.
\item
Hipótesis 3 (H3-OE3): Si el itinerario de aprendizaje en sNOOC está
basado en pedagogías inclusivas, se incrementará el compromiso activo
de las personas participantes, reflejado en una mayor interacción,
colaboración en equipo y corresponsabilidad en la construcción
colectiva del conocimiento.
\end{itemize}


\subsection{Muestra, instrumentos y análisis de
	datos}\label{sub-sec-muestrainstrumentos}
	
	El objeto de estudio de esta investigación son las interacciones del
	alumnado en la plataforma ALF de la UNED, contando con la participación
	de 79 personas, 57 mujeres y 22 hombres; 1 de nacionalidad croata y, el
	resto, española. Estos participantes han sido estudiantes del Máster
	Universitario en Educación y Comunicación en la Red y, dentro de este,
	de la asignatura ``Escenarios Virtuales para la participación'', una
	disciplina con contenidos relacionados con la educación mediática. En
	este caso concreto, para estructurar el método cuantitativo, se han
	utilizado los cuestionarios con preguntas diseñadas para recopilar datos
	cuantitativos correspondiente al curso 2023/2024.
	
	Referido a la plataforma ``tmooc.es'' donde este grupo de estudiantes
	creó los sNOOC, se ha realizado un análisis de estas propuestas tomando
	también esos entornos como objeto de estudio. Se tuvieron en cuenta los
	registros de datos relacionados con la dedicación en la creación de los
	sNOOC. Los sNOOC seleccionados son los siguientes: ``Introdúcete al
	mundo de Facebook'' (sN1), ``Senior 3.0'' (sN2), ``Correo electrónico
	son misterios: alfabetización digital para personas mayores'' (sN3),
	``Enredados en la edad dorada: dominar Facebook e Instagram con
	confianza'' (sN4), ``Healthy seniors network'' (sN5), ``Estas a un clic
	de conocer el mundo digital'' (sN6), ``Familias y aprendizaje en red''
	(sN7) y ``Google e inteligencia artificial, tus compañeros digitales''
	(sN8). En cuanto al enfoque cualitativo, se consideraron los datos
	generados a través de los sNOOC y las conclusiones del juicio de 22
	personas expertas internacionales, con el fin de validar hipótesis y
	evaluar riesgos o problemáticas presentes en el proyecto formativo. Para
	analizar los datos cuantitativos y cualitativos se utilizaron los
	programas SPSS y Atlas.ti, respectivamente. Estos aspectos se han
	organizado en categorías que se ajustan a las dimensiones de la
	educación inclusiva.

\section{Análise e discussão de dados}\label{sec-analiseediss}

Nos experimentos 1, 2 e 3 os grupos de alunos foram expostos,
respectivamente, ao ambiente imersivo em 360º ao ambiente de leitura com
glossário multimodal e ao ambiente imersivo em 360º + ambiente de
leitura com glossário multimodal. A seguir, esses experimentos serão
melhor discutidos.

\subsection{Experimento 1}\label{sub-sec-experimento1}

Neste experimento, 27 alunos foram expostos ao ambiente imersivo em
360º. Primeiramente, eles utilizaram os \emph{desktops} da escola para
explorar livremente o ambiente. Posteriormente, eles fizeram um
\emph{tour} guiado pela pesquisadora pelas cenas do ambiente, durante o
qual a pesquisadora narrou, interativamente, a peça shakespeariana,
fazendo perguntas aos alunos e direcionando sua atenção para objetos nas
cenas. Tais objetos são relevantes tanto para a história, por se
relacionarem a aspectos centrais da trama, quanto para o experimento,
por serem palavras alvo da atividade de aquisição lexical. Essa narração
interativa foi usada tanto para guiar a exploração dos alunos quanto
para ajudá-los a focar sua atenção nos objetos presentes no
ambiente/palavras alvo do experimento. Destaca-se que as palavras alvo
apareceram no ambiente na modalidade visual, sem apoio do verbal.

Simultaneamente à exploração guiada, os alunos responderam às perguntas
de compreensão, outra forma de salientar as palavras alvo e
apresentá-las aos alunos na modalidade escrita. Assim, cada item lexical
foi apresentado repetidas vezes e através de três modalidades
diferentes: visual, sonora e verbal. Esse reforço dos estímulos é
relevante, pois, como apontam pesquisas na área, apresentar o estímulo
em diferentes modalidades contribui significativamente para o
aprendizado e retenção do vocabulário da LE \cite{chun1996,saito2015,procopio2016,mayer2001,monteiro2021}.

Os resultados obtidos por meio dos testes de vocabulário aplicados aos
alunos antes e após a exposição ao ambiente são evidenciados no \Cref{graph-01}.

\begin{figure}[htpb]
    \centering
    \begin{minipage}{.75\textwidth}
    \includegraphics[width=\textwidth]{graph-01.png}
    \caption{Resultado dos Testes de vocabulário do Grupo 1}.
    \label{graph-01}
    \source{Elaborado pelas autoras (2024).}
    \end{minipage}
\end{figure}

Detalhando o gráfico, com base nos dados coletados, percebe-se que o
grupo de alunos exposto ao ambiente de aprendizagem imersivo em 360º
mostrou um ganho de aprendizagem para a maioria das palavras testadas.
Observa-se, entretanto, que algumas palavras tiveram ganho mais
expressivo de aprendizagem após a exposição ao ambiente, como
\emph{balcony, masquerade ball, dagger, banished} e \emph{tomb}
(respectivamente, 44\%, 37\%, 29\%, 37\% e 40\%), enquanto outras
apresentaram um ganho menos expressivo, como \emph{diary, poison} e
\emph{potion} (respectivamente, 4\%, 4\% e 7\%), ou nenhum ganho de
aprendizagem (\emph{fight}, \emph{quill pen} e \emph{enemies}). Essa
diferença pode ser atribuída à importância de cada palavra na narrativa,
pois, apesar de todas serem palavras-chave da tragédia, as que
apresentaram maior ganho de aprendizagem se relacionam diretamente aos
momentos de maior destaque/tensão da narrativa, a exemplo da famosa cena
de Romeo e Julieta na sacada (\emph{balcony}) e do final trágico do
casal na tumba (\emph{tomb}) da família Capuleto.

Conjectura-se que o relativamente baixo ganho de aprendizagem deste
grupo se deve à ausência do verbal (palavra escrita) no ambiente, visto
que apenas a narração não é suficiente para que alunos de nível
elementar consigam inferir o significado das palavras a partir do
contexto e relacioná-las aos objetos presentes no ambiente. Ademais, a
atividade de compreensão realizada durante a exploração do ambiente
também não foi eficaz devido à falta de simultaneidade na apresentação
das modalidades verbal e visual, o que contradiz o Princípio da
Contiguidade \cite{mayer2002} para a elaboração de materiais multimodais.
Segundo este princípio há uma aprendizagem mais profunda quando palavras
e imagens são apresentadas simultaneamente.

Quanto à análise qualitativa, o questionário deste grupo contou com 10
assertivas que visavam avaliar a experiência dos alunos com a atividade,
considerando fatores como usabilidade, engajamento, motivação, atenção e
aprendizagem multimodal. As respostas obtidas podem ser observadas no
\Cref{graph-02}.

\begin{figure}[htpb]
    \centering
    \begin{minipage}{.75\textwidth}
    \includegraphics[width=\textwidth]{graph-02.png}
    \caption{Questionário de avaliação da experiência do grupo 1.}
    \label{graph-02}
    \source{Elaborado pelas autoras (2024).}
    \end{minipage}
\end{figure}

A partir das assertivas 1 e 2 (``Usar e interagir com o ambiente foi
fácil para mim'' e ``Aprendi a usar o ambiente de forma rápida e
fácil'') comprova-se a usabilidade do ambiente, visto que 85\% e 92\%
dos alunos, respectivamente, concordaram com elas. Quanto à
aceitabilidade, 85\% concordaram com a assertiva 3 (``A experiência de
aprender vocabulário com ambiente imersivo foi gratificante para mim'').
Quanto ao envolvimento, 41\% afirmaram sentir-se envolvidos durante a
aprendizagem lexical no ambiente, concordando com a assertiva 4 (``Eu me
senti tão envolvido aprendendo vocabulário com o ambiente imersivo que
ignorei tudo ao meu redor''). Esses resultados evidenciam que o ambiente
é fácil e intuitivo de usar, além de favorecer ao engajamento dos alunos
nas atividades. Ele contribui também na criação de um espaço mais
receptivo e motivador para o aprendiz.

As assertivas 5 e 6 averiguaram o direcionamento do foco atencional
proporcionado pelo ambiente. 59\% dos alunos concordaram, 26\% não
souberam responder e 15\% discordaram da assertiva 5 (``O ambiente
ajudou a focar minha atenção nas informações relevantes para a
aprendizagem do vocabulário''). Isso aponta para o potencial do ambiente
enquanto recurso pedagógico, pois ajuda a direcionar a atenção dos
alunos ao objeto de aprendizagem, condição necessária para a ocorrência
do \emph{noticing} (registro cognitivo) e, consequentemente, do
aprendizado \cite{schmidit1990}. Ainda, através da assertiva 6 (``A narração
da professora durante a exploração do ambiente me ajudou a focar a
atenção nos objetos relacionados ao vocabulário''), verificou-se a
pertinência da narração durante o \emph{tour} guiado enquanto estratégia
essencial para ajudar os participantes a focarem sua atenção no objeto
de aprendizagem, como afirmaram 85\% dos alunos.

As assertivas 7 e 8 (``Eu gostei de estudar/aprender vocabulário com
esse ambiente'' e ``Eu me senti mais motivado aprendendo vocabulário com
esse ambiente que com materiais mais tradicionais (ex. livro
didático)''), comprovam a satisfação dos alunos com a aprendizagem
lexical mediada pelo ambiente. Os dados mostram 88\% dos alunos
afirmaram ter gostado de aprender vocabulário através do ambiente e 74\%
concordaram que ele motiva mais que materiais didáticos tradicionais.
Conjectura-se que a multimodalidade presente no ambiente, através da
interatividade e da novidade, é responsável por fazer com que os alunos
se sintam mais motivados em realizar as atividades e, assim, aprender a
LE.

As assertivas 9 e 10 (``Esse ambiente me permitiu compreender melhor o
significado das palavras, podendo ver objetos e relacioná-los à
história'' e ``O ambiente imersivo e os recursos utilizados durante sua
exploração foram suficientes para que eu aprendesse o vocabulário'')
corroboram a efetividade da aprendizagem multimodal, pois 74\% dos
alunos concordaram com a assertiva 9 e 67\%, com a 10, o que ratifica o
papel da multimodalidade como facilitadora do processo de ensino e
aprendizagem, pois a apresentação do material por meio de diferentes
modalidades facilita a compreensão do vocabulário alvo \cite{mayer2001}.

\subsection{Experimento 2}\label{sub-sec-experimento2}

Neste experimento, 27 alunos foram expostos ao ambiente de leitura com
glossário multimodal. Eles ficaram livres para explorá-lo, podendo ler
informações essenciais dos personagens principais e o hipertexto da peça
teatral Romeo e Julieta, cujos \emph{links} (palavras-alvo do
experimento) lhes direcionavam ao glossário multimodal, onde podiam ver
a representação em 2D, a pronúncia e o significado das palavras-alvo.
Não foi feita nenhuma intervenção durante a exploração desse ambiente,
deixando os alunos totalmente livres para navegá-lo e aprender o
vocabulário da forma como preferissem. Os resultados obtidos nos testes
de vocabulário deste experimento podem ser vistos no \Cref{graph-03}.

\begin{figure}[htpb]
    \centering
    \begin{minipage}{.75\textwidth}
    \includegraphics[width=\textwidth]{graph-03.png}
    \caption{Resultado dos Testes de vocabulário do Grupo 2.}
    \label{graph-03}
    \source{Elaborado pelas autoras (2024).}
    \end{minipage}
\end{figure}

Semelhantemente ao experimento 1, o grupo de alunos expostos ao ambiente
de leitura com glossário multimodal também mostrou ganho de aprendizagem
para quase todas as palavras, com exceção de \emph{fight,} e
\emph{enemies}, cujo conhecimento dos alunos antes e após a exposição se
manteve o mesmo. Contudo, percebe-se que, no geral, o ganho de
aprendizagem por palavra foi mais expressivo neste grupo. Por exemplo,
as palavras \emph{quill pen, masquerade ball} e \emph{tomb} apresentaram
um ganho de, respectivamente, 52\%, 52\% e 67\% no experimento 2, mas de
0\%, 38\% e 38\% no grupo 1. Pode-se atribuir essa diferença à presença
da multimodalidade neste ambiente, que se mostra mais eficaz para a
aprendizagem de alunos de proficiência elementar. Isso ocorre porque
este ambiente oferece o apoio do hipertexto e do glossário multimodal,
possibilitando a visualização da forma escrita das palavras-alvo, de seu
significado e imagens correspondentes. Assim, destaca-se a relevância da
apresentação simultânea de palavras escritas e imagem para a
aprendizagem de vocabulário \cite{mayer2001}, visto que as modalidades
verbal e visual são apresentadas aos alunos de forma simultânea.

Já questionário de avaliação da experiência deste grupo possuía 11
assertivas e foi adaptado para contemplar as características do ambiente
de leitura, mas o objetivo foi o mesmo: avaliar a experiência dos
alunos, considerando fatores como usabilidade, engajamento, motivação e
atenção para responder à segunda pergunta desta pesquisa. Os resultados
obtidos podem ser verificados no \Cref{graph-04}.

\begin{figure}[htpb]
    \centering
    \begin{minipage}{.75\textwidth}
    \includegraphics[width=\textwidth]{graph-04.png}
    \caption{Questionário de avaliação da experiência grupo 2.}
    \label{graph-04}
    \source{Elaborado pelas autoras (2024).}
    \end{minipage}
\end{figure}

Percebe-se que fator usabilidade deste ambiente foi corroborado pelos
alunos, pois 81\% concordaram com a assertiva 1 (``Usar e interagir com
o ambiente foi fácil para mim.'') e 74\%, com a assertiva 2 (``Aprendi a
usar o ambiente de forma rápida e fácil.''). Em relação à
aceitabilidade, 70\% concordaram que a experiência de aprender
vocabulário com o ambiente foi gratificante, fator averiguado através da
assertiva 3 (``A experiência de aprender vocabulário com o glossário
multimodal foi muito gratificante para mim.''). Quanto ao envolvimento
com o ambiente, 22\% concordaram, 52\% não souberam responder e 26\%
discordaram da assertiva 4 (``Eu me senti tão envolvido aprendendo
vocabulário com o glossário multimodal que ignorei tudo ao meu redor).

Quanto à atenção focada, 70\% concordaram com a assertiva 5 (``O
ambiente de leitura com glossário multimodal ajudou a focar minha
atenção nas informações relevantes para a aprendizagem do
vocabulário.''). Assim, evidencia-se que o ambiente ajuda o aluno a
direcionar sua atenção ao objeto de aprendizagem, o que, segundo a
\emph{Noticing Hypothesis} \cite{schmidit1990}, é necessário para ocorrer o
\emph{noticing}. Logo, o ambiente se mostra um recurso pedagógico de
relevância, pois ajuda a promover o registro cognitivo das informações,
promovendo a aprendizagem.

Ademais, 85\% concordaram com a assertiva 6 (``Eu gostei de
estudar/aprender vocabulário com o glossário multimodal.''). Isso aponta
para a agradabilidade proporcionada pelo ambiente \cite{schumann1999neurobiology}, a
qual está relacionada a fatores emocionais ligados à aprendizagem da LE,
como a motivação. Essa foi melhor averiguada através da assertiva 7
(``Eu me senti mais motivado estudando vocabulário com o glossário
multimodal que com materiais tradicionais (ex.: livro didático)''), com
a qual 63\% dos alunos concordaram, 22\% não souberam responder e 15\%
discordaram. Isso mostra que o ambiente possui mais potencial para
motivar os alunos durante a aprendizagem de vocabulário que materiais
didáticos tradicionais/lineares. Atribuímos isso à multimodalidade
presente no ambiente, um recurso que, por meio da interatividade e da
novidade, é capaz de fazer com que os alunos se sintam mais motivados
durante a atividade.

Quanto à aprendizagem multimodal, 92,5\% dos participantes concordaram
com a assertiva 8 (``O glossário me permitiu compreender melhor o
significado das palavras do texto por meio de imagens, pronúncias e
definições.''), evidenciando a eficácia da multimodalidade na
aprendizagem de vocabulário. Isso ocorre, pois a combinação de
diferentes modalidades no glossário permite melhor compreensão das
palavras-alvo, como defende a Teoria Cognitiva de Aprendizagem
Multimídia \cite{mayer2001}. Ademais, 48\% dos alunos concordaram com a
assertiva 9 (``O glossário multimodal foi suficiente para que eu
aprendesse o vocabulário.''), enquanto 30\% não souberam responder e
22\% discordaram. Isso mostra que, embora o glossário multimodal seja
eficaz, a utilização de recursos adicionais pode tornar a aprendizagem
lexical em ambientes multimodais ainda mais eficaz. Logo, surge a
necessidade de testar a eficácia da combinação de outras modalidades
além das utilizadas neste ambiente, o que será averiguado no terceiro
experimento.

Por questões técnicas, não foi possível reproduzir a pronúncia das
palavras presentes no glossário. Por meio da assertiva 10 (``\emph{A
impossibilidade de ouvir a pronúncia das palavras no glossário não
prejudicou minha aprendizagem do vocabulário''}), verificou-se se isso
gerou prejuízos à aprendizagem das palavras. 48\% dos alunos afirmaram
que isso não prejudicou a aprendizagem do vocabulário, enquanto 33\%
discordaram dessa assertiva e 19\% não souberam responder. Porém, o som
é um recurso relevante para os aprendizes de nível elementar e, por
isso, em futuras utilizações deste ambiente de aprendizagem, deve haver
um empenho para que todas as modalidades estejam disponíveis para o
pleno uso dos alunos.

Por fim, na assertiva 11 (``Qual dos recursos presentes no glossário
mais contribuíram para sua aprendizagem do vocabulário?''), buscou-se
averiguar qual das modalidades presentes no glossário foi mais eficaz
para a aprendizagem do vocabulário. Assim, 55,5\% dos alunos apontaram
``palavras + imagem'' como o recurso que mais contribui para sua
aprendizagem no ambiente, enquanto 29,5\% apontaram a ``imagem'' e 15\%,
a ``definição da palavra''. Esse resultado também corrobora a Teoria
Cognitiva de Aprendizagem Multimídia \cite{mayer2001}, a qual defende que
se aprende melhor através de palavras e imagens que apenas de palavras.

\subsection{Experimento 3}\label{sub-sec-experimento3}

No experimento 3, 27 alunos foram expostos ao ambiente de aprendizagem
imersivo em 360º + ambiente de leitura com glossário multimodal.
Primeiramente, eles exploraram o ambiente de leitura, podendo, a
qualquer momento, clicar nos \emph{links} e consultar o significado, a
pronúncia e a representação em 2D das palavras-alvo no glossário
multimodal. Isso pode facilitar a compreensão do significado dessas
palavras e constitui outra forma de salientá-las para além das
repetições no texto, pois a saliência é um fator importante para a
retenção lexical na memória a longo prazo \cite{procopio2016,saito2015}.

Após essa livre exploração do ambiente de leitura, os participantes
foram direcionados para o ambiente imersivo em 360º, no qual fizeram um
\emph{tour} pelas cenas do ambiente guiados pela pesquisadora, que
retomou oralmente o enredo de Romeu e Julieta, salientando as
palavras-alvo e direcionando a atenção dos participantes para os objetos
em 360º, também palavras-alvo do experimento. Essa estratégia teve o
intuito de salientar os objetos, pois a plataforma não permite a
inserção das palavras-alvo na modalidade escrita e o destaque desses
objetos para ficarem mais salientes para os alunos. Os resultados
obtidos neste experimento podem ser vistos no \Cref{graph-05}.

\begin{figure}[htpb]
    \centering
    \begin{minipage}{.75\textwidth}
    \includegraphics[width=\textwidth]{graph-05.png}
    \caption{Resultado dos Testes de vocabulário do Grupo 3.}
    \label{graph-05}
    \source{Elaborado pelas autoras (2024).}
    \end{minipage}
\end{figure}

Com base nos resultados dos testes de vocabulário dos alunos expostos à
terceira condição de testagem, observa-se um ganho de aprendizagem mais
expressivo para a maioria das palavras em comparação aos dois grupos
anteriores, com destaque para as palavras \emph{dagger, orchard},
\emph{masquerade ball} e \emph{quill pen}, com ganhos de,
respectivamente, 100\%, 73\%, 66\%, 68\% e 63\%.

Esses resultados nos levam a acreditar que, quando os ambientes são integrados,
a aprendizagem lexical é potencializada. Isso se deve aos diferentes recursos
presentes em cada um deles, que, combinados, fazem com que mais modalidades
estejam disponíveis. Isso é corroborado por \textcite{lemke2002}, o qual
postula que o significado das diferentes modalidades é multiplicativo, fazendo
do todo mais que a simples soma das partes e contribuindo, assim, para a
realização de inferências e retenção do significado inferido na memória.
Destaca-se também o fator saliência, essencial para a retenção lexical
\cite{procopio2016,saito2015}, pois o grupo 3 foi exposto a cada uma das
palavras-alvo no mínimo três vezes por meio das modalidades verbal, escrita e
visual.

Quanto à análise qualitativa, devido uma greve docente ocorrida durante
a coleta de dados deste estudo, no início ano de 2024, e à consequente
paralização das atividades do colégio de aplicação em que o experimento
foi aplicado, o questionário de avaliação da experiência deste grupo
precisou ser reduzido e, por isso, diferiu-se dos aplicados aos grupos
anteriores, enfocando mais diretamente a atenção, a motivação e a
aprendizagem multimodal. Através da primeira pergunta discursiva (Você
gostou da atividade? Justifique sua resposta.), averiguou-se que o
ambiente promove a agradabilidade, pois todos os participantes afirmaram
ter gostado da atividade. Segundo \cite{schumann1999neurobiology}, a agradabilidade
promove motivação e engajamento do aluno, facilitando o processo de
aprendizagem. Logo, conclui-se que o ambiente é capaz de promover a
motivação nos alunos durante a aprendizagem de vocabulário.

Na segunda pergunta (Você acha que a atividade te ajudou a aprender
vocabulário?), buscou-se averiguar se a aprendizagem mediada pelo
ambiente é eficaz para a aquisição lexical. Semelhantemente à pergunta
anterior, todos os alunos afirmaram que a atividade ajudou na sua
aprendizagem de vocabulário. Vê-se que o ambiente é capaz de auxiliar os
alunos a direcionarem sua atenção ao objeto de aprendizagem durante a
realização da atividade, promovendo o aprendizado. Isso corrobora a
\emph{Noticing Hypothsis} \cite{schmidit1990}, segundo a qual, para que o
aprendizado ocorra, o aluno deve direcionar sua atenção para o objeto de
aprendizagem para registrá-lo cognitivamente (\emph{noticing}) e
aprendê-lo. Assim, conclui-se que, sem atenção, não há aprendizagem.
Logo, o ambiente, capaz de atrair e direcionar a atenção dos alunos,
mostra-se um recurso pedagógico eficiente para a aprendizagem de LE.

Na terceira pergunta (Qual dos itens abaixo mais contribuiu para o seu
aprendizado de vocabulário durante a atividade?), buscou-se averiguar
qual dos recursos utilizados no ambiente mais contribuiu para a
aprendizagem do vocabulário em LE: a narração do texto; o glossário; a
peça teatral ``Romeu e Julieta''; ou a exploração do ambiente guiada
pela pesquisadora. A opção a mais escolhida foi a exploração guiada do
ambiente (54\%), seguida pelo glossário (27\%), narração (13\%) e peça
teatral (6\%). Logo, conclui-se que o tour guiado pela pesquisadora,
além de suprir limitações do ambiente, mostrou-se um recurso muito
eficaz para a aprendizagem do vocabulário, pois orientou a exploração do
ambiente e direcionou a atenção dos alunos, possibilitando a ocorrência
do \emph{noticing} (registro cognitivo), condição necessária para que o
aprendizado ocorra \cite{schmidit1990}. Além disso, aponta-se a eficácia do
glossário multimodal enquanto recurso que auxilia a aquisição lexical
mediada por ambientes multimodais, pois, além de fornecer um suporte
escrito aos alunos, facilita a aprendizagem ao apresentar o vocabulário
através de diferentes modalidades, o que corrobora a Teoria Cognitiva de
Aprendizagem Multimídia \cite{mayer2001}, pois o uso de duas ou mais
modalidades possibilita que o aluno construa representações mentais mais
ricas e estabeleça relação entre elas, facilitando a aprendizagem.

\section{Conclusão}\label{sec-conclusao}

Buscamos investigar o potencial de um ambiente imersivo na aprendizagem de
vocabulário de inglês como LE, observando, em particular, sua capacidade de
promover atenção e motivação nos alunos. Para responder às perguntas de
pesquisa, foram realizados três experimentos com alunos de nível elementar de
proficiência em língua inglesa de um colégio de aplicação em Minas Gerais.

A partir dos resultados dos testes de vocabulário, verificou-se que houve ganho
de aprendizagem nas três condições de testagem, respondendo positivamente à
primeira pergunta de pesquisa: ``\emph{O ambiente imersivo em 360º contribui para a
aprendizagem de vocabulário de inglês como língua estrangeira?}''. Além disso,
verificou-se que o ambiente imersivo em 360º, ao proporcionar uma experiência
nova aos alunos através da imersão e da multimodalidade, é capaz de fazer com
que se sintam mais motivados e, assim, direcionem melhor sua atenção ao objeto
de aprendizagem \cite{gardner2010}, o que resulta em mais \emph{noticing} \cite{schmidit1990}
e possibilita o aprendizado do vocabulário, respondendo também afirmativamente
à segunda pergunta: ``\emph{O ambiente imersivo em 360º contribui para promoção da
motivação e da atenção durante o aprendizado do vocabulário de inglês como
LE?}''.

Ainda, verificou-se que o ganho de aprendizagem foi mais expressivo em uma
condição que em outras. Assim, calculou-se o ganho médio de aprendizagem
lexical para cada grupo. Os grupos expostos ao ambiente imersivo em 360º, ao
ambiente de leitura com glossário multimodal e ao ambiente imersivo em 360º +
ambiente de leitura com glossário multimodal apresentaram ganho médio de
aprendizagem de, respectivamente, 18\%, 28\% e 62\%. Logo, responde-se à terceira
pergunta de pesquisa (``\emph{Comparativamente, qual das três condições testadas –
ambiente imersivo em 360º, ambiente hipermodal de leitura com glossário ou
ambiente imersivo em 360º + ambiente hipermodal de leitura com glossário – é
mais eficiente para a aprendizagem de vocabulário em LE?}''), pois a condição
mais eficiente para a aprendizagem de vocabulário foi a terceira, com 62\%. Isso
evidencia a importância do verbal e da apresentação simultânea das diferentes
modalidades aos alunos de nível elementar \cite{procopio2016}, pois os grupos que
contaram com um suporte da escrita (grupos 2 e 3), simultaneamente às imagens,
obtiveram maior ganho de aprendizagem que o grupo que contava apenas com as
modalidades sonora e visual (grupo 1).

Conclui-se que o ambiente imersivo em 360º tem um grande potencial pedagógico,
pois, ao apresentar o estímulo de forma multimodal, contribui
significativamente para o aprendizado e a retenção do vocabulário da LE, como
mostra a literatura \cite{chun1996,saito2015,procopio2016,mayer2001,monteiro2021}.
Ademais, ele é uma alternativa mais acessível para a
inserção da RV em salas de aula de LE, pois é mais econômica e propicia a
interação do aprendiz com elementos virtuais de forma dinâmica, fazendo com que
ele engaje mais nas atividades propostas.

Contudo, algumas limitações precisam ser apontadas, como a impossibilidade de
reproduzir as pronúncias das palavras no glossário devido à falta de
dispositivos de reprodução de som nos computadores da escola, impedindo que os
alunos acessassem uma das modalidades presentes do ambiente (a sonora). Outra
limitação que precisa ser destacada é a impossibilidade de inserir um glossário
multimodal no ambiente imersivo em 360º, o que permitiria a consulta das
palavras alvo de forma simultânea à sua exploração, oferecendo um suporte
verbal durante a atividade.

Ainda, mais estudos precisam ser desenvolvidos para que o uso ambientes de RV
seja cada vez mais popularizado e integrado ao ensino de LE, estando cada vez
mais presente nas salas de aula brasileiras. Há também necessidade de modelos e
teorias de ASL que considerem em suas abordagens atenção e motivação de forma
integrada e como elementos centrais do processo de aprendizagem de línguas,
visto que esses construtos, quando abordados, o são de forma periférica e não
integrados em estudos encontrados na literatura \cite{krashen1985,ellis1997,gass1997,hede2002,mayer2002,moreno2007,saito2015,monteiro2021}.





\printbibliography\label{sec-bib}
%conceptualization,datacuration,formalanalysis,funding,investigation,methodology,projadm,resources,software,supervision,validation,visualization,writing,review
\begin{contributors}[sec-contributors]
\authorcontribution{Rafaela Lemos Sales}[conceptualization,formalanalysis,methodology,projadm,supervision,writing,review]
\authorcontribution{Patrícia Nora de Souza Ribeiro}[conceptualization,formalanalysis,investigation,methodology,projadm,supervision,review]
\end{contributors}
\end{document}
