\section{Introdução}\label{sec-intro}

A aquisição de línguas tem despertado o interesse de pesquisadores desde
a antiguidade. Na sociedade moderna, esse interesse tendeu, por um
tempo, a enfatizar o ensino em detrimento da aprendizagem. Isso,
gradativamente, foi se transformando devido, sobretudo, às evoluções
ocorridas na Linguística e aos debates promovidos pela psicologia quanto
à natureza da aprendizagem. Logo, a atenção de pesquisadores voltou-se
para a aprendizagem, introduzindo uma nova agenda no campo de estudos
linguísticos e definindo uma nova área de pesquisa: Aquisição de Segunda
Língua\footnote{Constantemente, encontramos na literatura uma distinção
entre os termos \emph{aprendizagem} e \emph{aquisição}. \textcite{krashen1985}
entende por aprendizagem um processo consciente e explícito que ocorre
por meio de instruções formais em salas de aula, já a aquisição é por
ele considerada um processo inconsciente e implícito que ocorre
naturalmente em ambientes em que a língua é falada, ou seja, fora da
sala de aula. É importante esclarecer que, neste trabalho, não será
feita uma distinção entre esses termos, sendo eles utilizados de forma
genérica e intercambiável.} (ASL). Porém, essa mudança de foco não
significa ignorar o papel da instrução no processo de ASL, visto que um
dos objetivos das pesquisas dessa área é facilitar esse processo através
de um melhor entendimento de como ele ocorre, assim como propor teorias
que tornem o ensino mais eficaz e agradável para os alunos.

Várias teorias de ASL foram desenvolvidas, mas nenhuma delas pode ser
considerada completa no sentido de conseguir dar conta de toda a
complexidade envolvida no processo de aprender uma segunda
língua\footnote{Na literatura, observa-se uma diferenciação entre os
termos segunda língua e língua estrangeira. \textcite{krashen1985} define
segunda língua como uma língua que esteja sendo aprendida em ambiente
nativo e língua estrangeira como uma língua que esteja sendo aprendida
fora desse ambiente. Neste estudo, não será feita distinção formal
entre esses termos, que também serão utilizados de forma genérica e
intercambiável, pois não nos interessa aprofundarmos essa discussão.
Destaca-se, entretanto, que o contexto de aplicação do experimento é o
do inglês como língua estrangeira.}. Ainda, considerando as
tecnologias digitais e a sua inserção no processo de ensino e
aprendizagem, pode-se perceber uma divisão entre teorias de ASL de
acordo com o ambiente de aprendizagem (linear ou não linear). Destaca-se
que as teorias de ASL encontradas na literatura raramente englobam, em
uma única abordagem, atenção e motivação, essenciais para a
aprendizagem, e, quando abordam, é de forma periférica. Este estudo
busca investigar o uso de um ambiente imersivo em 360º para promover a
atenção e a motivação nos alunos durante a aprendizagem de vocabulário
de inglês como LE. Para tanto, busca-se responder às seguintes
perguntas:

\begin{enumerate}[label=\arabic{*}.]
  \item O ambiente imersivo em 360º contribui para a aprendizagem de
  vocabulário de inglês como língua estrangeira?
  
  \item  O ambiente imersivo em 360º contribui para promoção da motivação e a
  da atenção durante o aprendizado do vocabulário de inglês como LE?
  
  \item Comparativamente, qual das três condições testadas -- ambiente
  imersivo em 360º, ambiente de leitura com glossário multimodal ou
  ambiente imersivo em 360º + ambiente de leitura com glossário multimodal
  -- é mais eficiente para a aprendizagem de vocabulário em LE?
  
\end{enumerate}

Para alcançar tal objetivo, inicialmente, são apresentadas as teorias
que fundamentam o trabalho, discutindo, nas \Cref{sec-atencao,sec-motivacao,sec-multimodalidade} % seções 2.1, 2.2 e 2.3, %Criar hyperlinks
respectivamente, a atenção através da \emph{Noticing Hypothesis}
\cite{schmidit1990}, a motivação com base na Teoria Motivacional de \textcite{gardner1979,gardner2010} e a multimodalidade por meio da Teoria Cognitiva
de Aprendizagem Multimídia \cite{mayer2001}. Na \Cref{sec-metodologia}, apresenta-se a
metodologia utilizada no trabalho e, na \Cref{sec-analiseediss}, a análise dos dados. Na
\Cref{sec-conclusao}, são feitas as considerações finais deste trabalho.
