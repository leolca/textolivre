\section{Aporte Teórico}\label{sec-aporteteo}

\subsection{Atenção}\label{sec-atencao}

Quando se fala de aprendizagem, há construtos que são essenciais e precisam ser
considerados, pois, sem eles, ela não ocorre. Um desses construtos é a atenção.
Os mecanismos atencionais estão presentes em diversas espécies e resolvem um
problema muito comum: o congestionamento de informações. A todo momento, nosso
cérebro é bombardeado por estímulos de natureza diversa, mas é impossível que
todas elas sejam registradas devido aos recursos limitados de nosso cérebro
\cite{dehaene2022,james1890principles,lima2005,posner1990}. Dessa forma, a atenção é um
mecanismo de extrema importância, pois atua como um filtro e executa a triagem
seletiva dessa ampla gama de estímulos. Logo, em cada estágio do processamento
de informação, nosso cérebro decide quanta importância deve dar a cada estímulo
recebido e aloca recursos somente às informações consideradas mais essenciais.

A atenção é amplamente estudada por diversas áreas do conhecimento e pode ser
mensurada em diferentes níveis, como o anatômico e o cognitivo. Na
Neurociência, ela é um campo de estudo abrangente e pode ser definida de
diferentes formas, sobretudo, como um processo cognitivo complexo que envolve
várias redes neurais e funções cerebrais. \textcite{posner1971,posner1990}
define atenção como um sistema integrado de controle do processamento mental
dividido em três grandes subsistemas: alerta, orientação e atenção executiva. O
alerta nos indica o momento em que devemos permanecer vigilantes, ou seja,
quando é necessário direcionar nossa atenção a um estímulo específico. O de
orientação está associado à escolha dos objetos aos quais devemos dedicar nossa
atenção. O da atenção executiva determina como a informação em questão será
processada, estando ligado à limitação da capacidade de processamento no
sistema de controle executivo, o qual guia, direciona e supervisiona nossos
processos mentais.

Na Linguística, a atenção é mais enfatizada em descrições cognitivas de
desenvolvimento da LE, sobretudo, nas abordagens psicolinguísticas,
sendo a atenção ao \emph{input} vista como essencial para o
armazenamento da informação e para a formação e a testagem de hipóteses
sobre a língua. Nos estudos sobre a ASL, ela é considerada um construto
necessário para que o \emph{input} seja convertido em \emph{intake} e se
torne disponível para posteriores processamentos mentais.

\textcite{schmidit1990,schmidt1993awareness,schmidit2001} investiga a relação entre atenção, processos
controlados (como a consciência) e aprendizagem. Fundamentado nos
princípios da psicologia cognitiva e da neurociência, ele argumenta que
a atenção exerce controle sobre o acesso à consciência. Assim, quando
uma pessoa direciona sua atenção para um estímulo, ela se torna
consciente dele. Para o autor, o grau de consciência do aluno se divide
em três níveis cognitivos fundamentais: 1) percepção
(\emph{perception}), relacionada à representação e à organização mental
dos eventos externos; 2) registro cognitivo (\emph{noticing}), isto é,
notar conscientemente os aspectos formais da língua alvo; e 3)
compreensão (\emph{understanding}), relacionada à reflexão acerca dos
aspectos linguísticos notados conscientemente e à tentativa de
compreendê-los.

A partir disso, ele postula a \emph{Noticing Hypothesis}, segundo a
qual, para que o aprendizado ocorra, o aprendiz deve alocar,
conscientemente, sua atenção aos aspectos linguísticos presentes no
\emph{input} da LE para registrá-los cognitivamente (\emph{noticing}) e
compreendê-los. Logo, percebe-se que a atenção é indispensável para a
transformação do \emph{input} em \emph{intake} durante a aprendizagem de
uma LE, visto que sem ela, não há \emph{noticing} e, consequentemente,
não há aprendizado. Assim, o que acontece no espaço atencional determina
significativamente o curso da aprendizagem e do desenvolvimento da
língua. Contudo, um fator determinante para que o aluno se envolva com a
atividade e, desse modo, direcione melhor seu foco atencional para o
objeto de aprendizagem é a motivação, melhor discutida a seguir.

\subsection{Motivação}\label{sec-motivacao}

A motivação é um componente essencial no emprego dos recursos
individuais para alcançar determinado objetivo. Assim, um indivíduo
motivado apresenta um comportamento ativo e empenhado no processo de
aprendizagem, sentindo a necessidade de aprender e atribuir significado
àquilo que foi aprendido, o que reforça o papel central da motivação no
processo de ensino-aprendizagem, despertando o interesse de diversos
pesquisadores e professores pelo tema, sobretudo na área de ASL.
Pesquisas indicam que a motivação tem um impacto direto na frequência
com que os alunos utilizam estratégias de aprendizagem em LE, na
intensidade de interação com falantes nativos, na quantidade de
\emph{input} recebido na língua-alvo, no desempenho em testes, no nível
de proficiência alcançado e na duração das habilidades linguísticas após
a conclusão dos estudos (Christopher, 1986; Gardner, 1992;Robin; Oxford, 1992 \emph{apud} \textcite{oxford1994language}).

A Teoria Motivacional de \textcite{gardner1979} tem sido um alicerce para o
desenvolvimento de estudos sobre a motivação no processo de aprendizagem
de LE. Ele postula que a motivação é responsável, entre outras coisas,
pela atenção ao conteúdo ministrado em sala de aula e pela prática do
conteúdo aprendido em contextos informais (fora da sala de aula).
Segundo essa teoria, o sucesso na aprendizagem de LE se baseia na
cognição individual, que envolve inteligência e aptidão linguística, e
em variáveis motivacionais. \textcite{gardner2010} argumenta que a motivação é
essencial para a aprendizagem da LE, visto que alunos motivados aprendem
mais, pois, estando motivados, prestam mais atenção, o que resulta em
mais \emph{noticing}, e isso faz com que níveis mais altos de
consciência e resultados mais significativos de aprendizagem sejam
alcançados. O autor também destaca que a motivação resulta da combinação
de quatro elementos: objetivo, desejo de alcançar o objetivo, atitudes
positivas diante da aprendizagem da língua e esforço.

Além da atenção e da motivação, fatores essenciais para que a
aprendizagem ocorra, deve-se também considerar recursos tecnológicos
enquanto potenciais ferramentas pedagógicas, visto que podem promover
atenção e motivação nos alunos e auxiliar a aprendizagem. Nesse sentido,
um recurso importante é a multimodalidade, melhor discutida a seguir.

\subsection{Multimodalidade}\label{sec-multimodalidade}

Com a evolução das tecnologias digitais e o advento e a relativa
democratização de dispositivos móveis, o uso de aparatos tecnológicos
com fins pedagógicos se torna cada vez mais viável e diversificado, o
que pode favorecer o ensino e a aprendizagem através da multimodalidade
presente em ambientes digitais. Dessa forma, é pertinente entendermos
melhor como ela pode ser integrada ao ensino de LE, mais especificamente
de vocabulário.

A multimodalidade é um dos componentes da hipermodalidade, que combina
hipertextos com recursos multimodais. Essa combinação permite apresentar
informações de várias formas e organizar o conteúdo de maneira não
linear, integrando diferentes modalidades e promovendo a interação entre
o aprendiz e o material pedagógico \textcite{braga2004}. Nesse sentido, neste
estudo, enfocaremos a multimodalidade para investigar o potencial de
ambientes imersivos como mediadores do processo de ensino e
aprendizagem, entendendo que ela está presente também em ambientes
hipermodais. \textcite{leeuwen2015} define a multimodalidade como um
fenômeno, segundo o qual quase todo discurso é multimodal. Ele
compreende a multimodalidade como a combinação e a integração de
diferentes modos semióticos em determinada instância ou tipo de
discurso.

A multimodalidade, ao permitir que o aluno faça a seleção, a conexão e a
relação entre as informações multimodais apresentadas, partindo de seus
interesses, necessidades e estilos cognitivo e de aprendizagem, pode
potencializar a aprendizagem de LE. Logo, ambientes multimodais de
aprendizagem podem motivar os alunos e fazer com que engajem mais nas
atividades, tornando esse processo mais prazeroso e promovendo uma
aprendizagem mais significativa. Nessa direção, surgem modelos e teorias
de aprendizagem multimodal, sendo os cognitivistas mais comumente
utilizados na investigação desse processo, principalmente aqueles
mediados por computadores e dispositivos móveis.

Entre as teorias e os modelos cognitivistas de processamento da
informação na literatura, destaca-se a Teoria Cognitiva de Aprendizagem
Multimídia\footnote{Em sua teoria, \textcite{mayer2001} utiliza o termo
\emph{multimídia} em vez de \emph{multimodalidade} para se referir ao
uso simultâneo de palavras e imagens. Porém, é comum vermos na
literatura esses termos sendo utilizados como sinônimos. O próprio
Mayer, em trabalho posterior \cite{moreno2007}, ao passar a usar
o termo \emph{multimodalidade} em vez de \emph{multimídia}, sugere que
esses termos podem ser utilizados de forma intercambiável. No entanto,
neste trabalho, priorizaremos o uso do termo \emph{multimodalidade}
por entendermos mídias como ``simplesmente meios, isto é, suportes
materiais, canais físicos, nos quais as linguagens se corporificam e
pelos quais transitam.'' \cite[p.~77]{santaella2007linguagens}, enquanto entendemos
modalidades como subjacentes à multiplicidade dos sistemas sígnicos, a
partir das quais, por processos de combinações e misturas de
diferentes modos (verbais e não verbais), originam-se diferentes
formas de linguagem e processos de comunicação.} \cite{mayer2001}, a qual
sugere que o processamento da informação ocorre em dois sistemas
separados: o visual e o verbal. Porém, essa teoria, apesar de possuir
orientação cognitivista e ter sido inicialmente desenvolvida para
facilitar a compreensão de informações científicas, tem sido amplamente
aplicada, por conta de sua natureza abrangente, na compreensão e na
elaboração de materiais instrucionais em outras disciplinas e áreas do
conhecimento, como a ASL, mais especificamente, na aprendizagem de
vocabulário de LE mediada por tecnologias digitais \cite{souza2004,procopio2016,monteiro2021}.

Segundo o modelo de aprendizagem proposto por \textcite{mayer2001}, palavras
entram no sistema cognitivo por meio dos ouvidos (em caso de palavras
faladas) e imagens entram por meio dos olhos. Na seleção de palavras e
imagens, o aprendiz presta atenção a determinadas palavras e aspectos da
imagem, o que desencadeia a construção do som da palavra e da imagem na
memória de trabalho. Na organização, o aprendiz mentalmente organiza as
palavras e as imagens selecionadas em representações coerentes na
memória de trabalho, o que o autor chama, respectivamente, de modo
verbal e modo imagético. Por fim, na integração, o aprendiz mentalmente
integra esses dois modos ao conhecimento prévio na memória de longo
prazo. Assim, ocorre o aprendizado da informação multimodal à qual o
aprendiz foi exposto. Logo, a teoria propõe que se aprende melhor com
palavras e imagens que apenas com palavras, pois utilizar dois canais
para apresentar a informação (verbal e visual) é como apresentar o
material duas vezes, expondo o aluno duas vezes mais ao conteúdo alvo.

Considerando o papel central da atenção e da motivação no ensino e
aprendizagem de LE, além do uso de tecnologias digitais como mediadoras
desse processo, nossa hipótese é que a multimodalidade presente em
ambientes digitais é um recurso com grande potencial para promover e
facilitar a aprendizagem. Ela pode tornar o processo de aprendizagem
mais agradável e eficiente ao motivar os alunos durante as atividades e
ajudá-los a direcionar melhor sua atenção ao objeto de aprendizagem para
registrá-lo cognitivamente (\emph{noticing}) e, consequentemente,
aprendê-lo. Porém, não são encontrados na literatura modelos e teorias
de ASL que considerem em suas abordagens a atenção e a motivação de
forma integrada e como elementos centrais do processo de aprendizagem de
LE, pois esses construtos, quando abordados, o são de forma periférica e
não integrada \cite{krashen1985,ellis1997,gass1997,hede2002,mayer2002,moreno2007,saito2015,monteiro2021}. Logo,
surge a necessidade de modelos e teorias que integrem esses construtos
caros à aprendizagem de línguas e os considere como centrais nesse
processo.
