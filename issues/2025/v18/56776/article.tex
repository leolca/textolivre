% !TEX TS-program = XeLaTeX
% use the following command:
% all document files must be coded in UTF-8
\documentclass[english]{textolivre}
\providecommand{\tightlist}{}
% build HTML with: make4ht -e build.lua -c textolivre.cfg -x -u article "fn-in,svg,pic-align"

\journalname{Texto Livre}
\thevolume{18}
%\thenumber{1} % old template
\theyear{2025}
\receiveddate{\DTMdisplaydate{2024}{12}{30}{-1}} % YYYY MM DD
\accepteddate{\DTMdisplaydate{2025}{4}{4}{-1}}
\publisheddate{\DTMdisplaydate{2025}{6}{3}{-1}}
\corrauthor{Alejandro Romero-Muñoz}
\articledoi{10.1590/1983-3652.2025.56776}
%\articleid{NNNN} % if the article ID is not the last 5 numbers of its DOI, provide it using \articleid{} commmand 
% list of available sesscions in the journal: articles, dossier, reports, essays, reviews, interviews, editorial
\articlesessionname{dossier}
\runningauthor{Romero-Muñoz} 
%\editorname{Leonardo Araújo} % old template
\sectioneditorname{Hugo Heredia Ponce}
\layouteditorname{Leonado Araújo}

\title{Didactic intersemiotic and interlingual audio descriptions: some applications for L1 and L2 learning and teaching}
\othertitle{Audiodescrições didáticas intersemióticas e interlinguais: algumas aplicações para o ensino e aprendizagem de L1 e L2}
% if there is a third language title, add here:
%\othertitle{Artikelvorlage zur Einreichung beim Texto Livre Journal}

\author[1]{Alejandro Romero-Muñoz~\orcid{0000-0002-5869-0652}\thanks{Email: \href{mailto:alromero@uji.es}{alromero@uji.es}}}
\affil[1]{Universitat Jaume I, Castelló de la Plana, Castelló, Spain.}

\addbibresource{article.bib}
% use biber instead of bibtex
% $ biber article

% used to create dummy text for the template file
\definecolor{dark-gray}{gray}{0.35} % color used to display dummy texts
\usepackage{lipsum}
\SetLipsumParListSurrounders{\colorlet{oldcolor}{.}\color{dark-gray}}{\color{oldcolor}}

% used here only to provide the XeLaTeX and BibTeX logos
\usepackage{hologo}

% if you use multirows in a table, include the multirow package
\usepackage{multirow}

% provides sidewaysfigure environment
\usepackage{rotating}

% CUSTOM EPIGRAPH - BEGIN 
%%% https://tex.stackexchange.com/questions/193178/specific-epigraph-style
\usepackage{epigraph}
\renewcommand\textflush{flushright}
\makeatletter
\newlength\epitextskip
\pretocmd{\@epitext}{\em}{}{}
\apptocmd{\@epitext}{\em}{}{}
\patchcmd{\epigraph}{\@epitext{#1}\\}{\@epitext{#1}\\[\epitextskip]}{}{}
\makeatother
\setlength\epigraphrule{0pt}
\setlength\epitextskip{0.5ex}
\setlength\epigraphwidth{.7\textwidth}
% CUSTOM EPIGRAPH - END

% to use IPA symbols in unicode add
%\usepackage{fontspec}
%\newfontfamily\ipafont{CMU Serif}
%\newcommand{\ipa}[1]{{\ipafont #1}}
% and in the text you may use the \ipa{...} command passing the symbols in unicode

% LANGUAGE - BEGIN
% ARABIC
% for languages that use special fonts, you must provide the typeface that will be used
% \setotherlanguage{arabic}
% \newfontfamily\arabicfont[Script=Arabic]{Amiri}
% \newfontfamily\arabicfontsf[Script=Arabic]{Amiri}
% \newfontfamily\arabicfonttt[Script=Arabic]{Amiri}
%
% in the article, to add arabic text use: \textlang{arabic}{ ... }
%
% RUSSIAN
% for russian text we also need to define fonts with support for Cyrillic script
% \usepackage{fontspec}
% \setotherlanguage{russian}
% \newfontfamily\cyrillicfont{Times New Roman}
% \newfontfamily\cyrillicfontsf{Times New Roman}[Script=Cyrillic]
% \newfontfamily\cyrillicfonttt{Times New Roman}[Script=Cyrillic]
%
% in the text use \begin{russian} ... \end{russian}
% LANGUAGE - END

% EMOJIS - BEGIN
% to use emoticons in your manuscript
% https://stackoverflow.com/questions/190145/how-to-insert-emoticons-in-latex/57076064
% using font Symbola, which has full support
% the font may be downloaded at:
% https://dn-works.com/ufas/
% add to preamble:
% \newfontfamily\Symbola{Symbola}
% in the text use:
% {\Symbola }
% EMOJIS - END

% LABEL REFERENCE TO DESCRIPTIVE LIST - BEGIN
% reference itens in a descriptive list using their labels instead of numbers
% insert the code below in the preambule:
%\makeatletter
%\let\orgdescriptionlabel\descriptionlabel
%\renewcommand*{\descriptionlabel}[1]{%
%  \let\orglabel\label
%  \let\label\@gobble
%  \phantomsection
%  \edef\@currentlabel{#1\unskip}%
%  \let\label\orglabel
%  \orgdescriptionlabel{#1}%
%}
%\makeatother
%
% in your document, use as illustraded here:
%\begin{description}
%  \item[first\label{itm1}] this is only an example;
%  % ...  add more items
%\end{description}
% LABEL REFERENCE TO DESCRIPTIVE LIST - END


% add line numbers for submission
%\usepackage{lineno}
%\linenumbers

\begin{document}
\maketitle

\begin{polyabstract}
\begin{abstract}
Audio description (AD) is an audiovisual translation (AVT) and media accessibility mode aimed primarily at visually impaired users. Traditionally, AD has tended to use images and some sounds as the source text to create the AD script (the target text), therefore AD is known as an intersemiotic translation. However, some researchers have focused on the theoretical feasibility of opting for the interlingual translation of ADs into another language. Moreover, research has explored the combination of AVT with language teaching and learning, a discipline called “didactic audiovisual translation” with branches like didactic audio description (DAD). In this proposal, we argue that DAD needs to consider both the linguistic nature and application of AD to better differentiate the didactic potential of intersemiotic and interlingual AD, since they have different applications for L1 and L2 learning and teaching. More specifically, intersemiotic AD seems to be suitable for learning L1, for instance, to consolidate grammatical structures, to proofread texts, to develop writing and oral skills, to learn about L1 linguistic variation, etc. On the other hand, interlingual AD can be used to learn about vocabulary and grammar in L2, to develop foreign language oral and written skills, to develop translation skills, etc. All in all, given the potential of DAD for language learning and teaching, this line of research should be further explored with other combinations.

\keywords{Audiovisual translation \sep Didactic audiovisual translation \sep Didactic audio description \sep Didactic intersemiotic audio description \sep Didactic interlingual audio description}
\end{abstract}

\begin{portuguese}
\begin{abstract}
A audiodescrição (AD) é uma modalidade de tradução audiovisual (TAV) e de acessibilidade midiática voltada principalmente para usuários com deficiência visual. Tradicionalmente, a AD tende a utilizar imagens e sons como texto de partida  para a criação do roteiro de AD (texto de chegada), sendo, por isso, considerada  uma tradução intersemiótica. No entanto, alguns pesquisadores têm se debruçado
sobre a viabilidade teórica de optar pela tradução interlingual de roteiros de AD para outros idiomas. Além disso, pesquisas têm explorado a combinação da TAV com o ensino e aprendizagem de línguas, um campo de estudos conhecido como “tradução audiovisual didática”, com ramificações como, por exemplo, a
audiodescrição didática (ADD). Neste trabalho, argumenta-se que a ADD precisa considerar tanto a natureza linguística quanto a aplicação prática da AD para melhor diferenciar o potencial didático da AD intersemiótica do potencial didático da AD interlingual, já que elas possuem aplicações distintas no ensino e aprendizagem de L1 e L2. Mais especificamente, a AD intersemiótica parece ser mais adequada para o aprendizado de L1, por exemplo, para consolidar estruturas gramaticais, revisar textos, desenvolver habilidades de escrita e oralidade, aprender sobre variações linguísticas da L1 etc. Por outro lado, a AD interlingual pode ser usada para aprender vocabulário e gramática da L2, desenvolver habilidades de oralidade e escrita em língua estrangeira, desenvolver habilidades
tradutórias, entre outras possibilidades. Em suma, considerando o potencial da ADD para o ensino e aprendizagem de línguas, sugere-se que essa linha de pesquisa seja ampliada e aprofundada por meio de outras combinações e aplicações de seus usos em diferentes contextos linguísticos e educacionais.

\keywords{Tradução audiovisual \sep Tradução audiovisual didática \sep Audiodescrição didática \sep Audiodescrição intersemiótica didática \sep Audiodescrição interlingual didática}

\end{abstract}
\end{portuguese}
% if there is another abstract, insert it here using the same scheme
\end{polyabstract}

\section{Introduction}\label{sec-intro}
Audiovisual translation (AVT) is a discipline that belongs to Translation Studies and audiovisual texts convey information through at least two communication channels, which transfer encoded meanings by means of sign systems. Despite the various ways of cataloguing AVT modes — such as dubbing, subtitling, videogame localization, and audio description (AD) — these are usually considered part of AVT. The latter, AD, is an AVT mode that aims to make audiovisual products and events accessible to visually impaired or partially sighted users. On the other hand, following the development of AVT research, we will cover the emerging discipline called “didactic audiovisual translation” (DAT), which considers AVT as a tool for language teaching and learning. More specifically, this paper focuses on a particular part of DAT, didactic audio description (DAD), with the aim of answering a main research question: should DAD consider different linguistic dimensions beyond the common combinations? Thus, in this paper, we will support the hypothesis that research on DAD should make a distinction based on the linguistic nature of AD (which results in intersemiotic and interlingual ADs) and on the linguistic application of AD (which leads to a different didactic potential between L1 and L2). The main aim of the paper, therefore, is to provide appropriate tools to distinguish between these types of ADs and to highlight their distinctive didactic potential.

To fully portray the characteristics of DAD, we will first address the development of AVT and its link to DAT. After that, we will tackle some proposals related to the notion of translated or interlingual AD. The methodology will then be explained with a particular emphasis on the linguistic axis of AD (interlingual vs intersemiotic AD, and DAD for L1 and L2) followed by a concluding discussion.

\section{Audiovisual translation, audio description, and didactic audiovisual translation}\label{sec-normas}
In AVT, there is an intralingual (within the same language), interlingual (between different languages), or intersemiotic (between linguistic and non-linguistic sign systems) transfer of audiovisual texts, whose multimodal contents convey complex information through at least the visual and acoustic communication channels, although a third one, the tactile channel, may also be involved in audiovisual products like videogames \cite{mejias-climent2021}. Through these channels, different encoded sign systems (linguistic, paralinguistic, musical, iconographic, graphic, mobility, haptic, etc.) interrelate to create complex meanings. We can find multiple classifications of AVT modes, such as the one proposed by \textcite{chaume2004,chaume2012}, who distinguishes between revoicing (where a soundtrack replaces or adheres to the original soundtrack) and captioning (where a written translation is inserted on screen). Within revoicing, we find modes like dubbing or AD, among others, while within captioning, we find subtitling or subtitling for the deaf and hard of hearing (SDH), for example. Of these options, this paper focuses on AD, an AVT and media accessibility (MA) mode that aims to make audiovisual products (films, series, documentaries, etc.) and events (such as theater or opera) accessible by means of an oral narration that translates visual elements (and some acoustic references that might be difficult to understand) for a primary audience with significant visual impairment. Furthermore, drawing on Jakobson’s \citeyear{jakobson1959} classification, AD is often categorized as a form of intersemiotic translation, since there is a transfer from a non-verbal sign system (the image and some sounds, which are the source text or ST) to a verbal sign system (the AD script, which is the target text or TT). 

Despite the variety of AVT modes, it is important to note that this branch of Translation Studies is relatively recent. In fact, although the first academic contributions date back to the 1950s and 1960s, AVT research gained greater visibility during the 1990s and was established in the 2000s \cite{mangiron2022}, and has experienced dynamic development in recent years. To represent this dynamism, the “turn” metaphor has often been used, although there is no consensus on the exact turns AVT has undergone, to the point that there has been “a trivializing understanding of ‘turns’ as little more than relatively self-standing research themes” \cite[p. 1]{perez-gonzalez2019}. Thus, we find references to the descriptive, audiovisual, technological, sociological, cultural, cognitive turns, etc. \cite{remael2010,ohagan2013,chaume2018,perez-gonzalez2019}. This diversity of methodological turns is unusually vibrant in MA, which is currently experiencing a period of academic flourishing \cite{neves2022}. 

Evidence of the academic progress in both AVT and MA can be found in the development of a relatively recent discipline, didactic audiovisual translation (DAT), which is “the active use of the different audiovisual translation (AVT) modes […] by students in language learning […], as the focus of a lesson plan or didactic sequence, or as an isolated task” \cite[p. 1]{talavan_lertola_fernandez-costales2024}. As the authors explain, DAT has been primarily associated with foreign language (L2) teaching, although it can also be applied to L1 teaching and learning, to minority languages, or in bilingual education contexts. One of the oldest links between translation and language learning can be found in the 19th century grammar-translation method, but by the end of the century, the direct method emerged, and “translation was abandoned in favor of the teacher and the students speaking together” \cite[p. 56]{harmer2019}. This rejection of translation continued in many 20th and 21st century methodologies: humanistic approaches (silent way, suggestopedia, total physical response, and community language learning), communicative approaches, task-based learning, the Dogme approach, the action-oriented approach, etc. Stemming from this methodological development, the modern tendency to reject translation in language teaching likely originates because of a misconception that associates translation with the grammar-translation method. However, \textcite{talavan_lertola_fernandez-costales2024} argue that the use of DAT in the classroom is beneficial and can be used at all levels, from primary to university education \cite{talavan_lertola_fernandez-costales2024}.

DAT includes the didactic application of any AVT mode, so in the case of dubbing we should speak of didactic dubbing (see, for instance, \textcite{navarrete2013,talavan2013,talavan_costal2017,sanchez-requena2018}, didactic subtitling \textcite{borghetti2014,vanderplank2016,fernandez-costales2017,avila-cabrera2018,diaz-cintas2018a,sokoli2018,talavan2020}, didactic subtitling for the deaf and hard of hearing \cite{agullo2019, talavan2019, garcia-munoz_vizcaino2024}, or didactic respeaking \cite{belenguer-cortes2024} among others. In the case of didactic audio description (DAD), it has traditionally been used so that students write (or even voice) AD scripts using their L2 in various combinations \cite{talavan_lertola_fernandez-costales2024}: intersemiotic (from a non-verbal sign system to L2), interlingual (from L1 to L2), or intralingual (from L2 to L2). 

\section{Audio description beyond intersemiotic translation}\label{sec-conduta}
As we have stated, AD is usually classified as a form of intersemiotic translation where the ST is a non-verbal sign system and the TT is a verbal sign system. Hence, traditional AD translates images and certain sounds into a script that is typically in the same language as the audio. Therefore, an AD in Spanish is created drawing on an audiovisual product in this language (either the original Spanish version or the dubbed version into Spanish) \cite{aenor2005}. However, some studies have explored the feasibility of translating ADs: an existing AD would serve as the ST, which is used as a template in order to save costs, reduce production times, expand the linguistic offer, etc. In such cases, the translated AD would not be an example of intersemiotic translation, but rather an interlingual translation (from language A to language B) of an intersemiotic translation (the original AD used as a template). In fact, some studies have analyzed the theoretical advantages and disadvantages of hypothetical interlingual ADs. \textcite{hyks2005} was perhaps the first one to envision translated AD as a strategy to save time but ultimately rejected its suitability, arguing that it could take more time than traditional AD. \textcite{vanderheijden2007} shared this view, considering AD translation an overly time-consuming solution \cite{remael-vercauteren2010}. Conversely, \textcite{lopez_vera2006} argued that translated AD could reduce costs compared to traditional AD, which could increase the number of audio-described audiovisual products available. \textcite{remael-vercauteren2010} explored the challenges of translating AD and highlighted several benefits: cost reduction, the ability to engage more translators capable of working on AD (compared to the smaller number of trained audio describers who can analyze images and select the appropriate content for visually impaired or partially sighted users), and the potential to produce AD in countries with more limited educational means to train audio describers. \textcite{bourne2007} also saw benefits in terms of time and cost compared to traditional AD. Finally, \textcite{jankowska2015} conducted a triple experiment examining the time required to create translated AD, reactions from visually impaired audiences, and a cognitive comparison of AD created by audio describers versus translators. Her results showed that translated ADs were produced faster than traditional ADs and participants preferred the translated versions.

Therefore, it seems convenient to make a distinction between two types of ADs: intersemiotic and interlingual. Beyond the theoretical debate on their hypothetical advantages or disadvantages, we should consider which type of AD typically appears in professional contexts. AD scripts follow the recommendations of national, international, academic, or streaming platforms’ quality standards, which have generally opted for the intersemiotic AD model. For instance, Spain’s UNE 153020 norm \cite{aenor2005} specifies the following AD process: preliminary analysis, script preparation, script revision and correction, recording, editing, and final product review. The analysis step requires that the AD be created in the same language in which the acoustic information is presented in the audiovisual work \cite{aenor2005}. It also provides some analysis suggestions which are incompatible with interlingual AD, such as doing some research about the topic to ensure appropriate vocabulary and the fact that AD units must be inserted into the message gaps while respecting dramatic action or atmosphere, for example. Similarly, \textit{France’s La Charte de l'audiodescription} \cite{morisset2008} outlines an AD process starting with one or two viewings of the audiovisual product, followed by an initial description, research, a polished description, time code integration (indicating when each AD unit begins and ends), and so on. Again, this process does not consider AD translation, which would omit steps like time insertion, given the fact that this information would already exist in the original AD. In the UK, the \textit{ITC Guidance} \citeyear{itc2000} outlines steps such as selecting the program to describe, viewing it, drafting the AD, revising it, sound adjustments, recording, and reviewing the recording. In our view, steps like viewing or drafting point to the creation of AD from scratch using images and sound as the ST, not to the translation of an existing AD. In the US, the \textit{Standards for Audio Description and Code of Professional Conduct for Describers} \cite{audio_description_coalition2009} recommends viewing the material to identify key visual information not accessible to visually impaired audiences and describing the essential elements first. Such steps would be unnecessary in intralingual AD. Similarly, the Audio Description Standards \cite{california_audio_describers_alliance2009} offer advice applicable only to intersemiotic AD, such as describing essential details and specifying the description of “where”, “when”, “what”, etc. Joel Snyder’s \textit{The Visual Made Verbal} \citeyear{snyder2014} emphasizes four key pillars for learning to create AD: observation, editing (deciding what to include or omit in the AD), language, and vocal technique (for recording the AD). Again, interlingual AD would make observation and editing unnecessary since these elements are already part of the ST. Netflix’s AD guidelines \citeyear{netflix2024} include describing complex elements, such as action, credits, and on-screen elements. Once more, this approach suggests that Netflix supports the traditional intersemiotic AD process. 

In this vein, a few notions can be extracted from the two previous theoretical sections. First, even if DAD could be used to foster L1, many of the studies covering this discipline have been used to foster L2 learning. Moreover, they seem to be using intersemiotic AD, the type of AD that is encouraged by professional guidelines. Therefore, in the following section, we will provide some proposals so that future DAD can benefit from the distinctive use of intersemiotic and interlingual ADs to enhance L1 and L2 skills.


\section{Methodology}\label{sec-fmt-manuscrito}
Drawing on the theoretical concepts presented, this study aims to propose a methodological approach to make a systematic distinction between interlingual and intersemiotic AD to obtain the maximum didactic potential for L1 and L2 teaching and learning. Therefore, in order to magnify the potential of DAD, we develop two dimensions in the following subsections that need to be considered: the linguistic nature of AD and the linguistic application of AD.


\subsection{The linguistic nature of AD}\label{sec-formato}
The linguistic nature of AD determines whether a given AD has an intersemiotic or interlingual origin. In order to work with these different types of AD, a corpus of ADs belonging to different streaming platforms was compiled and analyzed. To do so, a series of selection criteria were applied, such as availability (streaming platforms, which led to Netflix, Amazon Prime Video, Disney+, and Apple TV+), production (series), language (AD in English and Spanish), the number of series (two per platform), and genre (drama and thriller). Based on these criteria, the following eight series were chosen: \textit{Sky Rojo} and \textit{Elite} (Netflix), \textit{The Wheel of Time and The Rings of Power} (Amazon Prime Video), \textit{The Clearing and Great Expectations} (Disney+), and \textit{Truth Be Told} and \textit{Lisey’s Story} (Apple TV+). After selecting the eight series, some excerpts were transcribed in English and Spanish, and they were aligned to examine whether they were intersemiotic or interlingual ADs. To determine their linguistic nature, three differentiation parameters were established (\Cref{tbl1}). Meeting these parameters would indicate that, when analyzing ADs in two (or more) languages, ADs in these languages are distinctive enough so as to be considered intersemiotic. However, failing to meet the parameters could suggest that the ADs in these languages are too similar and follow almost identical patterns that point to them being interlingual ADs.

The first differentiation parameter concerned the use of any available gaps (i.e., the non-dialogue pauses or moments with music, sound effects and ambient noise in which AD could be placed) \cite{isoiec2015}. We consider that two intersemiotic ADs in two different languages drawing on the same audiovisual product will insert AD units with different starting and ending times, that is, they will use the available gaps in different ways. The second parameter had to do with the linguistic structures and contents used in both ADs, which are expected to vary in the case of intersemiotic AD, since two traditional ADs about the same audiovisual product will certainly include slightly different information. Finally, the third parameter involved the absence of calqued structures or lexical forms, given that intersemiotic AD uses the image and sound as a source text (ST) to provide a perfectly polished AD script in the target language, where no interlingual interference is expected.

\begin{table}[h!]
\centering
\begin{threeparttable}
\caption{Differentiation parameters.}
\label{tbl1}
\begin{tabular}{l}
\toprule
Differentiation parameters between intersemiotic and interlingual AD \\ 
\midrule
Different use of available gaps \\
Variation of linguistic structures and contents \\
Absence of calques \\
\bottomrule
\end{tabular}
\source{Own elaboration (2024)}
\end{threeparttable}
\end{table}



\subsection{The linguistic application of AD}\label{sec-modelo}
On the other hand, another interesting distinction about DAD that can be made involves which type of language is applied to foster teaching or learning. As we have seen, \textcite{talavan_lertola_fernandez-costales2024} point out that DAD has typically focused on L2 (from a non-verbal system to L2, from L1 to L2, or from L2 to L2). However, given the potential of DAD, we propose the need to distinguish between didactic intersemiotic AD and didactic interlingual AD to maximize the benefits of DAD for both L1 and L2 learning and teaching.

\subsubsection{Didactic intersemiotic AD}\label{sec-organizacao}
If we consider some DAD research examples, we can mention studies such as those carried out by \textcite{ibanez-moreno2013} with their study on lexical skills; \textcite{ibanez-moreno2017} and the integrated development of skills; \textcite{navarrete2018}, \textcite{talavan_lertola2016}, or \textcite{navarrete2024} and oral skills; \textcite{calduch2018}, or \textcite{talavan_lertola_ibanez2022} and writing skills; \textcite{schaeffer-lacroix2020} and morphology; \textcite{bartolini2024} and didactic museum AD; \textcite{herrero2018} and media literacy; \textcite{bausells-espin2022} and students’ perception, etc. Having these proposals in mind, we observe two trends: they use intersemiotic AD and they focus mostly on L2 teaching. Considering these studies, it is our view that intersemiotic AD would be useful to consolidate grammar in students’ L1 (for example, linguistic variation), because AD tends to neutralize most variation traits \cite{romero2025linguistic}, so this can be the perfect scenario for students to reflect on the way linguistic variation manifests in their own native language. L1 proofreading skills could also be enhanced by asking students to polish AD scripts in their own language. Similarly, written expression skills in L1 could also be enhanced through the drafting of AD scripts. Finally, oral expression skills in L1 could be improved as well through the voicing of AD scripts, among other applications (see \Cref{tbl2}).

\begin{table}[htbp]
\centering
\begin{threeparttable}
\caption{Didactic intersemiotic AD.}
\label{tbl2}
\begin{tabular}{p{4cm} p{10cm}}
\toprule
\multicolumn{2}{l}{Didactic intersemiotic AD applications} \\ 
\midrule
Linguistic combinations: & From a non-verbal system/L1/L2 to L2/L1   \\ 
Areas: & Lexical skills, integrated development of skills, oral skills, writing skills, morphology, media literacy, students’ perception, linguistic variation, L1 expression skills, L1 proofreading skills, etc.    \\ 
\bottomrule
\end{tabular}
\source{Own elaboration (2024).}
\end{threeparttable}
\end{table}

\subsubsection{Didactic interlingual AD}\label{sec-organizacao-latex}
To the best of our knowledge, no studies have explored the use of interlingual AD in L1 or L2 learning and teaching. Despite this situation, we believe that interlingual AD could have significant relevance in the usual study of L2 (from a non-verbal system to L2, from L1 to L2, or from L2 to the same L2). For instance, it could be useful as a tool to learn L2 vocabulary and grammar if students are asked to translate an AD script covering a specific topic with rich vocabulary (such as cultural-related vocabulary about a country) or with certain structures (such as the use of the passive voice or the present tense). Interlingual AD could be used to develop written and oral skills if students write or voice the AD scripts in their foreign language. Moreover, interlingual AD could also be useful for developing translation competence (L1-L2), since students would be practicing the linguistic and cultural transfer from a given source language to a target language. Finally, interlingual AD could also be used to enhance L1, for instance, by means of proofreading exercises where students need to provide a polished version of the translated AD in their native language (see \Cref{tbl3}).

\begin{table}[htbp]
\centering
\begin{threeparttable}
\caption{Didactic interlingual AD.}
\label{tbl3}
\begin{tabular}{p{4cm} p{10cm}}
\toprule
\multicolumn{2}{l}{Didactic intersemiotic AD applications} \\ 
\midrule
Linguistic combinations: & From a non-verbal system/L1/L2 to L2/L1   \\ 
Areas: & Vocabulary, grammar, written skills, oral skills, translation competence, proofreading, etc.    \\ 
\bottomrule
\end{tabular}
\source{Own elaboration (2024).}
\end{threeparttable}
\end{table}

\section{Results}\label{sec-5}
After analyzing the eight series from four streaming platforms, we obtained revealing data about the nature of the AD scripts. Regarding the results from Netflix (\textit{Sky Rojo} and \textit{Elite}), Amazon Prime Video (\textit{The Wheel of Time} and \textit{The Rings of Power}), and Disney+ (\textit{The Clearing} and \textit{Great Expectations}), the contrastive analysis of the English and Spanish AD versions shows that the differentiation parameters are met in these examples: the AD in English and Spanish make different use of available gaps, they provide varied linguistic structures and contents, and they do not have calques. 

As \Cref{tbl4} shows, in this fragment from \textit{Sky Rojo} (Netflix), Wendy delivers her whole speech in English, whereas the Spanish version uses a brief pause to insert a descriptive segment (\textit{Coral vuelve a su lado}) at minute 22 and second 15, so the available gap parameter is met. Moreover, at minute 22 and second 37, the Spanish AD informs about Coral’s movements (\textit{lo deja en el suelo y se aproxima a la caja fuerte} [“she leaves him on the floor and approaches the safe”]), while the English AD starts by describing Romeo’s appearance (“blood steaks his nose”) and then describes Coral’s movements (“she gets up and opens a cabinet by a wall”), so in this case the linguistic structures and contents parameter is also met. Finally, no calques are perceived in either of the languages, so the calque absence parameter is met.

\begin{table}[h!]
\centering
\begin{threeparttable}
\caption{AD script fragment from \textit{Sky Rojo}.}
\label{tbl4}
\begin{tabular}{p{7cm} p{7cm}}
\toprule
\textbf{Coral}: Es la segunda vez que vuelves de entre los muertos, no tientes más a la suerte. &
\textbf{Coral}: This is the second time you’ve managed to come back from the dead. Don’t push your luck.  \\ 
\texttt{[22:37]} Lo deja en el suelo y se aproxima a la caja fuerte. Introduce en una mochila unos fajos de billetes, el cuaderno de cuentas y unos pasaportes. Coge dos pistolas y las coloca entre el cinturón y la cintura de su falda. Romeo llora afligido en el suelo. &
\texttt{[22:37]} Blood streaks his nose. She gets up and opens a cabinet by a wall. She punches a code on a safe inside and opens it to find money, passports, and ledgers. Romeo watches her stuff them into a backpack.    \\ 
\textbf{Wendy}: Si la hija de puta que amaste hasta la locura te abre la cabeza, te humilla y te roba… &
\textbf{Wendy}: If the fucking bitch you love like crazy smashes your head open, humiliates you, and steals from you, it’s okay to cry out of anger or helplessness. But Romeo’s were tears of hope, because instead of leaving him to die, Coral had just saved his life.     \\ 
\texttt{[22:15]} Coral vuelve a su lado. & \\
\textbf{Wendy}: Es normal llorar, de rabia o de impotencia. Pero Romeo lloraba de esperanza, porque en lugar de dejarlo morir, Coral le había salvado la vida. & \\
\texttt{[22:04]} Se agacha y le coge unas llaves del bolsillo. &
\texttt{[22:04]} She finds his keys and opens the door. \\
\textbf{Coral}: Ese momento mágico y sentimental merecía una celebración. &
\textbf{Coral}: And that magical and emotional moment warranted a celebration. \\
\bottomrule
\end{tabular}
\source{Own elaboration (2024).}
\end{threeparttable}
\end{table}

However, in the case of Apple TV+ (\textit{Truth Be Told} and \textit{Lisey’s Story}), the three parameters are not met systematically. First, as \Cref{tbl5} shows, both ADs begin and end at the same time (at minute 3 and second 27, at minute 3 and second 34, and at minute 3 and second 47). Moreover, they provide the same information using nearly identical linguistic structures, such as \textit{la mujer acaricia los labios del hombre de pelo oscuro} [“the woman caresses the dark-haired man’s lips”] in Spanish and “the woman touches a dark-haired man’s mouth” in English. Finally, even though calques are less evident, the absence of some indefinite articles in Spanish suggests potential calques that would hinder the third differentiation parameter. For instance, where the English AD states “framed posters cover a wall”, the Spanish AD uses carteles enmarcados decoran una pared [“framed posters decorate a wall”] instead of a more natural solution like  \textit{unos carteles} or \textit{algunos carteles} [“some posters”]. Similarly, where the English AD says “books lie on shelves”, the Spanish AD opts for libros ocupan un estante [“books occupy a shelf”] rather than \textit{unos libros} or \textit{algunos libros} [“some books”]. Considering that these calques occur in Spanish due to the influence of English, we can venture that these examples are interlingual AD, where the English AD would be the ST and the Spanish AD would be the TT.

\begin{table}[htbp]
\centering
\begin{threeparttable}
\caption{AD script fragment from \textit{Lisey’s Story}.}
\label{tbl5}
\begin{tabular}{p{7cm} p{7cm}}
\toprule
\textbf{Lisey}: ¿Por qué son tan importantes para ti? & \textbf{Lisey}: Why are they so important to you?   \\ 
\textbf{Scott}: Las historias son lo único que tengo. & \textbf{Scott}: Stories are all I have.  \\
\texttt{[03:27]} La mujer acaricia los labios del hombre de pelo oscuro. & \texttt{[03:27]} The woman touches a dark-haired man’s mouth. \\
\textbf{Scott}: Aunque ahora te tengo a ti. & \textbf{Scott}: And now I have you. \\
\texttt{[03:34]} El hombre la mira y le coge una mano. Libros ocupan un estante. & \texttt{[03:34]} He gazes at her, holding her hand. Elsewhere, books lie on shelves. \\
\textbf{Lisey}: Te quiero. & \textbf{Lisey}: I love you. \\
\textbf{Scott}: Y yo te quiero a ti. & \textbf{Scott}: I love you. \\
\texttt{[03:47]} Framed posters cover a wall. & \texttt{[03:47]} Carteles enmarcados decoran una pared. \\
\textbf{Scott}: Eres todas mis historias. & \textbf{Scott}: You’re every story. \\
\bottomrule
\end{tabular}
\source{Own elaboration (2024).}
\end{threeparttable}
\end{table}

Considering these results, we could conclude that our ADs from Netflix, Amazon Prime Video, and Disney+ are most likely intersemiotic ADs whose ST is the image and some sounds. Conversely, the differentiation parameters suggest that the two ADs from Apple TV+ are interlingual ADs, where the ST is an intersemiotic AD. Given that the Apple TV+ series analyzed are American and considering that the calques in Spanish were influenced by English, we can state that English served as a pivot language to create an AD template in turn used to create the Spanish AD (a situation that could possibly be extrapolated to other languages, since Apple TV+ offers a significantly broader linguistic AD catalogue compared to the other three platforms studied here). Regarding DAD, even if this discipline has traditionally used intersemiotic AD in L2 teaching and learning, it could be convenient to make a distinction between didactic intersemiotic and interlingual AD, whose potential includes both L1 and L2. Intersemiotic AD can have a didactic application to consolidate students’ L1 (grammar, proofreading, written and oral skills, etc.) and to enhance their L2 (lexical skills, integrated development of skills, oral skills, writing skills, morphology, media literacy, etc.). On the other hand, interlingual AD may have a clearer application in students’ L2 (to work on written expression skills, oral comprehension skills, or translation competence, for instance), although this type of AD can also be useful to enhance students’ L1 by means of proofreading exercises. 

\section{Conclusions}\label{sec-idioma}
In conclusion, after reviewing the results obtained through the contrastive analysis of our AD corpus, we can state that interlingual AD is not merely a theoretical notion limited to academic debates about its feasibility, but it is rather a reality in some streaming platforms’ catalogues. While these findings should be expanded by analyzing additional examples from each platform in larger corpora, our differentiation parameters (the different use of available gaps, the presence of varied linguistic structures and contents, and the absence of calques) suggest that the catalogues from Netflix, Amazon Prime Video, and Disney+ make use of intersemiotic AD, where the ST is the image and some sounds. However, it appears that Apple TV+ includes interlingual AD, in which case the ST is likely to be the English AD, which is then used to create the Spanish AD (and likely the AD in any other language). Moreover, we believe that research on DAD could maximize its potential if the linguistic nature and the linguistic application of AD are considered. By doing so, a distinction between didactic intersemiotic AD and didactic interlingual AD should be made, since they can be used in different ways to teach and learn both L1 and L2.

Therefore, this study should be considered as a first step toward creating a more comprehensive overview of the presence of intersemiotic and interlingual ADs on streaming platforms, as well as their didactic potential. It must be stated that this article focused on the need to distinguish between these two types of DAD in a systematic way, so their didactic applications are taken here as proposals that should be implemented in the classroom. Accordingly, future studies should address several issues, including the compilation of a larger corpora with more languages and streaming platforms, a temporal reference to identify when interlingual AD began to be incorporated, linguistic studies to explore the characteristics of both types of AD, reception studies to consider end-users’ opinions, the different ways in which these ADs are successfully applied in an L1 or L2 classroom, etc. Be that as it may, these possibilities highlight the long road that research on DAD can still explore, which is symptomatic of DAT’s vitality.


\printbibliography\label{sec-bib}
% if the text is not in Portuguese, it might be necessary to use the code below instead to print the correct ABNT abbreviations [s.n.], [s.l.]


\end{document}

