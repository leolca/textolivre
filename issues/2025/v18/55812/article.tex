\documentclass[portuguese]{textolivre}

% metadata
\journalname{Texto Livre}
\thevolume{18}
%\thenumber{1} % old template
\theyear{2025}
\receiveddate{\DTMdisplaydate{2024}{11}{11}{-1}}
\accepteddate{\DTMdisplaydate{2024}{12}{28}{-1}}
\publisheddate{\DTMdisplaydate{2025}{2}{23}{-1}}
\corrauthor{Marta Santos Silva}
\articledoi{10.1590/1983-3652.2025.55812}
%\articleid{NNNN} % if the article ID is not the last 5 numbers of its DOI, provide it using \articleid{} commmand 
% list of available sesscions in the journal: articles, dossier, reports, essays, reviews, interviews, editorial
\articlesessionname{articles}
\runningauthor{Santos Silva, Gonçalves e Morais}
%\editorname{Leonardo Araújo} % old template
\sectioneditorname{Daniervelin Pereira}
\layouteditorname{Leonardo Araújo}

\title{Tecnologia, inovação e modelos de negócio nos jornais regionais portugueses}
\othertitle{Technology, innovation, and business models in Portugal’s local newspapers}

\author[1]{Marta Santos Silva~\orcid{0009-0004-9148-3307}\thanks{Email: \href{mailto:marta.santos.silva@ubi.pt}{marta.santos.silva@ubi.pt}}}
\author[2]{Adriana Gonçalves~\orcid{0000-0002-8028-8248}\thanks{Email: \href{mailto:adriana.goncalves@ubi.pt}{adriana.goncalves@ubi.pt}}}
\author[3]{Ricardo Morais~\orcid{0000-0001-8827-0299}\thanks{Email: \href{mailto:rjmorais@letras.up.pt}{rjmorais@letras.up.pt}}}
\affil[1]{Universidade da Beira Interior, Faculdade de Artes e Letras, Labcom: Comunicação e Artes, Covilhã, Portugal.}
\affil[2]{Universidade do Porto, Faculdade de Artes e Humanidades, Porto, Portugal.}
\affil[3]{Universidade do Porto, Faculdade de Letras, Porto, Portugal.}

\addbibresource{article.bib}

\usepackage{array}

\begin{document}
\maketitle
\begin{polyabstract}
\begin{abstract}
A inovação tecnológica desempenha um papel transformador nos media, perante os desafios da transição digital e a crise financeira do setor. Este artigo analisa o papel que a inovação ocupa nos jornais regionais portugueses e nos seus modelos de financiamento, através de uma abordagem qualitativa (entrevistas com os diretores de três jornais tradicionais e três nativos digitais). Os entrevistados veem o conceito de inovação interligado com a tecnologia, mas também incluem o modelo de negócio e a orientação editorial como possíveis focos de inovação. Contudo, os diretores dos jornais tradicionais demonstram algum ceticismo em relação à sustentabilidade do jornalismo digital, destacando a relevância financeira da sua edição impressa. Para os nativos digitais, o financiamento diversifica-se entre investimentos privados, bolsas de apoio, assinaturas digitais e doações. Os entrevistados mostram-se preocupados com a crise no setor e atentos aos desafios do digital, reforçando que estão a atravessar um momento de transição.

\keywords{inovação \sep financiamento \sep jornalismo regional \sep Portugal \sep tecnologia}
\end{abstract}

\begin{english}
\begin{abstract}
Technological innovation plays a transformative role in media, in the face of digital transition and a financial crisis that plagues the sector. This article analyses the role that innovation plays in Portuguese regional newspapers, and in their funding models, through a qualitative approach (interviews with the heads of three digital native outlets and three storied print newspapers). The interviewees define the concept of innovation as interwoven with technology, but they also consider that business models and editorial angles can be innovative. However, the heads of the traditional newspapers are skeptical towards the sustainability of digital journalism, highlighting the continued financial relevance of their print editions. For the digital natives, funding sources split between private investors, fellowships, digital subscriptions and donations. All interviewees show concern about the crisis in the sector and they are attentive to the challenges brought by digital technologies; they reinforce that they are living a moment of transition.

\keywords{innovation \sep funding \sep local journalism \sep Portugal \sep technology}
\end{abstract}
\end{english}
\end{polyabstract}

Existem publicações regionais desde o princípio da atividade
jornalística \cite{Tengarrinha2013}, mas os desafios das últimas duas
décadas, das crises financeiras e da crise estrutural de contornos
sociais e políticos que a imprensa enfrenta \cite{DeMateo2010,Christofoletti2019,Mota2023}, têm colocado grandes desafios
ao jornalismo regional\footnote{Embora estudem, na
  maior parte dos casos, o mesmo objeto, são vários os conceitos
  utilizados para fazer referência ``ao campo do jornalismo que se
  centra nos pequenos territórios, no compromisso com eles e com as
  respetivas comunidades'' \cite[p.~24]{Jeronimo2015}. Neste trabalho
  optamos pelo conceito de jornalismo regional, uma vez que, ao
  contrário do que acontece, por exemplo, em Espanha, jornais locais e
  regionais circulam, na maior parte dos casos, nos mesmos territórios.
  A própria legislação (Lei da Imprensa; Estatuto do Jornalista;
  Estatuto da Imprensa Regional) não clarifica os conceitos, usando de
  forma indiferenciada a imprensa local e regional.}. A crescente
precarização do setor e redução no número de publicações tem levado a
uma ``erosão da influência do jornalismo de proximidade''
\cite[p.~2]{Mota2023} e à multiplicação dos ``desertos de notícias''
\cite{Jeronimo2022b}.

Os meios de comunicação regionais que ainda resistem enfrentam a
diminuição da receita publicitária e das vendas em banca, lançando-se na
procura de novas fontes de receita e modelos de negócio
\cite{DeMateo2010}. Em contracorrente, surgem novos media regionais na
\emph{web}, com estratégias de financiamento diferentes daquelas que a
imprensa tradicional segue.

O surgimento destes veículos de comunicação num período de crise
estrutural e multifacetada \cite{Christofoletti2019} levanta dúvidas:
Como sobrevivem estes meios de comunicação? De que forma a inovação está
presente nestes media? E em relação aos media regionais resistentes, de
que forma o seu modelo de negócio se relaciona com estratégias de
inovação?

Estas são algumas das questões que guiaram este artigo, com a intenção
de reforçar a literatura que relaciona a inovação tecnológica, os
modelos de negócio e o jornalismo regional em Portugal. O estudo adota
uma abordagem qualitativa, utilizando como técnica metodológica
entrevistas com dirigentes de seis jornais regionais (``Jornal do
Fundão''; ``Região de Leiria''; ``A Voz de Trás-Os-Montes''; ``A
Mensagem de Lisboa''; ``Sul Informação''; ``Médio Tejo'') escolhidos de
acordo com a sua natureza (três tradicionais e três nativos digitais) e
distribuição geográfica.

Este trabalho tem como objetivo compreender de que forma a inovação
tecnológica se relaciona com o modelo de financiamento no panorama
regional português. Para dar resposta a esta inquietação, o artigo
descreve o contexto de crise do modelo de negócio do jornalismo, aborda
a sustentabilidade nos media regionais e elenca alguns caminhos teóricos
sobre a inovação tecnológica para, de seguida, passar à metodologia e
aos resultados obtidos.

\section{Enquadramento teórico}
\subsection{Crise do modelo de negócio tradicional do jornalismo}

Durante muita da sua história, a imprensa teve como principal fonte de
receita a publicidade \cite{Ohlsson2017}. Com a revolução digital, os
jornais que eram detentores do monopólio da atenção dos leitores, passam
a ser mais um concorrente na luta por essa atenção. Plataformas como a
Google, o Facebook ou a Amazon ficam com a maior parte dos investimentos
publicitários digitais \cite{Breiner2016}, mesmo perante exigências do
Parlamentos Canadiano e Europeu para a repartição de lucros
\cite{Chan2023}.

Perante a crise multifacetada que o setor enfrenta, os jornais enfrentam
rigidez e dificuldade em inovar, o que \textcite{Meyer2009} justifica em parte
pelo ``conservadorismo inerente'' dos jornais e pelo enfraquecimento das
estruturas dos jornais na busca de maiores lucros. \textcite{IhlstromEriksson2016} acrescentam a dificuldade em obter rendimento do lado dos
consumidores, habituados agora a ter acesso gratuito à informação na
Internet. A \emph{paywall}, limitação do acesso apenas a subscritores
pagos, não é de uso generalizado: \textcite{Reis2019} assinalava que poucos
jornais diários no Brasil tinham introduzido este recurso.

No entanto, a implementação de uma \emph{paywall} não garante a
sustentabilidade dos jornais, levando à busca de financiamentos
alternativos como o \emph{crowdfunding}, \cite{Fonseca2016}, já com exemplos bem-sucedidos de meios independentes em Portugal
que se mantiveram em funcionamento com as contribuições dos leitores
\cite{Teixeira2021}; ou o apoio financeiro de fundações e organizações
não-governamentais procurado por jornais como a ``Divergente'' e o
``Fumaça'' \cite{Chaves2021,Bonixe2022}.

Um dos desafios é a falta de disponibilidade da audiência para pagar
pelo acesso às notícias: a percentagem de pessoas que pagam uma
subscrição digital estabilizou desde 2021 nos 17\% nos 20 países
analisados no \emph{Digital News Report} \cite{Newman2024}.
Portugal ``continua a destacar-se negativamente como um dos países onde
menos se paga por notícias em formato digital'' \cite[p.~104]{Cardoso2022}. 
Entre os 12\% dos inquiridos que dizem
ter pago por notícias online no último ano, a preferência continua a ser
a subscrição de forma contínua (34\%), seguindo-se aqueles que pagam de
forma indireta, ou seja, através da subscrição de um serviço que inclui
o acesso a notícias (30\%). Às pessoas que não pagaram por notícias no
último ano, perguntou-se quanto estariam dispostas a pagar. Nos 20
mercados analisados, 57\% das pessoas inquiridas não estão dispostas a
pagar nada \cite[p.~51]{Newman2024}.

Este problema de financiamento também se reflete no jornalismo regional,
que a partir de 2014-2015 foi especialmente afetado pelo aparecimento da
publicidade programática \cite{Sjovaag2021}. Esta permite que os
anúncios sejam mostrados ao público geograficamente relevante, em
qualquer plataforma, deixando de ser preciso anunciar em jornais
regionais \cite{Ohlsson2017}.

Para os jornais, esta perda não é apenas do seu financiamento
publicitário, mas também da sua ligação com os negócios locais,
construída ao longo de décadas de relações interpessoais
\cite{Sjovaag2021}. O jornal regional tem vantagens que outras
plataformas não têm --- os leitores valorizam ver-se representados nos
jornais, e estes ajudam a nutrir a compreensão do outro dentro da
comunidade \cite{CosteraMeijer2016}; o jornalismo regional é também uma
fonte de ``capital social'' \cite{Hess2015}, para ligar os membros de
uma comunidade. No entanto, estas vantagens podem ser difíceis de
alavancar perante os anunciantes.

Com a sua rentabilidade cada vez menos assegurada pela publicidade, os
jornais voltam-se para os seus leitores \cite{Jenkins2017}, com dois
grandes desafios: por um lado, os seus subscritores das edições
impressas estão a envelhecer; por outro, a captura de leitores mais
jovens no meio digital tarda a acontecer \cite{Sjovaag2021}. Os dados
de 51 jornais regionais nos Estados Unidos mostram que as edições em
papel são mais lidas localmente do que as edições digitais, mesmo entre
os jovens até aos 24 anos \cite{Chyi2019}. Estes investigadores sugerem
que os jornais regionais parem de investir em ``irreais sonhos
digitais'' quando, nos EUA, são as subscrições em papel que os mantêm
vivos, uma tendência que também se verifica na Europa. A maior parte dos
jornais regionais no Reino Unido, Alemanha, França e Finlândia dependem
de 80\% a 90\% das suas subscrições do jornal em papel ou da venda de
publicidade impressa \cite{Jenkins2017}.

Em Portugal, as quedas no volume de circulação impressa paga têm sido
inferiores no jornalismo regional quando comparadas com o nacional, e a
transição para o digital tem sido mais lenta. ``Ao contrário daquilo que
acontece para a imprensa nacional, (...) os utilizadores de jornais
regionais/locais continuam a preferir o formato impresso destas
publicações, em detrimento ainda do formato online'' \cite[p.~10]{Cardoso2018}. Em 2020 ainda se registrava um maior número de peças
publicadas nas edições em papel de jornais regionais do que nas suas
edições online \cite{Alves2020}.

Quase metade dos jornalistas de publicações regionais em Portugal
trabalham para jornais que têm presença impressa e digital (45,5\%), mas
é de assinalar que ``existem mais inquiridos a trabalhar para
publicações que atuam exclusivamente no formato em papel, do que aqueles
que consideram trabalhar para publicações que atuam exclusivamente no
formato online'' \cite[p.~47]{Cardoso2018}. Os casos de
jornais que apenas têm formato impresso ``são mais comuns em regiões de
menor urbanização'' \cite[p.~24]{Ramos2021}.

A principal fonte de receita dos jornais regionais continua a ser a
publicidade no formato impresso \cite{Cardoso2018}. Dos
jornais que possuem website, quase 90\% não colocam qualquer restrição
ao acesso às notícias, e ``não possuem estratégias para arrecadação de
{[}financiamento{]} junto dos seus públicos'', ou seja, não têm
mecanismos de subscrição ou contribuição online
\cite[p.~25]{Ramos2021}. Apenas 23 dos 247 jornais contabilizados por
Ramos têm \emph{paywall}, e 18 destes têm alguns conteúdos abertos e
outros fechados. ``Nenhum jornal que só atua na Internet restringe
conteúdo, e apenas quatro pedem apoio aos leitores através de
\emph{crowdfunding}'' \cite[p.~26]{Ramos2021}. Por outro lado, vários
dos jornais com \emph{paywall }só permitem assinar a sua versão em
papel, o que acaba por resultar numa exclusão do público que pode aceder
apenas digitalmente \cite{Ramos2021}. Desta forma, a estratégia de
financiamento do jornalismo regional parece alheada das potencialidades
que a inovação tecnológica pode trazer.

\subsection{Caminhos para a inovação tecnológica}
Desde que os jornais se lançaram na Internet que a inovação no
jornalismo tem sido alvo de especial atenção académica \cite{GarciaAviles2021,Paulusen2016,Dogruel2013}, tendo conhecido um novo fôlego
com a chegada da Inteligência Artificial (IA) às redações \cite{Paschen2020}. Os estudos espelham uma preocupação generalizada em
compreender como as tecnologias emergentes moldam as práticas
jornalísticas, e quais as oportunidades e desafios que colocam ao campo
\cite{Goncalves2024,Canavilhas2023}.

Porém, a revisão da literatura deixa evidente uma lacuna: a falta de uma
definição unívoca de inovação aplicada aos media. O conceito surgiu na
área da economia e migrou para o campo do jornalismo sem que tenha
acontecido uma clarificação teórica universal, o que gera incerteza no
que é considerado inovador no jornalismo e, particularmente, no
jornalismo regional \cite{Granado2020,Posetti2018}.

No campo da economia, o conceito de inovação traduz o objetivo de
produzir maior lucro empresarial, e se no campo do jornalismo também
existem lógicas de mercado no relacionamento com os públicos, pode haver
outros objetivos por detrás da busca por inovação, desde procurar
estratégias para facilitar o trabalho jornalístico ao desenvolvimento de
novos formatos informativos \cite{Sixto-garcia2023}. No caso
concreto do jornalismo de proximidade, agregar valor à sua comunidade é
um dos objetivos primordiais da inovação \cite{Jeronimo2020,Carvalheiro2021,Correia2021}.

Um dos trabalhos seminais sobre o tema propõe quatro categorias de
inovação, conhecidas como os 4 P's: ``introduzir ou melhorar Produtos;
introduzir ou melhorar Processos; definir ou redefinir o Posicionamento
da empresa ou dos seus produtos no mercado; definir ou redefinir o
Paradigma dominante na empresa'' \cite[p.~172]{Francis2005}. Este
esquema pode ser aplicado ao jornalismo, se lhe somarmos a dimensão de
inovação social:

\begin{quote}
``A utilização inovadora dos media e dos serviços de comunicação para
fins sociais não implica necessariamente um novo produto ou serviço, mas
pode também preocupar-se em utilizar criativamente os serviços ou
produtos existentes para promover objetivos sociais''
\cite[p.~17]{Dogruel2013}
\end{quote}

Segundo \textcite{Dogruel2013}, a falta de uma definição homogênea de inovação
deve-se ao número residual de estudos teóricos em comparação com os
estudos empíricos. \textcite{Weiss2010} reforçam que a falta de
instrumentos teóricos impossibilita o estudo sistemático dos processos
de inovação nas redações. Mas a dificuldade em encontrar uma definição
unívoca pode também estar relacionada com a própria natureza do
conceito: ``não é uma fórmula fixa ou única, senão um movimento
constante'' \cite[p.~166]{Flores2017}. E, por essa razão, não será
possível delinear uma única definição capaz de abarcar toda a
complexidade e mutabilidade da inovação \cite{Flores2017}.

Uma das principais linhas de pesquisa sobre inovação no jornalismo
assenta no determinismo tecnológico, que coloca a tecnologia como
principal impulsionadora de mudança \cite{Boczkowski2004,Pavlik2000}. A inovação é, assim, vista como imperativa para a sobrevivência
de qualquer empresa na era digital \cite{Francis2005}. \textcite{pavlik2013innovation}
defende que a melhor estratégia de superação da crise dos media seria a
inovação nas práticas e nos produtos jornalísticos, mantendo o
compromisso com a qualidade e com as normas éticas que sempre guiaram a
profissão.

Contudo, perspetivar a inovação como a chave para os problemas dos media
pode ser simplista: oculta fatores de ordem estrutural, histórica,
social e cultural que condicionam as mudanças no setor \cite{Creech2018,Steensen2011}. É importante que se procure ``uma mudança de
foco, voltando-se para preocupações persistentes e historicamente
enraizadas sobre o valor democrático sustentado no jornalismo'' \cite[p.~194-195]{Creech2018}. Uma visão inovadora do jornalismo deve
enquadrar fatores económicos, sociais e culturais, bem como os valores
da profissão.

A inovação encontra-se também na adoção de estratégias para produzir
conteúdo relevante para os leitores, mantendo os princípios de
veracidade, transparência e credibilidade \cite{GarciaAviles2021}. No
entanto, é condicionada por ``restrições de recursos'' económicos e de
``competências, tempo e base de conhecimentos''
\cite[p.~182]{Francis2005}, especialmente no âmbito regional.

Inovar é um processo complexo e multidimensional em que participam
diferentes atores das esferas tecnológica, económica e social \cite{Dogruel2013,Paulusen2016}. Para \textcite{Ornebring2010}, a inovação é também
condicionada pelo ``sistema de valores já existentes, e estes sistemas
de valores têm raízes culturais, sociais e económicas'' (p.~68).

À falta de uma visão unívoca sobre a inovação nos media, escolhemos para
este trabalho interpretá-la como um processo complexo e
multidimensional, que engloba fatores de ordem social, económica,
cultural e política.

Diversos estudos recentes têm-se dedicado a analisar o impacto que a
Internet, as plataformas digitais e a IA têm tido nos media regionais.
De modo geral, são identificadas fragilidades que condicionam e atrasam
a adoção de inovações tecnológicas, entre elas, a falta de recursos
humanos e financeiros, a resistência à mudança, o medo de perder
credibilidade, a falta de formação para lidar com as novas ferramentas,
a dificuldade em contratar profissionais especializados ou mesmo a
dificuldade em adaptar o modelo de negócio ao digital \cite{Goncalves2024,Morais2023,Jeronimo2022b,Correia2021,Jeronimo2020,Morais2020,Godinho2020,Ramos2020b,Goncalves2020,Jeronimo2015,Carvalheiro2021}.

Estes fatores traduzem-se num cenário de precariedade no setor, e
condicionam a transição para o digital e o aproveitamento pleno das
plataformas digitais \cite{Alves2020}. Como resultado, os media
regionais continuam a perder território para o digital, e a publicação
de conteúdos online continua a ser deixada para segundo plano em muitos
meios regionais tradicionais \cite{Quintanilha2018,Alves2020}. O aproveitamento de tecnologias como a IA parece estar
longe de ser uma realidade em grande parte da imprensa regional em
Portugal \cite{Goncalves2024}. Contudo, esta investigação irá mostrar
que os nativos digitais de âmbito regional têm revelado uma maior
predisposição para o uso de tecnologias, ligada à necessidade de
sustentabilidade no ambiente digital.

\section{Metodologia e questões de investigação}

Este estudo segue uma abordagem qualitativa, baseada em entrevistas
semiestruturadas com os responsáveis de seis jornais regionais
portugueses. Optou-se pela pesquisa qualitativa por ser a mais adequada
para o desenvolvimento de conceitos e interpretações a partir de ideias
recolhidas \cite{Soares2019}. Dado que se pretendia aprofundar o tema
da inovação nos media regionais, identificando problemas, interpretações
e características \cite{Duarte2005,bauer2008social}, escolheu-se a
entrevista semi-estruturada como técnica metodológica. Os entrevistados
concordaram com a gravação das entrevistas para posterior transcrição e
tratamento dos dados.

A seleção da amostra seguiu dois critérios: o primeiro, eleger três
jornais regionais impressos de referência nacional e três jornais
regionais nativos digitais; o segundo, procurar uma distribuição
geográfica diversificada em Portugal Continental. Assim, foram
selecionados: ``Jornal do Fundão'', ``Região de Leiria'', ``A Voz de
Trás-Os-Montes'', ``A Mensagem de Lisboa'', ``Sul Informação'' e ``Médio
Tejo'' (\Cref{tab01}).

%%%%% TABELA
\begin{table}
    \centering
    \footnotesize
    \begin{threeparttable}
    \caption{Caraterísticas da amostra e nomes dos entrevistados}
    \label{tab01}
    \begin{tabular}{ll
    >{\raggedright\arraybackslash}p{1.2cm}
    >{\raggedright\arraybackslash}p{1.8cm}
    *{2}{>{\raggedright\arraybackslash}p{2.2cm}}}
    \toprule
    Identificação do jornal & \multicolumn{1}{>{\raggedright\arraybackslash}p{1.2cm}}{Ano de fundação} & Suporte & Região (NUTS 2) & Periodicidade &   Profissional entrevistado (função) \\
    \midrule
    1. Jornal do Fundão	& 1946 & Papel e online & Centro & Semanal (papel); Diária (online) & Nuno Francisco (Diretor) \\
    2. Região de Leiria	& 1935 & Papel e online & Centro & Semanal (papel); Diária (online) & Patrícia Duarte (Diretora-adjunta) \\
    3. A Voz de Trás-os-Montes & 1947 & Papel e online & Norte & Semanal (papel); Diária (online) & João Vilela (Diretor) \\
    4. A Mensagem de Lisboa & 2020 & Online & Área Metropolitana de Lisboa & Diária & Catarina Carvalho (Diretora) \\
    5. Sul Informação & 2012 & Online & Alentejo e Algarve & Diária & Elisabete Rodrigues (Diretora) \\
    6. Médio Tejo & 2015 & Online & Centro & Diária & Patrícia Fonseca (Diretora) \\
    \bottomrule
    \end{tabular}
    \source{elaboração própria.}
    \end{threeparttable}
\end{table}

O estudo não pretende ser representativo e portanto os seus resultados
não podem ser generalizados \cite{Stake2009}. Em vez disso, esta
investigação visa compreender a vivência nestes seis media,
estabelecendo uma comparação entre os tradicionais e os nativos digitais
selecionados.

As entrevistas foram realizadas entre os dias 16 de dezembro de 2022 e
27 de março de 2023, uma presencialmente e cinco a distância, através
das plataformas Google Meet e Zoom, tendo a duração aproximada de uma
hora. O guião continha um total de 15 questões, agrupadas em dois
blocos: o primeiro destinado à inovação tecnológica e estratégia
digital; o segundo sobre fontes de financiamento e modelos de negócio.

Para comparar as situações dos jornais tradicionais e dos nativos
digitais, formularam-se as perguntas de investigação:
\begin{itemize}
    \item[RQ1.] Como é que os diretores dos jornais estudados definem a inovação?
    \item[RQ2.] Quais as estratégias de financiamento nestes jornais?
    \item[RQ3.] De que forma a utilização de tecnologias contribui para o modelo de
negócio do jornalismo regional?
\end{itemize}
 
A transcrição e análise das entrevistas foi feita manualmente, sem
recurso a \emph{softwares}. Os investigadores elaboraram notas
explicativas, identificaram a concordância e dissonância nos temas
debatidos e construíram uma tabela com a representação gráfica dos
assuntos. A partir deste procedimento, foi feita uma análise
interpretativa dos dados, justificada a partir de excertos das próprias
entrevistas \cite{bauer2008social}. Optou-se pela análise
interpretativa das entrevistas por se entender que desta forma se
realizaria uma comparação mais fidedigna das perspetivas dos
entrevistados.


\section{Resultados}
Os resultados das entrevistas permitem ter uma visão mais aprofundada
sobre a realidade vivida nestes seis jornais regionais portugueses.

Na abertura da entrevista, procurou-se explorar o conceito de inovação
perguntando aos diretores: da atividade do jornal que dirigem, o que
consideram inovador? Cinco dos seis entrevistados identificaram
diferentes ferramentas tecnológicas, como \emph{websites, apps} ou
\emph{sites} responsivos, como a principal área de inovação.

Para as diretoras do ``Sul Informação'' e do ``Região de Leiria'',
soma-se a procura de financiamento diversificado, com candidaturas a
fundos atribuídos pela \emph{Google}, a \emph{Meta }ou a \emph{North
Star Foundation}. Patrícia Duarte (``Região de Leiria'') especifica que
o jornal deu um salto tecnológico a partir do financiamento da
\emph{Google} (DNI), que lhes permitiu contratar dois engenheiros de
\emph{software}, e efetivar uma viragem digital na redação.

\begin{quote}
``Fomos contemplados com um valor para desenvolver uma aplicação, e isso
(...) permitiu começar a efetuar mudanças na redação, ou seja,
sobretudo, mudanças que têm a ver com a nossa mentalidade.'' -- Patrícia
Duarte, ``Região de Leiria''
\end{quote}

A exceção entre estas respostas foi ``A Mensagem'', que destacou como
inovadora a sua orientação editorial focada no ``jornalismo local de
índole comunitária'', sugerindo uma visão da inovação mais associada ao
relacionamento com a comunidade e ao ângulo editorial do que ao
desenvolvimento tecnológico.

\begin{quote}
``A Mensagem faz jornalismo local de índole comunitária, ou seja, o foco
(...) não são os actores habituais dos jornais locais, as instituições
ou processos mais institucionais, mas sim as pessoas da cidade. Ou seja,
o nosso foco é sempre comunitário, de baixo para cima, de pequenas
histórias.'' -- Catarina Carvalho, ``A Mensagem de Lisboa''
\end{quote}

Questionados sobre o uso de ferramentas tecnológicas no trabalho
jornalístico, os entrevistados assinalam o uso destas em todas as fases
da produção jornalística, desde a recolha de informação até à
distribuição. Entre as ferramentas e funcionalidades mencionadas
encontram-se as notificações \emph{push}, as redes sociais, aplicações
de mensagens (como o \emph{WhatsApp}) para comunicação interna e com o
público, \emph{newsletters}, \emph{smartphones}, tripés ou microfones.
Além disso, o uso de agendas digitais partilhadas foi referido como uma
prática comum. A tecnologia permite a vários dos jornais o trabalho à
distância, tal como no ``Médio Tejo'', que nunca teve uma redação
física:

\begin{quote}
``Começamos no facto de não sermos uma redação física, desde o início.
Juntamo-nos sempre que possível, tentamos que seja pelo menos uma vez
por mês. (...) Falamos todos os dias e já usámos quase tudo, desde o
\emph{Slack}, ao \emph{Asana}, e acabámos por ficar, sete anos depois,
com as coisas mais básicas. Falamos todos os dias num grupo privado de
\emph{Messenger} no \emph{Facebook}.'' -- Patrícia Fonseca, ``Médio
Tejo''
\end{quote}

Para atrair a atenção do público online, todos os jornais utilizam as
redes sociais, com uns a darem prioridade ao \emph{Instagram}, como ``A
Mensagem'', e outros ao \emph{Facebook}, como ``A Voz de
Trás-os-Montes'', que tem essa rede social como a sua ``montra
principal''. Vários dos jornais incluídos distribuem \emph{newsletters}
para se ligarem aos seus leitores.

Surgem também estratégias fora das redes para cultivar relações com os
públicos. O ``Sul Informação'' e ``A Voz de Trás-os-Montes'' visitam
escolas para reforçar a proximidade entre a comunidade e o jornalismo
local.

Já o ``Jornal do Fundão'' e o ``Região de Leiria'', embora também
utilizem ferramentas tecnológicas para se ligarem aos leitores,
acrescentam que a sua estratégia para captação da atenção das audiências
é a sua orientação editorial: no caso do ``Jornal do Fundão'', ``fazer
bom jornalismo'' e procurar manter a credibilidade a par da velocidade.
No ``Região de Leiria'', a disponibilidade ``para ouvir {[}os
leitores{]}, para responder às interações. Se nós queremos a atenção dos
leitores, não nos podemos demitir de lhes dar atenção''. Estas respostas
denotam a importância dada à manutenção dos princípios centrais do
jornalismo, mesmo no meio digital, colocada acima da inovação
tecnológica.

Todos os diretores da amostra referiram o \emph{Google Analytics }como a
ferramenta utilizada para monitorizar as audiências, mas com
regularidades diferentes: nos nativos digitais esta monitorização é
contínua, enquanto o ``Jornal do Fundão'' e o ``Região de Leiria''
realizam balanços mensais para avaliar o desempenho dos seus jornais.

Uma grande clivagem separa os meios tradicionais e os nativos digitais
na contribuição dos leitores para o seu financiamento. Os três nativos
digitais têm conteúdos abertos e apelam a donativos nos seus
\emph{websites}.

\begin{quote}
``Começámos a ter pessoas que todos os meses nos dão algum dinheiro. E
curiosamente, também por vivermos no Algarve, temos muitos apoiantes que
são estrangeiros. Alguns residentes, e outros que estão fora mas que vêm
cá e que nos leem. E percebe-se a diferença de perspetiva: os ingleses
estão habituados a apoiar coisas, (...) e as pessoas estão habituadas a
que seja a sociedade civil a apoiar determinados projetos. (...) Também
temos, felizmente, muitos portugueses.'' -- Elisabete Rodrigues, ``Sul
Informação''
\end{quote}

Na ``A Mensagem'', as contribuições dos leitores permitem-lhes participar na
comunidade do jornal:

\begin{quote}
``Não é subscrição, é doação, \emph{membership}. As pessoas têm maneira
de nos apoiarem e em troca recebem algumas coisas como participação
privilegiada em alguns eventos e coisas desse género, e a possibilidade
de estarem connosco nisto, de estarem connosco nesta aventura.'' --
Catarina Carvalho, ``A Mensagem de Lisboa''
\end{quote}

Já os três jornais tradicionais têm conteúdos fechados a subscritores, e
respondem que procuram angariar subscrições através de informação de
qualidade e com credibilidade. João Vilela, diretor da ``Voz de
Trás-os-Montes'', acrescentou que o lançamento de um novo \emph{website}
e \emph{app} resultou num aumento dos subscritores quase para o dobro,
embora o investimento que fazem no digital venha das receitas obtidas da
publicidade impressa. Para os jornais tradicionais, o financiamento vem
principalmente do investimento dos anunciantes para se publicitarem no
papel, e das assinaturas do impresso.

Todos os jornais em análise têm uma forte dependência da publicidade. O
``Sul Informação'', que realiza candidaturas a bolsas de apoio à
imprensa regional para suplementar o seu rendimento, identifica-a como
principal fonte de receita. O ``Região de Leiria'', o ``Jornal do
Fundão'' e ``A Voz de Trás-os-Montes'' obtêm a maior parte da sua
receita através da publicidade para a edição impressa, suplementada pela
venda de publicidade digital e pelas assinaturas.

\begin{quote}
``O problema aqui é que o digital nunca teve tanto sucesso a nível de
assinaturas como teve o impresso, por diversas razões, (...) As fontes
de financiamento do jornal sempre funcionaram da mesma forma, ou seja, o
jornal vive da receita publicitária e da receita dos seus leitores.'' --
Nuno Francisco, ``Jornal do Fundão''
\end{quote}

``A Mensagem'' é uma exceção. O veículo mais recente da amostra foi
fundado tendo como principal financiador um grupo empresarial local.

São de assinalar os sinais de mudança: quatro dos seis entrevistados
afirmam que o seu modelo de negócio está em transição, com o objetivo de
reduzir a dependência da publicidade. O ``Sul Informação'' está a
experimentar apoios de entidades específicas para projetos de
investigação ou de reportagem.

\begin{quote}
``Estamos a procurar projetos em que podemos ir buscar financiamento
para determinado tipo de conteúdos. Por exemplo, para uma série de
reportagens sobre `economia azul', ver que tipo de entidades portuguesas
ou da União Europeia existem que possam financiar esse trabalho. (...)
Gostávamos de aumentar o nível de financiamento que vem por esse tipo de
projetos para pelo menos 25\%, um quarto da nossa receita vir daí,
porque isso nos daria outra tranquilidade.'' -- Elisabete Rodrigues,
``Sul Informação''
\end{quote}

O ``Região de Leiria'' e a ``Voz de Trás-os-Montes'' pretendem depender
mais do digital do que do papel, mas esse futuro pode estar distante.

\begin{quote}
``Não podemos largar o modelo tradicional em papel, porque ainda temos
muitos anunciantes e muitos assinantes em papel.'' -- Patrícia Duarte,
``Região de Leiria''
\end{quote}

O ``Médio Tejo'' considera que, no seu caso, ``a mudança no modelo de
negócio está a acontecer até 2024'', com intenção de passar a criar
conteúdos só para assinantes, como ``\emph{podcasts} e \emph{stories}''.
Embora no passado tenham feito experiências com um mecanismo de
contribuições voluntárias dos leitores, não obtiveram resultados
significativos.

Apenas o ``Jornal do Fundão'' assume um modelo de negócio ``clássico, ou
seja, produção noticiosa que é alavancada economicamente pelos seus
leitores'' através de vendas em banca ou subscrições, com a publicidade
como fonte principal de financiamento -- um modelo do qual dependem
``desde sempre'' (com fundação em 1946), e que não têm planos concretos
de mudar. No entanto, o diretor Nuno Francisco assume o decréscimo da
receita tanto do lado dos leitores como do lado da publicidade, e
assinala que o número de páginas do jornal impresso já foi reduzido para
cortar custos. Recusa porém o rótulo de conservador no uso de
tecnologias, e explica:

\begin{quote}
``A tendência é ficar agarrado ao modelo tradicional que ainda garante
retorno financeiro, embora esteja em degradação. Ainda não foi
encontrada uma fórmula de encontrar retorno online. (...). E isso pode
travar o uso de tecnologias e resultar no tal conservadorismo de que se
fala.'' -- Nuno Francisco, ``Jornal do Fundão''
\end{quote}

Para atrair os anunciantes, os jornais tradicionais apostam na sua
reputação e prestígio acumulados ao longo do tempo, e no relacionamento
saudável com as empresas locais. No ``Região de Leiria'', essa qualidade
jornalística é aliada à inovação: a diretora-adjunta Patrícia Duarte
sublinha que, ``queira {[}o anunciante{]} anunciar no jornal impresso,
na revista, ou no \emph{podcast}, na \emph{newsletter}, temos uma
panóplia de formatos''.

Do lado dos nativos digitais, menciona-se a divulgação de métricas como
\emph{pageviews} e \emph{unique users} para cativar investimento
publicitário, embora Catarina Carvalho se afirme ``surpreendida'' por
ainda não haver ``diferença de paradigma'' para se valorizarem mais os
valores de interação nas redes ou de leitores recorrentes.

Os diretores têm visões díspares da inovação tecnológica. Para o
``Jornal do Fundão'' e ``A Voz de Trás-os-Montes'', os benefícios das
tecnologias para o modelo de negócio ainda não são totalmente claros, e
a sustentabilidade no digital afigura-se incerta.

\begin{quote}
``A imprensa regional, não toda mas uma parte substancial, tem vindo a
ficar retraída nos avanços tecnológicos, precisamente porque estamos
agarrados ao modelo tradicional porque é aquilo que ainda nos dá
garantias, embora esteja em evidente degradação. Sabemos que vamos
chegar a um ponto em que esse modelo não vai resistir. A tecnologia é
importantíssima e é fundamental para conhecer os públicos. (...) Temos
de abrir algumas portas para alguns caminhos que possam complementar a
estrutura tradicional.'' -- Nuno Francisco, ``Jornal do Fundão''
\end{quote}

Para o ``Sul Informação'' e o ``Médio Tejo'', as ferramentas
tecnológicas são a base do modelo de negócio, permitindo automação de
pagamentos e novas formas de promoção.

Todos os diretores concordam que as inovações tecnológicas podem
beneficiar o jornalismo regional. Os diretores de ``A Mensagem'' e do
``Jornal do Fundão'' veem a tecnologia como uma forma de aproximar os
jornais das suas comunidades.

\begin{quote}
``Todos os nossos eventos são difundidos através da tecnologia, são
comunicados através da tecnologia, são respondidos através da
tecnologia... Há uma forte componente da `vida real' mas há também uma
forte componente desta relação com a comunidade, que passa pela
tecnologia.'' -- Catarina Carvalho, ``A Mensagem de Lisboa''
\end{quote}

Para a diretora do ``Médio Tejo'', pode trazer ``uma revitalização da
imprensa regional''.

\begin{quote}
``Eu adoro o papel, e vou gostar sempre de papel. Não digo que vá
acabar, (...) acho que vai ser cada vez mais uma coisa \emph{premium} e
para longos formatos, e mais de nicho. (...) Por um lado {[}a
tecnologia{]} pode ser a salvação de muitos projetos que não têm a
capacidade de continuar com os custos de impressão, e por outro pode ser
a oportunidade de nascerem novos projetos, com ferramentas gratuitas ou
quase gratuitas.'' -- Patrícia Fonseca, diretora do ``Médio Tejo''
\end{quote}

No ``Região de Leiria'', considera-se que a tecnologia pode um dia
ajudar a colmatar a falta de recursos humanos nas redações, através da
automação de certas tarefas:

\begin{quote}
``A equipa de um jornal é cara, se conseguirmos que a tecnologia nos
ajude a colmatar algumas das lacunas que temos, excelente. Não acredito
que o fator humano seja dispensável, porque será sempre necessário fazer
as perguntas certas às pessoas certas. Mas se houver certas tarefas que
possam ser automatizadas, por que não, talvez seja esse o caminho do
futuro.'' -- Patrícia Duarte, ``Região de Leiria''
\end{quote}

Embora o uso de IA de modo intencional ainda seja residual em media
regionais \cite{Goncalves2024}, começa a existir uma maior abertura
para o seu uso, uma vez que esta tecnologia está incorporada em
plataformas digitais como o \emph{Wordpress}. Exemplo disso é o ``Sul
Informação'', que após a data desta entrevista, implementou um
\emph{chatbot }destinado a pedidos da audiência para verificação de
conteúdos ou sugestão de temas, e, segundo a diretora, já obteve
participações relevantes dos leitores\footnote{A
  implementação do \emph{chatbot} ocorreu após a entrevista realizada
  para os propósitos deste artigo. Esta perceção da diretora do ``Sul
  Informação'' resultou de uma conversa informal no contexto das
  comemorações do Dia Nacional da Imprensa, no dia 17 de maio, em
  Lisboa, e a informação foi posteriormente confirmada com ela antes de
  ser incluída neste artigo.}.

Os resultados indicam que estes media regionais portugueses convergem no
uso de plataformas digitais em todas as fases do processo jornalístico,
referindo esse uso como uma estratégia de inovação no conteúdo, no
trabalho e na captação de audiência. Contudo, as abordagens à inovação e
ao modelo de negócio diferem entre os nativos digitais e os jornais
tradicionais. Embora todos considerem inovador o investimento feito em
\emph{sites} responsivos e \emph{apps}, os nativos digitais propõem
definições de inovação mais variadas, desde a diversificação dos modelos
de financiamento até à abordagem editorial de proximidade. Entre os
jornais tradicionais, ressurge o foco nos valores da profissão, como a
credibilidade e a relação de confiança com os leitores.

No modelo de negócio, a clivagem é particularmente acentuada. Os nativos
digitais mantêm os seus conteúdos abertos, apelando a contribuições
voluntárias, enquanto os tradicionais têm o acesso digital parcialmente
limitado para não subscritores, e dependem da publicidade impressa para
se financiarem, com caminho ainda a percorrer na transição para o
digital.

Todos os jornais procuram diversificar as suas receitas, mas os nativos
digitais demonstram um melhor aproveitamento dos recursos online, como o
trabalho remoto, que permite poupar os custos associados ao espaço da
redação, ou a diversificação de conteúdos, como os stories nas redes
sociais.


\section{Discussão e conclusões}
Este trabalho promove uma discussão sobre o que pode ser considerado
inovação nos media regionais, sobretudo se considerarmos, como nos diz
\textcite[p.~7]{Posetti2018}, que ``a indústria noticiosa tem um problema de foco
(...) uma busca obsessiva por tecnologia e a ausência de estratégias
claras e informadas para a sua aplicação''.

Respondendo às questões de investigação, a começar pela forma como os
entrevistados definem o que é inovação (RQ1), esta é percebida através
de três dimensões. O primeiro eixo relaciona este conceito com o uso de
tecnologias e a adaptação às plataformas digitais, uma perspetiva da
inovação no seu aspeto tecnológico, ou a categoria de `produto' de
\textcite{Francis2005}.

Um segundo ponto de vista é o da inovação nas formas diversificadas de
financiamento, associadas à sobrevivência financeira do jornal. Aqui
podemos entrever uma coincidência com o proposto pelo determinismo
tecnológico, que sugere que a sobrevivência dos media virá da inovação
\cite{Francis2005,pavlik2013innovation}.

Um último ponto de vista é o da responsável de ``A Mensagem'', que
considera que a inovação reside na orientação editorial social e
comunitária. Esta visão aproxima-se da dimensão de inovação social de
Dogruel (2013), ao oferecer um serviço jornalístico diferenciado com um
objetivo social. Uma perspetiva que pode estar contextualizada pelo
sustento deste meio (através de um grupo económico privado) e a sua
maior liberdade em relação a estruturas de poder, como as autarquias.

Já no que diz respeito às estratégias de financiamento dos jornais
regionais (RQ2), continua a verificar-se uma predominância da
publicidade, embora o panorama se comece a diversificar. Do lado dos
jornais impressos, há indicações de que a transição digital ainda está
em curso, o que é sustentado por estudos anteriores \cite{Cardoso2018,Sjovaag2021}, e a receita vem ainda principalmente
do impresso, seja pela publicidade, seja pelas subscrições. O mesmo
acontece no jornalismo regional nos Estados Unidos, Reino Unido,
Alemanha, França e Finlândia \cite{Jenkins2017,Chyi2019}. \textcite{Chyi2019} sugerem mesmo que os jornais locais
larguem ``irreais sonhos digitais'' e se dediquem ao seu produto de
sucesso --- o jornal impresso. Uma sugestão que pode não ser assim tão
descabida para o panorama português, onde os leitores de jornais locais
também preferem lê-los em papel \cite{Cardoso2018}, embora se
registem quedas nos números de assinantes \cite{Foa2016}.

Por outro lado, os nativos digitais seguem uma estratégia diferente. ``A
Mensagem'' destaca-se por ter o apoio do grupo empresarial ``Valor do
Tempo'', que suplementa com bolsas de jornalismo. Já o ``Sul
Informação'' e o ``Médio Tejo'' exploram o financiamento através de
projetos e de subsídios à imprensa regional. Foram ainda mencionados
recursos alternativos como os donativos dos leitores. Inovar no modelo
de negócio do jornalismo significa aproveitar as plataformas digitais e
combinar diferentes fontes de receita \cite{canavilhas2015nuevos}, e estes
resultados evidenciam uma diferença entre os nativos digitais que, ainda
dependentes da publicidade, aparentam estar mais dedicados à exploração
de recursos múltiplos, e os jornais tradicionais que, mesmo afirmando
estar num processo transitório, ainda sobrevivem a partir do modelo de
negócio clássico. Também encontramos nesta amostra uma correspondência
com o averiguado por Ramos (2021), em que os jornais tradicionais
aplicaram já uma \emph{paywall} que requer um pagamento para aceder aos
conteúdos, enquanto os nativos digitais preferem os conteúdos abertos e
subscrições voluntárias. Até a data deste artigo, o Médio Tejo ainda não
implementou uma \emph{paywall}, apesar da menção dessa intenção na
entrevista.

Por fim, procurou-se saber de que forma a utilização de tecnologias
contribui para o modelo de negócio do jornalismo regional (RQ3).
Identificaram-se novas estratégias de captação de audiências e
otimização de processos da produção jornalística. Ferramentas como
\emph{Google Analytics}, notificações \emph{push}, redes sociais e
\emph{apps} de mensagem são utilizadas em todas as fases da produção
jornalística nas seis redações.

Os diretores destes media regionais mostram-se atentos às novas
exigências do mercado e reforçam que estão numa fase de transição,
mantendo firme o objetivo de revitalizar a imprensa regional e encontrar
formas sustentáveis de financiamento no digital, sem nunca perder a
credibilidade e compromisso com as suas comunidades.

Este estudo apresenta como principal limitação a sua amostra reduzida,
impossibilitando conclusões generalizadas. Esta investigação enriquece a
literatura existente com uma visão comparada entre os jornais
tradicionais e os nativos digitais. Pesquisas futuras poderão avaliar o
estado do processo de transição digital e as perspetivas de
sobrevivência do modelo impresso, cujo destino parece ter sido anunciado
cedo demais. Também seria importante explorar o esbatimento das
fronteiras entre os modos de trabalho e o uso da tecnologia nos nativos
digitais e nos jornais tradicionais. Para além disso, o uso de
Inteligência Artificial por meios regionais deverá ser objeto de estudo,
uma vez que esta tecnologia se encontra em expansão.


\printbibliography\label{sec-bib}
%conceptualization,datacuration,formalanalysis,funding,investigation,methodology,projadm,resources,software,supervision,validation,visualization,writing,review
\begin{contributors}[sec-contributors]
\authorcontribution{Marta Santos Silva}[conceptualization,investigation,methodology,writing,review]
\authorcontribution{Adriana Gonçalves}[conceptualization,investigation,methodology,writing,review]
\authorcontribution{Ricardo Morais}[validation,review]
\end{contributors}
\end{document}
