% !TEX TS-program = XeLaTeX
% use the following command:
% all document files must be coded in UTF-8
\documentclass[english]{textolivre}
% build HTML with: make4ht -e build.lua -c textolivre.cfg -x -u article "fn-in,svg,pic-align"
\usepackage{tabularx}
\usepackage{array}
%\usepackage{makecell}
\usepackage{longtable}

\journalname{Texto Livre}
\thevolume{18}
%\thenumber{1} % old template
\theyear{2025}
\receiveddate{\DTMdisplaydate{2025}{2}{16}{-1}} % YYYY MM DD
\accepteddate{\DTMdisplaydate{2025}{5}{12}{-1}}
\publisheddate{\DTMdisplaydate{2025}{8}{11}{-1}}
\corrauthor{Dana Mahadin}
\articledoi{10.1590/1983-3652.2025.57583}
%\articleid{NNNN} % if the article ID is not the last 5 numbers of its DOI, provide it using \articleid{} commmand 
% list of available sesscions in the journal: articles, dossier, reports, essays, reviews, interviews, editorial
\articlesessionname{articles}
\runningauthor{Mahadin, Olimat and Almahasees} 
%\editorname{Leonardo Araújo} % old template
\sectioneditorname{Daniervelin Pereira}
\layouteditorname{Saula Cecília}

\title{Artificial Intelligence in translation studies: a cross-sectional survey of Jordanian academics’ knowledge, attitudes and practices}
\othertitle{Inteligência Artificial em estudos de tradução: uma pesquisa transversal sobre conhecimento, atitudes e práticas de acadêmicos jordanianos}
% if there is a third language title, add here:
%\othertitle{Artikelvorlage zur Einreichung beim Texto Livre Journal}

\author[1]{Dana Mahadin~\orcid{0000-0003-1488-0515}\thanks{Email: \href{mailto:d.mahadin@bau.edu.jo}{d.mahadin@bau.edu.jo}}}
\author[2]{Sameer Naser Olimat~\orcid{0000-0002-2767-6751}\thanks{Email: \href{mailto:solimat@hu.edu.jo}{solimat@hu.edu.jo}}}
\author[3]{Zakaryia Almahasees~\orcid{0000-0002-4035-7165}\thanks{Email: \href{mailto:z_almhasees@asu.edu.jo}{z\_almhasees@asu.edu.jo}}}
\affil[1]{Al-Balqa Applied University, Salt Faculty of Human Sciences, English Language Department, Salt, Jordan.}
\affil[2]{The Hashemite University, Faculty of Arts, Department of English Language and Literature, Zarqa, Jordan.}
\affil[3]{Applied Science Private University, Faculty of Arts and Science, Department of English Language and Translation, Amman, Jordan.}

\addbibresource{article.bib}
% use biber instead of bibtex
% $ biber article

% used to create dummy text for the template file
\definecolor{dark-gray}{gray}{0.35} % color used to display dummy texts
\usepackage{lipsum}
\SetLipsumParListSurrounders{\colorlet{oldcolor}{.}\color{dark-gray}}{\color{oldcolor}}

% used here only to provide the XeLaTeX and BibTeX logos
\usepackage{hologo}

% if you use multirows in a table, include the multirow package
\usepackage{multirow}

% provides sidewaysfigure environment
\usepackage{rotating}

% CUSTOM EPIGRAPH - BEGIN 
%%% https://tex.stackexchange.com/questions/193178/specific-epigraph-style
% \usepackage{epigraph}
% \renewcommand\textflush{flushright}
% \makeatletter
% \newlength\epitextskip
% \pretocmd{\@epitext}{\em}{}{}
% \apptocmd{\@epitext}{\em}{}{}
% \patchcmd{\epigraph}{\@epitext{#1}\\}{\@epitext{#1}\\[\epitextskip]}{}{}
% \makeatother
% \setlength\epigraphrule{0pt}
% \setlength\epitextskip{0.5ex}
% \setlength\epigraphwidth{.7\textwidth}
% CUSTOM EPIGRAPH - END

% to use IPA symbols in unicode add
%\usepackage{fontspec}
%\newfontfamily\ipafont{CMU Serif}
%\newcommand{\ipa}[1]{{\ipafont #1}}
% and in the text you may use the \ipa{...} command passing the symbols in unicode

% LANGUAGE - BEGIN
% ARABIC
% for languages that use special fonts, you must provide the typeface that will be used
% \setotherlanguage{arabic}
% \newfontfamily\arabicfont[Script=Arabic]{Amiri}
% \newfontfamily\arabicfontsf[Script=Arabic]{Amiri}
% \newfontfamily\arabicfonttt[Script=Arabic]{Amiri}
%
% in the article, to add arabic text use: \textlang{arabic}{ ... }
%
% RUSSIAN
% for russian text we also need to define fonts with support for Cyrillic script
% \usepackage{fontspec}
% \setotherlanguage{russian}
% \newfontfamily\cyrillicfont{Times New Roman}
% \newfontfamily\cyrillicfontsf{Times New Roman}[Script=Cyrillic]
% \newfontfamily\cyrillicfonttt{Times New Roman}[Script=Cyrillic]
%
% in the text use \begin{russian} ... \end{russian}
% LANGUAGE - END

% EMOJIS - BEGIN
% to use emoticons in your manuscript
% https://stackoverflow.com/questions/190145/how-to-insert-emoticons-in-latex/57076064
% using font Symbola, which has full support
% the font may be downloaded at:
% https://dn-works.com/ufas/
% add to preamble:
% \newfontfamily\Symbola{Symbola}
% in the text use:
% {\Symbola }
% EMOJIS - END

% LABEL REFERENCE TO DESCRIPTIVE LIST - BEGIN
% reference itens in a descriptive list using their labels instead of numbers
% insert the code below in the preambule:
%\makeatletter
%\let\orgdescriptionlabel\descriptionlabel
%\renewcommand*{\descriptionlabel}[1]{%
%  \let\orglabel\label
%  \let\label\@gobble
%  \phantomsection
%  \edef\@currentlabel{#1\unskip}%
%  \let\label\orglabel
%  \orgdescriptionlabel{#1}%
%}
%\makeatother
%
% in your document, use as illustraded here:
%\begin{description}
%  \item[first\label{itm1}] this is only an example;
%  % ...  add more items
%\end{description}
% LABEL REFERENCE TO DESCRIPTIVE LIST - END


% add line numbers for submission
%\usepackage{lineno}
%\linenumbers

\begin{document}
\maketitle

\begin{polyabstract}
\begin{abstract}
Artificial Intelligence (AI) is revolutionizing translation teaching, training, and practices, yet its adoption amongst the translation studies community remains under-researched. The current study investigates Jordanian translation academics’ knowledge, attitudes and AI practices in teaching and academic research. The study employs a 5-likert scale questionnaire to explore translation academics’ awareness and attitudes about AI, their use of AI in research practices and their perception of AI use in the classroom. The study findings suggest that despite academics’ awareness of AI and its applications, they are reluctant to use it in their research practices. Furthermore, academics are hesitant to use AI in the translation classroom and more research is required to explore the underlying reasons behind this hesitation. The study recommends further research to understand the impact of AI adoption on teaching and learning, AI implications for academics and students’ training needs, and the role of higher education institutions in addressing AI within academic environments.

\keywords{Translation studies\sep Translation training\sep Artificial Intelligence\sep Knowledge, attitudes and practices\sep Jordanian higher education institutions}
\end{abstract}

\begin{portuguese}
\begin{abstract}
A Inteligência Artificial (IA) está revolucionando o ensino, a formação e as práticas de tradução, contudo, sua adoção na comunidade de estudos da tradução ainda é pouco pesquisada. O presente estudo investiga o conhecimento, as atitudes e as práticas de IA no ensino e na pesquisa acadêmica entre docentes jordanianos de tradução. A pesquisa utiliza um questionário com escala Likert de 5 pontos para explorar a conscientização e as percepções dos acadêmicos sobre a IA, além do uso dessa tecnologia em práticas de pesquisa e no ambiente de sala de aula. Os resultados sugerem que, apesar do conhecimento dos acadêmicos sobre a IA e suas aplicações, há uma relutância em utilizá-la em suas pesquisas. Além disso, os docentes demonstram hesitação em incorporar a IA ao ensino da tradução, sendo necessária uma investigação mais aprofundada para compreender as razões subjacentes a essa resistência. O estudo recomenda pesquisas adicionais para entender o impacto da adoção da IA no ensino e na aprendizagem, suas implicações para os acadêmicos e as necessidades de formação dos estudantes, bem como o papel das instituições de ensino superior na abordagem da IA em contextos acadêmicos.

\keywords{Estudos de tradução\sep Treinamento de tradução\sep Inteligência Artificial\sep Conhecimentos, atitudes e práticas\sep Instituições de ensino superior da Jordânia}
\end{abstract}
\end{portuguese}
% if there is another abstract, insert it here using the same scheme
\end{polyabstract}

\section{Introduction}\label{sec-intro}
Artificial intelligence (AI) is defined as:

\begin{quote}
the science and engineering of making intelligent machines, especially intelligent computer programs. It is related to the similar task of using computers to understand human intelligence, but AI does not have to confine itself to methods that are biologically observable \cite[p. 2]{mccarthy2007}.
\end{quote}

The recent advancements in AI have been projected to have a significant impact on everything, including individuals and organizations \cite{dwivedi2023}. One such development has been the meteoric rise of Generative Pre-Trained Transformer (ChatGPT). ChatGPT is a large language model (LLM) which had over 100 million users within the first two months of its launch. Its quick prevalence prompted UNESCO to launch its own quick start guide for its use \cite{unesco2023}. Although ChatGPT is not the only LLM available online, it is probably the most prominent. A Google research yields more than 668,000,000 results on it alone. In addition to the large natural language models, other AI applications are gaining momentum and there are numerous tools and systems dedicated to text creation and paraphrase, image design and formation, presentation construction, and speech recognition, among others.

The explosion of AI online models, such as ChatGPT, has similarly ignited lively debates and controversy in academia about its use in health care \cite{kitamura2023}, business \cite{george2023}, education \cite{lo2023}, student use \cite{cotton2023}, student perceptions \cite{chan2023}, assessment \cite{smolansky2023}, tourism \cite{ali2023}, academia and libraries \cite{lund2023}, in language learning \cite{hong2023}, and translation \cite{sahari2024, shi2025}, to mention but a few. Academics have even utilized ChatGPT in writing research papers \cite{oconnor2023}. This has led several academic publishing houses to update their authorship guidelines \cite{elsevier2023, taylorfrancis2023}. Concerns about the abilities of evolving AI, such as GPT-3 and Auto GPT in content creation, and the inability to distinguish between human generated text and ChatGPT text have raised questions about the creation of fake news, manipulation of public opinion, and misinformation \cite{floridi2020}. There are also concerns about the sustainability of traditional assessment tools in education \cite{rudolph2023, stokel-walker2022}, and more existential questions about the role of higher education \cite{kramm2023}.

In Translation Studies (TS), recent studies have shown AI models’ popularity amongst students \cite{sahari2023}, evaluated human translators and ChatGPT translations of figurative language \cite{sahari2024}, addressed the performance of GPT-4 in dubbing Egyptian comedy and highlighted its superiority to human translators \cite{abu-rayyash2024}, and investigated AI utility as a promising translation training pedagogical platform \cite{xu2025}. More research is needed, however, to understand how AI can be integrated into translation pedagogy and how it will impact different aspects of the TS community. Such research is necessary to enable TS academics to respond to AI integration demands and challenges and enhance academics’ ability to provide their students with the tools and skills that allow them to navigate this transformation. This study comes to survey Jordanian translation academics’ knowledge, attitudes and practices of AI and their employment of AI in research practices and classrooms. The next section provides a brief review of existing literature on using AI, particularly, in translation.

\section{Literature Review}
This section presents a general overview of the literature on the use of AI with particular focus on translation. TS is not new to technological innovation. It has long embraced numerous technological tools, such as computer-assisted tools (CAT), terminology management tools (TMT), Quality Assurance tools (QA), Corpus Analysis tools, Translation Memory (TM), and Machine Translation (MT), to mention but a few. There have been numerous studies on incorporating these technological tools in the training environment of translation. For example, research has looked at electronic corpora \cite{rodriguez-ines2010}, examined free open source software for the translation classroom \cite{florez2011}, assessed the design and evaluation of statistical MT syllabus for translation students \cite{doherty2014}, investigated neural MT in the classroom \cite{moorkens2018}, researched student autonomy in CAT teaching \cite{vieira2021} and explored trends in Audiovisual Translation (AVT) technologies \cite{bolanos-garcia-escribano2021}. Furthermore, TS scholars have continued to call for improved integration of technology in Translator Training Programs (TTPs) \cite{mahadin2018, tao2022} in managing crises \cite{mahadinolimat2022, mahadinolimat2023}, and in bridging the gap between the industry and academia \cite{gaspari2015, sanchez-castany2022}.

In addition, all TS competence models and frameworks, such as the EMT competence framework \cite{emt2017, europeanunion2022}, have acknowledged the significant role of technology within the field of translation and have emphasized the need to equip graduates for work in a ``dynamic and highly technologized workplace" \cite[p. 2]{europeanunion2022}. The EMT competence frameworks list a number of technological skills that students need to develop, including using IT applications, adapting to new technologies, managing different forms of digital content, and applying other tools ``in support of language and translation technology" \cite[p. 9]{europeanunion2022}. AI is projected to be the biggest disrupter for teaching, learning, and academic research \cite{dwivedi2023}. 

The introduction and the subsequent evolutions of LLMs, such as ChatGPT, have sparked scholarly interest in TS, particularly within MT. Different studies have examined ChatGPT’s performance against different MT models and systems \cite{kocmi2022, gao2023, peng2023, lu2023, wang2023, hendy2023}. The majority of these studies have shown the potential of ChatGPT against other established systems. These studies, however, are not without their limitations. On the one hand, some of the studies have either focused on general domains \cite{kocmi2022} limiting investigation of ChatGPT performance ability on specialised jargon, or on high-resource languages \cite{gao2023}. The focus on high-resource languages mainly raises questions about ChatGPT’s potential for low-resource languages. \textcite{hendy2023} addressed this limitation by comparing both high-resource and low-resource languages in 18 translation directions. While GPT models were found to perform very well for high-resource languages, their capabilities were limited for low-resource languages. A hybrid model that combines GPT and other systems may go a long way in enhancing translation quality. Using effective prompts has also been a key element in delivering good ChatGPT outcomes \cite{peng2023}. In the same vein, ChatGPT has been found to have the potential to become ``a new and promising paradigm for document-level translation" \cite[p. 1]{wang2023}, provided that well-designed prompts are used.

ChatGPT integration in translator training in the classroom has become another research area. For example, \textcite{sahari2023} reported on translation students’ preference for using ChatGPT instead of Google Translate. This preference may be due to ChatGPT’s outperformance in mechanical tasks like typing, spelling and grammar checks, and paraphrasing. By contrast, instructors leaned towards Google Translate, as ChatGPT was less effective in tasks that required nuanced judgements such as fine tuning and domain-specific translation. The study is limited in its sample size; only 15 students and 19 instructors, making generalization difficult. Integrating ChatGPT into the translation classroom via a platform is examined by \textcite{xu2025}. The platform provides real time feedback, ability to customize teaching scenarios per subject, and track learner performance. However, challenges, such as ethical considerations, over-reliance on AI output and limitation in creativity, were underscored, supporting previous findings such as \textcite{sahari2023}. Even though AI platforms may have their limitations, such as lack of robust assessment metrics for feedback, and absence of strategies to prevent AI over-reliance, they may be beneficial as a tool in translator training contexts \cite{xu2025}.

In addition to translator training, professional translators’ perceptions towards AI and its adoption in translation workflows are gaining momentum. In more detail, professional translators’ knowledge, attitudes, concerns, and experience of ChatGPT translation quality were examined \cite{shi2025}. ChatGPT was viewed as an effective tool for drafting and proofreading and as a complementary tool rather than a replacement. Again, challenges in addressing cultural sensitivities, ethical concerns, data security and accuracy and consistency were reported. Other studies have reported on professional translators’ use of AI primarily to enhance both translation quality and productivity through contextual meaning identification, word search, sentence restatements and summaries \cite{farrell2024}. While a significant link was found between MT and AI use by professional translators, ChatGPT was the most popular among them \cite{farrell2024}. Although professional translators’ perceptions towards AI use in the workplace were investigated, part-time and freelance translators’ fears and concerns over AI replacement remain under-researched. In addition, examining academics’ views may provide a broader understanding of the impact of AI technologies in translator training, which is the scope of the present study.

Notwithstanding the significant potential of AI for TS, as these previous studies have shown, it is also important to examine how AI will affect other aspects of the translation industry. According to \textcite{pym2023}, the logical approach would be to test AI in translation empirically, compare it within the larger context of automation that TS has already witnessed, and to determine and propose translation training needs, as well as academics’ practices and adoption of AI systems and tools. The present study investigates the knowledge, attitudes and practices of translation academics in AI in their profession. To the best of the researcher’s knowledge, there has been little research on this issue thus far in the Jordanian context. The next section describes the adopted methodology to achieve the research purposes.

\section{Methodology}
This section presents a description of the study instrument, recruiting participants, and data collection and analysis. To investigate the knowledge, attitude and practices of Jordanian translation academics towards AI in their academic careers, a close-ended 5-Likert scale questionnaire was developed. The questionnaire consisted of 18 statements exploring three main areas, namely, academics’ knowledge and attitudes about AI and their AI research and classroom practices. The questionnaire sections and statements were constructed after consulting existing literature.

To check the validity and reliability of the questionnaire, a draft version was sent to three expert linguistics and translation professors in different Jordanian universities to evaluate language clarity and statement construction. All the expert recommendations, which focused mainly on the order of the statements and clarity of some expressions and terms, were considered and incorporated into the final version of the questionnaire. In addition, a pilot study with 10 academics was conducted, and their responses were excluded from the final analysis.

For ethical considerations, ethical approval was obtained through the Institutional Review Board (IRB) committee of the Hashemite University (No. 35/4/2023/2024). The survey also included an introduction on the significance and objectives of the research and a consent form. The consent form underscored respondents’ rights in terms of their voluntary participation, right to withdrawal, anonymity and contact details of the researchers for any questions. To ensure confidentiality, the participants were informed that the collected data would be password-protected, with access only to the researchers, and that only anonymized data would be published in academic journals. The second part of the questionnaire elicited demographic information about the respondents. The main part included the questionnaire statements and was divided into three main sections: Knowledge and attitudes about AI, AI practices in academic research, and AI practices in the translation classroom. The final version was developed using Google Forms and was distributed to translation academics working in several Jordanian public universities. The questionnaire was open for two weeks and was filled by 80 participants. 

The next sub-section will present an analysis of the respondent demographics, followed by the data findings and discussion of the study results.

\subsection{Participant Demographic Information}
Table \ref{tab-1} below illustrates the gender and academic rank distribution of the study respondents. It shows that the number of academics participating in the survey was 80 academics from several public universities in Jordan. Of the 80 participants, 42, constituting 52.5\% of the respondents, were female, while 38, or 47.5\%, were male. The academic rank distribution of the participants was as follows: 10 professors, 32 associate professors, 30 assistant professors, and 8 instructors.

%---- CÓDIGO DA TABELA 1 ----%
\begin{table}[ht]
\centering
\begin{threeparttable}
\caption{Respondent demographics by academic rank and gender.}\label{tab-1}
\begin{tabular}{lcc}
\toprule
Academic Rank & Male & Female \\
\midrule
Professor           & 8  & 2  \\
Associate Professor & 14 & 18 \\
Assistant Professor & 12 & 18 \\
Instructor          & 4  & 4  \\
\bottomrule
\end{tabular}
\source{The authors.}
\end{threeparttable}
\end{table}

\section{Data Analysis}

\subsection{Academics’ Knowledge and Attitudes about AI}

This section presents the findings on participants’ knowledge and attitude about AI, as shown in Table \ref{tab-2} below. In statement one, 85\% of the participants agreed that they were aware of AI and its uses in the fields of education, academia, and research. In contrast, 10\% were not aware of the concept of AI, and 5\% of the participants provided a neutral response. In statement two, about three quarters of the participants signalled their knowledge of the latest developments in the area of AI while 12.5\% of the participants were neutral, and 15\% were not aware of recent advancements in AI. 

A large proportion of the academics, making up 70\% of the participants, indicated that they were aware of AI tools and applications devoted to several academic tasks, such as writing, image creation, brainstorming, conversation, and research. However, 15\% of the participants provided a neutral response, and similarly, 15\% of them were not aware of these AI tools and applications. Regarding the use of AI in the translation classroom, more than 60\% of the participants were aware that AI can be employed for different educational purposes, such as lesson planning, quizzes, and assessments. However, 17.5\% of the academics were neutral, and only 20\% showed unawareness of the AI’s use in translation classes.

In the final statement, academics were asked if they were aware of the use of AI to automate repetitive tasks. Only 12.5\% of the respondents agreed, 50\% disagreed, and 37.5\% were neutral. This indicates that Jordanian academics are not aware of the capabilities AI has in automating some repetitive academic duties, such as scheduling meetings, responding to queries, detecting plagiarism, and designing learning experiences.

To sum up, the data analysis of the participants’ response to the first section shows that Jordanian translation academics had a high level of knowledge and awareness of AI. They also expressed adequate knowledge of AI’s use for educational goals. However, the participants had a low level of awareness of how AI may automate repetitive tasks commonly featured in their translation teaching practices.

%---- CÓDIGO DA TABELA 2 ----%
\begin{small}
\begin{longtable}{l >{\raggedright\arraybackslash}p{3.85cm}lllll}
\caption{Knowledge and Attitudes about AI.}\label{tab-2} \\
\toprule
\# & Statement & \multicolumn{1}{>{\raggedright\arraybackslash}p{1.5cm}}{Strongly agree} & Agree & Neutral & Disagree &  \multicolumn{1}{>{\raggedright\arraybackslash}p{1.5cm}}{Strongly disagree} \\
\midrule
\endfirsthead

\bottomrule
\source{The authors.}
\endlastfoot

1 & I am aware of the concept of AI and its current use in different areas such as education, academia, etc. & 32 (40\%) & 36 (45\%) & 4 (5\%) & 4 (5\%) & 4 (5\%) \\
2 & I am aware of the vast developments in AI in the past few months. & 28 (35\%) & 30 (37.5\%) & 10 (12.5\%) & 10 (12.5\%) & 2 (2.5\%) \\
3 & I am aware that there are dedicated AI tools and applications for different tasks such as writing, image creation, brainstorming, conversation, research etc. & 26 (32.5\%) & 30 (37.5\%) & 12 (15\%) & 12 (15\%) & 0 (0.0\%) \\
4 & I am aware that AI can be used for different educational purposes, i.e. lesson planning, quizzes, assessments, etc. & 18 (22.5\%) & 32 (40\%) & 14 (17.5\%) & 16 (20\%) & 0 (0.0\%) \\
5 & I am aware that AI can be used to automate some repetitive tasks, such as scheduling student-teacher meetings, responding to common student queries, detecting student’s plagiarism, and designing individual learning experiences. & 2 (2.5\%) & 8 (10\%) & 30 (37.5\%) & 26 (32.5\%) & 14 (17.5\%) \\
\end{longtable}
\end{small}

\subsection{AI Practices in Academic Research}

This section elicited information via six statements on academics’ use of AI tools for academic research purposes, as shown in Table \ref{tab-3}. In statement one, more than half of the respondents expressed disagreement on the use of AI tools to develop research ideas. In addition, 17.5\% of the respondents were neutral, and 30\% agreed with the statement. There was a similar high disagreement on using AI tools as conservation buddies to brainstorm research ideas, as shown in statement two. In more specific details, more than half of the respondents showed disagreement, 15\% were neutral, and about a third of the respondents expressed agreement with this statement.

Similarly, the respondents showed high disagreement on using AI tools for help in finding relevant publications for literature review. Less than half of the respondents (47.5\%) showed disagreement, 22.5\% were neutral, and only 30\% showed agreement. In terms of academics’ use of AI tools to summarize research papers, disagreement was very high among the survey participants. Over half of the respondents disagreed with the statement, 17.5\% were neutral and 30\% agreed.

Furthermore, regarding the use of AI to increase research productivity, as mentioned in statement five, the data analysis shows very close agreement and neutral responses, where the total agreement stood at 32.5\% and neutral responses were 30\%. However, the total disagreement was higher at 37.5\%. On the other hand, 40\% of the participants viewed that AI can provide them with better access to academic information. In contrast, 40\% of them were neutral, and 20\% showed their disagreement.

To summarize, while academics indicated high knowledge of AI and its applications in section one, the findings in section two suggest a resistance by Jordanian academics to use AI in their research practices. Indeed, the total agreement in all statements ranged between 30\% and 32.5\%, with only one exception for the final statement which received 40\%. On the opposite end, the first four statements in the section received high disagreement, ranging between 47.5\% and 52.5\% among respondents. The last two statements had relatively high neutral responses at 30\% and 40\%, respectively.

The section findings would suggest that academics have resistance to the use of AI in developing research ideas, in using AI as a brainstorming buddy, reviewing literature, and in summarizing research papers. Moreover, the findings would reveal academics’ reluctance or uncertainty in using AI to increase their productivity in research and in getting better access to academic information. Further research may illuminate the reason behind this reluctance, which could be due to lack of training or academics’ fear of impinging on academic integrity.

%---- CÓDIGO DA TABELA 3 ----%
\begin{small}
\begin{longtable}{l >{\raggedright\arraybackslash}p{3.85cm}lllll}
\caption{Academics' AI Practices in Research.}\label{tab-3} \\
\toprule
\# & Statement & \multicolumn{1}{>{\raggedright\arraybackslash}p{1.5cm}}{Strongly agree} & Agree & Neutral & Disagree &  \multicolumn{1}{>{\raggedright\arraybackslash}p{1.5cm}}{Strongly disagree} \\
\midrule
\endfirsthead
\bottomrule
\source{The authors.}
\endlastfoot

1 & I use AI tools to help me develop my research ideas. & 2 (2.5\%) & 22 (27.5\%) & 14 (17.5\%) & 32 (40\%) & 10 (12.5\%) \\
2 & I use AI tools as a conversation buddy to brainstorm academic research ideas. & 6 (7.5\%) & 20 (25\%) & 12 (15\%) & 34 (42.5\%) & 8 (10\%) \\
3 & I use AI tools to help in my literature review by finding relevant academic publications. & 6 (7.5\%) & 18 (22.5\%) & 18 (22.5\%) & 30 (37.5\%) & 8 (10\%) \\
4 & I use AI tools to summarize research papers. & 6 (7.5\%) & 18 (22.5\%) & 14 (17.5\%) & 30 (37.5\%) & 12 (15\%) \\
5 & I use AI tools to increase my productivity as a researcher. & 6 (7.5\%) & 20 (25\%) & 24 (30\%) & 22 (27.5\%) & 8 (10\%) \\
6 & I use AI to get better access to academic information. & 10 (12.5\%) & 22 (27.5\%) & 32 (40\%) & 10 (12.5\%) & 6 (7.5\%) \\
\end{longtable}
\end{small}


\subsection{Academics’ AI Practices in the Classroom}

This section includes seven statements exploring the academics’ use of AI in the translation classroom, as Table \ref{tab-4} below shows. The respondents showed high disagreement in all of the statements, indicating that translation academics are reluctant to use AI tools in their classrooms. In statement one, 65\% of the respondents expressed disagreement on utilizing AI for lesson planning, while 25\% were neutral, and only 10\% agreed. Similarly, in statement two, more than two third of the respondents showed disagreement on utilizing the AI tools for making oral talks or PowerPoint presentations, whereas 17.5\% were neutral and only 15\% expressed agreement. In statement three, respondents were also reluctant to use AI tools to create teaching aides, such as games and flashcards. In more detail, 57.5\% of the respondents showed disagreement, 25\% were neutral, and only 17.5\% agreed.

There was high disagreement for statement four at 67.5\% on using AI tools to help academics with providing more relevant information, definitions, and examples for translation class content. There was very low agreement, only 12.5\%, and neutral responses stood at 20\%. As for statement five, the highest level of disagreement in this section was witnessed in utilizing AI to create quizzes and assessments for translation classrooms. Disagreement stood at 77.5\%, while 12.5\% were neutral and only 10\% agreed.

In statement six, on academics’ tendency to re-examine their activities in the translation classroom in light of AI revolution, agreement stood only at 25\%, while total disagreement reached 57.5\%, and neutral answers made up 17.5\% of the responses. In the last statement, academics had a similar reluctance to using AI tools for getting new teaching ideas in their classes, where 67.5\% disagreed, while 22.5\% agreed, and 20\% were neutral.

To conclude, similar to section two, academics showed high levels of disagreement to using AI in their translation classrooms, ranging between 57.5\%-77.5\%, whereas total agreement did not exceed 25\% at maximum, and neutral responses ranged between 12.5\% and 25\%. The reasons for these responses require further research.

%---- CÓDIGO DA TABELA 4 ----%
\begin{small}
\begin{longtable}{l >{\raggedright\arraybackslash}p{3.85cm}lllll}
\caption{Academics’ AI Practices in the Classroom.}\label{tab-4} \\
\toprule
\# & Statement & \multicolumn{1}{>{\raggedright\arraybackslash}p{1.5cm}}{Strongly agree} & Agree & Neutral & Disagree &  \multicolumn{1}{>{\raggedright\arraybackslash}p{1.5cm}}{Strongly disagree} \\
\midrule
\endfirsthead
\bottomrule
\source{The authors.}
\endlastfoot

1 & I use AI tools for lesson planning. & 0 (0.0\%) & 8 (10\%) & 20 (25\%) & 36 (45\%) & 16 (20\%) \\
2 & I use AI tools to make class presentations in my courses. & 0 (0.0\%) & 12 (15\%) & 14 (17.5\%) & 44 (55\%) & 10 (12.5\%) \\
3 & I use AI tools to create teaching aides (flash cards, games etc.) in my courses. & 2 (2.5\%) & 12 (15\%) & 20 (25\%) & 34 (42.5\%) & 12 (15\%) \\
4 & I use AI tools to provide relevant information, examples and definitions in my courses. & 4 (5\%) & 6 (7.5\%) & 16 (20\%) & 42 (52.5\%) & 12 (15\%) \\
5 & I use AI tools to create quizzes and assessments for my courses. & 2 (2.5\%) & 6 (7.5\%) & 10 (12.5\%) & 38 (47.5\%) & 24 (30\%) \\
6 & I have re-examined my teaching activities in light of AI advancements. & 0 (0.0\%) & 20 (25\%) & 14 (17.5\%) & 36 (45\%) & 10 (12.5\%) \\
7 & I have used AI to provide me with new teaching ideas in my courses. & 8 (10\%) & 10 (12.5\%) & 16 (20\%) & 34 (42.5\%) & 12 (15\%) \\
\end{longtable}
\end{small}


\section{Discussion}
The data analysis of the survey shows that Jordanian translation academics exhibit a high level of knowledge and awareness of AI and its different applications and adequate knowledge of its use in the translation classroom. This could be attributed to the explosion of AI applications and tools on the internet and media. According to the media monitoring company \textcite{meltwater2023}, there were over 2.8 million news articles on AI in 2022, and as such, academics were exposed to AI everywhere. In contrast, the level of awareness of AI use in automating repetitive tasks was very low, which may be attributed to lack of institutional support and training.

Despite their awareness of AI uses and applications, academics showed high levels of resistance to using it in any of the activities associated with research and publication, such as developing research ideas, engaging with it as a conversation buddy, conducting literature review with the assistance of AI, or summarizing research papers. Although \textcite{rahman2023} had shown that AI applications such as ChatGPT can help in generating research ideas, brainstorming, and providing research outlines, it also highlighted the need to verify the accuracy of any information that is generated by such models. Furthermore, \textcite{day2023} cautioned against fake AI generated citations. As such, academics are wary of relying on AI applications that have not proven their reliability.

In addition, many academics expressed neutral responses on AI’s potential to increase research productivity or in using it to get better access to academic information. This could be attributed to fear of misinformation \cite{baidoo-anu2023, sahari2024, shi2025} along with reports to exercise caution on LLMs as they still have issues when used for ``literature synthesis, citations, problem statements, research gaps, and data analysis" \cite[p. 1]{rahman2023}. As AI becomes indispensable to TS, it is important to highlight both its benefits and limitations to alleviate academics’ concerns and reluctance.

One of the main reasons that may also be limiting academics’ use of AI in research and publication practices could be related to their institutions’ policies. In case such policies exist, they may cause the limitations mentioned. Although certain organizations and bodies have initiated steps to regulate the use of AI \cite{europeanparliament2023}, Jordanian higher education institutions have not addressed this issue, and further research is needed to examine their engagement or lack thereof. This is particularly pertinent as, in the Jordanian academic environment, research publications are essential for academics to secure promotion and tenure. Another reason why academics are resistant to use AI in their academic work may be attributed to a fear of relinquishing their academic identity. In this context, \textcite{mirbabaie2022} have identified a number of AI identity threat predictors in the workplace, such as changes to work and loss of status position. Therefore, further research is required to understand the underlying reasons for this strong resistance.

Academics were also resistant to using AI in the translation classroom despite the versatility and wealth of applications AI offers, such as tutoring systems, personalized learning, automated grading, essay scoring, learning analytics tools, intelligent content, virtual assistant, and gamification tools \cite{owan2023}. This reluctance may be attributed to a number of reasons. Firstly, it could be the need for capacity building and better understanding of the role of these AI applications in classroom activities. Another reason may be the fear that these technologies may replace academics, and as such, they are reluctant to introduce them to their class activities \cite{owan2023}. Other studies had also reported that academics with transmissionist teaching orientations are less likely to integrate technology in their classes \cite{choi2023}. \textcite{mahadin2018} investigation of TTPs in public higher education institutions in the Jordanian context highlighted that the majority of Jordanian academics are more likely to use teacher-centred transmissionist approaches instead of constructivist ones. Hence, the relationship between teaching orientation and AI adoption in the translation classroom could be another underlying factor worthy of investigation.

Access to AI tools and applications, ethical considerations, governance, data ownership and privacy and equity may influence academics’ adoption of AI in their research and classroom practices, which supports previous research findings \cite{sahari2023, shi2025}. It is also possible that academics are fearful of becoming complacent by relying on AI. Recent research highlighted concerns about laziness and adverse reliance on AI applications among learners and teachers alike \cite{kasneci2023}. Other studies have shown that anxiety, self-efficacy, perceived usefulness, perceived ease of use, and attitudes toward use are key factors influencing academics’ adoption and use of AI. Among these, attitude toward use was the strongest factor for AI adoption \cite{wang2021}. Other studies have reported on academics’ reluctance to include new technologies due to their perceived complexity and variations \cite{celik2022}. \textcite{kaplan-rakowski2023} also argued that prior knowledge, exposure, and use can alleviate the trepidation associated with new technologies. It is possible that while Jordanian TS academics are aware of these AI applications, they have not been trained to use them by their higher education institutions which are ``intrinsically linked with developments on new technologies and computing capacities of the new intelligent machines" \cite[p. 1]{popenici2017}. \textcite{gentile2023} argue that academics need to adapt to new technologies and re-examine their roles to benefit from AI and enhance their capabilities. Higher education institutions in Jordan need to assume their responsibilities in setting policies and guidelines, supporting and training their academics, and clarifying their expectations of AI use and its ethical ramifications on translation training.

\section{Conclusion and Research Limitations}
This study examined Jordanian translation academics’ knowledge, attitudes and practices of AI in academic research and in the classroom via a 5-Likert scale questionnaire. It has revealed that while academics are aware of AI applications and tools, they expressed resistance to AI use in academic research and classroom activities. Based on this main finding, the study recommends identifying the reasons behind the academics’ reluctance or fear of using AI in research and teaching. Furthermore, the recognition and benefits of using AI in fulfilling automated repetitive tasks by TS academics should be explored. 

Since the research is confined to a quantitative approach, qualitative research investigating the employment, support, training, regulations and policies of AI in the Jordanian higher education environment is recommended. The study is also limited in the Jordanian context, i.e., public higher education institutions. Although the number of respondents is restricted to 80, the study presents insight on AI knowledge, attitude and practices at an important time of AI integration and adoption in TS. This study may help translation academics to enhance their understanding of AI potential and use. It also contributes to the scholarly debate on how AI can be effectively harnessed to enhance translation training and education.


\printbibliography\label{sec-bib}

\begin{contributors}[sec-contributors]
\authorcontribution{Dana K. Mahadin}[conceptualization,methodology,investigation,writing,review]
\authorcontribution{Sameer Naser Olimat}[conceptualization,methodology,investigation,writing,review]
\authorcontribution{Zakaryia Almahasees}[datacuration,investigation,review]
\end{contributors}

\end{document}

