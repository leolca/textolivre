\documentclass[portuguese]{textolivre}

% metadata
\journalname{Texto Livre}
\thevolume{18}
%\thenumber{1} % old template
\theyear{2025}
\receiveddate{\DTMdisplaydate{2024}{11}{26}{-1}}
\accepteddate{\DTMdisplaydate{2025}{1}{28}{-1}}
\publisheddate{\today}
\corrauthor{Camila Rezende Oliveira}
\articledoi{10.1590/1983-3652.2025.56096}
%\articleid{NNNN} % if the article ID is not the last 5 numbers of its DOI, provide it using \articleid{} commmand 
% list of available sesscions in the journal: articles, dossier, reports, essays, reviews, interviews, editorial
\articlesessionname{articles}
\runningauthor{Oliveira, Lopes e Oliveira}
%\editorname{Leonardo Araújo} % old template
\sectioneditorname{Daniervelin Pereira}
\layouteditorname{Leonardo Araújo}

\title{Tecnologias assistivas aplicadas à educação matemática inclusiva para
estudantes com Transtorno de Espectro Autista nos anos iniciais
do ensino fundamental}
\othertitle{Assistive technologies applied to inclusive mathematics education for
students with Autism Spectrum Disorder in the early years of
elementary school}

\author[1]{Camila Rezende Oliveira~\orcid{0000-0003-0115-8996}\thanks{Email: \href{mailto:oliveira.camila@ufu.br}{oliveira.camila@ufu.br}}}
\author[2]{Cjanna Vieira Lopes~\orcid{0000-0001-8733-9565}\thanks{Email: \href{mailto:cjannalopes@gmail.com}{cjannalopes@gmail.com}}}
\author[3]{Guilherme Saramago de Oliveira~\orcid{0000-0001-6638-7621}\thanks{Email: \href{mailto:gsoliveira@ufu.br}{gsoliveira@ufu.br}}}
\affil[1]{Universidade Federal de Uberlândia, Instituto de Ciências Humanas do Pontal, Ituiutaba, MG, Brasil.}
\affil[2]{Secretária Municipal de Educação, Coordenadora de  Planejamento, Caldas Novas, GO, Brasil.}
\affil[3]{Universidade Federal de Uberlândia, Faculdade de Educação, Uberlândia, MG, Brasil}

\addbibresource{article.bib}


\usepackage{enumitem,longtable,array}


\begin{document}
\maketitle
\begin{polyabstract}
\begin{abstract}
A presente pesquisa tem como temática as Tecnologias
Assistivas (TA) utilizadas para o ensino de Matemática para estudantes
com Transtorno do Espectro Autista (TEA), nos anos iniciais do Ensino
Fundamental. Tem-se como objetivo analisar as TA que podem auxiliar no
processo de ensino-aprendizagem de Matemática de alunos com TEA nessa
etapa da educação. A metodologia adotada foi uma pesquisa bibliográfica,
com ênfase em uma revisão da literatura do tipo Metanálise. Esta última
foi realizada nos bancos de dados: Scientific Electronic Library Online
(SciELO), Portal de Periódicos da Coordenação de Aperfeiçoamento de
Pessoal de Nível Superior (CAPES), Biblioteca Digital de Teses e
Dissertações (BDTD) e no Google Acadêmico. Aplicou-se um recorte
temporal de 2015 a 2022 tendo como critérios de inclusão quatro itens:
a) Falar sobre TEA; b) Abordar sobre ensino-aprendizagem de Matemática;
c) Tem delimitado os anos iniciais do Ensino Fundamental; d) Traz
exemplos de Tecnologias Assistivas. Selecionou-se ao final sete textos,
5 dissertações e 2 artigos, para compor a Metanálise. Como resultados
foram encontrados vários tipos de Tecnologias Digitais Assistivas, sendo
classificadas em: TA Tecnológicas, TA Metodológicas e TA Pedagógicas.
Além disso, identificou-se certa preponderância dos jogos e dos
Materiais Adaptados como TA para o ensino de Matemática tanto dentro da
sala regular como nas Salas de Recursos Multifuncionais. Conclui-se com
esta investigação que, independente do tipo de Tecnologia Assistiva que
se opte em utilizar, deve-se antes compreender as necessidades
educacionais dos estudantes com TEA e escolher uma TA que vá ao encontro
delas.

\keywords{Tecnologia assistiva \sep Transtorno do espectro autista \sep Anos iniciais.}
\end{abstract}

\begin{english}
\begin{abstract}
This research focuses on Assistive Technologies (AT)
used to teach Mathematics to students with Autism Spectrum Disorder
(ASD) in the early years of Elementary School. The objective is to
analyze the AT that can assist in the teaching-learning process of
Mathematics for students with ASD at this stage of education. To achieve
this objective, a bibliographic research was established, with emphasis
on a meta-analysis type literature review. The latter was carried out in
the following databases: Scientific Electronic Library Online (SciELO),
Portal de Periódicos da Coordenação de Aperfeiçoamento de Pessoal de
Nível Superior (CAPES), Biblioteca Digital de Teses e Dissertações
(BDTD), and Google Scholar. A time frame from 2015 to 2022 was applied,
with four inclusion criteria: a) Discusses ASD; b) Addresses the
teaching-learning of Mathematics; c) Has delimited the early years of
Elementary School; d) Provides examples of Assistive Technologies. Seven
texts, 5 dissertations and 2 articles, were selected to compose the
Meta-Analysis. As a result, several types of Assistive Digital
Technologies were found, classified as: Technological AT, Methodological
AT, and Pedagogical AT. In addition, a certain preponderance of games
and Adapted Materials was identified as AT for teaching Mathematics both
in the regular classroom and in Multifunctional Resource Rooms. It is
concluded from this research that regardless of the type of Assistive
Technology that one chooses to use, one must first understand the
educational needs of students with ASD and choose an AT that meets them.

\keywords{Assistive technology \sep Autism spectrum disorder \sep Early years}
\end{abstract}
\end{english}
\end{polyabstract}

\section{Introdução}
A Tecnologia Assistiva (TA) é um termo utilizado para identificar
recursos e serviços voltados às pessoas com deficiência visando
proporcionar a elas, autonomia, independência, qualidade de vida e
inclusão social. As Tecnologias Assistivas Digitais (TAD) fazem parte
das TA sendo utilizadas no meio digital para proporcionar aprimoramento
da aprendizagem para pessoas com deficiência. No contexto escolar, em
especial nas aulas de Matemática são instrumentos, metodologias,
didáticas elaboradas pelo professor para conduzir o estudante com TEA ao
aprendizado matemático. É um assunto ainda pouco discutido e, por vezes,
desconhecido dos professores da área.

Mediante essas considerações, a questão norteadora desta pesquisa,
advinda de uma metanálise, é: quais Tecnologias Assistivas podem
auxiliar no processo de ensino-aprendizagem de Matemática, nos anos
iniciais do Ensino Fundamental, de estudantes com Transtorno do Espectro
Autista --- TEA?

Diante dessa problemática de pesquisa elaborou-se o seguinte objetivo
geral: Identificar, descrever e analisar as Tecnologias Assistivas que
podem auxiliar no processo de ensino-aprendizagem de Matemática de
alunos com Transtorno do Espectro Autista (TEA), nos anos iniciais do
Ensino Fundamental.

Os objetivos específicos dessa pesquisa foram: a) descrever o contexto
da Educação Matemática e da Educação Matemática inclusiva no Brasil; b)
investigar e descrever sobre o Transtorno do Espectro Autista (TEA) e
suas características, com ênfase no processo de ensino-aprendizagem; c)
apresentar a ideias de autores da área sobre Educação Matemática
inclusiva para o ensino de estudantes com TEA, dando ênfase às
Tecnologias Assistivas; d) elencar as tecnologias assistivas para o
ensino de estudantes com TEA, com base na Metanálise realizada.

Para alcançar os objetivos traçados optamos por realizar uma metanálise
de artigos, dissertações e teses, produzidos sobre o ensino de
Matemática para crianças com Transtorno do Espectro Autista e que trazem
exemplos e sugestões de Tecnologias Assistivas que podem ser utilizadas
nesse processo de ensino-aprendizagem.

Essa pesquisa se justifica por se tratar
de uma realização âmbito pessoal e profissional aliada à experiência e à
criticidade como ponto para o aprimoramento dos estudos referentes à
observação e aos docentes que irão ministrar aulas de Matemática, nos
anos iniciais do Ensino Fundamental, para estudantes com TEA.
Acredita-se que essa investigação agrega valor científico e social à
temática já que propõe uma síntese das possíveis Tecnologias Assistivas
a serem utilizadas nesse contexto, colaborando assim para a autoformação
dos professores e o aprendizado dos estudantes.



\section{Metodologia}
A pesquisa científica é realizada por aspectos que permitem a reflexão
por parte dos pesquisadores. Pela sua contribuição científica permite a
resolução de problemas assim como a demonstração de resultados. Para a
sua aplicação prática e teórica tem diferentes metodologias entre estas:
a abordagem qualitativa. Primeiramente, é preciso compreender a
diferença existente entre pesquisa quantitativa e pesquisa qualitativa.

No que tange a abordagem qualitativa esta tem sido abordada por diversas
áreas científicas, tanto no que se refere a área de humanas quanto na
área social ou mesmo nas exatas em alguns casos específicos. Na área de
educação, esta vem sendo cada vez mais utilizada por diferentes
pesquisadores em âmbitos diferenciados cujo enfoque depende do problema
de pesquisa que está sendo pesquisado. Cabe ressaltar que a pesquisa
qualitativa tem algumas características específicas as quais podemos
destacar, nos dizeres de \textcite{garnica2003}:
\begin{quote}
a) a transitoriedade de seus resultados; b) a impossibilidade de uma
hipótese a priori, cujo objetivo da pesquisa será comprovar ou refutar;
c) a não neutralidade do pesquisador que, no processo interpretativo,
vale-se de suas perspectivas e filtros vivenciais prévios dos quais não
consegue se desvencilhar; d) que a constituição de suas compreensões
dá-se não como resultado, mas numa trajetória em que essas mesmas
compreensões e também os meios de obtê-las podem ser (re)configuradas;
e) a impossibilidade de estabelecer regulamentações, em procedimentos
sistemáticos, prévios, estáticos e generalistas
\cite[p.~86]{garnica2003}.
\end{quote}

Tais características acima não devem ser vistas como regras, visto que a
noção de pesquisa qualitativa é embasada na noção de movimento, de
continuidade ao contrário do que pressupõe as demais pesquisas, cujo o
enfoque era a estagnidade.

Esse trabalho tem uma abordagem qualitativa e, juntamente com as suas
diferentes modalidades de pesquisa qualitativa, ainda terá como foco uma
Metanálise. A Metanálise na pesquisa científica é um tipo de abordagem
metodológica muito utilizada em diversos estudos que visam o
aprofundamento bibliográfico em determinada área de estudo o qual visa
estabelecer lacunas que dantes não foram vistas ou realizadas.

Sete passos devem ser observados ao se efetuar uma Metanálise. São eles,
segundo \textcite[p.~13]{bicudo2014}: formulação da pergunta; localização e
seleção dos estudos; avaliação crítica dos estudos; coleta dos dados;
análise e apresentação dos dados; interpretação dos dados e;
aprimoramento e atualização da Metanálise. Deve-se assumir uma atitude
rigorosa ao efetuar a interpretação de textos devendo esta ser realizada
em um ``terceiro nível'', sendo a ``interpretação da interpretação''.

Pormenorizamos algumas etapas da Metanálise que foram seguidas. A etapa
1, que é definição do objetivo de pesquisa, já foi evidenciada
anteriormente na introdução desse trabalho de pesquisa. Por esse motivo
é que aqui vamos para a etapa 2, que é a sistematização das informações.
Trata-se do local em que as fontes de pesquisa serão buscadas.
Estabelecemos artigos científicos, publicados em periódicos brasileiros,
e dissertações/teses dos últimos 8 anos, de janeiro de 2015 a janeiro de
2022.

O estabelecimento dos descritores para esta pesquisa passou por várias
fases e testes. Percebemos ambiguidades presentes nos três termos que
compuseram a varredura. A palavra Transtorno do Espectro Autista ainda é
inexistente em algumas pesquisas, mesmo com o recorte temporal mais
atualizado, há utilização dos termos ``Autismo'', ``Deficiência
Intelectual'', ``Autista(s)'' dentre suas variantes. O Termo Matemática
ora se referia à ciência ora à disciplina e havia grande variação da
etapa da educação básica a qual ele se referia. A palavra ``Tecnologia
Assistiva'', como vimos definida na LBI \cite{brasil2015} não abrange
somente os aparelhos ou equipamentos tecnológicos, então, dentro desses
textos estão contempladas Tecnologias Assistivas, sem necessariamente
que se utilizasse esse nome ao longo do texto.

Diante desses impasses optamos por estabelecer como descritores para a
nossa pesquisa "Autismo" and "Matemática" e ``Tecnologia Assistiva'' and
``Matemática'' e ``Tecnologia Assistiva'' and ``Matemática'' and
"Autismo". As palavras foram buscadas entre aspas para a maior precisão
da pesquisa. Dessa forma, abrangeríamos a temática aqui proposta e
conseguiríamos um número maior de textos sobre o Transtorno do Espectro
Autista e as Tecnologias Assistivas utilizadas para o seu processo de
ensino-aprendizagem, na área da Matemática.

Com as publicações já previamente selecionadas, na etapa anterior,
buscamos aquelas que tinham ligação com o objetivo geral dessa pesquisa
e, então, realizamos uma leitura superficial desse material ou ``leitura
rápida'', como sugere \textcite{gil2008}. Após essas leituras, foi possível
ainda eliminar aqueles que, mesmo tendo nos títulos os descritores e que
se enquadravam no recorte temporal, não tinha como foco o tema e
objetivo proposto em nossa investigação.

Ao longo da análise de dados foram feitas considerações a respeito dos
textos, do ensino de Matemática para crianças com Transtorno do Espectro
Autista e das Tecnologias Assistivas como possíveis colaboradoras para
seu processo de ensino-aprendizagem.




\section{Tecnologias assistivas e digitais e suas implicações no
Ensino de Matemática}

Mudanças ocorridas no cenário mundial têm provocado diversas discussões
a respeito da educação em âmbito geral. A demanda da implementação de
recursos que visam o aprendizado de estudantes da Educação Básica
provenientes da Educação Inclusiva e fora dela aumenta a cada dia. A
narrativa democrática que visa a oferta de uma educação de qualidade já
é preconizada nos documentos oficiais e nos discursos de acadêmicos e
profissionais. Tendo em vista esses desafios inerentes ao trabalho
pedagógico docente um novo conceito foi surgindo: as Tecnologias
Assistivas.

De acordo com a Ata da VII Reunião do
Comitê de Ajudas Técnicas (CAT) \cite{brasil2007}, as tecnologias
assistivas são:
\begin{quote}
    {[}...{]} uma área do conhecimento, de característica interdisciplinar,
que engloba produtos, recursos, metodologias, estratégias, práticas e
serviços que objetivam promover a funcionalidade, relacionada à
atividade e participação, de pessoas com deficiência, incapacidades ou
mobilidade reduzida, visando sua autonomia, independência, qualidade de
vida e inclusão social \cite[p.~3]{brasil2007}.
\end{quote}

A Tecnologia Assistiva (TA) é composta por serviços e recursos que
buscam resolver problemas funcionais de pessoas com algum tipo de
deficiência ou o uso de equipamento em si para o desenvolvimento de
tarefas, muitas vezes, bem simples, promovendo promover uma vida mais
independente e a inclusão de mundo geral. Apesar de ser uma área
relativamente nova no contexto brasileiro, a TA, segundo \textcite{soares2017},
iniciou-se a mais de meio século nos Estados Unidos com o termo
\emph{Assistive Technology.}

Após a Segunda Guerra Mundial, por volta dos anos de 1970, a
visibilidade do termo aumentou, haja vista o retorno dos soldados guerra
com mutilações. Antes mesmo do surgimento do computador, programa de
ajudas protéticas e sensoriais, visava a reabilitação desses veteranos
de guerra por meio de próteses e órteses \cite{brasil2009}.

O desenvolvimento estadunidense foi contínuo e a partir daí diversas
ferramentas na área da mecatrônica e da robótica foram criadas. Os
financiamentos também foram constantes e o avanço no setor é realizado
até o presente momento. A realidade brasileira se distingue dos países
europeus e dos Estados Unidos. No Brasil, o termo anteriormente
designado como Ajudas Técnicas foi cunhado na legislação somente em
1999, especificamente no Decreto nº 3.298 \cite{brasil1999}, o qual
regulamenta a Lei nº 7853, que contém disposições sobre a Política
Nacional para a Integração da Pessoa Portadora de Deficiência.

Nesse decreto, são apontadas questões como linhas de crédito e isenção
tributária na aquisição desses equipamentos. Em seu Art. 19 define-se
como ``ajudas técnicas'' ``{[}...{]} os elementos que permitem compensar
uma ou mais limitações funcionais motoras, sensoriais ou mentais
{[}...{]} com o objetivo de permitir-lhe superar as barreiras da
comunicação e da mobilidade e de possibilitar sua plena inclusão social
\cite[Art.~19]{brasil1999}.

Ainda nesse mesmo artigo, em seu parágrafo único, citam-se nove tipos de
ajudas técnicas. Enfatizamos aqui aqueles que mais seriam necessárias no
contexto escolar: I - próteses auditivas, visuais e físicas; II -
órteses que favoreçam a adequação funcional; {[}...{]} VI - elementos
especiais para facilitar a comunicação, a informação e a sinalização
para pessoa portadora de deficiência; VII - equipamentos e material
pedagógico especial para educação, capacitação e recreação da pessoa
portadora de deficiência; VIII - adaptações ambientais e outras que
garantam o acesso, a melhoria funcional e a autonomia pessoal; {[}...{]}
\cite[Art.~19]{brasil1999}.

Em 2007, com intuito na melhoria da TA foi regulamentado o Comitê de
Ajudas Técnicas --- CAT.~ Com a firmação do Comitê foi alterada a
terminologia de ``Ajudas Técnicas'' e/ou ``Tecnologias de Apoio'' para
Tecnologias Assistivas. Em 2012, foi inaugurado o Centro Nacional de
Referência em Tecnologia Assistiva (CNRTA) que se localiza na cidade de
Campinas/SP que tem como objetivo principal ``{[}...{]} orientar uma
rede de 20 núcleos de pesquisa em universidades públicas'' \cite[p.~63-64]{brasil2013}. Somado a isso, a CNRTA tem a incumbência de estabelecer
diretrizes e articular a atuação dos núcleos de produção científica e
tecnológica do país.

Para além da criação do centro, foi criada Rede Nacional de Pesquisa e
Desenvolvimento em Tecnologia Assistiva (NPDTA), que contém 91 núcleos
visando a formação e articulação entre os diversos setores sociais.~
Ainda em 2012, foi lançada a Pesquisa Nacional de Tecnologia Assistiva
(PNTA) no Brasil. A ligação entre o Instituto de Tecnologia Social e o
Ministério de Ciência Tecnologia e Inovação (MCTI) permitiu a
organização de pesquisadores do país inteiro foi uma prospecção na área
\cite{brasil2013}.

Apesar de ser uma área já estabelecida, um grande problema que se
enfrenta quando tange as questões da Tecnologia Assistiva é sobre o
custo de seus equipamentos. Como a maioria dos recursos são oriundos de
outros países, o custeio assim como a manutenção para as TA acaba se
tornando, por vezes, inviável para quem necessita destas. \textcite{soares2017}
entende que a gama de produtos que se tem para a pessoa com deficiência
é grande, mas a demanda de mercado acaba por impedir um aproveitamento
profícuo dos produtos.

Outra questão relevante a ser tratada com relação às Tecnologias
Assistivas é a falta de investimento nas pesquisas brasileiras dessa
área. Por ser uma área de pesquisa recente e ainda em construção muitos
pesquisadores acabam por não optar por essa área acadêmica. Diante
disso, a cientificidade das pesquisas em TA acabam por ficar um tanto
obscuras e as universidades optam por investir nesse tipo de pesquisa
\cite{bersch2017}.

Por esse motivo torna-se relevante que
se aprofunde os estudos quanto à Tecnologia Assistiva de modo que se
possa atingir pessoas com deficiência em todo âmbito nacional e
internacional, principalmente nos espaços escolares e na formação de
professores que podem utilizar tais ferramentas no aprimoramento
pedagógico em sala de aula escolar.~Pensando nisso, \textcite{bersch2013}
elaborou um documento com o objetivo de ajudar professores e gestores a
identificarem tecnologias assistivas que pudessem beneficiar os alunos
com deficiência. Nele a autora preparou um catálogo com fotografias,
descrição de alguns recursos e a indicação de sites onde estes recursos
poderiam ser visualizados e adquiridos.

Sobre as classificações que podem ser
estabelecidas para as TAs estas seguem alguns parâmetros tanto em nível
mundial quanto brasileiro. Essas variam conforme a região, pois alguns
países utilizam mais determinadas classificações do que outros. As
classificações brasileiras são baseadas em modelos europeus sendo pouco
produzidas especificamente no Brasil. As classificações que utilizamos
foram geradas de três importantes referências que apresentam diferentes
focos de organização e aplicação, conforme vemos na \Cref{tbl01}.

\begin{table}[h]
\centering
\begin{threeparttable}
\caption{Referências para o estabelecimento da Tecnologia Assistiva no Brasil}\label{tbl01}
\begin{tabular}{
>{\raggedright\arraybackslash}p{0.4\textwidth}
>{\raggedright\arraybackslash}p{0.4\textwidth}} 
\toprule
Referência & Especificaçãoes \\
\midrule
ISO 9999 & Utilizada em vários países, com foco específico em recursos, que são organizados em classes desdobradas em itens de produtos. \\
Classificação Horizontal European Archives in Rehabilitation Technology – HEART & Foca nos conceitos envolvidos na utilização de Tecnologia Assistiva, considerando três grandes áreas de formação em TA: componentes técnicos, componentes humanos e componentes socioeconômicos. \\
Classificação Nacional de Tecnologia Assistiva, do Instituto Nacional de Pesquisas em Deficiências e Reabilitação, dos Programas da Secretaria de Educação Especial, Departamento de Educação dos Estados Unidos & Desenvolvida a partir do conceito de TA presente na legislação norte-americana, integrando recursos e serviços. \\
\bottomrule
\end{tabular}
\source{Organizado pelos autores (2023) , com base em \cite{brasil2009}.}
\end{threeparttable}
\end{table}

Além das classificações acima há também as classificações estabelecidas por
%\textcite{tonolli1998}
Tonolli e Bersch, em 1998, e atualizada por \textcite{bersch2017} as quais se reportam em 12 categorias principais:
\begin{enumerate}[label=\alph*)]
\item Auxílios para a vida diária e vida prática;
\item CAA - Comunicação Aumentativa e Alternativa;
\item Recursos de acessibilidade ao computador;
\item Sistemas de controle de ambiente;
\item Projetos arquitetônicos para acessibilidade;
\item Órteses e próteses;
\item Adequação Postural;
\item Auxílios de mobilidade;
\item Auxílios para ampliação da função visual e recursos que traduzem conteúdos visuais em áudio ou informação tátil;
\item Auxílios para melhorar a função auditiva e recursos utilizados para traduzir os conteúdos de áudio em imagens, texto e língua de sinais;
\item Mobilidade em veículos;
\item Esporte e Lazer.
\end{enumerate}

Nem toda tecnologia pode ser considerada acessível. No contexto escola,
por exemplo, ela só poderá ser considera como Tecnologia Assistiva
quando for utilizada por um aluno com deficiência com o objetivo de
romper barreiras sensoriais, motoras ou cognitivas que limitam/impedem
seu acesso às informações ou limitam/impedem o registro e expressão
sobre os conhecimentos adquiridos por ele. Também, quando o instrumento
tecnológico favorecer seu acesso e participação ativa e autônoma em
projetos pedagógicos; quando possibilitar a manipulação de objetos de
estudos \cite{bersch2013}.

Os Objetos de Aprendizagem Acessíveis
(OAA) são categorias que se encontram dentro das Tecnologias Assistivas,
porém têm concepções específicas de serem tratadas. Segundo \textcite{mourao2020}, trata-se de

{[}...{]} quaisquer ~ materiais ~
digitais (imagens, vídeos,~páginas da web,~animações
ou~simulações),~desde~que~tragam informações destinadas à construção do
conhecimento, especifiquem seus objetivos pedagógicos, estejam em
conformidade com os padrões e diretrizes de acessibilidade, e
estruturados de modo que possam ser reutilizados
\cite[p.~44]{mourao2020}.

Os autores correlacionam os objetivos de aprendizagem acessível com a
Engenharia de \emph{Software}, pois permite desenvolver e manter
sistemas, permite um controle de qualidade efetivo e promove o
planejamento e gestão de atividades. Citam sete metodologias/processos
de, enfatizando no aplicativo ``MIDOAA'', ou seja, Modelo Inclusivo de
Desenvolvimento de Objetos de Aprendizagem Acessíveis. Trata-se de um
modelo desenvolvido com base no modelo que tem como foco questões
pedagógicas e computacionais, sendo desenvolvido e validado por
professores do Ensino Superior dos cursos de Computação e Pedagogia. O
programa foi projetado e desenvolvido considerando os seguintes itens:
Metodologia de Projetos, Abordagem Pedagógica, Abordagem Computacional,
e Padrões e Diretrizes de Acessibilidade \cite{mourao2020}.

Tendo como base essa conceituação e o enfoque na acessibilidade como
conceito principal, é também necessário aqui realizar alguns
apontamentos sobre esta. O termo acessibilidade vem do latim
\emph{acessibilite} que de acordo com o dicionário Michaelis
(\emph{on-line}), significa: facilidade de acesso, de obtenção e também
facilidade no trato. \textcite{sassaki2009} explica que a acessibilidade vai
além do acesso físico e pedagógico, ela é maneira de oportunizar os
cidadãos em garantia e segurança. Deve-se considerar ela deve ser
pensada de modo coletivo, porém com singularidade que cada pessoa com
deficiência necessita.

Frente a essa demanda, foi criado
também o conceito de Acessibilidade Digital (AD). Até meados da década
de 40 e 60 o conceito de acessibilidade se referia às barreiras
arquitetônicas, porém com os adventos das Tecnologias Digitais da
Informação em Comunicação --- TDIC tal conceito foi ampliado. Afim de
estabelecer a inclusão digital a todas as pessoas tendo como foco que
esta ``{[}...{]} é o direito de acesso ao mundo digital para o
desenvolvimento intelectual (educação, geração de conhecimento,
participação e criação) e para o desenvolvimento de capacidade técnica e
operacional'' \cite[p.~215]{passerino2007} o conceito de AD entrou em
vigor em diversas esferas.

É importante o destaque que o conceito de Acessibilidade Digital vai
além do simples fato de usabilidade ele propõe, principalmente, a
permissão do uso. A usabilidade destaca somente a facilidade de uso por
parte dos usuários. Por esse motivo, a AD tem relação direta com as
concepções de Tecnologia Assistiva, pois considera importante a inclusão
de todos os que estão na rede de computadores. Os Objetos de
Aprendizagens Acessíveis são recursos favoráveis ao desenvolvimento que
se encontram presentes tanto na TA quando na AD, por esse motivo, a
relevância de trabalhos que abordem sobre esses na educação.

Para a construção de Objetos de Aprendizagem de modo geral e também para
os acessíveis é necessário que alguns passos sejam seguidos. O
planejamento é o primeiro passo para o estabelecimento da construção de
um objeto de aprendizagem. Os objetivos e público-alvo ao qual se
destina devem estar claros para os demais indivíduos.~ Uma equipe de
técnicos e de pessoas relacionadas ao âmbito pedagógico em trabalho
interdisciplinar é essencial na construção de objetos de aprendizagem
\cite{mourao2020}.

As possíveis barreiras que possam surgir no Objeto de Aprendizagem
também são consideradas quando em sua construção. A linguagem usual e de
simples acesso é um ponto a ser ressaltado, pois linguagens pouco
acessíveis ou de difícil entendimento dificultam para os usuários. Por
esse motivo, \textcite{souza2021} cita que há duas características principais
para OA: pedagógica e técnica. Segundo esses mesmos autores, as
características pedagógicas as quais serão retratadas nesse trabalho de
pesquisa, com enfoque nos OAA de Matemática para alunos com TEA, fazem
referência à concepção de objetos que facilitem o trabalho de
professores e alunos, visando à aquisição do conhecimento.

Sob esse ponto de vista, os Objetos de Aprendizagem Acessível em
Matemática vêm ao encontro de aprimorar o trabalho do docente de
Matemática, em sala de aula, afim de que os estudantes público-alvo da
Educação Especial na perspectiva Inclusiva possam aprender de modo
significativo essa disciplina. Os OAA para o ensino de Matemática não se
diferem muito dos demais objetos de aprendizagem de modo geral, porém
estes são destinados a essa disciplina especificamente no contexto
escolar.

Apesar da relevância sobre a acessibilidade e os Objetos de Aprendizagem
Acessíveis para o ensino de Matemática a exploração dessa temática ainda
é pouco abordada no Brasil. Os OA para a Matemática podem ser vistas
discutidas em pesquisas como de \textcite{derossi2015,capellini2015} que
abordam OA e Lousa Digital; no estudo de \textcite{kalinke2016} que propõe uma
análise dos OA presentes no Programa Nacional do Livro e do Material
Didático (PNLD), de 2014; e o de \textcite{renaux2017} Renaux (2017), que investiga o uso de
OA por estudantes de um curso de Pedagogia que vemos como de grande
relevância para a discussão das OA~ na Matemática, porém especificamente
voltadas para a acessibilidade em Matemática não tem muitos estudos.

Os trabalhos como dos autores acima, assim como outros investigados por
nós, demonstram indícios da utilização das OA na coletividade e
oportunizam uma melhoria na interação dos sujeitos envolvidos. A
promoção de estratégias matemáticas para o aprendizado, assim como, os
modos de pensar e agir traduziram nas pesquisas apontadas que os OA são
aliados no fazer e sentir Matemática. Esses aspectos também podem ser
pensados para a inserção dessa ferramenta pedagógica nas aulas de
Matemática com alunos público-alvo da Educação Inclusiva, como é o caso
dos estudantes do Ensino Fundamental com Transtorno do Espectro
Autista.~

Nesse sentido é que, na seção seguinte, iremos retratar especificamente
sobre os resultados dessa pesquisa.



\section{Resultados}
Na \Cref{tbl02}, é possível ter um apanhado geral dos textos selecionados
que serão minunciosamente analisados no decorrer desta seção. Eles se
encontram organizados por ordem cronológica, sendo especificado(s) o(s)
autor(es), a instituição vinculada ou revista publicada, o gênero
(Artigo ou dissertação) e o local em que ele foi encontrado, tendo como
base de busca os cinco portais descritos anteriormente.

%%%% TABELA
{
\setlength\LTleft{-0.9in}
\setlength\LTright{-0.9in}
\begin{footnotesize}
\begin{longtable}{ll
    >{\raggedright\arraybackslash}p{0.15\textwidth}
    >{\raggedright\arraybackslash}p{0.3\textwidth}
    >{\raggedright\arraybackslash}p{0.2\textwidth}
    l
    >{\raggedright\arraybackslash}p{0.15\textwidth}
    }
\caption{Textos selecionadas para a Metanálise}
\label{tbl02}
\\
\toprule
Nº & Ano & Autor & Título& Instituição/Revista & Gênero & Local \\
\midrule
01 & 2015 & CAMINHA et al. & Tecnologias assistivas e coping familiar para a inclusão escolar da criança com autismo & Revista Diálogos e Perspectivas em Educação Especial & Artigo & Periódico da CAPES \\

02 & 2019 & SANTOS & O uso de recursos de tecnologia assistiva para o ensino de ciências e matemática em salas de recursos multifuncionais & Universidade Federal de Itajubá & Dissertação & Google Acadêmico \\

03 & 2019 & Moreira costa amaral & tecnologia assistiva no ensino da matemática para alunos com transtorno do espectro autista & Educação Matemática em Revista & Artigo & Google
 Acadêmico \\
04 & 2020 & GUIMARÃES & O processo de construção de um material educacional na perspectiva da educação matemática inclusiva para um aluno autista & Universidade Federal Rural do Rio de Janeiro & Dissertação & BDTD \\

05 & 2020 & SANTOS & Ensino de matemática e transtorno do espectro autista - TEA: possibilidades para a prática pedagógica nos anos iniciais do ensino fundamental & Universidade Federal de Uberlândia & Dissertação & UFU \\

06 & 2020 & CAMARGO & Estratégias metodológicas para o ensino de matemática: inclusão de um aluno autista no ensino fundamental & Universidade Federal de Sergipe & Dissertação & Google
 Acadêmico \\
 
07 & 2022 & GUBERT & Uso de tecnologias assistivas no ensino de matemática em salas de recursos multifuncionais em uma rede municipal de ensino & Universidade Estadual do Oeste do Paraná & Dissertação & BDTD \\

\bottomrule
\source{Texto de preenchimento gerado pelo pacote lipsum.}
\end{longtable}
\end{footnotesize}
}

Os sete textos, presentes da \Cref{tbl02}, datam: 1 de 2015, 2 de 2019, 3 de
2020 e 1 de 2022. Foram selecionados por discutirem sobre o Transtorno
do Espectro Autista, abordarem sobre ensino-aprendizagem de Matemática,
terem delimitado os anos iniciais do Ensino Fundamental e trazerem
exemplos de Tecnologias Assistivas para se ensinar crianças com Autismo
nesta área.

%%%% TABELA
\begin{footnotesize}
\begin{longtable}{
    >{\raggedright\arraybackslash}p{0.15\textwidth}
    >{\raggedright\arraybackslash}p{0.8\textwidth}
    }
\caption{Texto 1}
\label{tbl03}
\\
\toprule
Autor & Vera L. P. S. Caminha\newline
Adriano de O. Caminha\newline
Priscila P. Alves\newline
Claudiana Prudência dos Santos\\
Título & Tecnologias Assistivas e Coping Familiar para a Inclusão Escolar da Criança com Autismo\\
Revista ou Instituição & Revista Diálogos e Perspectivas em Educação Especial\\
ISSN & 2764-6440 \\
Qualis & Qualis Capes (2017-2020) Educação: B1 \\
Páginas & 14 \\
Objetivo(s) & Apresentar considerações sobre os recursos das TIC (Tecnologias da
Informação e Comunicação) para inclusão escolar da criança com TEA,
como também dispositivos que visam favorecer estratégias de coping para a
família.\\
Metodologia & Revisão bibliográfica sobre o Transtorno do Espectro Autista;\newline
Revisão bibliográfica dos processos de desenvolvimento de software para inclusão digital para pessoas com TEA;\newline
Pré-Teste sobre aplicação do software;\newline
Validação da efetividade dos programas e jogos aplicados à criança com Transtorno do Espectro Autista a partir da inscrição de voluntários no projeto;\newline
Observação do comportamento de interação entre a criança com TEA e família e intervenção para favorecer as estratégias de coping;\newline
Validação do uso da ferramenta computacional através do follow up a partir do protocolo de desenvolvimento da criança.\\
Resultados & O Ambiente Digital de Aprendizagem para Crianças Autistas (ADACA) é
uma ferramenta que auxiliará na aprendizagem de Português e Matemática,
pois é uma tecnologia assistiva de fundamental importância na comunicação
e, possuindo formas de rastrear todos os movimentos feitos por cada criança
ao realizar uma atividade e no final gerar relatórios. \\
Tecnologia Assistiva & Ambiente Digital de Aprendizagem para Crianças Autistas (ADACA)\\
Fragilidades & A seção de metodologia traz na verdade um roteiro contendo as etapas da pesquisa e não necessariamente os procedimentos metodológicos, com
embasamento teórico, como se espera geralmente.\newline
Apesar de ter sido publicado em 2015 a terminologia Autismo prevalece em detrimento do Transtorno de Espectro Autista.\newline
Uso do termo ``portadores''\\
\bottomrule
\source{Os autores (2023).}
\end{longtable}
\end{footnotesize}

O primeiro texto foi publicado por um grupo de quatro autores, sendo
dois da área da Engenharia de Sistemas e Computação, uma da Psicologia
Social e outra da Psicopedagogia. Foi publicado na Revista Diálogos e
Perspectivas em Educação Especial e apresenta como objetivo o uso de
Tecnologias da Informação e Comunicação (TIC) para favorecer sua
aprendizagem em áreas como Matemática, Português e Música. Na \Cref{tbl03},
é possível ter um apanhado geral dessa publicação datada de 2015.

O segundo texto se trata de uma dissertação de Mestrado em Educação e
Ciências defendida pela Universidade Federal de Itajubá, MG, em 2019. A
autora investiga o uso de Tecnologias Assistivas especificamente nas
Salas de Recursos Multifuncionais (SRM) tanto aquelas adquiridas pelas
escolas quanto aquelas produzidas pelos professores da SRM.

O Texto 3 se trata de um artigo científico publicado na ``Educação
Matemática em Revista'', sendo essa de Qualis A2, em 2019, por Moreira,
Costa e Amaral. Os autores são vinculados à Universidade do Estado de
Minas, sendo que dois são doutores em Educação e outro é Mestre em
Educação Matemática.

O Texto 4 se trata de uma dissertação defendida, em 2020, pelo Programa
de Mestrado em Educação em Ciências e Matemática da Universidade Federal
Rural do Rio de Janeiro. O texto é composto por 183 páginas e aborda
sobre a experiência vivenciada entre alunos do Programa de Residência
Pedagógica em Matemática da Universidade Federal Rural do Rio de
Janeiro, campus Seropédica, e de sua preceptora em uma escola regular de
Educação Básica \cite{guimaraes2020}.

O produto educacional de \textcite{guimaraes2020} foi elaborado com base em
observações da acerca de suas vivências, dos registros dos
residentes/estagiários que acompanharam as aulas e da entrevista de
grupo cinco graduandos que participaram como residentes. O produto tomou
a forma de um caderno pedagógico em que na sua primeira parte
apresenta-se, resumidamente, a fundamentação teórica.

Na segunda parte, há relato de atividades desenvolvidas pela
pesquisadora e os alunos do programa de residência pedagógica em
Matemática da Universidade Federal Rural do Rio de Janeiro, com um aluno
com TEA. Na sequência, apresentam-se sugestões de estratégias para o
desenvolvimento da aprendizagem a partir de eixos de interesse,
sugerindo uma sequência de atividades a ser desenvolvida em sala de
aula.

O Texto 5 se trata de uma dissertação de Mestrado em Educação, defendida
na Universidade Federal de Uberlândia, em 2020. É composto por 131
páginas com um formato de pesquisa bibliográfica, enfoque em uma
metanálise, tendo uma abordagem qualitativa dos dados coletados nas
obras. A autora teve como objetivo apresentar práticas pedagógicas que
os educadores pudessem utilizar para ensinar Matemática a alunos com
Transtorno do Espectro Autista.

Quando a autora disse ``{[}...{]} um avanço relacionado às Tecnologias
da Informação e Comunicação abriu espaço para as chamadas Tecnologias
Assistivas - TA'' \cite[p.~100]{santos2020} fez-nos entender que as TAs
estão especificamente atreladas às TICs e isso não é correto de ser
afirmado. Em nosso Referencial Teórico explicamos que, ainda na década
de 90, era utilizada a terminologia ``ajudas técnicas'' para o que hoje
chamamos de ``Tecnologias Assistivas'' e que estes não englobam e nunca
englobaram apenas equipamentos eletrônicos ou tecnológicos.

As Tecnologias Assistivas abrangem ``coisas'' e ``ações'' que permitem a
superação de barreiras que impedem a plena inclusão social de pessoas
com deficiência. Elas podem ser uma prótese ou órtese, objetos ou
equipamentos que facilitam a comunicação, adaptações feitas no ambiente
e, no caso da escola especificamente, abrange sobretudo equipamentos e
materiais pedagógicos \cite{brasil1999}, bem como, recursos,
metodologias, estratégias e práticas \cite{brasil2007} que são
utilizados para mobilidade, recreação, capacitação, autonomia, qualidade
de vida e inclusão educacional do público algo da Educação Especial.

O Texto 6 se trata também de uma dissertação de Mestrado em Educação
defendida pela Universidade Federal de Sergipe, em 2020. Para melhor
visualização do texto 6, está evidenciada a \Cref{tbl04}.

%%%% TABELA
\begin{footnotesize}
\begin{longtable}{
    >{\raggedright\arraybackslash}p{0.15\textwidth}
    >{\raggedright\arraybackslash}p{0.8\textwidth}
    }
\caption{Texto 6}
\label{tbl04}
\\
\toprule
Autor & Erica Daiane Ferreira Camargo \\
Título & Estratégias Metodológicas para o Ensino de Matemática: inclusão de um aluno autista no Ensino Fundamental \\
Revista ou Instituição & Universidade Federal de Sergipe \\
ISSN & --- \\
Qualis & --- \\
Páginas & 235 \\
Objetivo(s)	& Analisar estratégias metodológicas necessárias à mediação do processo de ensino e aprendizagem no caso de um aluno com o diagnóstico de Transtorno do Espectro Autista \\
Metodologia	& Pesquisa-ação com abordagem de uma pesquisa colaborativa-crítica;
Utilizou-se como instrumentos para coleta de dados: observação participante, reunião com professores e trabalho colaborativo em sala de aula; \newline
\emph{Realizou-se intervenções com planejamentos específicos tendo como foco o desenvolvimento de estratégias pedagógicas para um aluno do Ensino Fundamental com diagnóstico de Transtorno do Espectro do Autismo.} \\
Resultados & O resultado deste processo foi percebido em diferentes momentos: mudanças no pensamento das professoras que demonstraram ter percebido a necessidade de modificar suas práticas. Porém, o processo de mudança traz a tona práticas que são enraizadas em concepções antigas e que reverberam nas nossas escolas dificultam a ultrapassagem da barreira do professor e da criança com deficiência. \newline
Conclui-se que é possível transpor as barreiras impostas para efetivar a Educação Inclusiva e que elas não estão nessas crianças, mas, sim, nas falhas de formação e na desarmonia entre o que é preciso fazer e como se pode fazer. \\
Tecnologia Assistiva & jogos\newline
quebra-cabeças\newline
colagens\newline
Materiais concretos para ensinar sobre operações de adição\newline
Comunicação Alternativa\newline
Atividades com Resolução de Problemas\newline
Atividades abordando geometria e noções espaciais\newline
Atividades baseadas na Comunicação Alternativa e Ampliada (CAA)\newline
Atividades de correspondência \\
Fragilidades & Nas considerações finais não traz uma conclusão específica sobre o ensino de Matemática, foca mais no ensino colaborativo, nas estratégias pedagógicas e na inclusão dos estudantes com TEA. \\
\bottomrule
\source{Os autores (2023).}
\end{longtable}
\end{footnotesize}

A autora elaborou dois tipos de
atividades, primeiramente as mais lúdicas e com material concreto e
depois aquelas impressas. Considerou que o estudante não estabelecia uma
comunicação oralizada e que estava passando pelo processo de
implementação do sistema da Comunicação Alternativa e Ampliada (CAA) em
uma clínica de fonoaudiologia, utilizando, portanto, símbolos
pictográficos.

Tomando como base os relatos das atividades elaboradas para o estudante
com Transtorno do Espectro Autista, por \textcite{camargo2020}, temos aquelas
que são lúdicas e realizadas por meio de materiais concretos e as
impressas, ambas são adaptadas às necessidades educacionais do aluno.
Vejamos quais foram as utilizadas pela pesquisadora:

Lúdicas e com Materiais Concretos:
\begin{enumerate}[label=\alph*)]
\item jogos
\item quebra-cabeças
\item colagens
\item materiais concretos para ensinar sobre quantidade e adição
\item Comunicação Alternativa Aumentada
\end{enumerate}

Impressas:
\begin{enumerate}[label=\alph*)]
\item Atividades com Resolução de Problemas
\item Atividades abordando geometria
\item Atividades abordando sobre noções espaciais
\item Atividades baseadas na CAA
\item Atividades de correspondência entre objetos, números e formas.
\end{enumerate}

Apesar de elencar várias Tecnologias Assistivas, \textcite{camargo2020} só
definiu como TA a Comunicação Alternativa Aumentada quando disse que a
utilização da CAA foi primordial ``{[}...{]} uma vez que esse recurso de
tecnologia assistiva proporcionou a comunicação necessária para expor as
situações e fazer o aluno responder aos questionamentos, tornando
possível a relação entre aluno e conhecimento''
\cite[p.~80]{camargo2020}.

Nossa receptividade com relação a essa pesquisa foi muito boa. Essa
impressão se deu porque foi o único texto analisado por nós na
metanálise que traz exemplos concretos de atividades que podem ser
feitas aplicadas a estudantes com TEA, composto por plano de aula,
modelos de atividades, metodologias para aplicá-las e os resultados
encontrados. Acreditamos que esta dissertação possa servir como
embasamento para muitos docentes que têm alunos com esse transtorno.

Apesar de que, no título do trabalho, \textcite{camargo2020} proponha uma
pesquisa voltada para o ensino da Matemática, em suas considerações
finais não há um enfoque dos resultados de suas intervenções quanto aos
conhecimentos matemáticos adquiridos pelo aluno com TEA nesta área. A
autora foca mais em avaliar como foi a receptividade dos professores e
nos resultados da aplicação das atividades, assim como nos resultados
positivos que foram alcançados para a inclusão do estudante.

%%%% TABELA
\begin{footnotesize}
\begin{longtable}{
    >{\raggedright\arraybackslash}p{0.15\textwidth}
    >{\raggedright\arraybackslash}p{0.8\textwidth}
    }
\caption{Texto 7}
\label{tbl05}
\\
\toprule
Autor & Larissa Leal Scapin Gubert \\
Título & Uso de Tecnologias Assistivas no Ensino de Matemática em Salas de Recursos Multifuncionais em uma Rede Municipal de Ensino \\
Revista ou Instituição & Universidade Estadual do Oeste do Paraná \\
ISSN & --- \\
Qualis & --- \\
Páginas & 123 \\
Objetivo(s) & Compreender como as professoras de Salas de Recursos Multifuncionais,
atuantes nos anos iniciais do ensino básico em um município do oeste do
Paraná, estão utilizando as Tecnologias Assistivas para o ensino de
Matemática em sua prática e quais são as mais usadas pelas docentes.\\
Metodologia & Trata-se de uma pesquisa exploratória feita por meio de um levantamento bibliográfico\newline
Pesquisa de campo realizada com 15 professoras de Sala de Recursos de escolas municipais\\
Resultados & Ademais, a pesquisa realizada permitiu compreender como as professoras de
São Miguel do Iguaçu lidaram com as Tecnologias Assistivas bem como
com os Materiais Adaptados. Também foi possível ter um panorama geral
dessa rede de ensino. \\
Tecnologia Assistiva & Jogos matemáticos: concretos e digitais\newline
Materiais Adaptados (MA) pelas professoras da SRM\\
Fragilidades & Questionário com 26 questões mistas - 20 a 30 minutos para respondê-lo\newline 45 dias aberto\newline
14 (quatorze) Escolas Municipais e sete Colégios Estaduais\\
\bottomrule
\source{Os autores (2023).}
\end{longtable}
\end{footnotesize}



Nosso último texto, o Texto 7 (\Cref{tbl05}), é, também, uma dissertação de
Mestrado em Ensino da Universidade Estadual do Oeste do Paraná. O
trabalho de pesquisa é composto por 123 páginas e tem como objetivo
compreender como as professoras de Salas de Recursos Multifuncionais
(SRM) utilizam as Tecnologias Assistivas para o ensino de Matemática.
Apesar de não focar apenas no Transtorno do Espectro Autista em sua
pesquisa, pois apresenta outras deficiências, \textcite{gubert2022} discute a
respeito da SRM onde as crianças com TEA são atendidas.

A autora começa seu texto com a contextualização da inclusão no Brasil e
depois apresenta documentos oficiais da educação nesse contexto. Na
sequência, fala sobre a implantação de Salas de Recursos
Multifuncionais, das Tecnologias Assistivas que podem ser utilizadas
nesse ambiente escolar e, por fim, do ensino de Matemática numa
perspectiva inclusiva. Há um capítulo específico em que \textcite{gubert2022}
faz uma revisão narrativa, com três dissertações e duas teses, sobre o
ensino de Matemática em SRM.

\textcite{gubert2022} organizou a análise dos dados do questionário em três
etapas: Salas de Recursos Multifuncionais e Inclusão; Matemática e Salas
de Recursos Multifuncionais; Tecnologias Assistivas e Materiais
Adaptados. A pesquisa de campo apontou para uma equipe de professores
com uma boa formação especializada em inclusão. Além disso, apontou para
uma regular estrutura arquitetônica das SRM, assim como, de mobiliários
e materiais pedagógicos.


Chamou-nos a atenção a distinção que
a autora faz em relação a Tecnologia Assistiva e Materiais Adaptados.
Para \textcite{gubert2022}, enquanto as TAs se tratam de materiais ou métodos
que foram desenvolvidos pensando em uma determinada deficiência, os MA
surgem a partir de algo que já estava pronto e se realiza uma adaptação
para determinada deficiência. Ambos precisam cumprir o papel de auxiliar
o aluno em seu desenvolvimento e na busca do conhecimento e promover
novas formas de se ensinar.

Houve dificuldades das participantes em definir o que é Tecnologia
Assistiva, algumas confundiram TA e MA, outras copiaram a resposta da
internet, apenas duas disseram que se trata de uma área do conhecimento
que engloba desde recursos tecnológicos até práticas diferenciadas. Ao
definirem as TAs três palavras se destacaram: metodologias, materiais
concretos, tecnologias e adaptações.

Apresentou como exemplos de Tecnologias Assistivas para o ensino de
Matemática para os alunos da Sala de Recursos Multifuncionais os jogos
matemáticos concretos e as tecnologias digitais. \textcite{gubert2022} descreve
as dificuldades relatadas pelas participantes: o tempo de atendimento
insuficiente; o fato de terem que oferecer conhecimentos de várias
áreas; encontrar ou produzir materiais adequados; os materiais
existentes não são suficientes; o sentimento de rejeição da Matemática
por parte de algumas professoras; ao mesmo tempo em que acreditam que o
professor da SRM deve buscar uma autoformação constante, bem como ser um
pesquisador na área da Educação Especial.

Entendemos como uma fragilidade do texto o fato de \textcite{gubert2022} dizer
ter participado 15 professores, mas não especificar quais são as escolas
vinculadas ao município e quais ao estado. Além disso, acreditamos que
um questionário que se leva até 30 minutos para ser respondido é bem
extenso e, talvez, por isso tenha ficado aberto tanto tempo e ter havido
a necessidade de estender o período por mais 15 dias e, mesmo assim,
ainda ter várias questões sem responder no instrumento de coleta de
dados.

Como resultado desta Metanálise apontamos primeiramente as distinções
dos tipos de pesquisa realizadas nos sete Textos. Cada pesquisador a seu
modo abordou sobre o ensino de Matemática para pessoas com Transtorno do
Espectro Autista de um modo diferente. Dos sete, apenas um realizou uma
pesquisa apenas teórica, os demais foram a campo para realizar
entrevistas, intervenções ou testar possibilidades. Consideramos essas
ações em meio ao ambiente escolar como positiva, já que leva
conhecimentos científicos e proporciona maior concretização daquilo que
se escreve.

Dos sete textos analisados, há aqueles que acreditam que a participação
da família é fundamental para o processo de aprendizagem do estudante
(Textos 1 e 2). Aqueles que resolveram estabelecer sua pesquisa na sala
regular (Textos 1, 3, 4, 5 e 6) e outros que decidiram investigar sobre
TA nas Salas de Recursos Multifuncionais (Textos 2 e 7).

Organizamos os Tipos de Tecnologias Assistivas que foram citadas nos
sete Textos analisados nesta Metanálise. Classificamos nas em
tecnológicas, pedagógicas, metodológicas, mesmo que algumas delas se
confundam ou possam ser classificadas em dois ou três tipos ao mesmo
tempo.

Podemos notar que as Tecnologias Assistivas mais citadas para o ensino
de Matemática para estudantes com Autismo são as TA Pedagógicas (Textos,
2, 4, 6 e 7), seguidas das TA Tecnológicas (Textos 1, 3 e 7), e apenas o
Texto 5 deu um enfoque nas TA Metodológicas. Em comum, os textos têm a
ideia de que o importante é conhecer as necessidades dos alunos e
elaborar um meio de levá-los a aprender, independente do tipo de recurso
que se utilize.

Encontramos também que os Materiais Adaptados estão presentes como um
dos meios mais adequados para levar os estudantes com TEA a aprenderem
Matemática (Textos 2, 4, 6 e 7). Presentes em todos os sete Textos estão
os jogos, seja no formato físico ou por meio das tecnologias, é
unanimidade entre os autores que os jogos são TAs bastante eficiente
para se ensinar Matemática para crianças com deficiência, principalmente
porque oportuniza a interação social e elas aprendem brincando.




\section{Considerações Finais}
Ao final deste trabalho de pesquisa podemos dizer que alcançamos os
objetivos específicos traçados. Isso porque conseguimos descrever o
contexto da Educação Matemática e da Educação Matemática inclusiva no
Brasil; da mesma forma, sobre as características do Transtorno do
Espectro Autista (TEA), com ênfase no processo de ensino-aprendizagem; e
também elencamos tecnologias assistivas para o ensino de estudantes com
TEA, com base na Metanálise realizada.

Também, alcançamos nosso objetivo geral que foi identificar, descrever e
analisar as Tecnologias Assistivas que podem auxiliar no processo de
ensino-aprendizagem de Matemática de alunos com Transtorno do Espectro
Autista (TEA), nos anos iniciais do Ensino Fundamental.

O que se pode perceber é que os textos utilizados por nós para a
construção deste trabalho não se preocupam com o nível do Transtorno do
Espectro Autista, já que na diferença entre eles ainda há constrovérsias
e em cada pessoa se manifesta de uma forma distinta. O que importa,
segundo os autores aqui estudados, é que deve-se analisar as
necessidades educacionais do estudante e elaborar meios para saná-las,
tendo em vista que cada aluno é único, independente do nível de TEA que
apresente.

Além disso, há o consenso entre os autores de que é preciso realizar
adaptações e adequações no currículo, nas práticas educativas, nas
metodologias e nas avaliações, do contexto escolar, para incluir os
estudantes com Transtorno do Espectro Autista. No caso da disciplina da
Matemática, é preciso aplicar ferramentas que sejam concretas e façam
com que os alunos entendam o raciocínio matemático, bem como reconheçam
que ela faz parte do seu cotidiano.

Também concluímos que as Tecnologias Assistivas são utilizadas tanto nas
salas regulares, com o professor regente e/ou o professor de apoio,
quanto nas Salas de Recursos Multifuncionais pelo professor de AEE. Na
verdade, o que se propõe nos documentos norteadores da Educação Especial
é que um complemente o outro e que seja realizado um trabalho
colaborativo entre ambos.



\printbibliography\label{sec-bib}
%conceptualization,datacuration,formalanalysis,funding,investigation,methodology,projadm,resources,software,supervision,validation,visualization,writing,review
\begin{contributors}[sec-contributors]
\authorcontribution{Camila Rezende Oliveira}[conceptualization,writing,review]
\authorcontribution{Cjanna Vieira Lopes}[methodology,datacuration,investigation,writing,review]
\authorcontribution{Guilherme Saramago de Oliveira}[conceptualization,writing,review]
\end{contributors}
\end{document}
