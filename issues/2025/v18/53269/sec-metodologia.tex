% !TeX root = main.tex

\section{Metodologia}\label{sec-metodologia}

Nesta seção, delineamos os procedimentos metodológicos adotados para
criar o banco de dados a partir do qual selecionamos o \emph{corpus} da
pesquisa. Nosso estudo é de caráter não experimental, pois estamos
observando os fenômenos conforme ocorrem em seu ambiente natural,
conforme \textcite{sampieri2006}. Dentro desse contexto, a
geração de dados foi realizada de maneira qualitativa, seguindo os
critérios estabelecidos adiante. Para analisar o \emph{corpus}, é
importante destacar que ele consiste em \emph{songfics} (ou
canções-\emph{fic}), ou seja, narrativas escritas e divulgadas por fãs
no ciberespaço. Nesse ínterim, \cite{bakhtin_os_1997} enfatiza que o uso da
língua se manifesta por meio de enunciados concretos e singulares, os
quais refletem as condições específicas de cada contexto e, à medida que
são empregados, resultam nos gêneros do discurso. Essa perspectiva
teórica nos permite analisar as particularidades relacionadas ao
\emph{conteúdo temático}, \emph{estilo} e \emph{construção
composicional} desses enunciados.

Ademais, objetivamos analisar as estratégias intertextuais e
hipertextuais, bem como as relações dialógicas presentes em um exemplar
do gênero \emph{songfic}, atentando-nos aos elementos compartilhados
entre as histórias e as demais obras artísticas que as inspiraram. Dito
isso, também examinamos os recursos multimodais empregados pelos autores
das \emph{fics} na criação de suas obras, observando o papel desses
recursos no desenvolvimento da trama. Por fim, descrevemos os recursos
hipertextuais presentes no gênero, como a inclusão (ou ausência) de
\emph{links}, GIFs, vídeos e imagens. Dessa forma, nossa análise, para
além da superfície textual, estende-se aos discursos evocados por esses
elementos, bem como às relações materializadas intertextualmente. Para
encontrar a \emph{songfic} referida, exploramos o \emph{website} Spirit
Fanfics\footnote{Endereço eletrônico:
  \href{https://www.spiritfanfiction.com/home}{https://www.spiritfanfiction.com/home}.},
haja vista a possibilidade de efetuar a busca por palavras-chave na
plataforma, bem como a disponibilidade da opção \emph{Musical
(songfics)} na aba \emph{Gêneros.} O critério de seleção abarcou os
seguintes critérios: 1) popularidade e interação, posto que a troca
entre os membros dos \emph{fandoms} são um traço vital desse gênero
discursivo; 2) relação explícita, nas etiquetas da história, a algum
elemento da indústria musical (canção, álbum, artista etc.).

Essencialmente, este artigo enfoca a natureza responsiva da linguagem,
uma vez que frequentemente estamos aludindo a enunciados de outras
pessoas --- ou a nós mesmos. Em tais circunstâncias, acreditamos que a
análise do gênero \emph{songfic} pode contribuir para a sociedade no que
diz respeito aos fenômenos de intertextualidade, hipertextualidade e
relações dialógicas. Isso nos leva à compreensão de que a própria
cultura se renova por meio da linguagem, nas mãos dos indivíduos que a
utilizam. Posto isso, vejamos, a seguir, de que modo os constructos
teóricos dialógicos defendidos por Bakhtin e o Círculo podem ser
aplicados a esse universo dominado pelos fãs.
