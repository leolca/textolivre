% !TeX root = main.tex
\section{Considerações finais}\label{sec-considerações}

Em suma, ao analisar a \emph{songfic} \emph{Cry Baby}, averiguamos uma
série de referências (diretas e indiretas) não somente às letras das
canções, mas, para a nossa surpresa, também aos videoclipes e encarte do
álbum. Também identificamos alguns Elementos de Reinterpretação
\cite{jenkins_textual_1992}, como o \emph{Crossover}, \emph{Expansão da Linha
Temporal}, entre outros, no decorrer da narrativa. Para coordenar todos
esses elementos, a \emph{ficwriter} utilizou recursos visuais, incluindo
imagens, bem como trechos do \emph{story book}. Nos capítulos seguintes,
ela até mesmo transcreve estrofes de outras canções da obra em inglês,
como \emph{Sippy Cup}, aprofundando assim as personalidades das
personagens e introduzindo várias reviravoltas ao longo da narrativa.
Cada capítulo estabelece conexões não apenas com as obras midiáticas
sinalizadas, mas também com outros discursos, enriquecendo assim o
processo de reinterpretação das \emph{songfics}. Isso resulta em um
diálogo intenso e responsivo \cite{bakhtin2017} da \emph{ficwriter} com
outras vozes no fluxo dialógico da língua, contribuindo para a complexa
rede de discursos e a valoração temática. A \emph{songfic} analisada
pode ser vista como uma continuação dos eventos do álbum, reafirmando os
temas abordados por Melanie Martinez.

Consoante \textcite[p.~129]{ribeiro_avaliacao_2011}, ``todas as novas formas de ler parecem
vilãs de um tempo sem calor, quando, na verdade, são apenas
possibilidades para algo que já se fazia e já se fez na história das
interfaces de leitura, interfaces homem/objeto de leitura''. Nessa
direção, a literatura para crianças e jovens tem usufruído das dinâmicas
estabelecidas no mundo digital, cujas ferramentas, potencializadas pelas
múltiplas mídias e \emph{hiperlinks}, provocam reelaborações de toda
sorte nos gêneros discursivos com os quais lidamos cotidianamente. Desse
modo, não se trata apenas de ler uma \emph{songfic}, mas de interagir
ativamente com ela, especialmente nos comentários e com os outros
integrantes dos \emph{fandoms}. Nesse panorama, no entanto, cabe
cautela, posto que são vários os textos publicados na \emph{web}, o que,
por consequência, demanda uma curadoria apurada; além disso, o próprio
modo de leitura, agora muito mais disperso, carece de um letramento
digital específico --- mas que não se distancia tanto das formas mais
tradicionais. À parte disso, com as \emph{songfics}, estamos lidando com
um gênero discursivo nativamente digital, próprio da contemporaneidade.
É a expressão artística da cibercultura, o vórtice da arte, a união de
mídias, fãs e obras literárias.


