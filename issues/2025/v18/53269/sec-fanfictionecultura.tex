% !TeX root = main.tex

\section{Fanfiction e cultura participativa: a manifestação do fandom}\label{sec-fanfictionecultura}

No ensaio \emph{Arte e responsabilidade}, \textcite{bakhtin_arte_2003} argumenta que a
\emph{ciência}, a \emph{arte} e a \emph{vida} são os três domínios da
cultura humana e eles só adquirem unidade no indivíduo que os incorpora
em sua própria unidade. Em outras palavras, as diferentes partes desse
todo, a cultura humana, não devem estar ligadas de maneira mecânica ou
externa, como quando ``o homem sai da `agitação do dia a dia' para a
criação como para outro mundo `de inspiração, sons doces e orações'\,''
\cite[p.~33]{bakhtin2011}. Isso resultaria em uma arte vazia e ousada, em
vez de responsiva à vida. \textcite{bakhtin2011} defende que, isto sim, é a
unidade da responsabilidade mútua que garante o elo interno entre os
elementos do indivíduo, ao afirmar que ``Arte e vida não são a mesma
coisa, mas devem se tornar algo singular em mim, na unidade da minha
responsabilidade'' \cite[p.~34]{bakhtin2011}. Consequentemente, essa ideia de
responsividade perpassa grande parte da teoria do autor. De outro modo,
para o pensador russo, viver é agir e a vida é permeada por relações
dialógicas, as quais se voltam tanto para o presente, no aqui e agora,
quanto para o passado/futuro, sempre direcionadas a um \emph{outro} com
o qual \emph{eu} me relaciono. Ao enunciar, tomamos uma posição perante
o próprio mundo, a partir de um lugar social, ideológico e axiológico.

Toda essa interação acarreta tipos de enunciados \emph{relativamente
estáveis}, que circulam em cada campo da comunicação humana e são
denominados ``gêneros do discurso''. Esses gêneros desempenham o papel
de organizar nossa fala e escrita \cite[p.~262]{bakhtin2011}. Eles são
considerados desse modo, porque, embora compartilhem características em
comum, a língua está em constante processo de transformação e os
enunciados não são formas acabadas ou totalmente inflexíveis. Embora
cada enunciação seja inerentemente única, de um ponto de vista
universal, os gêneros são sustentados por três pilares: o \emph{conteúdo
temático}, o \emph{estilo} e a \emph{forma composicional}.

O \emph{conteúdo}, segundo \textcite{bakhtin2010,bakhtin2011}, é um elemento
ético-cognitivo que engloba tanto o assunto evocado no gênero quanto os
aspectos ideológicos, objetivos e posição axiológica subsequentes; é
parte de uma macroestrutura e considera a singularidade do sujeito, sua
vontade e seus conhecimentos, orientando a comunicação discursiva com
abordagem valorativa do objeto e fatores linguísticos, textuais e
discursivos. O \emph{estilo}, por seu turno, abrange escolhas lexicais,
gramaticais e fraseológicas, em um contínuo de mais ou menos
formalidade, refletindo a individualidade do enunciador, mesmo em
gêneros mais regulados. O estilo individual é um epifenômeno do
enunciado, enquanto o estilo do gênero, ligado às práticas da
comunidade, garante sua estabilidade. Por fim, a \emph{forma
composicional} tem a ver com a estrutura geral, a organização textual e
as propriedades particulares do gênero, como a divisão de um artigo em
seções, por exemplo. Esses três pilares estão interconectados no
enunciado: o conteúdo requer uma forma para ser expresso
linguisticamente, o que, por sua vez, implica um estilo, tanto do gênero
quanto individual. Esses elementos estão organicamente integrados no
conjunto do enunciado e são moldados pelo caráter específico de cada
campo de comunicação, conforme orienta \textcite{bakhtin2011}.

Na incessante onda de transformações da \emph{internet} e redes sociais,
os gêneros discursivos também se adaptam e evoluem, assim como o
\emph{e-mail}, descendente da carta tradicional. \textcite{jenkins2009} destaca
três conceitos relevantes para esse fenômeno: a convergência de conteúdo
em múltiplas plataformas, a cultura participativa dos usuários e a
inteligência coletiva dispersa entre as pessoas. Esses elementos
influenciam as \emph{fanfictions} e a cultura midiática em geral.
Autores renomados, como Shakespeare, já faziam uso de ideias alheias em
suas criações antes que surgisse o conceito de propriedade intelectual,
semelhante ao Alonso Fernández de Avellaneda com sua sequência não
oficial de Dom Quixote, em 1614. Logo, a prática de criar obras baseadas
em outras precede a invenção do termo \emph{fanfiction}, no século XX.
Inicialmente, na década de 1930, aparecem os \emph{fanzines}, seguidos
pelas comunidades de fãs que compartilhavam e celebravam seus interesses
comuns. Com a ascensão da internet comercial, os \emph{fandoms}
encontraram um lar, agregando-se e produzindo conteúdo artístico
coletivamente. Em 1998, surge o FanFiction.Net, seguido por outros
fóruns e plataformas. Nessas comunidades \emph{on-line}, os fãs
engajam-se ativamente como criadores, alterando narrativas canônicas e
oferecendo novas perspectivas por meio das \emph{fanfics}.

Retornando às ideias de \textcite{bakhtin2011} sobre gêneros discursivos e
dialogismo, acreditamos que os participantes da interação, nesse
ambiente, são majoritariamente os membros dos \emph{fandoms},
estabelecendo-se um diálogo real e responsivo na confecção de suas
manifestações artísticas. A alternância dos sujeitos, uma das
peculiaridades do enunciado, é marcada por cada publicação ou comentário
nas comunidades, refletindo uma constante relação entre o \emph{eu} e o
\emph{outro}, repleta de retomadas, concordâncias e dissonâncias. Os
fanfiqueiros assumem uma posição axiológica no ato da interação, de modo
que tudo está inserido na arquitetônica da vida real --- da qual fazem
parte os indivíduos que compõem os \emph{fandoms}, considerando um
indivíduo responsável, ativo e responsivo --- e na arquitetônica dos
enunciados. Esses dois aspectos estão interligados porque a vida se
integra à língua por meio de enunciados concretos; e é por meio de
enunciados concretos que a língua adentra a vida, como defende \textcite{bakhtin2011} .

À vista disso, o gênero \emph{fanfiction} está relativamente estável
considerando seus três pilares: o \emph{conteúdo temático}, nesse caso,
está ligado aos aspectos ideológicos dos enunciados, apresentando uma
variedade de temas muito ampla, incapaz de ser mensurada, dada a
quantidade de \emph{fics} publicadas até hoje; o \emph{estilo} é
gerenciado por duas forças importantes: de um lado, o gênero
\emph{fanfiction} e suas características determinadas pelas condições
sócio-históricas que o cercam; de outro, o(s) \emph{ficwriter(s)}, isto
é, o(s) indivíduo(s) por trás daquele novo produto midiático, com suas
marcas pessoais e criativas --- além disso, tendo em vista a
sobreposição do estilo individual, os fanfiqueiros empregam recursos
multissemióticos e construções fraseológicas específicas; por fim, a
\emph{forma composicional} das \emph{fics} é multimodal e adaptável,
dependendo do suporte em que estão inseridas. Elas seguem uma
regularidade na apresentação de informações, como título, pseudônimo,
capa etc. A criatividade e os desejos dos \emph{fandoms} impulsionam as
narrativas, explorando lacunas deixadas nas histórias originais ou
representando comportamentos alternativos ao \emph{mainstream}. Posto
isso, na subseção a seguir, discorremos a respeito de algumas nuances
dos elementos imagéticos que integram as \emph{songfics}.

\subsection{Laços Intertextuais} \label{sub-sec-laços}
  

Constantemente recorremos ao discurso de outras pessoas. De forma geral,
tudo o que dizemos ou interpretamos tem uma origem, que pode ser mais ou
menos visível. Nesse contexto, \textcite{koch2008} retomam
as ideias de Kristeva, autora francesa que cunhou o termo
"intertextualidade" com base em sua interpretação das obras de Bakhtin,
apesar das polêmicas envolvendo essa questão. Para ela, cada texto é um
intertexto, inserido em uma sequência de textos já existentes ou que
ainda serão escritos. Contudo, é importante observar que, em nossa
análise, não consideramos esse conceito como sinônimo de dialogismo.
Isso se deve ao fato de que a intertextualidade se concentra,
majoritariamente, nas relações entre textos; por outro lado, a teoria
bakhtiniana aborda as relações dialógicas de maneira situada, destacando
a natureza responsiva dos enunciados.

Seguindo a mesma lógica, \textcite{koch2008} afirmam que a
Linguística Textual abraça o princípio dialógico de \textcite{bakhtin_marxismo_2009}, reconhecendo que textos estão sempre dialogando com outros e não
podem ser analisados isoladamente. Cada texto interage com seu entorno,
ganhando sentido em relação aos outros. Conforme \textcite{koch2008}, existem duas formas de intertextualidade, quais
sejam: a) estrita, quando há uma presença clara de um texto inserido no
outro; b) ampla, referindo-se a indícios mais sutis ligados ao formato,
estilo ou temas. \apud{barthes_morte_2004}{koch1991}
sustenta que um texto é um mosaico de diferentes textos que lhe dão forma e significado. Portanto, a intertextualidade é semelhante à interdiscursividade, na
medida em que ambos respondem a discursos anteriores, mas o dialogismo é
distinto, pois também está preocupado com as respostas futuras -- cada
enunciado espera por compreensão e resposta ativa, nos termos de \textcite{fiorin_interdiscursividade_2006}. Na prática, neste artigo, usamos interdiscursividade para
descrever relações dialógicas mais amplas, enquanto o conceito de
intertextualidade é utilizado mais especificamente para relações
manifestadas textualmente.

De acordo com \textcite{fiorin_interdiscursividade_2006}, é importante observar que a
intertextualidade sempre implica em uma interdiscursividade, mas o
inverso nem sempre é verdadeiro. Levando isso em conta, podemos analisar
alguns tipos de \emph{intertextualidade estrita}, baseados na obra de
\textcite{genette_palimpsestos:_2010}, conforme apresentado por \textcite{koch2008}: a \emph{citação}, a mais explícita, envolve a inclusão direta de
um trecho de um texto em outro, frequentemente indicado por aspas; o
\emph{parafraseamento} consiste na reformulação de um texto para servir
a diferentes propósitos, públicos ou contextos, resultando em
modificações de sentido; e, por fim, a \emph{alusão estrita} é uma
referência indireta e sutil a um texto, muitas vezes introduzindo
alterações formais e adaptando-o para fins diversos, como humor ou
crítica. Cada uma dessas formas desempenha um papel crucial na
construção do significado.