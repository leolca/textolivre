% !TeX root = main.tex
\section{Análise e discussão: da superfície às profundezas do discurso}\label{sec-analise}

Com o objetivo de examinar as estratégias intertextuais e hipertextuais
utilizadas pelos escritores de \emph{fics}, bem como as conexões
dialógicas estabelecidas entre as canções e as histórias inspiradas
nelas, vejamos a \emph{songfic} intitulada \emph{Cry Baby}, escrita e
atualizada por BrookeYoongi entre 2016 e 2019. Durante esse período, a
fã-autora produziu 16 capítulos, cada um dedicado a uma das músicas do
álbum homônimo da cantora norte-americana Melanie Martinez, totalizando
aproximadamente 48 mil palavras. Nessa abordagem, os temas evocados na
narrativa se entrelaçam com aqueles presentes no álbum que a inspirou.
Com esse movimento, os conteúdos temáticos evocados na narrativa
dialogam com aqueles do disco inspirador; e, ao reenunciá-lo, a autora
adota uma atitude responsiva não apenas com relação às canções, mas
também à própria vida, afinal é por meio de enunciados concretos que a
vida entra na língua --- e, como afirma \textcite{bakhtin2011}, a arte deve
responder à vida. Dessa forma, um \emph{eu} concreto, neste caso, a
escritora da \emph{songfic}, volta-se para um \emph{outro}, que não se
limita apenas a indivíduos empíricos, mas também as canções, neste caso,
estabelecendo um diálogo vivo e axiológico.

Na sinopse da \emph{fic}, somos apresentados à narrativa que gira em
torno da vida de Cry Baby, também conhecida como Mary, uma garota que
possui uma amiga imaginária chamada Lena, possivelmente ``a única pessoa
que realmente tenha lhe mostrado `amor'\,'' \cite{brookeyoongi}.
Apesar de sua tenra idade, Mary teve sua inocência e pureza roubadas, o
que a impede de vivenciar o mundo com a mesma alegria das outras
crianças. Sua vida se assemelha a uma representação teatral, mas sem
roteiro definido. Como BrookeYoongi assevera, é como um conto de fadas,
embora desprovido de pureza e ausente de uma fada madrinha. A capa da
\emph{songfic}\footnote{Disponível em:
  \href{https://www.spiritfanfiction.com/historia/cry-baby-7084287}{https://www.spiritfanfiction.com/historia/cry-baby-7084287}.}
é, na verdade, uma montagem da arte do álbum \emph{Cry Baby}, de Melanie
Martinez, onde as lágrimas da cantora formam gotas de chuva que inundam
a cidade, com um fundo rosa, sobreposta à figura de uma menina,
provavelmente a personagem Lucy. A \emph{canção-fic} também fornece
informações como a data de início e término de publicação, o idioma em
que foi escrita, o número de visualizações (mais de 3000), o número de
pessoas que a adicionaram aos seus favoritos (118), e outros elementos
comuns ao gênero. Embora as personagens sejam descritas como originais,
ou seja, criações da autora da \emph{fic}, há similaridades notáveis
entre essas personagens e aquelas apresentadas por Melanie Martinez em
seu trabalho. A classificação indicativa é de 16 anos, devido aos temas
abordados na história.

Para contextualizar, é necessário compreender que a intérprete, além de
lançar as músicas, enriqueceu ainda mais sua criação, ao incluir um
pequeno livro de histórias com 16 páginas no encarte do álbum. Esse
livro expande o conteúdo temático de cada uma das 16 canções do disco,
apresentando ilustrações que aprofundam a experiência, ao enriquecer a
rede de referências e verbo-visualidade das \emph{fics}. Com esse
detalhe, fica claro que a fã-autora não se limitou apenas às letras das
músicas, mas foi além, estabelecendo um diálogo também com o livro de
histórias (\emph{story book}) que acompanha o CD. Além disso, a
narrativa é influenciada pelos videoclipes da cantora. Algumas cenas do
universo cinematográfico de Melanie Martinez foram adaptadas na
história, com acréscimos e ajustes. Portanto, em uma única obra,
observamos a atuação de três influências midiáticas: as letras das
músicas, o \emph{story book} e os videoclipes. Nesse contexto, \textcite{jenkins2009} já tratava do fenômeno da convergência, no qual as mídias
integram e se complementam, resultando em novas formas de produção,
distribuição e consumo de conteúdo, como é evidenciado na interação
entre essas diferentes formas de expressão artística na \emph{songfic}.

No início da narrativa, BrookeYoongi segue fielmente os acontecimentos
que antecedem o nascimento de Cry Baby, valendo-se de um
``\textbf{Flashback ON \textasciitilde\textasciitilde{}}'', estilizado
em negrito. A mãe de Cry Baby, na \emph{songfic}, é descrita como uma
mulher loira, pele pálida, quase como uma assombração, se vista de
longe. Apesar de tais características não serem mencionadas na canção
original, ao analisarmos o clipe, percebemos a relação empreendida, uma
vez que a atriz que interpreta a mãe, Stella R. S. Clair, corresponde
aos traços da narrativa. O nome da bebê é atribuído devido ao seu choro
constante, refletindo a história da personagem, marcada por emoções
intensas e uma sensibilidade, muitas vezes, ignorada pelas pessoas ao
seu redor. Até esse ponto, a \emph{songfic} tece uma intertextualidade
implícita com o encarte do álbum, referenciando a letra da canção onde
Melanie Martinez descreve a personagem como ``A mais triste garota'' \cite{crybaby2015} e menciona que ``Seu coração é grande demais para seu corpo /
É por isso que ele não cabe dentro / Você o derrama onde todo mundo pode
vê-lo''\footnote{``Your heart\textquotesingle s too big for your body,
it\textquotesingle s why it won\textquotesingle t fit inside / You
pour it out where everyone can see'' \cite{crybaby2015}.} \cite{crybaby2015}, alinhando-se com a essência da personagem.

Na história, Cry Baby conversa com sua amiga imaginária, Lena, enquanto
celebra seus 10 anos de idade. A presença de Lena é uma adição da autora
e é considerada uma \emph{Expansão da Linha Temporal}, segundo \textcite{jenkins_textual_1992}, ampliando a narrativa para além dos 7 anos descritos
originalmente. Ao interagir com Lena e um ursinho de pelúcia, há uma
referência indireta à canção \emph{Teddy Bear}, de Melanie Martinez, o
que enriquece o enredo original e adiciona camadas de sentidos, ao
estabelecer conexões com outros elementos do trabalho. Esta abordagem
expande a trama, aumentando sua complexidade e profundidade.

Pode-se inferir, a seguir, um pouco mais do ambiente familiar de Mary
(nome que Cry Baby recebeu de sua amiga imaginária): ``Chegou na
cozinha, e olhou para a mesa. Estava vazia. Pegou um pouco de arroz, um
frango e colve {[}\emph{sic}{]}. O suco, ela tomaria depois de acabar
tudo. Foi assim que ela sempre fez'' ou mesmo em ``Seu pai, já tinha
saído {[}\ldots{]} / Seu irmão {[}\ldots{]} estava no quarto, local de
onde vinham barulhos, que a pequena ainda não entendia. Já sua mãe,
estava sentada {[}\ldots{]} bebendo o que tinha na garrafa, {[}\ldots{]}
sua companheira leal e fiel'' \cite{brookeyoongi}. Esse cenário
revela uma intertextualidade implícita com outra canção do álbum,
\emph{Dollhouse}, na qual a família de Cry Baby é retratada como uma
farsa, vivendo em uma casa de bonecas, de sorte que ``todo mundo pensa
que somos perfeitos / por favor não deixe que vejam através das
cortinas''\footnote{``Everyone thinks that we're perfect / Please don't
  let `em look through the curtains'' \cite{dollhouse2015}.} \cite{dollhouse2015}. Embora a autora da \emph{canção-fic} não tenha mencionado
explicitamente essa ligação, a presença de elementos como o cenário da
infância, as relações familiares disfuncionais e a dinâmica de
aparências podem ser interpretadas como uma sutil referência a essa
canção.

Lançando mão do dialogismo bakhtiniano \citeyear{bakhtin2011}, podemos inferir que esses
trechos não apenas refletem a experiência pessoal da personagem, mas
também apontam para a ideia de que, assim como não existem pessoas
perfeitas, de tal modo não há famílias perfeitas, por mais que possam
parecer à primeira vista. Trata-se de um retrato da realidade por meio
da arte. Posteriormente, na \emph{songfic}, em seu primeiro dia de aula,
Mary (ou Cry Baby) vai sozinha para a escola, sem estar acompanhada da
mãe:

\begin{quote}
Enquanto ela andava, as outras crianças olhavam para ela, já comentando
algumas coisas sobre a mesma {[}\emph{sic}{]} em um sussurro baixo, mas
que algumas vezes ela conseguia ouvir.

\emph{``Por que ela veio sozinha?''}

\emph{``Será que ela tem pais?''}

\emph{``Ela é estranha\ldots''} \cite[cap.~1, grifos da autora]{brookeyoongi} %(BrookeYoongi, 2016--2019, cap. 1, grifos da autora).
\end{quote}

Assim, no excerto, a fanfiqueira utiliza itálico para destacar a fala
das personagens secundárias, além de empregar o tempo verbal no passado,
principalmente o pretérito perfeito do indicativo e o imperfeito. Isso,
juntamente ao narrador em terceira pessoa, possibilita diferentes
perspectivas e o distanciamento da personagem principal, Mary. O estilo
adotado combina elementos do gênero, semelhante ao que ocorre em
romances, contos e crônicas, com toques individuais, como a demarcação
de falas em itálico e entre aspas. Outro indício de intertextualidade
implícita, na \emph{songfic}, é a descrição e o nome atribuídos à
professora da turma: ``{[}\ldots{]} uma figura mais velha entrava pela
porta / {[}\ldots{]} --- Bem, como muitos devem saber, sou a professora
Minerva. - falou a mulher de cabelos meio grisalhos, encarando as
crianças / Seu olhar parecia lhes dizer `eu sei tudo que vocês fizeram
de errado'' \cite{brookeyoongi}. Na saga Harry Potter, de J.K.
Rowling, também temos uma personagem com esse nome, a Profa. Minerva
McGonagall, descrita como ``uma mulher de aspecto severo que usava
óculos de lestes quadradas exatamente do formato das marcas que o gato
tinha em volta dos olhos'' \cite[p.13]{rowling2000}, sendo bastante
rígida, mas justa com os alunos. Isso é, inclusive, percebido por uma
internauta, que elogia a obra, diz estar ansiosa pelos próximos
capítulos e utiliza caracteres de coração 
({\Symbola ♡} e {\Symbola ♥}).

A estratégia de reinterpretação adotada nesse ponto pode ser
identificada como um \emph{Crossover}, onde elementos de uma história
são incorporados em outra. Embora não tenha havido uma referência
explícita à saga do ``bruxinho'', essa associação pôde ser inferida com
base nas características da personagem. No entanto, para leitores não
familiarizados com esse contexto, essa pista passaria despercebida, ao
contrário do que aconteceu com uma internauta que comentou o texto. De
fato, os membros dos \emph{fandoms} desempenham o papel de
interlocutores nesse ambiente, engajando-se em diálogos tanto entre eles
como com os discursos presentes nas obras que admiram. Posteriormente,
na \emph{songfic}, um caso de \emph{bullying} é descrito durante a
chamada: ``Riram do nome que ela tinha, só por seu significado?
Não\ldots{} Eles riram dela, por ela ter esse nome / Sentia os olhares a
acusando de algo que não fez'' \cite{brookeyoongi}. Momentos
depois, alguém lhe atira uma bolinha de papel, um bilhete amassado, em
que se lê: ``Você realmente não conseguia parar de chorar. Eles não
deveriam aguentar você, e colocam esse nome só por colocar. Vai chorar
agora, bebê chorona?'' \cite{brookeyoongi}. Essa situação foi o
gatilho para que a menina começasse a chorar incessantemente, ansiando
ser "normal" como as outras crianças de sua idade. Nesse aspecto,
pode-se estabelecer uma conexão temática com a canção-fonte da
narrativa, quando Melanie canta ``Você é uma pessoa única que ninguém
entende / Mas essas lágrimas de bebê chorona sempre voltam'' \cite{dollhouse2015}.

Na canção, a cantora compara-se à sensibilidade de Cry Baby para
confortar seu \emph{alter ego}, incentivando-a a expressar suas emoções.
Tanto na letra quanto na \emph{songfic}, Cry Baby (ou Mary) começa a
chorar antes de conseguir se explicar, estabelecendo assim uma
intertextualidade implícita entre as obras. Embora a canção não seja
transcrita na íntegra, seu conteúdo é diluído na história. Logo depois,
na vida de Mary, ocorre mais um evento perturbador. Durante o intervalo,
por não ter brinquedos além de uma boneca suja e um ursinho de pelúcia
destruído, ela decide brincar com a Barbie de uma colega, que logo
percebe e inicia uma discussão. Como resultado, ``Não demorou muito,
para a cabeça da boneca ficar nas mãos da loira, e o restante do corpo,
nas mãos de Cry Baby'' \cite{brookeyoongi}. Com efeito, a colega
de Mary, irritada, esbraveja:

\begin{quote}
--- Você... - falou a loira, irritada. A menina de cabelos enrolados mal
conseguiu ver, quando a loira pegou a cabeça da boneca, e atirou com
toda a força que podia na testa da garota. Assim que ela sentiu o
impacto, fechou os olhos. Quando os abriu novamente, depois de alguns
segundos, a loira estava com a sua boneca na mão, e o urso de pelúcia na
outra. Ela lançava um olhar um tanto demoníaco a Cry Baby.

--- Não... - falou a enrolada, como se pudesse ler os pensamentos da
loira. Mas, mesmo que tenha dito aquilo em um tom choroso, de nada
adiantou para mudar a ideia da loira \cite{brookeyoongi}.
\end{quote}

Após a agressão física descrita, a "loira", conforme detalhado,
prosseguiu destruindo a boneca, pisoteando-a e chutando-a com raiva.
Mais uma vez, lágrimas escorriam. No fim do primeiro capítulo, 
Mary
\cite{crybaby2015} volta a interagir com sua amiga imaginária, Lena, que pergunta
como foi seu dia na escola. Mary, em resposta, mente, dizendo que tudo
correu bem e pede um momento a sós. Depois disso, Mary "permitiu liberar
tudo aquilo que sentia. Toda aquela dor, raiva, mágoa. Eles agora eram
liberados em rios de lágrimas, que inundavam seus travesseiros pouco a
pouco" \cite{brookeyoongi}. Na seção de comentários, além dos
elogios de vários leitores, BrookeYoongi comenta que a música é triste,
mas reflete realisticamente nossa sociedade. Além disso, a
\emph{songfic} possui "118 Favoritos", ou seja, 118 pessoas adicionaram
a ficção às suas listas, o que atesta a circulação desse gênero. No
segundo capítulo, \emph{Dollhouse}, cuja análise excederia o limite de
páginas deste artigo, percebemos um movimento frequente da fã-autora:
ela faz uma captura de tela do videoclipe (que, por razões de direitos
autorais, não podemos reproduzir aqui), insere o título correspondente à
canção e inclui, abaixo, um trecho do \emph{story book} presente no
encarte do disco. Isso ilustra seu estilo individual, que, por sua vez,
influencia a composição do gênero, que naturalmente dialoga com outras
artes, mídias e códigos. A ocorrência do estilo individual é mais
evidente em gêneros mais flexíveis, como é o caso da \emph{songfic}.
Temos, assim, uma amostra de literatura dinâmica, feita, muitas vezes,
de jovens para jovens, vozeando suas preocupações, desejos e visões de
mundo. Há muitas outras temáticas, para várias outras classificações
indicativas, bastando um olhar atento e aguçado na curadoria dessas
histórias.
