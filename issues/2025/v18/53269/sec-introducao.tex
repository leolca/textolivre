% !TeX root = main.tex
\section{Introdução} \label{sec-introdução}

Com as transformações tecnológicas e a ascensão das novas mídias, outras
formas de enunciação vêm à tona, sobretudo no ciberespaço. Desse modo,
diariamente, lidamos com gêneros discursivos cada vez mais multimodais e
dinâmicos, o que ocasiona, com efeito, novos modos de leitura e de
escrita (e letramento digital). Nesse aspecto, no intenso fluxo das
redes, os internautas publicam textos e interagem entre si
instantaneamente, nutrindo interesses em comum, nos diversos campos da
atividade humana. Cabe ressaltar que, de acordo com \textcite{bakhtin2011}, na composição do discurso, o elemento central é a relação
dialógica estabelecida entre os participantes, ao passo que o fator
sócio-histórico tem ligação com a natureza extra-verbal do enunciado; de
fato, há um forte diálogo nas plataformas virtuais, inclusive no domínio
da literatura infanto-juvenil, haja vista a presença efervescente dos
\emph{fandoms} (reinos de fãs, em tradução livre). Tudo isso ao alcance
de um clique, um toque na tela, no emaranhado de \emph{hiperlinks} por
onde trafegamos virtualmente.

Nesse âmbito, conforme \textcite{xavier2002}, firmamos um novo modo de
enunciação (o digital), ocasionando a produção de gêneros discursivos
desse ambiente, como \emph{posts}, \emph{vlogs}, memes, entre outros.
Esses gêneros, concebidos por \textcite{bakhtin2011} como tipos
relativamente estáveis de enunciados, apresentam características
específicas, como propósito comunicativo, conteúdo temático, estilo,
construção composicional e meio de propagação \cite{bakhtin2011}. Entre as diversas formas
de expressão em voga na internet, as \emph{fanfics}, que podem ser
traduzidas como "ficções de fãs", destacam-se como uma das mais
independentes. Isso ocorre porque elas são resultantes de um processo de
escrita não hierarquizado, caracterizado pela coautoria e pela permuta
de papéis. Esses textos são criados e compartilhados nos \emph{fandoms},
definidos como um sistema digital que engloba diversas formas de
expressão no campo literário. Isso inclui desde a apreciação de
histórias até a análise e a produção de obras artísticas. Essa abordagem
inovadora vai além das atitudes passivas associadas à leitura e à
crítica tradicional e acadêmica, conforme destacado por \textcite{miranda2009}.

Em outros termos, conforme argumentado por \textcite{ribeiro2019}, o
\emph{fandom} oferece aos fãs um ambiente no qual eles podem (co)criar,
transformar e expressar perspectivas únicas com base em um ou mais
elementos de seu interesse, tais como obras literárias, personalidades
públicas, séries, filmes e até mesmo músicas. Como resultado, os fãs,
frequentemente referidos como \emph{ficwriters}, têm a capacidade de
conceber suas próprias histórias, transformando-as em várias formas
genéricas, incluindo as \emph{songfics}, objeto deste artigo. Elas são
elaboradas a partir de canções, de modo que as letras podem ser
incorporadas diretamente na narrativa ou utilizadas de maneira menos
explícita. O termo utilizado para descrever o processo por trás dessas
transformações é conhecido como reelaboração \cite{araujo2009,azevedo2022,zavam2012}; ou seja, trata-se do ato de renovar um gênero
discursivo específico por meio de adaptações e inovações, de modo a
satisfazer as demandas de novos contextos de uso da língua.

\textcite{damiani2016} observa que cada meio de publicação tem suas
particularidades, e que os digitais requerem novas abordagens
interpretativas. Ela argumenta que o meio digital pode ser uma forma
eficaz de promover a leitura nas escolas, dada a sua acessibilidade e o
envolvimento multissensorial proporcionado por recursos como vídeos,
GIFs e áudios, especialmente relevante para a literatura
infanto-juvenil. O hipertexto, segundo \textcite{xavier2002}, é notável pela sua
ubiquidade, permitindo acesso simultâneo de vários usuários. \textcite{jenkins2009} acrescenta que a convergência é um fenômeno onde o conteúdo flui
entre múltiplas plataformas, exemplificado por filmes que geram
discussões e conteúdos em diversas mídias digitais, a destacar a cultura
participativa atual em que os consumidores são ativos na interação,
produção e no diálogo.

É nessa interação discursiva que tomam forma os gêneros discursivos
digitais, como as \emph{songfics}. Para mais, independentemente da forma
como as canções são incorporadas, ocorre o fenômeno do
dialogismo\footnote{Cabe ressaltar que a teoria de Bakhtin e o Círculo
  trata da Translinguística, indo além da Linguística Tradicional e sua
  abstração da língua; todavia, no Brasil, os estudiosos adotam mais
  frequentemente o termo ``dialogismo''.}, conceituado por \textcite{bakhtin2010} como o princípio fundamental da existência humana. Isso se
justifica porque, para viver em sociedade, não é possível conceber o ser
humano sem considerar suas relações dialógicas. Nesse ponto, a
comunicação verbal está intrinsecamente ligada à situação concreta, de
modo que o discurso só adquire seu valor axiológico quando situado em um
contexto sócio-histórico específico. Por meio dele, as palavras
significam --- são mais do que palavras, são marcas axiológicas do
enunciador.

Nesse bojo, as relações dialógicas, fundamentadas na interação eu/outro,
podem ser identificadas nos enunciados, que representam as unidades
reais da comunicação, sendo únicos e irrepetíveis, neles se manifestando
os discursos alheios. De acordo com Bakhtin e seu Círculo, os
interlocutores (não apenas os textos concretos) estão em constante
interação uns com os outros, estabelecendo um diálogo por meio de
enunciados concretos. Todavia, em nossa análise, também incorporamos o
conceito de intertextualidade proposto por \textcite{kristeva2005}, que não deve
ser confundido com o dialogismo, tampouco tomado como sinônimo deste,
uma vez que a intertextualidade se refere exclusivamente às relações
externas entre textos, de forma abstrata, sem levar em consideração o
sujeito histórico, enquanto a teoria do Círculo de Bakhtin explora as
relações dialógicas de maneira mais ampla e contextualizada, decorrentes
da natureza responsiva dos enunciados. Portanto, uma vez que
estabelecemos essas distinções, a teoria da intertextualidade será
aplicada com o propósito específico de identificar outros textos
incorporados nas \emph{songfics}, ao passo que as relações dialógicas,
que são inerentes a todos os enunciados, apontam para uma discussão mais
ampla envolvendo a totalidade das narrativas, os sujeitos envolvidos e o
contexto de produção.

Posto isso, neste contexto, nosso objetivo é analisar as estratégias
intertextuais e hipertextuais, bem como as relações dialógicas presentes
em um exemplar do gênero \emph{songfic}, qual seja: \emph{Cry Baby},
escrita por BrookeYoongi e disponibilizada na plataforma Spirit Fanfics.
A escolha dessa única narrativa ampara-se na extensa dimensão da obra
(47 mil palavras), sendo ela adequada para abordar nosso propósito
dentro dos limites deste texto, a fim de elucidar um tipo de literatura
infanto-juvenil que, por categoria, dialoga com outras artes, mídias e
códigos. Nesse âmbito, a partir das \emph{songfics}¸ pode-se criar um
vínculo entre música e literatura, o que contribui para o estudo das
relações dialógicas, hipertexto e intertextualidade no campo acadêmico
e, também, no ambiente escolar, cenário propício a práticas de leitura e
escrita autônomas e concretas. Sob esse ponto de vista, \textcite{bandoli_por_2015} argumentam que a escola deve propor uma educação linguística que
não se limite ao ensino da norma culta, mas que também considere as
diferentes variações linguísticas e práticas sociolinguísticas.

Para efeitos práticos, este artigo está dividido da seguinte maneira,
além desta \emph{introdução} e das Considerações Finais, há 5 subseções:
2) \emph{Fanfiction e cultura participativa: a manifestação do fandom},
em que alinhamos a teoria bakhtiniana com os tipos relativamente
estáveis de enunciados de que tratamos; 2.1) \emph{Laços intertextuais},
parte na qual deslindamos tal fenômeno, tendo em vista a ressalva
supracitada; 3) \emph{Songfics: o encontro das artes}, em que
fundamentamos os traços gerais e específicos desse gênero, bem como a
sua conexão com as fanfics; 4) \emph{Metodologia}, em que descrevemos os
passos adotados em nossa análise; 5) \emph{Análise e discussão: da
superfície às profundezas do discurso}, em que examinamos a narrativa
\emph{Cry Baby}.