\documentclass[spanish]{textolivre}

% metadata
\journalname{Texto Livre}
\thevolume{18}
%\thenumber{1} % old template
\theyear{2025}
\receiveddate{\DTMdisplaydate{2024}{8}{28}{-1}}
\accepteddate{\DTMdisplaydate{2024}{9}{26}{-1}}
\publisheddate{\DTMdisplaydate{2025}{11}{29}{-1}}
\corrauthor{Blanca Florido Zarazaga}
\articledoi{10.1590/1983-3652.2024.54211}
%\articleid{NNNN} % if the article ID is not the last 5 numbers of its DOI, provide it using \articleid{} commmand 
% list of available sesscions in the journal: articles, dossier, reports, essays, reviews, interviews, editorial
\articlesessionname{articles}
\runningauthor{Florido Zarazaga y Romero Oliva}
%\editorname{Leonardo Araújo} % old template
\sectioneditorname{Daniervelin Pereira}
\layouteditorname{João Mesquita}

\title{Validación de una lista de verificación para el diseño de una Nubeteca}
%\footnote{Este trabajo se enmarca en la tesis doctoral de Blanca Florido Zarazaga Diseño e implementación de un espacio multimodal para la formación de ciudadanos desde el Horizonte 2030 y los libros ilustrados de no ficción Programa de doctorado 8219 Investigación y práctica educativa, Universidad de Cádiz}
\othertitle{Validação de uma lista de verificação para o projeto de uma Nubeteca}
\othertitle{Validation of a checklist for the design of a Nubeteca}

\author[1]{Blanca Florido Zarazaga~\orcid{0000-0003-0865-380X}\thanks{Email: \href{mailto:blanca.florido@uca.es}{blanca.florido@uca.es}}}
\author[1]{Manuel Francisco Romero Oliva~\orcid{0000-0002-6854-0682}\thanks{Email: \href{mailto:manuelfrancisco.romero@uca.es}{manuelfrancisco.romero@uca.es}}}
\affil[1]{Universidad de Cádiz, Facultad de Ciencias de la Educación, Departamento de Didáctica de la Lengua y la Literatura, Cádiz, España.}

\addbibresource{article.bib}

\usepackage{multirow}
\usepackage{array}
\usepackage{longtable}

\begin{document}
\maketitle
\begin{polyabstract}
\begin{abstract}
En las últimas décadas, la transformación digital ha redefinido la gestión y el consumo de información, impulsando la evolución de la lectura y la formación de lectores en un nuevo ecosistema digital. En este contexto, las bibliotecas, como guardianas del legado cultural de la lectura, deben adaptarse para seguir siendo relevantes, lo que ha dado lugar a la creación de "Nubetecas", espacios que integran lo físico y lo digital. Por ello, el objetivo de este estudio es desarrollar y validar una lista de verificación para evaluar los elementos clave que deben caracterizar este espacio multimodal, enfocada en tres dimensiones: organización espacial, equipamientos tecnológicos y política de gestión bibliotecaria. Nuestra investigación emplea una metodología mixta y se basa en la validación de una lista de verificación mediante una escala valorativa compuesta por un total de 33 ítems, en la que participaron siete expertos seleccionados por sus conocimientos especializados en el tema. Los resultados revelan que las aportaciones de los expertos condujeron a ajustes significativos en varios ítems, mejorando así la precisión, relevancia y efectividad del instrumento en consonancia con las necesidades actuales. Sin embargo, el libro impreso sigue siendo crucial para desarrollar una lectura en profundidad y habilidades de reflexión y concentración.

\keywords{Bibliotecas \sep Nubetecas \sep Espacio multimodal \sep Lista de verificación \sep Promoción de la lectura}
\end{abstract}

\begin{portuguese}
\begin{abstract}
Nas últimas décadas, a transformação digital redefiniu a gestão e o consumo de informação, impulsionando a evolução da leitura e a formação de leitores em um novo ecossistema digital. Neste contexto, as bibliotecas, como guardiãs do legado cultural da leitura, devem se adaptar para seguir se forem relevantes, o que deu lugar à criação de "Nubetecas", espaços que integram o físico e o digital. Por isso, o objetivo deste estudo é desenvolver e validar uma lista de verificação para avaliar os elementos chave que devem caracterizar esse espaço multimodal, enfocado em três dimensões: organização espacial, biblioteca, equipamentos tecnológicos e política de gestão. Nossa investigação emprega uma metodologia mista e se baseia na validação de uma lista de verificação por meio de uma escala valorativa composta por um total de 33 itens, na qual participam seus sete especialistas selecionados por seus conhecimentos especializados no tema. Os resultados revelam que as transferências dos especialistas conduziram a ajustes significativos em vários itens, melhorando também a precisão, relevância e eficácia do instrumento em consonância com as necessidades atuais. No entanto, a impressão do livro ainda é crucial para desenvolver uma leitura em profundidade e habilidades de reflexão e concentração.

\keywords{Bibliotecas \sep Nubetecas \sep Espaço multimodal \sep Lista de verificação \sep Promoção da leitura}
\end{abstract}
\end{portuguese}

\begin{english}
\begin{abstract}
In recent decades, digital transformation has redefined the management and consumption of information, boosting the evolution of reading and the training of readers in a new digital ecosystem. In this context, libraries, as guardians of the cultural legacy of reading, must adapt to continue being relevant, which has given way to the creation of "Nubetecas", spaces that integrate the physical and the digital. Therefore, the objective of this study is to develop and validate a checklist to evaluate the key elements that must characterize this multimodal space, focused on three dimensions: spatial organization, technological equipment and library management policy. Our investigation employs a mixed methodology and is based on the validation of a checklist using a valuation scale comprising a total of 33 items, in which seven experts selected by their specialized knowledge on the topic participated. The results reveal that the contributions of experts led to significant adjustments in several items, improving the precision, relevance and effectiveness of the instrument in line with current needs. However, the printed book remains crucial for developing in-depth reading and reflection and concentration skills.

\keywords{Libraries \sep Nubetecas \sep Multimodal space \sep Checklist \sep Promotion of reading}
\end{abstract}
\end{english}
\end{polyabstract}

\section{Introducción}
En las últimas décadas, hemos presenciado una transformación sin precedentes en la manera en que la información es generada, distribuida y consumida, dando lugar a lo que se conoce como sociedad de la información \cite{bell_coming_1973}, en donde el conocimiento teórico se convierte en referencia para los medios y servicios que deberán gestionarla hacia una ciudadanía crítica.  Este fenómeno no solo ha revolucionado los métodos tradicionales de gestión y comunicación de datos, sino que también ha desencadenado un profundo cambio en la forma en que interactuamos con el conocimiento. Nos encontramos, por tanto, en un momento de transición significativo entre modelos culturales diversos que incluyen términos como posinternet, posmodernidad y posverdad \cite{tabernero_formar_2022a}; conceptos que reflejan un cambio de paradigma en el que las antiguas formas de entender y manejar la información están en proceso de redefinición, dejando un horizonte abierto a nuevas configuraciones y prácticas culturales.

Este momento de transición se enmarca dentro de lo que \textcite{bauman_modernidad_2004,bauman_acerca_2019} denomina modernidad líquida. Este término captura la esencia de una época en la que la inestabilidad, la incertidumbre y la ausencia de certezas se han convertido en rasgos definitorios de este período de cambio. Consideramos más acertado representar esta realidad contemporánea con la metáfora de sociedad gaseosa propuesta por \textcite{scolari_cultura_2021}, por su relación con los medios, el consumo evanescente de contenidos, la brevedad y fragmentación de textos mediáticos, lo viral y la cultura snack en la que nos movemos. \textcite[p. 157]{scolari_cultura_2021} analiza las producciones mediáticas actuales a partir de diez conceptos claves que definen este entorno: “brevedad, miniaturización, fugacidad, fragmentación, viralidad, remixabilidad, infoxicación, movilidad, aceleración y afterpost”.

En este panorama actual, los nuevos soportes y formatos digitales han afectado a la sociedad, en general, y a los contextos educativos, en particular, transformando tanto la manera en que se transmiten los contenidos, como la forma en que son recepcionados \cite{heredia_tecnicas_2019,tabernero_habitos_2020}. Esto ha propiciado nuevos entornos de aprendizaje que exigen una actualización constante en la formación de docentes, en activo y en formación, en el buen uso de las Tecnologías del Aprendizaje y del Conocimiento —en adelante, TAC— \cite{alvarez_tac_2017} y, en especial, en una alfabetización mediática e informacional —en adelante, AMI— \cite{santiago_del_pino_consulta_2019}. Así mismo, para que esta formación resulte efectiva, debe estar situada, es decir, ser real y estar bien contextualizada, integrando metodologías y recursos adecuados a los tiempos actuales, con experiencias y prácticas concretas en el aula que promuevan una reflexión crítica sobre el contexto social y educativo en el que se lleva a cabo la enseñanza \cite{imbernon_ser_2017,castaneda_quintero_formacion_2019,rovira-collado_intertextualidad_2021}.

Concretamente, en el contexto de la formación de lectores y de la promoción de la lectura, nos adentramos en un nuevo ecosistema lector, dado que el desarrollo tecnológico ha dado lugar a nuevas etapas, modos de leer y preferencias lectoras \cite{tabernero_habitos_2020}. Este escenario ha conllevado no solo una modificación en la formación de hábitos lectores \cite{tabernero_leer_2022b}, sino que también redefine “la concepción del libro en todos sus niveles, desde el concepto de autoría hasta su materialidad, proponiendo una nueva experiencia de lectura” \cite[p. 74]{samperiz_libro_2020}. En este sentido, la integración de textos multimodales —es que combinan elementos visuales y textuales y que abarcan desde contenidos literarios hasta informativos— ese vuelve crucial.  Así se refleja en las Instrucciones de 21 de junio de 2023, que subrayan la importancia de incorporar diversos tipos de textos —continuos y discontinuos (multimodales)—, para preparar al alumnado para manejar eficazmente las nuevas textualidades de la era digital.

Si bien los adolescentes actuales muestran destreza técnica en el manejo de herramientas y dispositivos tecnológicos, diversos estudios \cite{cruces_como_2017,kovac_lectura_2020} advierten que, aunque los formatos digitales han facilitado el acceso a gran cantidad de contenidos con un solo clic, tienden a promover una lectura en barrido —propia de las redes sociales—, evidenciada en prácticas como la captura de pantalla y el consumo fragmentado de información. En contraposición, la lectura en papel se asocia con una comprensión más profunda de los textos y una mejor memorización, ya que el formato impreso favorece una mayor concentración y reflexión, facilitando el desarrollo del pensamiento crítico y analítico y contribuyendo a una retención más efectiva de la información. Así, \textcite{cordon-garcia_combates_2018,stole_mito_2020} señalan que la competencia entre ambos soportes ha llevado al sector impreso a enfatizar las cualidades físicas del libro, como su diseño y materialidad. Por tanto, “los dispositivos y la lectura digital, lejos de provocar la suplantación o desplazamiento del entorno impreso, lo han reforzado. Se está produciendo una retroalimentación de dos sistemas que [...] han explorado sus rasgos diferenciales para someterlos a una optimización inexistente anteriormente \cite[p. 477]{cordon-garcia_combates_2018}.

A este respecto, especialistas en la adquisición de la lectura y el aprendizaje de lenguas, como \textcite[p. 204]{wolf_lector_2020}, propone el desarrollo de un "cerebro lector bialfabetizado", que implica comenzar con la alfabetización en formato impreso y, a partir de los cinco años, introducir de forma gradual los medios digitales. \textcite{wolf_lector_2020} sugiere que, en un primer momento, los niños deben familiarizarse por separado con cada tipo de soporte para luego integrarlos progresivamente, permitiendo que desarrollen habilidades en ambos formatos y adaptarse a las características específicas de cada uno. En un entorno marcado por la infoxicación y la saturación de información, es esencial desarrollar una sólida competencia informacional y digital desde una edad temprana, que permita a los lectores seleccionar, filtrar y organizar la información para transformarla en conocimiento útil, abarcando tres ámbitos fundamentales que deben ser trabajados a lo largo de la vida en todos los programas/proyectos educativos \cite[p. 65-66]{area_alfabetizacion_2012}.

\begin{itemize}
    \item Adquisición y comprensión de la información: aprender a buscar, localizar y comprender la información empleando todos los tipos de recursos y herramientas (libros, ordenadores, Internet, tabletas, entre otros dispositivos).
    \item Expresión y difusión de la información: aprender a expresarse mediante distintos tipos de lenguajes, formas simbólicas y tecnológicas y, en consecuencia, saber difundir públicamente las ideas propias sea mediante presentaciones multimedia, blogs, wikis o cualquier otro recurso digital.
    \item Comunicación e intercambio social de información: aprender a comunicarse e interaccionar socialmente con otras personas a través de los recursos de la red (e-mail, foros, redes sociales, videoconferencias, entre otros).
\end{itemize}

Partiendo de este ámbito de comunicación e intercambio social de información, es evidente que la lectura ha adquirido un carácter eminentemente social y colaborativo, impulsado por las TIC y las redes sociales. Estas plataformas han logrado crear espacios de comunicación que combinan relación, entretenimiento y conversación, donde blogs, podcasts y vídeos no solo presentan novedades y establecen diálogos con lectores, sino que también fomentan la creación, el intercambio y la colaboración, pilares fundamentales de la cultura 2.0 \cite{alonso_lectura_2014,ravettino_booktubers_2015,cruces_como_2017,alvarez_epitextos_2018}. Este enfoque permite a los lectores compartir ideas, conectar con otros aficionados y formar nuevas amistades, mientras se convierten en promotores globales de la lectura, también considerados como mediadores no institucionales \cite{romero_oliva_habitos_2020}, quienes, a su vez, amplifican su influencia a través de comunidades virtuales que redefinen las dinámicas tradicionales del consumo literario.

Asimismo, estos medios sociales han propiciado la aparición de epitextos virtuales públicos —booktráilers, reseñas y blogs—, que sirven tanto para promocionar libros como para crear sentido y orientar al lector sobre el contenido del texto, despertando su curiosidad y motivándolos a profundizar \cite{lluch_epitextos_2015,romero_oliva_epitextos_2021,tabernero-sala_promocion_2022}.  La importancia de estos epitextos radica en su capacidad para transformar la promoción literaria en un discurso transmedia, donde la interacción entre texto e imagen en diferentes plataformas digitales enriquece la comprensión del contenido y motiva al lector a adentrarse en el mundo del autor y su obra. Estas estrategias hipervinculares han llevado al mercado editorial a adaptarse a las demandas actuales, diversificando su oferta con obras como los libros ilustrados de no ficción, conocidos como libros informacionales \cite{duke_case_2004}, que se han convertido en una respuesta a esta tendencia de promoción y exploración digital.

Sin embargo, la inserción de la experiencia digital y la multimodalidad en la vida diaria ha marcado el inicio de una nueva era, que no solo afecta a la lectura y al lector, sino también a los espacios de promoción del libro, como es el caso de las bibliotecas, dentro de los contextos no formales. Estas, como instituciones de referencia en la conservación y difusión del conocimiento, se convierten en pilares de referencia en la enseñanza, la información y la cultura, proporcionando un legado cultural accesible y diverso para toda la ciudadanía en su conjunto \cite{yarrow_bibliotecas_2009,romero_oliva_entre_2023}. Su misión como referentes culturales es incuestionable, pero en un contexto globalizado y tecnológico deben reinventarse para seguir siendo relevantes. En este sentido, es crucial que se adapten a las necesidades y exigencias de una ciudadanía que espera respuestas rápidas, acceso inmediato y una experiencia más interactiva y personalizada, incorporando y actualizando su oferta con nuevos lenguajes, soportes y formatos.

En la actualidad, con la posibilidad de buscar y acceder a contenidos de forma instantánea desde casi cualquier dispositivo móvil y en cualquier lugar, ya sea en la escuela, el hogar, el trabajo o incluso en el transporte público, las bibliotecas se enfrentan a un reto significativo. Este cambio en los hábitos de consumo de información y el auge de la lectura digital, ha disminuido la necesidad de que los usuarios visiten los espacios físicos de las bibliotecas, cuestionando así la viabilidad de su modelo tradicional \cite{dantas_adaptabilidad_2017}. En respuesta a esta realidad, es fundamental concebir un nuevo modelo de biblioteca que integre lo mejor de ambos mundos: lo analógico y lo digital. Este modelo debe preservar la esencia de la biblioteca tradicional como un espacio de encuentro presencial y, a la vez, aprovechar las oportunidades que brinda la accesibilidad digital para dinamizar su uso y atraer a un público más amplio. Es así como surge el proyecto enfocado en la creación de espacios Nubeteca.

Estos espacios, auspiciados por la Fundación Germán Sánchez Ruipérez y la Diputación de Badajoz, representan un modelo de integración entre lo físico y lo digital, convirtiendo a las bibliotecas en un punto de conexión entre ambos ecosistemas —\Cref{tbl1}—. Según \textcite{valbuena_impacto_2019}, es la denominación de una nueva concepción de la biblioteca basada en la investigación sobre la lectura digital y el libro electrónico. En este marco, una Nubeteca permite poner  a disposición de los usuarios una amplia gama de servicios —préstamo de materiales, clubes de lectura, autopublicación, formación, gestión de proyectos, difusión cultural, actividades de dinamización y generación de informes— que, además de facilitar el acceso a recursos, crean entornos amigables que fomentan la interacción y el diálogo entre lectores, bibliotecarios, autores y otros actores del ámbito cultural, priorizando la conversación y la conexión social sobre la mera acumulación de fondos bibliográficos \cite{moreno_mulas_florencia_2019}.

\begin{table}[htbp]
\footnotesize
\centering
\begin{threeparttable}
\caption{Características de la biblioteca tradicional y Nubeteca.}
\label{tbl1}
\centering
\begin{tabular}{p{0.5\textwidth} p{0.5\textwidth}}
\toprule
Biblioteca tradicional & Nubeteca \\ 
\midrule
\multicolumn{2}{c}{Espacio físico} \\
\midrule
Áreas específicas dedicadas a la lectura, consulta y eventos, principalmente orientadas al uso de recursos físicos como libros y revistas. Las zonas están diseñadas para la interacción directa con estos materiales. & Espacios diseñados para la integración de lo digital con lo físico, incorporando zonas con acceso a dispositivos electrónicos y recursos digitales que facilitan el acceso a una amplia gama de contenidos digitales y servicios en línea. \\ 
\midrule
\multicolumn{2}{c}{Equipamientos} \\
\midrule
El equipamiento se centra en estanterías con libros impresos, áreas de lectura, ordenadores para consultas básicas y espacios dedicados a actividades presenciales. & Estos espacios están equipados con una variedad de dispositivos digitales como tabletas, ordenadores, y pantallas interactivas, así como herramientas para acceder a catálogos digitales, realizar préstamos electrónicos, y consultar recursos a través de códigos QR, integrando lo analógico con lo digital. \\
\midrule
\multicolumn{2}{c}{Alcance y disponibilidad de recursos} \\
\midrule
El acceso a los recursos está limitado por la colección física disponible en la biblioteca. En algunos casos, se pueden gestionar intercambios de materiales con otras bibliotecas, pero esto depende de la red de bibliotecas y sus políticas. & Mayor flexibilidad con acceso a recursos digitales desde cualquier dispositivo conectado, más allá del horario físico de la biblioteca. \\
\midrule
\multicolumn{2}{c}{Acceso al catálogo} \\
\midrule
Los usuarios consultan el catálogo en formato físico o en línea, generalmente mediante terminales fijos (PC) dentro de la biblioteca. & Acceso al catálogo digital mediante dispositivos electrónicos como ordenadores, tabletas y códigos QR, en el espacio Nubeteca. \\
\midrule
\multicolumn{2}{c}{Préstamos} \\
\midrule
El préstamo de libros físicos es gestionado manualmente o a través de sistemas informáticos básicos, con un enfoque en la interacción personal durante el proceso de préstamo y devolución. & Préstamo de libros electrónicos a través de dispositivos como ordenadores, tabletas y códigos QR, dentro del espacio Nubeteca.
Los usuarios pueden descargar libros digitales directamente a sus dispositivos sin necesidad de interacción física. \\
\midrule
\multicolumn{2}{c}{Formación} \\
\midrule
La formación se centra en la orientación proporcionada por los bibliotecarios sobre el uso de los recursos físicos. Esto incluye cómo buscar en el catálogo, utilizar las instalaciones, y aprovechar al máximo la colección física. & Ofrece una formación digital (ITV digital dirigida tanto por bibliotecarios como por pares, enfocada en la configuración y uso de dispositivos electrónicos, la navegación por catálogos digitales, y la gestión de préstamos electrónicos. \\
\midrule
\multicolumn{2}{c}{Interacción usuario-biblioteca} \\
\midrule
La interacción se realiza principalmente de manera presencial, donde los usuarios se relacionan directamente con los bibliotecarios para buscar asistencia, realizar préstamos y participar en actividades. & En la Nubeteca, la interacción entre el usuario y la biblioteca es más flexible y digital. Los usuarios pueden gestionar préstamos, acceder al catálogo, y participar en actividades tanto de forma presencial como virtual. \\
\midrule
\multicolumn{2}{c}{Recomendaciones de lecturas} \\
\midrule
Las recomendaciones son ofrecidas principalmente por los bibliotecarios en persona o a través de listas impresas y catálogos físicos, facilitando la selección de lecturas tradicionales. & Las recomendaciones de lecturas se ofrecen a través de monitores y dispositivos electrónicos. Los usuarios pueden recibir sugerencias personalizadas y explorar opciones digitales mediante ordenadores de sobremesa, tablets y otros dispositivos conectados. \\
\midrule
\multicolumn{2}{c}{Promoción de la lectura} \\
\midrule
La promoción de la lectura se lleva a cabo principalmente a través de clubes de lectura presenciales, charlas con autores y eventos organizados en el espacio físico de la biblioteca. & La promoción de la lectura se expande más allá de lo presencial, incorporando la organización de clubes de lectura nubetecos, presentaciones de libros, talleres interactivos  y otros eventos que combinan lo digital con lo presencial, aprovechando las herramientas tecnológicas disponibles en el espacio Nubeteca. \\
\bottomrule
\end{tabular}
\source{Elaboración propia. Adaptación de \textcite{ifla_directrices_2001,valbuena_nubeteca:_2014,dantas_adaptabilidad_2017,cordon-garcia_mpacto_2020}.}
\end{threeparttable}
\end{table}

Por ello, el objetivo principal del presente estudio se centra en sistematizar los aspectos que deberían contemplarse en el diseño de uma biblioteca multimodal o Nubeteca mediante la elaboración de una lista de verificación. De esta forma, se estaria contribuyendo a comprobar los elementos que deben conformar un espacio multimodal, como es el caso de una Nubeteca, para la promoción de la lectura y la información entre la ciudadanía. En este sentido, el instrumento objeto de validación, que puede considerarse de utilidad para los mediadores y promoción de la lectura, se ha  diseñado a partir de  tres dimensiones de estudio:

\begin{itemize}
    \item Dimensión 1: Organización espacial, donde se analizan la ubicación y la visibilidad del espacio multimodal. El objetivo es examinar la ubicación de los elementos y la visibilidad del espacio multimodal.
    \item Dimensión 2: Equipamientos, donde se inspeccionan los recursos tecnológicos del espacio multimodal. El objetivo se centra en indagar en los recursos tecnológicos del espacio multimodal.
    \item Dimensión 3: Política de gestión de la biblioteca, donde se indaga la política de promoción de lectura y la información referida a la gestión de los catálogos y los servicios públicos con los usuarios y la sociedad. El objetivo contempla situar cuál es la política de promoción de la lectura y la información referida a la gestión de los catálogos y los servicios públicos con los usuarios y la sociedad.
\end{itemize}

\section{Metodología}\label{sec-normas}
El proceso de validación de la lista de verificación \cite{lewis_tests_2003} presenta un carácter mixto ya que confluyen aspectos cualitativos y cuantitativos. Previamente, se acotó el perfil de los expertos que serían consultados en relación con el objeto de estudio: la Nubeteca. Se tuvo en cuenta el sexo, la universidad y la formación que tiene cada uno de los jueces —\Cref{tbl2}—:

\begin{table}[htbp]
\centering
\begin{threeparttable}
\caption{Sexo, universidad y formación de los jueces.}
\label{tbl2}
\centering
\begin{tabular}{llll}
\toprule
Jueces & Sexo & Universidad & Formación \\ \midrule
1 & Hombre & Universidad de Cádiz & Doctor \\
2 & Mujer & Universidad de Valladolid & Doctora \\
3 & Mujer & Universidad de Valladolid & Doctora \\
4 & Hombre & Universidad de Málaga & Doctor \\
5 & Mujer & Universidad de Zaragoza & Doctora \\
6 & Hombre & Universidad de Alicante & Doctor \\
7 & Mujer & Universidad de Zaragoza & Doctora \\
\bottomrule
\end{tabular}
\source{Elaboración propia.}
\end{threeparttable}
\end{table}

Para llevar a cabo este proceso de validación, los jueces debieron tener unos conocimientos acerca de la Nubeteca. Para ello se realizó el coeficiente de competencia (K) —\Cref{tbl3}— utilizado por \textcite{almenara_utilizacion_2013}, que se calcula de la siguiente forma: K= 0.5 [kc(0.10) +Ka]. Como se puede observar en la formula se debe calcular por una parte  el coeficiente de conocimiento (Kc) en una escala del 1 al 10 y, por otra, el coeficiente de argumentación (Ka) a partir de los ítems y valores de la \Cref{tbl3}:

% Falta Cabero y Barroso (2013)

\begin{table}[htbp]
\centering
\begin{threeparttable}
\caption{Fuentes de argumentación de los jueces.}
\label{tbl3}
\centering
\begin{tabular}{llll}
\toprule
Fuentes de argumentación & Alto & Medio & Bajo \\ 
\midrule
Conocimientos Nubeteca & 0.5 & 0.4 & 0.2  \\
Conocimientos sobre la  promoción de la lectura & 0.3 & 0.2 & 0.1 \\
Conocimientos dinamización de colecciones digitales & 0.05 & 0.05 & 0.05 \\
Conocimientos herramientas digitales relacionadas con la lectura & 0.05 & 0.05 & 0.05 \\
Conocimientos sobre proyectos difusión de las colecciones & 0.05 & 0.05 & 0.05 \\
Conocimientos programas de difusión de las colecciones & 0.05 & 0.05 & 0.05 \\
\bottomrule
\end{tabular}
\source{Elaboración propia.}
\end{threeparttable}
\end{table}

La lista de verificación estaba compuesta de una escala valorativa con 33 ítems de tipo de Likert del 1 al 5 (1: Valoración nula, 2: Valoración deficiente, 3: Valoración mejorable, 4: Valoración adecuada y 5: Valoración máxima) diferenciados en tres dimensiones y sus respectivas subdimensiones, atendiendo a las características propias de una Nubeteca ——\Cref{tbl1}—

Para la manipulación de los datos se utilizó el programa estadístico Statistical Package for the Social Sciences (SPSS), versión 24. El procedimiento empleado se corresponde con el propuesto por \textcite{agreda_diseno_2016}:

\begin{description}
    \item[Fase 1.] Describir las fase de las dimensiones, es decir, se determinaron las diferentes dimensiones con sus respectivos ítems que van a crear para realizar el análisis —\Cref{tbl4}—:
\begin{table}[h]
\centering
\begin{threeparttable}
\caption{Dimensiones con el número de ítems.}
\label{tbl4}
\centering
\begin{tabular}{ll}
\toprule
Dimensiones & ítems \\ 
\midrule
1.Organización espacial & 4 \\
2.Equipamientos & 4 \\
3.Política de gestión de la biblioteca & 4 \\
3.1.Catálogo & 4 \\
3.2.Préstamos & 4 \\
3.3. Promoción de la lectura & 4 \\
3.4. Integración en las comunidades universitarias & 4 \\
3.5. Transferencia social en la universidad & 4 \\
\bottomrule
\end{tabular}
\source{Elaboración propia.}
\end{threeparttable}
\end{table}

    \item[Fases 2.] Redactar los ítems de cada dimensión
    \item[Fase 3.] Validar el instrumento. Aquí hay dos momentos:
        \subitem Delimitar el coeficiente de competencia de cada uno de los jueces. 
        \subitem Realizar las pruebas estadísticas pertinentes: valores descriptivos y  alfa de Cronbach.
    \item[Fase 4.] Modificar y crear la escala valorativa
\end{description}

\section{Resultados}\label{sec-conduta}
Los resultados se comentarán atendiendo a las fases establecidas en el punto anterior. Por lo tanto, el primer aspecto fue analizar el coeficiente de conocimiento (K) de cada uno de los jueces —\Cref{tbl5}—:

\begin{table}[htbp]
\centering
\begin{threeparttable}
\caption{Coeficiente de conocimiento de los jueces.}
\label{tbl5}
\centering
\begin{tabular}{llll}
\toprule
Jueces & kc & Ka & K \\ 
\midrule
1 & 0.8 & 0.8 & 0.8 \\
2 & 0.9 & 1 & 0.95 \\
3 & 1 & 0.9 & 0.95 \\
4 & 1 & 0.8 & 0.9 \\
5 & 0.8 & 0.9 & 0.85 \\
6 & 1 & 0.9 & 0.95 \\
7 & 0.8 & 0.9 & 0.85 \\
\bottomrule
\end{tabular}
\source{Elaboración propia.}
\end{threeparttable}
\end{table}

En la \Cref{tbl5} se puede observar que el K tiene un valor en la mayoría de los casos a 0.8, por lo tanto, se podría considerar que es alto. De esta forma, se conviene que todos los jueces tienen autoridad para realizar el proceso de validación, viendo, además, que  jueces, como el 2 y el 3, presentan un valor muy cercano a 1.

Tras verificar la idoneidad de los jueces, correspondió pasar a analizar en los resultados de valoración del propio instrumento.

\begin{table}[htbp]
\centering
\begin{threeparttable}
\caption{Datos descriptivos de las dimensiones.}
\label{tbl6}
\centering
\begin{tabular}{llllll}
\toprule
 & N & Mínimo & Máximo & Media & Desv. estándar \\ 
\midrule
D1 & 7 & 5.00 & 5.00 & 5.0000 & .00000 \\
D2 & 7 & 5.00 & 5.00 & 5.0000 & .00000 \\
D3 & 7 & 3.96 & 5.00 & 4.7802 & .39891 \\
N válido (por lista) & 7 & & & & \\
\bottomrule
\end{tabular}
\source{Elaboración propia.}
\end{threeparttable}
\end{table}

En relación con las dimensiones de análisis —\Cref{tbl6}—, observamos que la 1 y 2 tienen una media de 5 y la tercera está por encima del 4. Por lo tanto, apreciamos que son valores aceptables.

Adentrándonos en las subdimensiones de la 3 —\Cref{tbl7}—, todas están por encima del 4, siendo la 3.2. la que ofrece el valor más bajo (3.2857) con respecto a las otras. No obstante, todas son aceptables. Destaca, además, la 3.6 tiene una media de 5.0.

\begin{table}[htbp]
\centering
\begin{threeparttable}
\caption{Datos descriptivos de las subdimensiones de la dimensión 3.}
\label{tbl7}
\centering
\begin{tabular}{llllll}
\toprule
 & N & Mínimo & Máximo & Media & Desv. estándar \\ 
\midrule
D3.1 & 7 & 4.75 & 5.00 & 4.9286 & .12199 \\
D3.2 & 7 & 1.00 & 5.00 & 4.2857 & 1.49603 \\
D3.3 & 7 & 4.00 & 5.00 & 4.7143 & .48795 \\
D3.4 & 7 & 3.50 & 5.00 & 4.7857 & .56695 \\
D3.5 & 7 & 4.25 & 5.00 & 4.8929 & .28347 \\
D3.6 & 7 & 5.00 & 5.00 & 5.0000 & .00000 \\
N válido (por lista) & 7 & & & & \\
\bottomrule
\end{tabular}
\source{Elaboración propia.}
\end{threeparttable}
\end{table}

Tras ver que todas las dimensiones y subdimensiones tienen una media superior al 4 y muy cercas del 5, se procedió a analizar el índice de fiabilidad de las dimensiones a través del Alfa de Cronbach que es .912. El estudio de los datos muestra un valor del Alfa de Cronbach de .912  y, por lo tanto, según \textcite{george_spss_2003}, es excelente porque K>0.9. También se decidió investigar la incidencia de eliminar una dimensión en el Alfa de Cronbach subiría o no —\Cref{tbl8}—:

\begin{table}[htbp]
\centering
\begin{threeparttable}
\begin{small}
\caption{Alfa de Cronbach si se elimina una dimensión.}
\label{tbl8}
\centering
\begin{tabular}{lllll}
\toprule
& \multicolumn{1}{>{\raggedright\arraybackslash}p{2.5cm}}{Media de escala si el elemento se ha suprimido} & \multicolumn{1}{>{\raggedright\arraybackslash}p{2.5cm}}{Varianza de escala si el elemento se ha suprimido} & \multicolumn{1}{>{\raggedright\arraybackslash}p{2.5cm}}{Correlación total de elementos corregida} & \multicolumn{1}{>{\raggedright\arraybackslash}p{2.5cm}}{Alfa de Cronbach si el elemento se ha suprimido} \\ 
\midrule
D1 & 9.5659 & .627 & .753 & .999 \\
D2 & 9.6731 & .319 & .959 & .760 \\
D3 & 9.6786 & .307 & .984 & .738 \\
\bottomrule
\end{tabular}
\source{Elaboración propia.}
\end{small}
\end{threeparttable}
\end{table}

En este caso, si se descartara la dimensión 1, subiría 0.87, pero al estar dentro de los valores excelente, se decidió mantenerla. Analizando las subdimensiones, también nos sale un Alfa de Cronbach excelente, pues el coeficientes de .900.

A partir de estudiar las diferentes dimensiones, nos adentramos en los diferentes ítems  y en este sentido, el Alfa de Cronbach es de .932, es decir, excelente. Al igual que ocurrió en las dimensiones, queremos saber si al eliminar un ítems, el coeficiente sube considerablemente, pero en este caso no lo es —\Cref{tbl9}—:

\begin{table}[htbp]
\centering
\begin{threeparttable}
\begin{small}
\caption{Alfa de Cronbach si se elimina una dimensión.}
\label{tbl9}
\centering
\begin{tabular}{lllll}%{p{1cm} p{2,5cm} p{2,5cm} p{2,5cm} p{2,5cm}}
\toprule
& \multicolumn{1}{>{\raggedright\arraybackslash}p{2.5cm}}{Media de escala si el elemento se ha suprimido} & \multicolumn{1}{>{\raggedright\arraybackslash}p{2.5cm}}{Varianza de escala si el elemento se ha suprimido} & \multicolumn{1}{>{\raggedright\arraybackslash}p{2.5cm}}{Correlación total de elementos corregida} & \multicolumn{1}{>{\raggedright\arraybackslash}p{2.5cm}}{Alfa de Cronbach si el elemento se ha suprimido} \\ 
\midrule
I1 & 153.00 & 157.667 & .000 & .933 \\
I2 & 153.14 & 159.476 & -.205 & .935 \\
I3 & 153.14 & 149.143 & .908 & .928 \\
I4 & 153.14 & 149.143 & .908 & .928 \\
I5 & 153.29 & 153.571 & .188 & .934 \\
I6 & 153.29 & 140.905 & .902 & .926 \\
I7 & 153.14 & 149.143 & .908 & .928 \\
I8 & 153.14 & 149.143 & .908 & .928 \\
I9 & 153.00 & 157.667 & .000 & .933 \\
I10 & 153.00 & 157.667 & .000 & .933 \\
I11 & 153.29 & 150.905 & .544 & .930 \\
I12 & 153.00 & 157.667 & .000 & .933 \\
I13 & 153.71 & 122.905 & .981 & .924 \\
I14 & 153.71 & 122.905 & .981 & .924 \\
I15 & 153.71 & 122.905 & .981 & .924 \\
I16 & 153.71 & 122.905 & .981 & .924 \\
I17 & 153.29 & 146.905 & .890 & .928 \\
I18 & 153.29 & 146.905 & .890 & .928 \\
I19 & 153.29 & 146.905 & .890 & .928 \\
I20 & 153.29 & 146.905 & .890 & .928 \\
I21 & 153.00 & 157.667 & .000 & .933 \\
I22 & 153.43 & 132.952 & .896 & .925 \\
I23 & 153.43 & 132.952 & .896 & .925 \\
I24 & 153.00 & 157.667 & .000 & .933 \\
I25 & 153.14 & 155.476 & .217 & .933 \\
I26 & 153.14 & 155.476 & .217 & .933 \\
I27 & 153.14 & 155.476 & .217 & .933 \\
I28 & 153.14 & 155.476 & .217 & .933 \\
I29 & 153.00 & 157.667 & .000 & .933 \\
I30 & 153.00 & 157,667 & .000 & .933 \\
I31 & 153.00 & 157.667 & .000 & .933 \\
I32 & 153.00 & 157.667 & .000 & .933 \\
I33 & 153.00 & 157.667 & .000 & .933 \\
\bottomrule
\end{tabular}
\source{Elaboración propia.}
\end{small}
\end{threeparttable}
\end{table}

Tras los datos cuantitativos, se procedió a la interpretación y valoración de los comentarios que realizan los diferentes jueces —\Cref{tb10}—:

\begin{table}[htbp]
\footnotesize
\centering
\begin{threeparttable}
\caption{Análisis cualitativo a partir de los comentarios de los jueces.}
\label{tb10}
\centering
\begin{tabular}{ll p{5cm} p{5cm}}
\toprule
Dimensión & Ítems & Antes & Ahora \\ 
\midrule
1 & 1 & La Nubeteca se encuentra en un lugar claramente identificable por los usuarios de la biblioteca & La Nubeteca se encuentra en un lugar claramente identificable por los usuarios y las usuarias de la biblioteca \\
2 & 5 & Los recursos digitales están integrados como un monitor Smart TV & 
Los recursos digitales están integrados como un monitor Smart TV o también un monitor de ordenador \\
2 & 6 & Los recursos digitales están integrados como un iPad & Los recursos digitales están integrados como un tabletas digitales \\
2 & 7 & Los recursos digitales están integrados como un ordenador táctil de sobremesa & Los recursos digitales están integrados como un ordenador táctil de sobremesa o con teclados \\
2 & 8 & Apuesta por la integración de otros recursos digitales & Apuesta por la integración de otros recursos digitales  como por ejemplo gafas de realidad aumentada \\
3 & 9 & Se lleva a cabo una formación entre bibliotecarios y usuarios dirigidas a la lectura digital (tutoriales sobre dispositivo, plataformas de lectura, la socialización de la lectura…) & Se lleva a cabo de bibliotecarios y bibliotecarias  con usuarios dirigidas a la lectura digital (tutoriales sobre dispositivo, plataformas de lectura, la socialización de la lectura…) \\
3 & 12 & Se promueven otras actividades de formación de lectura con PDI/PTGAS
Se promueven otras actividades de formación de lectura con Personal Docente &  Investigador (PDI)/Personal Técnico, de Gestión y de Administración y Servicios (PTGAS) \\
3 & 17 & Posibilidad de realizar préstamos a través del ordenador & 
Existe posibilidad de realizar préstamos a través del ordenador \\
3 & 18 & Posibilidad de realizar préstamos a través de la tableta & Existe posibilidad de realizar préstamos a través de la tableta \\
3 & 19 & Posibilidad de realizar préstamos a través de los códigos QR & 
Existe posibilidad  de realizar préstamos a través de los códigos QR \\
3 & 20 & Posibilidad de realizar préstamos a través de otros dispositivos electrónicos & Existe posibilidad de realizar préstamos a través de otros dispositivos electrónicos \\
3 & 22 & La existencia de acompañamiento de lectura por medio de clubes de lectura está visible para los usuarios & \multirow{2}{=}{La existencia de clubes, talleres y otro tipo de actividades para la promoción de la lectura} \\
3 & 23 & La existencia de actividades de acompañamiento de lectura por medio de talleres está visible para los usuarios & \\
3 & 25 & Se puede localizar la existencia de actividades de cooperación con el ámbito universitario para el desarrollo de proyectos: bibliotecas universitarias naciones o internacionales & 
Se puede localizar la existencia de actividades de cooperación con el ámbito universitario para el desarrollo de proyectos: bibliotecas universitarias nacionales o internacionales \\
3 & 27 & Se puede localizar la existencia de actividades de cooperación con el ámbito universitario para el desarrollo de proyectos: PTGAS & 
Se puede localizar la existencia de actividades de cooperación con el ámbito universitario para el desarrollo de proyectos: Personal Técnico, de Gestión y de Administración y Servicios (PTGAS) \\
3 & 28 & Se puede localizar la existencia de actividades de cooperación con el ámbito universitario para el desarrollo de proyectos: otros & 
Se puede localizar la existencia de actividades de cooperación con el ámbito universitario para el desarrollo de proyectos: otros (empresas, asociaciones culturales, ONG…) \\
3 & 33 & Existen planes de actuación para el desarrollo de proyectos: otros
Existen planes de actuación para el desarrollo de otros proyectos \\
\bottomrule
\end{tabular}
\source{Elaboración propia.}
\end{threeparttable}
\end{table}

En este sentido, se han atendido las aportaciones de los jueces por considerarse que aportaban matices de mejora desde una interpretación externa, como muestra el dato significativo de la dimensión 3, concretamente, en la subdimensión “Promoción de la lectura” que pasó de 4 a 3 ítems al unificarse dos que se proponían inicialmente (Ver \Cref{apx-longtable}).

\section{Discusiones y conclusiones}\label{sec-fmt-manuscrito}
Una vez visto el estado de la cuestión, es indudable que la influencia de las TIC ha desencadenado cambios profundos y multifacéticos en todos los ámbitos de la vida moderna, desde lo social y político hasta lo laboral y, especialmente, lo educativo. En este último ámbito, se observa la emergencia de una nueva generación de estudiantes, comúnmente referidos como nativos digitales \cite{prensky_digital_2001} o Generación Z \cite{alvarez_ramos_generacion_2019}. Estos jóvenes se caracterizan por no concebir su interacción con el mundo sin el uso de las herramientas que les ofrece Internet; han crecido rodeados de tecnología, usándola de manera natural y desarrollando habilidades digitales superiores a las de generaciones anteriores. Por ello, se hace necesario crear espacios en consonancia con esta sociedad gaseosa \cite{scolari_cultura_2021} para que la ciudadanía, en general, y los docentes, estudiantes universitarios y la escuela, en particular, puedan acceder a recursos digitales de manera eficiente, adaptándose a las dinámicas cambiantes y fluidas de la realidad actual.

En este contexto, las bibliotecas se posicionan como elementos fundamentales para facilitar el acceso a la información, ofreciendo un entorno que integre los continuos avances tecnológicos. Frente al nuevo paradigma educativo y social, es crucial que las bibliotecas se adapten y evolucionen hacia centros de recursos digitales capaces de satisfacer las necesidades de la sociedad líquida actual. Estos cambios, aunque abren nuevas oportunidades, también presentan desafíos cognitivos en la formación de ciudadanos y lectores críticos del siglo XXI. De ahí que nuestro objetivo se haya centrado en diseñar y validar una lista de verificación práctica y accesible que identifique los elementos esenciales para la creación de una Nubeteca, la cual será de gran utilidad para todos los agentes -institucionales y no institucionales- \cite{romero_oliva_habitos_2020} interesados en la promoción de la lectura y la difusión de información. Aunque ya se han realizado estudios sobre Nubetecas \cite{valbuena_impacto_2019,cordon-garcia_mpacto_2020}, consideramos necesario validar un modelo específico para la creación de estos espacios multimodales mediante este instrumento que facilite la sistematización y comparación de las bibliotecas multimodales, que combinan recursos analógicos y digitales, con las tradicionales.

Para alcanzar este objetivo, se empleó una metodología que se centró en la creación de una lista de verificación y control a través de una escala valorativa, que abarcó diversas dimensiones y subdimensiones relacionadas con la gestión de la Nubeteca. El proceso fue enriquecido con el criterio de expertos en el ámbito, todos con formación doctoral y conocimientos especializados en la promoción de la lectura y la gestión de bibliotecas digitales, quienes proporcionaron una visión crítica fundamental para la revisión del instrumento. Posteriormente, se realizó un proceso de validación estadística utilizando el Alfa de Cronbach, que arrojó un valor excelente de 0.912, lo que no solo confirmó la fiabilidad del instrumento, sino que también indicó su potencial para ser extrapolado a otros contextos de investigación similares. En respuesta a las recomendaciones recibidas durante la revisión, se llevaron a cabo modificaciones en varios ítems, lo que fortaleció aún más la robustez y precisión de la herramienta evaluativa. Este proceso de validación y ajuste garantizó que la lista de verificación y control cumpliera con los estándares necesarios para una evaluación efectiva y aplicable en diversos entornos:

\begin{itemize}
    \item Los jueces expertos realizaron algunas sugerencias que fueron incorporadas para mejorar la precisión y la inclusividad de los ítems. Un ejemplo de estas modificaciones se encuentra en la Dimensión 1, "Organización espacial", donde se evalúa la localización y la percepción visual del espacio multimodal. En este caso, se propuso ajustar el lenguaje del ítem correspondiente para que, en lugar de referirse solo a "usuarios", incluyera también a "usuarias". Este cambio se realizó con el objetivo de asegurar un lenguaje más inclusivo, que refleje adecuadamente la diversidad de género entre las personas que utilizan la biblioteca.
    \item En la Dimensión 2, "Equipamientos", que evalúa los recursos tecnológicos del espacio multimodal, se realizaron varias modificaciones clave para mejorar la claridad y precisión de los ítems. Por ejemplo, se sugirió ampliar las descripciones de los recursos digitales, especificando que, además de los monitores Smart TV, se consideren también monitores de ordenador, y que las tabletas digitales se mencionen en lugar de marcas específicas como iPads. Asimismo, se incluyó la opción de que los ordenadores táctiles de sobremesa puedan utilizarse con teclados, y se propuso concretar ejemplos de recursos adicionales, como las gafas de realidad aumentada, para reflejar mejor la diversidad tecnológica que puede integrarse en estos espacios. Estos cambios ayudan a que la escala sea más comprensiva y adaptable a diferentes contextos tecnológicos.
    \item Por último, para la Dimensión 3 “Política de gestión de la biblioteca”, que examina la política de promoción de la lectura y la gestión de los catálogos y servicios públicos, propusieron varias mejoras. Estas sugerencias incluyeron la revisión de varios ítems para asegurar un lenguaje más inclusivo y detallado. Por ejemplo, se ajustaron las referencias a las formaciones dirigidas a bibliotecarios, especificando la inclusión de "bibliotecarios y bibliotecarias", y se amplió la descripción de las actividades de formación, explicando que estas incluyen a distintos perfiles de personal, como el Personal Docente Investigador (PDI) y el Personal Técnico, de Gestión y de Administración y Servicios (PTGAS). Además, se revisaron las opciones de préstamos digitales, asegurando que se mencionara la posibilidad de realizarlos a través de una variedad de dispositivos, como ordenadores, tabletas y códigos QR. Asimismo, se mejoró la claridad en los ítems relacionados con las actividades de promoción de la lectura, ampliando las menciones a clubes de lectura y talleres para reflejar una mayor diversidad de iniciativas. También se especificaron las actividades de cooperación con el ámbito universitario, detallando las colaboraciones con diferentes tipos de personal y otras entidades como empresas y asociaciones culturales. Por último, se precisó la mención a los planes de actuación, destacando su enfoque en el desarrollo de una variedad de proyectos.
\end{itemize}

Finalmente, se ha de destacar que el proyecto I+D+i PID2021-126392OB-I00 Lecturas no ficcionales para la integración de ciudadanas y ciudadanos críticos en el nuevo ecosistema cultural (Lenficec),, en el que participan diversas universidades españolas como Zaragoza, Vigo, Granada y Cádiz, tiene como objetivo analizar la variedad de contextos que conforman los nuevos ecosistemas culturales, con un enfoque especial en las bibliotecas como espacios clave para la promoción de la lectura. En línea con esta iniciativa, y como parte de una investigación superior, que corresponde a la Tesis Doctoral Diseño e implementación de un espacio multimodal para la formación de ciudadanos desde el Horizonte 2030 y los libros ilustrados de no ficción como prospección de este estudio, se propone la creación de un espacio multimodal en la Biblioteca del Campus de Puerto Real como un tercer espacio educativo. Este espacio permitirá la participación de futuros maestros, la escuela y estudiantes de magisterio, fomentando un vínculo directo entre la formación inicial del profesorado y la realidad educativa. En este espacio, se promoverán las lecturas no ficcionales como eje central de actividades que buscarán integrar a una ciudadanía crítica en el nuevo ecosistema cultural, mientras se promoverán las lecturas no ficcionales como eje central de actividades que integren a una ciudadanía crítica en el nuevo ecosistema cultural.

%Tabela no anexo

\section*{Contexto de la Investigación}
Este trabajo se enmarca en la tesis doctoral de Blanca Florido Zarazaga Diseño e implementación de un espacio multimodal para la formación de ciudadanos desde el Horizonte 2030 y los libros ilustrados de no ficción Programa de doctorado 8219 Investigación y práctica educativa, Universidad de Cádiz.

\printbibliography\label{sec-bib}
%conceptualization,datacuration,formalanalysis,funding,investigation,methodology,projadm,resources,software,supervision,validation,visualization,writing,review
\begin{contributors}[sec-contributors]
\authorcontribution{Blanca Florido Zarazaga}[conceptualization,investigation,methodology,writing]
\authorcontribution{Manuel Francisco Romero Oliva}[conceptualization,investigation,writing,review]
\end{contributors}



\appendix 
\section{Anexo 1}\label{apx-longtable}

\begin{small}
\centering
\begin{longtable}{p{8cm} |l|l|l|l|l}
\caption{DIMENSIÓN 1. ORGANIZACIÓN ESPACIAL}
\label{longtbl-01}
\\
\toprule
ÍTEM & 1 & 2 & 3 & 4 & 5 \\
\midrule
La Nubeteca se encuentra en un lugar claramente identificable por los usuarios y las usuarias de la biblioteca & & & & & \\
\midrule
Existe un plano de la distribución de los recursos de la Nubeteca para usuarios & & & & & \\
\midrule
La Nubeteca da visibilidad al catálogo digital en el espacio físico a través de tarjetas de estantes, pegatinas, carteles… & & & & & \\
\midrule
El espacio físico de la Nubeteca propicia la interacción entre los usuarios (mobiliario integrado como mesas, para equipos de aprendizaje; tabletas, para expandir el conocimiento; pantallas, para videoconferencias…). & & & & & \\
\bottomrule
\end{longtable}
\end{small}

\begin{small}
\centering
\begin{longtable}{p{8cm} |l|l|l|l|l}
\caption{DIMENSIÓN 2. EQUIPAMIENTOS}
\label{longtbl-02}
\\
\toprule
ÍTEM & 1 & 2 & 3 & 4 & 5 \\
\midrule
Los recursos digitales están integrados como un monitor Smart TV o también un monitor de ordenador & & & & & \\
\midrule
Los recursos digitales están integrados como un tabletas digitales & & & & & \\
\midrule
Los recursos digitales están integrados como un ordenador táctil de sobremesa o con teclados & & & & & \\
\midrule
Apuesta por la integración de otros recursos digitales  como por ejemplo Gafas de Realidad Aumentada & & & & & \\
\bottomrule
\end{longtable}
\end{small}

\begin{small}
\centering
\begin{longtable}{p{8cm} |l|l|l|l|l}
\caption{DIMENSIÓN 3. POLÍTICA DE GESTIÓN DE LA BIBLIOTECA}
\label{longtbl-03}
\\
\toprule
ÍTEM & 1 & 2 & 3 & 4 & 5 \\
\midrule
Se lleva a cabo de bibliotecarios y bibliotecarias  con usuarios dirigidas a la lectura digital (tutoriales sobre dispositivo, plataformas de lectura, la socialización de la lectura…) & & & & & \\
\midrule
La recomendación de lecturas se realiza a través del monitor y de otros dispositivos (ordenador de sobremesa y tableta) & & & & & \\
\midrule
Existe coordinación con responsables de titulaciones universitarias para el uso de los recursos de la biblioteca & & & & & \\
\midrule
Se promueven otras actividades de formación de lectura con Personal Docente Investigador (PDI)/Personal Técnico, de Gestión y de Administración y Servicios (PTGAS) & & & & & \\
\midrule
\multicolumn{6}{c}{CATÁLOGO} \\
\midrule
Dispone de un catálogo a través del ordenador & & & & & \\
\midrule
Dispone de un catálogo a través de la tableta & & & & & \\
\midrule
Dispone de un catálogo a través de los códigos QR & & & & & \\
\midrule
Dispone de un catálogo a través de otros dispositivos electrónicos & & & & & \\
\midrule
\multicolumn{6}{c}{PRÉSTAMOS} \\
\midrule
Existe posibilidad de realizar préstamos a través del ordenador & & & & & \\
\midrule
Existe posibilidad de realizar préstamos a través de la tableta & & & & & \\
\midrule
Existe posibilidad  de realizar préstamos a través de los códigos QR & & & & & \\
\midrule
Posibilidad de realizar préstamos a través de otros dispositivos electrónicos & & & & & \\ 
\midrule
\multicolumn{6}{c}{PROMOCIÓN DE LA LECTURA} \\
\midrule
La existencia de clubes, talleres y otro tipo de actividades para la promoción de la lectura & & & & & \\
\midrule
La existencia de actividades de acompañamiento de lectura por medio de talleres está visible para los usuarios & & & & & \\
\midrule
Se dinamizan ahí las colecciones digitales mediante propuestas didácticas de carácter multimodal & & & & & \\
\midrule
\multicolumn{6}{c}{INTEGRACIÓN EN LAS COMUNIDADES UNIVERSITARIAS} \\
\midrule
Se puede localizar la existencia de actividades de cooperación con el ámbito universitario para el desarrollo de proyectos: bibliotecas universitarias nacionales o internacionales & & & & & \\
\midrule
Se puede localizar la existencia de actividades de cooperación con el ámbito universitario para el desarrollo de proyectos: docentes universitarios & & & & & \\
\midrule
existencia de actividades de cooperación con el ámbito universitario para el desarrollo de proyectos: Personal Técnico, de Gestión y de Administración y Servicios (PTGAS) & & & & & \\
\midrule
Se puede localizar la existencia de actividades de cooperación con el ámbito universitario para el desarrollo de proyectos: otros (empresas, asociaciones culturales, ONG…) & & & & & \\
\midrule
\multicolumn{6}{c}{TRANSFERENCIA SOCIAL EN LA UNIVERSIDAD} \\
\midrule
Fortalece la cooperación con el ámbito no universitario para el desarrollo de proyectos: centros escolares & & & & & \\
\midrule
Existen planes de actuación para el desarrollo de proyectos con bibliotecas no universitarias & & & & & \\
\midrule
Existen planes de actuación para el desarrollo de proyectos con centros escolares & & & & & \\
\midrule
Existen planes de actuación para el desarrollo de proyectos con el entorno comunicatorio (familias, asociaciones, entidades…) & & & & & \\
\midrule
Existen planes de actuación para el desarrollo de otros proyectos & & & & & \\
\bottomrule
\end{longtable}
\end{small}

\end{document}
