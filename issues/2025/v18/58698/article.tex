% !TEX TS-program = XeLaTeX
% use the following command:
% all document files must be coded in UTF-8
\documentclass[spanish]{textolivre}
% build HTML with: make4ht -e build.lua -c textolivre.cfg -x -u article "fn-in,svg,pic-align"

\journalname{Texto Livre}
\thevolume{19}
%\thenumber{1} % old template
\theyear{2026}
\receiveddate{\DTMdisplaydate{2025}{4}{18}{-1}} % YYYY MM DD
\accepteddate{\DTMdisplaydate{2025}{8}{8}{-1}}
\publisheddate{\DTMdisplaydate{2025}{11}{18}{-1}}
\corrauthor{José Pablo Salazar Aguilar}
\articledoi{10.1590/1983-3652.2026.58698}
%\articleid{NNNN} % if the article ID is not the last 5 numbers of its DOI, provide it using \articleid{} commmand 
% list of available sesscions in the journal: articles, dossier, reports, essays, reviews, interviews, editorial
\articlesessionname{articles}
\runningauthor{Salazar Aguilar y Bonilla-Cruz} 
%\editorname{Leonardo Araújo} % old template
\sectioneditorname{Hugo Heredia Ponce~\orcid{0000-0003-3657-1369}}
\layouteditorname{Saula Cecília~\orcid{0009-0006-3069-8480 }}

\title{Conferencias de prensa en línea y la legitimación del Poder Ejecutivo en Costa Rica (2022-2024)}
\othertitle{Conferências de imprensa \textit{online} e a legitimação do Poder Executivo na Costa Rica (2022-2024)}
% if there is a third language title, add here:
\othertitle{Online press conferences and the legitimization of the Executive Branch in Costa Rica (2022-2024)}

\author[1]{José Pablo Salazar Aguilar~\orcid{0000-0001-6372-6915}\thanks{Email: \href{mailto:jose.salazar@ulatina.cr}{jose.salazar@ulatina.cr}}}
\author[1]{Cristian Bonilla-Cruz~\orcid{0000-0003-1864-9791}\thanks{Email: \href{mailto:cristian.bonilla@ulatina.cr}{cristian.bonilla@ulatina.cr}}}
\affil[1]{Universidad Latina de Costa Rica, Costa Rica.}


\addbibresource{article.bib}
% use biber instead of bibtex
% $ biber article

% used to create dummy text for the template file
\definecolor{dark-gray}{gray}{0.35} % color used to display dummy texts
\usepackage{lipsum}
\SetLipsumParListSurrounders{\colorlet{oldcolor}{.}\color{dark-gray}}{\color{oldcolor}}

% used here only to provide the XeLaTeX and BibTeX logos
\usepackage{hologo}

% if you use multirows in a table, include the multirow package
\usepackage{multirow}

% provides sidewaysfigure environment
\usepackage{rotating}

% CUSTOM EPIGRAPH - BEGIN 
%%% https://tex.stackexchange.com/questions/193178/specific-epigraph-style
\usepackage{epigraph}
\renewcommand\textflush{flushright}
\makeatletter
\newlength\epitextskip
\pretocmd{\@epitext}{\em}{}{}
\apptocmd{\@epitext}{\em}{}{}
\patchcmd{\epigraph}{\@epitext{#1}\\}{\@epitext{#1}\\[\epitextskip]}{}{}
\makeatother
\setlength\epigraphrule{0pt}
\setlength\epitextskip{0.5ex}
\setlength\epigraphwidth{.7\textwidth}
% CUSTOM EPIGRAPH - END

% to use IPA symbols in unicode add
%\usepackage{fontspec}
%\newfontfamily\ipafont{CMU Serif}
%\newcommand{\ipa}[1]{{\ipafont #1}}
% and in the text you may use the \ipa{...} command passing the symbols in unicode

% LANGUAGE - BEGIN
% ARABIC
% for languages that use special fonts, you must provide the typeface that will be used
% \setotherlanguage{arabic}
% \newfontfamily\arabicfont[Script=Arabic]{Amiri}
% \newfontfamily\arabicfontsf[Script=Arabic]{Amiri}
% \newfontfamily\arabicfonttt[Script=Arabic]{Amiri}
%
% in the article, to add arabic text use: \textlang{arabic}{ ... }
%
% RUSSIAN
% for russian text we also need to define fonts with support for Cyrillic script
% \usepackage{fontspec}
% \setotherlanguage{russian}
% \newfontfamily\cyrillicfont{Times New Roman}
% \newfontfamily\cyrillicfontsf{Times New Roman}[Script=Cyrillic]
% \newfontfamily\cyrillicfonttt{Times New Roman}[Script=Cyrillic]
%
% in the text use \begin{russian} ... \end{russian}
% LANGUAGE - END

% EMOJIS - BEGIN
% to use emoticons in your manuscript
% https://stackoverflow.com/questions/190145/how-to-insert-emoticons-in-latex/57076064
% using font Symbola, which has full support
% the font may be downloaded at:
% https://dn-works.com/ufas/
% add to preamble:
% \newfontfamily\Symbola{Symbola}
% in the text use:
% {\Symbola }
% EMOJIS - END

% LABEL REFERENCE TO DESCRIPTIVE LIST - BEGIN
% reference itens in a descriptive list using their labels instead of numbers
% insert the code below in the preambule:
%\makeatletter
%\let\orgdescriptionlabel\descriptionlabel
%\renewcommand*{\descriptionlabel}[1]{%
%  \let\orglabel\label
%  \let\label\@gobble
%  \phantomsection
%  \edef\@currentlabel{#1\unskip}%
%  \let\label\orglabel
%  \orgdescriptionlabel{#1}%
%}
%\makeatother
%
% in your document, use as illustraded here:
%\begin{description}
%  \item[first\label{itm1}] this is only an example;
%  % ...  add more items
%\end{description}
% LABEL REFERENCE TO DESCRIPTIVE LIST - END


% add line numbers for submission
%\usepackage{lineno}
%\linenumbers

\begin{document}
\maketitle

\begin{polyabstract}
\begin{abstract}
Las conferencias de prensa en vivo y en directo desde la Presidencia de la República de Costa Rica, encabezadas por el presidente Rodrigo Chaves Robles, resultan un condicionante de la legitimación de la gestión política del actual gobierno costarricense. Dichas conferencias presentan altos registros de audiencias, en el marco de la confrontación directa entre el Poder Ejecutivo, medios de comunicación y otros poderes públicos. En el presente estudio  de corte cuantitativo, se monitorea la audiencia durante 76 comparecencias ante la prensa comprendidas entre 2022 y 2024, y se analiza de la escucha digital en la web pública durante ese mismo lapso, para determinar la tendencia de audiencia digital de estas; identificar los puntos de mayor audiencia digital de dichas conferencias, como elemento de atención a los mensajes del Poder Ejecutivo; y establecer la correlación entre el nivel de audiencia digital alcanzada y el sentimiento en la conversación digital. Por tanto, este recurso de comunicación ha dejado de ser meros espacios informativos. Se han convertido en una táctica clave para la legitimación del Ejecutivo costarricense, al reducir la mediación de los medios tradicionales y priorizar la comunicación directa con la ciudadanía.

\keywords{Conferencia virtual\sep Prensa\sep Poder Ejecutivo\sep Electrónica\sep Gobierno}
\end{abstract}

\begin{portuguese}
\begin{abstract}
As coletivas de imprensa ao vivo da Presidência da República da Costa Rica, lideradas pelo Presidente Rodrigo Chaves Robles, são um fator determinante na legitimidade da administração política do atual governo costarriquenho. Essas coletivas têm altos números de audiência, em meio ao confronto direto entre o Poder Executivo, a mídia e outros poderes públicos. Este estudo quantitativo monitora o público durante 76 coletivas de imprensa entre 2022 e 2024 e analisa a escuta digital no site público durante o mesmo período para determinar as tendências de audiência digital; identificar os pontos de maior audiência digital dessas coletivas, como um fator na atenção dada às mensagens do Poder Executivo; e estabelecer a correlação entre o nível de audiência digital alcançada e o sentimento na conversa digital. Portanto, esse recurso de comunicação deixou de ser meros espaços informativos. Tornou-se uma tática fundamental para legitimar o governo costarriquenho, reduzindo a mediação da mídia tradicional e priorizando a comunicação direta com os cidadãos.

\keywords{Conferência virtual\sep Imprensa\sep Poder Executivo\sep Eletrônica\sep Governo}
\end{abstract}
\end{portuguese}

\begin{english}
\begin{abstract}
Live press conferences from the Presidency of the Republic of Costa Rica, led by President Rodrigo Chaves Robles, are a determining factor in the legitimacy of the current Costa Rican government's political management. These conferences have high audience ratings, in the context of direct confrontation between the executive branch, the media, and other public authorities. In this quantitative study, the audience is monitored during 76 press appearances between 2022 and 2024, and digital listening on the public web during the same period is analyzed to determine the digital audience trend for these appearances; identify the points of highest digital audience for these conferences, as an element of attention to the messages of the Executive Branch; and establish the correlation between the level of digital audience reached and the sentiment in the digital conversation. Therefore, this communication resource is no longer merely an informative space. It has become a key tactic for the legitimization of the Costa Rican Executive, reducing the mediation of traditional media and prioritizing direct communication with citizens.

\keywords{Virtual conference\sep Digital press\sep Executive Branch\sep Electronics\sep Government}
\end{abstract}
\end{english}
\end{polyabstract}

\section{Introducción}
La comunicación política en la era digital ha transformado radicalmente la forma en que los gobiernos interactúan con los ciudadanos \cite{asimakopoulos2025, dneprovskaya2018, manoharan2018}. En Costa Rica, las transmisiones en vivo de la Presidencia de la República, cada semana, se han consolidado como una herramienta clave para la gestión de la comunicación gubernamental, caracterizada por una alta polarización mediática, así lo ha registrado el Observatorio de Comunicación Digital (OCD) de la Universidad Latina de Costa Rica, desde octubre de 2021, y lo ha constatado desde inicio de la presente administración Chaves Robles, en mayo de 2022. 

De manera puntual, los nuevos formatos que combinan la inmediatez de los medios digitales y el acceso directo a información de la fuente oficial han permitido a los gobiernos establecer un canal de comunicación más directo con la ciudadanía, sin depender exclusivamente de la mediación de la prensa tradicional \cite{andrade2021, llano2020conferencias, salgado2023, vera2020}. No obstante, persisten retos de legitimación política por el antagonismo de la prensa hegemónica. 

El Observatorio de Comunicación Digital de la Universidad Latina de Costa Rica recopila evidencia cuantitativa de manera sostenida, la cual permite examinar el impacto de estas conferencias de prensa en la percepción pública del gobierno \cite{observatorio2023a, observatorio2023b, observatorio2023c, observatorio2023e}. Los datos sugieren que las sesiones informativas han sido clave en la construcción de la narrativa política oficial, mientras que los medios tradicionales han adoptado una postura contraria al discurso del gobierno. Esta tensión ha generado un ambiente de confrontación comunicativa en el que las rondas de prensa sirven como un instrumento de legitimación política, ofreciendo al gobierno un espacio para contrarrestar la narrativa mediática dominante, siempre contraria a la gestión gubernamental.

Este estudio es de gran relevancia social y académica, ya que aborda las formas en que las conferencias de prensa digitales se han convertido en una herramienta clave de narrativa para la legitimación política del gobierno costarricense en un contexto de polarización mediática \cite{stone2001}. En un escenario donde las relaciones entre gobernantes y ciudadanos están cada vez más mediadas por las plataformas digitales, entender el impacto de estas conferencias en la percepción pública es fundamental para comprender cómo las estrategias de comunicación digital pueden influir en la estabilidad política y la confianza en las instituciones democráticas.

Los resultados de esta investigación no solo permiten un análisis teórico profundo, sino que también tienen aplicaciones prácticas inmediatas y los hallazgos servirán para mejorar las estrategias discursivas de comunicación política, al proporcionar datos sobre la audiencia digital y el sentimiento social, lo cual fortalece la relación entre el gobierno y los ciudadanos \cite{nicoll2018}.

\section{Revisión de la literatura}
El estudio de las conferencias y su influencia en la legitimación política se inscribe dentro del marco más amplio de la comunicación política digitalizada y revolucionada por la inteligencia artificial \cite{tegmark2018}. Con la evolución exponencial de las tecnologías de la información y la comunicación \cite{kurzweil2017}, las plataformas digitales permiten a casi todos comunicarse y a los gobiernos y actores políticos con la efectividad que les permite la transacción, y sin intermediarios explícitos, trascendiendo los modelos tradicionales de flujo de información \cite{chadwick2017}.

Las conferencias de prensa digitales han sido utilizadas como uno de los mecanismos para configurar la relación entre gobernantes y ciudadanos, durante periodos breves (un gobierno), medianos (más de un gobierno), e incluso largos periodos, usados por las derechas e izquierdas por igual. Ofrecen transparencia, aunque en ocasiones cuestionable, y acceso directo a información de interés público, a menudo parcializada, todo ello respaldado por diversos discursos políticos \cite{hermida2010, jalit2021, ramirez2022odio, novais2024, ramirezgarcia2022, rodriguezperez2021}.

La legitimación política puede alcanzarse mediante varias vías o una combinación de aquellas, como por ejemplo la mediatización o el lobbismo, lo que es lo mismo, instrumentalizando a los medios de comunicación masiva como vehículo para que la sociedad valide determinada política pública o gestión política procedimental \cite{crespo2021, granda2019, milsom2021}.

Los medios de comunicación se usan mediante diferentes tácticas como la pauta o el lobbismo político en social media, aunque también se presentan cuando las fuerzas o élites políticas y económicas están aliadas o fusionadas y los medios de comunicación están a su servicio \cite{robles2021}. Con el conocimiento de esta realidad, la producción de discursos y narrativas funciona con efectividad en la transición del sistema tradicional al digitalizado, en tanto convergencia y complejidad \cite{cantos2018,garciaorosa2017, lopezgarcia2021}. Antes de profundizar, para el caso costarricense, el enfrentamiento genuino entre el gobierno y los medios de comunicación poseedores de una postura agonista confrontativa veraz, resultan víctimas del silencio y la legitimidad institucional \cite{bhat2020, egelhofer2021, hinterleitner2017, koliska2020, kramer2018, macaraig2022, meeks2020, novais2023, van2021, weaver2018}.

Además, la legitimación política a través de las conferencias digitales se ve afectada por la percepción del público hacia aquellas. Estudios previos señalan que la proliferación de fuentes digitales ha llevado a una fragmentación o polarización a todo nivel y a una visibilización de dicha fragmentación preexistente, lo que podría reducir la efectividad de estos mecanismos en términos de legitimación \cite{arora2022, boczkowski2013, lee2016, wilson2020}. 

A medida que la comunicación política muta hacia un entorno cada vez más digitalizado, es crucial comprender cómo estas dinámicas afectan la relación entre los tomadores de decisión política, la legitimidad de los regímenes políticos, y las percepciones de la sociedad \textit{online}, considerando que los contextos de comunicación antagonista han estado vigentes \textit{ex ante} la comunicación digital \cite{garcia2019, gomezdiago2022, saintout2015}. En el mundo, los medios de comunicación [cuarto poder] consagran su rol determinante en la política global, en tanto posiciones extremas evidentes en lo histórico concreto y en la realidad actual, con una ``pausa'' forzada por el shock pandémico del Covid19 \cite{slimovich2021a, slimovich2021b}. Los grandes emporios mediáticos están vinculados inevitablemente a élites político-económicas, e instrumentalizan las plataformas de social media para sus fines de interés específico mediante el uso de todas las posibilidades creativas y tecnológicas disponibles y en evolución, siendo las conferencias, \textit{streaming} o transmisiones en vivo una de las más efectivas \cite{apablazacampos2018, llano2020conferencias, lopezgarcia2021, montes2022, ramirezplascencia2022}.

En Latinoamérica, países como Argentina, Brasil, Venezuela, México, El Salvador, son ejemplos de la efectividad de estos encuentros con los \textit{social media} y el uso de las redes sociales por parte de sus poderes ejecutivos, con casos ilustrativos como el español \cite{castilloesparcia2020, daniel2022, esteinou2021}. Si bien este fenómeno es global, se aborda el problema considerando las particularidades socioculturales de sus contextos propios y del complejo panorama tecnológico en uso \cite{du2016, farhall2019}.

En México, las mañaneras de López Obrador ilustran un fenómeno similar de legitimación política \cite{salgado2023}. Además del consumo diario y coyuntural de estas conferencias, van conformando un acervo digital relevante, cuyo análisis da cuenta de la agenda mediática y política con la que la Presidencia busca incidir \cite{ramirezsanhez2021}. 

\section{La Comunicación Institucional en la Presidencia de Costa Rica}
En el contexto costarricense, la investigación sobre la comunicación gubernamental digital es incipiente, aunque se ha avanzado en la comprensión de su impacto de manera empírica. El último reporte relacionado con las comparecencias públicas data de 2020, cuando este instrumento de comunicación era usado todos los días por el Ministerio de Salud de Costa Rica para actualizar al público sobre la pandemia por Covid19 \cite{campbell2021}, quedando luego disponibles en redes sociales. Otro trabajo responde a cómo el uso de la comunicación politizada de un medio de comunicación digital, creado para tal fin, contribuye a deslegitimar un gobierno, al tiempo que fortalece la percepción de cercanía con la población, ante el constante escrutinio mediático \cite{lopezgonzalez2021}.

En contraste con los medios digitales, la prensa hegemónica ha mantenido su rol antagónico al gobierno, actuando como contrapeso a la información oficial, frente a la fuerza política que ocupa la administración del Estado costarricense, puesto que resulta un evidente conflicto contra élites históricamente representadas desde la prensa en Costa Rica \cite{molina2022, robles2021, urbina2017}, en el amplio marco del periodo en que se realiza la presente investigación (2022-2024). Este rivalismo entre medios de comunicación y gobierno es común, mas no beneficioso en sistemas de medios pluralistas, como el costarricense \cite{hallin2004}.

La pertinencia de este trabajo recae en su efecto espejo para Latinoamérica. El activismo desde el Poder Ejecutivo para comunicar las acciones y expresar intereses, es un poder informal que ha sido profundamente aprovechado, con un efecto evidente en la legitimación de su quehacer, siendo mecanismo autenticados las ``declaraciones, discursos y conferencias de prensa, hasta manifestaciones y respuestas a consultas de periodistas o medios de comunicación'', en lo que se ha denominado a nivel analítico el poder informal ``del megáfono'' \cite[p. 46]{consejo2024}.

En concreto, la contribución es al debate sobre el uso configurativo de estas, ofreciendo un análisis cuantitativo del impacto de la comunicación de la Presidencia de la República de Costa Rica durante el periodo 2022-2024 en la legitimación política del gobierno, sin omitir la fricción que tiende a afectar al régimen político vigente. Es menester del lector, situar este estudio en aquellos regímenes de bienestar que guarden similitudes, en el presente histórico, en países del Sur Global, aquejados por posiciones populistas in extremis, de derecha o de izquierda.

Entiéndase que ``la conversación digital refleja momentos en la coyuntura y evolución del proceso político-electoral y que se debe tomar como un elemento importante a considerar; ya que al igual que los medios tradicionales, las redes sociales llegaron a convertirse en elementos que son utilizados en forma masiva por la población costarricense'' \cite[p. 117]{arce2023}, de ahí la necesidad de darle seguimiento a este tipo de estudios en la vida política de las democracias occidentales.

\section{Metodología}
Esta investigación tiene como objetivo general (OG) analizar la relación existente entre las transmisiones digitales de la Presidencia de la República de Costa Rica y la legitimación política del gobierno, considerando el alcance de la comunicación política y el sentimiento de la audiencia digital (mayo 2022 - mayo 2024). De este, se desglosan los siguientes objetivos específicos:

\begin{itemize}
\item OE1: Determinar la tendencia de audiencia digital que atiende a conferencias de prensa de consejo de gobierno generadas por la Presidencia de la República de Costa Rica, durante ese periodo.
    
\item OE2: Identificar los puntos de mayor audiencia digital de dichas conferencias, como elemento de atención a los mensajes discursivos por parte del Poder Ejecutivo de Costa Rica.

\item OE3: Establecer una correlación entre el nivel de audiencia digital alcanzada y el sentimiento en la conversación en redes sociales y web pública hacia la figura presidencial de Costa Rica, producto de las comparecencias ante los medios del gobierno.
\end{itemize}

Para lograr los objetivos, se realiza un estudio cuantitativo de la audiencia digital que sigue las conferencias de prensa. Para ello, se utiliza la herramienta Kantar Social TV Ratings \cite{kantar2024}, que proporciona datos sobre las dos redes sociales donde se transmiten las conferencias: Facebook (META) y YouTube (Google). Este programa computarizado escanea todos los dispositivos conectados y lleva un registro cuantitativo minuto a minuto de la audiencia en los perfiles de redes sociales que previamente se han configurado para estos fines. Finalmente, el sistema genera dos datos relevantes para esta investigación:

\begin{itemize}
\item Cantidad de dispositivos digitales (ordenadores, tabletas o teléfonos celulares) conectados en vivo durante la transmisión, lo cual permite determinar los ``picos de audiencia'' y el ``promedio de audiencia'' que estuvo conectada en vivo en el transcurso de la transmisión.

\item  Alcance de dispositivos que revisaron posteriormente la transmisión en las siguientes 48 horas de haber sido transmitido \cite{facebook2024}. En YouTube, una visualización se considera cuando un usuario inicia la visualización del video de forma intencionada y ve al menos 30 segundos de contenido.
\end{itemize}

Durante el periodo de análisis (8 de mayo de 2022 al 31 de mayo de 2024) se registran 76 conferencias de prensa de gobierno transmitidas por las redes sociales Facebook y YouTube de la Presidencia de la República de Costa Rica, efectuadas los miércoles de forma recurrente y cuyo propósito principal es dar anuncios oficiales desde el Poder Ejecutivo, confrontar a algunos entes que adversan al gobierno y dar un espacio de consultas por parte de medios de comunicación que asisten presencial a dichos encuentros en Casa Presidencia, San José, Costa Rica. La duración promedio de estas conferencias es de una hora, 39 minutos y 22 segundos, en todos los casos transmitidas por los perfiles oficiales de Presidencia de Costa Rica. Adicionalmente al proceso de monitoreo de audiencia, se utiliza el método denominado escucha social \textit{(social listening)}, de la conversación en redes sociales y web pública, para lo cual se utilizan las plataformas Mention® y Digital Adspend®, las cuales hicieron un seguimiento diario desde el 8 de mayo de 2022 hasta el 31 de mayo de 2024. Este monitoreo se centra en menciones vinculadas a diversas al presidente de la República de Costa Rica, Rodrigo Chaves Robles.

Durante este intervalo, se acumularon 460.572 menciones que incorporaron las palabras claves predefinidas en los sistemas correspondientes de recolección. Estas palabras claves abarcan el nombre completo del Presidente, así como pseudónimos comúnmente utilizados para referirse a dicha figura; mientras que adicionalmente se agregan etiquetas usadas en redes sociales (\Cref{tab-1}).

%--- codigo da tabela 1 ---%
\begin{table}[h!]
\centering
\begin{threeparttable}
\caption{Palabras para la generación: Alerta presidente.}\label{tab-1}
\begin{tabular}{p{3.5cm} p{10cm}}
\toprule
 & Palabras \\
\midrule
Rodrigo Chaves Robles & Presidente Rodrigo Chaves Robles; Rodrigo (Near to) Presidente (near to Costa Rica); Presidencia (near to Costa Rica); Partido Progreso Social Democrático; PPSD; Progreso Social Democrático; Rodrigo Chavez (near to Presidente) (near to Costa Rica); MrC; @conocearodrigo; @rodrigochavesr; \#RodrigoChavesRenuncie; \#Zoodrigo; \#MeComoLaBronca; \#RodrigoChavesPresidente; \#TrabajandoDecidiendoMejorando; Trabajando, Decidiendo, Mejorando; Partido (Near to) Rodriguista; Partido Aquí Costa Rica Manda (near to Rodrigo); Aquí Costa Rica Manda (near to Rodrigo) \\
\bottomrule
\end{tabular}
\source{Elaboración propia.}
\end{threeparttable}
\end{table}


La metodología utilizada no establece una muestra de publicaciones digitales en términos probabilísticos, sino que recopila todas las menciones realizadas en redes sociales y web pública, durante el periodo especificado antes. Por su parte, los sistemas evalúan la carga emotiva o tono de cada comentario en elementos positivos, negativos o neutros. Es decir, las menciones positivas expresan apoyo explícito al presidente; las negativas, rechazo o crítica; y las neutras, simple descripción objetiva sin valoración.

\section{Resultados}
Acerca del desempeño de las conferencias de prensa del Consejo de Gobierno costarricense en el periodo estudiado, destacan seis grandes bloques de comportamiento de la audiencia digital. A continuación, se detallan los hallazgos al respecto de la tendencia de audiencias con base en la totalidad del periodo considerado.

\begin{enumerate}[label=\alph*)]
    \item \emph{Altos niveles iniciales (mayo -- julio 2022).} Se presenta durante el proceso de los primeros 100 días de gobierno. En estas transmisiones se capta un interés significativo de la audiencia. La administración Chaves Robles utiliza estas conferencias como una herramienta efectiva de comunicación, impulsado por el interés alrededor del anuncio de las medidas en contra del Parque Viva, propiedad del Grupo Nación, uno de los emporios mediáticos que ha sido catalogado por el presidente Rodrigo Chaves como ``prensa canalla'', debido a sus posiciones en contra de la actual administración.

    \item \emph{Disminución en audiencia y alcance (agosto -- noviembre 2022).} A partir de agosto de 2022, y especialmente en noviembre de 2022, se observa un descenso notable en la audiencia digital (ver Figura \ref{fig-1}). En comparación con los primeros meses de gobierno, el número de dispositivos conectados en vivo disminuyó 36 \%, y el alcance de visualizaciones experimentó una caída del 89 \% (ver Figura \ref{fig-2}). Este descenso estuvo motivado por una pérdida de novedad en las conferencias y una posible falta de temas nuevos que capturan el interés de la audiencia; aun así, durante este periodo, el presidente Chaves y varios funcionarios de su administración empezaron a utilizar un lenguaje directo y en ocasiones, despectivo contra medios de comunicación como La Nación, CR Hoy, El Observador, Telenoticias, entre otros.

    \item \emph{Reducción continua (diciembre 2022 -- marzo 2023).} En el primer trimestre de 2023, la caída en la audiencia de dispositivos conectados y en el alcance digital fue aún más marcada. En este periodo, la audiencia promedio de dispositivos        conectados se redujo un 40 \%, y el alcance de las reproducciones digitales cayó un 92 \% en comparación con los primeros meses del mandato. Esta pérdida de audiencia fue en parte debido a una estabilización del interés del público. Sin embargo, este comportamiento cambió a principios de 2023, cuando el Gobierno presentó el proyecto de `Ciudad Gobierno', que proponía eliminar alquileres de edificios pagados históricamente a familiares y allegados de un partido político opositor (ver Figura \ref{fig-1}).
\end{enumerate}

%--- código da figura 1 ---%
\begin{figure}[h!]
\centering
\begin{minipage}{\textwidth}
\includegraphics[width=\textwidth]{Imagens/Fig1.png}
\caption{Alcance promedio que tuvieron las transmisiones del Consejo de Gobierno del presidente Chaves Robles (8 de mayo de 2022 al 31 de mayo de 2024).}
\label{fig-1}
\source{Kantar Social TV Ratings, mayo 2024.}
\end{minipage}
\end{figure}


\begin{enumerate}[resume, label=\alph*)]
    \item \emph{Estabilización y recuperación parcial (abril -- agosto 2023).} A mediados de 2023, la audiencia digital experimentó una estabilización con una ligera recuperación. Se observa un aumento de 8,91 \% en la audiencia conectada en vivo en comparación con el trimestre anterior, aunque estos niveles seguían siendo un 57 \% menores que los observados en el mismo periodo de 2022. Esta recuperación temporal puede estar vinculada a temas coyunturales abordados en las ruedas de medios de comunicación, tales como la presentación de un Recurso de Amparo en contra del Periódico La Nación, por ``información falsa'' que, según el Ejecutivo, les daña la imagen.

\item \emph{Nueva caída en el último trimestre de 2023 (setiembre -- diciembre 2023).} En el cuarto trimestre de 2023, la audiencia promedio de dispositivos conectados volvió a disminuir 45 \%. Este descenso fue consistente con una tendencia de reducción en el interés general hacia las rondas informativas, salvo en momentos puntuales de alto impacto mediático. Este periodo no hubo mayor confrontación, más que con algunos diputados de la Asamblea Legislativa, que han sido detractores de las políticas de seguridad de la administración Chaves Robles. 

\item \emph{Recuperación de altos niveles de conexión (enero -- mayo 2024).} En este      periodo, dos eventos específicos lograron captar nuevamente una gran cantidad de dispositivos conectados y visualizaciones: ``Reunión con la Contraloría General de la República'', en febrero 2024, lo cual desembocó en una nueva confrontación, con amenazas y violencia verbal por parte del Ejecutivo (ver Figuras \ref{fig-2} y \ref{fig-3}). 
\end{enumerate}

%--- código da figura 2 ---%
\begin{figure}[h!]
\centering
\begin{minipage}{\textwidth}
\includegraphics[width=\textwidth]{Imagens/Fig2.png}
\caption{Promedio de dispositivos conectados ``en vivo'' a las transmisiones del Consejo de Gobierno del presidente Chaves Robles (8 de mayo de 2022 al 31 de mayo de 2024).}
\label{fig-2}
\source{Kantar Social TV Ratings, mayo 2024.}
\end{minipage}
\end{figure}

%--- código da figura 3 ---%
\begin{figure}[h!]
\centering
\begin{minipage}{\textwidth}
\includegraphics[width=\textwidth]{Imagens/Fig3.jpg}
\caption{Pico de audiencia de dispositivos conectados ``en vivo'' a las transmisiones del Consejo de Gobierno del presidente Chaves Robles (8 de mayo de 2022 al 31 de mayo de 2024).}
\label{fig-3}
\source{Kantar Social TV Ratings, mayo 2024.}
\end{minipage}
\end{figure}


La evolución de la audiencia de las comparecencias ante los medios está mediada por algunos factores que favorecen la atención de los cibernautas. Como se nota en las \Cref{fig-1,fig-2,fig-3} hay una coincidencia de mayor atención al inicio de esta administración, ya que el factor de novedad mueve a un grupo de personas a observar un producto ``nuevo'' de comunicación, en tanto previo a esta administración, no se transmitían estos encuentros informativos con frecuencia fija.

Por otra parte, en los primeros días también se nota un pico máximo de atención y que está vinculado con las Medidas de Cierre del Parque Viva (unidad de negocio del Periódico La Nación), en una clara confrontación del Poder Ejecutivo, hacia un medio de comunicación, con el cual tenía un enfrentamiento desde la época electoral. De ahí que ese elemento de confrontación genera a su vez un ``morbo'' en la audiencia y un grupo importante de internautas conectados a estas transmisiones. Esa atención también se disminuye en la medida que va pasando el tiempo y se va diluyendo la novedad, llevando los dispositivos conectados a una baja considerable de 36 \% de audiencia, al comparar el primer periodo de transmisiones dentro los primeros 100 días de administración.

Algo interesante que es tangible en el comparativo de las \Cref{fig-1,fig-2,fig-3} es que los ``picos de audiencia'' más importantes tienen una respuesta similar en cuanto el promedio de audiencia y las visualizaciones posteriores, de forma que se ``calca'' el comportamiento en los tres escenarios. Otro fenómeno interesante de la evolución de la audiencia que atiende a dichas transmisiones es que se fidelizan los cibernautas en un bloque, el cual no se mantiene homogéneo, sino que tiene ``picos'' de atención y otros momentos donde el mismo se va perdiendo. Sin embargo, esos momentos de mayor atención responden a rondas de preguntas con medios donde continúa la disputa con el medio de prensa ``La Nación'' y otros de confrontación con las instituciones públicas, como la Contraloría General de la República y los Partidos de Oposición (Proyecto de Ciudad Gobierno).

A continuación, se detalla el desempeño de estos periodos de análisis, considerando las variables de promedio de dispositivos conectados ``en vivo'', alcance promedio del periodo, pico máximo de audiencia y el porcentaje de crecimiento o descenso, con respecto al periodo anterior (\Cref{tab-2}).

%--- código da tabela 2 ---%
\begin{table}[h!]
\centering
\begin{threeparttable}
\caption{ Resumen de variables relacionadas con el desempeño de las conferencias de prensa del Consejo de Gobierno (8 de mayo de 2022 al 31 de mayo de 2024).}\label{tab-2}
\begin{tabular}{p{4cm} p{2.2cm} p{2.2cm} p{2.2cm} p{2.2cm}}
\toprule
Periodo de Análisis & Promedio Dispositivos Conectados & Alcance \newline Promedio & Pico de \newline Audiencia & Crecimiento/\newline Descenso (\%) \\
\midrule
Altos niveles iniciales (mayo a julio 2022) & 5.905 & 129.297 & 17.209 & -- \\
Disminución en audiencia y alcance (agosto – noviembre 2022) & 4.160 & 71.435 & 13.665 & -36 \% \\
Reducción continua (diciembre 2022 – marzo 2023) & 3.543 & 10.144 & 4.822 & -14.8 \% \\
Estabilización y recuperación parcial (abril – agosto 2023) & 3.684 & 74.680 & 7.433 & +8.91 \% \\
Nueva caída en el último trimestre de 2023 (setiembre – diciembre 2023) & 2.310 & 94.031 & 4.822 & -45.46 \% \\
Recuperación de altos niveles de conexión (enero – mayo 2024) & 3.353 & 101.569 & 7.433 & +7.06 \% \\
\bottomrule
\end{tabular}
\source{Elaboración propia.}
\end{threeparttable}
\end{table}



Ahora, en relación con el sentir social \textit{online} y su correlación con la audiencia en dichas conferencias, se destaca que en los primeros tres meses de la administración Chaves Robles, se observa un alto nivel de audiencia y un sentimiento digital marcadamente positivo. En este periodo, medidas como el cierre de Parque Viva generaron polémica, pero también apoyo, manteniendo un 52,8 \% de comentarios positivos y alto respaldo digital.

A partir de agosto de 2022 y hasta noviembre de ese mismo año, el interés y la percepción positiva hacia estas transmisiones experimentaron un descenso notable y marcó en correspondencia, una caída en el sentimiento positivo, manteniéndose en torno al 45 \% y casi duplicando ese sentimiento frente a los comentarios negativos, los cuales fueron aumentando paulatinamente. Además, en este periodo, la Sala Constitucional anuló el cierre de Parque Viva, lo que aumentó la percepción negativa hacia el gobierno en ciertos círculos y evidenció un declive en el interés por las conferencias.

Entre diciembre de 2022 y marzo de 2023, la disminución en audiencia y sentimiento positivo se intensificó. La audiencia promedio en dispositivos conectados descendió a 3543, mientras que el alcance total cayó a solo 10144 visualizaciones, representando una reducción del 92 \% respecto al periodo inicial de la administración. Paralelamente, el sentimiento positivo cayó a niveles críticos, y los comentarios negativos aumentaron en 40 \%. La caída en audiencia se vinculó a escándalos como el caso `Piero Calandrelli' y denuncias de financiamiento indebido, que generaron desconfianza y críticas. Esta percepción se reflejó en el distanciamiento de la audiencia de las transmisiones por redes sociales lo cual evidencia la relación entre temas de ética gubernamental y la pérdida de interés en los canales oficiales de comunicación.

Entre abril y agosto de 2023, la audiencia y el sentimiento hacia el gobierno experimentaron una estabilización y ligera recuperación. En cuanto al sentimiento digital, se observó un balance entre comentarios positivos y negativos, con aproximadamente un 38-40 \% de positividad. Este cambio se debió, en parte, a temas coyunturales, como la implementación de la `Operación Soberanía' y el decomiso de droga en los puertos, lo cual despertó un renovado interés en el tema de seguridad nacional. Sin embargo, el sentimiento de escepticismo seguía presente, y aunque el interés en las conferencias aumentó levemente, este interés era aún menor en comparación con el periodo inicial de la administración.

Durante el último trimestre de 2023, de septiembre a diciembre, el gobierno experimentó otra caída en la audiencia y en el sentimiento positivo. En términos de sentimiento, los comentarios negativos superaron el 36 \%, particularmente debido a la creciente inseguridad y el aumento récord en la tasa de homicidios. Además, los fallos de la Sala Constitucional, que obligaron al Ejecutivo a revertir decisiones en la Caja Costarricense de Seguro Social (CCSS), generaron críticas adicionales y un debilitamiento en la confianza del público, a lo cual es indispensable sumar los reportajes generados por el Grupo Nación denominados ``Los Audios de Presidencia'', en los cuales la exministra de comunicación de este gobierno compartió con el medio de comunicación conversaciones privadas, donde se compromete la integridad del Ejecutivo.

Por último, entre enero y mayo de 2024, se observó una recuperación en la audiencia y en el sentimiento hacia la figura presidencial, impulsada por eventos como el anuncio de un posible referéndum y la reunión con la Contraloría General de la República. El sentimiento positivo temporalmente alcanzó 43 \%, generando una discusión profunda en redes sociales, sobre el papel que debe cumplir la contralora general y otros órganos de control del gasto público (ver Figura \ref{fig-4}).

%--- código da figura 4 ---%
\begin{figure}[h!]
\centering
\begin{minipage}{\textwidth}
\includegraphics[width=\textwidth]{Imagens/Fig4.png}
\caption{Evolución de los sentimientos en la conversación digital hacia la figura del presidente Chaves Robles (8 de mayo de 2022 al 31 de mayo de 2024).}
\label{fig-4}
\source{Elaboración propia.}
\end{minipage}
\end{figure}


\section{Discusión}
Estos espacios con los medios de comunicación funcionan como un recurso estratégico del Ejecutivo para controlar narrativas \cite{swanson1990}, pero con una (a) simetría clave: los picos de audiencia digital no siempre se correlacionan con una validación del actual del gobierno costarricense. Esto sugiere que, aunque \textit{agenda setting} \cite{mccombs1993} logra posicionar temas, el \textit{framing} \cite{entman2010, entman1993} aplicado no siempre consigue legitimación automática. Al tiempo que la exposición mediática semanal garantiza consenso, también acciona polarización con determinados grupos de interés específico, minorías o mayorías minimizadas. 

La alta validación en contextos de confrontación (contra medios como Grupo Nación o la Contraloría General de la República) refleja el rol del Presidente y su buró en el contexto de este recurso de comunicación, validado y viralizado por medios masivos. El Presidente, al asumir un rol de ``líder en disputa'', encuadra la narrativa hacia ``nosotros vs. ellos'' que moviliza a su base \textit{online}. De tal forma, los eventos conflictivos generan mayor \textit{engagement} digital: la performancia de enfrentamiento dramatiza el poder y redefine la conversación pública, como si fuese un ritual político \cite{alexander2011}.

La correlación entre sentimiento y audiencia muestra que el encuadre confrontativo no solo atrae, sino que guía la discusión semanal, definiendo qué temas importan; aunque la audiencia digital reconfigura esa agenda desde la emoción, desafiando los modelos clásicos.

Es necesario comparar los hallazgos de este estudio con investigaciones previas sobre el uso de conferencias de prensa en la legitimación política. Estudios como los de \textcite{mccombs1993} sobre \textit{agenda setting} y \textcite{entman2010} sobre \textit{framing} muestran cómo los gobiernos utilizan plataformas para influir en la percepción pública. Este trabajo confirma que este espacio de comunicación gubernamental es clave para la legitimación, en línea con otros estudios latinoamericanos \cite{llano2020conferencias, salgado2023}, que destacan el papel creciente de la comunicación digital en la política.

Al comparar este estudio con investigaciones previas \cite{gomezdiago2022}, se observa cómo la polarización mediática y la confrontación con los medios tradicionales se intensifican en contextos digitales. Este análisis refuerza la idea de que las plataformas como Facebook y YouTube están siendo utilizadas estratégicamente por los gobiernos, lo que contribuye a la evolución de la comunicación política en un entorno digitalizado.

En síntesis, estas conferencias han dejado de ser meros espacios informativos para convertirse en una pieza central dentro de la estrategia de legitimación del Ejecutivo costarricense. Al reducir la intermediación de los medios de comunicación tradicionales y priorizar la comunicación directa con la ciudadanía, el gobierno no solo controla su narrativa, sino que refuerza una relación simbiótica con una audiencia digital predispuesta a validar su discurso confrontativo. Estos hallazgos evidencian cómo las dinámicas de comunicación política contemporánea dependen cada vez más de la performancia digital y de la capacidad de los actores políticos para moldear la agenda y el encuadre de la realidad a través de sus propios canales de difusión.

\section{Conclusiones}
Al tenor de cada objetivo planteado, este trabajo concluye que, en primer lugar, OE1:  [1] El efecto de novedad genera una tendencia de atención y aumento de dispositivos conectados ``en vivo'' o bien que revisan posteriormente las conferencias de prensa de la Presidencia de la República; [2] transcurrido el nivel de novedad, la tendencia es a la baja y se consolida un grupo de cibernautas que continúa viendo “en vivo” dichas transmisiones; y [3] los picos de audiencia son proporcionales a los picos más altos de dispositivos conectados en promedio y que revisan en forma posterior a la cobertura digital.

Ahora bien, en relación con los puntos de mayor audiencia digital de dichas conferencias (OE2): Durante dos años, la tendencia de nuevos ``picos de audiencia'' se da donde hay conflicto y confrontación en dos vías, a saber, el conflicto contra medios de comunicación como La Nación, y en contra de poderes públicos como la Contraloría General de la República (CGR) y partidos políticos opositores.

Finalmente, al correlacionar el nivel de audiencia digital alcanzada con el sentimiento en la conversación en redes sociales y web pública hacia la figura presidencia, consecuencia de estas transmisiones (OE3), se establece que: [1] El mayor volumen positivo hacia el presidente Rodrigo Chaves está conectado con el elemento de confrontación y conflicto con el emporio mediático Grupo Nación y el antagonismo con CGR. Pareciera que el Ejecutivo se ha dispuesto a mantener un discurso de confrontación para mantener niveles altos de sentimiento positivo a nivel de la conversación digital.

Este estudio se limita a conferencias digitales en Costa Rica, lo que restringe la generalización. La medición de audiencia no refleja calidad de interacción y el análisis de sentimiento puede tener sesgos. Pese a ello, aporta insumos valiosos y abre líneas para futuras investigaciones.

Esta investigación abre la posibilidad de estudiar estos fenómenos digitales, reconociendo que los modelos clásicos de comunicación política no son suficientes para comprender la realidad. Al centrarse en la comunicación digital, descuida la comunicación \textit{offline}, y viceversa. Hoy no se logran diferenciar los efectos de un mundo específico sobre otro, qué es realidad y qué es ficción, un reto \textit{sine qua non} para la academia y la ciencia, en términos de comprender el efecto de las tecnologías como relación social.


\printbibliography\label{sec-bib}
% if the text is not in Portuguese, it might be necessary to use the code below instead to print the correct ABNT abbreviations [s.n.], [s.l.]
%\begin{portuguese}
%\printbibliography[title={Bibliography}]
%\end{portuguese}


%full list: conceptualization,datacuration,formalanalysis,funding,investigation,methodology,projadm,resources,software,supervision,validation,visualization,writing,review
\begin{contributors}[sec-contributors]
\authorcontribution{José Pablo Salazar Aguilar}[conceptualization,formalanalysis,validation,writing,review]
\authorcontribution{Cristian Bonilla-Cruz}[conceptualization,datacuration,formalanalysis,visualization,writing,review]
\end{contributors}

\begin{dataavailability}
\txtdataavailability{dataavailable} % options: dataavailable, dataonly, databody, datanotav, nodata
\end{dataavailability}


\end{document}

