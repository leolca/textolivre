% !TEX TS-program = XeLaTeX
% use the following command:
% all document files must be coded in UTF-8
\documentclass[portuguese]{textolivre}
% build HTML with: make4ht -e build.lua -c textolivre.cfg -x -u article "fn-in,svg,pic-align"

\journalname{Texto Livre}
\thevolume{18}
%\thenumber{1} % old template
\theyear{2025}
\receiveddate{\DTMdisplaydate{2024}{1}{3}{-1}} % YYYY MM DD
\accepteddate{\DTMdisplaydate{2025}{4}{16}{-1}}
\publisheddate{\DTMdisplaydate{2025}{6}{14}{-1}}
\corrauthor{Larissa Bortoluzzi Rigo}
\articledoi{10.1590/1983-3652.2025.56822}
%\articleid{NNNN} % if the article ID is not the last 5 numbers of its DOI, provide it using \articleid{} commmand 
% list of available sesscions in the journal: articles, dossier, reports, essays, reviews, interviews, editorial
\articlesessionname{articles}
\runningauthor{Rigo e Hohlfeldt} 
%\editorname{Leonardo Araújo} % old template
\sectioneditorname{Daniervelin Pereira}
\layouteditorname{Leonado Araújo}

\title{Jornalismo Cultural e o \textit{webreview} do século XXI, inovação ou tradição?}
\othertitle{Cultural Journalism and the webreview of the XXI century, innovation or tradition?}
% if there is a third language title, add here:
%\othertitle{Artikelvorlage zur Einreichung beim Texto Livre Journal}

\author[1]{Larissa Bortoluzzi Rigo~\orcid{0000-0001-6308-334X}\thanks{Email: \href{mailto:lary_rigo@yahoo.com.br}{lary\_rigo@yahoo.com.br}}}
\author[2]{Antonio Hohlfeldt~\orcid{0000-0001-5284-8730}\thanks{Email: \href{mailto:hohlfeldt@yahoo.com.br}{hohlfeldt@yahoo.com.br}}}
\affil[1]{Universidade Regional Integrada do Alto Uruguai e das Missões, campus Frederico Westphalen, RS, Brasil.}
\affil[2]{Universidade Federal do Rio Grande do Sul, Faculdade de Biblioteconomia e Comunicação, Porto Alegre, RS, Brasil.}

\addbibresource{article.bib}
% use biber instead of bibtex
% $ biber article

% used to create dummy text for the template file
\definecolor{dark-gray}{gray}{0.35} % color used to display dummy texts
\usepackage{lipsum}
\SetLipsumParListSurrounders{\colorlet{oldcolor}{.}\color{dark-gray}}{\color{oldcolor}}

% used here only to provide the XeLaTeX and BibTeX logos
\usepackage{hologo}

% if you use multirows in a table, include the multirow package
\usepackage{multirow}

% provides sidewaysfigure environment
\usepackage{rotating}

% CUSTOM EPIGRAPH - BEGIN 
%%% https://tex.stackexchange.com/questions/193178/specific-epigraph-style
\usepackage{epigraph}
\renewcommand\textflush{flushright}
\makeatletter
\newlength\epitextskip
\pretocmd{\@epitext}{\em}{}{}
\apptocmd{\@epitext}{\em}{}{}
\patchcmd{\epigraph}{\@epitext{#1}\\}{\@epitext{#1}\\[\epitextskip]}{}{}
\makeatother
\setlength\epigraphrule{0pt}
\setlength\epitextskip{0.5ex}
\setlength\epigraphwidth{.7\textwidth}
% CUSTOM EPIGRAPH - END

% to use IPA symbols in unicode add
%\usepackage{fontspec}
%\newfontfamily\ipafont{CMU Serif}
%\newcommand{\ipa}[1]{{\ipafont #1}}
% and in the text you may use the \ipa{...} command passing the symbols in unicode

% LANGUAGE - BEGIN
% ARABIC
% for languages that use special fonts, you must provide the typeface that will be used
% \setotherlanguage{arabic}
% \newfontfamily\arabicfont[Script=Arabic]{Amiri}
% \newfontfamily\arabicfontsf[Script=Arabic]{Amiri}
% \newfontfamily\arabicfonttt[Script=Arabic]{Amiri}
%
% in the article, to add arabic text use: \textlang{arabic}{ ... }
%
% RUSSIAN
% for russian text we also need to define fonts with support for Cyrillic script
% \usepackage{fontspec}
% \setotherlanguage{russian}
% \newfontfamily\cyrillicfont{Times New Roman}
% \newfontfamily\cyrillicfontsf{Times New Roman}[Script=Cyrillic]
% \newfontfamily\cyrillicfonttt{Times New Roman}[Script=Cyrillic]
%
% in the text use \begin{russian} ... \end{russian}
% LANGUAGE - END

% EMOJIS - BEGIN
% to use emoticons in your manuscript
% https://stackoverflow.com/questions/190145/how-to-insert-emoticons-in-latex/57076064
% using font Symbola, which has full support
% the font may be downloaded at:
% https://dn-works.com/ufas/
% add to preamble:
% \newfontfamily\Symbola{Symbola}
% in the text use:
% {\Symbola }
% EMOJIS - END

% LABEL REFERENCE TO DESCRIPTIVE LIST - BEGIN
% reference itens in a descriptive list using their labels instead of numbers
% insert the code below in the preambule:
%\makeatletter
%\let\orgdescriptionlabel\descriptionlabel
%\renewcommand*{\descriptionlabel}[1]{%
%  \let\orglabel\label
%  \let\label\@gobble
%  \phantomsection
%  \edef\@currentlabel{#1\unskip}%
%  \let\label\orglabel
%  \orgdescriptionlabel{#1}%
%}
%\makeatother
%
% in your document, use as illustraded here:
%\begin{description}
%  \item[first\label{itm1}] this is only an example;
%  % ...  add more items
%\end{description}
% LABEL REFERENCE TO DESCRIPTIVE LIST - END


% add line numbers for submission
%\usepackage{lineno}
%\linenumbers

\begin{document}
\maketitle

\begin{polyabstract}
\begin{abstract}
Com o olhar que contempla as configurações sociais e históricas em torno das aproximações e intercomunicabilidade \cite{lajolo1996} entre Jornalismo e Literatura, em espaços constituídos pelo Jornalismo Cultural \cite{hohlfeldt2003,lima2013,soares2014}, traçamos o questionamento desta reflexão. Debruçamo-nos em caracterizar etapas históricas vividas por suplementos literários e \textit{webreviews}, do século XIX ao século XXI. Utilizamos como metodologia, a investigação exploratória, de caráter histórico-crítico-bibliográfica, em uma análise qualitativa e de conteúdo \cite{herscovitz2012,bardin1977}. Nossas reflexões estão centradas em seis edições \cite{krippendorff1990} do \textit{webreview} \textit{Peixe Elétrico}. Inspiramo-nos nas particularidades para identificarmos as tendências do Jornalismo Cultural do século XXI. Evoluiu-se do suplemento, que buscava um público amplo, mas indistinto, porque vinha encartado em jornais de referência, para publicações específicas, dirigidas diretamente para o interessado. Apesar das mudanças de suporte, da modernização das mídias, no que se refere aos aparatos tecnológicos, ao longo de cinquenta anos, os suplementos continuam basicamente com o mesmo formato. Considerando o impacto das mudanças tecnológicas nos comportamentos de leitores e modos de produção do jornalismo digital, o tipo de conteúdo proposto e a divisão de editorias do Jornalismo Cultural tiveram apenas mudanças de suporte – do impresso ao digital, e de \textit{layout}. \textit{Peixe Elétrico} não possui as características do ciberjornalismo, logo, visualizamos uma tendência à manutenção do estilo de se fazer Jornalismo Cultural no Brasil.

\keywords{Jornalismo Cultural \sep Literatura \sep Jornalismo \sep \textit{Peixe Elétrico} \sep Suplementos}
\end{abstract}

\begin{english}
\begin{abstract}
With a view that contemplates the social and historical configurations around approximations and intercommunicability \cite{lajolo1996} between Journalism and Literature, in spaces constituted by Cultural Journalism \cite{hohlfeldt2003,lima2013,soares2014}, we trace the questioning of this reflection. We focus on characterizing historical stages experienced by literary supplements and webreviews, from the nineteenth century to the twenty-first century. We used as a methodology the exploratory investigation, of a historical-critical-bibliographic nature, in a qualitative and content analysis \cite{herscovitz2012,bardin1977}. Our reflections are centered on six editions \cite{krippendorff1990} of the webreview "Electric Fish". We were inspired by the particularities to identify the trends of Cultural Journalism in the 21st century. It evolved from the supplement, which sought a broad but indistinct audience, because it was published in reference newspapers, to specific publications, directed to the interested party. Despite the changes in support and the modernization of the media, with regard to technological devices, over fifty years, the supplements continue basically with the same format. Considering the impact of technological changes on reader behaviors and modes of production of digital journalism, the type of content proposed and the division of editorials of Cultural Journalism had only changes in support – from print to digital, and in layout. \textit{Peixe Elétrico} does not have the characteristics of cyberjournalism, so we see a tendency to maintain the style of doing Cultural Journalism in Brazil.

\keywords{Cultural Journalism \sep Literature \sep Journalism \sep Electric Fish \sep Supplements}
\end{abstract}
\end{english}
% if there is another abstract, insert it here using the same scheme
\end{polyabstract}

\section{A arte literária e os escritores na imprensa}\label{sec-intro}
Desde o surgimento dos jornais, a cultura é um de seus objetos de atenção. Dentre os assuntos contemplados nos espaços jornalísticos chamados de Jornalismo cultural, uma das primeiras áreas a ocupar as páginas dos periódicos foi a Literatura. Importante ressaltar que entendemos o Jornalismo Cultural como uma especialidade do Jornalismo e nessa reflexão também propomos aproximações com outro termo, jornalismo literário, que indica, não a imprensa especializada em Literatura, mas a presença de escritores nos periódicos. Os dois termos possuem como referências o Jornalismo e a Literatura.

Propomos algumas aproximações entre os campos jornalístico e cultural, por meio de um panorama histórico e social. Nosso intuito é demonstrar que o jornalismo cultural possui conotação híbrida, interligada aos gêneros jornalístico e literário. Neste sentido, evidenciamos duas perspectivas dependentes: a presença da crítica literária e de escritores nos periódicos em espaços específicos dentro dos jornais e os suplementos literários que evoluíram, como iremos demonstrar em cadernos de cultura.

Para atender ao objetivo de concretizar a conceituação destes espaços nos jornais, e uma prospecção futura do campo do Jornalismo Cultural, temos como corpus, a análise de um \textit{webreview}, intitulado \textit{Peixe Elétrico}. Para os recursos metodológicos, utilizamos a análise de conteúdo \cite{gil1989,bardin1977}, em um estudo qualitativo e exploratório \cite[p. 28]{freitas2002} que contempla uma leitura crítica dos textos presentes nas edições.

Além disso, consideramos que a análise de conteúdo se apresenta como a metodologia mais eficiente para esta reflexão, pois contribui para a “interpretação da vida social” \cite[p. 123]{herscovitz2007analise}. Ao tomar o Jornalismo Cultural por meio do \textit{Webreview}, utilizamo-nos desta análise para detectar tendências e modelos, além da avaliação de características que nos permitem inferir a conceituação atual e, ainda, identificar uma configuração da visualização futura do campo. Nesse sentido, \textcite[p. 139]{herscovitz2012} aponta que a grande virtude do método “é a possibilidade de analisar uma grande quantidade de informações por um longo período de tempo, observando tendências em diferentes momentos históricos”.

Para a investigação exploratória de caráter histórico-crítico, relacionada a critérios qualitativos, a partir da análise de conteúdo, os materiais que embasam o corpus dessa reflexão foram escolhidos a partir da primeira edição do \textit{Peixe Elétrico}, e um total de seis edições. A hipótese com a qual trabalhamos decorre da constatação trazida pela obra de \textcite{krippendorff1990}, \textit{Metodologia de análisis de contenido: Teoria y practica}, a qual entende que há uma probabilidade estatística suficiente na análise de seis edições. O autor propõe não haver necessidade de se averiguar um grande número de matérias, pois não ocorrerão modificações no resultado da pesquisa:

\begin{quote}
    Stempel (1952) comparou amostras constituídas de seis, doze, dezoito, vinte e quatro e quarenta e oito números de um jornal com números que aparecem durante um ano inteiro, e comprovou, usando como medida a proporção média de materiais tratados, que o aumento no tamanho da amostra, além de doze, não produziu resultados significativamente mais precisos.
\end{quote}

É importante mencionar que a seleção do \textit{corpus} segue um percurso que envolve a análise de conteúdo, explorando os materiais e transcrevendo com exemplos, para posteriormente passarmos a reflexão sobre o pioneirismo que o veículo se propõe a trazer para os usuários-leitores. Iniciamos pelo contexto sócio-histórico que envolve o Jornalismo Cultural, para inserirmos as observações em torno de \textit{Peixe Elétrico}.

\section{Os suplementos literários como instrumento de leitura da Literatura}\label{sec-normas}
Iniciamos o percurso histórico contextualizando o século XIX, que é valorizado por \textcite{arnt2001}, por ser caracterizado como o período em que a história da imprensa jornalística se confunde com o jornalismo literário, já que, desde o surgimento do jornal, a Literatura marcou presença como um dos objetos apresentados aos leitores por meio de suas páginas. Para a autora, “uma análise do Jornalismo do século XIX deixa evidente um fato: a enorme participação de escritores na vida dos jornais” \cite[p. 07]{arnt2001}. Tanto no papel de editores dos periódicos, como no de cronistas, os escritores interferem no modo de fazer e conceber um jornal. “A influência dos escritores foi de tal ordem, que podemos qualificar esse período da história da imprensa de jornalismo literário” \cite[p. 07]{arnt2001}.

O termo jornalismo literário, para \textcite[p. 07-08]{arnt2001}, relaciona-se ao estilo de escrita e linguagem que se desenvolveu no século XIX e que teve como característica principal a militância de escritores na imprensa, juntamente com a publicação de gêneros híbridos ao Jornalismo e à Literatura, tais como o folhetim e a crônica.

Para \textcite{arnt2001}, é importante não confundir jornalismo literário com o papel da imprensa especializada em literatura. A crítica sobre literatura é um processo que nasceu com os jornais e ainda hoje está presente em suplementos culturais. Já o jornalismo literário é um estilo jornalístico que se desenvolveu no século XIX e tem como característica principal a utilização de técnicas narrativas próprias à Literatura.

A análise da imprensa do século XIX se relaciona também com a própria história de acesso do grande público à leitura, sobretudo nos países europeus e nos Estados Unidos, onde o desenvolvimento industrial possibilitou o alcance à leitura de modo mais amplo. Nesses contextos, as obras literárias serviram como propulsoras à disseminação de conteúdos, ou, como afirma \textcite[p. 08]{arnt2001}, “[...] a imprensa foi fundamental na democratização da cultura letrada”.

Em meados do século XIX, os europeus iniciaram um processo de alfabetização para formar mão de obra necessária às novas funções sociais criadas pela Revolução Industrial. Nesse contexto, o jornal teve papel central, pois preencheu necessidades culturais dos novos consumidores, ofertando Literatura para suprir as demandas de outros produtos, tais como os livros, que tinham altos preços: “Como o livro ainda era muito caro para os assalariados, o jornal preencheu essa lacuna, publicando folhetins, romances e contos” \cite[p. 09]{arnt2001}.

\textcite{arnt2001} menciona que, no contexto do século XIX, os Estados Unidos não aderiram ao movimento de folhetins publicados em jornais, tal como se evidenciou na França e na Inglaterra. A influência dos escritores estadunidenses ocorreu por meio de editores e cronistas que publicaram contos e novelas em revistas especializadas. Os textos desenvolvidos nestes espaços tornaram-se referência para o processo de alfabetização daquele país: “Essas publicações eram extremamente populares e tiveram um papel fundamental no programa de alfabetização de massa, desencadeado no país no final do século XX” \cite[p. 09]{arnt2001}. O espaço dos contos e novelas equivaleu à publicação dos folhetins na Europa e constituíram a chamada \textit{pulpfiction} ou \textit{penny press}.

Com as mudanças do Jornalismo, no final do século XIX, o estilo da crítica cultural, realizada nos periódicos, também se altera. Uma das representações dessas transformações pode ser identificada no irlandês George Bernard Shaw (1856-1950). Ele foi crítico literário, de música e de arte, escrevia para a \textit{Saturday Review} e o \textit{The World}.

O formato do novo Jornalismo cultural solidifica-se na virada do século XX. Até então, os jornais eram compostos por articulistas políticos, debates em torno de artes e livros, mas traziam pouco noticiário. Posteriormente a este período, escritores passam a dar maior atenção aos relatos dos fatos e para a construção das notícias. Do mesmo modo, o Jornalismo cultural segue essa tendência, com matérias que envolvem, além da crítica de arte, entrevistas e reportagens.

Como nosso percurso é cronológico, torna-se importante contemplar também o contexto brasileiro, sobretudo, o início da história da imprensa no país, para melhor contextualizar os primeiros passos do Jornalismo cultural e acompanhar os movimentos históricos indicados até aqui. \textit{O Correio Braziliense} ou \textit{Armazem Literário} é considerado, por \textcite[p. 03]{dines2000}, como o marco inicial da imprensa periódica brasileira, o primeiro jornal português livre de censura: “é o precursor do jornalismo político português”.

A demanda crescente por material impresso de leitura, apontada por \textcite[p. 89]{soares2014}, continua na década seguinte, época em que, justamente, observamos o surgimento da seção de folhetins no periodismo brasileiro, “mais propriamente no jornal \textit{O Moderador, Novo Correio do Brasil} e \textit{Jornal Político, Comercial e Literário}”. Com a divulgação de folhetins na Europa e contos e novelas nos Estados Unidos, logo essas publicações chegam ao Brasil, “que primeiro traduziu os folhetins franceses e, depois, começou a publicar, com Manuel Antônio de Almeida e José de Alencar, os próprios folhetins” \cite[p. 09]{arnt2001}. Como exemplo de divulgação dos folhetins, \textcite{meyer1996}, cita o romance de José de Alencar, \textit{O Guarani}, em 1857, no jornal \textit{Diário do Rio de Janeiro}. Assim, a aproximação entre a Literatura e a imprensa é fortalecida por meio do \textit{feuilleton}.

A adoção do \textit{feuilleton} acaba se tornando, na imprensa brasileira, uma parte da repercussão mundial. Observando esse cenário de produções que envolvem os periódicos supracitados, é importante considerar que, no Brasil do século XIX, não ocorreu somente uma mera cópia ou repetição dos produtos franceses, mas, sim, esse foi o modelo privilegiado para inspirar nossas produções nacionais.

Já no século XX, com todas essas mudanças sociais e históricas, o crítico torna-se menos moralista, mais incisivo e, assim, de certo modo, afasta-se do papel exercido pela figura do jornalista. No Brasil, até 1950, as duas áreas eram dependentes, sobretudo porque escritores trabalhavam em jornais e jornalistas se dedicavam a textos que tinham caráter essencialmente literário. No entanto, nessa época, assemelhando-se ao contexto estadunidense de separar a informação da opinião, e aludindo à teoria do espelho – em que o Jornalismo opera como um reflexo da realidade e por isso é vinculado ao mito da (im)parcialidade – os veículos precisaram se adaptar a essas regras e separar os textos jornalísticos daqueles não-jornalísticos (informação e opinião).

Com essa separação, os jornalistas, para não misturar conteúdos de caráter opinativo, criaram os suplementos literários. “Se até a década de 1950 a crítica era um produto fácil de encontrar nas páginas do jornal, daí em diante ela foi parar nos cadernos culturais e nos suplementos” \cite[p. 40]{lima2013}. O afastamento entre informação e opinião se caracteriza no processo que \textcite[p. 13]{santiago1993critica} chama de desliteraturalização do jornal. Os suplementos se transformam no instrumento de leitura da Literatura, com um olhar crítico ao social, inserido nas artes e politização, como acrescenta \textcite[p. 157]{lima2013}: “ao dar espaço, escolher enfoques e noticiar características e obras da literatura, esse tipo de publicação seleciona e recorta a história literária; evidencia seus valores estéticos e de grupos de intelectuais que o utilizam como espaço para veiculação de suas ideias”.

O conceito se constitui, dessa maneira, como uma “plataforma interpretadora” sobre a cultura e o pensamento de uma época. Periódicos como \textit{O Estado de São Paulo} e \textit{Jornal do Brasil} foram pioneiros desse tipo de publicação.

\textcite{lima2013} destaca a reorganização necessária dos suplementos frente aos processos que envolveram a indústria cultural. Por exemplo, nas décadas de 1950-60, a definição de cultura popular estava ligada a centros populares, ao contrário da década de 1970, que relaciona o popular ao mercado, a programas televisivos, voltados ao consumo da massa, isto é, à ótica do entretenimento. Todo esse contexto se ampliou ainda mais na década de 1990, com o início da televisão por assinatura e a proliferação da internet.

Acompanhando o contexto histórico, a partir da década de 1970, já é possível afirmar que temos os cadernos de cultura, e estes vivem uma nova fase, demarcada pelas alterações em produção e conteúdo. O modelo seguido pelos segundos cadernos consolida-se na década de 1980, período em que os jornais circulam com encartes diários sobre cultura \cite{gadini2009}. Nesse cenário, ocorreram modificações no design gráfico, o que valoriza as imagens em \textit{layouts} mais ousados e leves. Como exemplo, \textcite[p. 192]{golin2010} citam o suplemento diário do jornal \textit{Folha de São Paulo}, a \textit{Ilustrada}: “[o caderno] traduziu uma estratégia mercadológica que apresentava os bens culturais com base em critérios como grandes audiências, internacionalização, serviço e hibridações entre o erudito e o popular”.

Assim, em um contexto marcado pela crise financeira das empresas jornalísticas, os cadernos passam a ser constituídos por informações sintonizadas com a agenda televisiva “e do mercado em detrimento do caráter crítico e analítico dos assuntos artístico-culturais, frequente em períodos anteriores” \cite[p. 191-192]{golin2010}. Com equipes menores nas redações, há a diminuição de espaços para o ensaio literário. Todos esses fatores contribuem para a configuração do que \textcite[p. 191-192]{golin2010} classificam como uma fase centrada na divulgação e celebrismo: “A aposta no Jornalismo de serviço privilegiou o espaço dedicado aos roteiros de programação, tendo como parâmetro o consumo do leitor de classe média urbana”. Marcados por essa transição, os cadernos fortalecem a cobertura televisiva e de programas de lazer, mantendo-se pautados pela agenda midiática.

Nesse mesmo contexto de crise, surgem, a partir da segunda metade do século XX, as fusões entre grupos internacionais de mídia. “Os veículos tradicionais, além de serem afetados pela concentração do capital das empresas, foram prejudicados pelo incremento das novas tecnologias da comunicação” \cite[p. 28]{lima2013}. Os textos dos periódicos moldam-se, assim, de acordo com uma linguagem capaz de ser consumida em escala industrial, objetivando a mecanização da produção jornalística.

As demarcações históricas e sociais, atribuídas por \textcite{lima2013,marcondes2002} ao desenvolvimento do Jornalismo, apontam para os processos de transformação, tanto do Jornalismo, quanto da Literatura, ao longo dos anos. Diante de uma perspectiva que se relaciona à Sociedade da Informação \cite{castells2001}, ou ainda, como denomina \textcite{jenkins2008convergence}, “cultura da informação”, ou \textcite{shirky2011}, “cultura participativa”, entendemos a importância de se verificar as modificações ocorridas nas relações entre os campos jornalístico e literário no contexto atual. A despeito das transformações no modo de fazer e conceber o Jornalismo cultural, no intervalo de 67 anos desde as primeiras publicações, estas acompanharam os suplementos, seja na distribuição de temáticas, seja no número de páginas, presença (ou ausência) de críticos literários. Os veículos de comunicação tiveram que se adaptar, portanto, às modificações de ordem tecnológica e estrutural.

Não obstante, aquele antigo Jornalismo cultural se apropriar de obras literárias e transformá-las em objetos de trabalho foi a concepção adotada por um veículo contemporâneo, utilizando as particularidades do meio digital, para discutir a Literatura e suas temáticas, que teve sua primeira publicação circulando em julho de 2015. \textit{Peixe Elétrico} se intitula como revista cultural, sendo disponibilizado por aplicativos, via internet, em formato de \textit{e-book}. O \textit{webreview} é produzido pela E-galáxia, uma editora especializada no segmento de livros digitais e possui periodicidade bimestral. Para acessá-lo, é possível baixar uma amostra gratuita, para conhecer o periódico, ou adquirir o \textit{e-book} pelo valor de R\$9,99, por meio de aplicativos como Amazon, Apple, Google Play, Kobo, Livraria Cultura e Saraiva.

Contemplando a evolução da divulgação dos espaços dedicados a análise de produtos literários, observamos que as transformações dos veículos, ao longo dos anos, inserem-se em adaptações frente ao horizonte mercadológico. Além disso, as aproximações e o hibridismo da Literatura e do Jornalismo sinalizam alternativas para o grande campo do Jornalismo cultural. Desse modo, no próximo item, contextualizaremos e caracterizaremos o \textit{corpus} desta reflexão, para observar as estratégias editoriais e gráficas adotadas pelo \textit{Peixe Elétrico}.

\section{\textit{Peixe Elétrico}: um \textit{webreview} pioneiro?}\label{sec-conduta}
O \textit{corpus} dessa reflexão se compõe pelas primeiras seis edições do veículo. Tivemos problemas técnicos para a compra, então, ficamos com as edições \#01 e \#06, na Saraiva e as \#02, \#03, \#04 e \#05, adquiridos da Amazon. Por esta editora, não tivemos qualquer problema técnico. Todos os exemplares carregaram de forma rápida, não travaram e, sobretudo para a leitura, a página da Amazon é mais eficiente. O relato em relação às dificuldades técnicas encontradas com o \textit{Peixe Elétrico} auxilia-nos a entender o contexto utilizado pelo veículo para chegar até seus leitores. O pioneirismo de uma publicação que se intitula como revista cultural, e é pioneira nesse formato, quando se fala em Jornalismo Cultural, acaba sendo prejudicado, pelo percurso tecnológico.

Transitando pela inovação, o veículo insere, nas seis edições analisadas, conteúdos multimídia para agregar ainda mais subsídios ao processo de leitura. Mas, novamente, a tecnologia não contribui para esse processo. Muitos dos \textit{links} e conteúdos não estão disponíveis.

No que se refere aos traços da linguagem da web, para identificar as edições, o veículo emprega o símbolo da expressão \textit{hashtag} “\#”, antes do número da publicação. Por exemplo, no primeiro exemplar, é utilizado “\#01”; na edição dois, “\#02”, e assim, sucessivamente, até o número “\#06”. A \textit{hashtag} é comum entre os usuários-leitores e serve para categorizar os conteúdos das redes sociais. Por exemplo, ao publicar uma mensagem e utilizar a “\#”, imediatamente a expressão é transformada em um \textit{hiperlink}. \textcite{levy1999} define que um \textit{hiperlink} é um texto que interliga dois elementos em uma estrutura de dados, cuja principal característica é a interatividade. Assim, estará disponível para outros usuários-leitores acessarem e pode ainda ser indexada nas buscas do Google. Esse processo facilita a consulta temática das expressões antecedidas pelas \textit{hashtags}.

Os leitores-usuários de \textit{Peixe Elétrico} podem escolher um destes quatro canais para acessar o conteúdo da publicação. A utilização da “\#” denota que o veículo se insere na web 3.0, ou, como denomina \textcite[p. 25]{castells2001}, que o meio social e os ambientes tecnológicos se relacionam em um “padrão interativo”: “a tecnologia é a sociedade, e a sociedade não pode ser entendida ou representada sem suas ferramentas tecnológicas”. Entendemos que esse contexto das tecnologias, além das produções de conteúdos exclusivos para o ambiente on-line, teoricamente ampliam os recursos de interatividade.

Além da interatividade, as novas tecnologias da comunicação e da informação, junto com o jornalismo digital, produziram efeitos diversos dentro da organização e da atividade jornalística. \textcite[p. 94]{pavlik2011} entende que o aspecto digital “trouxe mudanças radicais para o jornalismo e as instituições que ele serve”.

Voltando ao estilo que a publicação se propõe a seguir: no formato \textit{e-book}, assemelhando-se a um livro. Em grande parte dos textos, observamos citações no estilo acadêmico, ou seja, as falas externas não são indicadas entre aspas ou travessões – estilo comumente utilizado no âmbito jornalístico, mas sim, por meio do recuo.

Outra característica do veículo é inserir, no lugar da linha de apoio (que geralmente contém as principais informações do lead), epígrafes. Também identificamos uma característica comumente utilizada no âmbito acadêmico: a inserção de notas de rodapé, de fim de texto, e as referências de autores inseridos nos textos.

O projeto gráfico está diretamente interligado às escolhas editoriais do \textit{Peixe Elétrico}. Se, nas indicações de \textit{layout}, a publicação se propõe a seguir o estilo de livro acadêmico, nos conteúdos, essa também é a maior evidência. As imagens demonstram que o estilo jornalístico, comumente utilizado pelos veículos de comunicação, não estão presentes no \textit{webreview}, tais como os exemplos já mencionados.

Outro recurso presente na maioria das matérias jornalísticas, é o relato de pessoas entrevistadas – as fontes – que fornecem maior credibilidade às informações. Geralmente, os depoimentos são inseridos no veículo por meio de aspas ou travessões. Nesse sentido, \textit{Peixe Elétrico} desenvolve as matérias com fontes que vão desde documentos, dados ou excertos de livros/textos, até discursos de pessoas.

Os textos do \textit{corpus}, que mais se aproximam da característica jornalística são os editoriais. Todos eles, além de serem publicados nos aplicativos de leitura, também estão disponíveis no site de \textit{Peixe Elétrico} (disponível em: \url{https://www.peixe-eletrico.com/editorial}), possibilitando pleno acesso a qualquer leitor-usuário.

Outro ponto que chama a atenção, em \textit{Peixe Elétrico}, quanto ao acesso à informação por qualquer leitor-usuário, é que não há uma convergência midiática dentro dos conteúdos. \textcite{jenkins2009} destaca a convergência midiática, não como a substituição por outros meios digitais mais emergentes, mas sim, em como se modificam pela absorção de novas tecnologias. Com isso, deve-se entender que a convergência não altera apenas os meios de divulgação, mas também como os leitores-usuários processam as notícias. \textcite{jenkins2014} indicam que os públicos, antes vistos como passivos frentes às informações, tornam-se agentes ativos de configurações e compartilhamento de conteúdos midiáticos. No entanto, mesmo frente a esse cenário emergente na multiplicação de conteúdos, \textit{Peixe Elétrico} não apresenta as características mencionadas por \textcite{jenkins2009} de convergência midiática, pois não há uma transposição para outros meios de divulgação, nem mesmo, em plataformas de redes sociais, como o Instagram. A editora que publica as edições do \textit{webreview} – E-galáxia – possui uma página nesta rede social, e esporadicamente, apenas transpõe os conteúdos de \textit{Peixe Elétrico}, sem nenhuma alteração, visando as características específicas da rede social.

Visualizando esse cenário, \textcite{canavilhas2011} reconhecem que uma parte considerável de empresas de comunicação que possuem conteúdos jornalísticos em plataformas móveis ainda não explora suas potencialidades e sobretudo, o leque que se apresenta, quando o assunto é interação com os públicos e aumento considerável de audiência.

Com a formatação dos princípios desses espaços, conseguimos visualizar as características editoriais pelas quais o veículo se propõe a transitar. Com as indicações do posicionamento de \textit{Peixe Elétrico}, seja frente a convergência midiática ou às suas temáticas, é possível visualizar, nos editoriais, as estratégias do veículo. Iremos abordá-las à medida em que apresentarmos as suas caracterizações.

O editorial número \#01 expressou que a revista não terá limites de espaço. Nesse sentido, observamos, nas seis publicações, essa particularidade em relação à predominância de matérias longas, havendo também uma variação na quantidade de textos. O \#01 foi o único que apresentou apenas oito textos, enquanto as demais edições variam entre 12 e 15.

Além disso, há várias matérias extraídas de livros, de conferências, de palestras e de eventos ocorridos em universidades. Como exemplo, no \#01, “A musa falida”, de Alcir Pécora, “é resultado da transcrição da fala proferida pelo autor por ocasião da abertura do ano letivo 2014-2015 da Faculdade de Letras da Universidade de Coimbra” \cite[n.1, p. 107]{pecora2015musa}. Já o texto da edição \#03, “A pele de cebola”, integra o livro O impostor, “tradução de Bernardo Azjenberg, com publicação pela Biblioteca Azul, para novembro de 2015” \cite[n. 3, s/p]{cercas2015pele}. Utilizaremos essa forma de citar, pois somente a indicação do ano poderia confundir o leitor, já que há mais de uma edição por ano. Também “Atenção e indiferença: o sentido em Machado de Assis”, na mesma edição, é a versão de um capítulo inédito de Futuro abolido – Machado de Assis e o Memorial de Aires: Tempo, história e literatura. Em todas as edições, há, pelo menos, dois textos que são resultado de outras produções que são reproduzidos pelo \textit{Peixe Elétrico}.

A abordagem mais densa e “a formação de um público autônomo e independente”, como propõe o editorial do \#04, é justificada com os números. Nas seis edições, observamos que há o predomínio do gênero artigo, seguido pelas informações noticiosas. Consideramos que cada uma das edições de \textit{Peixe Elétrico} possui duas informações noticiosas: o editorial e a última página; além da contracapa, cujos textos explicam a fonte das fotografias inseridas ao longo das edições. Nas demais páginas, as matérias atendem às características da análise mais crítica das informações, e até ensaios com maior profundidade de reflexão, identificadas com o que compreendemos como o gênero artigo.

Para melhor exemplificar esses números, destacamos o texto intitulado “No espelho que o terror oferece”, de Gabriel Ferreira Zacarias, pertencente ao \#04. A matéria aproxima-se da descrição do editorial do \#02, quando menciona que quer tratar, por meio de seus autores, dos significados do mundo contemporâneo. \textcite[n. 4, s/p]{zacarias2016terror} cita exemplos contemporâneos que resultaram em violência física e moral: “A cada novo ato de terrorismo, a agressão é sempre apresentada como exógena, como barbárie. A violência pertence ao Outro, dizem, é exterior à nossa cultura, está em oposição a ela”. O autor menciona essa reflexão, tanto como aquela dada ao fato da derrubada das torres do World Trade Center, em 2001:

\begin{quote}
    Pouco importava a complexidade do ataque e os meios extremamente modernos com que foi executado; pouco importava que alguns dos terroristas tivessem vivido e estudado na Europa e nos EUA; a versão oficial foi a de que o Ocidente estava sendo ameaçado por homens escondidos em cavernas nas profundezas dos desertos afegãos. Uma ladainha semelhante foi evocada em janeiro do ano passado, quando o semanário Charlie Hebdo foi alvo de um ataque que dizimou sua redação. Mais uma vez, pouco importou que os assassinos fossem ambos nascidos e criados na França. Eram muçulmanos e, portanto (!), contrários à “cultura francesa”, que teria por um de seus pilares a “liberdade de expressão”, valor que, de uma hora para outra, parecia ter se encarnado no Charlie Hebdo (semanário de cuja existência pouquíssimos ainda se lembravam, à exceção, claro da Al Qaeda, que havia jurado seus desenhistas de morte por terem caricaturado o profeta Maomé) \cite[n.04, s/p, grifos nossos]{zacarias2016terror}.
\end{quote}

Utilizando os exemplos das mortes no ataque às torres gêmeas, de 2001, e ao jornal Charlie Hebdo, \textcite[n. 04]{zacarias2016terror} constrói seus argumentos, por meio de outros autores e textos que abordam a violência em distintos âmbitos, desde a obra Sociedade do espetáculo, de Guy Debord, até os textos de filósofos como Marx, Kant e Descartes.

O tom reflexivo e as referências a diferentes e amplos conteúdos estão presentes em praticamente todas as edições de \textit{Peixe Elétrico}. Por exemplo, no \#03, identificamos reflexões sobre Roland Barthes, nas duas matérias escritas por Beatriz Sarlo: “O romance de Barthes” e “Barthes, leitor de Loyola”, em que a autora traça comparações entre Santo Inácio de Loyola e Barthes. Também nesta edição, \textit{Peixe Elétrico} traz a abordagem da violência, dessa vez, em texto em primeira pessoa, de Selva Amada. A autora conta, pela sua perspectiva, histórias de mulheres violentadas e mortas, refletindo sobre o machismo.

A violência, sobretudo contra a mulher, é um dos assuntos recorrentes das demais edições. A título de exemplo, no \#01, Angela Davis reflete sobre o papel feminino no século 21, no texto “O legado da escravidão: parâmetros para uma nova condição da mulher”.

Já em relação aos gêneros literários abordados nas seis edições, identificamos a predominância do que intitulamos “Outras áreas do conhecimento” e nestas, do que intitulamos como “históricos”. Enquadram-se nesta denominação textos que possuam como pano de fundo a vida e a obra de escritores, criações artísticas, movimentos literários, análise e resenha de obras – todas cotejando fatos históricos e sociais. Além dessa abordagem oferecer bases para um debate, como indica o editorial do \#06, ao mesclar fatos históricos e sociais, o veículo utiliza diferentes formatos de textos. No \#03, Felipe Charbel, com a matéria “Diário de uma releitura”, conta sua história, em primeira pessoa, como se o leitor estivesse em contato com um diário. O autor utiliza obras de cunho histórico para refletir sobre os fatos da contemporaneidade. Também em primeira pessoa, o texto do \#02, “Uma verdade revolucionária”, de Lina Meruane, expõe a história pessoal da autora, apontando o preconceito que ela sofreu em países como Israel e Islã. Com o relato da autora, sentimo-nos atingidos pelas diferenças culturais. Notamos que os textos com caráter histórico servem, em sua maioria, para inserir temáticas que reflitam sobre o meio social. Para pensar as questões contemporâneas, os autores utilizam o subsídio histórico.

Além disso, outra particularidade evidente, em praticamente todos os editoriais, é a inserção sociológica e histórica de \textit{Peixe Elétrico}, tomando posicionamento sobre os temas em circulação. No \#02, os editores evidenciam que não farão nenhum acordo: nadarão contra a corrente, justamente por se identificarem como uma publicação contraideológica. No \#04, a metáfora da expressão “meninos mimados” se refere ao propósito de construir espaços alternativos, que valorizem a formação de um público autônomo e independente. Com essas indicações, o \#05 utiliza um tema contemporâneo à edição, declarando-se contrário ao impeachment da então presidente Dilma Rousseff e, por fim, no \#06, declara-se favorável ao prêmio Nobel de Literatura concedido a Bob Dylan.

O posicionamento proposto pelo veículo torna-se perceptível em boa parte dos textos. Como exemplo, no \#02, Bruno Rodrigues, em “A educação pela pedrada”, faz críticas ao mercado editorial e suas influências sobre a reflexão literária:

\begin{quote}
    Se o mercado quer estabelecer um meio conservador, é preciso negar o mercado, fazer um novo mercado. Para alguns, a escritura é passatempo, hobby. Para mim, para nós, ela é tudo que restou. O embate da escritura contra o poder, em variados graus, se inicia com a própria raiz do pensamento ocidental, quando Platão nos expulsou de sua República ideal. É uma luta milenar \cite[n.2, s/p]{rodrigues2015educacao}. 
\end{quote}

Outra particularidade de \textit{Peixe Elétrico}, é a continuidade dos textos. Essa característica também está realçada nos editoriais. Desde o \#01, os editores alertam que irão tecer as redes que os textos apontarem. Assim, a revista já surgiu com essa perspectiva para os leitores.

A continuidade ocorre em páginas consecutivas com os mesmos autores e assuntos, inclusive nas citações entre estes, nas suas matérias. No \#01, o primeiro texto, “Os livros da minha vida”, de Ricardo Piglia, é uma autobiografia do escritor. Já “A arte de ler” de Juan Villoro, conta a história de Ricardo Piglia. No \#04, “Na mira da teoria”, de Boris Groys, é sucedido por texto de Marcelo Moreschi, “Um convite à tumba de Boris Groys”, que fala justamente sobre Boris Groys.

Ainda que trazendo vários textos de autores com origem internacional, como aponta o editorial \#01, a predominância é de escritores nacionais. A nacionalidade brasileira é também outra marca dos autores abordados nos textos; no entanto, a diferença com os estrangeiros é pequena.

Como os números indicam, observamos que, nas edições \#03 e \#06, há predominância de autores estrangeiros em relação aos nacionais. Estes dados são confirmados pelas abordagens das temáticas. No editorial \#01, fica evidente que \textit{Peixe Elétrico} surgiu sob a inspiração de duas revistas, a estadunidense \textit{New Left Review} e a argentina \textit{Punto de Vista}. A revista \textit{New Left Review} foi fundada em 1960, no Reino Unido. Seu principal viés sempre foi a política, em suas raízes marxistas. Inicialmente, Stuart Hall esteve à frente do periódico. Já \textit{Punto de Vista}, foi criada em 1978, na Argentina, e contempla temáticas mais abrangentes, como a literatura, história, sociologia e crítica literária. Esses dois periódicos serviram de inspiração para \textit{Peixe Elétrico}. Da primeira, a relação com o viés esquerdista e de \textit{Punto de Vista}, a densidade e a extensão dos textos.

A partir destas publicações, os editores transitam entre escritores de renome internacional. Fato que se confirma também pelo editorial \#03, quando os editores relatam a viagem até a Feira do Livro em Frankfurt, para reafirmar outras parcerias com o veículo de comunicação.

Com a explicitação do editorial, notamos que os editores buscam a inspiração internacional – como a do intelectual norte-americano Fredric Jameson, que assina a matéria de destaque do \#02 – mas, ao mesmo tempo, aproximam essas temáticas ao âmbito nacional, pelo critério de proximidade com os leitores.

Se, como os editores afirmam, “uma coisa leva à outra”, os números confirmam a veiculação de textos de autores contemporâneos e novos, relacionando a identidade do veículo à sua inserção na contemporaneidade.

Embora os números apontem para o predomínio de escritores novos e contemporâneos, notamos que o veículo valoriza o prestígio intelectual dos autores. Em todas as edições, como já afirmado, ao final dos textos, há uma mini-biografia destes autores, e todos eles, ou possuem formação acadêmica como Doutorado e Mestrado, ou são (re)conhecidos por suas obras no meio acadêmico e literário. Como exemplo, o texto do \#03, “Laudato Si: uma encíclica antissistema”, de Michael Lowy.
Além do prestígio intelectual, outro traço valorizado por \textit{Peixe Elétrico} é que, apesar da presença de autores canônicos em todas as edições, os escritores contemporâneos são mais numerosos que os clássicos.

Quanto aos autores abordados nos textos, é possível notar a despreocupação do veículo para com escritores cujas obras pertençam ao cânone, já que indicam sobretudo autores contemporâneos e novos, e não clássicos. Em \textit{Peixe Elétrico} não houve repetições de colaboradores. Salvo os editores, que assim nomeamos por escreverem todos os editoriais das publicações. Quando os autores aparecem, mais de uma vez, isso ocorre na mesma edição.

A partir das características presentes nas edições de \textit{Peixe Elétrico} e que compõem este \textit{corpus} de análise, seguimos, com as considerações acerca do veículo e do campo do Jornalismo Cultural.


\section{Considerações finais}\label{sec-fmt-manuscrito}
A partir das análises realizadas nas seis primeiras edições de \textit{Peixe Elétrico}, é possível afirmar que o veículo consolida-se na transição entre a inovação do ambiente tecnológico de publicação e a tradição, que contempla as características dos Suplementos Literários. Nesse cenário, reconhecemos aspectos a serem considerados, sobretudo quanto aos problemas técnicos dos aplicativos de leitura e dos \textit{links}, que não funcionam plenamente para estabelecer a interatividade com os leitores-usuários. No que tange à tradição, a falta de recursos gráficos e elementos de atratividade dificultam processos de leitura mais fluentes e prazerosos. O elevado número de páginas, associado às características predominantes do estilo \textit{e-book}/livro também não cultivam o estabelecimento de uma nova proposta de Jornalismo cultural.

Essa prerrogativa se confirma, ainda, se considerarmos o cenário emergente e atual que o impacto das mudanças tecnológicas provocou nos comportamentos dos leitores e modos de produção do jornalismo digital. Ainda que \textcite{canavilhas2011} destaque que o texto é o conteúdo mais utilizado no ciberjornalismo, atentamos que há outras particularidades que precisam ser inseridas nesse contexto de produção. \textcite{bardoel_deuze} enfatizam quatro características: \textit{Hipertextualidade} (presença de \textit{links}), \textit{Multimidialidade} (combinação de textos, sons e imagens), Personalização de conteúdo e \textit{Interatividade}. \textcite{palacios1999} acrescenta a Memória e, com o passar dos anos e estudos, destacam-se ainda a \textit{Instantaneidade} e a \textit{Ubiquidade} (o acesso a conteúdos de qualquer lugar e plataforma). Se olharmos atentamente para cada uma delas, em \textit{Peixe Elétrico} percebemos somente a \textit{Hipertextualidade} e a \textit{Memória}, em suas edições.

Outro ponto a discutir é o conjunto de dados quantitativos de recepção e alcance de \textit{Peixe Elétrico}. Sobre a recepção, analisamos a publicação considerando o valor de cada edição, que varia de $R$ 14,90$ até $R$ 55,00$ para uma edição especial (valores atualizados em abril de 2025). Observando de forma generalista, a média salarial da população brasileira, índices de alfabetização e leitura, logo identificamos que o Webriew se direciona para uma camada da população que é minoria, ou seja, com maior poder aquisitivo. Somado a isso, ainda consideramos a Ubiquidade, que dificulta o acesso às edições.

Outra característica predominante é que \textit{Peixe Elétrico} valoriza contextos mais gerais e menos bairristas, já que o veículo preconiza o debate de forma ampla às temáticas. Isso se torna positivo, por abranger um maior número de usuários-leitores, mas, por outro lado, não há uma identificação personalizada com os assuntos abordados.

Conforme já mencionamos, a tendência em abordar aspectos biográficos, nos cadernos, ainda é uma prática recorrente. O espaço do jornal é visualizado como um ambiente de reconhecimento e legitimação. Assim, grupos de intelectuais utilizam os cadernos para divulgar suas ideias \cite{lima2013}. \textit{Peixe Elétrico} possui uma valorização dos escritores e essa relação fica ainda mais evidente, porque, em todos os textos publicados, há uma mini-biografia do autor que os produziu. Todos os escritores possuem uma notoriedade no mundo das letras, estando vinculados à área acadêmica ou com reconhecimento literário pela publicação de suas obras.

Identificamos nas edições um total de 61 escritores, destes, 17 mulheres e 44 homens. Fica evidente, a partir desses números, a escassez da presença feminina. A pesquisa, realizada por Regina Dalcastagnè, a qual contou com um recorte temporal de romances publicados ao longo de 15 anos, entre 1990 e 2004, pelas editoras Companhia das Letras, Rocco e Record, demonstrou a restrição ao gênero feminino, isso porque, nas editoras comerciais, somente 30\% das publicações têm mulheres como autoras \cite{dalcastagnè2005}. Observamos que mesmo \textit{Peixe Elétrico} sendo um veículo contemporâneo, não guardam um equilíbrio de publicações entre mulheres e homens. Pelos números apresentados na pesquisa, verificamos que a representatividade feminina ainda é restrita no âmbito literário, tanto na década de 1960, quanto na contemporaneidade.

Voltando aos espaços dos escritores, percebemos que em \textit{Peixe Elétrico} não há uma repetição de escritores e isso se deve ao fato do veículo explorar diferentes temáticas por meio de distintos escritores. Além disso, os usuários-leitores necessitam de um conhecimento prévio sobre certas temáticas para melhor entender os textos divulgados pelos veículos. Isso ocorre graças à densidade e à profundidade dos conteúdos desenvolvidos e pela forma com que são expostos.

Por fim, entendemos que o \textit{webreview} \textit{Peixe Elétrico}, pioneiro na produção de cultura para a plataforma digital, não traz inovações, a não ser em sua forma de publicação \textit{on-line}. Discutir Literatura, em textos que ultrapassem 50 páginas, na era das imagens e da interação audiovisual, não pode ser considerada uma forma de inovação. \textit{Peixe Elétrico} poderia melhor explorar a tecnologia para reestruturar seus conteúdos, provocar a interatividade, oferecer materiais que reforçassem a densidade e a reflexividade social, mas que fossem atrativos aos leitores contemporâneos.

A Literatura não pode ser vista como uma área do conhecimento que, pela sua extensão temática, deva ser rígida. Pelo contrário, ela pode ser discutida de uma maneira leve e atrativa. Além disso, ela não pode ser encarada a partir de uma concepção demasiadamente elitista. Nesse sentido, a internet pode ser uma aliada dessas produções. Conteúdos multimídia e linguagem audiovisual podem servir de suporte às reflexões sobre os textos.

Apesar das mudanças de suporte, da modernização das mídias, no que se refere aos aparatos tecnológicos, 50 anos depois, os suplementos continuam basicamente com o mesmo formato. O tipo de conteúdo proposto e a divisão de editoriais do Jornalismo Cultural tiveram apenas mudanças de suporte – do impresso ao digital – e de \textit{layout}, com a linguagem verbal predominante, mas também com a utilização da linguagem não-verbal, sobretudo em mecanismos de interação, com o uso de \# (\textit{hashtags}). Logo, visualizamos uma tendência à manutenção do estilo de se fazer Jornalismo cultural no Brasil.


\printbibliography\label{sec-bib}
% if the text is not in Portuguese, it might be necessary to use the code below instead to print the correct ABNT abbreviations [s.n.], [s.l.]
%\begin{portuguese}
%\printbibliography[title={Bibliography}]
%\end{portuguese}


%full list: conceptualization,datacuration,formalanalysis,funding,investigation,methodology,projadm,resources,software,supervision,validation,visualization,writing,review
\begin{contributors}[sec-contributors]
\authorcontribution{Larissa Bortoluzzi Rigo}[formalanalysis,investigation,methodology]
\authorcontribution{Antonio Hohlfeldt}[conceptualization,datacuration]
\end{contributors}


\end{document}

