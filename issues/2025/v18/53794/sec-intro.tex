\section{Introduction}\label{sec-intro}

Technology-assisted interpreting has experienced exponential growth in
recent years, reflected in the dizzying progress in the development of
information and communication technology (ICT) tools and resources
\cite{gutierrezArtacho2016,mezcua2019}. These innovative
technologies have greatly facilitated the interpretation and
comprehension of texts in different linguistic contexts \cite{olallaSoler2015}. In this sense, computer-assisted interpreting (CAI)
has become crucial in an increasingly globalised and connected society,
playing a pivotal role in breaking down language barriers and
facilitating effective communication in an environment characterised by
cultural and linguistic diversity \cite{mellinger2019,li2021}. In
response to the growing demand for instant online communication and the
need to overcome language limitations in a global environment, CAI has
established itself as an infallible tool for a wide range of
applications, from interpreting international business meetings to
translating multimedia content in real time \cite{fantinuoli2017a,alcaidemartinez2021}.

The integration of technology has revolutionised the way individuals
interact with language, opening new possibilities for increasing the
efficiency and accuracy of text interpretation, both in real time and
asynchronously \cite{gaber2023a,ramirezRodriguez2023}. The development
of ICT tools has led to improvements in natural language processing,
machine translation, speech recognition and other related technologies,
resulting in significant advances in text interpretation \cite{valeroGarces2024}. These tools have been particularly useful in overcoming language
barriers in multicultural environments, facilitating communication
between speakers of different languages and promoting intercultural
understanding \cite{perez2020}.

The use of CAI does, however, present certain difficulties. In the
phraseological domain, understanding idiomatic expressions remains a
significant challenge, as these linguistic constructions can be
particularly difficult to interpret accurately due to their cultural
embeddedness \cite{corpaspastorgaber2021,ramirezRodriguez2022}.
Cultural and contextual differences can lead to inaccurate or ambiguous
translations of these expressions, resulting in misunderstandings and
communication problems. In addition, CAI is often based on pre-defined
algorithms and databases, which can limit its ability to adapt to new
idioms and emerging expressions \cite{ortigoza2024}. This problem
identified in the interpretation of idiomatic expressions by CAI adds to
the previously mentioned challenges in technology-assisted phraseology,
where accuracy and fluency in the interpretation of texts in different
linguistic contexts are key to effective communication. The complexity
of idiomatic expressions, such as idioms, and their culturally
contextual nature require new approaches and specialised tools to
improve the interpretation of such expressions, which represents a
relevant area of research in the development of technology-assisted
phraseology.