\documentclass[english]{textolivre}

% metadata
\journalname{Texto Livre}
\thevolume{18}
%\thenumber{1} % old template
\theyear{2025}
\receiveddate{\DTMdisplaydate{2024}{7}{31}{-1}}
\accepteddate{\DTMdisplaydate{2024}{11}{29}{-1}}
\publisheddate{\DTMdisplaydate{2025}{4}{17}{-1}}
\corrauthor{Pablo Ramírez Rodríguez}
\articledoi{10.1590/1983-3652.2025.53794}
%\articleid{NNNN} % if the article ID is not the last 5 numbers of its DOI, provide it using \articleid{} commmand 
% list of available sesscions in the journal: articles, dossier, reports, essays, reviews, interviews, editorial
\articlesessionname{dossier}
\runningauthor{Ramírez Rodríguez y Antonov}
%\editorname{Leonardo Araújo} % old template
\sectioneditorname{Hugo Heredia Ponce}
\layouteditorname{João Mesquita}

\title{New challenges and opportunities in technology-assisted phraseology interpreting: the case of Yandex live stream translation}
\othertitle{Novos desafios e oportunidades na interpretação da fraseologia assistida pela tecnologia: o caso da tradução de vídeo ao vivo do Yandex}
%\othertitle{Nuevos retos y oportunidades en la interpretación de la fraseología asistida por la tecnología: el caso de la traducción de vídeo en directo de Yandex}

\author[1]{Pablo Ramírez Rodríguez~\orcid{0000-0002-6168-3736}\thanks{Email: \href{mailto:pablorarod@gmail.com}{pablorarod@gmail.com}}}
\author[2]{Evgeniy Antonov~\orcid{0000-0003-1498-9131}\thanks{Email: \href{mailto:eantonov@kaf65.ru}{eantonov@kaf65.ru}}}
\affil[1]{Universidad de Granada, Granada, Spain.}
\affil[2]{National Research Nuclear University, MEPhI, Moscow, Russia.}

\addbibresource{article.bib}
%\usepackage{easyReview}
% \usepackage[utf8]{inputenc}  % UTF-8 encoding for Unicode characters
% \usepackage[T2A]{fontenc}    % Cyrillic font encoding for Russian characters

\setotherlanguage{russian}   % Define another language
%\usepackage{fontspec}      % For font selection
%\newfontfamily\cyrillicfont[Script=Cyrillic]{Charis SIL}
%\newfontfamily\cyrillicfont{Liberation Serif}
\newfontfamily\cyrillicfontsf{Liberation Sans}

\begin{document}
\maketitle
\begin{polyabstract}
%\begin{spanish}
%\begin{abstract}
%  La interpretación asistida por ordenador (IAO) se ha convertido
%  en un elemento esencial en un mundo cada vez más globalizado e
%  interconectado. Con la creciente demanda de comunicación instantánea en
%  línea y la necesidad de superar las limitaciones lingüísticas en un
%  contexto global, la IAO se ha convertido en una valiosa herramienta para
%  una amplia gama de aplicaciones. Sin embargo, la IAO no está exenta de
%  dificultades. La interpretación de expresiones idiomáticas sigue siendo
%  un obstáculo importante, ya que estas construcciones lingüísticas pueden
%  ser especialmente difíciles de interpretar con precisión debido a su
%  naturaleza culturalmente arraigada. En este contexto, el objetivo de
%  este artículo es abordar el problema de la interpretación de expresiones
%  idiomáticas, en nuestro caso de 52 locuciones verbales en el marco de la
%  IAO, centrándonos en la combinación lingüística español-ruso. El
%  objetivo es analizar cómo esta tecnología responde a los retos de la
%  fraseología idiomática y cómo influye en una comunicación intercultural
%  precisa y eficaz. Para lograr este objetivo, se ha aplicado una
%  metodología exhaustiva que combina el análisis lingüístico con
%  observaciones prácticas de situaciones de interpretación asistida por
%  tecnología en tiempo real utilizando el traductor de voz en directo y
%  los contenidos multimedia proporcionados por Yandex. Los resultados de
%  esta investigación aportan una comprensión más profunda de cómo la
%  tecnología de la traducción y la interpretación aborda los retos de la
%  expresión idiomática, al tiempo que proporcionan una visión crítica de
%  la eficacia y las limitaciones de las soluciones tecnológicas en el
%  ámbito de la comunicación intercultural.
%  
%
%
%\keywords{IAO \sep Fraseología \sep Locuciones verbales \sep Traducción en
%directo \sep Interpretación automática}
%\end{abstract}
%\end{spanish}

\begin{english}
\begin{abstract}
Computer-assisted interpreting (CAI) has become an essential
element in an increasingly globalized and interconnected world. With the
growing demand for instant online communication and the need to overcome
language limitations in a global context, CAI has become a valuable tool
for a wide range of applications. However, the CAI is not without its
challenges. The interpretation of idiomatic expressions remains a
significant barrier, as these linguistic constructs can be particularly
difficult to interpret accurately due to their culturally embedded
nature. In this context, the objective of this article is to address the
problem of interpreting idiomatic expressions, in our case 52 verbal
idioms within the framework of CAI, focusing on the Spanish-Russian
language combination. The aim is to analyze how this technology meets
the challenges of idiomatic phraseology and how it influences accurate
and effective intercultural communication. To achieve this goal, a
comprehensive methodology has been applied, combining linguistic
analysis with practical observations of real-time technology-assisted
interpreting situations using the live on-air speech translator and
multimedia content provided by Yandex. The results of this research
provide a deeper understanding of how translation and interpreting
technology addresses the challenges of idiomatic expression, while also
providing critical insight into the effectiveness and limitations of
technological solutions in the field of intercultural communication.

\keywords{CAI \sep Phraseology \sep Verbal idioms \sep Live stream translation \sep
Automatic interpreting} 
\end{abstract}
\end{english}

\begin{portuguese}
\begin{abstract}
  A interpretação assistida por computador (IAC) tornou-se um
  elemento essencial em um mundo cada vez mais globalizado e
  interconectado. Com a crescente demanda por comunicação instantânea
  online e a necessidade de superar as limitações linguísticas em um
  contexto global, a IAC se tornou uma ferramenta valiosa para uma ampla
  gama de aplicações. No entanto, a IAC não está isenta de dificuldades. A
  interpretação de expressões idiomáticas continua sendo um obstáculo
  significativo, pois essas construções linguísticas podem ser
  especialmente difíceis de interpretar com precisão devido à sua natureza
  culturalmente enraizada. Nesse contexto, o objetivo deste artigo é
  abordar o problema da interpretação de expressões idiomáticas, no nosso
  caso de 52 locuções verbais no âmbito da IAC, focando na combinação
  linguística espanhol-russo. O objetivo é analisar como essa tecnologia
  responde aos desafios da fraseologia idiomática e como influencia uma
  comunicação intercultural precisa e eficaz. Para alcançar esse objetivo,
  foi aplicada uma metodologia exaustiva que combina a análise linguística
  com observações práticas de situações de interpretação assistida por
  tecnologia em tempo real, utilizando o tradutor de voz ao vivo e os
  conteúdos multimídia fornecidos pela Yandex. Os resultados desta
  pesquisa proporcionam uma compreensão mais profunda de como a tecnologia
  de tradução e interpretação enfrenta os desafios da expressão
  idiomática, ao mesmo tempo que oferecem uma visão crítica da eficácia e
  das limitações das soluções tecnológicas no âmbito da comunicação
  intercultural.
  
\keywords{IAC \sep Fraseologia \sep Locuções verbais \sep Tradução em tempo
real \sep Interpretação automática}
\end{abstract}
\end{portuguese}


\end{polyabstract}

\section{Introdução}\label{sec-intro}

O processo de escrita é uma das atividades mais complexas que o ser
humano é capaz de realizar, em razão de vários fatores, a exemplo das
exigências feitas à memória e ao raciocínio durante o momento de
produção \cite{garcez2020}. São inúmeros os conhecimentos e habilidades que
precisam ser articulados e harmonizados para que o texto tome forma.
Tendo em vista esse seu caráter complexo, ainda são recorrentes falsas
crenças sobre a produção textual, que levam pessoas a acreditarem que
podem dominá-la a partir de ``dicas'' desvinculadas de seu contexto de
produção.

A ideia de que fórmulas pré-fabricadas e ``dicas'' isoladas são métodos
cabíveis no ensino de produção textual, apenas negligencia as etapas
necessárias que caracterizam um texto adequado conforme seu contexto de
produção \cite{garcez2020}. O processo de escrita é uma atividade que
carece de idas e vindas, pois deve admitir três grandes momentos que se
intercalam e devem ser compreendidos de modo indissociável: o do
planejamento, o da escrita propriamente dita e o da revisão \cite{antunes2005}. Enquadrar esse processo em uma perspectiva prescrita e linear
pode resultar em estudantes frustrados pela construção de textos
truncados e artificiais.

Essa realidade se agrava quando observamos o cenário acadêmico, em que
as exigências com relação a produções textuais se intensificam. As
expectativas quanto a essas produções não se limitam à utilização
adequada da norma-padrão ou a vocabulários específicos; expandem-se para
aspectos implícitos de produção que precisam ser considerados, como o
que pode ser dito, por quem, de que forma, sob que ponto de vista e
fundamentado em qual autor \cite{oliveira2024}.

Pensando na comunidade discursiva acadêmica, uma das produções textuais
mais demandadas em cursos de graduação da área de humanas é o artigo
acadêmico \cite{motta-roth2010}. Por ser um dos principais
veículos de divulgação científica, a circulação desse gênero na academia
é incontornável, sendo bastante exigido o seu consumo e produção por
parte de professores, estudantes e pesquisadores. Embora seja uma
produção essencialmente ligada ao meio universitário, sua feitura é
quase sempre exigida sem antes ser ensinada. Esse fato pode levar os
alunos a somarem suas dificuldades com o processo de escrita à
dificuldade de produzir um texto do qual desconhecem seu contexto de
produção, estrutura composicional e outras ``dimensões escondidas''
\cite{street2010} que perpassam a construção de um artigo.

Ao exigir do autor capacidade de síntese, descrição, análise e
argumentação, utilizando-se das convenções próprias à determinada área,
o artigo contempla informações geradas em pesquisas a serem submetidas a
apreciações públicas \cite{motta-roth2010}. Sua relevância remonta
à popularização da ciência que, por sua vez, possui a potencialidade de
descrever fenômenos sociais e até mesmo gerar algum impacto benéfico ao
público em geral.

A partir dessas pontuações, torna-se clara a importância de produzir
artigos acadêmicos e a responsabilidade do seu produtor de popularizar
os conhecimentos produzidos na esfera acadêmica. Quando essa tarefa de
produção precisa ser desenvolvida por graduandos e estes normalmente não
recebem orientação para tal, muitas vezes, recorrem a materiais digitais
sobre esse assunto, pois lhes propiciam as mais variadas estratégias de
ensino de acordo com o ritmo e as preferências do estudante \cite{falkembach2005}. Um fator que pode justificar essa recorrência é a facilidade de
acesso a plataformas digitais, que disponibilizam, na maioria das vezes
de forma gratuita, conteúdos digitais educacionais. Antigamente, os
estudantes consultavam manuais impressos que ensinavam a como produzir
textos acadêmicos, hoje, frente aos recursos tecnológicos, os locais de
aprendizagem se ampliam para a cibercultura. Como cibercultura,
compreendem-se vários ambientes da esfera digital que abrigam
informações, até mesmo os que simulam uma sala de aula a partir de
vídeos \cite{martins2018,rocha2005}.

Tendo a cibercultura se tornado uma potencializadora de novas abordagens
educativas, deve-se averiguar sua eficiência enquanto ferramenta de
ensino, a forma como se ensina determinados conteúdos, a exemplo da
produção textual de artigo acadêmico e seus aspectos constitutivos, foco
do presente estudo. Nesse sentido, traçamos dois objetivos para este
trabalho: identificar e analisar objetos de ensino explorados em
videoaulas sobre artigo acadêmico publicadas na plataforma YouTube.

Para tanto, organizamos este artigo em 5 seções, a saber: esta
introdução, contendo uma contextualização inicial sobre o objeto de
investigação da pesquisa, a problemática que o envolve e os objetivos
delineados; o embasamento teórico, no qual apresentamos os pressupostos
que fundamentam o estudo --- as práticas de ensino de Língua Portuguesa
em contexto didático-digital \cite{laurentino2023}, o artigo
acadêmico \cite{motta-roth2010}, as etapas de produção textual
(Antunes, 2003) e os objetos de ensino \cite{linodearaujo2014}; a
metodologia, na qual explicitamos a abordagem e o tipo de pesquisa, bem
como os procedimentos de coleta e análise de dados; os resultados,
contendo a exploração dos objetos de ensino contemplados nas videoaulas
sobre ensino de produção de artigo acadêmico; as considerações finais,
nas quais sinalizamos algumas implicações advindas dos resultados
alcançados.
\section{Theoretical Framework}\label{sec-theoretical}
\subsection{The phraseology-technology binomial}\label{sub-sec-thephraseologytechnology}

Today, phraseology and technology have significantly converged with the
advent of the internet and social networks \cite{corpas2013}.
Phraseology constitutes a specialized field of linguistic study that
focuses on fixed or semi-fixed expressions within a language,
encompassing idiomatic expressions, collocations, and other multi-word
units. Idiomatic expressions are characterized by meanings that cannot
be directly inferred from the literal interpretation of their
constituent elements, as their comprehension relies heavily on cultural
and contextual factors. These expressions pose significant challenges in
interpretation and translation due to their language-specific,
figurative nature and nuanced usage. Thus, phraseology systematically
explores the structural, semantic, and functional dimensions of these
expressions, emphasizing their linguistic and cultural specificity.

In parallel, within the realm of technology, the concept of CAI pertains
to the application of technological tools and specialized software to
augment the interpreting process, enhancing both efficiency and
precision. CAI encompasses a range of resources, including terminology
management systems, speech recognition software, and real-time
transcription tools. These technologies play a pivotal role in fostering
terminological consistency and streamlining the workflow of professional
interpreters, thereby contributing to the overall quality and
effectiveness of interpreting practices.

Digital platforms allow users to share phrases, expressions, and
thoughts instantly and globally, creating new forms of interaction and
communication \cite{piccioni2017}. In this context,
artificial intelligence (AI) and natural language processing are
revolutionising the way we interact with technology. In this sense,
virtual assistants such as Siri, Alexa and Google Assistant are
understanding and responding to voice commands in increasingly
sophisticated ways, changing the way we use language in our daily lives.
In other words, the combination of phraseology and technology represents
a dynamic and fruitful interaction capable of transforming the way we
understand and use idiomatic expressions in different linguistic
contexts. Phraseology, as a linguistic discipline focused on the study
of phraseological units and their use in discourse, has been greatly
enriched by technological advances that have allowed the development of
increasingly sophisticated language analysis and processing tools
\cite{sarachoArnaiz2015}.

The integration of technology in the study of phraseology opens new
possibilities for the study and analysis of idiomatic expressions in
different languages \cite{mogorronHuerta2012}. Thus, using digital
linguistic corpora and natural language processing tools, patterns, and
regularities in the use of phraseological units can be identified, which
has enriched our understanding of how these expressions are used in
different discursive contexts \cite{fantinuoli2017b,corpaspastorrubio2023,gaber2023b}.
%\alert{(Fantinuoli, 2017b; Corpas Pastor \& Rubio, 2023; Gaber, 2023)}. 
However, in this context, the pairing of
phraseology and technology also presents challenges and limitations
\cite{sevillaMunoz2012}. In this sense, the development of CAI has been
catalysed by significant advances in fields such as AI, machine learning
and computational linguistics. These technologies have enabled the
development of increasingly sophisticated interpreting systems capable
of instantly and accurately translating real-time conversations and
written texts in multiple languages \cite{koponen2021,guo2023}.

Idiomatic expressions encapsulate cultural identity, reflecting a
community's values, traditions, and collective
experiences. For example, an English expression like "spill the beans"
(to reveal a secret) might be meaningless or misinterpreted if
translated literally into a language without a similar metaphorical
framework. Such cultural nuances demand interpretive skills that
transcend linguistic competence, requiring interpreters to access a deep
understanding of both the source and target cultures \cite{ramirezRodriguez2024}. 
When these nuances are ignored or mistranslated, the resulting
communication can lack authenticity, coherence, or even lead to
misunderstandings. In practice, the challenge lies not only in decoding
the meaning of idiomatic expressions but also in determining how to
convey their intended effect whether by using an equivalent idiom in the
target language, paraphrasing the underlying concept, or providing
additional contextual information.

Current CAI systems are ill-equipped to handle the complexity of
idiomatic expressions, largely because their underlying algorithms are
designed to prioritize literal, data-driven translations rather than
context-sensitive or culturally nuanced interpretations. Speech
recognition software often struggles with regional accents or informal
speech, resulting in errors at the input stage. Similarly, machine
translation engines, though increasingly sophisticated, rely on
probabilistic models that may fail to capture the figurative meanings of
idioms or their cultural connotations. Moreover, most CAI tools lack the
ability to dynamically adjust to the contextual or pragmatic needs of a
conversation \cite{corpasPastor2020,corpasPastor2022}.

\subsection{The role of technology in interpreting: the case of Yandex browser}\label{sub-sec-theroleoftechnology}

As mentioned above, technology has played a key role in the development
of interpreting, improving the quality of interpreting through the
availability of resources such as terminology databases and online
dictionaries \cite{CifuentesFerez2015,rockwell2022}. These
enable interpreters to find the accurate and up-to-date information they
need to do their job effectively. A prominent example in this area is
Yandex browser, a platform developed by the Russian company Yandex that
combines AI and speech recognition technology to provide real-time
interpreting services. Yandex browser represents a significant advance
in CAI, as it harnesses the ability of AI to process large amounts of
linguistic data efficiently and quickly \cite{jibreel2023}.

This tool is designed to be easily integrated into different platforms
and devices, allowing users to access real-time interpretation services
from anywhere, at any time. This system uses natural language processing
and machine learning algorithms to automatically translate conversations
into different languages \cite{erbsen2023}. In addition, Yandex
browser includes interpreting capabilities in various contexts, such as
business meetings, international conferences, video conferences and
online events. The use of this technology in real-time situations
demonstrates the ability of AI to adapt to different circumstances and
provide a practical and efficient interpreting experience \cite{novozhilova2020}.

Recently, new technological tools have been explored to improve
interpreting, such as the integration of AI for context analysis and the
detection of cultural nuances in discourse \cite{defrancqfantinuoli2021}. 
These advances are revolutionising the way interpreting
challenges are addressed, allowing professionals to adapt more
effectively to users' needs and preferences. Moreover,
the application of machine learning algorithms significantly contributes
to improving the accuracy and fluency of interpreting, opening new
possibilities for human-technology collaboration in this field, playing
a key role in improving intercultural communication \cite{alotaibi2020}.

In this context, the speech recognition technology built into Yandex
browser is another key aspect of its functionality. This system allows
spoken conversations to be instantly converted to text, which is then
analysed and translated by AI algorithms to provide seamless
interpretations \cite{shadievLiu2023}. The ability to translate both
speech and text in real time is essential to ensure effective
communication in different situations and contexts. Furthermore, Yandex
browser is characterised by its ability to adapt to different accents,
intonations and speaking styles, which in theory improves the accuracy
of translations and ensures smooth communication between interlocutors.
This versatility is particularly important in multicultural environments
where significant linguistic variations can occur. In this context, such
adaptability to possible variations is based on algorithms that can
detect patterns in speech and dynamically adjust the interpretation to
accurately reflect the original meaning of the message, as well as on
advanced acoustic modelling and speech analysis techniques \cite{kim2020}.

To further enhance its capabilities, the Yandex Live Multimedia platform
also benefits from regular updates and continuous improvements to its AI
algorithms. This allows it to evolve and provide increasingly advanced
and efficient interpretation solutions by exploring the use of new
technologies based on deep language modelling and recurrent neural
networks. These techniques enable the platform to better understand the
context, tone, and subtleties of human language, resulting in more
natural interpretation. In addition, the integration of reinforcement
learning systems into Yandex Live Multimedia's AI
algorithms allows the platform to improve its ability to adapt in real
time to new dynamic situations and contexts \cite{tao2021end}. This
innovative approach is redefining the way intercultural communication
challenges are addressed today, while helping to position Yandex as a
leader in the field of CAI and setting new standards in the quality and
accuracy of machine translation services.

However, despite advances in the field aimed at overcoming existing
barriers to automated interpreting, there still appear to be significant
challenges to the accuracy and fluency of real-time text interpretation.
Phraseology, understood as the study of language-specific idiomatic
expressions endowed with meanings, is an elusive area in the context of
CAI due to its complexity and variability. In the interpreting process,
phraseology poses a challenge to AI systems, as literal translations of
idiomatic expressions may not convey the correct meaning in the target
language. The variability of idiomatic expressions and the presence of
regionalisms and jargon make it difficult for CAI systems to accurately
recognise and translate the implied meaning of these fixed
constructions. The lack of standardisation of phraseology across
languages and the constant evolution of colloquial expressions also make
it difficult to develop automated interpreting systems that can
accurately handle these linguistic aspects.

\section{Methodology}\label{sec-methodology}

In this paper, a technology developed by Yandex in 2021 that translates
a live stream from English, Spanish, French, Italian, German, or Chinese
into Russian has been used. The translation of a live stream is a
challenging task that has been addressed by the development of a novel
technique based on neural networks. Our study is devoted to the
evaluation of this new technique and focuses on its ability to preserve
the subtleties of semantic meaning and cultural connotations inherent in
phraseological expressions in different linguistic contexts. Although
our study does not address the technical details underlying this tool,
we address these issues as they may help to inform translation. This
technology incorporates advanced machine learning algorithms to enable
the instantaneous translation of language during audiovisual broadcasts.
In essence, this algorithm comprises five fundamental steps, each
executed by a distinct neural network model.

Initially, the audio stream is captured and transcribed into plain text
using automatic speech recognition. The video may contain extraneous
sounds such as noise and music, people may speak with different accents,
speeds and diction, and there may be many speakers, so the technology
must ensure that context and coherence are maintained during the
translation process. Therefore, the algorithm takes a sequence of audio
chunks as input, extracts acoustic features, and passes them into the
neural network. The neural network in turn produces a set of word
sequences from which the language model selects the most plausible
hypothesis.

Subsequently, a machine translation model is employed to translate the
text into the desired target language. There are several problems here:
if you translate word-by-word or phrase-by-phrase, the quality will
suffer, and if you wait for a long pause to guarantee the end of a
sentence, there will be a long delay. So, the technology groups words
into sentences without losing meaning or making sentences too long. For
correct translation at this stage, it is also necessary to determine the
gender of the speaker to determine to whom a particular line belongs and
to reproduce the voice correctly. After selecting individual sentences
and lines, the translation is performed, for which Yandex uses its own
translator.

Once the translation is completed, the translated text is processed
through text-to-speech technology, which converts the written content
into spoken audio. This step ensures that the generated speech sounds
natural and coherent, considering various linguistic features such as
tone, rhythm, and inflection. Additionally, the gender of the speaker,
which was previously identified during the initial stages of the
process, is incorporated into the synthesis to ensure the voice matches
the intended speaker's profile. This level of
customization helps improve the overall quality and authenticity of the
synthesized speech, making it more relatable and contextually
appropriate for the target audience.

Furthermore, the algorithm ensures that the translated speech is
accurately synchronized with the corresponding segments of the video
stream, aligning seamlessly with the visual content, and maintaining
synchronization with the video frames. This process is crucial to ensure
that the audio matches the timing of the speaker's lip movements and
actions in the video. Additionally, the neural network addresses several
challenges in this stage, such as when the speaker delivers a sentence
rapidly, or when the translated sentence is significantly longer than
the original. In these cases, the system must dynamically adjust the
synthesized audio by compressing or shortening it to fit within the
allotted time frame, ensuring smooth and natural speech flow that aligns
with the visual context.

Finally, the translated speech is seamlessly integrated into the live
video stream, replacing the original audio with the newly generated
translated audio. This newly created audio is then encapsulated into an
audio stream, which is embedded directly into the browser interface of
the viewer, allowing them to hear the translated speech while watching
the video in real-time. The technology used for this process is
currently functional exclusively within the Yandex browser, which was
the platform selected for the study. \Cref{fig-01} displays a detailed
screenshot of the Yandex browser interface, clearly highlighting the key
components involved in the translation and audio synchronization
process. It provides a visual representation of how the translated audio
is integrated into the video stream, showing the alignment between the
original video content and the overlaid translated speech. The
screenshot also highlights the user interface elements that facilitate
the viewer's interaction with the translated video, such as volume
controls, language options, and playback features.

\begin{figure}[htpb]
  \centering
  \begin{minipage}{\textwidth}
  \caption{Screenshot of the Yandex browser interface when utilizing 
  the automatic translation features.}
  \label{fig-01}
  \includegraphics[width=\textwidth]{image1.png}
  \source{Author's own work}
  \end{minipage}
\end{figure}

To enhance the study's transparency and validity, it is
crucial to provide comprehensive details regarding the sample selection,
data collection methods, and the analytical techniques employed. The
live video functionality within the interface is activated via a clearly
visible and easily accessible button, allowing for straightforward
interaction with the system. A notable feature of the translation
process is the 40-50 second temporal delay between the original live
video stream and its translated counterpart in Spanish. This delay, as
derived from a detailed evaluation of the system's operation, is
purposefully incorporated to allow sufficient time for the system to
process the content contextually, which is essential for delivering an
accurate translation in real-time, especially when dealing with live
broadcasts. Such a delay also enables the system to manage the
complexities inherent in real-time translation, such as adjusting for
idiomatic expressions, speech nuances, and varying speaking speeds.

Moreover, the Yandex browser's functionality extends beyond mere passive
translation by allowing users to actively customize their experience.
This includes features like adjusting the volume of the original audio
track, which could be crucial in environments where background noise or
other factors might interfere with the audio quality. Additionally, the
availability of subtitles provides an extra layer of understanding and
flexibility for users, especially in cases where visual cues or context
may not fully suffice to ensure a clear understanding of the translated
content. These customizable features add a significant level of
adaptability to the translation process, catering to various user
preferences and enhancing the overall user experience. Such details are
integral to understanding the technical infrastructure of the study and
ensure its findings can be accurately interpreted and replicated in
future research.

This study employed the Yandex browser and a range of accessible
technological tools to explore the automated translation of live news
streams. The analysis focused on two YouTube channels, ``RTVE Noticias''
and ``Canal Sur Andalucía,'' both of which provide continuous news
coverage. These channels were selected as the primary subjects of
investigation due to their ongoing news broadcasts, providing a robust
sample for examining the translation of live content. The methodology
for the study, as outlined in \Cref{fig-02}, was structured to facilitate a
systematic comparison of the automated translation of phraseological
expressions in real-time news streams.

The approach for comparing the automated translation of phraseological
expressions involved several key stages:

\begin{enumerate}
\def\labelenumi{\arabic{enumi}.}
\item
  \emph{Audio recording:} The first step in the methodology involved
  recording both the original live stream (audio recording \#1) and its
  corresponding translation (audio recording \#2) simultaneously,
  ensuring that both recordings occurred at the same intervals during
  the live broadcast. Two separate devices were used to capture these
  audio recordings concurrently, ensuring precise alignment of the
  original and translated content.
\item
  \emph{Detection of phraseologisms (verbal idioms):} In the second
  stage, instances of phraseologisms, commonly used idiomatic
  expressions, were identified in audio recording \#1, which captured
  the live stream in the original language (Spanish). The identification
  process was carefully carried out to ensure that the expressions
  detected were contextually relevant and representative of the language
  used in the broadcast.
\item
  \emph{Translation matching:} Once the phraseologisms were identified
  in the original audio, the next step involved matching the
  corresponding translations of these expressions in audio recording
  \#2, which was in Russian. This step was crucial for establishing
  whether the automated translation system accurately rendered the
  idiomatic expressions in a culturally appropriate and linguistically
  accurate manner.
\item
  \emph{Comparison and analysis:} The final stage entailed a detailed
  comparison of the accuracy and correctness of the translations. A
  comprehensive comparison table was developed to facilitate this
  process, allowing for a side-by-side evaluation of the original and
  translated expressions. The total duration of the audio recordings in
  both Spanish (the original language) and Russian (the translated
  language) amounted to 243 minutes for each language. These recordings
  were made across different intervals, ranging from 15 to 60 minutes,
  and were captured on multiple occasions throughout the day over four
  non-consecutive days. This sampling strategy was employed to ensure a
  diverse representation of news topics covered during the broadcasts.
  As part of the analysis, 52 verbal idioms were identified and
  thoroughly examined within the context of the live news stream,
  providing insights into the effectiveness of the automated translation
  system in handling phraseological expressions in a dynamic, real-time
  setting.
\end{enumerate}

\begin{figure}[htpb]
  \centering
  \begin{minipage}{\textwidth}
  \caption{General outline of the methodology presented.}
  \label{fig-02}
  \includegraphics[width=\textwidth]{image2.png}
  \source{Author's own work.}
  \end{minipage}
\end{figure}




\section{Results}\label{sec-results}

\subsection{First Expectations}\label{sub-sec-firstexpectations}

The authors hypothesised that pupils would be able to achieve a higher
linguistic level as a result of the unquestionable stimulation provided
by the audio-visual tools, which have been demonstrated to address the
learning needs of students. Additionally, in accordance with the
findings of previous research, it was anticipated that the
children\textquotesingle s level of motivation for the subject would
increase.

Nevertheless, it is reasonable to anticipate certain challenges
associated with the computer skills and linguistic proficiency of the
pupils. With regard to the latter, it is important to recognise that
children are simultaneously acquiring three languages without formal
instruction, which could potentially lead to some difficulties.

\subsection{Results on Acquisition of the Language}\label{sub-sec-resultsonacquisition}

As previously stated, the Montessori Method does not include exams,
therefore it was not feasible to propose a pre-test and post-test
framework to collect quantitative data. Consequently, it is not possible
to provide evidence of language improvement, if any. Nonetheless, it was
possible to anticipate certain improvements in their productions, given
that the children were engaged in activities such as translating,
dubbing and subtitling videos, practising listening comprehension,
pronunciation, writing and speaking.

A further defining feature of Montessori is that children select the
areas of study that they wish to pursue. Thus, this approach presented a
significant challenge for them, as their choices were based on the
scenes they enjoyed, irrespective of the level of L2 proficiency
required. This is the reason why the implementation resulted in
unforeseen language difficulties for the children, which obliged the
teacher to provide linguistic support by scaffolding the language.

Although it was not possible to analyse in detail the language
improvement resulting from the implementation, some common mistakes
committed by the pupils could be identified. These can be divided into
two main groups: those related to written productions and those related
to oral ones. In general, the children committed grammatical mistakes in
order to fit the sentences with the speech, such as the removal of the
subjects or the elision of auxiliary verbs. The errors associated with
oral production were primarily related to the rhythm of the
conversations and the pronunciation, particularly that of the vocalic
phonemes, as the pupils read the words in a literal manner. It is
important to highlight the intricate nature of these phonemes for
speakers of Basque and Spanish. In comparison to English, which has
twelve vocalic sounds, the two languages in question possess only five.
Furthermore, the pace of the original dialogues also affected the
pupils\textquotesingle{} oral productions, necessitating adjustments in
their speaking rate to align with the tempo of the original dialogues.

\subsection{Results of the Questionnaire}\label{sub-sec-resultsofthequestionnaire}

Following the completion of the implementation, pupils completed a
survey about AVT and DAT (see \Cref{annex-01}). Unfortunately, as the teachers
did not compel the pupils to fill it, and children were being prepared
for their incorporation to the formal instruction of the next course
(1st of Secondary Education), only seven children answered to the
questionnaire. The questions addressed their opinions regarding the
utilisation of DAT, the motivation it provides and its potential
integration into future English classes. As the pupils are taught in the
Basque language, and given the inherent complexity of the vocabulary and
the dearth of knowledge about the subject matter, the survey was
developed in Basque. However, for the sake of clarity and accessibility,
the sentences have been translated into English in the title of the
figures.

The initial question (see \Cref{fig-01}) sought to ascertain the
children\textquotesingle s level of enjoyment in the workshop. All of
the participants provided a rating that was above the minimum passing
grade. Four children, representing over half of the sample, rated the
workshop with an 8 or 9, while two children assigned a 6, and one child
a 7. These ratings indicate that the children felt at ease and content.
It is noteworthy that none of the children reached the 10-point mark,
given that they live in a multimodal world surrounded by video and
gaming platforms, which are likely to exert a strong influence on their
preferences. It may be posited that the introduction of this novel
activity, coupled with the necessity to master the utilisation of
internet-based tools, has instilled a sense of unease and lack of
confidence amongst them.
\begin{figure}[htbp]
    \centering
    \begin{minipage}{.5\textwidth}
    \includegraphics[width=\textwidth]{fig01.png}
    \caption{Question 1. From 1 to 10, how much have you enjoyed
    the ambiance of the workshop?}
    \label{fig-01}
    \source{Owm elaboration.}
    \end{minipage}
\end{figure}

In question 2 (see \Cref{fig-02}), the respondents were asked about their
previous knowledge of AVT. Four of the participants were unaware of its
existence, while the remaining three were cognizant of it. These figures
were unanticipated, given that the children have access to audiovisual
products in three different languages. This may have led them to assume
that there is a process of translation behind that range of options. It
seems reasonable to posit that the most probable reason for this lack of
awareness is that, due to their age, they are accustomed to consuming
audiovisual content in those languages and have not considered the
processes involved.

\begin{figure}[htbp]
    \centering
    \begin{minipage}{.5\textwidth}
    \includegraphics[width=\textwidth]{fig02.png}
    \caption{Question 2. Did you know what audiovisual translation
    was? (green-yes, purple-no).}
    \label{fig-02}
    \source{Owm elaboration.}
    \end{minipage}
\end{figure}

The answers to question 3, if they had enjoyed learning the L2 by means
of the editing of videos (see \Cref{fig-03}) were also positive.
Three-quarters of the children, five, demonstrated a positive attitude
towards the methodology employed, while two of them expressed a negative
opinion. Once more, we may cite the necessity of learning something new
compulsory as the primary reason for their refusal. These results align
with the responses to questions 1, 5 and 6, which will be analysed
subsequently.

\begin{figure}[htbp]
    \centering
    \begin{minipage}{.5\textwidth}
    \includegraphics[width=\textwidth]{fig03.png}
    \caption{Question 3. Have you enjoyed learning English by
    means of video editing?}
    \label{fig-03}
    \source{Owm elaboration.}
    \end{minipage}
\end{figure}

In response to question 4, which pertains to the two modes implemented
in class and translation (see \Cref{fig-04}), it can be observed that data do
not provide a clear indication of the pupils\textquotesingle{}
preferences towards either mode. The preference for dubbing is indicated
by a single pupil, while the other two modes were selected by two pupils
each. It is noteworthy that two children selected translation, a written
activity that is not directly relevant to their lived experience, rather
than the other two modes, which are inherently more visual.

\begin{figure}[htbp]
    \centering
    \begin{minipage}{.5\textwidth}
    \includegraphics[width=\textwidth]{fig04.png}
    \caption{Question 4. Which tool have you liked the most?
    (Green: interlingual indirect translation of texts; purple: dubbing;
    blue: subtitling).}
    \label{fig-04}
    \source{Owm elaboration.}
    \end{minipage}
\end{figure}


In question 5, the children were asked about the process of language
acquisition. As illustrated in \Cref{fig-05}, all pupils demonstrated an
understanding that they would benefit from these educational
initiatives. It is noteworthy that five of the pupils awarded the
project an 8, a high rating that reflects their trust in DAT. One pupil
considered the quality to be satisfactory, while one rated it slightly
above average. Overall, the marks are deemed satisfactory, as none of
the responses were below the passing mark of 5. These responses are
consistent with those provided in question 2, in which two children
indicated a lack of knowledge regarding AVT.

\begin{figure}[htbp]
    \centering
    \begin{minipage}{.5\textwidth}
    \includegraphics[width=\textwidth]{fig05.png}
    \caption{Question 5. From 1 to 10, how much do you think you
    would learn?}
    \label{fig-05}
    \source{Owm elaboration.}
    \end{minipage}
\end{figure}

The final question (number 6) asked the pupils whether they would
like to engage in DAT activities within the classroom setting. Five
pupils responded in the affirmative, while the remaining two expressed
opposition to the proposal. However, their responses lacked sufficient
clarity or referenced their disapproval of the instructor rather than
the DAT activities themselves. It can be inferred that these two
students are the same individuals who, in question 1, rated the workshop
atmosphere as 6, who provided negative responses to question 3 regarding
their enjoyment along the process, and who rated the development as 6
and 7. This is somewhat surprising, given that these ratings are not
particularly low.

Question 6 (original answers in \Cref{annex-03})

\begin{itemize}
    \item \textbf{Pupil 1}: Yes, because I learn.
    \item \textbf{Pupil 2}: Yes, because I like making videos a lot.
    \item \textbf{Pupil 3}: Yes, because I have never tried it and I think it would be good.
    \item \textbf{Pupil 4}: Yes, because I like recording videos a lot.
    \item \textbf{Pupil 5}: Yes, because it is fun.
    \item \textbf{Pupil 6}: No, because I do not want X (a teacher) to appear.
    \item \textbf{Pupil 7}: No.
\end{itemize}

\section{Conclusion}\label{sec-conclusion}

The findings of this study suggest that the exercise of agency in initial education contexts is multifaceted. Like the findings of \posscite{mercer2011,mercer2012} empirical work focusing on language learners, the analysis of the narratives in this study indicates that agency is influenced by the intricate interconnectedness of various factors and elements coexisting in the participants' systems. It was possible to identify interpersonal factors, such as the perceived opportunities to interact with agents (human and non-human) in formal and informal online environments, as well as intrapersonal factors, such as the impact on emotions, attitudes, and beliefs about the best ways to learn. The data also show that the exercise of agency is dynamic, subject to change, and open, as it can be influenced by other systems.

Throughout the analysis, the relational nature of agency \cite{larsen2019} proved to be quite salient, as the data unveiled a reciprocal interaction between internal and external factors and the actions emerging from these relationships within contexts and their possibilities.

The results point to the potential of mobile devices in facilitating the exercise of agency among the participating pre-service teachers. These devices allow teachers to access information, speed up time, study, and even provide opportunities for distraction. When it comes to their praxis, these teachers also recognize the possibilities of mobile technology to motivate, bring the classroom to the 21st century, and engage learners. The potential for mobile technologies to impact social life was also evident from the data, as participants at various points in their stories emphasized how pervasive and important technology is to everyday life and citizenship.

In terms of the possible implications of this study for teacher education, one of the possible insights may stem from the recognition that in order to understand the possibilities of teacher agency – pre-service and in-service – it is important to consider the environments in which these agents circulate, the technologies and other agents with which they interact, the nested systems that make up their ecosystems, and the ways in which these dynamics affect and feed back into the interaction between intra- and interpersonal aspects.

We recognize the complexity of agency in the context studied, and although some dynamics and the complex fabric of teachers' agency have been highlighted in the data analyzed, further research can highlight other units of analysis in relation to inter- and intrapersonal aspects, elements, and systems that can further the understanding of the role of these "agents", thus contributing to research on teacher agency.

\printbibliography\label{sec-bib}
%conceptualization,datacuration,formalanalysis,funding,investigation,methodology,projadm,resources,software,supervision,validation,visualization,writing,review
\begin{contributors}[sec-contributors]
\authorcontribution{Pablo Ramírez Rodríguez}[conceptualization,datacuration,writing,investigation,resources,review,supervision,validation]
\authorcontribution{Evgeniy Antonov}[formalanalysis,datacuration,writing,methodology,resources,supervision,visualization]
\end{contributors}
\end{document}
