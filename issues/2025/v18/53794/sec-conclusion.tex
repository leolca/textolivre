\section{Conclusion}\label{sec-conclusion}

Overall, despite advances in AI and machine learning, there are still
limitations in the ability of machine translation systems to understand
and faithfully translate idiomatic expressions. This underlines the
importance of and need for human intervention and linguistic knowledge
in the translation of complex and culturally rich texts. Therefore,
further research and improvement of machine translation systems is
needed to deal more effectively with the translation of idiomatic
expressions and to improve the quality and accuracy of the
interpretations generated. It is also worth noting that advances in the
field of AI have enabled the development of machine translation systems
that incorporate more sophisticated techniques, such as the use of
neural translation models based on deep neural networks. These models
could capture the full context of a sentence more effectively and to
consider the syntax and semantics of words in context, resulting in more
accurate and natural translations of idiomatic expressions. However, AI,
and CAI developed by Yandex in particular, faces challenges such as
natural language processing, understanding the context and semantics of
utterances, and a lack of sufficient data to train translation models.

The findings of this study carry considerable practical implications for
professionals in the fields of translation, interpretation, and
education. For translators and interpreters, particularly those engaged
with idiomatic expressions and culturally specific language, the study
underscores key challenges, such as the limitations of AI and CAI
systems in grasping context and semantics, that highlight the continued
necessity of human expertise to ensure both linguistic accuracy and
cultural appropriateness in translations. While AI-driven tools like
Yandex's CAI can enhance operational efficiency in certain scenarios,
they are not yet capable of replicating the nuanced comprehension and
interpretative decisions that skilled human translators and interpreters
can provide. As such, practitioners should view CAI systems as
complementary tools, leveraging their capabilities to support the
translation process, but not as standalone solutions. Human expertise
remains essential for navigating ambiguities, managing cultural
subtleties, and accurately interpreting idiomatic expressions, which
continue to present significant challenges for current machine
translation systems.

Moreover, for educators in the fields of translation and interpretation,
the study's insights offer valuable guidance for
curriculum development. Emphasizing the limitations and potential of CAI
systems in interpreting complex language features like idioms can help
future translators and interpreters become more proficient in utilizing
technology effectively. Educators could integrate hands-on training with
machine translation tools, while also teaching students how to navigate
the cultural subtleties and contexts that technology may miss.
Understanding the need for continuous advancement in AI systems could be
woven into training, preparing students to bridge the gap between human
understanding and technological capabilities.
