\section{Theoretical Framework}\label{sec-theoretical}
\subsection{The phraseology-technology binomial}\label{sub-sec-thephraseologytechnology}

Today, phraseology and technology have significantly converged with the
advent of the internet and social networks \cite{corpas2013}.
Phraseology constitutes a specialized field of linguistic study that
focuses on fixed or semi-fixed expressions within a language,
encompassing idiomatic expressions, collocations, and other multi-word
units. Idiomatic expressions are characterized by meanings that cannot
be directly inferred from the literal interpretation of their
constituent elements, as their comprehension relies heavily on cultural
and contextual factors. These expressions pose significant challenges in
interpretation and translation due to their language-specific,
figurative nature and nuanced usage. Thus, phraseology systematically
explores the structural, semantic, and functional dimensions of these
expressions, emphasizing their linguistic and cultural specificity.

In parallel, within the realm of technology, the concept of CAI pertains
to the application of technological tools and specialized software to
augment the interpreting process, enhancing both efficiency and
precision. CAI encompasses a range of resources, including terminology
management systems, speech recognition software, and real-time
transcription tools. These technologies play a pivotal role in fostering
terminological consistency and streamlining the workflow of professional
interpreters, thereby contributing to the overall quality and
effectiveness of interpreting practices.

Digital platforms allow users to share phrases, expressions, and
thoughts instantly and globally, creating new forms of interaction and
communication \cite{piccioni2017}. In this context,
artificial intelligence (AI) and natural language processing are
revolutionising the way we interact with technology. In this sense,
virtual assistants such as Siri, Alexa and Google Assistant are
understanding and responding to voice commands in increasingly
sophisticated ways, changing the way we use language in our daily lives.
In other words, the combination of phraseology and technology represents
a dynamic and fruitful interaction capable of transforming the way we
understand and use idiomatic expressions in different linguistic
contexts. Phraseology, as a linguistic discipline focused on the study
of phraseological units and their use in discourse, has been greatly
enriched by technological advances that have allowed the development of
increasingly sophisticated language analysis and processing tools
\cite{sarachoArnaiz2015}.

The integration of technology in the study of phraseology opens new
possibilities for the study and analysis of idiomatic expressions in
different languages \cite{mogorronHuerta2012}. Thus, using digital
linguistic corpora and natural language processing tools, patterns, and
regularities in the use of phraseological units can be identified, which
has enriched our understanding of how these expressions are used in
different discursive contexts \cite{fantinuoli2017b,corpaspastorrubio2023,gaber2023b}.
%\alert{(Fantinuoli, 2017b; Corpas Pastor \& Rubio, 2023; Gaber, 2023)}. 
However, in this context, the pairing of
phraseology and technology also presents challenges and limitations
\cite{sevillaMunoz2012}. In this sense, the development of CAI has been
catalysed by significant advances in fields such as AI, machine learning
and computational linguistics. These technologies have enabled the
development of increasingly sophisticated interpreting systems capable
of instantly and accurately translating real-time conversations and
written texts in multiple languages \cite{koponen2021,guo2023}.

Idiomatic expressions encapsulate cultural identity, reflecting a
community's values, traditions, and collective
experiences. For example, an English expression like "spill the beans"
(to reveal a secret) might be meaningless or misinterpreted if
translated literally into a language without a similar metaphorical
framework. Such cultural nuances demand interpretive skills that
transcend linguistic competence, requiring interpreters to access a deep
understanding of both the source and target cultures \cite{ramirezRodriguez2024}. 
When these nuances are ignored or mistranslated, the resulting
communication can lack authenticity, coherence, or even lead to
misunderstandings. In practice, the challenge lies not only in decoding
the meaning of idiomatic expressions but also in determining how to
convey their intended effect whether by using an equivalent idiom in the
target language, paraphrasing the underlying concept, or providing
additional contextual information.

Current CAI systems are ill-equipped to handle the complexity of
idiomatic expressions, largely because their underlying algorithms are
designed to prioritize literal, data-driven translations rather than
context-sensitive or culturally nuanced interpretations. Speech
recognition software often struggles with regional accents or informal
speech, resulting in errors at the input stage. Similarly, machine
translation engines, though increasingly sophisticated, rely on
probabilistic models that may fail to capture the figurative meanings of
idioms or their cultural connotations. Moreover, most CAI tools lack the
ability to dynamically adjust to the contextual or pragmatic needs of a
conversation \cite{corpasPastor2020,corpasPastor2022}.

\subsection{The role of technology in interpreting: the case of Yandex browser}\label{sub-sec-theroleoftechnology}

As mentioned above, technology has played a key role in the development
of interpreting, improving the quality of interpreting through the
availability of resources such as terminology databases and online
dictionaries \cite{CifuentesFerez2015,rockwell2022}. These
enable interpreters to find the accurate and up-to-date information they
need to do their job effectively. A prominent example in this area is
Yandex browser, a platform developed by the Russian company Yandex that
combines AI and speech recognition technology to provide real-time
interpreting services. Yandex browser represents a significant advance
in CAI, as it harnesses the ability of AI to process large amounts of
linguistic data efficiently and quickly \cite{jibreel2023}.

This tool is designed to be easily integrated into different platforms
and devices, allowing users to access real-time interpretation services
from anywhere, at any time. This system uses natural language processing
and machine learning algorithms to automatically translate conversations
into different languages \cite{erbsen2023}. In addition, Yandex
browser includes interpreting capabilities in various contexts, such as
business meetings, international conferences, video conferences and
online events. The use of this technology in real-time situations
demonstrates the ability of AI to adapt to different circumstances and
provide a practical and efficient interpreting experience \cite{novozhilova2020}.

Recently, new technological tools have been explored to improve
interpreting, such as the integration of AI for context analysis and the
detection of cultural nuances in discourse \cite{defrancqfantinuoli2021}. 
These advances are revolutionising the way interpreting
challenges are addressed, allowing professionals to adapt more
effectively to users' needs and preferences. Moreover,
the application of machine learning algorithms significantly contributes
to improving the accuracy and fluency of interpreting, opening new
possibilities for human-technology collaboration in this field, playing
a key role in improving intercultural communication \cite{alotaibi2020}.

In this context, the speech recognition technology built into Yandex
browser is another key aspect of its functionality. This system allows
spoken conversations to be instantly converted to text, which is then
analysed and translated by AI algorithms to provide seamless
interpretations \cite{shadievLiu2023}. The ability to translate both
speech and text in real time is essential to ensure effective
communication in different situations and contexts. Furthermore, Yandex
browser is characterised by its ability to adapt to different accents,
intonations and speaking styles, which in theory improves the accuracy
of translations and ensures smooth communication between interlocutors.
This versatility is particularly important in multicultural environments
where significant linguistic variations can occur. In this context, such
adaptability to possible variations is based on algorithms that can
detect patterns in speech and dynamically adjust the interpretation to
accurately reflect the original meaning of the message, as well as on
advanced acoustic modelling and speech analysis techniques \cite{kim2020}.

To further enhance its capabilities, the Yandex Live Multimedia platform
also benefits from regular updates and continuous improvements to its AI
algorithms. This allows it to evolve and provide increasingly advanced
and efficient interpretation solutions by exploring the use of new
technologies based on deep language modelling and recurrent neural
networks. These techniques enable the platform to better understand the
context, tone, and subtleties of human language, resulting in more
natural interpretation. In addition, the integration of reinforcement
learning systems into Yandex Live Multimedia's AI
algorithms allows the platform to improve its ability to adapt in real
time to new dynamic situations and contexts \cite{tao2021end}. This
innovative approach is redefining the way intercultural communication
challenges are addressed today, while helping to position Yandex as a
leader in the field of CAI and setting new standards in the quality and
accuracy of machine translation services.

However, despite advances in the field aimed at overcoming existing
barriers to automated interpreting, there still appear to be significant
challenges to the accuracy and fluency of real-time text interpretation.
Phraseology, understood as the study of language-specific idiomatic
expressions endowed with meanings, is an elusive area in the context of
CAI due to its complexity and variability. In the interpreting process,
phraseology poses a challenge to AI systems, as literal translations of
idiomatic expressions may not convey the correct meaning in the target
language. The variability of idiomatic expressions and the presence of
regionalisms and jargon make it difficult for CAI systems to accurately
recognise and translate the implied meaning of these fixed
constructions. The lack of standardisation of phraseology across
languages and the constant evolution of colloquial expressions also make
it difficult to develop automated interpreting systems that can
accurately handle these linguistic aspects.
