\section{Results and discussion}\label{sec-results}

As we have mentioned in this paper, AI uses various interpretation
techniques to orally translate any kind of information, such as natural
language processing, machine learning, speech recognition and semantic
analysis. In this sense, AI not only analyses and understands the
meaning of words and phrases in the context in which they are used, but
also uses machine learning algorithms to continuously improve its
ability to translate orally, as well as using speech recognition
technologies to capture and transcribe in real time. Importantly, AI
also relies on the use of neural networks and sequence processing models
to improve translation accuracy. These architectures allow AI to
understand the temporal and contextual relationships between words in a
sentence, resulting in a more coherent and contextual translation.

In this case, Yandex used natural language processing to firstly analyze
the text of the analyzed messages and secondly to extract relevant
information related to phraseology. It used machine learning to improve
speech recognition to transcribe and understand human speech, and
semantic analysis to understand the meaning behind the verbal idioms
displayed. When interpreting phraseological units, especially idioms,
several factors need to be considered. Among them are the ambiguity of
the meaning of the idioms between literal and figurative meaning; the
understanding of their cultural and linguistic context, since most of
the proposed phraseologisms are immersed in Hispanic culture; the
ability to recognize them as a single semantic unit; and the ability to
translate them appropriately into different languages, in our case
Russian. All of this is achieved through good use of contextual
resources and good adaptation to natural conversation. Therefore, it is
extremely important for AI to have access to contextual resources, such
as linguistic databases or text corpora, which allow it to understand
the use and meaning of verbal idioms in different contexts and
communication situations. Moreover, idioms are often used in natural and
colloquial conversational contexts, so AI must be able to interpret them
in a fluent and natural way, maintaining their original tone and style.

Nevertheless, after the speech interpretation of the speeches taken from
the news and the analysis of verbal idioms, it was observed that several
factors influenced the way in which these expressions were interpreted.
According to the data obtained, of the 52 units analyzed, only 20 were
correctly translated by Yandex, i.e., less than half. This suggests that
the ability of machine translation systems to interpret and translate
idiomatic expressions remains a major challenge in the development of
advanced translation technologies. The complexity and diversity of
idiomatic expressions in different languages make their accurate and
contextual translation a complicated process that requires further
analysis and development of specialized translation algorithms.

The presence of ambiguity in the meaning of idiomatic expressions,
consideration of the cultural context in which these expressions are
used, and attention to the type of linguistic register were crucial
aspects that affected the accuracy of the translations performed by
Yandex. The ambiguity in the meaning of idiomatic expressions posed a
significant challenge to AI, as the presence of multiple possible
interpretations made it difficult to select the most appropriate
translation based on the context in which these expressions were used.
The lack of clarity in the meaning of idioms led to incorrect, literal
translations or even the omission of part of the original message.

Furthermore, the cultural context in which the idiomatic expressions
were rooted was critical to their accurate interpretation. Yandex had
difficulty identifying and understanding the cultural connotations and
meanings implicit in these expressions, resulting in translations that
did not adequately reflect the original meaning of the idiomatic
expressions. On the other hand, the nature of the linguistic register of
the idioms also had an impact on the quality of the translations
produced. The presence of colloquialisms for certain idioms, which had
no direct equivalent in the target language, made it difficult to
interpret these expressions correctly, resulting in incorrect or
incomplete translations. The following are the most representative
examples of translations of the idioms analyzed.

In the case of ambiguity, many of the verbal idioms analyzed, such as:
``pisar fuerte'' (to make a strong impression), ``abrir puertas'' (to
open doors), ``poner en pie'' (to get on one´s feet), ``abrir camino''
(to pave the way) o ``ajustar cuentas'' (to settle scores), may have
other meanings or interpretations, which undoubtedly makes it difficult
to translate them accurately. In this context, AI in general, and Yandex
in particular, may have difficulties in choosing the appropriate
translation depending on the context in which the expression is used. In
our case, Yandex had difficulties in determining the correct meaning of
certain idioms based on the context in which they are used due to
ambiguity, that led to inaccurate or incorrect translations. This was
the case with the following idioms: ``pegar ojo'' (to get some sleep),
``salir rana'' (to turn out badly), ``caérsele la casa encima'' (to feel
overwhelmed), ``dar tela'' (to give a hard time), ``echar una mano'' (to
lend a hand), ``chuparse los dedos'' (to lick your fingers) or ``tener
mala leche'' (to be in a bad mood). All these verbal idioms have been
translated literally, resulting in confusing and incomprehensible
translations in Russian. It is therefore important to take these
potential ambiguities into account when using machine translation tools,
and in some cases, it may be necessary to use human translators to
achieve an accurate interpretation of idiomatic expressions.

In addition to linguistic and semantic aspects, the Russian translation
of the idiomatic expressions was also influenced by the cultural context
in which they are rooted. This is because many of the idiomatic
expressions presented are unique to a particular culture and reflect
values, beliefs, and traditions specific to that cultural context.
Therefore, a literal translation of these expressions may result in a
loss of their original meaning and connotation. In our case, Yandex
chose to translate some of these idioms in a more general or
contextualized way, considering the underlying meaning they convey in
the culture of origin, i.e., in Spanish. In this way, Yandex tried to
preserve the essence and original intention of the expression, adapting
it to the cultural context of the target language. The problem with this
is that the two translations are not always equivalent or congruent in
meaning. Some examples of unsuccessful translations by Yandex are:
``estar en una nube'' (to be on cloud nine), ``meter salsa'' (to spill
the beans), ``ser la guinda del pastel'' (to be the icing on the cake),
``dar con la tecla'' (to hit the nail on the head) or ``dar cosa'' (to
feel uneasy). Although the Russian translation of these idioms was not
literal, Yandex did not consider the specific cultural context in which
they are embedded.

Thus, ``estar en una nube'' was translated as \textrussian{быть на облаке} (``to be in
a cloud''), an expression used in Russian when a person dies and not, as
in the Spanish case, when one is excited or distracted by something
positive that has happened. In the case of meter salsa, this expression
was translated as \textrussian{водить фальшивки} (``to cheat''). The Spanish context
was that of a person who interferes in other people\textquotesingle s
lives to criticize them, while the Russian translation as a person who
cheats in the game does not make sense in the given context. The
following expression is curious in that its Russian counterpart exists
in its literal form \textrussian{вишенка на торте} (``to be the icing on the cake'').
However, Yandex chose to change one of the components of the idiom,
translating it as \textrussian{глазурь на торте} (to be the frosting of the cake),
which is understandable but not idiomatic. As for the expression ``dar
con la Tecla'', which is usually used as a synonym for ``dar en el
blanco'' (to hit the target), it was translated into Russian as
\textrussian{опредеделить деталь} (to determine in detail), while the expression ``me
da cosa'' in the sense of ``me da apuro'' (I feel embarrassed) has been
translated into Russian as \textrussian{это много мне дает} (it gives me a lot) due to
lack of context.

In terms of linguistic register, it is important to note that some
idiomatic phrases contain colloquialisms that are specific to a
particular register or level of linguistic formality. These elements
also posed an additional challenge for machine translation, as they may
not have a direct equivalent in the target language or may be
incomprehensible to native speakers of the target language. In these
cases, Yandex faced the difficulty of deciding how to approach the
translation of these expressions, choosing to omit the expression in
question if it could not find a suitable equivalence or, on the
contrary, to look for an approximate equivalence that preserved the
general meaning of the original expression. Examples of omissions
include: ``echarle morro'' (to be cheeky), ``meterse en un fregao'' (to
get into a mess), ``estar al pie del cañón'' (to hold the fort) and
``meter la pata'' (to mes up). In these cases, Yandex did not recognize
their specific meaning or context in Spanish because they are very
colloquial idiomatic expressions that do not have a direct translation
in Russian, so the tool decided to omit them directly to avoid possible
errors of interpretation. Regarding the search for an approximate
equivalence in order to preserve the general meaning of the original
expression, the following examples of idioms could be given: ``quitarle
hierro a un asunto'' -- to defuse a situation (\textrussian{снять напряжение} --
quitar presión -- to relieve pressure), ``meterse en un lío'' -- to ge
tinto trouble (\textrussian{втянуть в беспорядок} -- arrastrarse al desorden -- to
descend into chaos), ``estar en auge'' -- to be on the rise (\textrussian{процветать}
-- florecer -- to prosper), ``tener entre algodones a alguien'' - to
treat someone with kid gloves (\textrussian{держать в объятиях} -- mantener en abrazos
- to envelop someone in hugs), ``plantar cara'' -- to face up
(\textrussian{противостоять} -- oponerse -- to oppose) o ``estar en un pozo sin
fondo'' - to be in a never-ending hole (\textrussian{быть в яме} -- estar en un hoyo
-- to be in a hole). All the above examples perfectly preserve the
metaphorical meaning of the source text, and their translations, even
though in many cases they are not phraseologisms, fit perfectly into
Russian speech.

It is also interesting and positive to analyze those verbal idioms that
Yandex translated correctly. This was the case with the expressions:
``tener en el bolsillo'' -- to have somene under control (\textrussian{есть в
кармане}), ``dar alas'' -- to inspire (\textrussian{дать крылья}), ``hacerse la boca
agua'' - to make one's mouth water (\textrussian{слюнки текут}),
``tocar madera'' - to knock on wood (\textrussian{постучать по дереву}) y ``no
levantar cabeza'' - to be overwhelmed (\textrussian{не поднимать головы}). In these
cases, Yandex may have recognized the contextual meaning of the
expressions and provided an appropriate translation based on their
Russian equivalents. This shows that in some cases machine translators
may be able to correctly translate idiomatic expressions if they have a
large and up-to-date database that allows them to recognize and
understand the meaning of such expressions in different languages. It is
also noticeable that most of the correctly translated expressions are
idioms whose counterparts exist in the Russian language. This is a clear
indication that both Yandex and other machine translators still process
a certain bias towards literal language, showing more correct
translations only in cases where there are full equivalences in both
languages.

The findings reveal that CAI systems like Yandex face considerable
challenges in accurately interpreting idiomatic expressions and
culturally specific language. This observation aligns with theoretical
perspectives emphasizing the limitations of current natural language
processing systems in capturing the contextual and pragmatic dimensions
of language. Such limitations reinforce the theoretical argument that
AI-driven systems lack the nuanced understanding required to manage
linguistic and cultural subtleties, a domain where human interpreters
excel.

Additionally, the study highlights a critical gap in the scalability of
CAI tools for real-time applications, particularly in scenarios that
demand high fidelity in the translation of phraseological units.
Existing theories often posit that AI can significantly augment human
capabilities in real-time interpretation. However, the
study's findings challenge this assumption by
demonstrating that the technology struggles with temporal constraints
and the need for cultural and contextual alignment. This discrepancy
underscores the necessity for advancements in natural language
processing algorithms that incorporate larger, more diverse training
datasets, particularly those focused on colloquial and idiomatic
language.

Furthermore, the study advances the theoretical discourse by
illustrating the importance of hybrid approaches that integrate human
expertise with AI capabilities. While CAI tools can enhance efficiency
and accessibility, their limitations necessitate human oversight to
address ambiguities, cultural nuances, and the interpretive depth
required for idiomatic language. This reinforces theories advocating for
a collaborative framework in which AI serves as an assistive, rather
than autonomous, tool.

The study's results highlight that while CAI tools like
Yandex offer notable advantages in terms of efficiency and
accessibility, they fall short in handling the complexities of idiomatic
expressions and culturally nuanced language. This limitation underscores
the theoretical argument that AI systems, even with advanced natural
language processing capabilities, currently lack the cognitive and
contextual understanding necessary to navigate the intricacies of
phraseological translation effectively. These insights align with
established theories suggesting that the interpretive depth required for
accurate and culturally appropriate translations remains beyond the
scope of autonomous systems.

Moreover, the study challenges theories advocating for full automation
in language interpretation, revealing that the nuanced decision-making
inherent to human interpreters is indispensable for managing linguistic
ambiguity and cultural specificity. It supports the proposition that
hybrid models, where human judgment complements AI''s
speed and processing capacity, offer the most effective framework for
real-world applications of CAI. This collaborative approach is
particularly crucial in high-stakes scenarios, such as live news
translation, where real-time accuracy and cultural resonance are
imperative.
