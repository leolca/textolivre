% !TEX TS-program = XeLaTeX
% use the following command:
% all document files must be coded in UTF-8
\documentclass[portuguese]{textolivre}
% build HTML with: make4ht -e build.lua -c textolivre.cfg -x -u article "fn-in,svg,pic-align"
\usepackage{multirow}

\journalname{Texto Livre}
\thevolume{18}
%\thenumber{1} % old template
\theyear{2025}
\receiveddate{\DTMdisplaydate{2025}{1}{26}{-1}} % YYYY MM DD
\accepteddate{\DTMdisplaydate{2025}{7}{30}{-1}}
\publisheddate{\DTMdisplaydate{2025}{11}{13}{-1}}
\corrauthor{Thais Lemos dos Santos}
\articledoi{10.1590/1983-3652.2025.57811}
%\articleid{NNNN} % if the article ID is not the last 5 numbers of its DOI, provide it using \articleid{} commmand 
% list of available sesscions in the journal: articles, dossier, reports, essays, reviews, interviews, editorial
\articlesessionname{articles}
\runningauthor{Santos e Stephens} 
%\editorname{Leonardo Araújo} % old template
\sectioneditorname{Daniervelin Pereira}
\layouteditorname{Saula Cecília}

\title{Educação ambiental para o Ensino Fundamental II: criação de um aplicativo para aulas de poluição ambiental}
\othertitle{Environmental education for Lower Secondary School: development of an application for environmental pollution classes}
% if there is a third language title, add here:
%\othertitle{Artikelvorlage zur Einreichung beim Texto Livre Journal}

\author[1]{Thais Lemos dos Santos ~\orcid{0009-0005-7358-9252}\thanks{Email: \href{mailto:thaislemosantos@gmail.com}{thaislemosantos@gmail.com}}}
\author[2]{Paulo Roberto Soares Stephens~\orcid{0000-0001-6389-1371}\thanks{Email: \href{mailto:paulor2000@yahoo.com.br}{paulor2000@yahoo.com.br}}}
\affil[1]{Fundação Oswaldo Cruz, Instituto Oswaldo Cruz (IOC), Ensino em Biociências e Saúde, Rio de Janeiro, RJ, Brasil.}
\affil[2]{Fundação Oswaldo Cruz, Instituto Oswaldo Cruz (IOC), Laboratório de Inovações em Terapias, Ensino e Bioprodutos  (LITEB), Rio de Janeiro, RJ, Brasil.}

\addbibresource{article.bib}
% use biber instead of bibtex
% $ biber article

% used to create dummy text for the template file
\definecolor{dark-gray}{gray}{0.35} % color used to display dummy texts
\usepackage{lipsum}
\SetLipsumParListSurrounders{\colorlet{oldcolor}{.}\color{dark-gray}}{\color{oldcolor}}

% used here only to provide the XeLaTeX and BibTeX logos
\usepackage{hologo}

% if you use multirows in a table, include the multirow package
\usepackage{multirow}

% provides sidewaysfigure environment
\usepackage{rotating}

% CUSTOM EPIGRAPH - BEGIN 
%%% https://tex.stackexchange.com/questions/193178/specific-epigraph-style
\usepackage{epigraph}
\renewcommand\textflush{flushright}
\makeatletter
\newlength\epitextskip
\pretocmd{\@epitext}{\em}{}{}
\apptocmd{\@epitext}{\em}{}{}
\patchcmd{\epigraph}{\@epitext{#1}\\}{\@epitext{#1}\\[\epitextskip]}{}{}
\makeatother
\setlength\epigraphrule{0pt}
\setlength\epitextskip{0.5ex}
\setlength\epigraphwidth{.7\textwidth}
% CUSTOM EPIGRAPH - END

% to use IPA symbols in unicode add
%\usepackage{fontspec}
%\newfontfamily\ipafont{CMU Serif}
%\newcommand{\ipa}[1]{{\ipafont #1}}
% and in the text you may use the \ipa{...} command passing the symbols in unicode

% LANGUAGE - BEGIN
% ARABIC
% for languages that use special fonts, you must provide the typeface that will be used
% \setotherlanguage{arabic}
% \newfontfamily\arabicfont[Script=Arabic]{Amiri}
% \newfontfamily\arabicfontsf[Script=Arabic]{Amiri}
% \newfontfamily\arabicfonttt[Script=Arabic]{Amiri}
%
% in the article, to add arabic text use: \textlang{arabic}{ ... }
%
% RUSSIAN
% for russian text we also need to define fonts with support for Cyrillic script
% \usepackage{fontspec}
% \setotherlanguage{russian}
% \newfontfamily\cyrillicfont{Times New Roman}
% \newfontfamily\cyrillicfontsf{Times New Roman}[Script=Cyrillic]
% \newfontfamily\cyrillicfonttt{Times New Roman}[Script=Cyrillic]
%
% in the text use \begin{russian} ... \end{russian}
% LANGUAGE - END

% EMOJIS - BEGIN
% to use emoticons in your manuscript
% https://stackoverflow.com/questions/190145/how-to-insert-emoticons-in-latex/57076064
% using font Symbola, which has full support
% the font may be downloaded at:
% https://dn-works.com/ufas/
% add to preamble:
% \newfontfamily\Symbola{Symbola}
% in the text use:
% {\Symbola }
% EMOJIS - END

% LABEL REFERENCE TO DESCRIPTIVE LIST - BEGIN
% reference itens in a descriptive list using their labels instead of numbers
% insert the code below in the preambule:
%\makeatletter
%\let\orgdescriptionlabel\descriptionlabel
%\renewcommand*{\descriptionlabel}[1]{%
%  \let\orglabel\label
%  \let\label\@gobble
%  \phantomsection
%  \edef\@currentlabel{#1\unskip}%
%  \let\label\orglabel
%  \orgdescriptionlabel{#1}%
%}
%\makeatother
%
% in your document, use as illustraded here:
%\begin{description}
%  \item[first\label{itm1}] this is only an example;
%  % ...  add more items
%\end{description}
% LABEL REFERENCE TO DESCRIPTIVE LIST - END


% add line numbers for submission
%\usepackage{lineno}
%\linenumbers

\begin{document}
\maketitle

\begin{polyabstract}
\begin{abstract}
A poluição ambiental se destaca como uma das problemáticas socioambientais mais urgentes e visíveis na realidade cotidiana, esse fenômeno está frequentemente relacionado às atividades econômicas em escala mundial. Nesse contexto, a Educação Ambiental desempenha um papel fundamental ao abordar essas questões de forma crítica e reflexiva, destacando-se a vertente crítica, que busca aprofundar a compreensão sobre as relações de poder e os conflitos socioambientais. Além disso, a tecnologia tem se consolidado como uma ferramenta essencial para potencializar a aprendizagem e estimular novas formas de ensino, tornando-se uma estratégia promissora para o ensino de Ciências. Este trabalho investiga o desenvolvimento do aplicativo Gaia em Alerta como ferramenta para o ensino de Ciências e Educação Ambiental, com foco no conteúdo de poluição ambiental e no estímulo ao pensamento crítico dos alunos. A pesquisa, de natureza qualitativa, foi realizada em uma escola estadual na Zona Norte do Rio de Janeiro, com alunos do 9º ano, utilizando o aplicativo já disponível no Google Play para explorar conceitos e abordagens sobre Poluição Ambiental. A implementação do aplicativo demonstrou seu potencial pedagógico, estimulando reflexões críticas e ampliando a compreensão dos alunos sobre a temática. A análise da nuvem de palavras revelou que alguns estudantes passaram a relacionar a poluição a fatores estruturais, como consumo indevido e indústria, alinhando-se à Educação Ambiental Crítica. Apesar desses avanços, observou-se a necessidade de aprofundar discussões sobre soluções sustentáveis, reforçando a importância da tecnologia na construção de aprendizagens dinâmicas e contextualizadas.

\keywords{Materiais digitais\sep Poluição\sep Ensino de Ciências\sep Educação Ambiental}
\end{abstract}

\begin{english}
\begin{abstract}
Environmental pollution stands out as one of the most urgent and visible socio-environmental issues in our daily lives. This phenomenon is frequently associated with economic activities on a global scale. In this context, Environmental Education plays a fundamental role in addressing these issues critically and reflectively, with emphasis on the critical approach, which seeks to deepen the understanding of power relations and socio-environmental conflicts. Additionally, technology has become an essential tool for enhancing learning and stimulating new teaching methods, making it a promising strategy for Science Education. This study investigates the development of the Gaia em Alerta application as a tool for teaching Science and Environmental Education, focusing on environmental pollution and fostering students’ critical thinking. The qualitative research was conducted in a public school in the North Zone of Rio de Janeiro, with 9th-grade students, using the application, which is already available on Google Play, to explore concepts and approaches related to Environmental Pollution. The implementation of the application demonstrated its pedagogical potential, stimulating critical reflections and expanding students’ understanding of the topic. The word cloud analysis revealed that some students began to relate pollution to structural factors, such as improper consumption and industry, aligning with the perspective of Critical Environmental Education. Despite these advances, there is a need to deepen discussions on sustainable solutions, reinforcing the importance of technology in building dynamic and contextualized learning experiences.

\keywords{Digital Materials\sep Pollution\sep Science Teaching\sep Environmental Education}
\end{abstract}
\end{english}
% if there is another abstract, insert it here using the same scheme
\end{polyabstract}

\section{Introdução}\label{sec-intro}
A poluição ambiental, seja do ar, do solo ou das águas, é um dos grandes desafios contemporâneos, frequentemente associado às atividades econômicas ao redor do mundo. Embora esse processo tenha se iniciado com as primeiras explorações de recursos naturais, foi durante a Revolução Industrial, no século XVIII, que seu agravamento se tornou expressivo \cite{siqueira2001}. Esse marco histórico, iniciado na Inglaterra, impulsionou avanços tecnológicos e o aumento da produção industrial \cite{pott2017}, ao mesmo tempo em que intensificou o uso de combustíveis fósseis e químicos, cujos impactos ambientais e sociais, muitas vezes, não foram considerados em uma escala global \cite{anjos2022}.

A poluição pode ser definida como a contaminação do ambiente por agentes que comprometem a qualidade de vida e o bem-estar dos seres vivos \cite{ribeiro2019}. Seus efeitos negativos ao meio ambiente, bem como às sociedades humanas, trouxeram um problema central nos debates ambientais e educacionais, uma vez que a conscientização sobre esses impactos é essencial para a construção de soluções sustentáveis \cite{mattos2013}. Nesse sentido, a Educação Ambiental desempenha um papel fundamental ao abordar essas questões de forma crítica e reflexiva. Dentre suas principais vertentes pedagógicas, destacam-se três macrotendências: Conservadora, Pragmática e Crítica \cite{layrargues2014}. A Educação Ambiental Crítica, em particular, propõe aprofundar a compreensão sobre as relações de poder e os conflitos socioambientais, ampliando a percepção da interdependência entre seres humanos e natureza \cite{guimaraes2004, queiroz2019}.

No contexto educacional, a tecnologia tem se consolidado como uma ferramenta essencial para potencializar a aprendizagem e estimular novas formas de ensino. Desde cedo, crianças e adolescentes têm contato com dispositivos como celulares e computadores, tornando a integração dessas tecnologias ao ambiente escolar uma estratégia promissora para o ensino de Ciências \cite{santos2017tecnologias}. O avanço das Tecnologias Digitais de Informação e Comunicação (TDICs) trouxe novas possibilidades pedagógicas, possibilitando a criação de metodologias mais interativas e dinâmicas, que vão ao encontro das necessidades das novas gerações \cite{santos2020}.

Dentre essas inovações, destaca-se o conceito de aprendizagem móvel, que vem sendo amplamente discutido por educadores e pesquisadores devido ao seu potencial de transformar experiências pedagógicas em vivências mais imersivas e acessíveis. O uso de dispositivos móveis permite que os alunos naveguem por vídeos, textos, mapas e outras mídias interativas, ampliando suas oportunidades de aprendizagem para além dos limites físicos da sala de aula \cite{oliveira2017}. Além disso, a aprendizagem mediada por tecnologias digitais favorece tanto a descentralização do acesso ao conhecimento, permitindo que os conteúdos sejam explorados de maneira flexível e autônoma, quanto a diversificação dos formatos didáticos, tornando o processo educativo mais inclusivo e engajador \cite{bortolazzo2020}.

Diante desse cenário, este estudo investiga o desenvolvimento do aplicativo Gaia em Alerta como ferramenta para o ensino de Ciências e Educação Ambiental, com foco no conteúdo de poluição ambiental e no estímulo ao pensamento crítico dos alunos.

\section{Metodologia}
O presente trabalho é de caráter qualitativo, comumente utilizado para investigar as opiniões, pensamentos e percepções dos público alvo, permitindo assim a construção de novos conhecimentos no processo da pesquisa \cite{minayo2000}. Esta última está baseada na metodologia de pesquisa-ação, que envolve a interação colaborativa entre o pesquisador e os participantes do estudo na identificação de problemas de caráter coletivo. Essa abordagem promove o empoderamento dos participantes ao incluí-los ativamente na investigação do problema e na busca por soluções \cite{costa2019}.

A pesquisa foi submetida ao Núcleo de Extensão, Pesquisa e Editoração (NEPE) do Instituto de Aplicação Fernando Rodrigues da Silveira -- CAp-UERJ, passando por uma comissão de pesquisa e no dia 01 de novembro de 2023, o projeto recebeu a Carta de autorização para pesquisa, uma vez que fosse enviado o Parecer Consubstanciado do Comitê de Ética (CEP). Após essa primeira etapa, foi submetido ao Comitê de Ética e Pesquisa Nacional Plataforma Brasil, conforme os procedimentos éticos da resolução 466/12 do Conselho Nacional de Saúde (CNS), no que diz respeito à pesquisa com seres humanos. Foi aceita no dia 20 de fevereiro de 2024 sob o parecer Nº 6.657.820.

O público-alvo deste estudo foi composto por alunos do 9º ano de uma escola pública estadual na Zona Norte do Rio de Janeiro. A participação exigiu a assinatura do Termo de Consentimento Livre e Esclarecido (TCLE) e do Termo de Assentimento pelos responsáveis, devido à idade dos alunos.


\subsection{Conteúdo e construção do aplicativo}
O tema Poluição Ambiental foi escolhido para o aplicativo por estar alinhado à grade curricular da turma e inserido no contexto da Educação Ambiental. Além disso, a abordagem seguiu a perspectiva da Educação Ambiental Crítica para embasar o conteúdo a ser trabalhado.

A revisão bibliográfica foi realizada em plataformas como Google Acadêmico, Scielo e CAPES, com as palavras-chave ``educação ambiental'', ``material digital'', ``tecnologia digital'', ``ensino de ciências'' e ``ensino fundamental'', combinadas entre si. Foram priorizados periódicos Qualis A ou B (2017-2020). Além disso, reportagens relevantes foram analisadas para exemplificar fenômenos da poluição no cotidiano.

O processo de criação do aplicativo iniciou a partir da interação com os pesquisadores do Laboratório de Comunicação Celular (LCC) -- Fiocruz, onde foi estabelecida uma parceria para o desenvolvimento do aplicativo. Foram realizadas diversas reuniões nas quais foram definidos os objetivos gerais do software. Após isso, o conteúdo estabelecido começou a ser esboçado em slides de apresentação no Canva para que todos os passos a serem executados na linguagem da programação, já partissem de um norte planejado anteriormente.

Para a criação do software, foi utilizado o React Native, um \textit{framework} multiplataforma, criado pelo Facebook, que permite o desenvolvimento para WEB, IOS (sistema de Iphones) e Android (sistema de celulares como Samsung, Motorola, entre outros).

Com base no planejamento no Canva, as ideias foram convertidas em códigos de programação, na linguagem JavaScript, iniciando o desenvolvimento do aplicativo. O React Native permitiu a visualização em tempo real das alterações no código, agilizando o processo. As imagens e vídeos do aplicativo foram gerados por uma rede neural generativa ou retirados do Canva \textit{Premium}. Além disso, o material ficou disponível em formato web, permitindo o acesso em computadores, notebooks, smartphones e tablets com conexão à internet.

\subsection{Coleta e análise de dados}
A coleta de dados foi realizada por meio de um questionário em duas etapas, aplicado após as devidas autorizações necessárias. A primeira etapa investigou o perfil dos alunos, e a relação deles com o acesso à tecnologia e à internet no cotidiano. Na segunda, buscou-se compreender os métodos alternativos de aprendizagem utilizados em sala e explorar o conteúdo de ``Poluição Ambiental'', por meio de perguntas fechadas e abertas bem específicas, para que não se fuja muito do padrão de respostas. Também foi incluído um quadro de adjetivos para avaliar as percepções dos alunos sobre o material e um tópico voltado à criação de uma ``Nuvem de Palavras''. As perguntas foram baseadas em questionários de estudos anteriores sobre o uso de materiais alternativos no ensino de Ciências.

Os dados coletados no questionário, tanto da primeira etapa quanto da segunda, serão examinados à luz da metodologia de análise de conteúdo proposta por \textcite{bardin2016}, por serem dados de natureza qualitativa. Seguindo as três etapas que consistem em pré-análise, exploração do material e tratamento dos resultados, inferência e interpretação \cite{bardin2016}.

\section{Resultados}
\subsection{Aplicativo ``Gaia em Alerta''}
O aplicativo ``Gaia em Alerta'' foi desenvolvido como uma ferramenta educacional voltada para dinâmicas que utilizem a temática da poluição ambiental. Ele está disponível para dispositivos Android e pode ser baixado gratuitamente na loja virtual Google Play. Também há a opção de acessá-lo por meio de navegadores, através do modelo Web no link: \url{https://cienciaimago.com/projeto_thais/site/index.html}.

Ao abrir o aplicativo, o usuário é recebido pelo Menu Principal, que apresenta uma interface intuitiva e simples de navegar. O menu conta com as seguintes abas:

\begin{itemize}
    \item Introdução: Esta seção oferece uma visão geral sobre o tema da poluição ambiental e seus diferentes tipos existentes.
    \item Sobre: Nesta aba, o usuário encontra informações sobre o período de desenvolvimento do aplicativo, os autores responsáveis e seu principal objetivo.
    \item Conceitos: Uma coleção de conceitos-chave relacionados à poluição ambiental, apresentados de forma clara e didática. Essas diversas abas foram projetadas para servir como um suporte teórico tanto para professores quanto para alunos.
    \item Tour por Gaia: A atividade interativa é o destaque do aplicativo. Durante o ``Tour por Gaia'', os usuários podem explorar diferentes situações, identificando problemas de poluição e propondo soluções. Esta funcionalidade incentiva o pensamento crítico e a aplicação prática dos conceitos abordados.
    \item Bibliografia: Nesta aba são encontradas as referências bibliográficas utilizadas para o desenvolvimento do conteúdo do aplicativo.
\end{itemize}

\begin{figure}[h!]
\centering
\begin{minipage}{.35\textwidth}
\includegraphics[width =\textwidth]{Imagens/fig001.png}
\caption{\textit{Print} do Menu Principal do aplicativo Gaia em Alerta.}
\label{fig-1}
\source{A autora.}
\end{minipage}
\end{figure}

Cada um dos conceitos listados -- como maré negra e chuva ácida -- segue um padrão dinâmico (\Cref{fig-1}). Ao clicar em um conceito, o usuário encontrará as opções:

\begin{itemize}
    \item Definição: Explicação clara e objetiva do conceito.
    \item Impactos: Um resumo dos principais impactos ambientais, sociais e econômicos associados ao tema.
    \item Acontecimentos: Exemplos práticos e situações reais, ilustrados a partir de notícias verídicas que mostram a ocorrência do conceito no contexto da vida real.
\end{itemize}

O aplicativo Gaia em Alerta conta com a atividade ``Tour por Gaia'' (Figura \ref{fig-2}), projetada para reforçar os conceitos apresentados no menu principal por meio de uma experiência interativa. Nessa atividade, os usuários exploram diferentes cartas de situações, cada uma abordando um problema ambiental e incentivando a reflexão sobre possíveis soluções. Ao resolver os desafios propostos, o aluno pode clicar no botão representado por um presente para abrir o Selo de reconhecimento, simbolizado por uma placa que representa sua contribuição para a mitigação dos impactos ambientais. Entre os selos disponíveis estão por exemplo ``Saúde Planetária'', ``Ação Coletiva'' e ``Ação Individual'', cada um acompanhado de uma explicação sobre o seu significado.

\begin{figure}[h!]
\centering
\begin{minipage}{.35\textwidth}
\includegraphics[width =\textwidth]{Imagens/fig002.png}
\caption{\textit{Print} do início da atividade Tour por Gaia.}
\label{fig-2}
\source{A autora.}
\end{minipage}
\end{figure}

\subsection{Perfil dos alunos}
O gráfico da \Cref{fig-3} foi montado a partir da união das perguntas fechadas presentes no primeiro questionário para mapear o perfil dos participantes da pesquisa, considerando cinco questões principais: gênero, idade, histórico de escolaridade, acesso à internet e posse de celular com acesso à internet. A maioria dos participantes se identifica como feminino (10), enquanto 4 se identificam como masculino. Não houve respostas para as categorias ``Outros'' e ``Prefiro não dizer''. Todos os participantes possuem idade entre 14 e 16 anos. Quanto ao histórico de escolaridade, 8 alunos indicaram que até o presente momento sempre estudaram em escola pública, enquanto 6 declararam o contrário, que não estudaram em escola pública desde sempre. Em relação ao acesso à internet, 13 participantes afirmaram ter internet em casa, mencionando tanto conexão por Wi-Fi quanto por dados móveis, enquanto apenas 1 declarou não ter acesso. Por fim, todos os participantes afirmaram possuir celular com acesso à internet.

Ainda no primeiro questionário perguntou-se aos alunos quais recursos didáticos a/o professor(a) de Biologia costumava utilizar. Eles puderam listar mais de uma opção em suas respostas, uma vez que se tratava de uma pergunta aberta. Os resultados revelaram que a maioria dos participantes (12) indicou o quadro, seguido pelas folhas de estudo dirigido (10), elaboradas pela própria professora sobre a temática da aula, como os materiais mais utilizados para apoiar as aulas. Esta resposta demonstra que geralmente as aulas ocorrem apenas com os materiais tradicionais de uma sala de aula (\Cref{fig-4}).

\begin{figure}[h!]
\centering
\begin{minipage}{.90\textwidth}
\includegraphics[width =\textwidth]{Imagens/Graficos-1.png}
\caption{Perfil dos alunos participantes da pesquisa.}
\label{fig-3}
\source{A autora.}
\end{minipage}
\end{figure}

\begin{figure}[h!]
\centering
\begin{minipage}{.90\textwidth}
\includegraphics[width =\textwidth]{Imagens/Graficos-2.png}
\caption{Recursos didáticos utilizados pelo professor.}
\label{fig-4}
\source{A autora.}
\end{minipage}
\end{figure}

Segundo dados da \textcite{gsma2024}, no ano de 2023, aproximadamente 69\% da população mundial, cerca de 5,6 bilhões de pessoas, possuía um serviço móvel, representando um crescimento significativo em relação a uma pesquisa feita em 2015. O uso da internet móvel apresentou uma expansão ainda mais acelerada, alcançando 58\% da população global, o equivalente a 4,7 bilhões de usuários.

Em relação ao Brasil, o \textcite{ibge2023} publicou um comunicado sobre 72,5 milhões de domicílios terem acesso à Internet, o que representa 92,5\% da população brasileira.

É notório que as sociedades, mesmo que em diferentes contextos de desenvolvimento econômico, estão imersas no mundo tecnológico, revelando assim um novo contexto sociocultural. Nesse cenário, a conectividade entre as pessoas se torna ampliada através da internet e do uso de tecnologias/plataformas digitais \cite{reboucas2019interatividade}.

\subsection{A perspectiva sobre Poluição Ambiental}
A pergunta ``O que é Poluição Ambiental?'', feita na primeira etapa do questionário, revelou três temas principais, que evidenciam diferentes compreensões sobre a temática. Como a pergunta era aberta e as respostas não eram excludentes, alguns alunos abordaram mais de um aspecto, o que demonstra as diferentes percepções sobre Poluição Ambiental, como pode ser observado na Tabela \ref{tab-1}.

\newpage
%--- CÓDIGO DA TABELA 1 ---%
\begin{table}[h]
    \centering
    \begin{threeparttable}
    \caption{``O que é Poluição Ambiental?'' Pergunta feita antes do uso do aplicativo.}
    \label{tab-1}
    \begin{tabular}{p{3cm} p{7cm} p{3cm}} 
        \toprule
        Subcategorias & Conceitos Norteadores & Estudantes \\
        \midrule
        Resíduos sólidos & Consulte ao descarte inadequado de lixo, como plásticos, papéis e embalagens, que poluem o meio ambiente. & E1, E2, E3, E9, E10, E11, E14, E8 \\
        Impactos ambientais & Abrange queimadas, desmatamento, desequilíbrio ecológico e outros danos ao meio ambiente. & E1, E3, E4, E5, E6, E7 \\
        Consequências para saúde e bem-estar & Enfatiza como a poluição relativa à saúde, segurança, bem-estar e a biodiversidade. & E1, E6, E12, E13 \\
        \bottomrule
    \end{tabular}
    \source{A autora.}
    \end{threeparttable}
\end{table}


O tema mais recorrente foi a abordagem sobre ``Resíduos Sólidos'', classificado por 57\% dos alunos, indicando que a poluição ambiental é frequentemente associada ao descarte inadequado de lixo, evidenciando uma visão mais cotidiana do problema, de uma maneira mais tangencial e ligada a aspectos visíveis e imediatos. Um exemplo dessa abordagem é falar sobre ``Lixos na rua'' (E1, E3, E11), para estes Poluição Ambiental é, por exemplo:

\begin{quote}
    ``Ruas com lixo, esgoto com lixo e lixo nas praias." -- E11
\end{quote}

Isso reflete o que é discutido por \textcite{layrargues2014}, ao ressaltar que as abordagens conservadoras e pragmáticas da educação ambiental, focadas em ações individuais e comportamentais no âmbito doméstico e privado, muitas vezes ignoram aspectos históricos e políticos. \textcite{beyer2024}, ao analisar ``como as pesquisas relacionadas ao ensino de Ciências e o Livro Didático retratam a Educação Ambiental'' também obtiveram como resultado uma predominância das abordagens conservacionistas e pragmáticas, que se concentram em soluções práticas e na preservação ambiental, mas geralmente não abordam as dinâmicas sociais e políticas.

\textcite{defreyn2022} mostram em seu trabalho como as abordagens conservadoras e pragmáticas, caracterizadas pelo foco no comportamento individual, ainda são as mais recorrentes. \textcite{rodrigues2019} analisaram algumas práticas didático-pedagógicas em Educação Ambiental e identificaram que a vertente pragmática foi predominante. Os autores revelam que maioria dos trabalhos analisados na pesquisa se concentrou em temas como ``Lixo'' e ``Reciclagem'', enquanto discussões mais amplas sobre questões socioambientais foram pouco exploradas. Esse cenário mostra uma abordagem escolar focada nas mudanças de comportamentos individuais em relação ao meio ambiente, sem aprofundamento nas causas e impactos estruturais dos problemas ambientais.

Vale ressaltar que o participante E8 mencionou ``...aquilo que degrada o ambiente, como por exemplo o lixo que polui o ambiente impossibilitando seu desenvolvimento'', falando sobre o comprometimento do desenvolvimento do ambiente, porém, ainda fixado na ideia do ``lixo''.

Em seguida, ``Impactos Ambientais'' foi apontado para 43\% dos alunos, trazendo questões como desmatamento, alagamentos e perda da biodiversidade. E por fim, ``Consequências na Saúde e Bem-Estar'' foi classificado por 28,5\% dos participantes. Esse dado mostra uma compreensão ampliada do problema, indo além da poluição visível e considerando suas consequências mais profundas para os ecossistemas. A poluição atmosférica é o principal agente de degradação da Terra \cite{cruz2019} e um dos grandes problemas da saúde pública \cite{ribeiro2019}.

Como o eixo menos mencionado nas respostas nota-se que é necessário desconstruir a relação antropocêntrica estabelecida com o meio ambiente \cite{rambo2019vivencia}. A Educação Ambiental é um processo formativo que aguça o senso crítico das pessoas, destacando a importância de um ensino interdisciplinar que, desde o ensino básico, promova a conscientização dos indivíduos sobre as questões socioambientais e sua responsabilidade na transformação do meio. \cite{velozo2022}. A qualidade de vida das pessoas depende do ambiente em que vivem, que muitas vezes é afetado pelas ações humanas. Portanto, essas mudanças no meio ambiente podem causar problemas que também impactam diretamente a saúde da população \cite{alencar2020}.

Evidenciando isso, \textcite{alencar2020} perceberam em seu trabalho que a Educação Ambiental nas escolas ainda é majoritariamente conservadora, refletindo a abordagem limitada dos livros didáticos. Com a vertente crítica pouco explorada, muitos educadores reproduzem essa perspectiva sem perceber, evidenciando a necessidade de ampliar debates que aprofundem a compreensão dos problemas socioambientais.

Após a realização da atividade ``Tour por Gaia'', os alunos responderam a outra pergunta do questionário, agora na segunda etapa do trabalho. A pergunta ``Cite três palavras ou expressões que você lembra ao falarmos de Poluição Ambiental'' originou uma nuvem de palavras (Figura \ref{fig-5}) que mostra que os alunos diversificaram suas respostas e incorporaram conceitos trabalhados na aula com o aplicativo. A intenção de se trabalhar com a perspectiva da Educação Ambiental Crítica, é que essa vertente busca criar espaços educacionais que incentivem a intervenção em problemas reais, promovendo um processo educativo que apoie e responda positivamente à atual crise socioambiental \cite{queiroz2019}.

\begin{figure}[h!]
\centering
\begin{minipage}{.85\textwidth}
\includegraphics[width =\textwidth]{Imagens/fig003.png}
\caption{Nuvem de palavras ``O que é Poluição Ambiental?''.}
\label{fig-5}
\source{A autora.}
\end{minipage}
\end{figure}

A análise da nuvem de palavras revela um repertório conceitual diversificado. Termos como ``efeito estufa'', ``chuva ácida'', ``desmatamento'' e ``maré negra'' foram os mais mencionados, indicando que os alunos associam a poluição a fenômenos ambientais amplamente debatidos no contexto escolar. Além disso, a presença de palavras como ``microplásticos'', ``petróleo'', ``resíduos'' e ``indústria'' sugere uma compreensão da poluição não apenas como um fenômeno isolado, mas também como consequência de processos produtivos e padrões de consumo. 

Adicionalmente, a inclusão de palavras relacionadas à mitigação dos impactos ambientais, como ``energia sustentável'', ``reciclagem'' e ``tratamento de esgoto'', relaciona a conexão feita pelos alunos entre problemas da poluição e as possíveis soluções. \textcite{campos2022} fizeram um estudo para investigar em produções acadêmicas as macrotendências identificadas em projetos, dessa vez não escolares, mas também obtiveram como resultado uma predominância da vertente conservadora, seguida da pragmática. Em relação a abordagem pragmática, foram destacados temas como resíduos sólidos, gerenciamento de resíduos e segurança/risco ambiental. Dessa maneira, notaram que a perspectiva crítica é uma tendência ainda pouco representativa.

Observa-se também a inclusão de termos como ``consumo indevido'', ``indústria'' e ``exploração do ambiente'', que apontam para uma visão mais crítica das questões ambientais. A presença dessas expressões sugere que parte dos alunos reconhece que os impactos ambientais não decorrem apenas do descarte inadequado de resíduos, mas também de dinâmicas econômicas e sociais que intensificam a degradação ambiental. Essa abordagem se alinha aos princípios da Educação Ambiental Crítica (EAC), que busca evidenciar as relações estruturais entre sociedade e meio ambiente, promovendo reflexões sobre a necessidade de mudanças sistêmicas.

Dessa forma, os dados demonstram que os alunos não apenas internalizaram conceitos fundamentais sobre poluição ambiental, mas também começaram a estabelecer conexões mais amplas com aspectos sociais e econômicos. A ampliação desse repertório e o fortalecimento da perspectiva crítica podem contribuir para um entendimento mais aprofundado sobre as causas estruturais da crise ambiental, alinhando-se aos objetivos da Educação Ambiental Crítica. A formação inicial e continuada dos docentes é apontada como essencial para que a EA seja um princípio estruturante da educação, possibilitando que alunos desenvolvam consciência crítica e assumam responsabilidades ambientais \cite{rosa2024, defreyn2022}. Porém, ainda há uma lacuna na discussão sobre políticas públicas de EA, com poucos estudos abordando essa questão de forma aprofundada \cite{rosa2024}.


\subsection{Avaliação sobre a ferramenta}
Após a utilização do aplicativo em sala de aula, foi avaliada a opinião dos alunos sobre o conteúdo do aplicativo. Para isso, foi apresentada uma lista de adjetivos, e a pergunta era do tipo múltipla escolha, permitindo que os participantes selecionassem as opções, sem limite estabelecido, que julgassem adequadas. Os resultados mostraram que todos os participantes (14) consideraram o conteúdo útil para a sua aprendizagem. Além disso, adjetivos como `bem apresentado' e `indicaria' também obtiveram alto índice de marcação, com 12 menções cada. `Legal' e `fácil' foram assinalados por 11 participantes, reforçando a percepção positiva do material. Não foram registradas opiniões negativas, como `chato', `difícil', `cansativo' ou `não indicaria' (Figura \ref{fig-6}).

\begin{figure}[h!]
\centering
\begin{minipage}{.80\textwidth}
\includegraphics[width =\textwidth]{Imagens/Graficos-3.png}
\caption{Opinião dos participantes sobre o conteúdo do aplicativo.}
\label{fig-6}
\source{A autora.}
\end{minipage}
\end{figure}

\textcite{aguiar2022} destacam que um dos principais benefícios do uso desses dispositivos em sala de aula é o impacto positivo no desenvolvimento da autonomia e do protagonismo dos estudantes. Uma característica essencial vinculada ao digital e suas vastas possibilidades é a promoção de uma educação personalizada, onde os processos de aprendizagem podem ocorrer por meio de diversos dispositivos e aplicativos, disponíveis a qualquer hora e lugar \cite{bortolazzo2020}.

\textcite{carmelo2021} falam sobre a necessidade de compreender a relação entre o ensino formal estruturado e o potencial transformador das tecnologias na educação. Uma vez que essas são um fator essencial para otimizar o ensino e enriquecer o aprendizado programado, tornando-o mais interativo, dinâmico e conectado com as demandas e interesses das novas gerações.

Na pergunta ``Você gostaria de ter mais dinâmicas como essa nas aulas de Biologia? Por quê?'', as respostas foram agrupadas em três categorias principais: ``Aulas mais dinâmicas e envolventes'', ``Facilidade de aprendizado e compreensão'' e ``Interação e uso da tecnologia no ensino'', conforme descrito na \Cref{tab-2}.

A categoria mais mencionada foi ``Aulas mais dinâmicas e envolventes'', com diferença de apenas um estudante em relação a segunda que foi abordada ``Facilidade de aprendizado e compreensão'', mencionada por cinco estudantes. Na primeira (E1, E7, E9, E10, E12 e E13), as respostas indicaram que a dinâmica tornou as aulas mais interessantes e motivadoras, estimulando o engajamento dos alunos. Já na segunda (E5, E6, E8, E11 e E14), os alunos destacaram que a atividade auxiliou na absorção do conteúdo e no aprimoramento da compreensão dos temas abordados.

Destaco o aluno E5, que comenta ``isso melhora minha compreensão; sair da rotina me ajuda a absorver melhor a matéria''.

\textcite{tavares2019} revelam, em seu estudo, que o uso cotidiano de tecnologias digitais é uma prática comum entre alunos. Eles possuem acesso a uma variedade de dispositivos, incluindo notebooks, tablets, computadores de mesa e smartphones, demonstrando a presença constante dessas ferramentas em seu dia a dia. E assim, indicam que a cibercultura influencia a aprendizagem dos jovens nativos digitais, uma vez que essas tecnologias digitais fazem parte de sua rotina tanto para acessar informações quanto para compreender conteúdos escolares.

Por fim, a categoria ``Interação e uso da tecnologia no ensino'' foi a menos mencionada, sendo citada por apenas três estudantes (E2, E3 e E4). Nessa categoria, as respostas ressaltaram que o uso de tecnologia e aplicativos favorece o aprendizado e promove uma experiência inovadora.

Com o avanço do virtual e o impacto crescente da conectividade, a relação com o saber se transforma, exigindo uma busca contínua por conhecimento. Essa dinâmica impõe aos sistemas educacionais a necessidade de constante atualização, já que os saberes se renovam em alta velocidade. Nesse cenário, o foco deixa de ser apenas a transmissão de conteúdo, passando a ser o desenvolvimento de competências que favoreçam a circulação e apropriação do conhecimento em um mundo cada vez mais conectado \cite{levy1999cibercultura, trindade2023}.

%--- CÓDIGO DA TABELA 2---%
\begin{table}[h]
    \centering
    \begin{threeparttable}
    \caption{Motivos pelos quais os alunos gostariam de ter mais aulas com essa dinâmica.}
    \label{tab-2}
    \begin{tabular}{p{5cm} p{5.5cm} p{2.5cm}} 
        \toprule
        Categoria & Conceito Norteador & Estudantes \\
        \midrule
        Aulas mais dinâmicas e envolventes & A dinâmica torna as aulas mais interessantes e motivadoras, estimulando o engajamento dos alunos. & E1, E7, E9, E10, E12, E13 \\
        Facilidade de aprendizado e compreensão & A atividade contribuiu para uma melhor absorção do conteúdo e aprimoramento da compreensão dos temas.
 & E5, E6, E8, E11, E14 \\
        Interação e uso da tecnologia no ensino & O uso de tecnologia e aplicativos favorece o aprendizado, promovendo uma experiência inovadora. & E2, E3, E4 \\
        \bottomrule
    \end{tabular}
    \source{A autora.}
    \end{threeparttable}
\end{table}

\section{Conclusão}
Os resultados obtidos com a implementação do aplicativo Gaia em Alerta evidenciam seu potencial como ferramenta pedagógica no ensino de Ciências, especialmente no contexto da Educação Ambiental Crítica. A aceitação positiva por parte dos alunos, expressa pelos adjetivos utilizados na avaliação do material, demonstra que a tecnologia pode atuar como um recurso significativo para tornar as aulas mais dinâmicas, envolventes e conectadas às realidades dos estudantes.

Além de despertar interesse, o uso do aplicativo contribuiu para ampliar a compreensão dos alunos sobre a poluição ambiental, permitindo que transcendesse uma visão restrita à geração de resíduos sólidos e passasse a considerar impactos mais amplos, como consequências sociais e ambientais. A atividade Tour por Gaia incentivou reflexões críticas e fomentou a construção de conhecimentos que vão além da mera reprodução de conceitos tradicionais, favorecendo a interatividade e o pensamento crítico. Respostas presentes na segunda etapa do questionário, como ``o desmatamento acaba matando os animais'', indicam que alguns alunos já conectam o desgaste ambiental com a perda da fauna e da flora, embora essa relação ainda precise ser aprofundada em atividades futuras.

Ainda na análise da nuvem de palavras, observou-se que alguns alunos incorporaram termos como consumo indevido e indústria, evidenciando uma compreensão mais ampla sobre os fatores que agravam a poluição. Essas palavras indicam uma relação direta com o modelo econômico vigente e sua influência sobre os problemas ambientais, assim como defende a vertente de Educação Ambiental Crítica, sugerindo que parte dos estudantes passou a perceber a poluição não apenas como ``resíduos sólidos'', como a maioria havia indicado na primeira etapa do questionário, mas como um fenômeno complexo e estruturalmente condicionado.

Contudo, os dados também indicam a necessidade de aprofundar discussões sobre soluções e estratégias sustentáveis para a mitigação dos impactos ambientais. Apesar de avanços na ampliação do repertório conceitual dos alunos, termos como reciclagem e energia sustentável ainda apareceram com menor frequência na nuvem de palavras, sugerindo a necessidade de fortalecer o debate sobre práticas efetivas de transformação socioambiental.

Dessa forma, o estudo reforça a relevância da integração de recursos tecnológicos no ensino de Ciências, evidenciando que o uso de aplicativos pode não apenas engajar os alunos, mas também promover aprendizagens mais contextualizadas e alinhadas às demandas contemporâneas da Educação Ambiental.


\printbibliography\label{sec-bib}
% if the text is not in Portuguese, it might be necessary to use the code below instead to print the correct ABNT abbreviations [s.n.], [s.l.]
%\begin{portuguese}
%\printbibliography[title={Bibliography}]
%\end{portuguese}


%full list: conceptualization,datacuration,formalanalysis,funding,investigation,methodology,projadm,resources,software,supervision,validation,visualization,writing,review
\begin{contributors}[sec-contributors]
\authorcontribution{Thais Lemos dos Santos}[conceptualization,datacuration,investigation,methodology,visualization,writing]
\authorcontribution{Paulo Roberto Soares Stephens}[funding,methodology,projadm,supervision,review]
\end{contributors}

\begin{dataavailability}
\txtdataavailability{databody} % options: dataavailable, dataonly, databody, datanotav, nodata
\end{dataavailability}


\end{document}

