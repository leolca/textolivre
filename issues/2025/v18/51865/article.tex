\documentclass[portuguese]{textolivre}

% metadata
\journalname{Texto Livre}
\thevolume{18}
%\thenumber{1} % old template
\theyear{2025}
\receiveddate{\DTMdisplaydate{2024}{3}{29}{-1}}
\accepteddate{\DTMdisplaydate{2024}{9}{02}{-1}}
\publisheddate{\DTMdisplaydate{2025}{2}{05}{-1}}
\corrauthor{Késsia Mileny de Paulo Moura}
\articledoi{10.1590/1983-3652.2025.51865}
%\articleid{NNNN} % if the article ID is not the last 5 numbers of its DOI, provide it using \articleid{} commmand 
% list of available sesscions in the journal: articles, dossier, reports, essays, reviews, interviews, editorial
\articlesessionname{articles}
\runningauthor{Lopes e Moura}
%\editorname{Leonardo Araújo} % old template
\sectioneditorname{Daniervelin Pereira}
\layouteditorname{Leonardo Araújo}

\title{Aprendizagens repercutidas a partir da metodologia \textit{Digital Storytelling}: estudo de caso em uma turma de Pedagogia}
\othertitle{Repercussed learning from Digital Storytelling methodology: case study in a Pedagogy class}

\author[1]{Marina Rodrigues da Silva Lopes~\orcid{0000-0003-4128-4583}\thanks{Email: \href{mailto:rodrigues.marina@discente.ufma.br}{rodrigues.marina@discente.ufma.br}}}
\author[1]{Késsia Mileny de Paulo Moura~\orcid{0000-0002-5124-1432}\thanks{Email: \href{mailto:kessiamileny@yahoo.com.br}{kessiamileny@yahoo.com.br}}}
\affil[1]{Universidade Federal do Maranhão, CCIM, Imperatriz, MA, Brasil.}

\addbibresource{article.bib}

\begin{document}
\maketitle
\begin{polyabstract}
\begin{abstract}
As histórias estão em toda a parte e são utilizadas para motivar os outros, transmitir informações, partilhar experiências, socializar e se conectar com novas pessoas. A habilidade de contar/comunicar histórias permite transcender estruturas pessoais e assumir perspectivas mais amplas, tornando-se uma importante ferramenta de aprendizagem. A metodologia \textit{Digital Storytelling}, por sua vez, tem se mostrado um mecanismo que pode fomentar aprendizagens. Este trabalho procurou captar as aprendizagens/habilidades obtidas por discentes do curso de pedagogia de uma universidade do nordeste brasileiro com a metodologia \textit{Digital Storytelling} em formato audiovisual. Recorreu-se a autores como \textcite{rodrigues2020narrativas,rodrigues2023para,pasinato2023educacao,moura2023narrativa,prado2017narrativas,cruz2016letramentos,robin2006usos}, dentre outros, para discutir as potencialidades da \textit{Digital Storytelling}. Teve como instrumento metodológico que subsidiou a geração de dados uma entrevista de grupo focal com 12 estudantes de Pedagogia de uma universidade pública do Nordeste, que buscou captar as mudanças em termos de aprendizagens, após a participação destes em uma oficina que versou sobre a produção de \textit{Digital Storytelling}. Como resultado, os participantes desenvolveram diversas habilidades, que podem reverberar em aprendizagens relevantes para a sua formação.

\keywords{Metodologias ativas \sep Digital storytelling \sep Aprendizagens \sep Habilidades}
\end{abstract}

\begin{english}
\begin{abstract}
Stories are everywhere and are used to motivate others, convey information, share experiences, socialize and connect with new people. The ability to tell/communicate stories allows you to transcend personal structures and take on broader perspectives, becoming an important learning tool. The digital storytelling methodology, in turn, has proven to be a mechanism that can foster learning. This work sought to capture the learning/skills obtained by students on the pedagogy course at a university in northeastern Brazil using the Digital Storytelling methodology in audiovisual format. Authors such as \textcite{rodrigues2020narrativas,rodrigues2023para,pasinato2023educacao,moura2023narrativa,prado2017narrativas,cruz2016letramentos,robin2006usos}, among others, were used to discuss the potential of digital storytelling. The methodological instrument that supported the generation of data was a focus group interview with 12 pedagogy students from a public university in the northeast, which sought to capture changes in terms of learning, after their participation in a workshop that focused on the production of digital storytelling. As a result, participants developed several skills, which can result in learning relevant to their training.

\keywords{Active methodologies \sep Digital storytelling \sep Learnings \sep Skills}
\end{abstract}
\end{english}
\end{polyabstract}

\section{Introdução}
Mediada pelo uso de novos recursos, ferramentas e metodologias, no contexto da cibercultura \cite{levy2001cibercultura}, surge a metodologia \textit{Digital Storytelling}, fenômeno que utiliza de mídias digitais para a contação, construção, interação e propagação de histórias compartilhadas em plataformas digitais. A narrativa, nesse contexto, assume a “roupagem” adequada às tendências do seu tempo \cite[p.71]{busatto2005narrando}, ou seja, a \textit{Digital Storytelling} utiliza-se de meios e múltiplas linguagens contemporâneas para sua composição, marca indelével desse gênero, que se opera em múltiplas linguagens e plataformas midiáticas utilizadas em sua construção e compartilhamento, em detrimento da oralidade ou escrita como são as narrativas tradicionais.

Sobre as narrativas digitais (ND), \textcite[p.~54]{rodrigues2023para} afirmam a possibilidade da articulação da história a diferentes contextos “por meio da utilização de recursos hipermidiáticos associando objetividade e subjetividade, potencializando a hibridização de mídias para representar e encadear fatos e pensamentos em um todo significativo”, numa construção inventiva e personalizada.

Nesse limiar, em contexto educativo sua construção pode enriquecer o processo de ensino e aprendizagem e favorecer a construção de conhecimentos sobre as matérias escolares, outras linguagens e usos de tecnologias \cite{pasinato2023educacao}. Frente a essas considerações, o presente trabalho propõe uma investigação acerca das potencialidades formativas trazidas pela metodologia \textit{Digital Storytelling}, com o seguinte objetivo geral: captar as aprendizagens/habilidades obtidas por discentes do curso de Pedagogia de uma universidade do nordeste brasileiro, a partir do uso da metodologia \textit{Digital Storytelling}, em formato audiovisual.

Este trabalho justifica-se pelas mudanças que os processos de ensino e seus sujeitos vêm passando nos últimos anos e a narrativa digital emerge como uma possibilidade de metodologia que pode desenvolver habilidades quanto à autonomia, personalização, autoria, criticidade, trabalho com as emoções, apropriação de novas linguagens digitais fomentados por outras mídias \cite{rodrigues2023para}.

Incorporar tecnologias digitais ao currículo escolar ainda pode apresentar entraves no contexto educacional em que estamos inseridos. Esses entraves revelam que, por mais que estejamos imersos na sociedade da tecnologia e das mídias, ainda é mais cômodo sua não incorporação ao currículo educacional de forma efetiva, por uma crença de que as mídias e tecnologias seriam algo ruim para a educação, um atraso, distração desnecessária para os alunos e que em nada têm a contribuir para o processo formativo. A formação inicial de professores com e sobre as mídias e recursos digitais pode agir como um facilitador na incorporação da mídia na educação, lançando mão das metodologias ativas de ensino e aprendizagem como mediadoras da aprendizagem significativa com tecnologias que promovam criticidade e problematização quanto à transposição didática nas salas de aula da educação básica. Aqui está a relevância desta proposta.

\section{A metodologia \textit{Digital Storytelling} à luz das metodologias ativas}\label{sec-normas}
\textcite[p.~41]{bacich2018metodologias} entendem as “metodologias como grandes diretrizes que orientam os processos de ensino e aprendizagem e que se concretizam em estratégias, abordagens e técnicas concretas, específicas e diferenciadas”. Portanto, consideramos que as metodologias são técnicas utilizadas para direcionar alunos e professores em seu processo de ensino e aprendizagem. Elas são capazes de direcionar e até supor possíveis objetivos que podem ser alcançados, mas o que determinará quais os ganhos educativos advindos da metodologia adotada é a experiência que cada sujeito terá ao longo do processo.

Existem diversas metodologias que contemplam diferentes tipos de abordagens, cada uma direcionando o processo educativo para uma potencialidade. Vale ressaltar que as metodologias são dinâmicas e flexíveis, podendo ser exploradas de diferentes formas e alcançando novos resultados a depender do envolvimento dos sujeitos no processo, sobretudo um envolvimento que traga significado a partir das experiências vividas.

Partindo dessa premissa, as metodologias ativas podem ser entendidas como estratégias pedagógicas que possibilitam o direcionamento do aluno a uma aprendizagem que busca a autonomia, criatividade, pensamento analítico, envolvimento crítico com questões sociais, políticas, culturais e econômicas. Elas podem acontecer “de forma flexível, interligada e híbrida” \cite[p.~41]{bacich2018metodologias}, facilitando o interesse e a participação dos estudantes no processo de ensino e aprendizagem.

\textcite[p.~465]{valente2017metodologias} reforçam ainda que

\begin{quote}
    [...] as metodologias ativas são estratégias pedagógicas para criar oportunidades de ensino nas quais os alunos passam a ter um comportamento mais ativo, envolvendo-os de modo que eles sejam mais engajados, realizando atividades que possam auxiliar o estabelecimento de relações com o contexto, o desenvolvimento de estratégias cognitivas e o processo de construção de conhecimento.
\end{quote}

Logo, temos que o principal direcionamento trazido pelas metodologias ativas é o engajamento dos alunos, possibilitando a construção de experiências de aprendizagem que sejam significativas e, como consequência dessa interação, os estudantes sejam capazes de, em diferentes situações, agir de forma autônoma. A flexibilidade e dinamicidade trazidas pelas metodologias ativas possibilitam a abertura e incorporação das diversas tecnologias para o contexto de aprendizagem ativa dos alunos.

Diante do exposto, as metodologias ativas aproximam-se das Tecnologias Digitais da Informação e Comunicação (TDIC) à medida que essas tecnologias vão sendo incorporadas no contexto educacional. O ensino passa a ser tecnologicamente híbrido, utilizando-se de tecnologias digitais simples, e com maior uso, como celulares e computadores com acesso à internet, onde os estudantes podem acessar, produzir, interagir e compartilhar de forma instantânea suas aprendizagens.

Consideramos assim que o uso de diferentes tipos de metodologias ativas, no contexto da educação para/com/por meio das mídias, apresenta potencialidades para a formação de um estudante engajado na sua própria aprendizagem. “Existem diversos formatos de metodologias ativas que trabalham em prol de diferentes objetivos na formação do educando” \cite[p.~146]{farias2015aprendizagem}. Desse modo, a produção de \textit{Digital Storytelling} fomenta a participação dos estudantes na produção de histórias.

Com base nessa argumentação, percebe-se a potencialidade dessa proposta educacional quanto à formação integral do estudante. Para além da apropriação dos conteúdos escolares, as metodologias ativas preparam os estudantes para a vida em sociedade, estimulando a resolução de problemas em situações reais do contexto social, trabalhando as emoções, princípios da ética e valores morais, a curiosidade e autonomia desses estudantes.

Doravante, inquietações sobre práticas educativas escolares baseadas nas metodologias ativas, como verificamos, vêm sendo colocadas nas últimas décadas. O advento das TDIC trouxe implicações para a dinâmica educativa escolar e exigiu dos professores e alunos outros papéis e usos significativos desses aparatos para dinamizar o processo de ensino e aprendizagem. O uso de tecnologias de forma apropriada ou não se faz presente da Educação Infantil ao Ensino Superior, promovendo alterações nas práticas e concepções, que passam a ser apontadas com viés inovador e, ancoradas às metodologias ativas, envolvem os educandos de outra maneira, ativa e colaborativamente.

Para a sociedade contemporânea, as mídias e tecnologias estão presentes em atividades cotidianas e corriqueiras. Para termos acesso a informações, por exemplo, basta apenas acessar o Google; ler um livro também se tornou uma atividade digital, ouvir músicas em um ambiente de \textit{streaming}, interagir com pessoas por meio das redes sociais, compartilhar fotos, vídeos, histórias, e muito mais, de forma instantânea.

As mídias estão presentes nos processos de produção, reprodução e transmissão da cultura, funcionando como espaços para socialização, que possibilitam novas formas não apenas de se conectar com o conhecimento, mas também de construir, perceber-se no mundo, interagir e aprender. A vivência nesses espaços híbridos e multimodais de hiperconexão, para \textcite[p.~458]{valente2017metodologias} cria novas formas de “[...] expressar emoções, produzir e compartilhar informações e conhecimentos, assim como aporta novos elementos à aprendizagem, podendo trazer novas contribuições e desafios aos processos educativos”.

De acordo com \textcite[p.~8]{pasinato2023educacao}, o aspecto crucial de uma formação com tecnologias é a incorporação de outras formas de comunicação, uma vez que as TDIC dispõem de interessantes oportunidades para melhorar a aprendizagem, “envolvendo os alunos em experiências de aprendizagem interativas, colaborativas e personalizadas”, sobretudo em tempos de inteligência artificial, que faz toda produção por meio de um comando, conforme apontam \textcite{rodrigues2023para}.

Contudo, não basta fazer uso das mídias e tecnologias que se encontram à disposição da sociedade de forma indiscriminada, sem reflexão e análise sobre o seu uso quando falamos do ambiente escolar e da construção do conhecimento. Educar sobre/para os meios, com e a partir dos meios, exige não apenas do professor, como orientador do processo educativo, mas também do aluno, que se torna autor, produtor, espectador, crítico, criativo e reflexivo das produções que nascem mediadas pelo uso das TDIC no ambiente escolar, mas que não se limitam apenas a esse ambiente, graças ao compartilhamento das produções por meio das redes, que ultrapassam as paredes da escola e podem ganhar espectadores de diversos lugares.

Percebemos, então, a crescente necessidade da incorporação das mídias digitais ao currículo educacional e somos levados a refletir sobre como tem se dado nossa práxis pedagógica, qual o tipo de estudante temos formado e quais potencialidades as ferramentas que utilizamos em sala de aula têm despertado nesses alunos. “O que constatamos, cada vez mais, é que a aprendizagem por meio da transmissão é importante, mas a aprendizagem por questionamento e experimentação é mais relevante para uma compreensão mais ampla e profunda” \cite[p.~37]{moran2015mudando}. Repensar as práticas educativas é fundamental para que novas formas de aprender e ensinar sejam discutidas e trabalhadas em sala de aula.

Nesse limiar, temos a contação de histórias ancorada nas mídias digitais, que, de acordo com \textcite[p.~25]{jenkins2015cultura}, “talvez nada seja mais humano do que dividir histórias, seja ao pé do fogo ou em ‘nuvem’, por assim dizer”. As narrativas são uma forma de organizar e perpetuar as experiências humanas \cite{bruner1991construcao}. Por muitos e muitos anos, a contação/narração de histórias acontecia de forma oral e escrita e servia para que saberes e culturas produzidos pela humanidade pudessem ser transmitidos e recontados pelas gerações seguintes.

Sobre isso, \textcite[p.~222]{valenca2019storytelling} nos falam que a contação de histórias sempre foi uma forma de transmissão de conceitos, valores, conhecimentos e representações sobre o mundo e as experiências vividas, sendo, nesses termos, ancestral, “já que não conhecemos épocas ou sociedades em que tal atividade não estivesse presente” nas práticas sociais.

Nesse limiar, ao mesmo tempo em que marcava a espécie humana por meio desse recurso que produzia, a dotava de uma capacidade de se produzir por meio desses processos narrativos orais ou escritos. Desde os tempos mais longínquos, mediante sua capacidade de comunicação, manifesta esse carecimento de contar histórias, narrativas, sustentadas por símbolos e códigos que evoluem com os tempos e acarretam também a evolução da espécie humana.

Diante desse pressuposto ontológico, \textcite[p.~9]{busatto2005narrando} nos coloca que a contação de histórias é um instrumento capaz de “servir de ponte para ligar as diferentes dimensões e conspirar para a recuperação dos significados que tornam as pessoas mais humanas, íntegras, solidárias, tolerantes, dotadas de compaixão e capazes de estar com”. Um ato individual e coletivo que materializa e historiciza o humano.

Por mais que a história narrada seja sobre um acontecimento, ela pode assumir diferentes perspectivas a partir de quem a conta. Nesse sentido, o ato de narrar como uma construção do pensamento humano ganha maior importância, uma vez que, a partir da construção do pensamento narrativo, o ser humano passa a “narrar-se”; a apropriação crítica dos acontecimentos históricos permite que eles sejam relembrados e recontados passando a dar sentido à existência humana \cite{rodrigues2020narrativas,rodrigues2023para}, colaborando para seu processo de historicidade.

Muito embora as narrativas tradicionais sejam práticas antigas e nunca obsoletas, pelos motivos já expostos, a contemporaneidade impulsionou uma transcendência real para a contação, que precisa agora dar conta, aderir e adquirir a outros formatos de captura das complexas vivências dos sujeitos, ou seja, assistimos às contações de histórias migrarem para plataformas midiáticas, como denominamos a narrativa digital, ou mais especificamente o termo adotado no nosso trabalho, a \textit{Digital Storytelling} que, sobretudo, insere a prática de contação de histórias no arcabouço das mídias digitais que podem ser utilizadas no processo educativo e potencializá-lo.

Com o passar dos anos e a evolução dos meios de comunicação e das tecnologias, as histórias passaram a ser narradas com o auxílio de plataformas virtuais e passaram a incorporar outros elementos, além da voz do narrador e da escrita; contando agora com elementos de imagens, gráficos, efeitos sonoros e musicais, animações visuais, dentre outros que são possibilitados pelo uso das TDIC e amplamente usados na hipermídia. Nas plataformas midiáticas, a narração de histórias ganha outros horizontes. Contudo, não se trata apenas de uma migração para as plataformas no sentido restrito, pura e simplesmente um deslocamento no espaço de publicização da narrativa.

É importante ter em conta que as narrativas contemporâneas demarcam outro espaço, que recria simbólica e esteticamente as produções das histórias, trazendo implicações para o espaço onde se elabora e veicula e para o próprio produto/narrador.

O narrador, em suas elaborações, marca e é marcado pela contemporaneidade, que por sua vez está erigida com e pela cibercultura, e segundo nos aponta \textcite[p.~93]{busatto2005narrando}, chegou e “se impôs, imponente ao homem, que no final do século XX se viu diante de uma nova dimensão do tempo, e foi instigado a mudar a lente com que olhava para o mundo e para as coisas”. Também, o formato digital ancorado nas plataformas digitais marca o narrador e desencadeia outras habilidades e domínios necessários à atuação no tempo vigente. Trata-se, portanto, de um movimento de interação e consequências que produzem efeitos no que e como produz a si e a essa \textit{Digital Storytelling}.

Ainda segundo \textcite[p.~93]{busatto2005narrando}, ao olharmos para as produções no digital, percebemos que se apresentam

\begin{quote}
 como uma linguagem que incorpora uma visão dinâmica e interativa do sentido. Ao apresentar novas simbolizações para o ser humano, o meio digital, o tempo virtual, propõe que a gente pense e repense os significados de se viver e conviver, produzir e consumir, ser e estar no mundo contemporâneo.   
\end{quote}

Conforme \textcite{canini2018narrativas} nos lembra, as formas narrativas desenvolvem-se à medida da evolução dos meios digitais de comunicação e dos usos que geramos das inovações tecnológicas,

\begin{quote}
    por isso mesmo, os conceitos de gêneros textuais tradicionalmente utilizados não são capazes de contemplar a diversidade de organização discursiva em meio às TDIC. Dessa forma, a partir dos meios digitais, as produções textuais tornam-se mais complexas e mescladas \cite[p.~57]{canini2018narrativas}.
\end{quote}

Dessa maneira, as narrativas já não são simplesmente contadas ou escritas, são assistidas e transmitidas também, como coloca \textcite[p.~58-59]{canini2018narrativas}, mediante plataformas e mídias com interfaces visuais intuitivas, que se diferem e se afastam dos modelos de tecnologias fundamentalmente textuais de outrora. Portanto, há a disseminação dos recursos tecnológicos possibilitando novas formas de produção de narrativas, além do texto escrito ou falado. Além disso, outras formas de produção de texto, advindas das práticas sociais com o uso de múltiplas linguagens midiáticas, propiciam a organização de nossas experiências por meio de histórias que articulam os acontecimentos com os quais lidamos, representados por meio de texto, imagem ou som.

Sobre essas características das plataformas, que incorporam diversas mídias de forma intuitiva, possibilitam semioses e construções diversas das expressões e comunicações dos sujeitos, podemos dizer que aqui reside o seu sucesso e migração consubstanciada. As possibilidades colocadas estimulam e materializam a criatividade, a tradução da imaginação dos produtores, as aprendizagens em mídias e em outros formatos textuais, ou seja, expandem-se, por meio das produções, as construções narrativas, que podem ganhar originalidade e ineditismo, incrementando os ambientes e os resultados das narrativas, uma vez que produzem e ampliam linguagens, bem como artefatos culturais e simbólicos que representam cada tempo vivido e tomado pelos sujeitos \cite{bruner1991construcao}.

Estamos delineando aqui a metodologia \textit{Digital Storytelling}. Doravante, precisamos nos aproximar do conceito desta, fruto do movimento evolutivo e ambiental em que se produz nos usos de linguagens e mídias digitais emergentes. É um modo de construir narrativas que surge nos anos 1990, que acompanha toda essa evolução das narrativas tradicionais. Apresenta em sua estrutura potencialidades para a construção de um novo leitor, um leitor que é capaz de ler para além das letras em textos, ele ganha repertório para ler o mundo por meio de uma variedade de linguagens. Segundo \textcite{robin2006usos} há várias definições de \textit{Digital Storytelling}, mas todas comungam com a ideia de combinar a arte de contar histórias com uma variedade de multimídia digital, como imagens, áudio e vídeo, “quase todas as histórias digitais reúnem alguma mistura de gráficos digitais, texto, narração em áudio gravada, vídeo e música para apresentar informações sobre um tópico específico” \cite[p.~1, tradução nossa]{robin2006usos}.

Ao utilizar a \textit{Digital Storytelling} seguindo a definição trazida por \textcite{robin2006usos}, percebe-se que as narrativas digitais se utilizam de efeitos sonoros e imagéticos que auxiliam na construção de um ambiente propício para tornar o ato de narrar histórias mais dinâmico e atraente para os espectadores, leitores, criadores e compartilhadores.

Na atualidade, também se percebe que a forma como nos relacionamos e interagimos com os meios tem mudado e vive em constante processo de evolução. Como uma resposta evolutiva e forma de adaptabilidade às formas de comunicação e as demandas que emergem com a sociedade contemporânea, temos nos tornado a sociedade do compartilhamento, comportamento que é intensificado pelos usos das redes sociais; narramos diariamente nossas vivências e experiências, construímos nossas histórias de forma virtual e deixamos nossos registros de forma autobiográfica para que qualquer pessoa possa ser telespectadora de nossa narrativa pessoal, ao mesmo tempo que somos espectadores de diversas outras narrativas.

No contexto escolar, não é diferente: utilizar-se dos elementos que as tecnologias trazem para construir narrativas curtas é poderoso para a apresentação, compreensão e interação com conteúdo didático. Embora ainda não muito explorado em nosso contexto educacional, com base em estudos realizados em países como Estados Unidos, Turquia, Canadá, Noruega, entre outros ao redor do mundo, as narrativas digitais apresentam elementos que podem contribuir significativamente no processo de ensino e aprendizagem.

Conforme \textcite[p.~1]{signes2010practical} nos assegura, a \textit{Digital Storytelling} é uma maneira envolvente de os alunos produzirem histórias em formas tradicionais e inovadoras, uma vez que “aprendem a combinar algumas ferramentas básicas de multimídia, como gráficos, animação, com habilidades como pesquisa, escrita, apresentação, tecnologia, entrevista, interpessoal, resolução de problemas e habilidades de avaliação”.

Apesar de parecer ser novidade, por ser um tipo de produção audiovisual que está no auge, a metodologia \textit{Digital Storytelling}, assim como o uso das tecnologias na educação, teve seus primeiros ensaios em 1990, com o fundador da organização Center for \textit{Digital Storytelling} (CDS), Joe Lambert, que oferecia treinamento e assistência para pessoas que demonstravam interesse em compartilhar suas histórias fazendo uso de ferramentas digitais.

De acordo com \textcite{signes2010practical}, a \textit{Digital Storytelling} tem a mesma estrutura de uma narrativa tradicional, porém, enriqueceu os formatos, modos de apresentação e compartilhamento fácil e rápido com o auxílio da tecnologia multimídia, que por sua vez favorece alargamentos quanto aos seus usos e dimensões pedagógicas. Nesse sentido, encontramos aqui sua maior diferença com relação à narrativa tradicional, no meio e suas implicações, ou seja, sua ambientação, onde ela é construída e as implicações disso na relação sujeito-meio, narrador-ouvinte; bem como na estética que o digital oferece, nas possibilidades de outras contações, recontações, aprofundamentos, com estes ou aqueles recursos multimídia, que darão outros resultados e significações ao enredo central.

\textcite[p.~3]{signes2010practical} afirma também que essa abertura para diversas combinações multimídias traz resultados originais e nos impõe “considerar é que a contação de histórias digital abriu novas formas de trabalhar tanto com o discurso quanto com as novas tecnologias que, sem dúvida, podem gerar todo tipo de atividades que podem ser úteis, atraentes e motivadoras para os alunos”.

Percebemos que, para a construção de \textit{Digital Storytelling}, os dispositivos da tecnologia funcionam apenas como um “braço” na construção das narrativas. Exige-se dos sujeitos, professores e alunos um envolvimento muito maior na apropriação, construção, interpretação e significação das histórias, fazendo assim que se crie um ambiente de aprendizagem colaborativa, por isso apontado pelos autores como metodologia, e a nosso ver uma metodologia ativa, uma vez que a contação de histórias combinada com método “de aprendizado ativo permite estimular processos de motivação e construir significados” \cite[p.~223]{valenca2019storytelling}.

\textcite{robin2006usos} categoriza três grandes grupos de narrativas digitais: 1) narrativas pessoais, que se ocupam em contar histórias que são importantes para uma pessoa, geralmente carregam relatos de vida e histórias de superação e motivação; 2) narrativas históricas, que examinam acontecimentos históricos a fim de entender o passado; e 3) narrativas que instruem, que são aquelas utilizadas para informar ou instruir o espectador sobre os mais variados conceitos e práticas. Essas categorias não dizem respeito a uma demarcação/separação de modos de narrativas por blocos individualizados que não possam interagir, servem apenas à divisão didática de como geralmente essas narrativas têm sido elaboradas.

A construção da \textit{Digital Storytelling} não se resume à transcrição de uma narrativa oral para o ambiente virtual. Ao trabalhar uma narrativa utilizando os elementos da hipermídia ou multimídia, é necessário que haja uma preocupação quanto à adaptação da linguagem e recursos a serem utilizados. \textcite[p.~697]{rodrigues2020narrativas} aponta que “as tecnologias agregam ao narrar componentes que passam a ser constitutivos do próprio processo de construção da narrativa, de como ela pode ser idealizada, materializada e disseminada por seus autores”.

Nessa perspectiva, \textcite[p.~328]{moura2023narrativa} nos colocam que esse tipo emergente de narrativa ancorada em artefatos digitais “acaba por ampliar e elaborar uma linguagem própria referente e formada a partir dos atributos técnicos e estéticos que cada aparato oferece, sem perder de vista os atributos e implicações que contar já oferece”, ampliando-se as possibilidades de enredos e formatos que as histórias podem adquirir.

\textcite{hack2013digital} discorrem que as histórias estão em toda a parte e as usamos para motivar os outros, na transmissão de informações e partilhas de experiências, socialização e conexão com novas pessoas. Narrando, “criamos oportunidades para expressar pontos de vista, revelar emoções e aspectos presentes da nossa vida pessoal e profissional” \cite[p.~13]{hack2013digital}. Para os autores, envolvemo-nos nessa atividade de contação, vale destacar, unicamente humana de forma criativa e, com isso, ela estimula a imaginação e melhora a memória e a capacidade de visualização. Assim, a habilidade de comunicar experiências permite transcender estruturas pessoais e assumir perspectivas mais amplas, tornando-se uma importante ferramenta de aprendizagem.

As potencialidades formativas do uso pedagógico da \textit{Digital Storytelling} na educação podem ser percebidas à medida que “as mídias digitais podem colaborar para com o letramento e formação do leitor, uma vez que o indivíduo letrado é apto a confrontar textos das mais distintas linguagens, tornando-se um leitor ávido de palavras, gestos e ações do mundo que o cerca” \cite[p. 1160]{prado2017narrativas}. Esse leitor traz a sagacidade de compreender contextos e produzir em diferentes estilos e gêneros textuais.

Nesse limiar, \textcite{robin2006usos} já apontava que, quando os educandos são capazes de produzir suas próprias \textit{Digitais Storytelling}, eles desenvolvem:
\begin{enumerate}
    \item habilidades de pesquisa: capacidade de encontrar, analisar e documentar informações;
    \item habilidade de escrita: capacidade de desenvolver um ponto de vista e formular um roteiro;
    \item habilidade de organização: capacidade de organizar os materiais a serem utilizados, organizar o tempo para cumprir a tarefa e entregar o resultado final;
    \item habilidades tecnológicas: capacidade de aprender a manejar diferentes ferramentas, como softwares, câmeras digitais, celulares etc.;
    \item habilidades de apresentação: capacidade de decidir a melhor forma de apresentar a narrativa para o público; habilidade de entrevista, em que encontrarão as fontes para entrevistar e determinarão as melhores perguntas a serem feitas;
    \item habilidades interpessoais: dizem respeito ao trabalho em grupo e o trabalho colaborativo;
    \item habilidades de resolução de problemas: capacidade de aprender a solucionar conflitos em todas as etapas do projeto;
    \item habilidade de avaliação: análise crítica e avaliativa do seu projeto e dos seus colegas.
\end{enumerate}

A produção de narrativa digital, portanto, oferece diferentes habilidades de letramento digital, uma vez que cada produção é única e diferente do que outros gêneros não digitais oferecem \cite{signes2010practical}. Nesse ínterim, os alunos adquirem capacidades de usar, compreender e se comunicar por diversas linguagens visuais, escritas, sonoras, dentre outras, bem como a hibridização dessas, de maneira autônoma, criativa e também colaborativa, já que os trabalhos podem ser pensados e construídos em grupo.

Para \textcite[p.~334]{moura2023narrativa}, o narrar, a partir dessa metodologia, ganha abrangência e torna mais fluídas as inventividades no contexto formativo, podendo “gerar um movimento de aprendizagens emergentes na forma de representar o mundo”, por meio do caminho trilhado por cada produtor, enredado por linguagens multimídias escolhidas para sua produção, deslocando o meio da condição de mero suporte, rompendo seus limites, instaurando outros modos convergentes de contar.

Além desses aspectos, \textcite{barrett2019storytelling} nos ajuda a pensar nesse potencial da produção de narrativa digital também pelo aspecto de fomento à criação de enredos, reflexão e, por consequência, ajuda a melhorar os níveis e eficácia da aprendizagem por ser motivador. Para o autor, “facilita a convergência de quatro estratégias de aprendizado centradas no aluno: envolvimento do aluno, reflexão para aprendizado profundo, aprendizado baseado em projetos e integração efetiva da tecnologia na instrução” \cite[p.~37]{barrett2019storytelling}.

A formação desse perfil de educando corrobora as teorias de metodologias ativas e a formação de sujeitos que são plenamente capazes de compreender os mais diversos contextos e usos das ferramentas que estão à sua disposição para a elaboração de potentes narrativas digitais. Sobre isso, \textcite[p.~223]{valenca2019storytelling} nos falam que essa metodologia se envereda pelo aprendizado ativo por permitir “estimular processos de motivação e construir significados por meio de narrativas em campos do saber em que seu uso tem sido restrito”.

Além disso, a produção de ND, na argumentação de \textcite[p.~51]{rodrigues2023para}, implica:
\begin{quote}
    i) dedicação e tempo de reflexão para estruturar a história, em contraposição à superficialidade e ao ritmo frenético dos \textit{posts} das redes sociais; ii) exercício crítico sobre o próprio processo de aprendizagem, em lugar de produção de respostas prontas que podem ser geradas, inclusive, por inteligências artificiais; iii) experimentação da autoria, desenvolvimento da criatividade e uso de diferentes linguagens para expressão do pensamento, e não a edição simplista de \textit{templates} pré-fabricados; e finalmente, iv) possibilidade e descoberta de recursos abertos para  produção de recursos digitais que podem ser usados também em outros contextos de ensino e aprendizagem.
\end{quote}
 
Diante do exposto, cabe destacar a \textit{Digital Storytelling} como uma metodologia ativa, que oferece múltiplas potencialidades formativas, podendo ser inserida desde a Educação Infantil ao Ensino Superior, visto que a lógica subjacente à metodologia ativa é que o aluno se torne sujeito ativo na produção do conhecimento. Nesses termos, colocam-nos \textcite[p.~223]{valenca2019storytelling}, “considera-se que o estudante assume papel tanto como receptor quanto criador de conhecimento, em uma lógica semelhante à proposta pelo ‘Paradigma do Aprendizado’”. Essa é a prerrogativa da metodologia, a constante e permanente preocupação em envolver os alunos no processo pedagógico, para que estes possam compreender as etapas e fazer as melhores escolhas para aprender.

\section{Aspectos metodológicos}\label{sec-conduta}
Concentramo-nos no tipo pesquisa-ação, que, segundo \textcite[p.~127]{severino2016metodologia}, é aquela que visa, além da compreensão do fenômeno, a intervenção do pesquisador em determinado cenário com vistas a promover mudanças, ou seja, “o conhecimento visado articula-se a uma finalidade intencional de alteração da situação pesquisada”, realizando ao mesmo tempo “um diagnóstico e a análise de uma determinada situação”, que, para o autor, serve aos envolvidos mudanças que aprimorem as práticas analisadas.

O lócus de nossa pesquisa deu-se em uma turma de sétimo período de Pedagogia, durante o segundo semestre de 2022, com a participação efetiva de 12 alunos matriculados na disciplina de Educação e Tecnologias, quando desenvolvemos uma oficina de construção de \textit{storytelling}. Nesse cenário, trata-se de um estudo de caso que, no dizer de \textcite{severino2016metodologia}, se concentra numa investigação de um caso particular, porém significativo, de modo que generalizações possam ser convenientes.   

Vale destacar que os participantes assinaram um Termo de Consentimento Livre e Esclarecido (TCLE), no qual esclarecemos que seus dados e informações pessoais seriam mantidos em sigilo; que poderiam, a qualquer momento durante a pesquisa e análise, retirar seu consentimento sem nenhum prejuízo; e ainda, que a participação na pesquisa não incorreria em riscos previsíveis, bem como não haveria pagamento ou benefícios financeiros.

Para identificar os participantes de nossa pesquisa, os enumeramos de 1 a 12. Sobre estes, estavam na faixa-etária dos 25 aos 30 anos, a maioria ainda não atuava na docência, e tinham nível de apropriação em recursos tecnológicos voltados à educação de médio a baixo, embora tenham desenvolvido bem a atividade proposta, sobretudo devido à fácil funcionalidade da mídia escolhida para desenvolver a proposta, neste caso, aplicativos de vídeo que ancoraram a história contada sobre nuances da cidade de Imperatriz/MA (Dados das anotações de campo do pesquisador, set. 2022).

A escolha por esse ambiente de ensino superior se deu, além dos aspectos de aproximação com o objeto de estudo que destacamos na introdução, também por vislumbrar que, no campo acadêmico, lugar onde há predomínio dos gêneros escritos, é preciso incentivar a comunicação, expressões e outras construções a partir de hipermídias e multimídias, conforme nos alerta \textcite{cruz2016letramentos}.

Os instrumentos que subsidiaram a geração de nossos dados foram, então, as anotações de campo do pesquisador, bem como uma entrevista de grupo focal que realizamos no último encontro da oficina, buscando captar as mudanças ocorridas em termos de aprendizagens em mídias e contação de histórias. A respeito da entrevista de grupo focal, \textcite{gil2014metodos} nos coloca ser um dos procedimentos mais adequados para respaldar pesquisas qualitativas, sendo introduzida com uma questão genérica que vai sendo discutida e detalhada até que o moderador entenda que a questão foi respondida.


\section{Aprendizagens repercutidas}\label{sec-fmt-manuscrito}
A primeira questão lançada ao grupo referiu-se a “O que mais chamou a sua atenção ao produzir o trabalho?”. A participante 1 respondeu:

\begin{quote}
    Gostei da experiência de ir a campo pesquisar, conversar com as pessoas, às vezes a gente busca na internet e acaba não encontrando essas histórias das pessoas (Participante 1, entrevista do grupo focal).
\end{quote}

Aqui, destaca-se o processo de busca de informações como significativo, mesmo que não se encontre material facilmente disponível. Pesquisar e conversar com pessoas nesse resgate foi importante para esse fim. Como sabemos, as pessoas são fontes de informação e histórias que subsidiam a representação do mundo em nossas histórias, conforme nos indica \textcite{bruner1991construcao}.

As participantes 2 e 3 destacaram, respectivamente, as dificuldades em encontrar material, colocando, respectivamente:

\begin{quote}
    Também no nosso foi encontrar material recente, é tudo muito desatualizado quando se fala da história de Imperatriz e do desenvolvimento dela. (Participante 2, entrevista do grupo focal).

    Tem algumas imagens que nós usamos na produção do nosso vídeo que foi registro nosso mesmo e outras foram tiradas da internet (Participante 3, entrevista do grupo focal).
\end{quote}

Esses apontamentos feitos pelas participantes corroboram as dificuldades de acesso a materiais recentes sobre a história da cidade. Realmente, a história da cidade carece de um material sistematizado e divulgado. Da mesma forma, anos atrás, durante o ensino fundamental, essa participante também se deparou com essa ausência. Percebemos que a relevância de atividades como essa, que desenvolvemos por meio da oficina, pode ser uma maneira de produzir e documentar, de forma consistente, histórias sobre a cidade.

Isto nos remete a \textcite{almeida2012integracao}, ao lembrarem que esse tipo de atividade, para além do aspecto do uso do recurso da tecnologia, trabalha a produção de conhecimento e, por consequência, a melhoria dos conteúdos, nesse caso, um repertório importante e que pode favorecer elaboração de versões diferentes que ampliem e aprofundem a história de Imperatriz.

Em seguida, perguntamos se essa metodologia que aplicamos era interessante e se tinha potencial para ser levada para outros ambientes, como, por exemplo, a Educação Básica. Quatro (4) participantes destacaram a relevância da metodologia e da atividade, argumentando:

\begin{quote}
    Até mesmo para Educação Infantil, porque as crianças são muito visuais, e por exemplo... eu tenho o livro Imperatriz: Cidade da Gente, anos iniciais, que é da minha filha, ela ganhou da rede municipal, mas os professores mal trabalharam o livro em sala, eu que trabalhei o livro com ela em casa, e ele é um livro bom, traz uma boa didática, mas é legal ter o livro e a possibilidade de trabalhar outras coisas, como vídeos, além do material físico, por que a gente pode trazer outras coisas, como as meninas que trouxeram depoimentos, poesia e outras formas de contar a história, que eu acho que pode chamar mais a atenção da criança, mais até do que o próprio livro, o livro fica como uma base, mas assistindo, vendo e ouvindo pode ser que elas apreenda mais a atenção da criança (Participante 4, entrevista do grupo focal).

    A questão das tecnologias hoje em dia é que elas conseguem fazer com que a gente (professores) entre em um mundo que pra eles é habitual, eles estão sempre em contato com as tecnologias, sempre em contato com as redes sociais, então trazer para a sala de aula é uma formar de tentar interagir melhor com eles (Participante 5, entrevista do grupo focal).

    É até uma forma de ampliar, de divulgar, por que assim, esses vídeos a gente pode postar no Instagram, pode postar no Youtube, então assim, é uma de outras pessoas conhecerem, ver, e utilizar do trabalho ou se inspirar para fazer alguma coisa, ou atividade, é uma forma também de mostrar a cidade com um outro olhar (Participante 6, entrevista do grupo focal).

    Deixar também registrado que, na minha época de escola, por exemplo, a gente não falava sobre a nossa cidade, sobre as nossas tradições, falava assim, coisas do Brasil, mas Rio de Janeiro, São Paulo, Minas, nada da gente, nada daqui. Quando minha filha recebeu o livro eu fiquei assim, nossa filha, que legal, que bacana, porque assim, eu não tive acesso a essa história na escola. Quando eu estava fazendo Estágio em Educação Infantil, em uma turma de 24 crianças, agitadíssimas, e em um dia que trouxemos um vídeo com o tema da aula foi mágico, eles ficaram mais calmos, eles assistiram todo o desenho e aprenderam, depois a gente começou a conversar sobre e eles já faziam as associações com o que tinha visto no desenho e foi ótimo (Participante 10, entrevista do grupo focal).
    \end{quote}

Pelo exposto, observamos que as respostas convergem quanto à potencialidade da metodologia, não apenas para dinamizar as aulas, trazendo elementos e circunstâncias que despertem mais a atenção, enriquecendo com outras linguagens como a visual e utilizando tecnologias de modo geral nas turmas da educação básica, mas também na possibilidade de construir um acervo sobre a história e cultura da cidade, como apontado anteriormente. Em metodologias como estas exploram o potencial das TDIC voltadas para o desenvolvimento de atividades curriculares das mais diferentes áreas do conhecimento. Conforme nos asseguram \textcite{rodrigues2020narrativas,rodrigues2023para,prado2017narrativas}, o uso de tecnologias no ensino pode promover a aprendizagem, principalmente ao proporcionar outras metodologias de ensino, o que pode resultar em uma formação mais completa dos alunos.

Além disso, questionamos se era possível adquirir aprendizagens e quais destacariam a partir desse exercício baseado na metodologia \textit{Digital Storytelling}? As respostas foram:

\begin{quote}
    Tem um estímulo muito grande na pesquisa, quando a gente não encontra a informação, instiga a gente a ir buscar mais, pesquisar mais (Participante 8, entrevista do grupo focal).

    Gera uma aprendizagem muito significativa, por que você só escutando, absorvendo o que o professor fala, muitas coisas podem passar despercebidas e no momento que você produz conhecimento, vai procurar, vai pesquisar, vai ler você acaba absorvendo aquilo melhor do que só ouvido (Participante 9, entrevista do grupo focal).

    Você se torna uma pessoa perfeccionista, de certo modo, porque você quer fazer algo bom, porque, se você não fizer algo bom você vai ficar frustrado e esse medo de se frustrar acaba fazendo com que você pesquise mais para trazer uma informação coesa (Participante 10, entrevista do grupo focal).

    Além da criatividade, fica cem por cento estimulada, uma coisa é você ter os dados, outra coisa é você conseguir passar esses dados sem ser um conteúdo cansativo e maçante, por exemplo, o vídeo das meninas que foi 15 minutos, passou super rápido e foi muito bom de assistir. Então assim, contar as histórias da nossa maneira é algo bem interessante (Participante 11, entrevista do grupo focal).

    Sim, por exemplo, a Dona Gonçala, já faleceu, então não tínhamos como trazer esse depoimento dela e a família não pode fornecer uma foto dela, então trouxemos uma ilustração de uma senhora de idade que tinha disponível no aplicativo para representar aquela personagem (Participante 1, entrevista do grupo focal).

    Cada plataforma tem recursos diferentes, que podem ser utilizados na ausência de informações, como no caso da foto da Dona Gonçala (Participante 12, entrevista do grupo focal).
\end{quote}

As histórias narradas e toda a troca de experiência trouxeram pontos importantes de entrelaçamento entre as metodologias ativas e as \textit{Digital Storytelling}, como pontuado não só nas produções, mas também nas falas dos participantes, por meio da reflexão que se fez sobre os aspectos formativos que as metodologias trazem como potencialidade ao serem trabalhadas em sala de aula \cite{pasinato2023educacao,moura2023narrativa}.

As habilidades já destacadas nos tópicos anteriores por \textcite{robin2006usos}, no que se refere à construção de \textit{Digital Storytelling}, comungam com as ideias de \textcite[p.~146]{farias2015aprendizagem} ao estabelecerem os elementos que compõem uma metodologia ativa. Destacamos alguns desses elementos presentes nas produções e nas falas dos participantes.

Os participantes apresentaram o desenvolvimento de habilidades interpessoais, ao trabalharem em grupo de forma colaborativa. Como destacado em uma das falas, o trabalho em grupo possibilita a troca de conhecimento e habilidades de forma orgânica, permitindo que as dificuldades sejam solucionadas de forma mais rápida e eficaz com a ajuda do outro.

Destacaram bastante o aparecimento da habilidade investigativa nas construções de \textit{Digital Storytelling}. Para construção de uma história, é imprescindível que haja um movimento de investigação. Os participantes da pesquisa trouxeram essa investigação por meio de pesquisas bibliográficas e digitais, coleta de depoimentos, pesquisa em museus digitais, entre outros, para o enriquecimento do conteúdo trabalhado. A habilidade investigativa não se limitou a uma pesquisa simples com as informações que já tinham em mãos; eles foram um pouco além e buscaram trazer novos elementos para as histórias, novas curiosidades e fatos interessantes, que são resultados da habilidade de investigação potencializada pela proposta metodológica.

A habilidade de organização é outro ponto relevante dos trabalhos. Os participantes conseguiram entregar as produções no prazo estabelecido, e também conseguiram trabalhar bem os recursos que tinham à disposição, sem grandes excessos no uso de elementos de forma desnecessária. Dentro das duplas, também foi possível perceber que todos se empenharam em construir o vídeo que seria apresentado, sem que o trabalho ficasse centralizado em uma única pessoa para realizar a execução.

As duplas apresentaram grande habilidade de apresentação, conseguindo organizar de forma lógica a ordem dos fatos apresentados nas narrativas e também escolhendo muito bem as informações relevantes para a temática trabalhada. As duplas que trouxeram depoimentos de terceiros souberam trabalhar as perguntas e histórias de forma que não se tornassem cansativas para o telespectador, demonstrando assim que a habilidade de apresentação foi bem trabalhada. Essa habilidade nos direciona para o que argumentam \textcite{rodrigues2023para} em relação à experimentação, autoria e criatividade para expressão do pensamento.

As habilidades tecnológicas foram compartilhadas não só entre as duplas, mas também entre os grupos, com o compartilhamento de aplicativos de edição e dicas de uso. Além disso, os participantes produziram algumas imagens e vídeos. Transformar todas as informações e ideias na produção final também se tornou um desafio estimulante para os participantes, uma vez que eles queriam ver e compartilhar o resultado final de toda a investigação, produção e criatividade que tiveram na construção das narrativas. Isso nos remete às descobertas de que falam \textcite{rodrigues2023para}, que puderam fazer com relação aos recursos e suas possibilidades de utilização em diversos contextos do ensino. Esse processo, conforme asseguraram \textcite[p.~339]{moura2023narrativa}, “engendra conhecimentos para criar os enredos, competências e habilidades técnicas de uso das mídias digitais e ainda capacidade de comunicar-se e expressar-se”, produzindo conhecimentos de forma original para os sujeitos de nossa pesquisa.

Por último, destacamos a habilidade de avaliação. Os participantes mostraram-se capazes de avaliar de forma crítica suas produções e as dos colegas, trazendo aquilo que mais sentiram dificuldades e também onde mais se desenvolveram, destacando a importância que a proposta metodológica teve para a sua formação e como pode ser usada e adaptada para o uso na educação básica ou em outros contextos educacionais.


\section{Considerações finais}\label{sec-formato}
A análise do que foi colocado pelos participantes nos aponta para uma contribuição no fomento à formação de um novo perfil de estudante, que seja mais autônomo e protagonista do seu próprio processo de ensino. Por meio das produções, identificamos que os estudantes desenvolveram diversas habilidades, dentre as quais destacamos as habilidades interpessoais, habilidades investigativas, habilidades de organização, habilidades de apresentação, habilidades tecnológicas e habilidades de avaliação. Todas essas habilidades são potencializadas pala construção da \textit{Digital Storytelling}, uma metodologia que mobiliza ferramentas digitais, recursos humanos e diversas habilidades para construção de conhecimento.

A \textit{Digital Storytelling} é uma nova metodologia que propõe o protagonismo do aluno em sala de aula. Um aluno que assume o papel de inventor, investigador, produtor e que faz uso das mídias e tecnologias para essa produção. Narrar histórias ganha outros espaços com o digital, passando a ser compartilhada de maneira mais dinâmica, rápida e interativa.

Como pesquisadora, entendemos que essa construção se mostrou rica por unir as habilidades advindas da contação de histórias em sua forma mais tradicional e do uso das tecnologias e elementos midiáticos, proporcionando que os produtores/leitores/telespectadores desenvolvam habilidades potencializadas pelo exercício dessas produções, sendo levados a refletir sobre aquilo que foi produzido, os meios investigativos, os usos da mídias e tecnologias em sala de aula e o protagonismo, autoria e personalização do estudante trazido por essa proposta metodológica. Nesse sentido, esta pesquisa contribuiu com o uso de tecnologias digitais, cada vez mais presentes no cotidiano dos nossos estudantes e, nesse sentido, ao fomentar a produção de narrativa nos processos formativos, traçamos um caminho pertinente para diversas e diferentes aprendizagens, conforme nos colocam \textcite{rodrigues2023para}, aprendizagens mais personalizadas, em contraponto às automatizadas elaboradas por inteligência artificial.



\printbibliography\label{sec-bib}
%conceptualization,datacuration,formalanalysis,funding,investigation,methodology,projadm,resources,software,supervision,validation,visualization,writing,review
\begin{contributors}[sec-contributors]
\authorcontribution{Marina Rodrigues da Silva Lopes}[conceptualization,formalanalysis,methodology,writing,review]
\authorcontribution{Késsia Mileny de Paulo Moura}[conceptualization,formalanalysis,methodology,writing,review]
\end{contributors}
\end{document}

