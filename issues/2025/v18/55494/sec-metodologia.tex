\section{Metodologia}\label{sec-metodologia}

Esta pesquisa, situada no campo da Linguística Textual, sob uma
abordagem quanti-qualitativa \cite{souza2017}, está embasada nos
pressupostos da abordagem netnográfica, que garante ao pesquisador a
imersão a comunidades virtuais, e, por consequência, seu monitoramento,
a fim de averiguar como a sociedade contemporânea é afetada pela
cibercultura \cite{rocha2005}.

Dado esse pressuposto, optar pelo emprego da netnografia ocorre em razão
das videoaulas também estarem inseridas na cibercultura, ``necessitando
da imersão do pesquisador em espaço on-line para observar práticas de
cultura através do uso da interface de um computador'' \cite[p.~51]{laurentino2023}. No caso da nossa pesquisa, a observação foi
desenvolvida a partir da plataforma YouTube, que, embora a princípio não
seja reservada exclusivamente para disponibilizar vídeos com teor
educativo, desempenha um papel de divulgadora de materiais audiovisuais
de ensino em grande escala.

Para que os objetivos desta pesquisa fossem atendidos, o processo de
coleta de dados se deu em 03 momentos principais. Em um primeiro
momento, acessamos a plataforma YouTube e, para a obtenção dos
resultados de busca, utilizamos o seguinte descritor: ``como escrever um
artigo científico''\footnote{No tocante ao processo de coleta de dados,
  observamos que não há diferenciação entre os vídeos da plataforma
  sobre os termos ``artigo acadêmico'' e ``artigo científico''. Os
  resultados foram os mesmos. Em razão disso, neste trabalho,
  continuamos a utilizar o termo ``artigo acadêmico'', conforme
  \cite{motta-roth2010}.}. Para além da utilização desse
descritor, consideramos as seguintes delimitações como critérios de
busca: (a) material selecionado se caracteriza como uma videoaula; (b)
produções feitas e voltadas para circulação em língua portuguesa; (c)
videoaulas publicadas em contas da plataforma YouTube voltadas para
práticas de letramento acadêmico ou ensino de produção textual; (d)
conteúdo da videoaula alinhado com indicadores de proposta de ensino de
produção de artigo acadêmico no título da videoaula; (e) durabilidade
mínima de cinco minutos e máxima de trinta minutos; (f) apenas uma
videoaula por canal; (g) aulas divididas em partes apenas se forem até
dois vídeos; (h) videoaulas com mais de duzentas visualizações.

Após a inserção do descritor e a aplicação desses critérios, obtivemos
quinze resultados. Para uma melhor organização dos resultados,
elencamo-los com base no critério de número de visualizações (do mais
acessado até o menos acessado) e os apresentamos na \Cref{tab-01}:

{\footnotesize
\begin{longtable}{l >{\raggedright\arraybackslash}p{.3\textwidth} l >{\raggedright\arraybackslash}p{.2\textwidth} l} \\
\caption{Videoaulas sobre artigo acadêmico com maior número de visualizações.}\label{tab-01} \\
\multicolumn{1}{c}{Videoaula} & Título & Visualizações & Canal & Link para acesso\\
\midrule
\endfirsthead
01 & Artigo Científico Pronto em 5 Passos & 959.200 & TCC SEM DRAMA &
\url{https://x.gd/exRap} \\
02 & Artigo Científico Pronto em 5 Passos (Vídeo 2) & 409.706 & TCC SEM
DRAMA & \url{https://x.gd/QADKY} \\
03 & Como elaborar um artigo científico? & 205.532 & Joana Barbosa --
TCC Academy & \url{https://x.gd/FzQNL} \\
04 & Como fazer um Artigo Científico? Estrutura Básica! & 193.856 &
Revista Científica Núcleo do Conhecimento & \url{https://x.gd/y8Ake} \\
05 & Como escrever um artigo científico - passo a passo - AULA 1 &
144.322 & Andre Campos Mesquita & \url{https://x.gd/1mgtM} \\
06 & Como escrever artigo acadêmico -- processo de elaboração de escrita
+ DICAS & 142.153 & Além do Lattes & \url{https://x.gd/ELptJ} \\
07 & Como fazer um TCC passo a passo: Artigo Científico - Guia com 07
passos simples, rápidos e fáceis & 135.280 & André Fontenelle &
\url{https://x.gd/GlLhV} \\
08 & Como escrever um artigo científico & 124.488 & Carla Estorilio &
\url{https://x.gd/V9A2Q} \\
09 & Como fazer um artigo científico -- TCC, Mestrado e Doutorado &
49.945 & Evandro Queiroz \textbar{} Pesquisa sem Mistério &
\url{https://x.gd/klQJC} \\
10 & Como escrever um artigo científico - Passo a passo

(Vídeo 2) & 23.850 & Andre Campos Mesquita & \url{https://x.gd/pmc6E} \\
11 & Escrita científica: Como escrever um artigo? - Pesquisa na prática
133 & 7.717 & Acadêmica & \url{https://x.gd/ypP63} \\
12 & Como Fazer um ARTIGO CIENTÍFICO Passo a Passo - Estrutura Completa
e Forma de Escrita & 4.313 & Professor Hernán & \url{https://x.gd/5Bw6O} \\
13 & COMO ESCREVER UM ARTIGO CIENTÍFICO? \textbar{} Como Aprender? &
1.523 & Como Aprender? & \url{https://x.gd/sFm2W} \\
14 & 9. Como escrever um ARTIGO CIENTÍFICO: GUIA PASSO A PASSO & 921 &
sejaphd & \url{https://x.gd/aOQjw} \\
15 & Como escrever um artigo científico? & 251 & Prof. Dr. Leonardo
Flach & \url{https://x.gd/NxKSD} \\
\bottomrule
\source{elaboração dos autores (2024).}
\end{longtable}
}


Os dados quantitativos expostos na \Cref{tab-01} apontam para a videoaula
publicada no canal TCC SEM DRAMA como aquela que detém não apenas a
primeira, mas as duas videoaulas com maior número de visualizações a
partir da busca empreendida. A soma das duas videoaulas beira a
exorbitante marca de 1 milhão e meio de acessos, ressaltando como a
temática abordada possui uma enorme procura e, por conseguinte, demanda.

A quantidade de visualização também viabiliza a necessidade de análise
quanto ao conteúdo que atinge a todos esses usuários, de modo a
justificar a presente pesquisa. É importante notar que, em cada título
dos vídeos selecionados, há menções sobre o tipo de texto que será
abordado como também um item lexical que indica o propósito de ensino. É
possível observar a variação entre ``como escrever'' (V05, V06, V08,
V10, V11, V13, V14, V15), ``como fazer'' (V04 V07, V09, V12), ``como
elaborar'' (V03), sugerindo o compromisso que os produtores desse
conteúdo assumem com os internautas (talvez, preferencialmente
estudantes), ao disponibilizarem um material dessa natureza na internet.

É imprescindível ressaltar que esses dados quantitativos dizem respeito
a números em constante mudança, dada a sua característica virtual que o
torna contínuo. Todavia, as alterações quanto ao número de visualizações
apenas reforçam seu vigor cibernético. Mediante tal incidência, a
delimitação do \emph{corpus} da pesquisa foi registrada no dia 04 de
outubro de 2024, das 14h12min às 15h20min.

No segundo momento do processo de coleta de dados, assistimos às quinze
videoaulas sobre artigo acadêmico com maior número de visualização. Já
no terceiro e último momento, categorizamos as informações presentes
nessas videoaulas a partir dos objetivos delineados neste artigo. Para
retextualizar as falas dos produtores de conteúdo em material escrito,
adotamos o método da transcrição oral, a partir dos critérios
estabelecidos por \textcite{dionisio2012}.
