\section{Introdução}\label{sec-intro}

O processo de escrita é uma das atividades mais complexas que o ser
humano é capaz de realizar, em razão de vários fatores, a exemplo das
exigências feitas à memória e ao raciocínio durante o momento de
produção \cite{garcez2020}. São inúmeros os conhecimentos e habilidades que
precisam ser articulados e harmonizados para que o texto tome forma.
Tendo em vista esse seu caráter complexo, ainda são recorrentes falsas
crenças sobre a produção textual, que levam pessoas a acreditarem que
podem dominá-la a partir de ``dicas'' desvinculadas de seu contexto de
produção.

A ideia de que fórmulas pré-fabricadas e ``dicas'' isoladas são métodos
cabíveis no ensino de produção textual, apenas negligencia as etapas
necessárias que caracterizam um texto adequado conforme seu contexto de
produção \cite{garcez2020}. O processo de escrita é uma atividade que
carece de idas e vindas, pois deve admitir três grandes momentos que se
intercalam e devem ser compreendidos de modo indissociável: o do
planejamento, o da escrita propriamente dita e o da revisão \cite{antunes2005}. Enquadrar esse processo em uma perspectiva prescrita e linear
pode resultar em estudantes frustrados pela construção de textos
truncados e artificiais.

Essa realidade se agrava quando observamos o cenário acadêmico, em que
as exigências com relação a produções textuais se intensificam. As
expectativas quanto a essas produções não se limitam à utilização
adequada da norma-padrão ou a vocabulários específicos; expandem-se para
aspectos implícitos de produção que precisam ser considerados, como o
que pode ser dito, por quem, de que forma, sob que ponto de vista e
fundamentado em qual autor \cite{oliveira2024}.

Pensando na comunidade discursiva acadêmica, uma das produções textuais
mais demandadas em cursos de graduação da área de humanas é o artigo
acadêmico \cite{motta-roth2010}. Por ser um dos principais
veículos de divulgação científica, a circulação desse gênero na academia
é incontornável, sendo bastante exigido o seu consumo e produção por
parte de professores, estudantes e pesquisadores. Embora seja uma
produção essencialmente ligada ao meio universitário, sua feitura é
quase sempre exigida sem antes ser ensinada. Esse fato pode levar os
alunos a somarem suas dificuldades com o processo de escrita à
dificuldade de produzir um texto do qual desconhecem seu contexto de
produção, estrutura composicional e outras ``dimensões escondidas''
\cite{street2010} que perpassam a construção de um artigo.

Ao exigir do autor capacidade de síntese, descrição, análise e
argumentação, utilizando-se das convenções próprias à determinada área,
o artigo contempla informações geradas em pesquisas a serem submetidas a
apreciações públicas \cite{motta-roth2010}. Sua relevância remonta
à popularização da ciência que, por sua vez, possui a potencialidade de
descrever fenômenos sociais e até mesmo gerar algum impacto benéfico ao
público em geral.

A partir dessas pontuações, torna-se clara a importância de produzir
artigos acadêmicos e a responsabilidade do seu produtor de popularizar
os conhecimentos produzidos na esfera acadêmica. Quando essa tarefa de
produção precisa ser desenvolvida por graduandos e estes normalmente não
recebem orientação para tal, muitas vezes, recorrem a materiais digitais
sobre esse assunto, pois lhes propiciam as mais variadas estratégias de
ensino de acordo com o ritmo e as preferências do estudante \cite{falkembach2005}. Um fator que pode justificar essa recorrência é a facilidade de
acesso a plataformas digitais, que disponibilizam, na maioria das vezes
de forma gratuita, conteúdos digitais educacionais. Antigamente, os
estudantes consultavam manuais impressos que ensinavam a como produzir
textos acadêmicos, hoje, frente aos recursos tecnológicos, os locais de
aprendizagem se ampliam para a cibercultura. Como cibercultura,
compreendem-se vários ambientes da esfera digital que abrigam
informações, até mesmo os que simulam uma sala de aula a partir de
vídeos \cite{martins2018,rocha2005}.

Tendo a cibercultura se tornado uma potencializadora de novas abordagens
educativas, deve-se averiguar sua eficiência enquanto ferramenta de
ensino, a forma como se ensina determinados conteúdos, a exemplo da
produção textual de artigo acadêmico e seus aspectos constitutivos, foco
do presente estudo. Nesse sentido, traçamos dois objetivos para este
trabalho: identificar e analisar objetos de ensino explorados em
videoaulas sobre artigo acadêmico publicadas na plataforma YouTube.

Para tanto, organizamos este artigo em 5 seções, a saber: esta
introdução, contendo uma contextualização inicial sobre o objeto de
investigação da pesquisa, a problemática que o envolve e os objetivos
delineados; o embasamento teórico, no qual apresentamos os pressupostos
que fundamentam o estudo --- as práticas de ensino de Língua Portuguesa
em contexto didático-digital \cite{laurentino2023}, o artigo
acadêmico \cite{motta-roth2010}, as etapas de produção textual
(Antunes, 2003) e os objetos de ensino \cite{linodearaujo2014}; a
metodologia, na qual explicitamos a abordagem e o tipo de pesquisa, bem
como os procedimentos de coleta e análise de dados; os resultados,
contendo a exploração dos objetos de ensino contemplados nas videoaulas
sobre ensino de produção de artigo acadêmico; as considerações finais,
nas quais sinalizamos algumas implicações advindas dos resultados
alcançados.