
\section{Considerações finais} \label{sec-considerações}

Neste artigo, objetivamos identificar e analisar objetos de ensino
explorados em videoaulas sobre artigo acadêmico publicadas na plataforma
Youtube. Para alcançar tal objetivo, fundamentamo-nos nos pressupostos
teóricos sobre objetos de ensino \cite{linodearaujo2014}, artigo
acadêmico \cite{motta-roth2010} e contexto de produção de
videoaulas em contexto didático-digital \cite{laurentino2023}.

Ao analisarmos as videoaulas selecionadas, percebemos a predominância de
objetos de ensino relacionados à estrutura composicional do gênero
artigo acadêmico, bem como, em menor escala, objetos relacionados à sua
função, delimitação da pesquisa e etapas de produção. Embora
reconheçamos a produtividade de focalizar esses objetos de ensino,
sobretudo para quem está se familiarizando com o gênero artigo
acadêmico, não podemos desconsiderar que a escrita acadêmica não é
homogênea, mas heterogênea, situada e plural \cite{lea1998}. Nesse
sentido, conforme sinalizamos na análise, a estrutura do gênero em tela
não poderia ser apresentada como algo estático, independente das
especificidades da cultura disciplinar na qual a pesquisa reportada foi
divulgada. Produzir conhecimentos na área de humanas não coincide com
produzi-los em outras áreas. Isso reflete, inevitavelmente, na
construção dos textos acadêmicos \cite{pereira2019}.

Nessa perspectiva, as videoaulas apresentam algumas lacunas e
inadequações que podem interferir no processo de construção de
conhecimentos acerca da produção do artigo acadêmico, de forma
específica, mas também da produção textual, de forma geral. Se os
estudantes que buscarem essas videoaulas não tiverem autonomia
intelectual, nem se empenharem para procurar outras fontes de
referência, podem ter sua formação acadêmica comprometida.

Consideramos este trabalho relevante para a área de ensino de escrita
acadêmica, visto que dá visibilidade a objetos de ensino explorados em
videoaulas com foco na produção de artigos acadêmicos. Além disso, este
estudo também contribui para o reconhecimento das mídias
didático-digitais como passíveis de estabelecer um ambiente de
ensino-aprendizagem. Entretanto, é importante salientar que os conteúdos
dispostos são trabalhados de forma genérica e ampla, em decorrência,
possivelmente, do objetivo de ter um maior número de acessos. Essas
características são esperadas em ambientes virtuais e assíncronos, que
determinam seu contexto de produção e seu método de ensino.

Ademais, todo material voltado para o ensino deve ser analisado e
estudado por nós professores, pesquisadores e estudantes. É claro que
não de modo a condená-lo, mas para estimular o seu aprimoramento,
sobretudo nos dias atuais em que as mídias alcançam um número maior de
pessoas a cada dia.