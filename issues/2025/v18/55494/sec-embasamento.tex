\section{Embassamento Teórico}\label{sec-embasamento}

O referencial teórico utilizado como base norteadora deste estudo parte
do entendimento de contextos didático-digitais como um espaço advindo de
``publicações on-line de materiais com objetivo de conduzir sujeitos ao
domínio de saberes provenientes de disciplinas curriculares''
\cite[p.~46]{laurentino2023}. A partir dessa concepção, percebe-se
que a flexibilidade quanto ao modo de se ofertar conteúdos pedagógicos
atualmente, frente ao avanço tecnológico, potencializa um amplo alcance
e facilidade de acesso. Entretanto, essas vantagens não asseguram a
qualidade e a credibilidade dos conteúdos veiculados nesses contextos,
levando pesquisadores a examinarem criticamente a consistência desses
materiais \cite{bessa2020,laurentino_videoaulas_2019}.

A depender do material em questão, a forma como se estabelece o contato
entre usuário e conteúdo pode sofrer alterações. Em se tratando da
cibercultura, ambiente que abriga um acúmulo inquantificável de
conhecimento, ``o usuário pode interagir não só com o objeto (a máquina
ou a ferramenta), mas também com a informação, conteúdo'' \cite[p.~8]{rocha2005}. O modo de interação com o conteúdo se altera
conforme os componentes didáticos que complementam o material didático.
No caso da videoaula, a interação atinge o nível de mediação pedagógica
graças à sincronização de outras mídias (áudio, texto ou imagem)
\cite{barrere2014}, favorecendo a possibilidade de concretude dos fins
didáticos pretendidos.

Para considerarmos a videoaula um material didático-digital, \textcite{laurentino2023} apresentam outros componentes necessários para sua
caracterização, a saber: (a) tratar de uma prática de ensino que
apresente um foco didático, relacionando conhecimento curricular, teoria
e metodologia; (b) ser realizada de modo assíncrono; (c) apresentar a
figura de um sujeito empenhado em conduzir as práticas de ensino,
empregando saberes e competências relativos à profissão docente.

Com base nas informações supracitadas, podemos nos questionar: o que
justifica a alta demanda de acessos a materiais inseridos em contextos
didático-digitais? A resposta para essa pergunta pode se relacionar com
as ``práticas institucionais do mistério'', conforme estipulado por
\textcite{lillis1999}. O conceito explorado pela autora diz respeito a
convenções, sobretudo de escrita, demandadas aos alunos que não são
explicitamente orientadas, pois parte-se do pressuposto que eles já as
dominem. Em outras palavras, ``os professores esperam que os alunos
saibam essas convenções que não lhes são explicitadas'' \cite[p.~363]{fiad2011}. Esse cenário é comum na comunidade acadêmica, sobretudo por ser um
ambiente no qual são exigidas produções textuais específicas, típicas da
esfera acadêmica.

A alta demanda de produção de artigos acadêmicos advém da política de
financiamento de bolsas e de projetos de pesquisa, próprios do sistema
universitário brasileiro. Desse modo, nesse contexto, a produtividade
intelectual é medida pela produtividade na publicação \cite{motta-roth2010}. Esse fator gera uma pressão para elaboração de textos de
qualidade na forma de artigos para periódicos acadêmicos.

\textcite{motta-roth2010}, ao definirem o artigo acadêmico, já o
caracterizam como uma produção que objetiva a publicação em periódicos
especializados. Nas palavras das autoras, ``esse gênero serve como uma
via de comunicação entre pesquisadores, profissionais, professores e
alunos de graduação e pós-graduação'' \cite[p.~65]{motta-roth2010}. A exigência de produção de artigos acadêmicos pode se enquadrar em
uma ``prática institucional do mistério'' \cite{lillis1999}, caso suas
dimensões não sejam explicitadas aos produtores desse gênero. Essas
dimensões podem ser entendidas como: o enquadramento do gênero e do
público, contribuição para o avanço dos estudos de uma área de pesquisa,
a voz do autor e seu ponto de vista, as marcas linguísticas e sua
estrutura composicional \cite{street2010}. O aluno, ao se predispor a
produzir um artigo acadêmico, muitas vezes desconhece essas dimensões.

Não obstante defendamos a necessidade de apresentação aos estudantes de
orientações explícitas para as produções textuais, entendemos que a
finalidade pretendida não é de fornecer subsídios ou ``dicas'' de
produção, mas de permitir aos alunos ``uma produção do saber e
estabelecer uma base sólida para a construção contínua e eficaz de
conhecimentos específicos, desenvolvendo, ao mesmo tempo, a habilidade
de aprender e recriar permanentemente'' \cite[p.~48]{fischer2007}.

Ao abordar as habilidades ``de aprender e recriar permanentemente'', é
possível fazer um recorte das ponderações de \textcite{fischer2007} e refleti-lo
a partir de uma ótica textual. Mesmo que o artigo acadêmico traga
elementos constitutivos, em sua estrutura, como revisão da literatura,
metodologia, análise e discussão de resultados \cite{motta-roth2010}, há outras etapas inerentes à produção textual em si, exigidas
para uma prática constante de criação e recriação textual.

Elaborar um texto escrito é uma tarefa que não se completa pela
codificação de escrita e sintetização de ideias \cite{garcez2020}. Também
não se trata de uma atividade que se inicia quando tomamos nas mãos
papel e caneta ou quando nos ajustamos na cadeira com um documento
aberto e posicionamos as mãos no teclado. A escrita compreende três
etapas distintas e integradas de realização: planejamento, operação e
revisão \cite{antunes2003}. A autora compreende essas três etapas da
produção textual da seguinte forma: \emph{planejamento}: ``delimitar o
tema de seu texto e aquilo que lhe dará unidade, eleger os objetivos,
escolher o gênero, delimitar os critérios de ordenação das ideias,
prever as condições de seus leitores e a forma linguística (mais formal
ou menos formal) que seu texto deve assumir'' \cite[p.~55]{antunes2003};
\emph{operação}: registro do que foi planejado, escolha das palavras e
das estruturas das frases; \emph{revisão}: análise do que foi escrito,
para aquele que escreve confirmar o cumprimento dos objetivos, a
concentração temática desejada, a coesão e coerência no desenvolvimento
das ideias e se cumpriu com as normas gramaticais.

Podemos observar que apenas a atividade de escrita propriamente dita não
sustenta o processo de composição de uma produção textual. É necessária
uma etapa anterior e outra posterior, pois cada uma desempenha uma
função indispensável para produzir um texto que atenda seus objetivos de
produção.

Embora distintas, é importante alertar que as três etapas propostas por
\textcite{antunes2003} são intercomplementares. Ou seja, são etapas que implicam
umas nas outras e por isso, durante o ato de produção, não precisam ser
seguidas à risca, como se fossem métodos fixos e inflexíveis. ``Quando
planejamos, já estamos em plena escrita e, quando escrevemos, revisamos
simultaneamente parcelas do texto'' \cite[p. 20]{garcez2020}. O processo é
recursivo e, portanto, abre espaço para idas e vindas conforme
necessário.

Nessa perspectiva, esperamos que os materiais disponibilizados nos
contextos didático-digitais possam se comprometer com um ensino que
forneça essa base sólida de produção, advertindo ao público sobre as
etapas que precisam ser perpassadas ao escrever, para que possam
produzir o artigo acadêmico atentando-se aos seus próprios objetivos e
aos objetivos inerentes ao gênero trabalhado. Para tal, faz-se
necessário que o processo de ensino-aprendizagem sobre a produção desse
gênero em tela se estabeleça a partir de um ou mais objetos de ensino,
que orientem a metodologia empreendida para alcançar a finalidade
pretendida. \textcite[p.~11-12]{linodearaujo2014} apresenta três princípios
pelos quais são validados esses objetos, sendo eles:

\begin{quote}
Princípio da legitimidade -- o objeto precisa fazer referência aos
elementos que emanam da cultura ou são elaborados por especialistas;
Princípio de pertinência -- o objeto precisa estar relacionado às
capacidades dos alunos, às finalidades e objetivos da escola, aos
processos de ensino-aprendizagem; Princípio de solidarização -- objeto
precisa tornar coerentes os conhecimentos em função dos objetivos
visados.
\end{quote}

A seleção de objetos de ensino pode evidenciar, portanto, o modo como se
ensina determinado conteúdo. Para o estabelecimento de um objeto, é
preciso considerar o contexto de ensino e o conhecimento do público a
ser alcançado. A depender do caso, adaptações e reajustes são
necessários para que nenhuma informação passe despercebida pelos alunos.

Com base nessa compreensão de objetos de ensino, identificamos e, em
seguida, analisamos, na seção de resultados, quais objetos são
explorados em videoaulas sobre artigo acadêmico publicadas na plataforma
YouTube. Antes, apresentamos o método empreendido para selecionar as
videoaulas que constituem o \emph{corpus} da análise.
