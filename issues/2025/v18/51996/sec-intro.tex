\section{Introducción}\label{sec-intro}

La transformación digital se ha convertido en un tema crítico para la
sociedad, ya que permea en todos los sectores. Particularmente, en el
ámbito educativo debido a distintos factores como cierre temporal de
escuelas en 2020 y sus implicaciones en los procesos de enseñanza y
aprendizaje ocasionados por la pandemia de COVID-19, aceleró el interés
global de las instituciones de educación superior (IES) por encaminarse
a estas alteraciones \cite{gomez2021,PonceLopez2021EstadoTIC} y para diversos
organismos internacionales (OI) se convirtió en un tema de interés en
sus agendas.

El término de transformación digital no cuenta con una definición
consensuada o asentada en la literatura académica ni con un ``autor
clásico'', dado que el concepto ha sido descrito a partir de los
continuos avances en tecnología y sus usos, así como de las necesidades
y expectativas de la sociedad y al ámbito al que se emplee \cite{fernandez-martínzez2019,schallmo2017}. No
obstante, esta se presenta en la educación superior como un fenómeno
evolutivo y multidimensional de factores internos y externos a las IES
que van más allá de la simple digitalización del conocimiento y de
procedimientos, donde se involucran distintos ámbitos como lo
tecnológico, educativo, económico, cultural, social y político.

En específico, se centra en un conjunto de cambios, tanto en el plano
tecnológico, ---digitalización física (\emph{digitization}) y
automatización de procesos (\emph{digitalization})---, como en lo
organizacional y social, sustentado en una cultura digital, en un
entorno de hiperconectividad y de avances digitales, donde las personas
son el elemento clave; implica un esfuerzo holístico para propiciar
estas transformaciones en los modelos educativos, la gestión y en la
propuesta de valor de la institución educativa \cite{bikse2021,delgado-fernández2020,grajek2020,llorens2021udigital,mergel2019,yavuz2023}.

Por consiguiente, representa un desafío mundial que engloba diversos
aspectos en la educación superior y que requieren de marcos conceptuales
y estratégicos para guiar la transición digital. En este contexto,
diversos OI han emitido informes, guías, estudios, proyectos y
recomendaciones que delinean los principales aspectos de la
transformación digital en la educación. No obstante, se aborda el
fenómeno desde múltiples perspectivas y enfoques de investigación en
distintos contextos. En conjunto, estas iniciativas pueden ofrecer una
caracterización amplia del tema que posibilita una mayor comprensión de
la misma en la educación.

En principales hallazgos de revisiones de la literatura, se señala que
la temática es un campo emergente, en el cual es fundamental abordarla
desde un enfoque más integral \cite{benavides2020-castro} y donde se
consideren perspectivas de organizaciones internacionales \cite{Wang2024}. Sin embargo, son escasas las sistematizaciones que
contemplan en su corpus de análisis lo abordado en la educación
terciaria por los diversos OI en su conjunto.

Asimismo, es importante tener una visión amplia y abonar en este vacío
bibliométrico en la literatura que abarque el período previo y posterior
a la pandemia, debido a que los OI desempeñan un rol crucial en la
configuración de las políticas y prácticas educativas a nivel mundial y
sus contribuciones son sustanciales en la generación de conocimiento, el
establecimiento de marcos de referencia globales, el fomento de la
cooperación entre países, así como la promoción de la calidad y la
equidad en los sistemas educativos \cite{lamprou2023role,maldonado-maldonado2023,martens2023}.

Por consiguiente, en este artículo se explora, a través de un Mapeo
Sistemático de la Literatura (MSL), el cómo ha sido la producción
documental a nivel internacional en lo referente al tema de la
transformación digital, particularmente en la educación superior, desde
la visión de OI. Entre estos organismos se encuentran el Banco
Interamericano para el Desarrollo (BID), la Comisión Económica para
América Latina y el Caribe (CEPAL), la Organización para la Cooperación
y el Desarrollo Económico (OCDE), la Organización de Estados
Iberoamericanos (OEI), la Organización de las Naciones Unidas (ONU) y la
Organización de las Naciones Unidas para la Educación, la Ciencia y la
Cultura (Unesco).

En el marco del estudio, se aporta una descripción general estructurada
sobre el estado actual del tema, según lo documentado por los OI
mencionados. Se clasifica las publicaciones de acuerdo con su
perspectiva temática, tipo de aporte y enfoque de investigación. Además,
se identifica los contextos geográficos mayormente estudiados y se
categoriza las perspectivas analíticas en función de palabras clave y
temas abordados por estos organismos. En conjunto, estos elementos
perfilan líneas de investigación futuras y proporcionan una base para la
implementación de estrategias o el diseño de políticas públicas e
institucionales.
