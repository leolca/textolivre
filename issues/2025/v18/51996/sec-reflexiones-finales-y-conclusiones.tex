% !TeX root = main.tex
\section{Reflexiones finales y conclusiones}\label{sec-reflexiones-finales-y-conclusiones}

En el marco de los resultados presentados, se destaca que abordar la
transformación digital conlleva a comprenderla como un proceso de
integración tecnológica digital en varias áreas de una organización,
tanto internas como externas, y se compone de un marco complejo de
cambios que involucra tecnologías digitales y emergentes,
infraestructura, modelos educativos, gestión educativa, desarrollo de
competencias digitales y política pública, y que converge en cuatro
dimensiones clave: tecnológica, organizacional, educacional y social.
Este enfoque sugiere que las instituciones educativas requieren un
análisis holístico y los hallazgos coinciden con lo señalado por \textcite{aditya2021,bikse2021} y \textcite{benavides2020-castro} para garantizar la transformación digital en las
IES.

Si bien, los OI presentan publicaciones desde todas las perspectivas
mencionadas ---excepto la OEI que solo se enfocó en lo tecnológico,
organizacional y social---, y el BID, la CEPAL, la OCDE y la ONU
presentan la transformación como una visión holística, el tema en la
educación superior ha sido estudiada e implementada de manera segmentada
y con un abordaje desde lo empresarial y comercial. Por ejemplo, la
promoción del modelo ``business to university'', los ecosistemas
\emph{Fintech} (finanza y tecnología) y el financiamiento para el
emprendimiento, así como la generación de capital humano competitivo en
aras de mejorar la empleabilidad y asegurar que los principales
beneficiarios de estos esfuerzos sean los estudiantes, graduados y
empleadores.

No obstante, los abordajes novedosos en los que se trata el tema y
pueden guiar futuras líneas de estudio están relacionadas con el
desarrollo de estrategias resilientes y programas educativos sostenibles
para un futuro digital, así como la promoción de prácticas pedagógicas
centradas en la inclusión de tecnologías para la reducción de brechas
pedagógicas y digitales mediante la transformación digital. Además, se
reconoció que tres cuartas partes de los enfoques identificados en los
OI revisados, están concentrados en procesos de evaluación, y que menos
de la mitad plantean propuestas de solución, iniciativas o proyectos.
Esto conlleva también a pensar en otras líneas de investigación
relacionadas con el desarrollo de programas y estrategias integrales de
la transformación digital en las IES, donde se consideren todas las
áreas y actores clave para su implementación en las universidades.

En suma, esto concuerda con lo puntualizado por \textcite{cerdá-suárez2021}, al señalar que estudiar este fenómeno complejo y
de interés actual para las universidades, tanto en países más avanzados
como en vías de desarrollo y particularmente en la región de América
Latina y el Caribe, es necesario tomar en cuenta una serie de elementos
para su implementación. Estos van desde la formulación de políticas y
agendas nacionales que respalden la transformación en el país hasta el
fomento de habilidades digitales, tanto en profesores como en
estudiantes y atender cuestiones de accesibilidad y equidad.

En conclusión, no se puede desestimar que el tema de la transformación
digital en el ámbito educativo, ha sido significativamente acelerado por
la pandemia de COVID-19 y que ha propiciado la necesidad global por
implementarlo en las instituciones educativas, así como el interés de
diversos OI por analizarlo desde múltiples perspectivas, enfoques y
contextos geográficos en el mundo, con el fin de impulsarla y ofrecer
una caracterización amplia del tema para una mayor comprensión de la
misma en la educación.

Si bien el MSL posibilita la exploración de publicaciones sobre un tema
de interés, sus limitaciones en este estudio están relacionadas con el
nivel general de análisis, por lo que no se ahondó en los marcos
conceptuales. Sin embargo, este enfoque resulta útil para seleccionar
documentos que permitan un análisis más profundo en una siguiente fase,
como se recomienda en la revisión sistemática de la literatura.
Asimismo, debido a que escasas publicaciones contenían palabras clave (7
de 23), fue necesario revisar información adicional en la sección de
resumen, objetivos, índices y conclusiones, por lo que, requirió de más
tiempo en la revisión de cada documento para determinar su potencial
relevancia.

No obstante, se reconoce que el MSL es una herramienta valiosa para
comprender el panorama actual del conocimiento, identificar áreas de
investigación prioritarias y respaldar la toma de decisiones
fundamentadas en el ámbito de la educación digital. El análisis de 23
documentos, posibilitó la clasificación de perspectivas temáticas, tipos
de aportes y enfoques de investigación, así como ubicar los contextos
geográficos analizados y vislumbrar futuras líneas de indagación. Por
consiguiente, este artículo proporciona el procedimiento metodológico
particular para publicaciones no catalogadas como científicas pero que
aportan al campo de conocimiento, y un marco referencial de cómo ha sido
la producción documental a nivel internacional en lo referente a la
transformación digital en el ámbito de la educación superior, desde la
perspectiva de seis OI.

