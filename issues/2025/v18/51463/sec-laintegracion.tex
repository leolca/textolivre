\section{La integración de la tecnología en la educación superior}\label{sec-laintegracióndelatecnologíaenlaeducaciónsuperior}

Teniendo en consideración, como se comentaba con anterioridad, de que el
proceso de digitalización se ha desarrollado en las diferentes etapas
educativas, esta investigación centra la atención en la educación
superior.

En esta etapa es fácilmente observable la integración de la tecnología
en las tres vertientes apuntadas anteriormente: equipamiento, currículum
y metodologías. En relación con el equipamiento, y aunque a nivel
general, está extendida la presencia de ordenadores, pizarras y
proyectores, así como la utilización de entornos virtuales de
aprendizaje, la disponibilidad de recursos depende de cada institución
y, en ocasiones, del contexto geográfico donde se ubican. Así, mientras
que en países como España, Uruguay, Cuba o Ecuador los centros
educativos cuentan de manera casi completa con ordenadores y conexión a
Internet, en otros como en Honduras, Guatemala o Paraguay el
equipamiento es mucho menor \cite{siteal}.

En cuanto a la integración curricular de la tecnología, y tomando en
consideración el contexto español, ha habido una tendencia a promover
las destrezas tecnológicas del alumnado de las diferentes etapas
educativas de un modo trasversal \cite{gabarda2021b}. El origen de este
enfoque hay que buscarlo en las directrices comunitarias que, desde el
año 2006, identificaron la competencia digital como una de las
habilidades clave a lo largo de la vida \cite{comision2006,consejo2018}, condicionando la política educativa de los diferentes
países a nivel organizativo y curricular. De este modo, la actual
legislación en España, la \emph{Ley Orgánica 3/2020, de 29 de diciembre,
por la que se modifica la Ley Orgánica 2/2006, de 3 de mayo, de
Educación}, reconoce la competencia digital como una competencia clave
en el sistema educativo, enfatizando la necesidad de incorporarla
curricular y metodológicamente en las diferentes etapas, incluyendo la
educación superior.

Por último, y precisamente en relación con el punto de vista
metodológico, la integración de la tecnología ha permitido potenciar y
optimizar propuestas que anteriormente se desarrollaban de un modo
analógico y, por otra, crear propuestas donde la tecnología es condición
sine qua non para poder desarrollarlas. En el primer grupo se podrían
encontrar opciones como el trabajo cooperativo, el aprendizaje basado en
problemas o la gamificación que, aunque no precisan en sí de recursos
digitales para implementarse, se han visto enriquecidos por sus
beneficios \cite{cuevas2021,pacheco2022,sanchezrivas2023}. En el segundo estarían recursos como la realidad aumentada, la
realidad virtual, la robótica, los videojuegos o los simuladores,
procesos como el pensamiento computacional o metodologías como el
flipped classroom, donde se constata que la tecnología es el medio para
la implementación metodológica \cite{gonzalez-martinez2024,sanchezsoto2023}.

Independientemente de que sea de un modo u otro, hay bastante consenso
en la literatura científica acerca del impacto de la tecnología en el
aprendizaje del alumnado de educación superior. Así, estudios como el de
\textcite{juanllamas2022} concluyen que la tecnología permite
mejorar la dinámica de las clases y aumentar la participación y la
motivación del alumnado a través de la introducción de aspectos lúdicos.
Por otro lado, también hay evidencias de que la tecnología favorece una
mayor personalización del aprendizaje y una mayor autorregulación por
parte del alumnado \cite{sáez-delgado_parra_jara-coatt_mella-norambuena_lópez-angulo_2023} o que su implementación
puede contribuir de un modo efectivo al desarrollo de competencias
comunicativas \cite{mesarave2023}, personales \cite{guerrero2023},
sociales \cite{garcia2022}, académicas \cite{cabrera2023},
digitales \cite{cupido2022} o profesionales \cite{gabarda2021b}. Por último, no hay que olvidar que la tecnología puede
contribuir a la inclusión, sirviendo de mecanismo para hacer más
accesible la educación superior a la población diversa. Así lo confirman
estudios como los de \textcite{fernandezcerero2024} o
\textcite{fernandezbatanero2023}.