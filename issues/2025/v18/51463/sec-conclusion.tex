\section{Conclusión}\label{sec-conclusión}

El trabajo de revisión sistemático realizado en este estudio ha
permitido corroborar el interés de la comunidad científica por conocer
en qué medida la tecnología contribuye al aprendizaje del alumnado de
educación superior y, concretamente, en el desarrollo de diferentes
competencias que se vinculan a su campo formativo.

Este interés está presente en diferentes contextos geográficos dentro
del panorama iberoamericano, habiendo sido la pandemia uno de los
elementos clave del impulso de políticas tanto por parte de los
diferentes estados \cite{marinsuelves2023} como de las
instituciones a través de estrategias propias \cite{colomo2023}.

Así, se ha podido constatar, de manera mayoritaria, la tecnología cuenta
con posibilidades para optimizar los procesos formativos \cite{cuevas2022}, siendo la educación superior la que ofrece mayores posibilidades
por las características de los usuarios y la diversidad de estudios que
ampara.

Concretamente, este análisis ha podido demostrar el potencial de los
recursos y metodologías mediadas para la adquisición de conceptos
teóricos \cite{cabrera2023}, consolidándose como un promotor de la
mejora del rendimiento académico \cite{davila2023}. Asimismo, la
literatura estudiada pone de relieve que la tecnología contribuye al
desarrollo de destrezas prácticas \cite{gabarda2021b,perez2023}, complementando los aprendizajes de carácter conceptual. Estas
habilidades de carácter académico se acompañan con el favorecimiento de
habilidades sociales \cite{garcia2022,guerrero2023} y
personales \cite{cordero2021,marques2023}, poniendo
de manifiesto que la tecnología puede contribuir al desarrollo integral
del alumnado.

Para finalizar, cabe exponer que, a pesar de que hay un amplio consenso
sobre el potencial que la tecnología aporta a los procesos formativos,
resultaría de gran interés poder cotejar este fenómeno en otras etapas
educativas y en otros contextos geográficos (Unión Europea, América y el
resto de los continentes), así como poder explorar qué cuestiones
condicionan el diseño e implementación de estrategias mediadas por ella.
En este sentido, analizar aspectos como el equipamiento, la política a
nivel macro y a nivel de cada institución o la formación del profesorado
podrían permitir conocer qué obstáculos hay para su integración y poder
diseñar, en base a ellos, propuestas específicas para la promoción de
procesos formativos más ajustados a las necesidades de cada contexto.
