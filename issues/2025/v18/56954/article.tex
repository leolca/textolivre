% !TEX TS-program = XeLaTeX
% use the following command:
% all document files must be coded in UTF-8
\documentclass[portuguese]{textolivre}
% build HTML with: make4ht -e build.lua -c textolivre.cfg -x -u article "fn-in,svg,pic-align"

\journalname{Texto Livre}
\thevolume{18}
%\thenumber{1} % old template
\theyear{2025}
\receiveddate{\DTMdisplaydate{2025}{1}{13}{-1}} % YYYY MM DD
\accepteddate{\DTMdisplaydate{2025}{2}{12}{-1}}
\publisheddate{\DTMdisplaydate{2025}{6}{25}{-1}}
\corrauthor{Jose Isavam Oliveira Silva}
\articledoi{10.1590/1983-3652.2025.56954}
%\articleid{NNNN} % if the article ID is not the last 5 numbers of its DOI, provide it using \articleid{} commmand 
% list of available sesscions in the journal: articles, dossier, reports, essays, reviews, interviews, editorial
\articlesessionname{articles}
\runningauthor{Silva e Pinho} 
%\editorname{Leonardo Araújo} % old template
\sectioneditorname{Daniervelin Pereira}
\layouteditorname{Leonado Araújo}

\title{As tecnologias digitais no ensino de língua portuguesa}
\othertitle{Digital technologies in portuguese language teaching}
% if there is a third language title, add here:
%\othertitle{Artikelvorlage zur Einreichung beim Texto Livre Journal}

\author[1]{Jose Isavam Oliveira Silva~\orcid{0009-0005-2026-7129}\thanks{Email: \href{mailto:isavam.silva@unemat.br}{isavam.silva@unemat.br}}}
\author[2]{Albina Pereira de Pinho~\orcid{0000-0002-5139-9299}\thanks{Email: \href{mailto:albina@unemat.br}{albina@unemat.br}}}
\affil[1]{Universidade do Estado de Mato Grosso, Programa de Pós-Graduação em Letras, Sinop, MT, Brasil.}
\affil[2]{Universidade do Estado de Mato Grosso, Faculdade de Ciências Humanas e Letras de Sinop, Sinop, MT, Brasil.}

\addbibresource{article.bib}
% use biber instead of bibtex
% $ biber article

% used to create dummy text for the template file
\definecolor{dark-gray}{gray}{0.35} % color used to display dummy texts
\usepackage{lipsum}
\SetLipsumParListSurrounders{\colorlet{oldcolor}{.}\color{dark-gray}}{\color{oldcolor}}

% used here only to provide the XeLaTeX and BibTeX logos
\usepackage{hologo}

% if you use multirows in a table, include the multirow package
\usepackage{multirow}

% provides sidewaysfigure environment
\usepackage{rotating}

% CUSTOM EPIGRAPH - BEGIN 
%%% https://tex.stackexchange.com/questions/193178/specific-epigraph-style
\usepackage{epigraph}
\renewcommand\textflush{flushright}
\makeatletter
\newlength\epitextskip
\pretocmd{\@epitext}{\em}{}{}
\apptocmd{\@epitext}{\em}{}{}
\patchcmd{\epigraph}{\@epitext{#1}\\}{\@epitext{#1}\\[\epitextskip]}{}{}
\makeatother
\setlength\epigraphrule{0pt}
\setlength\epitextskip{0.5ex}
\setlength\epigraphwidth{.7\textwidth}
% CUSTOM EPIGRAPH - END

% to use IPA symbols in unicode add
%\usepackage{fontspec}
%\newfontfamily\ipafont{CMU Serif}
%\newcommand{\ipa}[1]{{\ipafont #1}}
% and in the text you may use the \ipa{...} command passing the symbols in unicode

% LANGUAGE - BEGIN
% ARABIC
% for languages that use special fonts, you must provide the typeface that will be used
% \setotherlanguage{arabic}
% \newfontfamily\arabicfont[Script=Arabic]{Amiri}
% \newfontfamily\arabicfontsf[Script=Arabic]{Amiri}
% \newfontfamily\arabicfonttt[Script=Arabic]{Amiri}
%
% in the article, to add arabic text use: \textlang{arabic}{ ... }
%
% RUSSIAN
% for russian text we also need to define fonts with support for Cyrillic script
% \usepackage{fontspec}
% \setotherlanguage{russian}
% \newfontfamily\cyrillicfont{Times New Roman}
% \newfontfamily\cyrillicfontsf{Times New Roman}[Script=Cyrillic]
% \newfontfamily\cyrillicfonttt{Times New Roman}[Script=Cyrillic]
%
% in the text use \begin{russian} ... \end{russian}
% LANGUAGE - END

% EMOJIS - BEGIN
% to use emoticons in your manuscript
% https://stackoverflow.com/questions/190145/how-to-insert-emoticons-in-latex/57076064
% using font Symbola, which has full support
% the font may be downloaded at:
% https://dn-works.com/ufas/
% add to preamble:
% \newfontfamily\Symbola{Symbola}
% in the text use:
% {\Symbola }
% EMOJIS - END

% LABEL REFERENCE TO DESCRIPTIVE LIST - BEGIN
% reference itens in a descriptive list using their labels instead of numbers
% insert the code below in the preambule:
%\makeatletter
%\let\orgdescriptionlabel\descriptionlabel
%\renewcommand*{\descriptionlabel}[1]{%
%  \let\orglabel\label
%  \let\label\@gobble
%  \phantomsection
%  \edef\@currentlabel{#1\unskip}%
%  \let\label\orglabel
%  \orgdescriptionlabel{#1}%
%}
%\makeatother
%
% in your document, use as illustraded here:
%\begin{description}
%  \item[first\label{itm1}] this is only an example;
%  % ...  add more items
%\end{description}
% LABEL REFERENCE TO DESCRIPTIVE LIST - END


% add line numbers for submission
%\usepackage{lineno}
%\linenumbers

\begin{document}
\maketitle

\begin{polyabstract}
\begin{abstract}
Esta pesquisa analisa as percepções de jovens do Ensino Médio sobre o ensino de Língua Portuguesa aliado ao uso das Tecnologias Digitais em contexto do Ensino Remoto Emergencial. O estudo filia-se aos fundamentos da investigação qualitativa, sob o enfoque da pesquisa com narrativas, conforme \textcite{todorov1983}. AA geração dos dados aconteceu por meio de Entrevista Narrativa, conforme proposto por \textcite{jovchelovitch2002}. A análise dos dados se baseia no modelo proposto por \textcite{lieblich1998}, que permitiu uma compreensão aprofundada das narrativas obtidas. O resultado da análise aponta a importância do uso das Tecnologias Digitais durante o Ensino Remoto Emergencial; evidencia a necessária adaptação dos educadores e jovens às condições difíceis; destaca o papel das Tecnologias Digitais na manutenção das práticas de leitura e escrita; e ressalta o potencial dessas tecnologias no processo educacional. Contudo, aponta para desafios como a desigualdade no acesso às Tecnologias Digitais e a necessidade de uma abordagem crítica em sua utilização. Os resultados da pesquisa reafirmam a necessidade de investimentos na formação de educadores e na infraestrutura tecnológica para promover uma educação de qualidade alinhada às demandas contemporâneas.

\keywords{Linguística Aplicada \sep Língua Portuguesa \sep Ensino Remoto Emergencial \sep Entrevista narrativa \sep Tecnologias digitais}
\end{abstract}

\begin{english}
\begin{abstract}
This research analyzes the perceptions of high school students regarding the teaching of the Portuguese language combined with the use of Digital Technologies in the context of Emergency Remote Teaching. The study is based on the principles of qualitative research, with a focus on narrative inquiry, as proposed by \textcite{todorov1983}. Data generation was conducted through Narrative Interviews, following the approaches of \textcite{jovchelovitch2002}. Data analysis follows the model proposed by \textcite{lieblich1998}, which allowed for an in-depth understanding of the collected narratives. The analysis results highlight the importance of using Digital Technologies during Emergency Remote Teaching; emphasize the necessary adaptation of educators and students to challenging conditions; underscore the role of Digital Technologies in maintaining reading and writing practices; and highlight the potential of these technologies in the educational process. However, they also point to challenges such as inequality in access to Digital Technologies and the need for a critical approach to their use. The research findings reaffirm the necessity of investing in teacher training and technological infrastructure to promote quality education that meets contemporary demands.

\keywords{Applied Linguistics \sep Portuguese language \sep Emergency Remote Teaching \sep Narrative interview \sep Digital technologies}
\end{abstract}
\end{english}
% if there is another abstract, insert it here using the same scheme
\end{polyabstract}

\section{Introdução}\label{sec-intro}
Ao considerar os desafios que os educadores enfrentaram ao ministrar aulas de Língua Portuguesa, doravante denominado, LP, aliadas ao uso das Tecnologias Digitais (TD\footnote{\textcite{ribeiro2012} define as Tecnologias Digitais (TD) como recursos tecnológicos que se popularizaram no Brasil durante a década de 1990, incluindo computadores e softwares que transformaram as práticas de leitura e escrita. Essas tecnologias, longe de substituir os métodos convencionais de letramento, promovem uma reconfiguração da importância de diversos aspectos do aprendizado, especialmente no que se refere à cultura impressa. Dessa forma, as TD influenciam tanto o ensino quanto a pesquisa \cite{ribeiro2012}.}), o Ensino Remoto Emergencial, ERE\footnote{O Ensino Remoto Emergencial (ERE) refere-se a um modelo de ensino implementado de forma temporária e emergencial, em resposta a crises que impedem a realização das aulas presenciais, como ocorreu durante a pandemia da Covid-19. Diferente da Educação a Distância (EaD), que é planejada e estruturada previamente, o ERE caracteriza-se pela rápida transposição das aulas presenciais para o ambiente virtual, utilizando tecnologias síncronas, como videoconferências e transmissões ao vivo, e assíncronas, como plataformas de ensino e materiais digitais. \cite{dutra2025}.}), trouxe desafios para educadores e estudantes, surgiu nosso interesse por compreender essas dinâmicas sob as percepções dos estudantes do Ensino Médio. As experiências que eles viveram durante esse período na educação chamaram atenção em compreender esse modelo de ensino durante a pandemia da Covid-19. Ao fazer um levantamento da literatura disponível, percebemos que a maioria dos estudos e relatos se concentrava nas percepções dos docentes e da comunidade escolar em geral. Esse fato nos mobilizou a questionar: como, de acordo com as narrativas dos jovens, foram conduzidas as aulas de LP integradas ao uso das TD no Ensino Médio durante o ERE\footnote{Este artigo é um recorte da dissertação intitulada “Narrativas sobre a Interface entre a Língua Portuguesa e as Tecnologias Digitais: o que retratam as vozes de jovens do Ensino Médio no Ensino Remoto Emergencial?”, defendida em 03 de maio de 2024 no Programa de Mestrado Acadêmico em Letras (PPGLetras) da Universidade do Estado de Mato Grosso (UNEMAT), campus Sinop, sob a orientação da coautora deste manuscrito.}?

Nos últimos anos as TD ganharam espaço no ensino da língua portuguesa, oferecem oportunidades para aprimorar o ensino e a aprendizagem, além de torná-los mais acessíveis. No entanto, a incorporação desses recursos tecnológicos ao currículo escolar ainda é uma questão que merece um olhar crítico. É importante reconhecer que o uso das TD, por si só, não resolve os vários dilemas que se apresentam ao ensino; é necessário fortalecer a educação de forma abrangente, o que inclui estrutura, valorização e qualificação profissional. Além disso, as TD englobam uma ampla gama de ferramentas, desde plataformas de videoconferência, como Zoom e Google Meet, até aplicativos de gestão de aprendizagem e recursos interativos, que se tornaram fundamentais durante o ERE. Conforme apontado por \textcite[p. 15]{coscarelli2020}, a implementação do ERE ocorreu sem planejamento prévio, e a escolha dos ambientes de aprendizagem foi feita às pressas, sem tempo adequado para os professores se prepararem.

Nesse cenário, percebemos como as TD influenciaram as relações estabelecidas pelos indivíduos com a leitura e a escrita. Elas proporcionam acesso a textos multimodais e abrem novos horizontes para a escrita on-line, conforme apontado por \textcite{barton2015}. Em ambientes como aplicativos de mensagens instantâneas, \textit{blogs}, revistas eletrônicas, sites de relacionamento e Redes Sociais Virtuais, doravante, RSV, é notável o uso da escrita para disseminar conhecimentos, postar comentários, realizar reflexões e relatos pessoais, bem como engajar-se em debates com outras pessoas.

\textcite{paiva2020ensino} destaca que, apesar da resistência às inovações, alguns meios digitais disponibilizados, durante o período de confinamento, podem passar a ser cobradas no futuro. O exemplo citado é o uso do Google Meet como recurso de comunicação e gravação, que, durante o ERE, era oferecido gratuitamente, mas agora apresenta custos associados. Este cenário aponta para desafios financeiros adicionais que podem afetar a sustentabilidade e acessibilidade das práticas educacionais mediadas por TD.

\textcite{barton2015} destacam que os espaços de escrita on-line possibilitam interações e favorecem o compartilhamento, a avaliação e a produção de opiniões públicas. Os autores denominam esse comportamento como "postura," ao referir-se ao posicionamento adotado em relação ao que é expresso. Dessa forma, a escrita on-line amplia as práticas de comunicação e promove a construção ativa e participativa de opiniões públicas.

Estas ideias introdutórias acenam para discussões em torno do ensino de LP e os desafios do ERE em tempos da pandemia da covid-19, das reflexões sobre as aulas de LP mediadas pela cultura digital, bem como do uso das TD aliadas ao ensino de LP.

Na sequência, apresentamos, ainda, a metodologia adotada, como também a análise dos dados e os resultados da pesquisa. E, por fim, as considerações acerca dos resultados apontados 	pela pesquisa.

\section{As aulas de língua portuguesa e os desafios do Ensino Remoto Emergencial}\label{sec-normas}
O ERE configurou-se como uma resposta imediata às adversidades impostas pela pandemia da covid-19, que impactou profundamente as práticas de ensino-aprendizagem em todo o mundo. Diante da necessidade de distanciamento social para conter a disseminação do coronavírus, escolas e instituições de ensino adotaram soluções tecnológicas para garantir a continuidade das atividades educacionais. O uso de plataformas digitais, aplicativos de videoconferência e ambientes virtuais de aprendizagem tornaram-se peças fundamentais nesse período, trazendo tanto desafios quanto oportunidades para alunos e professores.

O campo do ensino de LP, essa mudança evidenciou as potencialidades e limitações das TD como mediadoras do processo educacional. Com base nas perspectivas teóricas de \textcite{andrea2020pesquisando}, que analisa a plataformização e as dinâmicas de controle exercidas por grandes corporações sobre os ambientes digitais, e de \textcite{coscarelli2016}, que explora as possibilidades pedagógicas das novas tecnologias, esta seção aborda as implicações do ERE no ensino de LP. A crescente dependência de plataformas privadas, como Google Meet e Microsoft Teams, durante a pandemia da covid-19 exemplifica como os serviços digitais possibilitam novas formas de interação e impõem restrições que afetam o acesso e a equidade no ensino. Além disso, \textcite{moran2015} fornece subsídios para compreender como as TD podem contribuir para práticas educativas inovadoras, enquanto \textcite{freire1996pedagogia} fundamenta a necessidade de um ensino dialógico e crítico, mesmo em contextos mediados por tecnologias.

Essa base teórica orienta a análise dos desafios e adaptações enfrentados por professores e jovens no cenário emergencial, forneceu um pano de fundo para a discussão dos dados apresentados na seção \ref{sec-modelo}.

\subsection{As aulas de língua portuguesa no ensino médio e a cultura digital}\label{sec-conduta}
No contexto do Ensino Médio, as aulas de LP têm sido diretamente impactadas pela Cultura Digital\footnote{A cultura digital, conforme abordada por \textcite{martino2014}, refere-se ao conjunto de práticas, valores e formas de interação que emergem a partir da digitalização e da incorporação das tecnologias digitais na sociedade. Ela abrange as transformações nas formas de comunicação, as interações sociais e os processos educativos, provocadas pela massificação do uso das tecnologias da informação e da comunicação. A cultura digital não se limita ao uso de ferramentas tecnológicas, mas também envolve a criação de novos comportamentos, hábitos e formas de pensar que emergem com a inserção das tecnologias digitais na vida cotidiana. A obra de Sá Martino destaca como as mídias digitais remodelam as relações entre indivíduos, grupos e instituições, criando um contexto de comunicação caracterizado pela interatividade, pela instantaneidade e pela convergência de diferentes linguagens e mídias.}, que se caracteriza pelo uso intensivo de TD e pela internet em diversas esferas da vida contemporânea. Essa nova configuração cultural oferece possibilidades e desafios ao ensino de LP, especialmente em um cenário em que a BNCC enfatiza o desenvolvimento de competências digitais críticas e éticas \cite{brasil2018}.

Com referência à Cultura Digital na escola, a BNCC preconiza:

\begin{quote}
    Compreender, utilizar e criar TD de informação e comunicação de forma crítica, significativa, reflexiva e ética nas diversas práticas sociais (incluindo as escolares) para se comunicar, acessar e disseminar informações, produzir conhecimentos, resolver problemas e exercer protagonismo e autoria na vida pessoal e coletiva \cite[p. 67]{brasil2018}.
\end{quote}

A BNCC destaca a importância das TD no desenvolvimento do pensamento crítico e na formação dos jovens para o mundo contemporâneo. Contudo, essa inserção exige acompanhamento e responsabilidade no uso. A Competência 5, "Cultura Digital", enfatiza o uso ético das TD, o que possibilita que os jovens utilizem multimídia para aprender e compreender seu papel nas práticas sociais, especialmente nas escolares \cite{brasil2018}.

A Cultura Digital tem ampliado as possibilidades de aprendizado e ensino, oferece recursos como bibliotecas digitais, dicionários on-line, bases de dados linguísticas e plataformas multimídia. No entanto, é importante destacar que, embora esses recursos possam contribuir para a aprendizagem, o acesso à Cultura Digital não pode ser considerado amplamente democratizado. Persistem barreiras, como desigualdades socioeconômicas e desafios tecnológicos, especialmente em um país como o Brasil, marcado por disparidades regionais e sociais. Além disso, as plataformas digitais, ao coletarem e processarem dados dos usuários, impõem limitações no uso pleno de seus recursos. A ideia de que o ensino de LP se torna mais acessível devido à Cultura Digital também pode ser repensada, pois, embora a tecnologia ofereça novos meios de interação, a acessibilidade não é um fenômeno universal. As desigualdades no acesso às tecnologias digitais e as diferentes realidades dos jovens e professores precisam ser consideradas para uma análise mais crítica e realista desse impacto no ensino de LP.

A integração das TD no ensino de LP permite o uso de Ambientes Virtuais de Aprendizagem (AVA), como Moodle e Google Classroom, que facilitam a organização pedagógica e eliminam barreiras geográficas. Paralelamente, os Recursos Educacionais Abertos (REA) oferecem materiais adaptáveis que potencializam o ensino e fomentam a autonomia dos estudantes. Por meio de metodologias ativas, como a aprendizagem baseada em projetos, as TD colocam os jovens como protagonistas do processo educativo, o que os incentiva a participar por meio de pesquisa, criação e compartilhamento de conteúdo digital.

Embora as TD apresentem grandes benefícios, sua integração no ensino de LP exige uma abordagem crítica. Como aponta \textcite{coscarelli2016}, é necessário enfrentar desafios como a avaliação da credibilidade da informação on-line e o estímulo ao desenvolvimento do senso crítico dos jovens. Além disso, a produção textual no ambiente digital é influenciada pela Cultura Digital. Plataformas como \textit{blogs}, \textit{wikis} e redes sociais promovem o desenvolvimento de habilidades de escrita e colaboração, além de incentivar a autoria e a participação em práticas sociais. No entanto, é fundamental questionar em que medida esse processo de escrita é reflexivo e crítico. A literacia midiática é um aspecto fundamental nesse contexto, pois o uso dessas ferramentas digitais sem uma abordagem crítica pode, na verdade, ser prejudicial ao ensino de Língua Portuguesa. Para que o uso das plataformas seja verdadeiramente enriquecedor, é necessário que os alunos desenvolvam a capacidade de analisar e refletir sobre o conteúdo produzido, considerando as implicações sociais, culturais e ideológicas presentes nas interações digitais.

Conforme \textcite{barton2015}, as TD transformaram as atividades de comunicação e redefiniram as práticas culturais e pedagógicas. Esse cenário dinâmico exige que professores e jovens estejam em constante adaptação para aproveitar os benefícios dessas tecnologias enquanto lidam com seus desafios.

A análise da relação entre tecnologia, educação e sociedade exige uma reflexão sobre as dinâmicas atuais das plataformas digitais e sua centralidade no contexto educacional. Ao longo dos últimos anos, o impacto da digitalização no ensino tem sido indiscutível. As TD, especialmente as ferramentas digitais e as plataformas \textit{on-line}, oferecem novas oportunidades de aprendizado.

Por um lado, essas plataformas ampliam o acesso a conteúdos, informações e materiais educativos, proporcionam aos alunos e professores recursos antes inacessíveis. No entanto, por outro, a centralização do acesso a essas ferramentas por grandes corporações digitais cria uma concentração de poder sobre os dados e o controle dos processos comunicacionais, o que limita a ideia de uma verdadeira democratização do conhecimento. Em vez de ser um espaço neutro e acessível a todos, a internet se tornou um ambiente mediado por grandes plataformas que ditam as regras e as condições de acesso e de participação. Esse controle corporativo tem implicações diretas na forma como a informação circula e é consumida.

Além disso, a desigualdade no acesso às TD continua sendo um dos maiores obstáculos para a inclusão digital e, consequentemente, para uma educação de qualidade. A disparidade no acesso à infraestrutura digital e à capacitação dos docentes nas novas tecnologias amplia as diferenças entre escolas localizadas em áreas urbanas e rurais, e entre alunos de diferentes classes sociais. A falta de recursos adequados e de formação contínua para os educadores ainda é uma barreira significativa para a implementação eficaz da tecnologia nas salas de aula.

Portanto, é imprescindível que o debate sobre a educação digital se concentre não apenas na potencialidade das tecnologias, mas também nas questões sociais e estruturais que condicionam seu uso. Políticas públicas voltadas para a educação digital podem promover o acesso às tecnologias e garantir que as condições necessárias para seu uso estejam presentes em todas as escolas, o que inclui desde a infraestrutura tecnológica até a formação crítica e pedagógica dos professores, permitindo que eles possam aproveitar o potencial das plataformas digitais para o aprendizado, sem depender exclusivamente das grandes corporações que dominam o cenário digital.

A verdadeira democratização do acesso à informação e ao conhecimento digital passa por uma reflexão crítica sobre os modelos atuais de plataformização e as implicações sociais e políticas que essas plataformas impõem.

Assim, a Cultura Digital transforma as aulas de LP no Ensino Médio, ao proporcionar novas possibilidades de ensino e aprendizado e impor desafios que exigem reflexão crítica e adaptação constante. O professor de LP desempenha uma função central nesse processo, sendo necessário que esteja em constante atualização para acompanhar as mudanças tecnológicas e pedagógicas que moldam o cenário educacional contemporâneo.


\subsection{O uso das tecnologias digitais aliado ao ensino de língua portuguesa no Ensino Remoto Emergencial}\label{sec-fmt-manuscrito}
O ERE surgiu como uma resposta rápida às circunstâncias impostas pela pandemia de covid-19. Nesse cenário, o uso das TD tornou-se imprescindível para garantir a continuidade do ensino, que incluiu as aulas de LP. Essa modalidade de ensino, no entanto, evidenciou tanto as potencialidades quanto as limitações das TD no processo educativo.

O componente de LP no Ensino Médio, assim como Matemática, pode ser oferecido ao longo dos três anos, como estipulado pela Lei nº 13.415/2017. A BNCC preconiza um ensino centrado no texto, que permita o trabalho de gêneros textuais contemporâneos e digitais, além dos convencionais. O foco recai na análise das linguagens, no fortalecimento das habilidades analíticas de leitura e escrita.

\begin{quote}
    As práticas de linguagem contemporâneas são caracterizadas por uma diversidade de gêneros e textos que incorporam múltiplas formas de comunicação, que inclui elementos visuais, sonoros e multimídia. Além disso, a proliferação de meios de edição tornou possível a qualquer pessoa produzir e compartilhar textos multimídia nas RSV e na internet. Que permite o acesso a uma ampla variedade de conteúdos em diferentes mídias e possibilita a criação e publicação de fotos, vídeos, podcasts, infográficos e outros tipos de conteúdo digital. As pessoas agora têm a capacidade de interagir com a Cultura Digital, como fazer comentários em RSV após ler um livro ou assistir a um filme, seguindo diretores, autores e escritores, que criam playlists, vlogs, vídeos curtos, que escrevem fanfics e tornam-se criadores de conteúdo, como booktubers, entre outras opções. A internet e as RSV sejam espaços democráticos e utilizados por crianças, adolescentes e jovens, surge a questão de como a escola pode considerar essa realidade. É importante que a escola reconheça e integre essas práticas de linguagem contemporâneas em sua abordagem educacional, a fim de preparar os jovens para navegar e criar de forma crítica e responsável no mundo digital \cite[p. 70]{brasil2018}.
\end{quote}


As TD no ensino de LP apresentam desafios e oportunidades. A transformação trazida pelos recursos digitais na metodologia de ensino e aprendizado é indiscutível. Assim, torna-se importante analisar as maneiras pelas quais essas tecnologias podem ser inseridas no currículo de LP. Ao ampliar as fronteiras do conhecimento, as TD inauguram uma nova era para o ensino de LP, promove uma aprendizagem mais dinâmica.

Portanto,

\begin{quote}
    As TD da informação e da comunicação já tiveram várias siglas: TIC, NTIC, TDIC. Esta última parece ter sido assumida, de alguns anos para cá, como a mais interessante, já que integra a palavra digital como especificidade dessas tecnologias das quais queremos tratar. O N, de novas, tem sido abandonado porque, justamente, essas tecnologias já deixaram de ser novas, e a sigla TIC específica muito pouco o que são as tecnologias da informação e da comunicação, que podem, por exemplo, incluir a televisão e o rádio, segundo argumentam alguns pesquisadores \cite[p. 55]{ribeiro2023linguistica}.
\end{quote}

A evolução das Tecnologias Digitais de Informação e Comunicação (TDIC) pode ser observada na transição das siglas TIC (Tecnologias da Informação e Comunicação) e NTIC (Novas Tecnologias da Informação e Comunicação) para TDIC. Essa mudança terminológica reflete o desenvolvimento das próprias tecnologias e a crescente relevância da dimensão digital no cenário atual. A sigla TDIC é mais adequada para capturar a essência das tecnologias contemporâneas, que transformaram os meios de comunicação e influenciam profundamente os processos de acesso à informação e a interação com o mundo.

No contexto educacional, as TDIC representam um avanço fundamental, ao proporcionar uma ampla gama de recursos que tornam o ensino e a aprendizagem mais dinâmicos, interativos e acessíveis. Elas oferecem novas possibilidades no ensino de LP, ao criar ambientes de aprendizagem mais imersivos e conectados à realidade digital dos estudantes. Além disso, as TDIC possibilitam a utilização de plataformas digitais, recursos multimídia e ferramentas colaborativas, para o desenvolvimento de competências críticas e analíticas, fundamentais no ensino contemporâneo.

Contudo, o desafio de adaptar os professores ao ERE evidenciou uma lacuna ponderante: muitos educadores não estavam preparados nem tinham experiência no uso das TDIC no ambiente remoto. Além da dimensão técnica do uso das ferramentas, também se mostrou a necessidade de uma compreensão crítica sobre essas tecnologias, incluindo seus impactos pedagógicos, sociais e éticos. A transição para as aulas remotas não foi apenas uma questão de aprendizado de novas ferramentas, mas também de reflexão sobre a forma como essas tecnologias podem ser utilizadas de maneira crítica no processo de ensino-aprendizagem.

No entanto, é imperativo reconhecer, que os desafios subjacentes e a equidade ao acesso às TD são um ponto forte a ser abordado. Garantir que todos os jovens, independentemente de suas circunstâncias, tenham acesso aos meios necessários para a aprendizagem on-line é fundamental para evitar disparidades educacionais.

O ERE trouxe à tona a importância das TD como suportes mediadores no ensino de LP. Plataformas digitais como Google Classroom, Microsoft Teams e Zoom foram amplamente utilizadas para promover a interação entre professores e jovens, enquanto aplicativos de comunicação instantânea, como WhatsApp, serviram como suporte complementar. Além disso, o uso de suportes colaborativas, como Google Docs e Padlet, incentivou a produção e revisão textual de forma coletiva e remota.

As TD ampliaram as possibilidades pedagógicas, o que permitiu que os jovens acessassem materiais multimídia, como vídeos, áudios e infográficos, que enriqueceram as aulas de LP. Segundo \textcite{moran2015}, a utilização de recursos digitais no ensino possibilita a criação de experiências de aprendizagem mais dinâmicas e engajadoras, que promove a autonomia dos jovens e adapta os conteúdos às suas realidades.

No entanto, o uso das TD no ERE não se limitou à simples transmissão de informações. Metodologias ativas, como a aprendizagem baseada em projetos e a gamificação, foram implementadas para incentivar a participação dos jovens e contextualizar os conteúdos de LP. Por exemplo, atividades como criação de \textit{blogs}, \textit{podcasts} e vídeos estimularam a autoria e a produção textual em formatos digitais, alinhando-se às práticas comunicativas contemporâneas.

Apesar das vantagens oferecidas pelas TD, o ERE também revelou desafios.  Um dos principais problemas foi a desigualdade no acesso às TD e à internet. A efetividade das plataformas digitais é limitada pelas disparidades socioeconômicas, o que impactou a participação de jovens, especialmente aqueles em áreas rurais ou de baixa renda, nas aulas remotas, evidenciando a exclusão digital. De acordo com Carlos \textcite{andrea2020pesquisando}, a plataformização dos serviços digitais e o controle exercido pelas grandes corporações sobre as ferramentas utilizadas no ensino remoto exacerbam essas desigualdades, criando barreiras adicionais ao acesso. Esse fenômeno demonstra como, apesar das possibilidades abertas pelas tecnologias, o controle das plataformas digitais tem um impacto profundo na equidade no acesso à educação.

A adoção do ERE trouxe desafios para os professores de LP, que precisaram aprender rapidamente e dominar TD para o planejamento e condução de aulas em ambientes digitais.

A falta de formação específica tem sido apontada por diversos estudos como um dos principais desafios na integração das TD no ensino. Segundo \textcite[p. 52]{coscarelli2016}, a implementação das TD na educação depende não apenas da disponibilidade de infraestrutura tecnológica, mas também de uma capacitação contínua dos docentes, garantindo que possam utilizar essas ferramentas de maneira pedagógica.

\textcite{moran2015} reforça que a adoção das TD exige não só o domínio técnico, mas também uma compreensão crítica dessas tecnologias, de modo que os professores consigam incorporá-las de forma reflexiva em suas práticas pedagógicas. Para \textcite{kenski2012}, a capacitação docente pode ir além do ensino instrumental das ferramentas digitais, promovendo metodologias inovadoras que favoreçam a aprendizagem ativa dos alunos. Já \textcite{valente2019} destaca que a ausência de formação adequada compromete a apropriação significativa dessas tecnologias, resultando em um uso superficial ou ineficaz no contexto escolar.

Esse cenário evidencia a necessidade de políticas educacionais que priorizem a formação docente como um pilar na adoção das TD, assegurando que seu uso contribua efetivamente para a melhoria do ensino e da aprendizagem. Além disso, é fundamental esclarecer o funcionamento de algumas ferramentas mencionadas, como o Chromebook\footnote{O Chromebook é um \textit{laptop} desenvolvido pelo Google, com o sistema operacional Chrome OS, focado no uso de serviços on-line e armazenamento na nuvem. Ideal para tarefas básicas, como navegação na web e aplicativos de produtividade, requer conexão à internet para acesso aos arquivos. É uma opção popular em ambientes educacionais devido ao seu baixo custo e eficiência \cite{berredo2025}.}, um laptop desenvolvido pelo Google que opera com o sistema Chrome OS. Esse dispositivo é amplamente utilizado no contexto educacional por sua leveza, segurança e integração com plataformas de ensino remoto, tornando-se uma alternativa viável para escolas que adotam o uso de TD no processo de ensino-aprendizagem.

Um dos desafios mais expressivos durante o ERE foi manter o engajamento dos jovens em um ambiente virtual. A ausência de interações presenciais, combinada com as distrações oferecidas pelos meios digitais, comprometeu a concentração e a participação de muitos jovens. \textcite[p. 89]{moran2015} destaca que o uso das TD pode ser acompanhado de estratégias pedagógicas que promovam a interação e a construção coletiva do conhecimento, maximizando a efetividade dos processos de ensino e aprendizagem.

Apesar das dificuldades, o ERE também revelou o potencial transformador das TD na educação. Essa experiência emergencial indicou caminhos para a incorporação definitiva dessas TD às práticas pedagógicas, não como substitutas do ensino presencial, mas como suportes complementares que ampliam as possibilidades de aprendizado. \textcite[p. 104)]{ribeiro2018},  enfatiza que a escrita é uma prática social moldada por contextos históricos e tecnológicos. Assim, o uso das TD durante o ERE garantiu a continuidade do ensino e reforçou a necessidade de preparar os jovens para navegar em ambientes digitais dinâmicos, indispensáveis no cenário atual.

A crítica às práticas tradicionais de ensino é reforçada por \textcite[p. 37]{soares2002}, que argumenta que a escola pode ir além do ensino mecânico, conectando as habilidades de leitura e escrita a práticas sociais reais. Relatos indicam que o uso de metodologias interativas durante o ensino remoto aproximou a experiência virtual das características do ensino presencial, atendendo melhor às demandas contemporâneas.

\textcite[p. 78]{ribeiro2016} observa que a integração das TD no ensino é um processo contínuo e adaptativo, que pode considerar tanto as necessidades dos jovens quanto as possibilidades pedagógicas. O uso combinado de dispositivos, como computadores e celulares, mostrou-se essencial para atender às demandas de plataformas educacionais, como Google Meet e Classroom, proporcionando maior flexibilidade no acesso aos conteúdos e na interação entre professores e jovens.

Nesse sentido, \textcite[p. 45]{barton2015} ressaltam que a adoção das TD pode levar em conta tanto as capacidades técnicas dos dispositivos quanto a maneira como influenciam as interações dos jovens com o conteúdo e entre si. A utilização de Chromebooks nas escolas e dispositivos pessoais em casa reflete essa integração das TD no cotidiano educacional. Essa abordagem está em consonância com a Base Nacional Comum Curricular \cite{brasil2018}, que propõe o desenvolvimento da competência de “Cultura Digital”, focada no uso crítico, reflexivo e responsável das tecnologias, que prepara os jovens para os desafios do século XXI.

O retorno às aulas presenciais não significou o abandono das TD, mas sua incorporação de forma mais equilibrada. Relatos destacam a preocupação com o uso excessivo de dispositivos como celulares, sugere a busca de um equilíbrio entre as TD e as interações presenciais. \textcite[p. 63]{moran2001} reforça essa ideia ao destacar a necessidade de harmonizar os avanços tecnológicos com os valores humanos e a interação direta no processo educativo.

A distribuição de Chromebooks como política pública foi destacada como um esforço relevante para democratizar o acesso às TD. Essa iniciativa reflete as ideias de \textcite[p. 112]{papert1996}, que defende a equidade no acesso às tecnologias como elemento-chave para potencializar as oportunidades de aprendizagem. Da mesma forma, \textcite[p. 94]{castells2009} argumenta que a inclusão digital é um fator essencial para a transformação da sociedade contemporânea, ampliando oportunidades educacionais e fortalecendo a cidadania.

\textcite{papert1996} sugere, também, que, em vez de temer ou limitar o uso das TD pelos jovens, as famílias e os educadores podem adotar uma abordagem construcionista. Nessa perspectiva, as tecnologias são vistas como suportes favoráveis à aprendizagem e à criatividade. Ele propõe que pais e professores incentivem a exploração, a experimentação e a construção de conhecimento mediado por tecnologias, permitindo que as crianças se envolvam ativamente na solução de problemas e na criação de projetos.

Portanto, a experiência com o ERE e o retorno às aulas presenciais reafirmam a relevância das TD como instrumentos transformadores na educação. Investimentos em formação docente, infraestrutura e políticas públicas que promovam a equidade no acesso às TD são indispensáveis para construir uma educação mais inclusiva, dinâmica e conectada às demandas do século XXI.

\section{Metodologia da pesquisa}\label{sec-formato}
Nesta seção descrevemos os procedimentos metodológicos adotados para compreender as experiências de estudantes do EM com o ERE durante a pandemia. A pesquisa é de cunho qualitativo fundamentada em metodologias narrativas que exploram as percepções e histórias dos participantes, o que permitiu uma análise detalhada de suas vivências.

A abordagem qualitativa é caracterizada por seu foco na compreensão de contextos sociais e culturais. \textcite{creswell2014} enfatiza que essa abordagem é adequada para investigações que buscam explorar as subjetividades dos indivíduos em suas interações com o ambiente. Dentro desse escopo, foi utilizada a metodologia narrativa, que, segundo Jovchelovitch e Bauer (2002), oferece suportes para acessar as interpretações dos sujeitos, com respeito as suas construções de significado no contexto sociocultural em que estão inseridos.
Conforme \textcite[p. 23]{flick2009pt},

\begin{quote}
    [...] consiste na escolha adequada de métodos e teorias convenientes; no reconhecimento e análise de diferentes perspectivas; nas reflexões dos pesquisadores a respeito de suas pesquisas como parte do processo de produção de conhecimento; e na variedade de abordagens e métodos.
\end{quote}


Essa citação oferece uma análise sobre a pesquisa qualitativa, com ênfase as contribuições de autores notáveis, como \textcite{denzin2006introducao}.

A pesquisa foi realizada em uma escola pública estadual, localizada em um ambiente urbano de médio porte, com participantes da 3ª série do Ensino Médio. Esses estudantes, com idades entre 17 e 18 anos, vivenciaram o ERE nos anos de 2020 e 2021, período em que o ensino ocorreu predominantemente em plataformas digitais, enfrentando desafios como dificuldades de acesso e adaptação tecnológica. \textcite{bortoni2008} ressalta a importância de considerar o contexto sociocultural para compreender as percepções dos sujeitos sobre os fenômenos estudados, uma premissa central nesta investigação.

Selecionamos oito (8) participantes com base em critérios específicos: matrícula na 3ª série do Ensino Médio durante o ERE, participação ativa nas aulas remotas de LP e consentimento para participação na pesquisa. A escolha do número de participantes foi influenciada por considerações metodológicas que equilibram a profundidade da análise com a viabilidade prática da pesquisa. Segundo \textcite{flick2009pt}, amostras menores em pesquisas qualitativas permitem uma exploração mais rica e detalhada dos fenômenos, uma vez que possibilitam entrevistas mais profundas e maior atenção aos dados individuais. Embora houvesse outros alunos elegíveis, a seleção focou em um grupo que pudesse oferecer uma variedade de perspectivas dentro das limitações do tempo e dos recursos disponíveis. Além disso, a seleção de um número reduzido de participantes pode contribuir para uma análise mais focada e contextualizada, sem perder a relevância do fenômeno em questão.

A identidade dos participantes foi preservada por meio da adoção de nomes fictícios, assegurando a confidencialidade das informações. As entrevistas foram realizadas individualmente em julho de 2023, em um ambiente reservado na escola, com duração média de 40 minutos. O roteiro, composto por perguntas abertas, foi elaborado para explorar as experiências dos jovens com o ensino de LP e o uso de TD. Questões como “Como foram as aulas de LP no ERE?” e “Quais TD você utilizou nas aulas remotas?” foram incluídas para estimular relatos espontâneos e detalhados, permitindo uma compreensão mais profunda das percepções dos estudantes. Em resumo, “[...] uma narrativa é uma sequência singular de eventos, estados mentais e ocorrências que envolvem seres humanos como personagens ou autores” \cite[p. 46]{bruner2002atos}. Bruner enfatiza que essa narrativa pode ser tanto "real" quanto "imaginária" sem perder efetividade enquanto história (p. 47). Em contraposição à visão chomskiana, Bruner (2002, p. 69) destaca, sem diminuir a importância da sintaxe, que os seres humanos "entram na linguagem" com predisposições pré-linguísticas para o significado. Ele resumiu essa ideia ao afirmar que, inicialmente, somos equipados, se não com uma "teoria" da mente, certamente com um conjunto de predisposições para interpretar o mundo social de maneira específica e agir de acordo com essas interpretações.

A geração de dados foi precedida por um teste piloto com três (3) participantes, o que permitiu ajustes no roteiro. Posteriormente, cinco (5) entrevistas adicionais foram realizadas, o que culminou nas oito (8) que formaram o corpus final. Todas as entrevistas foram gravadas e transcritas para análise posterior. Adotamos a metodologia de análise narrativa, conforme descrita por \textcite{lieblich1998}, para estruturar a análise em etapas claras e sistemáticas. Essas etapas incluíram transcrição, codificação inicial, agrupamento e refinamento de temas, além de interpretação e relato dos achados. A combinação das abordagens holística e categórica permitiu uma compreensão integrada do conteúdo e da forma das narrativas.

Do ponto de vista ético, a pesquisa seguiu rigorosamente as orientações do Comitê de Ética em Pesquisa, tendo sido aprovada sob o Parecer nº 6.169.026. Todos os participantes assinaram os Termos de Consentimento Livre e Esclarecido (TCLE) e, no caso de menores de idade, também os Termos de Assentimento Livre e Esclarecido (TALE). \textcite{oliveira2016} destaca a importância de negociações formais com as instituições envolvidas, procedimento que foi cuidadosamente seguido nesta pesquisa. Além disso, preservamos a integridade narrativa dos participantes, conforme recomendam \textcite{todorov1983,barcelos2020}, garantindo que suas histórias fossem representadas de maneira autêntica e respeitosa.

Essa metodologia permitiu compreender como o ensino de LP foi mediado pelas TD no contexto do ERE, a fim de capturar as experiências e percepções dos jovens e, ao mesmo tempo, oferecer contribuições para o campo da educação.


\section{O ensino de língua portuguesa aliado ao uso das TD em contexto ERE: o que narram os jovens do ensino médio?}\label{sec-modelo}
Nesta seção, analisamos o panorama do Ensino de Língua Portuguesa aliado ao uso das TD em contexto ERE: o que narram os jovens do ensino médio? E para alcançar compreensão deste cenário, organizamos a análise em duas (2) categorias, as quais incluem: i) o ensino da Língua Portuguesa; ii) o uso de TD no contexto do ERE, voltadas para o estudo da LP.

No tópico seguinte, apresentamos os dados narrativos seguidos da análise.

\subsection{O ensino língua portuguesa no contexto do Ensino Remoto Emergencial}\label{sec-organizacao}
A análise das experiências dos jovens com o ensino de LP durante o ERE, no período da pandemia, mostrou como as estratégias pedagógicas se adaptaram à nova realidade. A transição para o AVA exigiu a utilização de TD e metodologias inovadoras para garantir a continuidade do ensino, alinhando-se às diretrizes da BNCC (\Cref{tbl1}).

\begin{table}[htbp]
\begin{small}
\caption{O Ensino de língua portuguesa.}
\label{tbl1}
\centering
\begin{tabular}{p{1,5cm} >{\raggedright\arraybackslash}p{4.5cm} >{\raggedright\arraybackslash}p{4cm} >{\raggedright\arraybackslash}p{3cm}}
\toprule
Participante & Transcrição & Primeira redução & Palavras-chave \\
\midrule
Ana & No período da pandemia, a gente tinha uma plataforma
Plurall. Os professores gravavam as aulas ou eram ao vivo. A
gente participava ativamente, e as atividades eram baseadas na apostila.
Utilizávamos o celular para enviar fotos das atividades pela plataforma,
onde recebíamos correções e orientações. & Uso de plataforma digital
para aulas e atividades. Participação ativa dos jovens. Comunicação e
correção de atividades via plataforma. & Plataforma digital,
participação, comunicação, correção de atividades. \\
Ari & Não lembro de quase nada, apenas que a maioria das interações era
por WhatsApp. Eram enviados PDF e textos para leitura e
posterior resposta de perguntas. & Uso predominante do WhatsApp
para envio de materiais e comunicação. & WhatsApp, PDF, textos,
leitura, perguntas. \\
Carol & As aulas foram muito boas, quase como presenciais, graças a uma
professora que explicava bem. Utilizamos livros e o celular para
leituras. Apreciava a convivência e interação na sala de aula virtual. &
Aulas similares às presenciais com uso intensivo de leitura. Boa
comunicação e interação professor-aluno. & Aulas virtuais, leitura,
comunicação, interação. \\
Gabriel & As aulas foram excelentes, com continuidade e bem
desenvolvidas. Tivemos aulas de redação, produzindo textos e cartas
abertas, enviadas através do Google Teams ou Google
Classroom. & Continuidade pedagógica com atividades de escrita. Uso de
suportes Google para envio de trabalhos. & Continuidade, escrita,
Google Teams, Google Classroom. \\
Gabrielly & Mais usava o WhatsApp para receber PDF para
leitura. Respondíamos às perguntas no caderno, tirávamos fotos e
enviávamos. & Uso do WhatsApp para recebimento de materiais e envio de
atividades por foto. & WhatsApp, PDF, atividades, foto. \\
Helena & Lembro pouco, mas percebia o esforço dos professores. Usávamos
WhatsApp e Facebook para receber textos e recomendações de
livros. Respondíamos em apostilas e enviávamos fotos para o professor. &
Esforço dos professores em manter o ensino. Uso de redes sociais para
distribuição de materiais. & Esforço docente, WhatsApp, Facebook,
apostilas. \\
Isabella & Recebíamos videoaulas, inclusive do próprio professor. Com
apostilas ao lado, assistíamos às aulas e respondíamos às atividades.
Produzíamos textos baseados nas videoaulas. & Uso de videoaulas para
instrução e atividades de escrita com apoio de apostilas. & Videoaulas,
apostilas, atividades de escrita. \\
José & As aulas foram boas, com foco na escrita. Durante a pandemia, li
muito. As provas eram baseadas em leituras de livros. & Enfoque na
leitura e escrita. Uso de livros para provas e atividades. & Leitura,
escrita, provas, livros. \\
\bottomrule
\end{tabular}
\source{Dados gerados por meio de entrevista narrativa, 2023.}
\end{small}
\end{table}


O ensino de LP no ERE destacou a importância da TD como suporte pedagógico. A utilização de plataformas como Plurall, Google Teams e WhatsApp permitiu que professores mantivessem a interação com os jovens e garantissem o envio e a correção das atividades. Para Ana, por exemplo, o uso da plataforma digital foi necessário para manter a participação ativa e o acompanhamento pedagógico.

\textcite{ribeiro2018} contribui para essa discussão ao afirmar que a escrita é uma prática social moldada por contextos históricos e tecnológicos. O uso de TD durante o ERE viabilizou o ensino e reforçou a necessidade de ensinar os jovens a navegar em um ambiente digital dinâmico. Essa abordagem é central em um cenário onde a fluência digital é indispensável.
Por outro lado, o relato de Gabriel sobre a continuidade das aulas de redação ilustra como as práticas pedagógicas se alinharam às diretrizes da BNCC. A produção de textos e o trabalho com gêneros textuais, incluindo os multissemióticos e multimidiáticos, são exemplos de como o currículo foi adaptado ao ambiente virtual. Segundo a BNCC, essas práticas visam formar jovens aptos a lidar com uma sociedade informacional e multimodal.

\textcite{soares2002} critica as práticas tradicionais de ensino, argumenta que a escola pode ir além do ensino mecânico e conectar as habilidades de leitura e escrita a práticas sociais reais. A experiência de Carol, que descreveu suas aulas on-line como quase presenciais, reflete essa transição para metodologias mais interativas e conectadas às necessidades do mundo contemporâneo.

É importante reconhecer as desigualdades no acesso às TD, evidenciadas pelos relatos de participantes como Ari e Gabrielly, que dependiam do WhatsApp para receber materiais. Esse cenário aponta para a necessidade de políticas públicas que garantam infraestrutura tecnológica e formação docente para que o ensino de LP seja inclusivo e acessível.

O período de ERE destacou tanto os desafios quanto as oportunidades para o ensino de LP. O uso de TD demonstrou potencial para enriquecer as práticas pedagógicas e aproximá-las das diretrizes da BNCC, mas também evidenciou a urgência de investimentos em infraestrutura e formação docente. Autores como \textcite{soares2002,ribeiro2018} oferecem perspectivas importantes para repensar o ensino, ao enfatizar a importância de metodologias que integrem tecnologia e práticas sociais em uma abordagem holística.

\subsection{As tecnologias digitais utilizadas no Ensino Remoto Emergencial}\label{sec-organizacao-latex}
No cenário educacional contemporâneo, marcado pela influência das TD, o ERE emergiu como uma resposta prática e imediata à pandemia de Covid-19, proporcionando novas experiências e desafios no ensino de LP. Neste contexto, as TD foram integradas como meios centrais no processo de ensino e aprendizagem, possibilitou estratégias diversas, como aulas gravadas, encontros em tempo real e atividades assíncronas, adaptadas às diferentes rotinas dos jovens.

Plataformas como Plurall, Google Teams e Google Classroom foram utilizadas para fomentar a interação entre professores e jovens, viabilizar a entrega de tarefas e promover a troca de conhecimentos. Esses suportes ilustram como as TD assumiram um papel central na continuidade do ensino durante a pandemia e abriram novas perspectivas para abordagens pedagógicas, caracterizando-se pela flexibilidade e constante evolução no cenário educacional contemporâneo (\Cref{tbl2}).

\begin{table}[htbp]
\caption{As tecnologias digitais utilizadas no Ensino Remoto
Emergencial.}
\label{tbl2}
\small
\centering
\begin{tabular}{p{1.5cm} >{\raggedright\arraybackslash}p{4.5cm} >{\raggedright\arraybackslash}p{4cm} >{\raggedright\arraybackslash}p{3cm}}
\toprule
Participante & Transcrição & Primeira redução & Palavras-chave \\
\midrule
Ana & Computador, a gente tem mais ações. & Uso do computador para aulas
e atividades. & Computador, atividades. \\
Ari & Era muito por WhatsApp só. & Uso predominante do
WhatsApp para comunicação e recebimento de materiais. &
WhatsApp. \\
Carol & Foi pelo celular. & Uso do celular para acesso às aulas e
atividades. & Celular. \\
Gabriel & Eu acho que o notebook foi melhor. & Preferência pelo uso do
\emph{notebook} para as aulas remotas. & Notebook. \\
Gabrielly & Assistir vídeos de aulas, que o professor recomendava. & Uso
de vídeos de aulas recomendados pelo professor. & Vídeos de aulas,
recomendação. \\
Helena & Telefone, celular, notebook. E de aplicativo a gente
tinha o Classroom, que o professor colocava as atividades lá e a
gente, todo dia, dava para encaminhar as apostilas também em casa. & Uso
de múltiplos dispositivos e o Google Classroom para gestão das
atividades. & Telefone, celular, notebook, Google Classroom. \\
Isabella & Celular, computador, \emph{tablet}... & Uso de diversos
dispositivos para as aulas remotas. & Celular, computador,
\emph{tablet}. \\
José & Olha, eu usei os dois, porque geralmente a escola exigia o
computador, né, pra aparecer nosso rosto, e o celular pra acompanhar as
tarefas que eles passavam nas plataformas, como o Classroom, Meet,
Teams. & Uso combinado de computador e celular para participação em
vídeo e acompanhamento de tarefas em plataformas digitais. & Computador,
celular, Google Classroom, Google Meet, Microsoft Teams. \\
\bottomrule
\end{tabular}
\source{Dados coletados por meio de entrevista narrativa, 2023.}
\end{table}

A análise das narrativas evidencia uma diversidade de dispositivos e plataformas utilizadas pelos jovens, refletindo tanto suas preferências pessoais quanto as condições socioeconômicas e pedagógicas.

\textcite{ribeiro2016} observa que a integração das TD no ensino é um processo contínuo e adaptativo, que pode considerar as necessidades dos jovens e as possibilidades pedagógicas. O relato de José exemplifica essa realidade: o uso combinado de computador e celular atende às diferentes demandas das plataformas utilizadas para ensino, como o Google Meet e o Classroom, permitiu maior flexibilidade no acesso ao conteúdo e interação com os professores.

O acesso a múltiplos dispositivos, como destacado por Gabriel e Helena, mostrou a importância da infraestrutura tecnológica para o sucesso do ERE. Gabriel, ao preferir o notebook para as aulas, ilustra como dispositivos mais robustos facilitam a realização de atividades mais complexas, enquanto Helena descreve o uso do Google Classroom como um suporte central para a gestão das tarefas. \textcite{barton2015} ressaltam que a adoção de TD pode levar em conta tanto as capacidades técnicas dos suportes quanto a maneira como elas influenciam a interação dos jovens com o conteúdo e entre si.

Por outro lado, Ari e Carol destacam o uso predominante do celular e do WhatsApp, suportes mais acessíveis e amplamente difundidas. Este cenário aponta para a desigualdade no acesso às tecnologias, um dos principais desafios do ERE, como também discutido por \textcite{soares2002}, que enfatiza a necessidade de políticas públicas para democratizar o acesso aos recursos educacionais.

A análise histórica de \textcite{barton2015} sobre a evolução das mídias digitais na educação complementa essa discussão, evidencia que, embora suportes como e-mails e mensagens instantâneas possam ser consideradas "antigas", elas continuam sendo relevantes no contexto educacional contemporâneo. A narrativa de Gabrielly sobre o uso de vídeos recomendados pelo professor reforça a importância de integrar diferentes mídias para enriquecer o processo de ensino e aprendizagem.

A inserção das TD no ERE transcendeu a simples adaptação ao ERE, ao oferecer novas possibilidades pedagógicas e destacar a necessidade de reflexão contínua sobre a função dessas TD na educação. Os relatos analisados ilustram os benefícios das TD, como maior flexibilidade e interatividade e os desafios, como o acesso desigual aos dispositivos e a necessidade de formação contínua aos docentes.

Autores como \textcite{ribeiro2016,barton2015,soares2002} nos lembram que a utilização das TD no ensino não é um fenômeno recente, mas um processo em constante evolução, que requer a integração de práticas inovadoras e inclusivas. No futuro, será necessário que as instituições de ensino e os formuladores de políticas educacionais invistam em infraestrutura e formação contínua, garantindo que a educação mediada por tecnologias seja efetivamente acessível e equitativa para todos os jovens.

\section{Conclusão}\label{sec-titulo}
A essência desta pesquisa reside na análise das narrativas de jovens do EM sobre a interface entre a LP e o uso das TD no ERE durante a pandemia da covid-19. O propósito central foi desvendar quais TD e recursos digitais foram selecionados e de que maneira esses suportes foram integrados nas aulas de LP, além de avaliar o impacto dessas TD no processo de ensino e aprendizagem.

Quanto às TD e plataformas utilizadas para a condução das aulas no formato remoto, os jovens narraram sobre uma diversidade de dispositivos, como computadores, celulares, tablets e notebooks. Além disso, citaram plataformas como Plurall, Google Teams, Google Classroom e Google Meet, que destacam a importância desses meios para viabilizar as atividades escolares durante o período remoto. Apesar das limitações do ambiente virtual, os educadores buscaram manter a continuidade do processo de ensino. Estratégias como o envio de PDF, textos para leitura e correções por meio de plataformas on-line demonstraram esforços para garantir a interatividade e a avaliação do aprendizado.

No contexto de adaptação, os dados narrativos mostraram a capacidade de professores e jovens em lidar com as condições desafiadoras impostas pelo ERE. Os educadores ajustaram suas práticas pedagógicas, incorporaram meios digitais e exploraram novas metodologias para manter o engajamento dos jovens. Da mesma forma, os estudantes demonstraram resiliência ao adaptarem-se a ambientes virtuais de aprendizado, exibindo disposição para superar obstáculos e aproveitar ao máximo as oportunidades oferecidas.

Embora o acesso às tecnologias tenha se ampliado nos últimos anos, as narrativas apontam para a persistência de desigualdades no acesso às TD. Essa constatação ressalta a importância de abordar questões relacionadas à democratização do uso das tecnologias, garantindo acesso universal e gratuito à informação e aos recursos educacionais.

As reflexões de estudiosos indicam que a educação passou por transformações profundas durante o ERE, expondo desigualdades sociais, tecnológicas e econômicas. Apesar das contribuições deste estudo, algumas limitações são inerentes. A principal reside na dependência das respostas dos participantes, sujeitas à subjetividade e possivelmente incapazes de abranger a totalidade das experiências dos jovens durante o ERE. Além disso, a ausência de acesso universal às TD pode ter influenciado as percepções relatadas, ampliando as desigualdades no uso desses recursos.

Diante das limitações mencionadas, sugerimos que futuras pesquisas explorem mais profundamente as desigualdades no acesso às TD, propondo estratégias para garantir uma distribuição mais equitativa. Além disso, investigações sobre os efeitos a longo prazo da integração das TD no ensino de LP podem fornecer reflexões para o desenvolvimento de políticas públicas educacionais.

A metodologia de pesquisa narrativa adotada neste estudo permitiu uma compreensão holística dos fenômenos estudados, destacando as perspectivas dos jovens, que muitas vezes não são ouvidas nas discussões sobre educação. Ao dar voz aos estudantes, essa abordagem proporcionou uma visão mais ampla das experiências vividas por eles, valorizando suas histórias pessoais como fontes importantes de compreensão das interações linguísticas e culturais em diversos contextos.

As contribuições e perspectivas de estudos futuros em pesquisa narrativa são amplas e promissoras. Investigações interculturais poderiam destacar as diferenças nas experiências linguísticas e tecnológicas de estudantes em diferentes regiões do Brasil. Ademais, explorar o desenvolvimento de habilidades digitais e sua relação com a proficiência linguística pode fornecer reflexões para aprimorar práticas pedagógicas e promover uma educação mais inclusiva. Além disso, seria relevante examinar as narrativas de professores sobre o ensino de LP e TD, buscando compreender suas experiências e identificar práticas que possam contribuir para o desenvolvimento de políticas e programas educacionais alinhados às necessidades de educadores e estudantes.

Portanto, o estudo baseado em narrativas ofereceu uma visão sobre as experiências vivenciadas pelos jovens estudantes, destacando a importância de uma abordagem reflexiva e inclusiva na educação. Os desafios e benefícios das TD exigem investimentos contínuos em formação docente e infraestrutura tecnológica, promovendo ambientes de aprendizagem mais dinâmicos e conectados às demandas do século XXI.


\printbibliography\label{sec-bib}
% if the text is not in Portuguese, it might be necessary to use the code below instead to print the correct ABNT abbreviations [s.n.], [s.l.]
%\begin{portuguese}
%\printbibliography[title={Bibliography}]
%\end{portuguese}


%full list: conceptualization,datacuration,formalanalysis,funding,investigation,methodology,projadm,resources,software,supervision,validation,visualization,writing,review
\begin{contributors}[sec-contributors]
\authorcontribution{Jose Isavam Oliveira Silva}[conceptualization,datacuration,investigation,methodology,software,visualization,writing,review]
\authorcontribution{Albina Pereira de Pinho}[supervision,software,validation]
\end{contributors}


\end{document}

