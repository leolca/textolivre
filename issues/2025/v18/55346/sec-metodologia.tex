\section{Metodologia}\label{sec-metologia}

Como esta pesquisa possui um viés linguístico dentro do PLN, a proposta
dos métodos e de seus respectivos algoritmos refletirá essa perspectiva,
como apontam \textcite{rodrigues2022}. Destaca-se que o ambiente
utilizado para criar, treinar e testar os modelos apresentados neste
trabalho foram configurados na versão 3.12.3 do Python, tendo como
sistema operacional o POP!\_OS 22.04. Ademais, as bibliotecas utilizadas
em Python serão especificadas ao longo desta seção. Destaca-se que o
código completo para desenvolvimento deste estudo, bem como sua
documentação detalhada, está disponível na plataforma GitHub dos
autores\footnote{Disponível em: \url{https://github.com/semcovici/AMR-BP-prediction}}.

A proposta de classificação de papéis semânticos neste trabalho segue a
proposta metodológica de \textcite{ilmy2021}. Os autores propõem
identificar relações linguísticas utilizando uma estrutura hierárquica
entre \emph{parent} e \emph{child}, sendo, respectivamente, as palavras
de saída e de chegada em uma representação em grafo de sentenças
anotadas com AMR. Após isso, constrói-se um arquivo no formato texto
(com extensão txt) em que todas essas informações estão organizadas. A
fim de exemplificação, apresentamos a sentença anotada em \ref{itm4}, bem como
o \Cref{tab-02}, em que as informações linguísticas estão organizadas.

\begin{enumerate}[start=4,label={(\arabic{enumi})}]
  \item\label{itm4}
    \begin{enumerate}[label=(\arabic{enumi}.\alph*)]
      \item\label{itm4a} Sentença: Meyer tem duas explicações para esses resultados.
      \item\label{itm4b} Anotação: (h3 / ter-01 :ARG0 (p0 / person :name (n1 / name :op1
  "Meyer") :ARG1 (e5 / explicar-01 :ARG0 p0 :ARG1 (t8 / coisa :mod (t7 /
  esse) :ARG2-of (r9 / resultar-01)) :quant 2) :wiki -))
    \end{enumerate}
\end{enumerate}


\begin{table}[htpb]
  \centering
  \begin{threeparttable}
  \caption{Mapeamento estrutural em \emph{parent} e \emph{child} de informações linguísticas.}
  \label{tab-02}
  \begin{tabular}{
  >{\raggedright\arraybackslash}p{.35\textwidth} 
  >{\raggedright\arraybackslash}p{.3\textwidth} 
  >{\raggedright\arraybackslash}p{.3\textwidth}
  }
  \toprule
  \multirow{3}{=}{Dependência sintática} & \emph{parent} &  ter \\
  &  \emph{child} &  Pessoa (\emph{person})  \\
  & 5 dep &  nsubj \\
  \multirow{4}{=}{Características ligadas a \emph{parent}} &  Texto (\emph{Text}) &  tem \\
  & Lema (\emph{Lemma}) & ter \\
  & Etiqueta morfossintática (POS) & verbo (VERB) \\
  & Entidade nomeada &  \emph{não se aplica} \\ 
  \multirow{4}{=}{Características ligadas a \emph{child}}
  & Texto (\emph{Text}) & Meyer \\
  & Lema (\emph{Lemma}) & Meyer \\
  & Etiqueta morfossintática (POS) & Nome próprio (PROPN) \\
  & Entidade nomeada & Pessoa (pers) \\
  Papel semântico & Rótulo (Label) &  Arg0 \\
  \bottomrule
  \end{tabular}
\source{Elaborado pelos autores.}
\end{threeparttable}
\end{table}

No \Cref{tab-02}, observam-se características extraídas da relação apenas
entre as palavras ``Meyer'' e ``ter''. Para que seja feita a
classificação do papel (a saber, Arg0), são consideradas informações
morfológicas, morfossintáticas e de Entidades nomeadas tanto da palavra
na posição \emph{parent}, quanto da palavra na posição \emph{child}.
Tais informações servirão de \emph{features} para os modelos automáticos
de classificação, ao passo que o valor da aresta entre \emph{parent} e
\emph{child} (a saber, a dependência sintática) servirá como variável
\emph{target}.

Nesta pesquisa, foram utilizados os quatro \emph{corpora} do AMR-BP, os
quais foram padronizados na extensão txt. Tal procedimento não foi
necessário para o \emph{corpus} PropBank, que estava em formato CoNLL.
Em seguida, foi gerado um arquivo de extensão json para todos os
\emph{corpora}. Para extrair as informações linguísticas exemplificadas
no \Cref{tab-02}, foi utilizada a biblioteca spaCy \cite{spacy2024}. Ao final, o
arquivo json contém as seguintes informações:

\begin{itemize}
\item ID da sentença;
\item Nome do \emph{corpus} de origem;
\item Informações dos nós: IDs dos nós e seus respectivos valores;
\item Informações das arestas: ID das arestas e os nós que elas conectam;
\item Sentença original;
\item Sentença anotada em AMR;
\item Alinhamentos entre itens léxicos e conceitos AMR, quando for possível
  criar ou estiverem disponíveis. O \emph{corpus news} possui parte de
  suas sentenças com alinhamentos; no caso do PropBank, é possível criar
  os alinhamentos dada a formatação CoNLL do \emph{corpus};
\item Sentença com anotação linguística, contendo texto, lema, POS,
  dependência e Entidades nomeadas.
\end{itemize}

Algumas sentenças dos \emph{corpora} não apresentaram alinhamentos entre
as palavras analisadas como \emph{parent} e \emph{child}. Nesse sentido,
optou-se pela inferência utilizando o método \emph{match} da
\emph{string} do AMR com a sentença original. Porém, houve casos em que
isso não foi possível, já que a anotação semântica não apresentava uma
palavra/\emph{token} correspondente ao texto original, como ``Tenho de
admitir'' em relação à anotação ``\emph{obligate}''. Assim, ao final,
somaram-se 2.212 casos em foi possível inferir o alinhamento, frente a
3.760 em que não foi possível, sendo estes últimos desconsiderados neste
trabalho.

Ao final, obteve-se a seguinte distribuição de papéis semânticos: (i)
Arg0 com 4.154 exemplares; (ii) Arg1 com 7.052; (iii) Arg2 com 1.714;
(iv) Arg3 com 188; (v) Arg4 com 137; e (vi) Arg5 com 1 único exemplar.
Diante dessa distribuição, optou-se por não considerar Arg5 por conta de
sua baixa representatividade. Além disso, dado o desbalanceamento dos
outros papéis, optou-se por realizar dois experimentos para a construção
dos classificadores automáticos: Experimento 1, utilizando apenas os
dados relativos a Arg0 e Arg1; Experimento 2, utilizando os dados
relativos a Arg0 até Arg4.

A construção do modelo de predição de papéis semânticos deste trabalho
baseou-se em seis etapas. Na primeira, realizou-se a separação dos
conjuntos de treino e teste. Os dados provenientes do PropBank foram
utilizados apenas para treino; já os dados restantes foram divididos em
80\% para treino e 20\% para teste, considerando a informação
``\emph{label}'' das instâncias que, neste trabalho, são os próprios
papéis semânticos.

Na segunda etapa foi necessário aplicar a vetorização em variáveis
preenchidas com valor textual (ou seja, categóricas). Para tanto, foi
aplicada a técnica \emph{One-Hot Encoding} (Rodríguez \emph{et al}.,
2018). Ainda, para as \emph{features} relacionadas ao lema das palavras
das instâncias, foi feita a vetorização utilizando o modelo Word2Vec de
\emph{Continuos Bag-of-Words} com 300 dimensões de \textcite{hartmann2017}.

A terceira etapa consistiu na sequência organizada de subetapas. Foi
utilizado o método Pipeline do \emph{imbalanced-learn}\footnote{Disponível
  em:
  \href{https://imbalanced-learn.org/stable/references/generated/imblearn.pipeline.Pipeline.html}{Pipeline
  --- Version 0.12.3}. Acesso em: 20 jul.2024.}, consistindo em:

\begin{itemize}
\item \emph{Normalização}, em que as variáveis foram normalizadas utilizando
  \emph{Z-score}\footnote{O Z-score mede quantos desvios padrão um valor
    está distante da média da distribuição. Trata-se de uma forma de
    transformação dos dados que torna a distribuição deles mais
    compreensível e comparável entre si.} através do método
  \emph{StandardScaler}\footnote{Disponível em: \href{https://scikit-learn.org/stable/modules/generated/sklearn.preprocessing.StandardScaler.html}{StandardScaler
    -- scikit-learn 1.5.1 documentation}. Acesso em: 20 jul.2023.} do
  \emph{scikit-learn} (Pedregosa \emph{et al.}, 2011);
\item \emph{Seleção de melhores features}, em que foram selecionadas as
  melhores features por meio do método
  \emph{SelectPercentile}\footnote{Disponível em:
    \href{https://scikit-learn.org/stable/modules/generated/sklearn.feature_selection.SelectPercentile.html}{SelectPercentile
    -- scikit-learn 1.5.1 documentation}. Acesso em: 20 jul.2023.} do
  \emph{scikit-learn};
\item \emph{Sobre-amostragem}, em que o método \emph{Synthetic Minority
  Over-sampling Technique} (SMOTE), do
  \emph{imbalanced-learn}\footnote{Disponível em:
    \href{https://imbalanced-learn.org/stable/references/generated/imblearn.over_sampling.SMOTE.html}{SMOTE
    --- Version 0.12.3}. Acesso em: 20 jul. 2023.} \cite{lemaitre2017}, foi utilizado para sobre-amostrar as
  classes minoritárias até que elas possuíssem o mesmo número de
  amostras da classe majoritária;
\item \emph{Classificação}, em que foi utilizado o algoritmo \emph{Extreme
  Gradient Boosting} (XGBoost) \cite{chen2016} da biblioteca
  XGBoost\footnote{Disponível em:
    \href{https://xgboost.readthedocs.io/en/stable/python/python_api.html\#xgboost.XGBClassifier}{Python
    API Reference -- xgboost 2.1.0 documentation}. Acesso em: 20
    jul.2023.}.
\end{itemize}

A quarta etapa consistiu no ajuste de hiperparâmetros do algoritmo
XGBoost. Trata-se de uma etapa necessária para otimizar o desempenho do
modelo de classificação. Para tanto, foi utilizado o método
\emph{RandomizedSearch} do \emph{scikit-learn}\footnote{Disponível em:
  \href{https://scikit-learn.org/stable/modules/generated/sklearn.model_selection.RandomizedSearchCV.html}{RandomizedSearchCV
  --- scikit-learn 1.5.1 documentation}. Acesso em: 20 jul.2024.}, que
realiza uma validação cruzada com combinações aleatórias de parâmetros
pré-definidos.

A quinta etapa de avaliação foi realizada a partir das métricas
clássicas de AM, a saber:

\begin{itemize}
\item
  \emph{Precisão} (P), em que é medida a proporção de verdadeiros
  positivos entre os exemplos classificados como positivos pelo modelo.
  Para tanto, baseia-se no seguinte cálculo: \(P = \frac{VP}{VP + FP},\)
  em que VP é a quantidade de verdadeiros positivos, e FP é o número de
  falsos positivos.
\item
  \emph{Revocação} (R), em que é medida a proporção de verdadeiros
  positivos entre os exemplos que são realmente positivos. Para tanto,
  baseia-se no seguinte cálculo: \(R = \frac{VP}{VP + FN},\) em que FN é
  o número de falsos negativos.
\item
  \emph{Medida-F} (MF), que é a média harmônica entre P e R,
  proporcionando balanceamento entre as duas métricas. Para tanto,
  baseia-se no seguinte cálculo:
  \(MF = 2\frac{Precision \cdot Recall}{Precision + Recall}\)
\item
  \emph{Acurácia} (A), em que é medida a proporção de exemplos
  corretamente classificados pelo modelo entre todos os exemplos. Para
  tanto, o cálculo é feito da seguinte maneira:
  \(acur\acute{a}cia = \frac{VP + VN}{VP + VN + FP + FN}\), em que VN é a
  quantidade de falsos negativos.
\end{itemize}

Por fim, a sexta etapa consistiu na seleção de \emph{features} mais
relevantes para a classificação de papéis semânticos. Foram utilizados
os métodos \emph{feature\_importance} do algoritmo XGBoost e a
biblioteca SHAP (\emph{SHapley Additive exPlanations}) \cite{lundberg2017}, a qual permitiu comparar as \emph{features} mais importantes para
cada tipo de argumento.
