\section{Considerações finais}\label{sec-considerações}

Ao longo deste trabalho, como demonstrado, utilizaram-se métodos
automáticos de anotação de papéis semânticos em textos da língua
portuguesa a partir de um \emph{corpus} multigênero anotado com o modelo
AMR. É importante ressaltar que a proposta AMR objetiva explicitar o
conhecimento semântico em construções linguísticas visando a
implementação computacional. Porém, sua base teórica tem nascedouro e
correlação direta com teorias linguísticas, como apontado neste
trabalho.

Nesse sentido, destaca-se a importância de implementar
computacionalmente uma proposta que modele o sentido de maneira
explícita. Esta demonstra ser uma alternativa bastante interessante
frente às propostas neurais atuais da Inteligência Artificial, em que há
dificuldade de se obter explicabilidade das correlações semânticas, que
podem produzir equívocos e ``alucinações''.

Os resultados deste trabalho evidenciam uma melhor performance do modelo
com relação à distinção entre Arg0 e Arg1. Um dos fatos que justifica
esse apontamento é a grande quantidade de casos no \emph{corpus},
refletindo a realidade para além do conjunto de dados. Entretanto, o
modelo enfrenta desafios significativos quando a tarefa de classificação
se estende a um conjunto maior e mais diversificado de argumentos, como
o acréscimo de Arg2, Arg3 e Arg4.

Se, por um lado, as dificuldades enfrentadas pelo modelo desenvolvido
neste trabalho salientam os desafios impostos pela própria língua, por
outro indicam a necessidade de técnicas adicionais para lidar com o
desbalanceamento entre os argumentos, como demonstrado. Nesse sentido,
em trabalhos futuros, caberá o uso de estratégias de reamostragem (real
ou artificial) e/ou ajuste de hiperparâmetros que penalizem mais
severamente equívocos na classificação em classes com menor quantidade
de exemplares.

Após observar as métricas de avaliação, é possível constatar que, a
despeito dessas limitações, os resultados do modelo de classificação
desenvolvido neste trabalho são compatíveis com o cenário atual de PLN
em PB. Tais constatações também vão ao encontro das reflexões
linguísticas tecidas e demonstradas ao longo da discussão. Nesse
sentido, é possível destacar que o trabalho foi bem-sucedido.

Quanto aos trabalhos futuros, destaca-se ainda que os resultados são
condizentes com a quantidade de dados que se tinha à disposição na época
para compor etapas de treinamento e teste do modelo de classificação.
Caso a quantidade de instâncias analisadas fosse maior, sobretudo nas
classes Arg3, Arg4 e Arg5, é possível que o desempenho do modelo pudesse
apresentar melhores resultados de classificação. Assim, caberá aumentar
a quantidade de exemplos das classes minoritárias e refazer as etapas de
treinamento e teste, avaliando o modelo novamente.

Além dessa estratégia, uma outra que pode ser adotada em trabalhos
futuros está relacionada à padronização dos dados. É possível que
utilizar um alinhador AMR para inferir os alinhamentos das instâncias
aprimore o desempenho do classificador. Porém, em PB, há escassez de
ferramentas com esse propósito, quando comparado ao cenário de pesquisas
em língua portuguesa.

Para o leitor interessado, mais detalhes sobre este trabalho podem ser
encontrados no portal web do projeto POeTiSA\footnote{https://sites.google.com/icmc.usp.br/poetisa}.

