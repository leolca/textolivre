\section{Introdução}\label{sec-intro}

As pesquisas em Processamento de Linguagem Natural (PLN), de maneira
geral, buscam ``investigar e propor métodos e sistemas de processamento
computacional da linguagem humana'' \cite[p.~10]{caseli2023}. 
%\alert{(Caseli; Nunes; Pagano, 2024, p.10)}. 
Nesse ínterim, as pesquisas podem desenvolver ferramentas, recursos
e/ou aplicações, que abarcam fenômenos linguísticos entre os níveis
fonético/fonológico e discursivo. A depender do nível linguístico a ser
analisado, os fenômenos podem ser mais ou menos complexos e abstratos do
ponto de vista de representação formal e de processamento automático.
Assim, quanto mais próximo do nível discursivo, mais subjetivos os
fenômenos tendem a ser, ao passo que, quanto mais próximo do nível
fonético/fonológico, mais objetivos.

Quanto ao processamento automático do nível semântico, em particular,
tem-se desafios complexos, já que há fenômenos com menos possibilidades
de serem descritos objetivamente. Com efeito, nem todas as propostas de
metodologias e teorias que nascem nos estudos linguísticos são passíveis
de pronta implementação computacional, dado o desafio de modelar
computacionalmente fenômenos e perspectivas não discretas. Isso
significa dizer que, diante da ausência de traços característicos que
possibilitem distinção entre classes observáveis, por exemplo, sistemas
computacionais podem não atingir bom desempenho.

Apesar do desafio de modelar computacionalmente o sentido, é possível
que utilizar aspectos sintáticos contribua positivamente para essa
tarefa, tal como a perspectiva de papéis temáticos. \textcite[p. 39]{cançado2017} definem esse conceito como noções semânticas que
``apresentam relações diretas com estruturações e propriedades
sintáticas''. Considere as sentenças ilustradas abaixo.

\begin{enumerate}[label={(\arabic{enumi})}]
  \item\label{itm1}
    \begin{enumerate}[label=(\arabic{enumi}.\alph*)]
      \item\label{itm1a} João quebrou a mesa.
      \item\label{itm1b} A mesa se quebrou.
      \item\label{itm1c} A mesa foi quebrada por João.
    \end{enumerate}
\end{enumerate}

Nas sentenças em \ref{itm1}, a função semântica de ``a mesa'' será sempre de
paciente, sendo caracterizada como entidade que sofre o efeito de alguma
ação, havendo mudança de estado \cite{cançado2017}. Entretanto, em
\ref{itm1a}, a função sintática é de complemento, enquanto em \ref{itm1b} e em \ref{itm1c} é
de sujeito. Já ``João'' exerce a função semântica de agente, que seria o
desencadeador de alguma ação, capaz de agir com controle \cite{cançado2017}; entretanto, em \ref{itm1a}, sintaticamente exerce a função de
sujeito e em \ref{itm1c} aparece na posição de adjunção. Nesse sentido, \textcite[p.~40]{cançado2017} destacam que ``apesar de os argumentos estarem em
diferentes posições sintáticas, as sentenças não são distintas e sem
relação'', pois são assertivas sobre o mesmo evento, diferenciando-se
apenas de pontos de vista.

Há diversos autores que apresentam conjuntos de papéis temáticos a
partir de descrições linguísticas, como \textcite{halliday1966}, \textcite{fillmore1968}, \textcite{chafe1970} e \textcite{jackendoff1976} e, para citar estudos do
Português do Brasil (PB), \textcite{geraldi1987} e \textcite{cançado2016} De maneira mais específica, \textcite{cançado2012} propôs uma tipologia
mais abrangente de papéis para o PB, conforme demonstrado no \Cref{tab-01}.

\begin{table}[htpb]
  \centering
  \footnotesize
  \begin{threeparttable}
  \caption{Caracterização dos papéis temáticos para o PB.}
  \label{tab-01}
  \begin{tabular}{
  >{\raggedright\arraybackslash}p{.2\textwidth} 
  >{\raggedright\arraybackslash}p{.35\textwidth} 
  >{\raggedright\arraybackslash}p{.35\textwidth}}
    \toprule
    PAPEL & DEFINIÇÃO & EXEMPLO \\
    \midrule
    Agente & Desencadeador de alguma ação, capaz de agir com controle & O \emph{motorista} lavou o carro. O \emph{atleta} correu. \\
    Causa & Desencadeador de alguma ação, sem controle & As \emph{provas} preocupam a Maria. O \emph{sol} queimou a plantação. \\
    Paciente & Entidade que sofre o efeito de alguma ação, havendo mudança de estado & O João quebrou \emph{o vaso}. O acidente machucou \emph{a Maria}. \\
    Tema & Entidade transferida, física ou abstratamente, por uma ação & O colega jogou \emph{a bola} para a menina. O pai deu \emph{uma viagem} para a filha. \\
    Experienciador & Ser animado que está ou passa a estar em determinado estado mental, perceptual ou psicológico & O \emph{namorado} pensou na amada. O \emph{colecionador} viu um pássaro diferente. As provas preocupam \emph{a Maria}. \\
    Resultativo & Resultado de uma ação, ou seja, alguma entidade que não existia e passa a existir ou vice-versa & O pedreiro construiu \emph{a casa}. A bruxa comeu \emph{a maçã}. \\
    Beneficiário & Ser animado que é beneficiado ou prejudicado no evento descrito & O patrão pagou \emph{o funcionário}. \emph{A mulher} perdeu a carteira. A bibliotecária emprestou o livro para \emph{o aluno}. \\
    Objeto estativo & Entidade ou situação à qual se faz referência, sem que esta desencadeie uma ação ou seja afetada por uma ação & O aluno leu \emph{um livro do Chomsky}. O marido ama \emph{a mulher}. \\
    Locativo & Lugar de onde algo se desloca, para onde algo se desloca ou em que algo está situado ou acontece & A modelo voltou de \emph{Paris}. A Sara jogou a bola para \emph{o alto}. Eu moro em \emph{Belo Horizonte}. O show aconteceu no \emph{teatro}. \\
    Instrumento & Instrumento usado por um agente no desencadeamento da ação & \emph{Uma tesoura sem ponta} cortou as gravuras. \\
    \bottomrule
  \end{tabular}
  \source{\textcite{cançado2017}.}
  \end{threeparttable}
\end{table}

A ordem ocupada pelos predicados nas sentenças está contida nas
informações semânticas e sintáticas dos itens lexicais verbais, dando
origem a outro conceito importante à noção de papéis temáticos, a saber,
a \emph{estrutura argumental}, como exemplificado em \ref{itm2}.

\begin{enumerate}[start=2,label={(\arabic{enumi})}]
  \item\label{itm2}
    \begin{enumerate}[label=(\arabic{enumi}.\alph*)]
      \item\label{itm2a} João\textsubscript{{[}Agente{]}} correu
      \item\label{itm2b} João\textsubscript{{[}Causa{]}} quebrou o vaso\textsubscript{{[}Paciente{]}}
      \item\label{itm2c} João\textsubscript{{[}Agente{]}} colocou seus pertences\textsubscript{{[}Tema{]}} na estante\textsubscript{{[}Locativo{]}}
    \end{enumerate}
\end{enumerate}

A estrutura argumental não aponta necessariamente para a ordem com que
os predicados ocorrem nas sentenças, nem para as suas possibilidades de
flexão morfológica; a estrutura destaca apenas a exigência de cada um
dos argumentos. É importante destacar que, no escopo deste trabalho,
essa estrutura, apesar de não incluir informações sintáticas, é o que
determina a organização dos itens lexicais nas construções linguísticas,
incluindo as possibilidades e restrições de alternâncias entre eles.
Nesse caso, há uma relação bastante próxima entre estrutura argumental e
sintaxe, como destacado por \textcite{camacho1999}, já que a ordem dos elementos
na sentença contribui para a construção do sentido. A estrutura
argumental de \ref{itm2a}, por exemplo, pode ser usada tanto para o verbo
\emph{correr} no pretérito perfeito, como está, ou no imperfeito. O
mesmo ocorre em \ref{itm2b}: tanto na voz ativa quanto na voz passiva a
estrutura argumental será idêntica.

Há uma série de limitações que podem dificultar que papéis temáticos
sejam reconhecidos ou interpretados por humanos, bem como por sistemas
computacionais, como a interpretação de \emph{Tema} e \emph{Objeto
estativo}, conforme ilustrado no Quadro 1. Nesses casos, humanos têm à
disposição uma série de mecanismos (cognitivos e de conhecimento de
mundo, por exemplo) que podem auxiliar na interpretação das construções
linguísticas. Por conta desse aspecto e de o conhecimento nesse tipo de
construto teórico não estar explícito, impõem-se mais desafios de
implementar computacionalmente esse tipo de abordagem.

Ainda que os conceitos de papéis temáticos e semânticos se refiram à
interpretação e representação de sentidos, a noção de papéis temáticos
remonta à organização entre argumentos de um predicado na estrutura
sintática, atrelando-se, por vezes, a uma perspectiva gramatical, como
demonstrado em \ref{itm2}. Já a noção de papéis semânticos consiste normalmente
em uma abordagem mais ampla, calcada no significado e na interpretação
da sentença em relação ao evento descrito nela, não apenas na
organização gramatical.

\textcite{gildea2002}, em uma perspectiva computacional, propõem-se a
trabalhar com papéis semânticos. Os autores destacam que os conjuntos de
papéis temáticos mais abstratos foram propostos com o objetivo de
explicar generalizações entre predicados e argumentos, apontando para
uma teoria gramatical. Já os conjuntos de papéis propostos no âmbito do
PLN, conforme ainda insistem, são mais específicos, enfatizando as
realizações sintáticas entre predicados e argumentos.

Nesse sentido, a teoria dos papéis semânticos precisa estar formalizada
de maneira tal que a relação entre argumentos e predicados possa ser
explícita. Isso, a princípio, garante que os sistemas computacionais
obtenham maior acurácia (ou acerto) ao identificar e/ou rotular as
relações e papéis semânticos. Para tanto, o modelo \emph{Abstract
Meaning Representation} (AMR) \cite{banarescu2013,weischedel2013} tem ganhado destaque nos estudos e aplicações em PLN,
por ser um modelo que garante explicitude sobre a estrutura argumental,
permitindo explicabilidade não apenas para humanos, mas também para
sistemas computacionais, de como o sentido se estrutura em sentenças de
línguas naturais.

Tais aspectos são relevantes nas discussões atuais em PLN face aos
\emph{Large Language Models} (LLMs). Um dos desafios impostos por esses
modelos de língua é a dificuldade de se obter explicabilidade acerca de
inferências e correlações semânticas realizadas. Apesar do avanço
metodológico e técnico sobre a quantidade de dados que podem ser
processados em menor tempo, o aprendizado do ``sentido'' acontece
implicitamente, sem que um humano precise supervisionar o processo,
\emph{grosso modo}. A proposta da AMR é trabalhar com a estrutura
semântica de maneira explícita, permitindo maior entendimento não apenas
por humanos, mas também por sistemas computacionais.

Tendo a AMR como pressuposto teórico, objetiva-se neste trabalho
apresentar um algoritmo de classificação de papéis semânticos baseado em
Aprendizado de Máquina (AM). Para tanto, partiu-se de um \emph{corpus}
multigênero (literário, jornalístico, opinativo e científico) em PB, em
que os textos escritos já estavam pré-anotados com os papéis semânticos
do modelo AMR.

Com esse intuito, este artigo está organizado em cinco seções, além
desta Introdução. Na Seção 2, apresentamos com mais detalhes os
pressupostos teóricos da AMR, bem como os trabalhos relacionados à
tarefa de \emph{Semantic Role Labeling} (SRL). Na Seção 3, apresentamos
a metodologia empregada neste trabalho, bem como a descrição do
\emph{corpus} utilizado no desenvolvimento do algoritmo. Na Seção 4,
apresentamos o resultado obtido pelos algoritmos na tarefa de
classificação. Por fim, na Seção 5, tecemos algumas considerações
finais.