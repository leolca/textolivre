\section{Revisão bibliográica}\label{sec-revisãobibliográfica}

Para identificar de que forma o LD tem sido abordado na área de ensino
de LI, no âmbito nacional e internacional, e como os pesquisadores têm
definido este conceito, foi realizada uma pesquisa bibliográfica sobre
tema. Entende-se aqui que ``{[}\ldots{]} pesquisas bibliográficas
constituem o campo que revisam; não são exaustivas; são situadas,
parciais e feitas sob determinadas~perspectivas {[}\ldots{]}'' \cite[p.~52]{reis2008pesquisa}. Assim sendo, possibilitam uma compreensão ampla e situada dos
conhecimentos já produzidos acerca de uma questão conceitual em
específico sobre a qual se deseja aprofundar.

O procedimento inicial foi uma busca na base de dados \emph{Web of
Science}\footnote{\url{https://www.webofscience.com/wos/woscc/basic-search}},
escolhida por ser referência em disponibilização de materiais
científicos de alta qualidade e grande número de produções
científico-acadêmicas brasileiras e internacionais. A base de dados foi
acessada entre os dias 11 e 15 de julho de 2022 pelo Portal da
Coordenação de Aperfeiçoamento de Pessoal de Nível Superior (CAPES), com
o acesso restrito Cafe\footnote{Acesso Cafe (Comunidade Acadêmica
  Federada, um serviço da Rede Nacional de Ensino e Pesquisa - RNP).
  Este serviço permite o acesso remoto ao conteúdo assinado do Portal de
  Periódicos da CAPES disponível para a instituição.
  \url{https://www.periodicos.capes.gov.br/index.php/acesso-cafe.html}.} . Os
seguintes critérios foram usados para a pesquisa dos materiais
científicos: palavra-chave ``digital literacy''; ano das publicações de
2017 a 2021. Tendo em vista que a pesquisa foi realizada no ano de 2022,
a delimitação temporal do levantamento foi estabelecida com o objetivo
de garantir a relevância e a atualização das discussões sobre letramento
digital no ensino de língua inglesa. Ao concentrar a análise em um
período recente, buscamos identificar as tendências mais atuais, as
novas abordagens e os debates mais recentes sobre o tema; tipos de
documentos: artigos científicos e artigos de revisão; áreas de pesquisa
de educação: Estudos da Linguagem e linguística; idiomas: português e
inglês. Para refinar os resultados encontrados foi usado também o Índice
do \emph{Web of Science Emerging Sources Citation Index} (ESCI), o qual,
segundo seu site, abrange uma grande variedade de disciplinas e vão
desde publicações internacionais e de amplo escopo até aquelas que
fornecem informações regionais ou cobertura da área de
especialidade\footnote{\url{https://clarivate.com/webofsciencegroup/solutions/webofscience-esci/}}.
Desta forma, seguindo os critérios descritos, foram encontrados 165
artigos acadêmicos.

Posteriormente, foi realizada a leitura dos resumos destes materiais, e
foram filtradas para esta revisão apenas as produções científicas
consideradas como diretamente relacionadas com o ensino de LI e que
continham como palavra-chave o letramento digital, o que resultou em 14
artigos. Deste montante, quatro não estavam disponíveis na íntegra,
restando, então dez artigos que foram analisados de forma mais profunda,
conforme descrito a seguir.

\subsection{Pesquisas sobre letramento digital na área de língua inglesa}\label{sub-sec-pesquisassobreletramentodigitalnaáreadelínguainglesa}

Visando um conhecimento mais aprofundado sobre o LD na área de LI, foi
realizada uma análise primária, recuperando os seguintes aspectos:
autor(es), ano de publicação, objetivos da pesquisa, metodologia,
sujeitos da pesquisa, universidades vinculadas e local de origem, e
principais resultados/conclusões (\Cref{tab-01}).

\begin{small}
{
\setlength{\tabcolsep}{1pt}
\begin{longtable}{
>{\raggedright\arraybackslash}p{0.12\textwidth}
*{5}{>{\raggedright\arraybackslash}p{0.17\textwidth}}
}
\caption{Pesquisas sobre letramento digital no ensino de língua inglesa}
\label{tab-01}\\
\toprule
Autor(es) e Ano & Objetivos & Metodologia &
Sujeitos & Universidades vinculadas e Local de origem
& Principais resultados/conclusões
\\ 
\midrule
\endhead
\textcite{roche2017assessing} & É investigar o impacto da inclusão de um componente
explícito de letramento digital nos programas de preparação para o
ensino superior para estudantes internacionais de língua inglesa
adicional. & Um estudo de caso comparativo com 125 estudantes
internacionais de língua inglesa adicional de uma universidade
australiana. Os estudantes foram divididos em dois grupos: um grupo que
ingressou na universidade por meio de um programa EAP com foco explícito
em letramento digital e outro grupo que ingressou por meio de outra via
sem esse tipo de treinamento. & Estudantes internacionais de inglês como
idioma adicional & Universidade Southern Cross

Lismore -- Austrália & O estudo sugere que programas de preparação
acadêmica para estudantes internacionais de inglês devem incorporar uma
abordagem específica de letramento digital. A inclusão desse componente
resultou em benefícios significativos, como menor dificuldade nos
estudos subsequentes e uma compreensão aprimorada dos requisitos do
curso e das práticas de integridade acadêmica para os participantes. \\
\textcite{oliveira2018multiletramentos} & Identificar as estratégias de escrita em
inglês / LE (Língua Estrangeira) no ambiente digital usadas por alunos
de uma escola pública de ensino médio, na produção de uma
\emph{wikipage}. & Trabalho descritivo-interpretativo, dados foram
coletados a partir de questionário, procedimentos de observação e de
coleta bibliográfica. & Alunos do 2º ano do Ensino Médio de uma escola
pública estadual do município. & Universidade Regional do Cariri (URCA)

Universidade Federal da Bahia (UFBA)

Universidade do Estado do Rio Grande do Norte (UERN)

Maracanaú - Fortaleza Ceará. & Alunos do ensino médio têm familiaridade
com o ambiente digital e utilizam ferramentas digitais para acessar
informações, produzir textos e refletir sobre suas práticas de
letramento. \\
\textcite{xavier2019construcao} & Apresentar e discutir uma sequência
didática de ensino de língua inglesa por meio da análise, construção e
compartilhamento de memes online. & Os alunos assistiram a um vídeo
sobre memes em inglês, responderam a um questionário sobre memes e, por
fim, criaram seus próprios memes. & Alunos de diversos cursos técnicos
integrados ao ensino médio. & Instituto Federal de Minas Gerais - Ouro
Preto

Ouro Preto -Minas Gerais & Sequência didática baseada em memes foi uma
experiência positiva para alunos, mas revelou dificuldades no uso de
ferramentas digitais, língua inglesa e capacidade crítica. \\
\textcite{almusharraf2020postsecondary} & Investigar o uso de práticas de
letramento digital multimodal em salas de aula de inglês como língua
estrangeira (EFL) no Reino da Arábia Saudita, identificar os obstáculos
ao seu uso eficaz e propor estratégias para superá-los. & Design de
métodos mistos a partir do qual os dados da pesquisa e da entrevista do
grupo focal foram triangulados. & Professores de inglês como língua
estrangeira (ILE) em instituições de ensino superior & Universidade
Prince Sultan /

Universidade Estadual de Nova Iorque em Buffalo

Arabia Saudita & Pesquisa mostra que professores de IEL na Arábia
Saudita usam tecnologias e práticas de letramento digital multimodal,
mas há obstáculos que precisam ser superados, assim, treinamento e
suporte são necessários para aumentar o uso e a eficácia. \\
\textcite{cladis2020shifting} & Avaliar o impacto das tecnologias digitais no domínio da
expressão da linguagem, incluindo leitura, escrita e faculdades de
imaginação e pensamento crítico. & A pesquisa foi realizada por meio de
um questionário e observação de atividades desenvolvidas nas aulas de
uma escola em Elmhurst, Illinois, EUA. & Estudantes do ensino médio. &
Universidade Fairfield

Fairfield, Connecticut, EUA & A dependência digital tem alterado a forma
como lemos, escrevemos e processamos informações, com implicações
negativas para o pensamento crítico, a criatividade e a interação
social. \\
\textcite{campbell2020developing} & É defender uma abordagem integrada e situada ao
letramento digital para professores em formação de inglês que leve em
consideração as identidades. & É um estudo de caso qualitativo
longitudinal, que coleta dados de participantes usando uma variedade de
métodos, incluindo questionários, planos de aula, reflexões escritas e
entrevistas em grupo focal. Os dados são analisados
\hspace{0pt}\hspace{0pt}usando o software NVIVO para identificar
categorias e temas nas percepções e práticas dos participantes. &
Professores de inglês em formação. & Universidade da Cidade do Cabo

Cidade do Cabo -África do Sul & De acordo com o estudo, os professores
em formação têm percepções variadas sobre suas habilidades digitais e
recursos tecnológicos, criando barreiras no uso de tecnologias. O ensino
do letramento digital, combinado com a prática e a reflexão, pode ajudar
os professores em formação a ver o digital como um recurso para a
aprendizagem. \\
\textcite{bozavli2021foreign} & É determinar as experiências de aprendizagem dos
alunos de línguas estrangeiras que participam do ensino à distância
durante a pandemia e suas crenças sobre a possibilidade de aprender um
idioma estrangeiro sem escola. & O estudo utilizou uma abordagem
quantitativa, com um desenho descritivo. Os dados coletados foram
analisados por meio do programa SPSS, com análises estatísticas
descritivas e análise de conteúdo. & Alunos de línguas estrangeiras dos
departamentos de alemão, francês e inglês. & Universidade de Atatürk

Erzurum - Turquia & Os alunos de língua estrangeira preferem o ensino
presencial e acreditam que não é possível aprender um idioma estrangeiro
sem ir à escola. No entanto, o autor argumenta que a tecnologia pode ser
usada para criar ambientes de aprendizagem mais eficazes. \\
\textcite{dhillon2021investigation} & É explorar as visões e experiências de
professores de EAP (Inglês para Fins Acadêmicos) sobre seu
desenvolvimento de habilidades digitais, sua aplicação de tecnologia de
aprendizado eletrônico (e-learning) em seu ensino e suas percepções de
seu valor como ferramenta de aprendizagem. & O estudo utilizou uma
abordagem mista para coletar e analisar dados. Um questionário foi
enviado a 100 professores de EAP de instituições de ensino superior no
Reino Unido, e um grupo focal foi realizado com um subconjunto de 10
professores. Os dados do questionário foram analisados
\hspace{0pt}\hspace{0pt}estatisticamente, e os dados do grupo focal
foram analisados \hspace{0pt}\hspace{0pt}qualitativamente para obter
informações adicionais. & Professores de Inglês para Fins Acadêmicos &
Universidade de Warwick

Coventry - Reino Unido & Professores de EAP estão usando e-learning, mas
há uma necessidade de melhor treinamento e suporte para que eles possam
usar essas ferramentas de forma mais eficaz. As universidades têm um
papel importante a desempenhar em fornecer esse treinamento e suporte, e
os professores também precisam se comprometer em desenvolver suas
próprias habilidades. \\
\textcite{krajka2021} & É promover a atitude de
professor-como-pesquisador-de-linguagem nos programas de formação de
professores de línguas. & A metodologia do artigo é qualitativa, com
foco na análise de dados coletados por meio de questionários e diários.
Os dados coletados são usados para ilustrar a viabilidade de uma
abordagem de ensino de línguas que coloca os professores na posição de
pesquisadores. & Professores-alunos de um programa de pós-graduação em
TEFL (ensino de inglês como língua estrangeira) em uma universidade
privada de médio. & Universidade Maria Curie-Skłodowska

Lublin - Polônia & Os professores de línguas consideram as habilidades
de pesquisa auxiliadas pela linguística de corpus e pelo aprendizado
baseado em dados como uma parte útil do seu letramento digital. No
entanto, o maior obstáculo à aquisição dessas habilidades é a falta de
compreensão de como esse aprendizado pode ser útil para o ensino de
línguas. \\
\textcite{valadares2021videogames} & O objetivo do estudo foi verificar a
percepção dos estudantes sobre o uso de videogames como ferramenta de
aprendizagem de inglês. & Dados coletados por questionários, Action Logs
e narrativas escritas no início e final do semestre. Questionários
exploraram experiências prévias com videogames para aprendizagem de
inglês. Atividades baseadas em videogames e outros recursos foram
aplicadas ao longo do semestre. Action Logs e narrativas analisados no
contexto do marco de experiências de aprendizagem formal de línguas de
Miccoli, Bambirra e Vianini (2020). & Alunos universitários de turma do
Nível III de curso de extensão. & Universidade Federal de

Viçosa (UFV)

Viçosa - Minas Gerais & Videogames podem ser uma ferramenta eficaz para
a aprendizagem de línguas, mas é importante considerar as diferentes
realidades dos alunos e o papel do professor como mediador. Jogos em que
a participação ativa dos alunos é esperada são mais bem recebidos. \\
\bottomrule
\source{Elaborado pela autora (2023)}
\end{longtable}
}
\end{small}

Conforme os dados dispostos no \Cref{tab-01}, é possível observar que as
pesquisas sobre LD no ensino de LI foram realizadas em diferentes países
e com diferentes metodologias e objetivos, mas todos estão relacionados
à investigação da importância do LD para o ensino-aprendizagem da LI.
Algumas pesquisas se concentram em compreender as experiências dos
alunos com o uso de tecnologias digitais para aprender línguas
adicionais, por exemplo, o estudo de \textcite{roche2017assessing} que investiga o
impacto da inclusão do LD nos programas de preparação para o ensino
superior para estudantes internacionais de LI como LA. Outras pesquisas
se concentram em avaliar os benefícios do uso de tecnologias digitais
para a aprendizagem, como o estudo de \textcite{valadares2021videogames}, que
verifica a percepção dos estudantes sobre o uso de videogames como
ferramenta de aprendizagem de LI. Ainda, outras pesquisas se concentram
no desenvolvimento de práticas pedagógicas que integram o LD à
aprendizagem de LA por exemplo, o estudo de \textcite{campbell2020developing} que
defende uma abordagem integrada e situada ao LD para professores em
formação de LI. Observa-se uma crescente diversidade de tecnologias
empregadas no ensino de inglês, cada qual com suas especificidades e
potencialidades. Plataformas de gamificação, ambientes virtuais de
aprendizagem e redes sociais emergem como ferramentas chave para
promover a interação, a colaboração e a autonomia dos aprendizes. A
pesquisa de \textcite{valadares2021videogames}, por exemplo, evidencia o
potencial dos videogames para tornar o aprendizado mais engajador. Já os
estudos de \textcite{oliveira2018multiletramentos} e \textcite{xavier2019construcao}
demonstram como wikis e memes podem estimular a produção de conteúdo
autêntico e a reflexão crítica.

No que se refere às metodologias, observou-se que algumas pesquisas
possuem perspectiva quantitativa, como os estudos de \textcite{roche2017assessing} e
\textcite{bozavli2021foreign} que utilizaram apenas questionários como instrumentos de
coleta de dados, pois seus objetivos estavam direcionados a
conhecer/compreender as experiências dos alunos. Outras pesquisas
utilizam abordagem qualitativa, como é o caso \textcite{oliveira2018multiletramentos}, que utilizou entrevistas e observações para coletar dados sobre
as estratégias de escrita em LI no ambiente digital usadas por alunos de
uma escola pública de ensino médio, na produção de uma \emph{wikipage}.
\textcite{dhillon2021investigation}, \textcite{almusharraf2020postsecondary} e \textcite{campbell2020developing} usaram também o método de grupo focal como forma de coleta de
dados para suas pesquisas.

Outras pesquisas se baseiam em métodos mistos, ou seja, combinando o
quantitativo e o qualitativo, tal como o estudo de \textcite{valadares2021videogames} que usou questionários, \emph{Action Logs} e narrativas escritas
para conhecer a percepção dos estudantes sobre o uso de videogames como
ferramenta de aprendizagem de LI.

Dos dez estudos analisados nesse escopo, seis tiveram como sujeitos da
pesquisa os alunos. Destes, três foram realizados com estudantes
universitários \cite{roche2017assessing,bozavli2021foreign,valadares2021videogames} e
os outros três com estudantes do Ensino Médio \cite{oliveira2018multiletramentos,xavier2019construcao,cladis2020shifting}. Os demais estudos têm
como participantes professores, sendo dois voltados para docentes do
ensino superior \cite{dhillon2021investigation,almusharraf2020postsecondary},
enquanto os outros analisam professores em formação em programas de
pós-graduação \cite{krajka2021,campbell2020developing}. Alguns deles se
concentram em compreender as percepções dos professores sobre o uso de
tecnologias digitais para ensino, \textcite{dhillon2021investigation} investigam as
visões e experiências de professores com ensino de Inglês para fins
acadêmicos sobre seu desenvolvimento de habilidades digitais, sua
aplicação de tecnologia de aprendizado eletrônico \emph{(e-learning)} em
seu ensino e suas percepções de seu valor como ferramenta de
aprendizagem. Outros, se concentram no desenvolvimento de práticas
pedagógicas que integrem o LD ao ensino, como o estudo de \textcite{almusharraf2020postsecondary} que investiga o uso de práticas de letramento digital
multimodal em salas de aula de inglês como língua estrangeira (IEL) no
Reino da Arábia Saudita, identifica os obstáculos ao seu uso eficaz e
propõe estratégias para superá-los.

Com base nos resultados e conclusões apresentados pelos
estudos, podemos perceber que o LD é uma ferramenta valiosa
para a aprendizagem de LA, pois pode contribuir para o aumento da
motivação e da participação dos alunos, a melhoria na compreensão de
textos e na produção de gêneros textuais, o desenvolvimento de
habilidades de pensamento crítico entre outras. \textcite{cladis2020shifting} destacou
que a dependência digital tem alterado a forma como lemos, escrevemos e
processamos informações, com implicações negativas para o pensamento
crítico, a criatividade e a interação social. Já \textcite{campbell2020developing}
observaram que os professores em formação têm percepções variadas sobre
suas próprias habilidades digitais e sobre os recursos tecnológicos, o
que cria barreiras no uso de tecnologias. E, por fim, \textcite{dhillon2021investigation} constataram que professores de Inglês para Fins Acadêmicos (IFA)
estão utilizando \emph{e-learning}, mas há uma necessidade de melhor
treinamento e suporte para que eles possam usar essas ferramentas de
forma mais eficaz.

Outro ponto relevante diz respeito às limitações das pesquisas
analisadas. \textcite{oliveira2018multiletramentos}, destacam que, embora os
alunos demonstrem habilidades digitais, ainda há lacunas, como
dificuldades com vocabulário específico e o uso de tradutores online.
\textcite{valadares2021videogames} apontam barreiras sociais e o acesso restrito
às tecnologias, além da importância do papel do professor como mediador.
\textcite{campbell2020developing} mencionam a limitação das competências digitais
dos professores em formação, enquanto \textcite{xavier2019construcao}
sugerem mais reflexões sobre o uso de memes como ferramenta pedagógica.
\textcite{almusharraf2020postsecondary} destacam a limitação do tamanho da amostra
e possíveis vieses nos dados, e \textcite{dhillon2021investigation} indicam a falta
de treinamento contínuo e suporte técnico como obstáculos no uso de
\emph{e-learning}. \textcite{krajka2021} e \textcite{bozavli2021foreign} reconhecem limitações
na amostra e no método qualitativo, o que compromete a generalização dos
resultados.

No que se refere aos desafios tecnológicos, \textcite{roche2017assessing} alerta para a
dificuldade de estabelecer uma relação causal entre o programa de EAP e
o sucesso acadêmico. Essa limitação indica a necessidade de mais
pesquisas e a implementação de treinamentos e suportes mais consistentes
para maximizar o uso das tecnologias digitais no ensino.

Por fim, ficou evidente que, embora muitos alunos estejam familiarizados
com o ambiente digital e utilizem ferramentas digitais para a
aprendizagem, ainda há desafios significativos a serem enfrentados. É
preciso que os professores de LI estejam familiarizados com as
tecnologias digitais e com as práticas de LD para poderem integrar essas
ferramentas de maneira significativa aos contextos educacionais. Além
disso, é preciso enfrentar os obstáculos que surgem no uso das
tecnologias digitais, como a falta de formação e de suporte tanto para
os professores quanto para os alunos.

\subsection{Discussões sobre o conceito de letramento digital}\label{sub-sec-discussõesconceitoletramentodigital}

A discussão sobre o uso do termo "letramento" no singular ou no plural
faz parte da compreensão do conceito de LD. Ao tratar da temática alguns
autores como \textcite{dudeney2016letramentos} e \textcite{rojo2019letramentos},
defendem o uso do termo "letramentos digitais", no plural, para
enfatizar a diversidade de formas de letramento que existem no mundo
digital. Em contraste com essa perspectiva, \textcite{paiva2021letramento} argumenta que o
termo "letramento" é suficiente para abranger todas essas formas, sendo
que

\begin{quote}
{[}\ldots{]} letramento é um termo guarda-chuva e seria suficiente para
definir qualquer prática social de linguagem. Observe que uso linguagem
também no singular, pois também entendo que é um termo guarda-chuva e
que inclui tanto a linguagem verbal, a linguagem visual e a linguagem
multimodal \cite[p. 1165]{paiva2021letramento}.
\end{quote}

Concordamos com a perspectiva de \textcite[p. 18]{dudeney2016letramentos},
que argumentam que o letramento é mais do que um conjunto de habilidades
ou competências individuais. Essa perspectiva é importante para a
compreensão do conceito de letramento digital, pois enfatiza a
importância do contexto social e cultural na aquisição e uso do
letramento. Também concordamos que letramento é um termo guarda-chuva,
quando se pensa em práticas sociais de leitura e escrita de modo geral
dentro de um contexto sócio-histórico-cultural. Portanto, optamos por
usar o termo `letramento' no singular
na presente pesquisa, seguindo a perspectiva de \textcite{paiva2021letramento}.

Partindo do objetivo desta revisão, o qual centras-se na conceituação do termo
LD na área de ensino-aprendizagem de LI foram identificadas algumas
definições do termo nos artigos revisados, conforme \Cref{tab-02}.

\begin{small}
\begin{longtable}{p{0.18\textwidth} p{0.76\textwidth}}
\caption{Definições de letramento digital}
\label{tab-02}\\
\toprule
Autores & Definição de Letramento Digital \\
\midrule
\endhead

\textcite{oliveira2018multiletramentos} & Para além dos vários conceitos e
    perspectivas que o letramento digital, por sua parte, possui, aqui o
    compreendemos conforme Xavier (2011, p. 6) o define, para quem o
    letramento digital ``significa o domínio pelo indivíduo de funções e
    ações necessárias à utilização eficiente e rápida de equipamentos
    dotados de tecnologia digital''. Assim, trata-se de novas práticas
    lecto-escritas e interacionais efetuadas em ambiente digital com intenso
    uso de hipertextos \emph{on} e \emph{off-line} (XAVIER, 2009), bem como
    se caracteriza por uma intensa prática de comunicação por meio dos novos
    gêneros digitais mediados por aparelhos tecnológicos (XAVIER, 2011, p.
    6) \cite[p.253]{oliveira2018multiletramentos}. \\
    \textcite{valadares2021videogames} & Os multiletramentos, como sugerido pelo New
    London Group (1996), surgem como um elo para explorar a multiplicidade
    de conhecimentos existentes dentro e fora da sala de aula. A partir
    desse diálogo, Rojo (2013) explica que: {[}\ldots{]} isso implica negociar
    uma crescente variedade de linguagens e discursos: interagir com outras
    línguas e linguagens, interpretando ou traduzindo, usando interlínguas
    específicas de certos contextos, usando inglês como língua franca;
    criando sentido da multidão de dialetos, acentos, discursos, estilos e
    registros presentes na vida cotidiana, no mais pleno plurilinguismo
    Bakhtiniano (Rojo, 2013, p. 17) \cite[p.~213]{valadares2021videogames}. \\
    \textcite{xavier2019construcao} & (\ldots) Rojo (2012) expande a noção de
    letramento para o letramento digital: ``{[}\ldots{]} trabalhar com
    multiletramentos pode ou não envolver (normalmente envolverá) o uso de
    novas tecnologias da comunicação e de informação (`novos letramentos'),
    mas caracteriza-se como um trabalho que parte das culturas de referência
    do alunado (popular, local, de massa) e de gêneros, mídias e linguagens
    por eles conhecidos, para buscar um enfoque crítico, pluralista, ético e
    democrático - que envolva agência -- de textos/discursos que ampliem o
    repertório cultural, na direção de outros letramentos'' (p. 8) \cite[p.~146]{xavier2019construcao}. \\
    \textcite{campbell2020developing} & 
    A noção de letramento como prática social (Barton, Hamilton, \& Ivanič,
    2000), expande o conceito para incluir outros modos além da linguagem
    escrita que são ``múltiplos, multimodais e multifacetados'' (Alverman,
    Ruddell, \& Unrau, 2013, p. 1150). A definição de Martin (2008) de
    letramento digital captura a conexão inextricável entre o digital e o
    contexto no qual é usado.
    {[}Letramento digital é{]} a consciência, atitude e capacidade dos
    indivíduos de usar adequadamente ferramentas e recursos digitais para
    identificar, acessar, gerenciar, integrar, avaliar, analisar e
    sintetizar recursos digitais, construir novos conhecimentos, criar
    expressões de mídia e se comunicar com os outros, no contexto de
    situações especificas da vida, de modo a possibilitar uma ação social
    construtiva; e refletir sobre esse processo \cite[p.~167, tradução nossa]{campbell2020developing}\footnote{\textbf{Original:} The notion of
      literacy as social practice (Barton, Hamilton, \& Ivanič, 2000),
      expands the concept to include modes other than written language that
      are ``multiple, multimodal, and multifaceted'' (Alverman, Ruddell, \&
      Unrau, 2013, p. 1150). Martin's (2008) definition of digital literacy
      captures the inextricable connection between the digital and the
      context in which it is used. {[}Digital literacy is{]} the awareness,
      attitude and ability of individuals to appropriately use digital tools
      and facilities to identify, access, manage, integrate, evaluate,
      analyse and synthesize digital resources, construct new knowledge,
      create media expressions, and communicate with others, in the context
      of specific life situations, in order to enable constructive social
      action; and to reflect upon this process.}.\strut
     \\
    \textcite{almusharraf2020postsecondary} & O letramento digital é outro termo que se
    refere a um constituinte do letramento midiático em várias plataformas
    digitais; refere-se à capacidade do aluno de explorar, avaliar, criar e
    fornecer evidências claras de aprendizado por meio da escrita e outras
    formas de comunicação \cite[p.~86, tradução
    nossa]{almusharraf2020postsecondary}\footnote{\textbf{Original:} Digital literacy is another term that
      refers to a constituent of media literacy on various digital
      platforms; it refers to a learner's ability to explore, assess,
      create, and deliver clear evidence of learning through writing and
      other forms of communication (p.86).}. \\
    \textcite{krajka2021} & A construção do letramento digital dos professores
    precisa estar ancorada em teorias de aprendizagem estabelecidas, pois o
    letramento digital é um conjunto de habilidades, conhecimentos e
    atitudes que permitem o ensino de línguas dentro de uma estrutura
    educacional específica \cite[p.~587, tradução
    nossa]{krajka2021}\footnote{\textbf{Original:} Building teachers' digital literacy
      needs to be anchored in established learning theories as digital
      literacy is a set of skills, knowledge, and attitudes enabling
      language teaching within a particular educational framework (p. 587).}. \\
    \textcite{dhillon2021investigation} & (\ldots) letramento digital {[}3-5{]}, definida
    por Dudeney, Hockly e Pegrum {[}6{]} como ``as habilidades individuais e
    sociais necessárias para interpretar, gerenciar, compartilhar e criar
    efetivamente significado na crescente gama de canais de comunicação
    digital''. Isso, dizem eles, requer não apenas competência
    técnica, mas também a capacidade de usar a tecnologia ``efetivamente
    para localizar recursos, comunicar ideias e construir colaborações além
    das fronteiras pessoais, sociais, econômicas, políticas e culturais''
    \cite[p.~2, tradução nossa]{dhillon2021investigation}\footnote{\textbf{Original:}
      (..)digital literacy {[}3--5{]}, defined by Dudeney, Hockly and Pegrum
      {[}6{]} as ``the individual and social skills needed to effectively
      interpret, manage, share and create meaning in the growing range of
      digital communication channels''. This, they say, requires not
      only technical competence but also the ability to use technology
      ``effectively to locate resources, communicate ideas, and build
      collaborations across personal, social, economic, political and
      cultural boundaries.''.}. \\
    \textcite{roche2017assessing} & (\ldots) a definição de letramento digital utilizada no
    presente artigo é a capacidade de acessar, avaliar criticamente, usar e
    criar informações por meio de mídias digitais em engajamento com
    indivíduos e comunidades. Mover nossa compreensão do letramento digital
    para além das habilidades de TIC e da decodificação da informação em
    modos digitais para explorar as relações de autoridade entre autores e
    leitores baseia-se em conceitos desenvolvidos no campo do letramento
    crítico (ver Freebody \& Luke, 1990; Luke, 2014), enfatizando a natureza
    socialmente situada de todas as práticas de letramento \cite[p.~73, tradução nossa]{roche2017assessing}\footnote{\textbf{Original:} (\ldots) the definition of
      digital literacy used in the current paper is the ability to access,
      critically assess, use and create information through digital media in
      engagement with individuals and communities. Moving our understanding
      of digital literacy beyond ICT skills and the decoding of information
      in digital modes to explore relations of authority between authors and
      readers draws on concepts developed in the field of critical literacy
      (see Freebody \& Luke, 1990; Luke, 2014), emphasising the socially
      situated nature of all literacy practices (p. 73).}. \\
\bottomrule
\source{Elaborado pela autora (2023)}
\end{longtable}
\end{small}
Dentre os dez artigos pesquisados, foi verificado que seis deles
\cite{oliveira2018multiletramentos,campbell2020developing,almusharraf2020postsecondary,krajka2021,dhillon2021investigation,roche2017assessing} trazem
uma definição de LD, na qual estão baseadas suas pesquisas, e outros
dois \cite{valadares2021videogames,xavier2019construcao} apesar de
não trazerem uma definição para o termo, trazem a definição de
multiletramento, como pode-se verificar no \Cref{tab-02}.

Cabe ressaltar que o termo multiletramento, que tem sido muito utilizado
atualmente, está relacionado com os estudos do \textcite{newlondon1996pedagogy}. Segundo eles

\begin{quote}
  Uma pedagogia de multiletramentos, (\ldots) concentra-se em modos de
  representação muito mais amplos do que apenas a linguagem. Estes diferem
  de acordo com a cultura e o contexto, e têm efeitos cognitivos,
  culturais e sociais específicos \cite[p. 64, tradução nossa]{newlondon1996pedagogy}.
  \footnote{\textbf{Original:} ``A pedagogy of multiliteracies
  (\ldots) focuses on modes of representation much broader than language
  alone. These differ according to culture and context, and have
  specific cognitive, cultural, and social effects'' \cite[p. 64]{newlondon1996pedagogy}.}
  \end{quote}

Podemos considerar assim que este traz uma concepção amplificada dentro
da qual podemos situar também o LD.

Ao analisar as definições de LD utilizada nos seis estudos identificados
\cite{oliveira2018multiletramentos,campbell2020developing,almusharraf2020postsecondary,krajka2021,dhillon2021investigation,roche2017assessing}, tendo
em vista a convergência e/ou proximidade entre as escolhas lexicais
utilizadas, podemos considerar que o conceito pode ser entendido como a
capacidade, habilidade ou domínio que uma pessoa possui para usar,
criar, avaliar e comunicar-se usando recursos digitais.

\textcite{campbell2020developing}; \textcite{dhillon2021investigation}, e \textcite{oliveira2018multiletramentos} fazem referência a outros autores quando apresentam a definição
de LD. Já \textcite{roche2017assessing}, após citar a definição dada para o termo por
alguns autores, apresenta sua própria definição.\textcite{almusharraf2020postsecondary}, e \textcite{krajka2021} também elaboraram sua própria definição, sendo
que \textcite{krajka2021} é a única que faz relação específica em sua definição
com a construção do LD dos professores.

Percebemos que LD envolve não apenas competências técnicas, mas também
habilidades relacionadas ao uso da tecnologia de forma eficaz para
localizar recursos, comunicar ideias e construir colaborações entre
fronteiras pessoais, sociais, econômicas, políticas e culturais. Estas
habilidades são adquiridas por meio da prática e experiência com os
recursos digitais, o que significa que o LD não é algo imutável, mas sim
desenvolvido ao longo do tempo e que está sempre interligado com os
recursos usados em diferentes contextos sócio-históricos.

Assim, fica claro que, tanto a linguagem quanto a tecnologia modificam
os contextos sócio-históricos e as experiências humanas como também são
modificadas por esses contextos e experiências. Conforme Barton e Lee:

\begin{quote}
As práticas sociais em que a linguagem está inserida têm importância
particular quando se examina a linguagem online, especialmente por causa
das constantes mudanças, do aprendizado contínuo e da fluidez dos
textos. Uma parte crucial do contexto de textos online é situá-los nas
práticas de sua cria~e~utilização \cite[p. 24]{barton2015linguagem}.
\end{quote}

Diante da gama de possibilidades, para esta pesquisa foi observado o
letramento no ambiente digital, pois, como afirma Ribeiro ``O
pesquisador precisa, portanto, fazer um recorte dentro do tema mais
amplo dos letramentos e chegar a um ambiente que deseje observar
{[}\ldots{]}'' \cite[p. 26]{ribeiro2017}. Portanto, nessa pesquisa, o termo LD
é adotado no singular refere-se às práticas de leitura e escrita
desenvolvidas no ambiente digital, onde a tecnologia é um elemento
importante para o desenvolvimento de novas habilidades de leitura e
escrita, e está relacionado às habilidades sociais e cognitivas
necessárias para usar a tecnologia de maneira eficaz e para o exercício
da cidadania.

\subsection{O papel do professor de língua inglesa no letramento digital}\label{sub-sec-papelprofessorletramentodigital}

Outro assunto abordado por alguns dos autores ao discutirem o LD, foi o
termo nativo digital, trazido primeiramente por \textcite{prensky2001digital}. Segundo
o autor, ``{[}\ldots{]} nossos alunos hoje são todos `falantes nativos' da
linguagem digital de computadores, videogames e Internet {[}\ldots{]}''
\cite[p.~1, tradução nossa]{prensky2001digital}. Com
relação aos estudos analisados, \textcite{oliveira2018multiletramentos} definem
que são nativos digitais quem usa a tecnologia com maior apropriação por
terem nascido e crescido com acesso às mídias. Bem como, \textcite{cladis2020shifting}
cita Palfrey (2009) para explicar que os nativos digitais nunca
experienciaram o mundo sem as tecnologias digitais, e que os
``colonizadores digitais'' são aqueles que aos poucos se acostumaram com
as tecnologias.

\textcite{dhillon2021investigation} referem-se a \textcite{prensky2001digital} ao abordar os termos
nativos digitais e ``imigrante digital'', esse último seria aquele que
não cresceu com aparelhos digitais e pode não ser tão fluente quanto os
nativos digitais. Os referidos autores também comentam que Pegrum (2009)
e White (2013) questionam sobre o fato dessa distinção ser feita usando
a idade como principal fator para essa classificação. Na perspectiva de
\textcite{roche2017assessing}, os alunos universitários são normalmente considerados
nativos digitais, mas que isso não quer dizer que eles tenham um LD que
seja relevante para o meio universitário.

Compreende-se assim que o termo nativo digital trazido por \textcite{prensky2001digital}, tem sido discutido por diversos autores, sendo definido
amplamente como aqueles que nasceram e cresceram com acesso às
tecnologias digitais. Contudo, do ponto de vista de Barton e Lee

\begin{quote}
Essa divisão pode ter sido útil durante algum tempo por volta do ano
2000, quando havia uma geração de pessoas sem contato com a internet; e
durante certo tempo pesquisadores que haviam presenciado a mudança do
impresso para a tela estudaram os mais jovens, que tinham crescido com
as novas tecnologias \cite[p. 23]{barton2015linguagem}.
\end{quote}

Estudos mais recentes, como o de \textcite{alexandre2023revisitando},
destacam que, diante das transformações tecnológicas e sociais das
últimas décadas, essa categorização deve ser revista para atender à
complexidade das relações contemporâneas com a tecnologia. Não se
contrapõem aqui ao uso do termo nativos digitais em si, pois sabe-se
que, ultimamente, muitos dos jovens têm tido mais acesso as tecnologias
digitais, o que favorece o desenvolvimento de seu LD, contudo
concordamos com \textcite[p.~26]{dudeney2016letramentos} que ``{[}\ldots{]} a noção de uma
geração digitalmente competente homogênea é mito {[}\ldots{]}''. Os autores ainda complementam que

\begin{quote}
Primeiro, fatores óbvios como posição socioeconômica, nível de educação
e localização geográfica, assim como fatores menos óbvios como gênero,
raça e língua, têm grande impacto tanto no acesso quanto nas habilidades
digitais, criando uma crescente~muralha~digital. \cite[p. 26]{dudeney2016letramentos}
\end{quote}

Portanto, acreditamos que é importante levar em consideração a realidade
de cada indivíduo e as variáveis que influenciam no seu acesso e uso das
tecnologias digitais para o seu LD. Não se pode partir do pressuposto
que todos os jovens hoje são nativos digitais e possuem o mesmo nível de
LD, pois existem diversos fatores que afetam o acesso e o uso das
tecnologias digitais. Mesmo que na atualidade variados recursos digitais
estejam mais acessíveis disponíveis aos jovens\footnote{Não é nosso
  propósito aprofundarmos na discussão acerca da inclusão digital,
  porém, é necessário fazermos a ressalva de que ainda hoje não podemos
  afirmar, por exemplo, que todo e qualquer cidadão brasileiro esteja
  incluso. A Pesquisa TICS Domicílios --- realizada desde 2005 pelo
  Centro Regional de Estudos para o Desenvolvimento da Sociedade da
  Informação, com apoio do apoio do Ministério da Ciência, Tecnologia,
  Inovações e Comunicações (MCTIC), do Instituo Brasileiro de Geografia
  e Estatística (IBGE), do Instituto de Pesquisa Econômica Aplicada
  (Ipea) --- visa mapear no Brasil ``o acesso às TIC nos domicílios
  urbanos e rurais do país e as suas formas de uso por indivíduos de 10
  anos de idade ou mais''. Os dados publicados em 2024 indicam que, por
  exemplo, 89\% da população brasileira é usuária de internet, sendo que
  28\% dos domicílios não possuem conexão por banda larga. Informações
  disponíveis em \url{https://cetic.br/pt/} .}, isso não resulta,
necessariamente, em maior capacidade de encontrar, interpretar e avaliar
de forma crítica as informações, sendo comum que confiem na primeira
informação recebida, sem avaliar sua relevância ou confiabilidade
\cite{bueno2022reflexoes}.Um exemplo disso é que muitos jovens têm grande
familiaridade com o uso de celulares para acesso às redes sociais do
momento, com a linguagem e com ferramentas usadas nesses ambientes,
porém muitas vezes não sabem como usar programas de edição de áudio,
vídeo e imagens. Com frequência, observamos a existência de uma inclusão
digital focada em acesso às tecnologias digitais, porém sem trazer
práticas de letramento de maneira a tornar esse acesso mais eficaz. Uma
verdadeira inclusão digital precisa criar oportunidades para que se
adquira as habilidades e competências necessárias. Em relação ao enfoque
deste estudo, os professores de LI devem considerar em sua prática
pedagógica todos os fatores que implicam no LD dos alunos.

Ainda, alguns autores do escopo analisado debatem sobre o papel do
professor de LI em contextos que as tecnologias digitais são utilizadas
para o desenvolvimento da escrita e leitura. Nessa perspectiva,
\textcite{oliveira2018multiletramentos} relatam sobre o dever dos professores de
estimular diversos letramentos e identificar práticas para serem usadas
em sala, a qual conforme os autores é o lugar para o exercício da
cidadania. Ao tratar esse tema, \textcite{valadares2021videogames} destacam o
papel do professor como mediador na adesão das práticas digitais do
diálogo com os alunos para inclusão de todos nessas práticas. \textcite{cladis2020shifting}, por sua vez, pontua que os educadores devem permitir o uso de
plataformas digitais e o diálogo digital, e cita como ponto principal
para aprendizagem no meio digital a ênfase no pensamento e o aprendizado
baseado em investigação.

\section{Conclusão}

Este artigo teve por objetivo explorar o conceito de LD no contexto do ensino de LI, tanto no âmbito nacional quanto no internacional. Para isso, realizamos uma revisão bibliográfica que analisou artigos acadêmicos selecionados. Essa análise permitiu destacar percepções significativas sobre esse conceito, no ensino dessa língua, com a utilização de diferentes abordagens para investigar as experiências de ensino-aprendizagem de LI ligadas ao LD, considerando alunos e professores de variados contextos.

Com base nas leituras realizadas, conclui-se que o professor de LI, tendo em vista a relevância global que o idioma assume, tem um papel importante em ambientes digitais. Considera-se assim que a aprendizagem da LI deve ser tratada como parte das práticas sociais contemporâneas, considerando o uso das tecnologias e a produção de gêneros textuais típicos da esfera virtual.

Os estudos analisados indicam que docentes de LI precisam estar familiarizados com as tecnologias digitais e saber como utilizá-las pedagogicamente para promover o letramento digital dos alunos. Ao adotar recursos digitais em sala de aula, o docente não apenas impulsiona a aprendizagem de seus alunos, mas também se torna um aprendiz e um modelo para eles, evidenciando a importância das práticas sociais mediadas pelo digital. Nesse sentido, o professor deve ser um mediador no processo de aprendizagem, ajudando os alunos a desenvolverem as habilidades e competências necessárias para se comunicar e interagir no mundo digital.

O conceito de LD, conforme os estudos abordados, envolve não apenas competências técnicas, mas também habilidades relacionadas ao uso eficaz da tecnologia para localizar recursos, comunicar ideias e construir colaborações em diversos contextos. Essas habilidades são adquiridas por meio da prática e da experiência, o que implica que o LD não é estático, mas sim um conjunto de habilidades em constante evolução, influenciado pelos contextos sócio-históricos em que se insere.

Nesse contexto, torna-se essencial desenvolver práticas educacionais na LI que abordem aspectos de LD e de Letramento Crítico, além de permitir a utilização dessas tecnologias como meio para ampliar o domínio da língua. Assim, proporcionar a possibilidade do exercício da cidadania do indivíduo, propiciado a ele uma maior compreensão dos contextos do mundo globalizado, os quais tem a LI e a tecnologia como seus pilares fundamentais. Como afirma \textcite{warschauer2000changing}, os aprendizes de inglês não são apenas consumidores passivos da língua, mas agentes ativos que utilizam a LI para expressar suas ideias e influenciar o mundo ao seu redor. Ao desenvolver essas habilidades, os alunos estarão mais preparados para enfrentar os desafios do século XXI e exercer sua cidadania de forma plena.

Por fim, ao compreender a importância do LD no ensino de LI, estamos buscando formas de auxiliar os alunos a enfrentar os desafios e aproveitar as oportunidades que um mundo cada vez mais globalizado e conectado pode oferecer. Nesse cenário em constante transformação, percebemos que há um vasto campo de pesquisa a ser explorado. Algumas áreas promissoras para pesquisas futuras incluem a avaliação do impacto da inteligência artificial no ensino de LI, a investigação de estratégias eficazes para o desenvolvimento de competências críticas de LD e o estudo da integração das tecnologias digitais ao currículo escolar. À medida que novas tecnologias e abordagens emergem, a pesquisa e a prática contínua são essenciais para garantir que os alunos estejam adequadamente preparados para enfrentar os desafios e aproveitar as oportunidades do mundo digital em constante evolução. Dessa forma, proporcionaremos aos alunos a possibilidade de serem cidadãos informados e participantes ativos em uma sociedade cada vez mais digital e globalizada.
