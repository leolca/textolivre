\section{Introdução}\label{sec-intro}

O mundo contemporâneo é marcado por vários avanços tecnológicos, que
continuamente transformam diversos aspectos da vida das pessoas
acarretando, também, transformações na linguagem. \textcite[p.~13]{barton2015linguagem}
apontam que

\begin{quote}
A linguagem tem um papel fundamental nessas mudanças contemporâneas, que
são, antes de tudo, transformações de comunicação e de construção de
sentidos. A linguagem é essencial na determinação de mudanças na vida e
nas experiências que fazemos. Ao mesmo tempo, ela é afetada e
transformada por essas mudanças.
\end{quote}

As tecnologias digitais têm transformado a linguagem por meio da
internet, redes sociais, \emph{smartphones}, entre outras ferramentas.
Além disso, têm ampliado as possibilidades de leitura e escrita, ao
fazer com que a linguagem possa ser mais fluida, e, assim, possibilitam
o surgimento de novas formas de letramento.

Para \textcite{soares2006letramento} o letramento é ``{[}\ldots{]} o resultado da ação de
ensinar ou de aprender a ler e escrever: o estado ou a condição que
adquire um grupo social ou um indivíduo como consequência de ter-se
apropriado da escrita{[}\ldots{]}'' \cite[p. 18]{soares2006letramento}. E ainda sobre o
uso do termo ela declara que ``{[}\ldots{]} ao buscar uma palavra que
designasse aquilo que em inglês já se designava por \emph{literacy},
tenha-se optado por verter a palavra inglesa para o português, criando a
nova palavra \textbf{letramento}''\cite[p.~18, grifo do autor]{soares2006letramento}. Assim, o letramento, para Soares, é um processo
social que envolve o aprendizado e a prática da leitura e da escrita
para o exercício da cidadania. Sobre o conceito, afirmam \textcite[p.~33]{alves2023alfabetizacao}:

\begin{quote}
    A palavra ‘letramento’ tem sua origem em uma perspectiva social de leitura, conforme explica \textcite{soares2020A-alfabetizacao,soares2020C-alfabetar}. Nessa visão, tal termo implica saber fazer o uso efetivo de práticas de leitura e escrita para a inserção nas práticas sociais entre os indivíduos. Trata-se, nas palavras de \textcite[p.~79]{soares2020A-alfabetizacao}, do ‘uso da escrita como discurso, isto é, como atividade real de enunciação, necessária e adequada a certas situações de interação, e concretizada em uma unidade estruturada – o texto – que obedece a regras discursivas próprias (recursos de coesão, coerências, informatividade, entre outros)’.
\end{quote}

Ao situar as práticas de letramento nos contextos mediados pela
tecnologia, o letramento digital (LD) surge como uma concepção que
abarca demandas da sociedade contemporânea. Como definem \textcite[p.~17]{dudeney2016letramentos}
 os letramentos digitais são ``{[}\ldots{]}
habilidades individuais e sociais necessárias para interpretar,
administrar, compartilhar e criar sentido eficazmente no âmbito
crescente dos canais de comunicação~digital {[}\ldots{]}'', sendo assim,
percebe-se como é importante abordar o impacto das tecnologias digitais
no ensino de línguas. Tendo em vista nosso contexto de atuação
profissional, como professoras de língua inglesa, nosso foco recai sobre
o ensino-aprendizagem do idioma no contexto do LD. Por meio de uma
revisão bibliográfica, buscamos compreender como as tecnologias digitais
influenciam as práticas pedagógicas em um mundo cada vez mais
globalizado e conectado.

Nesse sentido, a aprendizagem de uma língua adicional (LA) não deve se
limitar apenas ao domínio gramatical e de vocabulário, ela também
precisa incluir competências e habilidades digitais, pois essas são
necessárias para um cidadão do mundo contemporâneo. Ferramentas como
plataformas de aprendizado gamificadas, ambientes virtuais de
aprendizagem e redes sociais ampliam as possibilidades pedagógicas ao
integrar competências digitais diretamente nas práticas de ensino. Além
de promover o domínio do idioma, essas tecnologias podem desenvolver
habilidades cruciais para o século XXI, como a colaboração, a comunicação
em ambientes digitais e o pensamento crítico sobre as informações
consumidas e compartilhadas online. Além disso, \textcite{warschauer2000changing} ressalta que a tecnologia da informação está revolucionando a
comunicação humana, assim como a escrita e a imprensa revolucionaram no
passado. Seu impacto sobre a forma como as pessoas interagem, acessam e
compartilham informações é profundo e ocorre rapidamente.

Nesse sentido, partimos do pressuposto de que o LD no ensino de LI vai
além do simples uso de recursos tecnológicos em sala de aula; ele
envolve o desenvolvimento de competências e habilidades de pesquisa,
análise, seleção e avaliação de informações online, além da comunicação
mais eficaz em ambientes digitais. Conforme \textcite{dudeney2016letramentos}, essas habilidades são cruciais para que os alunos possam
participar ativamente da sociedade contemporânea, navegar em um mercado
de trabalho cada vez mais digital e se conectar com pessoas de
diferentes culturas.

Dessa forma, este artigo tem como objetivo situar a conceituação do
termo LD, tendo em vista a relevância desta temática no contexto do
ensino de LI, e busca fornecer informações valiosas para professores e
pesquisadores que desejam abordar esse tema. Este estudo deriva da
pesquisa de mestrado desenvolvida no âmbito do Programa de Pós-Graduação
em Letras da Universidade Tecnológica Federal do Paraná (UTFPR), a
qual investigou as representações de docentes de língua inglesa no
Ensino Médio, de uma rede privada de escolas do Estado do Paraná, acerca
de suas habilidades de letramento digital. A pesquisa buscou identificar
o grau de letramento digital desses profissionais, com base em um
questionário adaptado da Matriz de Letramento Digital de \textcite{dias2009matriz}, bem como analisar as maiores necessidades formativas e
dificuldades encontradas por esses professores no uso de tecnologias
digitais no ensino de língua inglesa \cite{farias2024letramento}.

Assim, este artigo está organizado em seções que visam detalhar as
etapas e os fundamentos deste estudo. Após esta breve introdução,
discutimos sobre os procedimentos de filtragem e os artigos encontrados
na revisão bibliográfica sobre LD no ensino de LI, no âmbito nacional e
internacional. A seguir, discutimos sobre o conceito de LD, como o termo
é apresentado nos artigos revisados e sobre o papel do professor de LI
nesse contexto. Por fim, apresentamos as conclusões da revisão
bibliográfica e sugestões para pesquisas futuras sobre o tema.
