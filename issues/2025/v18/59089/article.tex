% !TEX TS-program = XeLaTeX
% use the following command:
% all document files must be coded in UTF-8
\documentclass[portuguese]{textolivre}
% build HTML with: make4ht -e build.lua -c textolivre.cfg -x -u article "fn-in,svg,pic-align"
\widowpenalty=10000
\clubpenalty=10000
\linespread{1.14}

\journalname{Texto Livre}
\thevolume{18}
%\thenumber{1} % old template
\theyear{2025}
\receiveddate{\DTMdisplaydate{2025}{5}{13}{-1}} % YYYY MM DD
\accepteddate{\DTMdisplaydate{2025}{6}{17}{-1}}
\publisheddate{\DTMdisplaydate{2025}{10}{7}{-1}}
\corrauthor{Clayton Oliveira}
\articledoi{10.1590/1983-3652.2025.59089}
%\articleid{NNNN} % if the article ID is not the last 5 numbers of its DOI, provide it using \articleid{} commmand 
% list of available sesscions in the journal: articles, dossier, reports, essays, reviews, interviews, editorial
\articlesessionname{articles}
\runningauthor{Oliveira} 
%\editorname{Leonardo Araújo} % old template
\sectioneditorname{Rodrigo Nascimento de Queiroz}
\layouteditorname{Saula Cecília}

\title{Autoetnografia: a trajetória de um professor de inglês em plataformas digitais de ensino}
\othertitle{Autoethnography: the journey of an English teacher in digital teaching platforms}
% if there is a third language title, add here:
%\othertitle{Artikelvorlage zur Einreichung beim Texto Livre Journal}

\author[1]{Clayton Oliveira~\orcid{0009-0000-4527-9146}\thanks{Email: \href{mailto:xclaytonx@gmail.com}{xclaytonx@gmail.com}}}
\affil[1]{Universidade Federal de Minas Gerais, Faculdade de Letras, Programa de Pós-Graduação em Estudos Linguísticos, Belo Horizonte, MG, Brasil.}


\addbibresource{article.bib}
% use biber instead of bibtex
% $ biber article

% used to create dummy text for the template file
\definecolor{dark-gray}{gray}{0.35} % color used to display dummy texts
\usepackage{lipsum}
\SetLipsumParListSurrounders{\colorlet{oldcolor}{.}\color{dark-gray}}{\color{oldcolor}}

% used here only to provide the XeLaTeX and BibTeX logos
\usepackage{hologo}

% if you use multirows in a table, include the multirow package
\usepackage{multirow}

% provides sidewaysfigure environment
\usepackage{rotating}

% CUSTOM EPIGRAPH - BEGIN 
%%% https://tex.stackexchange.com/questions/193178/specific-epigraph-style
\usepackage{epigraph}
\renewcommand\textflush{flushright}
\makeatletter
\newlength\epitextskip
\pretocmd{\@epitext}{\em}{}{}
\apptocmd{\@epitext}{\em}{}{}
\patchcmd{\epigraph}{\@epitext{#1}\\}{\@epitext{#1}\\[\epitextskip]}{}{}
\makeatother
\setlength\epigraphrule{0pt}
\setlength\epitextskip{0.5ex}
\setlength\epigraphwidth{.7\textwidth}
% CUSTOM EPIGRAPH - END

% to use IPA symbols in unicode add
%\usepackage{fontspec}
%\newfontfamily\ipafont{CMU Serif}
%\newcommand{\ipa}[1]{{\ipafont #1}}
% and in the text you may use the \ipa{...} command passing the symbols in unicode

% LANGUAGE - BEGIN
% ARABIC
% for languages that use special fonts, you must provide the typeface that will be used
% \setotherlanguage{arabic}
% \newfontfamily\arabicfont[Script=Arabic]{Amiri}
% \newfontfamily\arabicfontsf[Script=Arabic]{Amiri}
% \newfontfamily\arabicfonttt[Script=Arabic]{Amiri}
%
% in the article, to add arabic text use: \textlang{arabic}{ ... }
%
% RUSSIAN
% for russian text we also need to define fonts with support for Cyrillic script
% \usepackage{fontspec}
% \setotherlanguage{russian}
% \newfontfamily\cyrillicfont{Times New Roman}
% \newfontfamily\cyrillicfontsf{Times New Roman}[Script=Cyrillic]
% \newfontfamily\cyrillicfonttt{Times New Roman}[Script=Cyrillic]
%
% in the text use \begin{russian} ... \end{russian}
% LANGUAGE - END

% EMOJIS - BEGIN
% to use emoticons in your manuscript
% https://stackoverflow.com/questions/190145/how-to-insert-emoticons-in-latex/57076064
% using font Symbola, which has full support
% the font may be downloaded at:
% https://dn-works.com/ufas/
% add to preamble:
% \newfontfamily\Symbola{Symbola}
% in the text use:
% {\Symbola }
% EMOJIS - END

% LABEL REFERENCE TO DESCRIPTIVE LIST - BEGIN
% reference itens in a descriptive list using their labels instead of numbers
% insert the code below in the preambule:
%\makeatletter
%\let\orgdescriptionlabel\descriptionlabel
%\renewcommand*{\descriptionlabel}[1]{%
%  \let\orglabel\label
%  \let\label\@gobble
%  \phantomsection
%  \edef\@currentlabel{#1\unskip}%
%  \let\label\orglabel
%  \orgdescriptionlabel{#1}%
%}
%\makeatother
%
% in your document, use as illustraded here:
%\begin{description}
%  \item[first\label{itm1}] this is only an example;
%  % ...  add more items
%\end{description}
% LABEL REFERENCE TO DESCRIPTIVE LIST - END


% add line numbers for submission
%\usepackage{lineno}
%\linenumbers

\begin{document}
\maketitle

\begin{polyabstract}
\begin{abstract}
O artigo tem como objetivo analisar criticamente os impactos da plataformização do ensino sobre a trajetória de um professor de inglês não nativo, a partir da perspectiva metodológica da autoetnografia. O estudo emprega a escrita reflexiva para articular experiências pessoais com discussões teóricas sobre a precarização do trabalho docente, a lógica neoliberal no ensino de línguas e as hierarquias linguísticas em ambientes digitais. O autor descreve sua transição da área de engenharia para o ensino de inglês, destacando os desafios enfrentados em plataformas digitais, como a exigência de autopromoção constante, a competição desigual com professores nativos e a mercantilização das relações pedagógicas. Os resultados apontam que, apesar da aparente flexibilidade, o modelo das plataformas impõe mecanismos de controle algorítmico, desvalorização profissional e pressão por desempenho baseado em métricas de mercado. Conclui-se que, embora o sistema acentue desigualdades e fragilize vínculos educativos, existem estratégias possíveis de resistência, como a atuação em nichos especializados e a valorização da identidade docente não nativa. A pesquisa contribui para os debates sobre o futuro do trabalho docente em ambientes digitais e para a construção de práticas pedagógicas mais críticas e significativas.


\keywords{Ensino de línguas online\sep Precarização do trabalho\sep Plataformização\sep Autoetnografia\sep Neoliberalismo}
\end{abstract}

\begin{english}
\begin{abstract}
This article aims to critically analyze the impacts of teaching platformization on the trajectory of a non-native English teacher, using autoethnography as a methodological approach. The study adopts reflective writing to connect personal experiences with theoretical discussions on the precarization of teaching, the neoliberal logic in language education, and linguistic hierarchies within digital environments. The author describes his transition from engineering to English teaching, highlighting challenges faced on digital platforms, such as constant self-promotion demands, unequal competition with native teachers, and the commodification of pedagogical relationships. The results indicate that, despite the apparent flexibility, the platform model imposes algorithmic control mechanisms, professional devaluation, and pressure to perform according to market metrics. It is concluded that although the system reinforces inequalities and weakens educational bonds, there are possible resistance strategies, such as working in specialized niches and valuing non-native teacher identity. This research contributes to debates on the future of teaching work in digital environments and the construction of more critical and meaningful pedagogical practices.

\keywords{Online language teaching\sep Labor precarization\sep Platformization\sep Autoethnography\sep Neoliberalism}
\end{abstract}
\end{english}
% if there is another abstract, insert it here using the same scheme
\end{polyabstract}

\section{Introdução}
Este trabalho apresenta uma autoetnografia que explora minha trajetória como professor de inglês em plataformas digitais de ensino, situando-a no contexto mais amplo da precarização do trabalho docente e da uberização na era digital. A escolha por relatar essa experiência pessoal justifica-se pela necessidade de refletir criticamente sobre as dinâmicas impostas pelas plataformas \textit{online}, que, embora prometam democratizar o acesso ao ensino, reproduzem lógicas neoliberais de competição, flexibilização e controle \cite{abilio2021, antunes2020}. A autoetnografia, como método, permite articular a subjetividade da experiência individual com análises teóricas sobre as transformações no mundo do trabalho, especialmente no ensino de línguas \cite{adams2015}.

Minha trajetória começou em 2019, quando, ainda estudante de Engenharia em uma universidade pública brasileira, decidi abandonar a carreira tradicional para me dedicar ao ensino de inglês. A pandemia de Covid-19 acelerou minha migração para as plataformas digitais, expondo-me às contradições desse modelo: de um lado, a flexibilidade e o acesso a alunos globais; de outro, a precariedade e a intensa concorrência \cite{cavazzani2024, standing2014, gonsales2020}. A plataforma que utilizei operava sob uma lógica de mercado, onde professores eram classificados por algoritmos e disputavam visibilidade em um sistema que privilegiava quem podia pagar por planos \textit{premium} ou aceitar tarifas mais baixas \cite{block-gray2012}.

Além disso, enfrentei dilemas identitários, como a desvalorização de professores não nativos de inglês e a pressão para me tornar um tipo de ``influenciador digital", distanciando-me da essência pedagógica. Essas experiências ilustram como as plataformas reconfiguram não apenas as relações de trabalho, mas também as identidades profissionais \cite{moura2021}.

Neste relato, articulo minha vivência com teorias críticas sobre a uberização \cite{antunes2020, abilio2021, braga2017rebeldia}, a colonialidade do saber \cite{lander2005} e a mercantilização da educação \cite{block-gray2012}. O objetivo deste estudo é contribuir para o debate acadêmico sobre os impactos das plataformas digitais no ensino de línguas, destacando resistências e alternativas, como a atuação em nichos específicos, que me permitiram escapar parcialmente da lógica predatória desses sistemas. A reflexão aqui apresentada busca, portanto, conectar a micro-história de um professor às macroestruturas do capitalismo contemporâneo, seguindo a tradição crítica de autores como \textcite{harvey2014neoliberalismo} e \textcite{antunes2020}.

\section{Autoetnografia}
Esta pesquisa adota a autoetnografia como método, seguindo os princípios discutidos por \textcite{adams2015}, que entendem a prática autoetnográfica como um processo de escrita reflexiva que conecta o pessoal ao cultural, transformando a experiência individual em ferramenta de análise crítica. Minha trajetória como professor de inglês em plataformas digitais serve não apenas como objeto de estudo, mas como lente através da qual examino as estruturas mais amplas da precarização do trabalho, da plataformização da educação e das hierarquias linguísticas colonizadas. A autoetnografia, neste contexto, permite articular a subjetividade da vivência com as teorias críticas que fundamentam a discussão, criando um diálogo entre a micro-história e os macrofenômenos sociais.

Os dados que compõem esta análise emergem de uma combinação de memória reflexiva, registros pessoais e interações documentadas durante meu período de atuação nas plataformas. Utilizei como material empírico meu próprio relato feito sobre desafios enfrentados, estratégias de adaptação e percepções dos alunos, além de métricas de desempenho da plataforma e avaliações recebidas que foram feitas por ex-alunos. Esses fragmentos foram organizados e revisitados criticamente, buscando identificar padrões que revelassem não apenas minhas próprias contradições, mas também as dinâmicas sistêmicas das plataformas. A escolha por focar em episódios específicos -- como a migração para nichos de ensino ou o confronto com a hierarquia nativo/não nativo -- segue a proposta de \textcite{adams2015} de selecionar ``momentos de virada" que encapsulam tensões culturais mais amplas.

A escrita autoetnográfica aqui desenvolvida alterna entre descrição densa e análise teórica, um movimento que os autores citados defendem como essencial para evitar tanto o foco desproporcional em minhas experiências subjetivas, quanto a generalização abstrata. Por exemplo, ao narrar a pressão para me tornar um influenciador digital, não apenas descrevo as exigências da plataforma, mas as relaciono com os conceitos de uberização do trabalho \cite{abilio2020}, precariado enquanto classe trabalhadora \cite{standing2014} e a mercantilização do trabalho \cite{harvey2014neoliberalismo}. Essa abordagem permite que a experiência pessoal seja interrogada como caso emblemático, e não como mera ilustração.

O precariado, conforme \textcite{standing2014}, representa uma nova classe trabalhadora. Diferente do proletariado, essa nova classe trabalhadora é marcada pela instabilidade, ausência de direitos e pela fragmentação das relações laborais, condição que se aplica aos professores das plataformas digitais, eles são privados de vínculos formais e submetidos a uma lógica de trabalho intermitente. Já a uberização do trabalho \cite{abilio2020} refere-se à transformação do trabalhador em um ``microempreendedor de si mesmo", submetido a condições laborais flexíveis, sem garantias trabalhistas e dependente de algoritmos que controlam sua produtividade e remuneração. No contexto do ensino em plataformas digitais, essa dinâmica se manifesta na exigência de autogerenciamento constante, na competição por visibilidade e na conversão do ato educativo em uma mercadoria regulada pelo mercado. Esses conceitos auxiliam a compreender como a plataformização do ensino reproduz e intensifica formas contemporâneas de exploração, nas quais a autonomia aparente esconde a precarização estrutural.

A reflexividade é um pilar central dessa metodologia. Reconheço minha posição como professor não nativo, ex-estudante de engenharia e trabalhador inserido em um mercado globalizado, fatores que inevitavelmente moldam minha interpretação. Seguindo a orientação de \textcite{adams2015}, busquei equilibrar autoexposição e rigor analítico, evitando tanto a vitimização quanto a neutralidade impossível. A autoetnografia, nesse sentido, não é apenas um relato, mas um ato ético e político -- uma forma de denúncia das estruturas que naturalizam a precariedade, mas também de afirmação de alternativas construídas na prática cotidiana.

Após essa breve seção na qual eu descrevi a autoetnografia como base metodológica deste artigo e os conceitos teóricos principais que orientam minhas interpretações, se encontra a seção em que começo a expor minha narrativa pessoal. Na seção a seguir, faço um breve resumo de como me tornei um professor de inglês e os motivos que me levaram a entrar em uma plataforma de ensino digital enquanto elaboro uma relação entre minha experiência e os construtos teóricos abordados aqui.

\section{Da Engenharia ao ensino de Inglês}
Em 2019, eu era estudante de Engenharia em uma universidade pública brasileira, minha segunda graduação, depois de ter me formado em Relações Internacionais. O inglês sempre fez parte da minha vida -- estudei desde jovem, fiz intercâmbio e, após a primeira faculdade, obtive uma famosa certificação internacional de proficiência. Mesmo com um nível avançado, nunca havia pensado em ensinar até que, sem trabalho na época, um amigo sugeriu que eu desse aulas para nossos colegas de faculdade.

A princípio, a ideia não me animou. Mas, com o incentivo dele e da minha namorada, resolvi tentar. Elaborei um cartaz simples oferecendo aulas particulares e colei no mural da universidade. Para minha surpresa, em pouco tempo, vários alunos me procuraram. Comecei a usar salas vazias no campus da universidade para dar aulas individuais e em pequenos grupos. O número de interessados cresceu rapidamente, incluindo até pessoas de fora da faculdade. No fim daquele ano, eu já sabia que queria me dedicar ao ensino e me tornar professor de inglês; além disso, eu já havia perdido completamente o interesse em continuar na engenharia.

Em 2020, veio a pandemia. Com o fechamento da universidade, perdi quase todos os meus alunos presenciais. Buscando uma alternativa, foi quando descobri as plataformas \textit{online} de ensino de idiomas. Elas funcionavam como um mercado digital: professores criavam perfis com suas informações, preços e disponibilidade, enquanto os alunos escolhiam com base em avaliações, vídeos de apresentação e outros critérios.

Me adaptar a esse sistema não parecia difícil a princípio, mas logo percebi que a concorrência era grande, os algoritmos privilegiavam professores com mais aulas concluídas e que pagavam a mensalidade por planos \textit{premium} dentro da plataforma. No começo, meu perfil não aparecia nas buscas por professores feitas por alunos. Para melhorar minha posição na classificação dentro da plataforma, migrei meus poucos alunos remanescentes para a plataforma, usando seu sistema de agendamento para acumular pontuação. Cada aula concluída utilizando o sistema de agendamentos da plataforma rendia pontos, quanto mais pontos o professor acumulava melhor era sua classificação dentro da plataforma. Os professores pagantes de planos mensais recebiam pontos extras de bônus. Assinei o plano pago, o valor era equivalente ao que eu recebia mensalmente de um aluno em média. Ou seja, do que eu recebia de todos meus alunos, pelo menos um era para financiar a plataforma. Aos poucos, meu perfil ganhou visibilidade, novos alunos começaram a me encontrar e eu comecei a ter retorno do meu investimento na plataforma.

Dois fatores foram essenciais para meu crescimento: primeiro, descobri nichos específicos, como inglês para a indústria de petróleo e gás e preparação para entrevistas de emprego. Essas demandas tinham menos concorrência e alunos mais comprometidos. Segundo, aprendi a valorizar meu diferencial como professor brasileiro -- entendia as dificuldades específicas de quem fala português e podia explicar conceitos de forma mais acessível. Meu conhecimento acerca de aspectos desafiadores do inglês como falsos cognatos e ``vogais fantasmas" \textit{(ghost vowels)} ajudaram a conquistar a confiança dos alunos.

Hoje, não dependo mais exclusivamente das plataformas, mas a experiência me mostrou que é possível construir uma carreira no ensino de idiomas mesmo em um mercado competitivo. A chave foi encontrar meu espaço, seja pela especialização, seja pela autenticidade -- ensinando não como um ``falante nativo", mas como alguém que já passou pelo mesmo processo de aprendizagem e consegue guiar outros nesse caminho.

Minha transição de estudante de Engenharia para professor de inglês, iniciada em 2019, exemplifica os processos de precarização e reinvenção laboral analisados por \textcite{antunes2020} no contexto da uberização do trabalho. Como descreve \textcite{standing2014}, minha trajetória reflete a condição do precariado contemporâneo: profissionais exercendo atividades intermitentes sem vínculos e relações de trabalho formais e consequentemente sem as proteções e benefícios sociais.

A migração de profissionais para plataformas digitais durante a pandemia, materializam o conceito de trabalhador \textit{just-in-time} \cite{abilio2020}. Nesse modelo, típico da uberização, o profissional é acionado sob demanda, como um recurso flexível que assume riscos (instabilidade financeira, ausência de benefícios) enquanto a plataforma ou instituição externaliza custos. Minha casa se transformou em local de trabalho, sem delimitação de jornada, minha disponibilidade precisava ser constante para aulas e responder mensagens de possíveis interessados. Passava os dias aguardando receber um chamado. De acordo com as normas da plataforma, caso uma demanda para uma aula surgisse, o professor deveria responder em poucos minutos sob pena de ter o aluno-interessado encaminhado para outro professor que estivesse ativo no momento. Mas a remuneração das plataformas se restringe aos períodos efetivos de aula, durante as horas de inatividade que eu estava à disposição aguardando a próxima aula não havia remuneração. O professor nessa dinâmica tornava-se um estoque humano a ser explorado conforme a flutuação do mercado, sem qualquer garantia de renda ou proteção social.

Esta trajetória pessoal explicita as contradições do trabalho docente uberizado na era digital: de um lado, a autonomia aparente de ser ``seu próprio patrão"; de outro, a submissão a algoritmos e métricas que convertem relações educacionais em transações de mercado.

A seguir, na próxima sessão do artigo, me aprofundo mais no funcionamento das plataformas digitais de ensino. Baseado em minha experiência ensinando inglês \textit{online} revelo em detalhes como as plataformas operacionalizam a mercantilização do trabalho docente \cite{antunes2020, standing2014}.

\section{Funcionamento da plataforma de ensino \textit{online}}
A plataforma de ensino de línguas onde atuei operava como um mercado digital que conectava professores e alunos através de um sistema complexo de perfis, classificações e mecanismos de busca. O processo iniciava com o cadastro gratuito dos professores, que precisavam preencher informações pessoais, formação acadêmica, experiência profissional e definir seu valor por aula. Um diferencial importante era a inclusão de um vídeo de apresentação, onde os professores podiam demonstrar suas habilidades de comunicação e didática.

Para os alunos interessados em aprender um novo idioma, a plataforma oferecia uma interface de busca de professores que incluía diversos filtros, incluindo preço, disponibilidade de horários, avaliações de outros alunos e características dos professores (como ser falante nativo ou não). O sistema exibia os perfis dos professores em uma lista ordenada por pontos. Esses pontos não eram visíveis aos alunos e cada professor podia ver a sua pontuação, mas não a de outros professores. A pontuação era atribuída ao perfil do professor por um algoritmo que considerava múltiplos fatores. Quanto mais pontos o perfil do professor recebia do algoritmo, mais próximo do topo da classificação o perfil era mostrado, e maiores eram as chances de um aluno interessado encontrá-lo e contratá-lo. O fator determinante que mais rendia pontos ao professor era ter uma assinatura \textit{premium} na plataforma.

Aos professores, a plataforma oferecia três níveis de assinatura -- básico, intermediário e \textit{premium} -- que garantiam maior visibilidade e destaque ao perfil. Professores que pagavam os planos de assinatura da plataforma apareciam mais no topo das buscas e conseguiam mais indicações de alunos pela plataforma. Outros fatores geravam pontos, cada aula concluída gerava 1.000 pontos para o professor, responder dúvidas em fóruns ou publicar conteúdos no blog da plataforma rendiam pontos adicionais, embora em menor quantidade. Após cada aula, os estudantes atribuíam uma nota de 1 a 5 estrelas e podiam deixar comentários públicos. Professores bem avaliados ganhavam destaque dentro do mecanismo de busca do algoritmo da plataforma. Mas os pontos não se acumulavam de forma permanente, eles tinham um prazo de validade e ao final do prazo eles desapareciam, sendo assim os professores tinham que buscar acumular pontos dentro do sistema continuamente. Por exemplo, os 1.000 pontos recebidos por concluir aulas com sucesso na plataforma eram reduzidos para 100 pontos após 30 dias e depois de três meses esses pontos eram reduzidos para zero.

O processo de contratação era simples. O professor precisava apenas se cadastrar na plataforma, inserir suas informações pessoais e montar o seu perfil do modo que fosse mais atrativo e convincente possível. Após montar o perfil o professor recebia uma oferta para escolher um plano de assinatura da plataforma. Fazer a assinatura significava que o professor pagava mensalmente à plataforma para obter benefícios exclusivos. Entre os quais, o benefício mais importante era aumentar consideravelmente a pontuação do perfil do professor.

Os alunos se cadastravam na plataforma e a partir de suas informações eram indicados a alguns professores pelo algoritmo, em geral cinco opções, para que o aluno interessado escolhesse aquele perfil que gostasse mais. Os alunos interessados podiam avaliar qual o melhor professor dentre as opções disponíveis de acordo com seu perfil, horário disponível em seu calendário e uma conversa rápida através mensagens. Os alunos podiam interagir com os professores através de mensagens de texto e tirar dúvidas antes de tomar uma decisão final sobre qual dos professores seria escolhido. Os professores no entanto, tinham um número limitado de mensagens que podiam enviar aos alunos interessados, de acordo com o plano de sua assinatura. Caso o professor consumisse todas as respostas para se comunicar com os alunos interessados, havia a possibilidade de comprar um pacote de respostas na plataforma. Cada pacote comprado na loja da plataforma permitia ao professor enviar entre 10 e 50 mensagens de texto, a depender do valor do pacote. Os professores que não pagavam um plano de assinatura mensal da plataforma não tinham como interagir com os alunos interessados a não ser que comprassem um pacote de mensagens.

Os alunos podiam escolher entre os professores indicados pelo algoritmo da plataforma, qual era de sua preferência, ou, caso o aluno não tivesse gostado de nenhum dos professores, era possível pedir por novas indicações desde que pagando uma taxa para a plataforma. Para poder ignorar as recomendações do algoritmo e escolher um professor diretamente, geralmente aqueles mais bem classificados na plataforma, o aluno precisava pagar uma taxa mais substancial à plataforma para obter o contato direto do professor.

Uma vez que o aluno tivesse selecionado o professor desejado e feito o pagamento das aulas à plataforma, as aulas podiam finalmente começar. As aulas aconteciam no ambiente virtual integrado ao sistema ou em aplicativos de teleconferência externos, conforme combinado. A plataforma recomendava sempre que seu ambiente virtual de videochamadas próprio fosse utilizado. Esse ambiente gerava uma série de métricas ao final de cada aula, como pontualidade dos participantes, duração total da aula e até mesmo tempo de fala e número de palavras ditas. A recomendação era que a maior parte das palavras ditas durante a aula fossem dos alunos e não dos professores.

Os pagamentos pelas aulas eram sempre feitos pelos alunos à plataforma, atuando como intermediadora de pagamentos ela cobrava uma taxa dos alunos. A remuneração do professor ocorria após a conclusão com sucesso de algumas aulas. Após confirmada a realização das aulas sem incidentes, a plataforma fazia o repasse do pagamento ao professor, depositando o dinheiro em sua conta bancária. A plataforma cobrava taxas de ambos os professores e alunos: dos alunos, a plataforma cobrava um adicional sobre o valor da aula do professor, uma taxa de serviços; dos professores, a plataforma recebia o valor da assinatura mensal \textit{premium}. Assinar o pacote \textit{premium} era condição para atuar na plataforma, ainda que fosse opcional na teoria, na prática professores sem assinatura mensal não conseguiam novos alunos da plataforma. O pagamento aos professores era periódico, geralmente mensal ou quinzenal. Os professores evitavam receber os repasses com mais frequência pois a cada transferência bancária a plataforma descontava o valor do TED bancário. É de interesse vital para as plataformas caracterizar essas relações financeiras como uma prestação de serviço da plataforma ao trabalhador, e jamais como uma relação trabalhista onde o profissional está sendo remunerado por seu trabalho na empresa-plataforma, conforme explica \textcite{abilio2020}.

Um aspecto importante era a flexibilidade do modelo. Os alunos podiam comprar aulas avulsas ou pacotes com desconto, e não havia qualquer vínculo de longo prazo, podiam trocar de professor ou interromper as aulas a qualquer momento. Para os professores, isso significava uma constante necessidade de manter bons índices de avaliação e preços competitivos e se esforçar para reter alunos. Afinal os alunos podiam trocar de professor a qualquer momento dentro da plataforma, mesmo que fosse por simples curiosidade.

Minha experiência revelou que o sucesso na plataforma dependia de vários fatores: investir em um plano pago, manter alta disponibilidade de horários, aceitar valores mais baixos por aula (especialmente no início) e cultivar uma boa reputação através das avaliações. Professores novos enfrentavam o desafio de aparecer em posições inferiores nas buscas, enquanto os mais estabelecidos conseguiam cobrar valores mais altos e manter uma carteira estável de alunos.

A arquitetura da plataforma de ensino de línguas descrita acima materializa concretamente os processos de precarização e uberização do trabalho docente conforme descrito por \textcite{antunes2020} e \textcite{standing2014}. Os professores são trabalhadores sem nenhuma estabilidade ou garantias mínimas geridos por um algoritmo que utiliza de métricas nem sempre muito claras. Obter sua remuneração depende de atender os desígnios e exigências da plataforma enquanto tentam agradar ao algoritmo, deixando os professores totalmente à mercê da plataforma. Os professores da plataforma enquanto trabalhadores tem suas relações trabalhistas distorcidas, enquanto a empresa-plataforma opera como aparente prestadora de serviços. Apesar disso, é a plataforma que impõe regras e padrões que os trabalhadores devem seguir para garantir os padrões de qualidade que a plataforma vende ao alunos-consumidores. A empresa-plataforma explora a mão de obra dos professores para gerar seus lucros ao mesmo tempo que transfere os riscos e custos ao trabalhadores. Os professores assumem todos os riscos inerentes ao trabalho, caso o professor não possa trabalhar, seja por motivos de saúde ou outros alheios a sua vontade como uma falha na prestação de serviços públicos, ele assume integralmente a responsabilidade. Além disso, os professores são os responsáveis pelos custos de adquirir e fazer a manutenção de suas ferramentas e infraestrutura de trabalho -- computador, conexão com internet, eletricidade, e outros. A empresa-plataforma não assume nenhuma responsabilidade.

A mercantilização do trabalho \cite{harvey2013} docente fica evidente no sistema de perfis competitivos, onde professores precisam constantemente competir para garantir visibilidade. Isso exemplifica a transformação do trabalho intelectual em \textit{commodity}\footnote{\textit{Commodity} é um termo utilizado para designar bens primários ou matérias-primas comercializados no mercado, caracterizados por sua padronização e fungibilidade, ou seja, possuem qualidade uniforme e são intercambiáveis, independentemente do produtor. No contexto deste estudo, compreende-se \textit{commodity} não apenas como mercadoria padronizada e intercambiável, mas como processo de transformação de atividades humanas -- inclusive intelectuais e afetivas -- em bens sujeitos às leis de oferta e demanda, esvaziando sua dimensão social.}, tal como analisado por \textcite{harvey2013} em suas reflexões sobre o capitalismo flexível. Os trabalhadores são tratados como descartáveis e prontamente substituídos por outros. A plataforma oferece os professores aos alunos do mesmo modo como um \textit{marketplace} oferece produtos. A remuneração dos professores ocorre exclusivamente em relação ao tempo de aula efetivo e não leva em consideração o tempo gasto com a produção de conteúdo e manutenção do perfil da plataforma.

O mecanismo de avaliações públicas por estrelas constitui uma forma contemporânea de controle disciplinar de acordo com o exposto em \textcite{antunes2020}. Essas avaliações, aparentemente democráticas, na prática funcionam como dispositivos de normalização que pressionam os professores a adequarem suas práticas pedagógicas às expectativas de consumo dos alunos num processo que subordina a educação às leis de mercado. A comissão cobrada sobre cada aula concretiza a mais-valia extraída pelo capital plataformizado, que se apropria de parte do valor gerado pelo trabalho docente sem oferecer contrapartidas trabalhistas.

A falsa autonomia oferecida pelo sistema -- onde professores supostamente escolhem seus preços e horários -- mascara o processo de autoexploração, no qual o trabalhador internaliza a lógica competitiva como se fosse empreendedor de si mesmo \cite{gonsales2020}. A possibilidade de os alunos trocarem de professor a qualquer momento, vendida como ``liberdade de escolha", na verdade intensifica a precariedade, enfraquecendo o vínculo entre educador e educando, onde relações pedagógicas de longo prazo são substituídas por transações pontuais em um ambiente altamente competitivo.

Na próxima sessão deste artigo passo a descrever em mais detalhes a minha experiência como professor de inglês em uma plataforma digital de ensino. Descrevo os desafios que eu encontrei e as estratégias que desenvolvi para enfrentar essas dificuldades.

\section{Minha experiência com plataformas de ensino}
No início, enfrentei dificuldades para conseguir alunos. Como meu perfil estava no final das buscas, era pouco visível. Percebi que os professores poderiam seguir três caminhos para obter sucesso: cobrar preços muito baixos para atrair alunos iniciantes, investir em planos pagos para aparecer no topo da lista ou ter avaliações altas após meses de trabalho na plataforma. Decidi não competir apenas por preço e, em vez disso, busquei um nicho específico: inglês para profissionais, também conhecido como \textit{Business English}, ou inglês para negócios.

Essa especialização fez toda a diferença. Enquanto muitos professores ofereciam aulas genéricas de conversação ou gramática, eu passei a focar em vocabulário técnico, situações profissionais e preparação para entrevistas de emprego nesse setor. Como eu já tinha familiaridade com a área -- minha cidade tinha forte presença da indústria petrolífera --, conseguia oferecer um conteúdo mais direcionado. Os alunos que buscavam esse tipo de aula eram menos sensíveis ao preço e mais fiéis, pois não encontravam facilmente outros professores com o mesmo perfil.

Outro nicho que explorei foi a preparação para entrevistas de emprego em inglês. Muitos profissionais precisavam treinar respostas, melhorar sua pronúncia em contextos corporativos e aprender a lidar com perguntas técnicas. Como esses alunos tinham um objetivo claro e urgente, estavam dispostos a pagar mais por aulas focadas em suas necessidades. Esse trabalho é especialmente satisfatório para mim, ajudar alguém a conquistar uma vaga de emprego me faz sentir que posso ter uma influência positiva na vida de outros.

A competição, no entanto, não desapareceu. Mesmo com a especialização, precisei manter meu perfil atualizado, me esforçava para responder rapidamente a mensagens e garantir avaliações positivas. Eu sabia que sempre que um aluno interessado entrava em contato comigo, a plataforma também havia indicado pelo menos outros 4 professores para aquela pessoa, a pressão da competição era constante. A plataforma recompensava professores ativos e bem avaliados com maior visibilidade, então eu sempre incentivava meus alunos a deixarem um comentário de avaliação após as aulas, era como se tivesse me tornado um criador de conteúdo digital, constantemente compelido a solicitar que alunos curtissem e compartilhassem as postagens.

Com o tempo, consegui estabelecer uma base de alunos recorrentes e reduzir minha dependência da plataforma para captar novos estudantes. A estratégia de focar em nichos me permitiu escapar da disputa por preço baixo e criar um diferencial que valorizava minha experiência específica, não apenas minha fluência no idioma. Essa abordagem mostrou que, mesmo em um mercado saturado, ainda havia espaço para quem soubesse identificar demandas não atendidas e se posicionar de forma estratégica, ou era isso que eu pensava. Naquela época eu estava totalmente imbuído da lógica de pensamento propagado pelo mercado sem nenhum olhar crítico.

Outra dificuldade era a pressão para estar sempre \textit{online}. Além das aulas, era preciso responder rapidamente a mensagens, participar de fóruns e interagir com publicações da plataforma. Muitos professores faziam isso de forma estratégica, postando em horários de pico para aumentar seu alcance. Eu preferia focar no preparo das aulas, mas percebia que essa postura me deixava menos visível.

O maior dilema veio quando alunos e colegas sugeriram que eu criasse perfis em redes sociais dedicados ao ensino. A ideia era postar conteúdos gratuitos para atrair seguidores e convertê-los em alunos pagos. Alguns exemplos que vi incluíam: contas no Instagram com frases diárias em inglês, vídeos curtos no TikTok explicando gramática de forma humorística e \textit{lives} no YouTube simulando aulas gratuitas.

Tentei algumas vezes, mas não consegui manter a consistência. Criar conteúdo demandava tempo e energia que eu preferia dedicar ao ensino em si. Além disso, sentia que estava forçando uma persona que não era natural para mim. Eventualmente, aceitei que meu caminho não seria o do professor `celebridade digital", e sim o do profissional que conquista alunos pela qualidade do trabalho, não pelo marketing.

O sistema é construído para valorizar professores que sabem se vender: aqueles com perfis chamativos, vídeos bem produzidos e presença ativa nas redes sociais. Descobri que muitos colegas criam conteúdos diários em redes sociais: posts com dicas de inglês, \textit{reels} mostrando aulas divertidas, até mesmo desafios virais para atrair alunos. Descobri que alguns professores até investiam em equipamentos de gravação e serviços de edição para parecerem mais atraentes e profissionais.

A plataforma incentivava essa exposição também. Quem tinha um perfil mais dinâmico, com fotos profissionais e vídeos bem editados, ganhava mais destaque nas buscas. Eu, porém, sempre fui reservado. A ideia de me expor constantemente, gravar vídeos performáticos ou tentar ser ``influencer" sempre me causou desconforto. Eu tentava fazer o possível dentro das limitações da minha personalidade mais introvertida. No meu perfil dentro da plataforma meu vídeo de apresentação era simples, sem efeitos ou \textit{scripts} elaborados, mas isso claramente me colocava em desvantagem.

Enquanto professores mais ativos nas redes conseguiam alunos novos constantemente, meu crescimento era mais lento e dependente de indicações. Aprendi a compensar isso aprofundando ainda mais meu nicho -- alunos com necessidades específicas, como inglês técnico ou preparação para entrevistas, estavam menos preocupados com meu número de seguidores e mais interessados na minha expertise.

A exigência de autopromoção constante revelou-se um dos aspectos mais contraditórios do trabalho nas plataformas de ensino, expondo a transformação do professor em ``trabalhador-empresa" \cite{abilio2020}. O sistema não apenas media a interação entre professores e alunos, mas impunha uma lógica de visibilidade que convertia a identidade docente em capital simbólico negociável \cite{block-gray2012}. Enquanto a plataforma apresentava essa dinâmica como meritocrática, onde os ``melhores" perfis seriam naturalmente recompensados, na prática, ela reproduzia formas veladas de precarização, exigindo investimento contínuo em auto marketing sem garantias de retorno \cite{standing2014}.

Minha resistência em aderir plenamente a essa performatividade digital não era apenas uma questão de personalidade introvertida, mas uma recusa inconsciente a um processo que reduz o ato educativo a conteúdos de entretenimento digital. A pressão mercadológica para produzir conteúdo para as redes sociais (atividades completamente alheias ao planejamento pedagógico) exemplificam a imposição do pensamento neoliberal sobre o trabalho, onde horas não remuneradas de produção de conteúdo tornam-se condição para acessar o direito de ser visto (e contratado) na plataforma.

A sugestão recorrente de que eu deveria me tornar um ``influenciador educacional" reflete o que \textcite{block-gray2012} identificam como a internalização do \textit{ethos} neoliberal na educação linguística: o professor não é mais avaliado apenas por sua competência didática, mas por sua capacidade de entretenimento e engajamento algorítmico. Essa exigência cria uma divisão hierárquica entre docentes: de um lado, os professores-influenciadores que dominam as linguagens das redes sociais; de outro, profissionais como eu, cujo valor reside no ensino propriamente dito, mas que são sistematicamente preteridos pelos mecanismos de busca da plataforma \cite{cavazzani2024}.

Minha solução de focar em nichos especializados que valorizassem expertise sobre performance pode ser lida como uma microrresistência à espetacularização do ensino. Ao direcionar meu trabalho para demandas específicas (inglês técnico, preparação para entrevistas), consegui parcialmente deslocar o eixo da avaliação do quão divertido/visível eu era para o quão eficaz era meu ensino. Essa estratégia ecoa as observações de \textcite{mignolo2005} sobre a possibilidade de contra-condutas que, sem romper completamente com o sistema, criam fissuras por onde escorrem outras lógicas de valor.

A experiência evidencia o paradoxo central da docência plataformizada: quanto mais o ensino se aproxima da lógica do entretenimento digital, mais se afasta de sua função crítica e transformadora. Minha trajetória sugere que a resistência a essa dinâmica exige não apenas coragem para nadar contra a corrente, mas a construção cuidadosa de alternativas que, mesmo dentro do sistema, preservem a dignidade do trabalho educativo como ato de conhecimento, não de consumo.

Nesta seção eu analisei as principais dificuldades que enfrentei trabalhando na plataforma e as alternativas que encontrei para navegar por essas dificuldades. Falei também das minhas inseguranças e limitações. A próxima seção do artigo é dedicada a um fenômeno particularmente complexo. Nesta seção final do artigo, eu abordo a questão dos falantes nativos no ensino de inglês, partindo de uma perspectiva crítica e decolonial \cite{mignolo2005, lander2005} eu analiso como eu vivenciei pressões, inseguranças e incoerências, frutos de uma estrutura histórica colonial, e como eu pude ressignificar conceitos e crenças para me reposicionar como profissional e falante de língua inglesa.

\section{A hierarquia linguística colonizada das plataformas digitais}
Um dos maiores obstáculos que enfrentei na plataforma foi a constante comparação com professores falantes nativos de inglês. Havia uma percepção generalizada entre muitos alunos de que aprender com um nativo era automaticamente melhor, como se o simples fato de ter crescido falando inglês tornasse alguém um professor mais qualificado.

No início, isso me causou muita insegurança. Eu havia estudado inglês por anos, feito intercâmbio, obtido certificação de proficiência, mas, ainda assim, me questionava: ``Será que tenho o mesmo direito de ensinar essa língua?". Após decidir abandonar a engenharia para me dedicar completamente ao ensino, eu busquei me aperfeiçoar profissionalmente. Já em 2020, mesmo durante a pandemia, eu consegui cursar o CELTA\footnote{Certificate in Teaching English to Speakers of Other Languages.} e obtive com sucesso essa certificação da famosa Universidade de Cambridge. E não parei, continuei estudando, aprimorando meus conhecimentos e obtendo vários outros certificados TEFL\footnote{Teaching English as a Foreign Language.}. Acabei entrando na pós-graduação de ensino de inglês e depois no mestrado. Estudei muito, sempre me dedicando para me aperfeiçoar, ensino de pronúncia, de gramática, planejamento de aulas, metodologias de ensino, desenvolvimento de currículos, elaboração de materiais e mais, a lista é longa.

Eu sentia sempre essa necessidade constante de me colocar à prova, mas nada do que eu fazia me parecia ser bom o suficiente. Via professores nativos, às vezes até sem nenhuma experiência de ensino, sendo contratados com mais facilidade apenas por sua origem e por valores mais altos, às vezes o dobro, que professores não nativos com formação acadêmica e ampla experiência. Eu, por outro lado, não era nem nativo, nem formado em Letras, e nem muito experiente.

A virada veio quando tive uma oportunidade nova e comecei a ensinar português para estrangeiros. Durante uma das aulas de português, o aluno me fez uma pergunta simples sobre pronúncia que me deixou pensativo. Eu sabia como eu falava, mas percebi que existiam muitas variações válidas em outras regiões e localidades. E eu não poderia afirmar que uma variante era mais correta que a outra. Foi quando entendi que o inglês também não é uma língua única: tem diversos sotaques e diversas variações linguísticas, e todos são igualmente válidos.

Essa experiência me fez repensar minha abordagem. Em vez de tentar imitar um sotaque nativo (o que seria inautêntico), passei a valorizar meu próprio inglês, claro, correto, mas com minha identidade linguística.

Com o tempo, descobri que muitos alunos preferiam justamente professores não nativos. Alguns se sentiam intimidados por falantes nativos, outros valorizavam a didática sobre origem. Meu público-alvo era de profissionais que trabalhavam em grandes empresas multinacionais e precisavam usar inglês no trabalho, muitas vezes para se comunicar com estrangeiros não nativos que também tinham o inglês como língua adicional, não precisavam (nem deveriam) soar como nativos.

A plataforma, no entanto, ainda favorecia os nativos no algoritmo. Eles apareciam primeiro nas buscas, podiam cobrar mais e tinham um selo especial em seus perfis. Tive que trabalhar bastante para provar meu valor para os alunos interessados, mas também para mim mesmo. Eu estava sempre criando materiais específicos, obtendo avaliações positivas e, principalmente, tentando sempre mostrar resultados concretos para meus alunos.

Aprendi que ensinar uma língua que não é a minha língua materna não é uma deficiência, mas sim uma perspectiva única. O desafio foi transformar o que eu via como desvantagem em meu diferencial. Eu não ensinava inglês de nativo, mas sim inglês real para quem, como eu, precisava usá-lo como ferramenta de trabalho e comunicação global.

A experiência como professor não nativo de inglês na plataforma expôs as estruturas de poder linguístico que privilegiam os falantes nativos, um fenômeno que \textcite{pennycook1998} identifica como herança colonial na disseminação global do inglês. O algoritmo da plataforma reforçava essa hierarquia ao conceder selos especiais e posicionamento privilegiado aos perfis de professores nativos, independentemente de sua formação pedagógica.

Minha crise identitária inicial -- questionar se tinha legitimidade para ensinar inglês -- refletia a internalização de que saberes periféricos são sistematicamente desvalorizados. A plataforma, ao naturalizar a superioridade dos nativos (através de filtros de busca, diferenciação visual de perfis e disparidade de preços), reproduzia a lógica que estabelece centros e periferias de legitimidade cultural \cite{mignolo2005, lander2005}.

A minha mudança de mentalidade ocorreu quando tive a oportunidade de ensinar português para estrangeiros, compreendi a pluralidade intrínseca às línguas. Essa experiência me permitiu ressignificar minha posição: deixei de ver meu inglês como versão deficiente do original e passei a entendê-lo como legítima variante pós-colonial.

Na prática pedagógica, essa consciência se traduziu na valorização do inglês como língua franca\footnote{O conceito de inglês como língua franca (ELF) refere-se ao uso do inglês como meio de comunicação entre falantes de diferentes línguas maternas, sem a necessidade de aderência aos padrões nativos da língua.} \cite{jenkins2007}, focado na comunicação eficaz, não na imitação de padrões nativos. Meu diferencial foi justamente a capacidade de mediar entre a língua alvo e as dificuldades específicas dos aprendizes brasileiros.

Contudo, mesmo estratégias bem-sucedidas de resistência são limitadas pela estrutura do mercado linguístico global. A plataforma continua a beneficiar os nativos em seu design algorítmico, comprovando o imperialismo linguístico nas políticas de difusão do inglês. Minha experiência revela como as plataformas digitais, longe de serem espaços neutros, materializam e amplificam hierarquias linguísticas históricas. Ao mesmo tempo, demonstra a possibilidade de contra-discursos, onde professores periféricos podem apropriar-se criticamente do inglês, transformando sua posição aparentemente marginal em espaço de resistência pedagógica. Definitivamente, resistir a essas pressões não é fácil, a prática docente contemporânea está atravessada pelo discurso neoliberal que se alinha em harmonia com a colonialidade do saber. Essa mentalidade se encontra muitas vezes naturalizada e internalizada sem que se perceba -- certamente, era meu caso -- e muito esforço e energia precisam ser empregados para sua superação.

Nesta seção eu realizei uma análise crítica da figura do falante nativo dentro da plataforma de ensino de idiomas a partir da perspectiva da colonialidade do saber \cite{lander2005}. Eu descrevi como essa mentalidade colonizada tem efeitos concretos, como eu sofri com medos e inseguranças até conseguir ressignificar meu papel como professor de inglês. Finalmente, na próxima seção do artigo faço uma breve conclusão retomando os pontos principais abordados neste trabalho.

\section{Conclusão}
Minha trajetória como professor de inglês em plataformas digitais revelou um cenário complexo, onde a precarização do trabalho docente convive com oportunidades de reinvenção profissional. A experiência mostrou que o modelo das plataformas, embora prometa democratização do ensino, reproduz desigualdades ao priorizar lógicas de mercado sobre necessidades pedagógicas. Professores trabalhando para a empresa-plataforma precisam lidar com muitos desafios estruturais, desde a competição acirrada entre colegas para obter remuneração sem direitos trabalhistas, até a hierarquização baseada em critérios como falante nativo e pressões para se tornar um tipo de influenciador digital.

No entanto, o relato também demonstra que espaços de agência existem. Ao identificar nichos específicos (inglês técnico, preparação para entrevistas) e valorizar minha expertise como professor não-nativo, consegui escapar parcialmente da padronização imposta pelo sistema. Essa estratégia não resolveu todos os desafios estruturais, foi necessário um esforço grande e eu precisei ressignificar conceitos que estavam internalizados em mim. Contudo, acredito que foi possível construir uma prática docente mais significativa e alinhada com minhas habilidades.

O caminho adiante exige equilíbrio: se por um lado as plataformas oferecem acesso a alunos e flexibilidade, por outro demandam constante vigilância crítica para não reduzir o ensino a uma mercadoria. Minha experiência sugere que o futuro do trabalho docente nesses espaços dependerá da capacidade de negociar individualmente essas contradições, enquanto coletivamente buscamos formas de regulamentação que protejam a dignidade do ofício de ensinar.


\printbibliography\label{sec-bib}
% if the text is not in Portuguese, it might be necessary to use the code below instead to print the correct ABNT abbreviations [s.n.], [s.l.]
%\begin{portuguese}
%\printbibliography[title={Bibliography}]
%\end{portuguese}


%full list: conceptualization,datacuration,formalanalysis,funding,investigation,methodology,projadm,resources,software,supervision,validation,visualization,writing,review



\end{document}

