\section{Theoretical framework}\label{sec-theoreticalframework}
\subsection{Artificial Intelligence in Education}\label{sub-sec-artificialintelligenceineducation}

The integration of AI in education has garnered significant attention in
recent years, revolutionizing educational goals, practices, and the
learning environment. In this context, \textcite{roll2016} explore the
evolutionary and revolutionary aspects of AI in education, emphasizing
the transformative impact on educational objectives, classroom
practices, and the broader learning environment. Building upon this,
\textcite{chen2020} contribute by highlighting the positive
outcomes and potential of AI systems across administrative,
instructional, and learning domains, providing a comprehensive review of
AI applications in education. \posscite{roll2016} 
study underscores the shift in educational goals, moving away from rigid
knowledge preparation for the workforce towards equipping students as
adaptive experts and on-the-job learners. The ubiquity of smartphones
has transformed educational goals, emphasizing knowledge application,
collaboration, and self-regulated learning. This shift requires a
corresponding change in assessments from summative to ongoing formative
measures. In terms of practices, Roll and Wylie note the incorporation
of authentic elements in classrooms, resulting in increased complexity
and the challenge of personalization. The learning environment has
expanded beyond traditional classrooms to include informal and workplace
learning, transforming teachers from the "sage on the stage" to the
"guide on the side" \cite[p.~592]{roll2016}. Roll and Wylie also
acknowledge challenges posed to AI in education, raising questions about
effective technology support for teachers. \posscite{chen2020} study
complements this by providing a comprehensive review of AI applications
in education, emphasizing AI's positive impact on
administrative tasks, instructional quality, and learning experiences.
AI has the capability in assisting teachers in tasks such as exam
generation and grading. Chen et al. highlight the increasing integration
of AI into education, from early childhood to higher levels, showcasing
the adaptability of AI, including the use of robots (cobots) in teaching
routine tasks. It recognizes the growth in AI-related publications and
explores the diverse applications of AI, emphasizing its transformative
impact on various educational aspects. The positive outcomes of AI
deployment include improved learning quality, collaboration, global
access, enhanced academic integrity, and personalized learning plans.

On the other hand, the increasing integration of AI in education has
prompted extensive exploration into its challenges, implications, and
ethical considerations. \textcite[p.~4]{rodrigues2023} delve into
the philosophical, historical, and practical dimensions associated with
tools such as ChatGPT and AI in education, emphasizing the necessity for
a thoughtful and ethical approach to their integration within
educational frameworks. This discussion is crucial as AI technologies,
including natural language processing and datification, become more
pervasive in educational settings. At the same time, \textcite[p.~4222]{nguyen2023} explore the ethical principles governing AI in
education, focusing on K-12 settings, emphasizing the need for global
standards. The study critically evaluates existing ethical guidelines
and explores the applications of AI, such as personalized learning
systems, automated assessments, facial recognition, and predictive
analytics, in supporting both teachers and students. However, the study
acknowledges ethical challenges, ranging from systemic bias to privacy
concerns, which call for the urgent need to develop comprehensive
ethical guidelines in the field, emphasizing the need for global
standards and unified ethical principles for trustworthy AI. \textcite{rodrigues2023} emphasize the importance of an ethical and
considerate approach to AI integration, situating their study within the
broader context of AI technologies' increasing
influence, particularly in natural language processing. The authors
address concerns about AI's impact on professions,
particularly within education, questioning whether ChatGPT and AI pose a
threat or a challenge to the educational landscape. They highlight the
significance of datification in the Web 4.0 era, emphasizing ethical
considerations in the intertwining of human-machine interactions,
particularly in education. \textcite{nguyen2023} stress the urgent need
to educate teachers and students about ethical concerns and justify the
development of ethical guidelines in the field. Also, \textcite[p.~170]{regan2019} identify six privacy concerns for teachers and learners.
These are: information privacy, anonymity, surveillance, autonomy,
non-discrimination and ownership of information.

In summary, these studies highlight the transformative impact of AI
integration in education, emphasizing shifts towards adaptive learning
methodologies, personalized approaches, and the redefined role of
teachers. AI emerges as an evolving reality within the educational
realm, prompting considerations of ethical, social, and methodological
implications as discussed by the authors. Clear guidelines are deemed
necessary, particularly concerning assessment practices, privacy issues,
and technological infrastructure. Additionally, effective technology
support for teachers is identified as essential, especially regarding AI
tools such as cobots and chatbots.