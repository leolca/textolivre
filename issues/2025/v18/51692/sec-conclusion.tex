In this study, we aimed to explore the attitudes and perspectives of
prospective foreign language teachers in secondary education in Spain
regarding the application of AI in their teaching practices, with a
particular focus on the AI tool for language teachers: TWEE. Through
data analysis, we gained valuable insights into
participants' engagement with AI tools for academic and
non-academic purposes, as well as their perceptions of the strengths,
weaknesses, opportunities, and threats associated with the TWEE
application.

Our findings reveal a diverse spectrum of AI use among participants,
ranging from active integration into academic tasks to more reserved
attitudes or abstention from AI use altogether. Participants exhibit
variability in their choice of AI platforms, with preferences ranging
from well-established tools like Deepl and ChatGPT to lesser-known
options such as TWEE and Canva. The data also highlights an openness
among participants to explore new AI technologies, as indicated by their
intentions to adopt novel tools and experiment with additional
functionalities of existing platforms. Furthermore, participants
frequently integrate multiple AI tools into their workflow,
demonstrating a holistic approach to leveraging technology to enhance
productivity and outcomes.

Regarding the TWEE application, our SWOT analysis reveals a range of
internal strengths and weaknesses, including practicality, variety of
resources, interface issues, and language constraints. External
opportunities and threats resonate with the internal analysis,
emphasizing the potential for quick creation of activities and
innovative resources, as well as concerns related to ethical
considerations and the risk of teacher superfluity.
Participants' perceptions and attitudes towards AI are
generally favorable, with the majority expressing intentions to
incorporate AI tools into their future teaching practice. They believe
AI is well-suited for written tasks such as comprehension activities and
vocabulary instruction, while also recognizing its potential to enhance
student motivation and technological literacy.

Pedagogical implications for the secondary school classroom include the
need for educators to adopt a balanced approach to AI integration,
maximizing its benefits while mitigating potential risks and challenges.
Teachers should consider the diverse preferences and approaches of
students and be mindful of issues related to accessibility, interface
design, and ethical concerns. Additionally, there is a need for ongoing
professional development to support educators in effectively integrating
AI tools into their teaching practice.

Despite the valuable insights gained from this study, it is essential to
acknowledge its limitations. The sample size and demographics of
participants may not fully represent the broader population of
prospective foreign language teachers in secondary education. Moreover,
the study focused primarily on perceptions and attitudes, and further
research is needed to explore the actual implementation and impact of AI
tools in classroom settings.

In conclusion, this study contributes to our understanding of how
prospective language teachers engage with AI tools and their perceptions
of AI's role in future teaching practice. By addressing
the multifaceted nature of AI use and its implications for pedagogy,
educators can better navigate the opportunities and challenges
associated with integrating AI into language learning environments.