\section{Conclusiones y reflexiones finales}\label{sec-conclusionesyreflexionesfinales}

Como se expuso, la netnografía es un método que permite el desarrollo de
áreas y líneas de investigación, las cuales fomentan la constitución de
una netnografía lingüística. Si bien la etnografía lingüística se ha
conformado como un enfoque interpretativo que estudia las acciones
locales e inmediatas de los actores desde su punto de vista y que
considera que las interacciones se insertan en contextos y estructuras
sociales más amplios \cite{copland2015linguistic}, la netnografía
lingüística también es un enfoque que interpreta que tales acciones y
relaciones se dan en el ciberespacio, donde las instituciones y las
prácticas sociales en que el lenguaje se usa afectan la vida social
contemporánea, o donde hay prácticas digitales que no logran ser
prácticas sociales recurrentes porque las comunidades virtuales pueden
extinguirse, pero interesa ser estudiadas por los comportamientos e
interacciones sociales que se produjeron. A propósito, la lingüística
busca entender las prácticas sociales y comunicativas a través de
espacios \emph{online} mediante la netnografía, sin distanciarse del
medio \emph{offline}. En este sentido, debe entenderse que lo digital
está inmerso en la sociedad, la cual se ve transformada por las
tecnologías digitales, por lo que el análisis del mundo virtual no
termina en la investigación de la comunicación en línea, sino que a su
vez se nutre del análisis post-digital \cite{berry2015postdigital}. Así, es
importante también explorar las comunidades fuera de su espacio virtual,
tales como aquellas que convocan encuentros o reuniones presenciales
posteriores a la interacción digital. Dicho de otro modo, la
comunicación digital está entrelazada con los contextos \emph{offline},
pero también se encuentra anclada en las actividades fuera de línea de
los propios individuos. Es decir, con la netnografía lingüística se
puede realizar observaciones \emph{online} y \emph{offline}, como
entrevistas en persona con los usuarios. Además, métodos y técnicas
\emph{offline} junto con la netnografía lingüística pueden asegurar una
triangulación y rigurosidad en una investigación. Como se ha explicado,
la netnografía es un enfoque en la investigación social y cultural
establecido y su aplicación en el campo lingüístico es coherente y
razonable, puesto que la comunicación en línea involucra aspectos
lingüísticos esenciales.

Quien emplea la netnografía en el análisis del lenguaje es el netnógrafo
lingüístico. Como investigador, precisamente, el lingüista puede
entender que con la netnografía es posible obtener una comprensión
holística de las vidas experimentadas en el espacio virtual por una
comunidad o grupo de personas en particular acerca del uso del lenguaje,
sin aislar de su estudio las normas sociolingüísticas, los valores
lingüístico-culturales, la manifestación de identidades
intralingüísticas, las relaciones personales interlingüísticas y las
prácticas cotidianas e institucionales lingüísticas que los afectan. En
este sentido, los lingüistas pueden entender que la netnografía no es
una mera recopilación de conocimientos; por el contrario, se trata de un
trabajo de interpretación de fenómenos en que participa el lenguaje en
la vida social de los internautas. \textcite{Blommaert2010} sostuvieron
que en las etnografías el lenguaje se percibe como una herramienta
cargada y evaluada socialmente que posibilita que los seres humanos
actúen como seres sociales. Esto mismo sucede en el espacio virtual.
Dentro de este, se analizan los significados y las funciones de los
recursos lingüísticos dentro de sus contextos para conocer la vida, la
interacción, la identidad, el control, el poder, la vigilancia, el
ciberodio, las negociaciones, las representaciones, las aspiraciones y
los activismos de los interlocutores digitales.

En el artículo se ha visto que algunas áreas de investigación son la
netnopragmática, la netnosociolingüística y la netnolingüística
aplicada. En relación con ello, cabe indicar que al igual que etnografía
lingüística, la netnografía lingüística reclama la presencia de enfoques
interpretativos más amplios desde la sociología, la antropología, la
lingüística aplicada, más que concentrarse en las tradiciones
antropológicas para el estudio del lenguaje, como la etnografía de la
comunicación \cite{hymes1968ethnography} y la sociolingüística interaccional \cite{gumperz1972sociolinguistics}. Al respecto, \textcite{costello2017} consideran que la
netnografía es una oportunidad para las investigaciones, pues permite
revisar información y hasta participar en actividades en línea de forma
activa, que es de interés para la netnografía lingüística, la cual va
más allá de la aplicación de métodos etnográficos presenciales en el
estudio de las lenguas o actos de habla \cite{ramajo2011etnografico}. En términos de
temporalidad, las indagaciones lingüísticas pueden ser breves o implicar
años de investigación. En términos de cantidad, puede centrarse en una
comunidad o en múltiples comunidades. Y pueden combinar la netnografía
con otros métodos interdisciplinarios de investigación sobre comunidades
en línea.

Con este trabajo se intenta reflexionar sobre el método netnográfico que
los lingüistas pueden emplear en sus investigaciones. Para ello deben
desarrollar netnografías lingüísticas desde visiones amplias que
consideren que las interacciones lingüísticas en el ciberespacio
muestran los comportamientos de los internautas y que estas se insertan
en contextos y estructuras sociales que afectan la vida social
contemporánea. Con la netnografía tienen la posibilidad de interpretar y
documentar la interconexión y la complejidad de los valores y las
conductas sobre el lenguaje que tienen los cibernautas y las diversas
comunidades digitales. Para ello, la netnografía ha venido adaptando las
técnicas etnográficas tradicionales al estudio de la ``red'', no solo
para recoger, grabar y clasificar los datos, sino también para
analizarlos y representarlos \cite{addeo2020netnography}. Finalmente, los investigadores que usan el método
netnográfico tienen que considerar que las comunidades virtuales son
dinámicas y, por tanto, sus acercamientos con este deben en
consencuencia adaptarse, particularmente con los avances en inteligencia
artificial, que afectan caracterizaciones e interacciones en el mundo
digital.
