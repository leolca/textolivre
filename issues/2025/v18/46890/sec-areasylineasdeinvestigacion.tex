\section{Áreas y líneas de investigación}\label{sec-areasylineasdeinvestigacion}

Teniendo en cuenta la diversidad de posibilidades, las preocupaciones de
las netnografías lingüísticas pueden resumirse en netnopragmática,
netnosociolingüística y netnolingüística aplicada, las cuales mantienen
diálogo entre sí, pues lo social, lo cultural, lo cognitivo y lo
lingüístico están imbricados entre sí en el análisis de actores,
comunidades y acciones virtuales. Ha de precisarse que una netnografía
lingüística es una investigación sobre un tema lingüístico que emplea el
método netnográfico. En ese orden, la netnopragmática es la búsqueda de
fenómenos estudiados por la pragmática utilizando la netnografía; la
netnosociolingüística es la investigación acerca de aspectos en que la
lengua y la sociedad se unen manejando la metodología netnográfica; la
netnolingüística aplicada es la indagación en que interviene la
lingüística aplicada valiéndose de la netnografía. A continuación, se
muestran las tres áreas con líneas de investigación que pueden ayudar al
lingüista netnógrafo en la realización de sus pesquisas:

\subsection{Netnopragmática}\label{sub-sec-netnopragmática}

\textcite{putrikusama2016} sostiene que la netnopragmática es el estudio pragmático
sobre el uso del lenguaje por parte de los miembros de la comunidad
\emph{online} y propone la realización de estudios netnopragmáticos,
como el análisis de las implicaturas y las comunicaciones entre los
miembros del mismo grupo de Facebook. Para él, la netnopragmática
considera que los datos en línea son actos sociales, que el mundo en
línea funciona como un entorno social y que la búsqueda del significado
de los actos está en relación con el contexto que precede o sigue. Y
señala que la netnopragmática pueden estudiar datos visuales como el
diseño de páginas y mensajes, los emoticonos, los colores y tipo de
letra, las imágenes y fotografías. El mismo autor indica que en la
observación de los actos de habla de los miembros de una cultura en el
ciberespacio hay que tener en cuenta los cumplidos, las invitaciones,
las disculpas, las negativas y las ofertas, así como los lazos de
relación entre los miembros, como el grado de solidaridad, el poder
relativo y las imposiciones.

Al respecto, se puede investigar la cortesía en Facebook, por ejemplo,
en el mantenimiento de las relaciones laborales. Algunos estudios han
evaluado las estrategias lingüísticas que se usan para establecer
amistades, realizar posteos, replicar información, saludar por
cumpleaños, lo que finalmente se entiende como rituales en la web \cite{west2013}.

En esta línea, pueden realizarse trabajos sobre los usos de figuras
retóricas en la comunicación virtual, como la ironía, el humor o la
metáfora. \textcite{taiwo2016analyzing} han encontrado relaciones entre la
pragmática y la comunicación virtual. \textcite{coker2016} examinan la
retórica del humor visual en Facebook y descubren que este es
gubernativo, institucional, cultural y grotesco, y se emplea para
ridiculizar los problemas de la sociedad de forma abierta o encubierta.
Ellos recurren al trabajo netnográfico para interpretar que el humor
visual suele estar moldeado por las visiones del mundo, los prejuicios y
los intereses de sus usuarios. Ellos señalan que metodológicamente
fueron observadores y participantes muy activos en la publicación y
lectura de comentarios en línea por casi siete años. Con su trabajo los
autores concluyen que el humor no es un discurso neutral, sino más bien
legitima y forma una serie de valores.

Por otro lado, la netnopragmática puede servir para estudiar las
interacciones lingüísticas que se dan en torno al consumo de productos
en línea. \textcite{septianasari2021} utilizaron el método
netnográfico con un enfoque pragmático para recoger información sobre el
consumo de publicidad cosmética y analizar los actos de habla empleados
para persuadir por medio del marketing \emph{online} en Indonesia. Esto
nos recuerda que la netnografía se ha utilizado ampliamente en la
investigación de marketing. Para \textcite{casas-romeo2014}, la
netnografía es una herramienta que se aprovecha para recoger información
del mundo virtual para la toma de decisiones de marketing. Los estudios
netnográficos han cobrado importancia en el campo de la administración
en la búsqueda de los significados que pueden servir a las empresas para
lograr la comprensión, de lo que los clientes dicen por escrito en el
espacio virtual y fomentan el desarrollo de etnografías de la
comunicación \cite{freitas2012netnografia}.

\subsection{Netnosociolingüística}\label{sub-sec-netnosociolingüística}

La netnosociolingüística puede estudiar la construcción de identidades y
relaciones sociales en el ciberespacio. Le interesa las preocupaciones
de las identidades en curso. Se puede analizar la identidad digitalizada
de un grupo social que se negocia en línea a través de la actuación
textual y el metalenguaje de los participantes \cite{campbell2006}, pero
también las identidades que se cambian por seguridad en espacios
virtuales, especialmente en el campo de la política, la migración o la
religión, por ejemplo. Las actuaciones de identidad en los entornos en
línea no siempre están disociadas de las identidades fuera de línea,
pues tienden a ser continuas. Sin embargo, las identidades \emph{online}
no coinciden siempre con las \emph{offline}, pues en el mundo virtual se
performan otras identidades que el agente social configura, muchas veces
para protegerse, como sucede con las comunidades LGTBIQ+, que por
seguridad pueden recrear identidades \emph{online}, incluso en
aplicaciones de búsqueda de pareja. A los netnógrafos puede llamarle la
atención las aplicaciones como Grindr, Moovz, Growlr, Scissr, Scruff,
LGBTQutie, Chappy, Wapo y Wapa, Tinder. Los investigadores que pueden
integrar la relación entre las identidades \emph{online} y
\emph{offline} de los miembros lo logran al reunirse con los informantes
cara a cara.

Asimismo, desde la netnosociolingüística, se puede investigar los
lenguajes y discursos de ciberodio que se generan en la web, las
ideologías lingüísticas sobre ortografía de los internautas, las
ideologías que se encuentran tras la enseñanza de idiomas en el virtual,
las prácticas de vigilancia sobre la normativa lingüística y las
conductas sociales que mantienen diversos grupos con discursos
normativistas, así como el activismo lingüístico que busca deconstruir
prácticas de dominación social. Según \textcite[p. 57]{londoñopalacio2012}, ``Las redes
de comunicación tienen un papel definitivo en las narrativas digitales,
pues de manera explícita o implícita modifican entornos, hábitos y
prácticas sociales''. Dado que las literacidades muestran la relación
entre el discurso y el poder \cite{bloome2005discourse,jimenez2015}, a través del estudio netnográfico se puede analizar las
literacidades virtuales como prácticas sociales de relaciones de poder,
entre actores que se perciben con autoridad frente a los que no. La
netnosociolingüística debe ayudar a develar que los entornos
electrónicos no proporcionan neutralidad, sino más bien son espacios de
relaciones de poder en que se usa el lenguaje. Los análisis pueden estar
sesgados por las historias personales o por la ausencia de voces no
escuchadas \cite{noveli2010}. A este enfoque también le interesa las
literacidades no hegemónicas que muchas veces se ven afectadas por
prácticas de poder que evitan la aceptación y generación de
literacidades vernáculas. Adicionalmente, estudian las desigualdades
sociales y las posibilidades de transformación social asociadas con las
tecnologías \cite{Ardèvol_Lanzeni_2014}.

Además, el netnógrafo puede interesarse por los motivos en que los
internautas dan a sus interacciones sociales. \textcite{leppanen2009} estudian las particularidades de la acción lingüística, social y
cultural de un grupo de jóvenes finlandeses en los espacios translocales
de los nuevos medios de comunicación, y analizan la forma en que ellos
mismos dan sentido y razón a sus acciones. Los autores muestran que la
translocalidad se manifiesta, en particular, en la elección del idioma y
en la heteroglosia lingüística y estilística, la coexistencia, la mezcla
y la alternancia de diferentes lenguas, registros, géneros y estilos.
Los jóvenes seleccionan el inglés como lengua de comunicación en lugar
de su primera lengua, y emplean y combinan recursos de más de una
lengua, como el inglés con el japonés, e integran géneros y estilos de
una lengua en sus discursos.

También puede emplearse la netnografía para el análisis del discurso y
armado de corpus amplios como se ofrece con los datos en Twitter. Se
puede usar para analizar muestras de discusión pública generadas en
dicho espacio \cite{Bonilla_2022}. A propósito, debe señalarse que ``La
netnografía, como forma de investigación, hace un énfasis en la
recolección y el procesamiento de cantidades significativas de datos
empíricos para producir conocimiento al respecto'' \cite[p. 7]{Bonilla_2022}.

\subsection{Netnolingüística aplicada} \label{sub-sec-netnolingüísticaaplicada}

Desde la netnolingüística aplicada se pueden realizar trabajos sobre la
adquisición y enseñanza de lenguas nacionales y extranjeras, como
primeras y segundas lenguas, de canal oral o visual, como la lengua de
señas, de uno o grupo de usuarios en la red. \textcite{kulavuz-onal2015} señala
que con la netnografía se puede entender la cultura de las comunidades
de aprendizaje y enseñanza de idiomas en línea. El uso de las
tecnologías ha atraído la atención de muchos estudiosos en el campo del
aprendizaje de segundas lenguas, entonces a través de la observación
participante y las entrevistas en línea semiestructuradas con los
estudiantes se puede conocer sus experiencias, gustos, desventajas con
los sitios web de aprendizaje, y también las maneras en que integran la
tecnología a sus modos de estudiar. Investigar la enseñanza de idiomas
favorece a otros para que tomen decisiones de diseño de los aprendizajes
en el mundo digital. \textcite{kulavuz-onal2013} ha ido investigando los
aprendizajes de los profesores de inglés para poder enseñar con la
tecnología participativa en una comunidad de práctica en línea. Al
respecto, \textcite{kesller2021} sostienen que las netnografías se
pueden aplicar en el estudio de la adquisición de segundas lenguas; por
ejemplo, se puede analizar los hábitos formales e informales de
aprendizaje de idiomas que tienen los usuarios en chat o foros. Entre
diversas investigaciones, ellos señalan que el estudio de \textcite{isbell2018online}
ofrece una mirada hacia los hábitos informales de aprendizaje de
personas que aprenden coreano a través de chats, los que permiten
evidenciar comportamientos similares a las prácticas tradicionales del
aula al ver que los aprendices se centran en aprender la parte formal o
estructural de la lengua.

Por otro lado, \textcite{kulavuzonal2018} han sugerido analizar por
medio de netnografías las interacciones virtuales translingüísticas que
ocurren en la comunidad global de educadores de inglés como lengua
extranjera, por ejemplo. Así, observa que los participantes recurren a
sus repertorios multilingües dentro de Facebook, y que a través de
entrevistas etnográficas con los profesores y documentos en línea de su
telecolaboración, descubre que si bien el grupo se construyó
discursivamente como una zona exclusiva para emplear el inglés con el
fin de que los profesores orienten y fomenten el uso del inglés en sus
alumnos, todos los participantes rompen las instrucciones del espacio,
pues recurren a otras lenguas como el español y el árabe para diversos
propósitos, sobre todo los maestros, como en los casos de
establecimiento de solidaridad, selección de un destinatario, y el
modelado de la sensibilidad intercultural.

También, se puede estudiar las plataformas que facilitan procesos de
lectura a usuarios. Los estudios netnográficos pueden contribuir con el
análisis de los tipos de herramientas hipermedia disponibles que
permiten modificar prácticas previamente preferidas por otras formas de
leer y comprender los textos hipermedia \cite{azman2017hypermedia}. O
pueden estudiarse canales de YouTube donde se crean relatos y poemas
como casos de literacidad con subtitulados en lenguas extranjeras y
personas señantes \cite{broullón-lonzano2019}.

Incluso, se puede investigar netnográficamente el uso de software del
diccionario creado para la adquisición o el perfeccionamiento de una
lengua materna y una lengua extranjera, más aún si son solicitados por
profesores que trabajan en zonas con presencia de inmigración de otros
países y buscan que sus estudiantes logren el aprendizaje lingüístico
\cite{turrini2000}. El netnógrafo lingüista puede aprovechar
en interactuar con comunidades de consumo de diccionarios electrónicos
que cuentan con foro o retroalimentación virtual hacia la confección y
alimentación de los diccionarios.

Como se colige, a los lingüistas les puede interesar no solo la
comunicación humana mediada por el ordenador, sino también los trabajos
y aprendizajes colaborativos en el ordenador, que en última instancia
pueden generar interacciones entre los participantes del espacio
virtual.

En general, se ha visto que la netnografía se ajusta a las necesidades
de los lingüistas y los intereses que redundan en el progreso académico
científico y social.

\subsubsection{Procedimientos en una investigación netnográfica: hacia una
experiencia en trabajo}

En el período de pandemia y pospandemia, ha sido crucial las
netnografías en el análisis de varios temas. Una investigación en línea
emprendida ha sido de las ideologías lingüísticas que presentan
estudiantes universitarios sobre el inglés y las lenguas originarias del
país. Los estudiantes entrevistados y a quienes se ha observado por un
semestre académico son de la especialidad de Humanidades y llevan sus
cursos de forma virtual. Desde el tiempo en confinamiento por la
pandemia de la covid-19, la universidad impartió sus cursos en la
modalidad de educación a distancia, lo que dio paso a generar
comunidades digitales de estudiantes. En este caso, se decidió conocer
las formas en que piensan la enseñanza y el aprendizaje de una lengua
extranjera desde su profesión, incluso en entornos virtuales a los que
acceden también en tiempo de pandemia. Para ello, se procedió a
solicitar su consentimiento en esta investigación. Se observó por varias
ocasiones sus intervenciones \emph{online} y sus redacciones en los
chats, cuando referían a la lengua que aprenden. Conocían de la
presencia del investigador, con quien también pudieron interactuar en
ciertas oportunidades al ahondar en sus aprendizajes y opiniones,
especialmente sobre el tópico de indagación, en sincronía. El poder
entrevistarlos y ver sus valoraciones, además, en sesiones
individualizadas posteriores, a través de una plataforma como Zoom o
Google Meet, sobre los idiomas permite conocer las maneras en que
posicionan las lenguas. Algunos establecen que el inglés es una lengua
que se aprende fácilmente porque cuenta con herramientas digitales de
acceso para el aprendizaje y atrae en múltiples sentidos, dado que es
considerada un instrumento para encontrar trabajo u obtener un mejor
estatus social, en un mundo globalizado y tecnológico, mientras que las
lenguas originarias aparentemente se aprecian de que carecen de estos
atractivos. Cabe señalar que el investigador también era estudiante del
curso de lengua extranjera, por lo que era considerado un integrante
más, por lo que la observación participante se facilitó. Los estudiantes
en general fueron grabados y sus discursos se han ido transcribiendo
para clasificar y determinar las ideologías lingüísticas que están
detrás de sus opiniones. De manera asincrónica, se pidió que completen
algunas preguntas relacionadas con la indagación, especialmente para
recabar sus opiniones sobre la proliferación de subtitulados que se dan
en inglés frente a cualquier otra lengua. La investigación viene
complementándose con entrevistas y observaciones de estudiantes de
traducción. Esto permite explorar aún más las ideologías lingüísticas
sobre la supuesta utilidad de las lenguas que tienen otro grupo de
estudiantes relacionados con la especialidad de idiomas. En términos
contrastivos, por el momento, se viene encontrando similitudes en la
forma en que se sobrevalora el inglés frente a lenguas originarias en la
traducción de series y películas. Este tipo de trabajos se inserta
dentro de las netnografías sociolingüísticas. En general, el
investigador se presenta y participa, garantiza la confiabilidad de los
informantes, identifica y estudia al grupo, ofrece conocimientos
generados por la comunidad. Recordemos que \textcite{kozinets2002} sugiere que
se sigan siempre los principios básicos en la netnografía.