\section{Netnografía y lingüística}\label{sec-netnografiaylinguisica}

La netnografía es un tipo de metodología de investigación cualitativa
que se ocupa de estudiar las comunidades y de las culturas emergentes en
el espacio \emph{online} o ciberespacio, donde la información se produce
de manera natural. Según \textcite{kozinets1998}, la netnografía es un enfoque
etnográfico que se interesa por las comunidades en línea. Con este
método, se puede comprender la vida social de los sujetos y las
comunidades virtuales en la sociedad contemporánea y en un periodo de
tiempo. La netnografía se ha ido empleando para estudiar las diferentes
percepciones y preferencias de consumo de los productos, así como
también los servicios que las comunidades en el entorno virtual demandan
en sus relaciones de interacción \cite{casas-romeo2014,moralesvasquez2020,xun2010netnography}. Y, más recientemente, como método de
investigación posibilita el abordaje de lo que acontece en las
comunidades virtuales \cite[p. 9]{turpo2008}. Dichas comunidades comparten
valores, creencias y comportamientos, y, además, cada vez aumentan en la
red con el fin de crear relaciones con otros usuarios, mantenerse
informadas o consumir algún tipo de información. Se tratan de
comunidades mediadas por un ordenador. \textcite[p. 147]{martinez2017etnografia} sostienen que la etnografía no convencional ha ido entendiendo que
el ciberespacio es una realidad donde se construyen significados, se
generan identidades y se establecen agrupaciones relativamente estables
y con intereses que se comparten. Este método permite, asimismo, pensar
en los efectos y la relación que tiene el lenguaje en la vida social de
los cibernautas, por lo que interesa pensar sobre la utilidad práctica
que tiene la netnografía como instrumento de la indagación en la
especialidad de la lingüística.

Desde los trabajos de \textcite{herring_2004}, con los trabajos hechos sobre la
Comunicación-Mediada-por-Computadoras (CMC), la lingüística ha prestado
más atención a los medios digitales. Desde entonces, se han desarrollado
investigaciones sobre los discursos que se generan en la web, así como
las reflexiones metalingüísticas que poseen los cibernautas sobre sus
lenguas \cite{tannen2013}. Los lingüistas se sumergen en el
análisis del uso del lenguaje en los videojuegos, sitios de video,
plataformas, foros en línea \cite{lovón-lóvon2022} y medios sociales u
otros de comunicación virtual, como Flickr, Facebook, Twitter, WhatsApp,
YouTube, Youku, Paideia o Blackboard Collaborate. Los investigadores de
netnografías se acercan a este método para conocer la vida social en
redes sociales, plataformas virtuales o páginas web que manejan los
cibernautas. Según \textcite[p.~41]{putrikusama2016}, ``Researchers can make
themselves familiarised with the netnographic approach to help them
study chats such as Skype, WhatsApp Messenger, and Line; blogs such as
Tumblr, Blogspot, and Wordpress; even social networking sites such as
Facebook, Twitter, and Path''. Con la netnografía, no solo se estudian
espacios virtuales recientes, sino también aquellos que ya no se emplean
o tienen pocos usuarios, tal es el caso de hi5. Así como los lingüistas
estudian lenguas en peligro de extinción o muertes lingüísticas
\cite{ramirez-cruz2021}, pueden también en la virtualidad
analizar comunidades de habla en espacios que existieron. El interés por
el mundo digital se centra en varias facetas: en los géneros discursivos
digitales, los estilos y la creatividad de los internautas, las
prácticas y los recursos multilingüísticos en la comunicación digital,
el desarrollo de literacidades, la redacción y el aprendizaje de
lenguas, la construcción de identidades y amistades en torno al
lenguaje, la comunicación virtual pública y sus consecuencias en la
publicación de textos, el rol de la lengua en la multimodalidad, el
desarrollo de habilidades lectoras \cite{georgakopoulou2016}.
Aunque, según \textcite{kulavuz-onal2015}, la netnografía no es tan popular en
el campo de la educación ni en la lingüística aplicada o el aprendizaje
de idiomas asistido por ordenador.

Si bien distintas investigaciones se centran en el análisis de la
ortografía y la escritura de los mensajes de textos, la variación
lingüística o el sistema lingüístico, los nuevos estudios en el mundo
digital ponen peso al estudio del cibernauta como agente social. Dicho
de otro modo, la agenda de investigación ha cambiado, por lo que se
amerita conocer cómo piensa, siente, percibe, se transforma ---dado que
se performan identidades o despliegan nuevas constantemente--- y se
conduce en su entorno digital. \textcite{varis2019digital} señalan que la
etnografía digital se interesa por analizar las formas en que las
personas usan el lenguaje, interactúan entre sí, emplean discursos y
construyen comunidades, colectivos, conocimientos e identidades, a
través de las tecnologías digitales al mismo tiempo que se encuentran
bajo su influencia. Precisamente, la netnografía analiza los comentarios
y las conversaciones que se generan entre los participantes internautas,
así como sus formas de pensar y sus mitos a través del internet. \textcite[p. 41]{putrikusama2016} precisa que los lingüistas también pueden desarrollar
trabajos netnográficos: ``the development of cultures, communication,
and communities online can also encourage linguistic researchers to
conduct their research online'', y ello implica su acercamiento al mundo
virtual. Los lingüistas que se acercan al análisis del lenguaje en el
ciberespacio lo hacen, porque conocen este mundo o tienen interés por lo
que sucede en este espacio virtual. Entre múltiples comunidades
virtuales, los grupos indígenas y de lenguas originarias son importantes
de ser estudiados para explorar sus interacciones y demandas.

En relación con el campo de indagación, según \textcite{kozinets2002}, algunas
comunidades virtuales se presentan en los foros y los blogs de
discusión, las salas multiusuario, las listas de servidores, las páginas
Web independientes o los boletines electrónicos. En ese sentido, los
campos de investigación netnográficas para los lingüistas son las salas
de chat, los espacios de juego de videojuegos en red, los sitios de
campo en línea, como los tablones de anuncios o los foros, los servicios
de listas y las páginas web, los sitios de redes sociales, los wikis,
los blogs, los videoblogs, los microblogs. Teniendo en cuenta la
locación, la netnografía permite al investigador comprender el
funcionamiento de una comunidad que está distribuida en uno o múltiples
sitios y lugares de Internet, así como la interconexión entre sus
sitios, sus actividades y sus tecnologías. Debe repararse que para la
netnografía, los lugares de estudio surgen, además, en el proceso de
compromiso reflexivo del etnógrafo con los entornos digitales \cite{varis2019digital}. Por tanto, el lingüista entra en un compromiso con el
ciberespacio.

Asimismo, en términos de modalidades, a la netnografía le interesan los
espacios de comunicación sincrónica, como asincrónica \cite{kozinets2010}.
Con la netnografía incluso se puede obtener datos de los internautas
asincrónicamente cuando los obtiene de correos electrónicos o crea foros
de investigación específicos, o sincrónicamente por medio de plataformas
digitales como Zoom, Meet o Skype. A los lingüistas puede llamarle la
atención la comunicación por correos electrónicos entre personas e
instituciones, como de aquellas que se inician en la escritura virtual.
Cabe señalar que la desconectividad puede implicar una desventaja sobre
todo si se generan distancias entre la investigación y las personas
involucradas en el estudio.

Como se expone, la netnografía se involucra en la lingüística al
emplearse como método de investigación. Las metodologías en lingüística
son variadas y la netnografía contribuye con sus indagaciones en el
ciberespacio. La relación método y ciencia se unen para poder investigar
hechos, fenómenos y vidas en un espacio virtual llamativo.

Por último, debe señalarse que los resultados de las investigaciones
lingüísticas con netnografías deben ser valorados acorde con los
objetivos y la naturaleza planteadas por los investigadores. En tales
resultados se consideran las fortalezas, las limitaciones, las lecturas
e interpretaciones de los datos.
