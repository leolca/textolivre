\section{El lingüista como netnógrafo
}\label{sec-ellinguistacomonetnografo}

Los investigadores de netnografías son llamados netnógrafos. Como
especialistas reconocen que las comunicaciones, las creencias y las
actuaciones de los participantes en línea cobran sentido. Su campo de
investigación se encuentra dentro de la pantalla. Los netnógrafos se
involucran en la observación participante en línea al conectarse a la
comunidad a través del ordenador. Como investigadores, toman notas de
apuntes sobre lo que observan y también preguntan en línea a la
comunidad, mientras recogen sus impresiones e interrogantes sobre la
vida social virtual. Sus notas sirven para ir reuniendo sus
interpretaciones. Los netnógrafos observan, por ejemplo, el discurso
textual \cite{kozinets2002}. Sin embargo, existen trabajos que se llevan a
cabo con una observación no participante, o sin entrevistas, o sin tomar
notas, aunque de preferencia en la netnografía hay que sumergirse en la
comunidad y participar en ella como un miembro. Asimismo, los
netnógrafos se inscriben en una comunidad en línea, participan en
actividades sincrónicas y asincrónicas seleccionadas, entrevistan a
miembros clave de la comunidad, archivan documentos relevantes para
apoyar sus entrevistas y notas de campo, capturan pantallazos, imágenes
o audios importantes que enriquecen su trabajo. Y pueden llegar en el
contexto en línea a más internautas a través de la técnica de la bola de
nieve \cite{Ardèvol_Lanzeni_2014}. Los analistas pueden basarse en los
datos textuales, imágenes, videos, sonidos.

El lingüista es netnógrafo cuando indaga aspectos del lenguaje y usa el
método de la netnografía para observar y entrevistar comunidades
digitales. Él usa notas o apuntes de campo sobre su trabajo realizado en
el ciberespacio. En su aplicación, es participante de las interacciones
e interpreta la vida social en que el lenguaje construye prácticas
sociales. En relación con ello, en la interpretación de información, aun
cuando los investigadores empleen \emph{software} de análisis de datos o
herramientas especializadas y se utilicen para establecer ideas lógicas
con sus interpretaciones, el análisis y la interpretación responde
finalmente a lo que concluye el investigador, pues es él quien maneja
los marcos conceptuales, conoce las visiones del mundo, abstrae
categorías, recolecta información directamente, a partir de trabajos de
campo con la comunidad e incluso previos, compara e integra información,
y reinterpreta los resultados acorde con sus progresos científicos y
humanísticos \cite[p. 10]{sanchez2017}. El lingüista es un
especialista preparado para las indagaciones lingüísticas \emph{face to
face} y virtual. Además, sabe que las comunidades virtuales que
investiga pueden estar en un espacio específico o en varios, pues se
tratan de comunidades desterritorializadas. Según \textcite[p. 4]{kulavuz-onalvásquez2013}, ``in netnography, field-sites can be diverse, because an
online community can exist in a single site or multiple sites''.

Como investigador se familiariza con los códigos lingüísticos y éticos
\cite{Tagg2017}. Así, al realizar netnografías debe
considerar el anonimato de los participantes, en el caso de que la
información que use la va a utilizar en alguna publicación o algunas
conferencias públicas o en cualquier otra forma de difusión. Lo más
recomendable es que consulte a sus interlocutores si puede usar sus
identificaciones o si prefieren ser ocultados en algún anonimato
\cite{boellstorff2012ethnography}. Sobre su responsabilidad en la
investigación, \textcite{kozinets2010} considera que se debe realizar siempre
netnografía éticas. Por eso, propone que los investigadores pidan
permiso; obtengan el consentimiento informado; citen, anonimicen o
acrediten a los participantes en la investigación; y tengan en cuenta
cualquier consideración legal relacionada con la investigación. Cabe
señalar que evita los prejuicios en su observación y análisis, si
encuentra por ejemplo faltas ortográficas en las redacciones o inclusive
agresiones verbales entre los usuarios, entonces toma el material de
estudio como lo halló, lo procesa en relación con las técnicas de
investigación correspondiente e interpreta con los lentes teóricos que
maneja. En relación con cuestionarios y preguntas de entrevistas, si se
aplican en el momento a los sujetos de estudio o se les entrega antes,
como podría ocurrir en algún proyecto, estos pasan por validaciones o
juicio de expertos.

Asimismo, como netnógrafo en comunidades virtuales en que participa se
revela como investigador en el sitio web que investiga, por lo que
informa sobre sus objetivos, los antecedentes y los detalles del
estudio. En su participación en línea, cuida no tener un papel
dominante, sino más bien participa de la misma manera que los
cibernautas reales lo realizan de forma rutinaria. El lingüista como
netnógrafo en sus entrevistas puede mantener una conversación
multimodal, no solo plantea preguntas oralmente, sino también, por
ejemplo, puede usar una presentación de diapositivas en una pizarra
virtual o presentar imágenes que fomenten la interacción en línea, como
puede ocurrir en una videollamada o una sala Zoom. De esta manera,
recoge datos para analizar. Como netnógrafo establece conexiones en la
comunidad y sigue los acontecimientos y las actividades. A través de las
narrativas digitales puede conocer modos de vida \textcite{londoñopalacio2012}. Como
el netnógrafo no está copresente en un espacio físico con la comunidad,
debe determinar su campo de estudio en función de las interacciones
discursivas que se generan entre los miembros. El netnógrafo lingüista
es quien decide qué interacción incluir y cuál no \cite{kulavuz-onal2015}.
Sin embargo, debe tener en cuenta que su netnografía depende de su
acceso a la tecnología que utiliza la comunidad en línea, como la
conexión fiable a internet. El desconectarse de una comunicación por un
tiempo corto o prolongado puede impactar en el estudio. Si bien algunas
investigaciones \emph{online} son sincrónicas, los investigadores por
diversos motivos ajenos a su indagación cesan de investigar los
fenómenos que buscaba comprender. El compromiso del investigador con la
comunidad en línea es variable: puede durar unos meses o unos años. Solo
su desvinculación con la comunidad determina la salida del campo.

En el caso de las entrevistas, como netnógrafo, puede grabar videos y
usar \emph{softwares} de captura de audio en sus ordenadores. Hasta el
momento, Facebook y Skype son medios útiles para estos casos. Aunque
debe considerar que una de las limitaciones es la poca posibilidad de
capturar el lenguaje no verbal, pues la mayoría de las netnografías
pueden correr el riesgo de no recoger la información tan igual que las
videograbaciones presenciales; además, la mayoría se basan más en la
data textual de los mensajes electrónicos, que incluso no requieren que
alguna cámara sea encendida, basta con escuchar al sujeto entrevistado
con un micrófono \cite{noveli2010}. Al respecto, como netnógrafo debe
considerar que la presencia de la cámara condiciona a los participantes
a sobreactuar o sentirse cohibidos \cite[p. 116]{boellstorff2012ethnography}, por lo que debe encontrar o construir un momento o ambiente de
familiarización que favorezca la interacción. Al respecto, el lingüista
requiere optar por protocolos de investigación adecuados.

Como se entiende, sus trabajos sobre el uso del lenguaje deben tener en
cuenta las normas sociolingüísticas, los valores lingüístico-culturales,
la manifestación de identidades intralingüísticas, las relaciones
personales interlingüísticas y las prácticas cotidianas e
institucionales lingüísticas de los internautas en el entorno
electrónico.