\documentclass[spanish]{textolivre}

% metadata
\journalname{Texto Livre}
\thevolume{18}
%\thenumber{1} % old template
\theyear{2025}
\receiveddate{\DTMdisplaydate{2024}{7}{15}{-1}}
\accepteddate{\DTMdisplaydate{2024}{8}{20}{-1}}
\publisheddate{\today}
\corrauthor{Marco Lovón-Cueva}
\articledoi{10.1590/1983-3652.2025.46890}
%\articleid{NNNN} % if the article ID is not the last 5 numbers of its DOI, provide it using \articleid{} commmand 
% list of available sesscions in the journal: articles, dossier, reports, essays, reviews, interviews, editorial
\articlesessionname{articles}
\runningauthor{Lovón-Cueva}
%\editorname{Leonardo Araújo} % old template
\sectioneditorname{Daniervelin Pereira}
\layouteditorname{João Mesquita}

\title{La netnografía en la investigación lingüística}
\othertitle{A netnografia na investigação linguística}
\othertitle{Netnography in linguistic research}


\author[1]{Marco Lovón-Cueva~\orcid{0000-0002-9182-6072}\thanks{Email: \href{mailto:marcolovoncueva@gmail.com}{marcolovoncueva@gmail.com}}}
\affil[1]{Universidad Nacional Mayor de San Marcos, Lima, Perú.}

\addbibresource{article.bib}

\begin{document}
\maketitle
\begin{polyabstract}

\begin{spanish}
\begin{abstract}
 La netnografía se emplea estratégicamente en diferentes investigaciones sociales, económicas y culturales. En el campo de la lingüística puede contribuir con el análisis de las identidades e interacciones sociales en las que el lenguaje tiene presencia en el ciberespacio. El propósito de este estudio es reflexionar sobre la utilidad práctica de la netnografía en la investigación lingüística. Para ello, se propone el rol que tiene el lingüista como netnógrafo y se sugieren un conjunto de áreas y líneas de investigación que le concierne en la gestación de una netnografía lingüística. El trabajo concluye que la netnografía es un método cualitativo que permite a los lingüistas interpretar la vida social de las comunidades virtuales, principalmente en relación con el lenguaje en uso y las estructuras sociales.  

\keywords{Etnografía \sep Netnografía \sep Lingüística \sep Trabajo de Campo \sep Transdisciplina}
\end{abstract}
\end{spanish}

\begin{portuguese}
\begin{abstract}
    A netnografia é usada estrategicamente em diferentes pesquisas sociais, econômicas e culturais. No campo da linguística, ela pode contribuir para a análise de identidades e interações sociais nas quais a linguagem está presente no ciberespaço. O objetivo deste estudo é refletir sobre a utilidade prática da netnografia na pesquisa linguística. Para isso, propõe o papel do linguista como netnógrafo e sugere um conjunto de áreas e linhas de pesquisa que lhe dizem respeito na gestação de uma netnografia linguística. O artigo conclui que a netnografia é um método qualitativo que permite aos linguistas interpretar a vida social das comunidades virtuais, principalmente em relação à linguagem em uso e às estruturas sociais.      
\keywords{Etnografia \sep Netnografia \sep Linguística \sep Trabalho de campo \sep Transdisciplina}
\end{abstract}
\end{portuguese}

\begin{english}
\begin{abstract}
    Netnography is used strategically in different social, economic and cultural research. In the field of linguistics it can contribute to the analysis of identities and social interactions in which language is present in cyberspace. The purpose of this study is to reflect on the practical usefulness of netnography in linguistic research. To this end, it proposes the role of the linguist as netnographer and suggests a set of areas and lines of research that concern him/her in the gestation of a linguistic netnography. The paper concludes that netnography is a qualitative method that allows linguists to interpret the social life of virtual communities, mainly in relation to language in use and social structures.

\keywords{Ethnography \sep Netnography \sep Linguistics \sep Field work \sep Transdiscipline}
\end{abstract}
\end{english}
\end{polyabstract}

\section{Introduction}\label{sec-intro}

Technology-assisted interpreting has experienced exponential growth in
recent years, reflected in the dizzying progress in the development of
information and communication technology (ICT) tools and resources
\cite{gutierrezArtacho2016,mezcua2019}. These innovative
technologies have greatly facilitated the interpretation and
comprehension of texts in different linguistic contexts \cite{olallaSoler2015}. In this sense, computer-assisted interpreting (CAI)
has become crucial in an increasingly globalised and connected society,
playing a pivotal role in breaking down language barriers and
facilitating effective communication in an environment characterised by
cultural and linguistic diversity \cite{mellinger2019,li2021}. In
response to the growing demand for instant online communication and the
need to overcome language limitations in a global environment, CAI has
established itself as an infallible tool for a wide range of
applications, from interpreting international business meetings to
translating multimedia content in real time \cite{fantinuoli2017a,alcaidemartinez2021}.

The integration of technology has revolutionised the way individuals
interact with language, opening new possibilities for increasing the
efficiency and accuracy of text interpretation, both in real time and
asynchronously \cite{gaber2023a,ramirezRodriguez2023}. The development
of ICT tools has led to improvements in natural language processing,
machine translation, speech recognition and other related technologies,
resulting in significant advances in text interpretation \cite{valeroGarces2024}. These tools have been particularly useful in overcoming language
barriers in multicultural environments, facilitating communication
between speakers of different languages and promoting intercultural
understanding \cite{perez2020}.

The use of CAI does, however, present certain difficulties. In the
phraseological domain, understanding idiomatic expressions remains a
significant challenge, as these linguistic constructions can be
particularly difficult to interpret accurately due to their cultural
embeddedness \cite{corpaspastorgaber2021,ramirezRodriguez2022}.
Cultural and contextual differences can lead to inaccurate or ambiguous
translations of these expressions, resulting in misunderstandings and
communication problems. In addition, CAI is often based on pre-defined
algorithms and databases, which can limit its ability to adapt to new
idioms and emerging expressions \cite{ortigoza2024}. This problem
identified in the interpretation of idiomatic expressions by CAI adds to
the previously mentioned challenges in technology-assisted phraseology,
where accuracy and fluency in the interpretation of texts in different
linguistic contexts are key to effective communication. The complexity
of idiomatic expressions, such as idioms, and their culturally
contextual nature require new approaches and specialised tools to
improve the interpretation of such expressions, which represents a
relevant area of research in the development of technology-assisted
phraseology.
\section{Netnografía y lingüística
}\label{sec-netnografiaylinguisica}

La netnografía es un tipo de metodología de investigación cualitativa
que se ocupa de estudiar las comunidades y de las culturas emergentes en
el espacio \emph{online} o ciberespacio, donde la información se produce
de manera natural. Según \textcite{kozinets1998}, la netnografía es un enfoque
etnográfico que se interesa por las comunidades en línea. Con este
método, se puede comprender la vida social de los sujetos y las
comunidades virtuales en la sociedad contemporánea y en un periodo de
tiempo. La netnografía se ha ido empleando para estudiar las diferentes
percepciones y preferencias de consumo de los productos, así como
también los servicios que las comunidades en el entorno virtual demandan
en sus relaciones de interacción \cite{casas-romeo2014,moralesvasquez2020,xun2010netnography}. Y, más recientemente, como método de
investigación posibilita el abordaje de lo que acontece en las
comunidades virtuales \cite[p. 9]{turpo2008}. Dichas comunidades comparten
valores, creencias y comportamientos, y, además, cada vez aumentan en la
red con el fin de crear relaciones con otros usuarios, mantenerse
informadas o consumir algún tipo de información. Se tratan de
comunidades mediadas por un ordenador. \textcite[p. 147]{martinez2017etnografia} sostienen que la etnografía no convencional ha ido entendiendo que
el ciberespacio es una realidad donde se construyen significados, se
generan identidades y se establecen agrupaciones relativamente estables
y con intereses que se comparten. Este método permite, asimismo, pensar
en los efectos y la relación que tiene el lenguaje en la vida social de
los cibernautas, por lo que interesa pensar sobre la utilidad práctica
que tiene la netnografía como instrumento de la indagación en la
especialidad de la lingüística.

Desde los trabajos de \textcite{herring_2004}, con los trabajos hechos sobre la
Comunicación-Mediada-por-Computadoras (CMC), la lingüística ha prestado
más atención a los medios digitales. Desde entonces, se han desarrollado
investigaciones sobre los discursos que se generan en la web, así como
las reflexiones metalingüísticas que poseen los cibernautas sobre sus
lenguas \cite{tannen2013}. Los lingüistas se sumergen en el
análisis del uso del lenguaje en los videojuegos, sitios de video,
plataformas, foros en línea \cite{lovón-lóvon2022} y medios sociales u
otros de comunicación virtual, como Flickr, Facebook, Twitter, WhatsApp,
YouTube, Youku, Paideia o Blackboard Collaborate. Los investigadores de
netnografías se acercan a este método para conocer la vida social en
redes sociales, plataformas virtuales o páginas web que manejan los
cibernautas. Según \textcite{putrikusama2016}, ``Researchers can make
themselves familiarised with the netnographic approach to help them
study chats such as Skype, WhatsApp Messenger, and Line; blogs such as
Tumblr, Blogspot, and Wordpress; even social networking sites such as
Facebook, Twitter, and Path''. Con la netnografía, no solo se estudian
espacios virtuales recientes, sino también aquellos que ya no se emplean
o tienen pocos usuarios, tal es el caso de hi5. Así como los lingüistas
estudian lenguas en peligro de extinción o muertes lingüísticas
\cite{ramirez-cruz2021}, pueden también en la virtualidad
analizar comunidades de habla en espacios que existieron. El interés por
el mundo digital se centra en varias facetas: en los géneros discursivos
digitales, los estilos y la creatividad de los internautas, las
prácticas y los recursos multilingüísticos en la comunicación digital,
el desarrollo de literacidades, la redacción y el aprendizaje de
lenguas, la construcción de identidades y amistades en torno al
lenguaje, la comunicación virtual pública y sus consecuencias en la
publicación de textos, el rol de la lengua en la multimodalidad, el
desarrollo de habilidades lectoras\cite{georgakopoulou2016}.
Aunque, según \cite{kulavuz-onal2015}, la netnografía no es tan popular en
el campo de la educación ni en la lingüística aplicada o el aprendizaje
de idiomas asistido por ordenador.

Si bien distintas investigaciones se centran en el análisis de la
ortografía y la escritura de los mensajes de textos, la variación
lingüística o el sistema lingüístico, los nuevos estudios en el mundo
digital ponen peso al estudio del cibernauta como agente social. Dicho
de otro modo, la agenda de investigación ha cambiado, por lo que se
amerita conocer cómo piensa, siente, percibe, se transforma ---dado que
se performan identidades o despliegan nuevas constantemente--- y se
conduce en su entorno digital. \textcite{varis2019digital} señalan que la
etnografía digital se interesa por analizar las formas en que las
personas usan el lenguaje, interactúan entre sí, emplean discursos y
construyen comunidades, colectivos, conocimientos e identidades, a
través de las tecnologías digitales al mismo tiempo que se encuentran
bajo su influencia. Precisamente, la netnografía analiza los comentarios
y las conversaciones que se generan entre los participantes internautas,
así como sus formas de pensar y sus mitos a través del internet. \textcite[p. 41]{putrikusama2016} precisa que los lingüistas también pueden desarrollar
trabajos netnográficos: ``the development of cultures, communication,
and communities online can also encourage linguistic researchers to
conduct their research online'', y ello implica su acercamiento al mundo
virtual. Los lingüistas que se acercan al análisis del lenguaje en el
ciberespacio lo hacen, porque conocen este mundo o tienen interés por lo
que sucede en este espacio virtual. Entre múltiples comunidades
virtuales, los grupos indígenas y de lenguas originarias son importantes
de ser estudiados para explorar sus interacciones y demandas.

En relación con el campo de indagación, según \textcite{kozinets2002}, algunas
comunidades virtuales se presentan en los foros y los blogs de
discusión, las salas multiusuario, las listas de servidores, las páginas
Web independientes o los boletines electrónicos. En ese sentido, los
campos de investigación netnográficas para los lingüistas son las salas
de chat, los espacios de juego de videojuegos en red, los sitios de
campo en línea, como los tablones de anuncios o los foros, los servicios
de listas y las páginas web, los sitios de redes sociales, los wikis,
los blogs, los videoblogs, los microblogs. Teniendo en cuenta la
locación, la netnografía permite al investigador comprender el
funcionamiento de una comunidad que está distribuida en uno o múltiples
sitios y lugares de Internet, así como la interconexión entre sus
sitios, sus actividades y sus tecnologías. Debe repararse que para la
netnografía, los lugares de estudio surgen, además, en el proceso de
compromiso reflexivo del etnógrafo con los entornos digitales \cite{varis2019digital}. Por tanto, el lingüista entra en un compromiso con el
ciberespacio.

Asimismo, en términos de modalidades, a la netnografía le interesan los
espacios de comunicación sincrónica, como asincrónica \cite{kozinets2010}.
Con la netnografía incluso se puede obtener datos de los internautas
asincrónicamente cuando los obtiene de correos electrónicos o crea foros
de investigación específicos, o sincrónicamente por medio de plataformas
digitales como Zoom, Meet o Skype. A los lingüistas puede llamarle la
atención la comunicación por correos electrónicos entre personas e
instituciones, como de aquellas que se inician en la escritura virtual.
Cabe señalar que la desconectividad puede implicar una desventaja sobre
todo si se generan distancias entre la investigación y las personas
involucradas en el estudio.

Como se expone, la netnografía se involucra en la lingüística al
emplearse como método de investigación. Las metodologías en lingüística
son variadas y la netnografía contribuye con sus indagaciones en el
ciberespacio. La relación método y ciencia se unen para poder investigar
hechos, fenómenos y vidas en un espacio virtual llamativo.

Por último, debe señalarse que los resultados de las investigaciones
lingüísticas con netnografías deben ser valorados acorde con los
objetivos y la naturaleza planteadas por los investigadores. En tales
resultados se consideran las fortalezas, las limitaciones, las lecturas
e interpretaciones de los datos.
\section{El lingüista como netnógrafo
}\label{sec-ellinguistacomonetnografo}

Los investigadores de netnografías son llamados netnógrafos. Como
especialistas reconocen que las comunicaciones, las creencias y las
actuaciones de los participantes en línea cobran sentido. Su campo de
investigación se encuentra dentro de la pantalla. Los netnógrafos se
involucran en la observación participante en línea al conectarse a la
comunidad a través del ordenador. Como investigadores, toman notas de
apuntes sobre lo que observan y también preguntan en línea a la
comunidad, mientras recogen sus impresiones e interrogantes sobre la
vida social virtual. Sus notas sirven para ir reuniendo sus
interpretaciones. Los netnógrafos observan, por ejemplo, el discurso
textual \cite{kozinets2002}. Sin embargo, existen trabajos que se llevan a
cabo con una observación no participante, o sin entrevistas, o sin tomar
notas, aunque de preferencia en la netnografía hay que sumergirse en la
comunidad y participar en ella como un miembro. Asimismo, los
netnógrafos se inscriben en una comunidad en línea, participan en
actividades sincrónicas y asincrónicas seleccionadas, entrevistan a
miembros clave de la comunidad, archivan documentos relevantes para
apoyar sus entrevistas y notas de campo, capturan pantallazos, imágenes
o audios importantes que enriquecen su trabajo. Y pueden llegar en el
contexto en línea a más internautas a través de la técnica de la bola de
nieve \cite{Ardèvol_Lanzeni_2014}. Los analistas pueden basarse en los
datos textuales, imágenes, videos, sonidos.

El lingüista es netnógrafo cuando indaga aspectos del lenguaje y usa el
método de la netnografía para observar y entrevistar comunidades
digitales. Él usa notas o apuntes de campo sobre su trabajo realizado en
el ciberespacio. En su aplicación, es participante de las interacciones
e interpreta la vida social en que el lenguaje construye prácticas
sociales. En relación con ello, en la interpretación de información, aun
cuando los investigadores empleen \emph{software} de análisis de datos o
herramientas especializadas y se utilicen para establecer ideas lógicas
con sus interpretaciones, el análisis y la interpretación responde
finalmente a lo que concluye el investigador, pues es él quien maneja
los marcos conceptuales, conoce las visiones del mundo, abstrae
categorías, recolecta información directamente, a partir de trabajos de
campo con la comunidad e incluso previos, compara e integra información,
y reinterpreta los resultados acorde con sus progresos científicos y
humanísticos \cite[p. 10]{sanchez2017}. El lingüista es un
especialista preparado para las indagaciones lingüísticas \emph{face to
face} y virtual. Además, sabe que las comunidades virtuales que
investiga pueden estar en un espacio específico o en varios, pues se
tratan de comunidades desterritorializadas. Según \textcite[p. 4]{kulavuz-onalvásquez2013}, ``in netnography, field-sites can be diverse, because an
online community can exist in a single site or multiple sites''.

Como investigador se familiariza con los códigos lingüísticos y éticos
\cite{Tagg2017}. Así, al realizar netnografías debe
considerar el anonimato de los participantes, en el caso de que la
información que use la va a utilizar en alguna publicación o algunas
conferencias públicas o en cualquier otra forma de difusión. Lo más
recomendable es que consulte a sus interlocutores si puede usar sus
identificaciones o si prefieren ser ocultados en algún anonimato
\cite{boellstorff2012ethnography}. Sobre su responsabilidad en la
investigación, \textcite{kozinets2010} considera que se debe realizar siempre
netnografía éticas. Por eso, propone que los investigadores pidan
permiso; obtengan el consentimiento informado; citen, anonimicen o
acrediten a los participantes en la investigación; y tengan en cuenta
cualquier consideración legal relacionada con la investigación. Cabe
señalar que evita los prejuicios en su observación y análisis, si
encuentra por ejemplo faltas ortográficas en las redacciones o inclusive
agresiones verbales entre los usuarios, entonces toma el material de
estudio como lo halló, lo procesa en relación con las técnicas de
investigación correspondiente e interpreta con los lentes teóricos que
maneja. En relación con cuestionarios y preguntas de entrevistas, si se
aplican en el momento a los sujetos de estudio o se les entrega antes,
como podría ocurrir en algún proyecto, estos pasan por validaciones o
juicio de expertos.

Asimismo, como netnógrafo en comunidades virtuales en que participa se
revela como investigador en el sitio web que investiga, por lo que
informa sobre sus objetivos, los antecedentes y los detalles del
estudio. En su participación en línea, cuida no tener un papel
dominante, sino más bien participa de la misma manera que los
cibernautas reales lo realizan de forma rutinaria. El lingüista como
netnógrafo en sus entrevistas puede mantener una conversación
multimodal, no solo plantea preguntas oralmente, sino también, por
ejemplo, puede usar una presentación de diapositivas en una pizarra
virtual o presentar imágenes que fomenten la interacción en línea, como
puede ocurrir en una videollamada o una sala Zoom. De esta manera,
recoge datos para analizar. Como netnógrafo establece conexiones en la
comunidad y sigue los acontecimientos y las actividades. A través de las
narrativas digitales puede conocer modos de vida \textcite{londoñopalacio2012}. Como
el netnógrafo no está copresente en un espacio físico con la comunidad,
debe determinar su campo de estudio en función de las interacciones
discursivas que se generan entre los miembros. El netnógrafo lingüista
es quien decide qué interacción incluir y cuál no \cite{kulavuz-onal2015}.
Sin embargo, debe tener en cuenta que su netnografía depende de su
acceso a la tecnología que utiliza la comunidad en línea, como la
conexión fiable a internet. El desconectarse de una comunicación por un
tiempo corto o prolongado puede impactar en el estudio. Si bien algunas
investigaciones \emph{online} son sincrónicas, los investigadores por
diversos motivos ajenos a su indagación cesan de investigar los
fenómenos que buscaba comprender. El compromiso del investigador con la
comunidad en línea es variable: puede durar unos meses o unos años. Solo
su desvinculación con la comunidad determina la salida del campo.

En el caso de las entrevistas, como netnógrafo, puede grabar videos y
usar \emph{softwares} de captura de audio en sus ordenadores. Hasta el
momento, Facebook y Skype son medios útiles para estos casos. Aunque
debe considerar que una de las limitaciones es la poca posibilidad de
capturar el lenguaje no verbal, pues la mayoría de las netnografías
pueden correr el riesgo de no recoger la información tan igual que las
videograbaciones presenciales; además, la mayoría se basan más en la
data textual de los mensajes electrónicos, que incluso no requieren que
alguna cámara sea encendida, basta con escuchar al sujeto entrevistado
con un micrófono \cite{noveli2010}. Al respecto, como netnógrafo debe
considerar que la presencia de la cámara condiciona a los participantes
a sobreactuar o sentirse cohibidos \cite[p. 116]{boellstorff2012ethnography}, por lo que debe encontrar o construir un momento o ambiente de
familiarización que favorezca la interacción. Al respecto, el lingüista
requiere optar por protocolos de investigación adecuados.

Como se entiende, sus trabajos sobre el uso del lenguaje deben tener en
cuenta las normas sociolingüísticas, los valores lingüístico-culturales,
la manifestación de identidades intralingüísticas, las relaciones
personales interlingüísticas y las prácticas cotidianas e
institucionales lingüísticas de los internautas en el entorno
electrónico.
\section{Áreas y líneas de investigación}\label{sec-areasylineasdeinvestigacion}

Teniendo en cuenta la diversidad de posibilidades, las preocupaciones de
las netnografías lingüísticas pueden resumirse en netnopragmática,
netnosociolingüística y netnolingüística aplicada, las cuales mantienen
diálogo entre sí, pues lo social, lo cultural, lo cognitivo y lo
lingüístico están imbricados entre sí en el análisis de actores,
comunidades y acciones virtuales. Ha de precisarse que una netnografía
lingüística es una investigación sobre un tema lingüístico que emplea el
método netnográfico. En ese orden, la netnopragmática es la búsqueda de
fenómenos estudiados por la pragmática utilizando la netnografía; la
netnosociolingüística es la investigación acerca de aspectos en que la
lengua y la sociedad se unen manejando la metodología netnográfica; la
netnolingüística aplicada es la indagación en que interviene la
lingüística aplicada valiéndose de la netnografía. A continuación, se
muestran las tres áreas con líneas de investigación que pueden ayudar al
lingüista netnógrafo en la realización de sus pesquisas:

\subsection{Netnopragmática}\label{sub-sec-netnopragmática}

\textcite{putrikusama2016} sostiene que la netnopragmática es el estudio pragmático
sobre el uso del lenguaje por parte de los miembros de la comunidad
\emph{online} y propone la realización de estudios netnopragmáticos,
como el análisis de las implicaturas y las comunicaciones entre los
miembros del mismo grupo de Facebook. Para él, la netnopragmática
considera que los datos en línea son actos sociales, que el mundo en
línea funciona como un entorno social y que la búsqueda del significado
de los actos está en relación con el contexto que precede o sigue. Y
señala que la netnopragmática pueden estudiar datos visuales como el
diseño de páginas y mensajes, los emoticonos, los colores y tipo de
letra, las imágenes y fotografías. El mismo autor indica que en la
observación de los actos de habla de los miembros de una cultura en el
ciberespacio hay que tener en cuenta los cumplidos, las invitaciones,
las disculpas, las negativas y las ofertas, así como los lazos de
relación entre los miembros, como el grado de solidaridad, el poder
relativo y las imposiciones.

Al respecto, se puede investigar la cortesía en Facebook, por ejemplo,
en el mantenimiento de las relaciones laborales. Algunos estudios han
evaluado las estrategias lingüísticas que se usan para establecer
amistades, realizar posteos, replicar información, saludar por
cumpleaños, lo que finalmente se entiende como rituales en la web \cite{west2013}.

En esta línea, pueden realizarse trabajos sobre los usos de figuras
retóricas en la comunicación virtual, como la ironía, el humor o la
metáfora. \textcite{taiwo2016analyzing} han encontrado relaciones entre la
pragmática y la comunicación virtual. \textcite{coker2016} examinan la
retórica del humor visual en Facebook y descubren que este es
gubernativo, institucional, cultural y grotesco, y se emplea para
ridiculizar los problemas de la sociedad de forma abierta o encubierta.
Ellos recurren al trabajo netnográfico para interpretar que el humor
visual suele estar moldeado por las visiones del mundo, los prejuicios y
los intereses de sus usuarios. Ellos señalan que metodológicamente
fueron observadores y participantes muy activos en la publicación y
lectura de comentarios en línea por casi siete años. Con su trabajo los
autores concluyen que el humor no es un discurso neutral, sino más bien
legitima y forma una serie de valores.

Por otro lado, la netnopragmática puede servir para estudiar las
interacciones lingüísticas que se dan en torno al consumo de productos
en línea. \textcite{septianasari2021} utilizaron el método
netnográfico con un enfoque pragmático para recoger información sobre el
consumo de publicidad cosmética y analizar los actos de habla empleados
para persuadir por medio del marketing \emph{online} en Indonesia. Esto
nos recuerda que la netnografía se ha utilizado ampliamente en la
investigación de marketing. Para \textcite{casas-romeo2014}, la
netnografía es una herramienta que se aprovecha para recoger información
del mundo virtual para la toma de decisiones de marketing. Los estudios
netnográficos han cobrado importancia en el campo de la administración
en la búsqueda de los significados que pueden servir a las empresas para
lograr la comprensión, de lo que los clientes dicen por escrito en el
espacio virtual y fomentan el desarrollo de etnografías de la
comunicación \cite{freitas2012netnografia}.

\subsection{Netnosociolingüística}\label{sub-sec-netnosociolingüística}

La netnosociolingüística puede estudiar la construcción de identidades y
relaciones sociales en el ciberespacio. Le interesa las preocupaciones
de las identidades en curso. Se puede analizar la identidad digitalizada
de un grupo social que se negocia en línea a través de la actuación
textual y el metalenguaje de los participantes \cite{campbell2006}, pero
también las identidades que se cambian por seguridad en espacios
virtuales, especialmente en el campo de la política, la migración o la
religión, por ejemplo. Las actuaciones de identidad en los entornos en
línea no siempre están disociadas de las identidades fuera de línea,
pues tienden a ser continuas. Sin embargo, las identidades \emph{online}
no coinciden siempre con las \emph{offline}, pues en el mundo virtual se
performan otras identidades que el agente social configura, muchas veces
para protegerse, como sucede con las comunidades LGTBIQ+, que por
seguridad pueden recrear identidades \emph{online}, incluso en
aplicaciones de búsqueda de pareja. A los netnógrafos puede llamarle la
atención las aplicaciones como Grindr, Moovz, Growlr, Scissr, Scruff,
LGBTQutie, Chappy, Wapo y Wapa, Tinder. Los investigadores que pueden
integrar la relación entre las identidades \emph{online} y
\emph{offline} de los miembros lo logran al reunirse con los informantes
cara a cara.

Asimismo, desde la netnosociolingüística, se puede investigar los
lenguajes y discursos de ciberodio que se generan en la web, las
ideologías lingüísticas sobre ortografía de los internautas, las
ideologías que se encuentran tras la enseñanza de idiomas en el virtual,
las prácticas de vigilancia sobre la normativa lingüística y las
conductas sociales que mantienen diversos grupos con discursos
normativistas, así como el activismo lingüístico que busca deconstruir
prácticas de dominación social. Según \textcite[p. 57]{londoñopalacio2012}, ``Las redes
de comunicación tienen un papel definitivo en las narrativas digitales,
pues de manera explícita o implícita modifican entornos, hábitos y
prácticas sociales''. Dado que las literacidades muestran la relación
entre el discurso y el poder \cite{bloome2005discourse,jimenez2015}, a través del estudio netnográfico se puede analizar las
literacidades virtuales como prácticas sociales de relaciones de poder,
entre actores que se perciben con autoridad frente a los que no. La
netnosociolingüística debe ayudar a develar que los entornos
electrónicos no proporcionan neutralidad, sino más bien son espacios de
relaciones de poder en que se usa el lenguaje. Los análisis pueden estar
sesgados por las historias personales o por la ausencia de voces no
escuchadas \cite{noveli2010}. A este enfoque también le interesa las
literacidades no hegemónicas que muchas veces se ven afectadas por
prácticas de poder que evitan la aceptación y generación de
literacidades vernáculas. Adicionalmente, estudian las desigualdades
sociales y las posibilidades de transformación social asociadas con las
tecnologías \cite{Ardèvol_Lanzeni_2014}.

Además, el netnógrafo puede interesarse por los motivos en que los
internautas dan a sus interacciones sociales. \textcite{leppanen2009} estudian las particularidades de la acción lingüística, social y
cultural de un grupo de jóvenes finlandeses en los espacios translocales
de los nuevos medios de comunicación, y analizan la forma en que ellos
mismos dan sentido y razón a sus acciones. Los autores muestran que la
translocalidad se manifiesta, en particular, en la elección del idioma y
en la heteroglosia lingüística y estilística, la coexistencia, la mezcla
y la alternancia de diferentes lenguas, registros, géneros y estilos.
Los jóvenes seleccionan el inglés como lengua de comunicación en lugar
de su primera lengua, y emplean y combinan recursos de más de una
lengua, como el inglés con el japonés, e integran géneros y estilos de
una lengua en sus discursos.

También puede emplearse la netnografía para el análisis del discurso y
armado de corpus amplios como se ofrece con los datos en Twitter. Se
puede usar para analizar muestras de discusión pública generadas en
dicho espacio \cite{Bonilla_2022}. A propósito, debe señalarse que ``La
netnografía, como forma de investigación, hace un énfasis en la
recolección y el procesamiento de cantidades significativas de datos
empíricos para producir conocimiento al respecto'' \cite[p. 7]{Bonilla_2022}.

\subsection{Netnolingüística aplicada} \label{sub-sec-netnolingüísticaaplicada}

Desde la netnolingüística aplicada se pueden realizar trabajos sobre la
adquisición y enseñanza de lenguas nacionales y extranjeras, como
primeras y segundas lenguas, de canal oral o visual, como la lengua de
señas, de uno o grupo de usuarios en la red. \textcite{kulavuz-onal2015} señala
que con la netnografía se puede entender la cultura de las comunidades
de aprendizaje y enseñanza de idiomas en línea. El uso de las
tecnologías ha atraído la atención de muchos estudiosos en el campo del
aprendizaje de segundas lenguas, entonces a través de la observación
participante y las entrevistas en línea semiestructuradas con los
estudiantes se puede conocer sus experiencias, gustos, desventajas con
los sitios web de aprendizaje, y también las maneras en que integran la
tecnología a sus modos de estudiar. Investigar la enseñanza de idiomas
favorece a otros para que tomen decisiones de diseño de los aprendizajes
en el mundo digital. \textcite{kulavuz-onal2013} ha ido investigando los
aprendizajes de los profesores de inglés para poder enseñar con la
tecnología participativa en una comunidad de práctica en línea. Al
respecto, \textcite{kesller2021} sostienen que las netnografías se
pueden aplicar en el estudio de la adquisición de segundas lenguas; por
ejemplo, se puede analizar los hábitos formales e informales de
aprendizaje de idiomas que tienen los usuarios en chat o foros. Entre
diversas investigaciones, ellos señalan que el estudio de \textcite{isbell2018online}
ofrece una mirada hacia los hábitos informales de aprendizaje de
personas que aprenden coreano a través de chats, los que permiten
evidenciar comportamientos similares a las prácticas tradicionales del
aula al ver que los aprendices se centran en aprender la parte formal o
estructural de la lengua.

Por otro lado, \textcite{kulavuzonal2018} han sugerido analizar por
medio de netnografías las interacciones virtuales translingüísticas que
ocurren en la comunidad global de educadores de inglés como lengua
extranjera, por ejemplo. Así, observa que los participantes recurren a
sus repertorios multilingües dentro de Facebook, y que a través de
entrevistas etnográficas con los profesores y documentos en línea de su
telecolaboración, descubre que si bien el grupo se construyó
discursivamente como una zona exclusiva para emplear el inglés con el
fin de que los profesores orienten y fomenten el uso del inglés en sus
alumnos, todos los participantes rompen las instrucciones del espacio,
pues recurren a otras lenguas como el español y el árabe para diversos
propósitos, sobre todo los maestros, como en los casos de
establecimiento de solidaridad, selección de un destinatario, y el
modelado de la sensibilidad intercultural.

También, se puede estudiar las plataformas que facilitan procesos de
lectura a usuarios. Los estudios netnográficos pueden contribuir con el
análisis de los tipos de herramientas hipermedia disponibles que
permiten modificar prácticas previamente preferidas por otras formas de
leer y comprender los textos hipermedia \cite{azman2017hypermedia}. O
pueden estudiarse canales de YouTube donde se crean relatos y poemas
como casos de literacidad con subtitulados en lenguas extranjeras y
personas señantes \cite{broullón-lonzano2019}.

Incluso, se puede investigar netnográficamente el uso de software del
diccionario creado para la adquisición o el perfeccionamiento de una
lengua materna y una lengua extranjera, más aún si son solicitados por
profesores que trabajan en zonas con presencia de inmigración de otros
países y buscan que sus estudiantes logren el aprendizaje lingüístico
\cite{turrini2000}. El netnógrafo lingüista puede aprovechar
en interactuar con comunidades de consumo de diccionarios electrónicos
que cuentan con foro o retroalimentación virtual hacia la confección y
alimentación de los diccionarios.

Como se colige, a los lingüistas les puede interesar no solo la
comunicación humana mediada por el ordenador, sino también los trabajos
y aprendizajes colaborativos en el ordenador, que en última instancia
pueden generar interacciones entre los participantes del espacio
virtual.

En general, se ha visto que la netnografía se ajusta a las necesidades
de los lingüistas y los intereses que redundan en el progreso académico
científico y social.

\subsubsection{Procedimientos en una investigación netnográfica: hacia una
experiencia en trabajo}

En el período de pandemia y pospandemia, ha sido crucial las
netnografías en el análisis de varios temas. Una investigación en línea
emprendida ha sido de las ideologías lingüísticas que presentan
estudiantes universitarios sobre el inglés y las lenguas originarias del
país. Los estudiantes entrevistados y a quienes se ha observado por un
semestre académico son de la especialidad de Humanidades y llevan sus
cursos de forma virtual. Desde el tiempo en confinamiento por la
pandemia de la covid-19, la universidad impartió sus cursos en la
modalidad de educación a distancia, lo que dio paso a generar
comunidades digitales de estudiantes. En este caso, se decidió conocer
las formas en que piensan la enseñanza y el aprendizaje de una lengua
extranjera desde su profesión, incluso en entornos virtuales a los que
acceden también en tiempo de pandemia. Para ello, se procedió a
solicitar su consentimiento en esta investigación. Se observó por varias
ocasiones sus intervenciones \emph{online} y sus redacciones en los
chats, cuando referían a la lengua que aprenden. Conocían de la
presencia del investigador, con quien también pudieron interactuar en
ciertas oportunidades al ahondar en sus aprendizajes y opiniones,
especialmente sobre el tópico de indagación, en sincronía. El poder
entrevistarlos y ver sus valoraciones, además, en sesiones
individualizadas posteriores, a través de una plataforma como Zoom o
Google Meet, sobre los idiomas permite conocer las maneras en que
posicionan las lenguas. Algunos establecen que el inglés es una lengua
que se aprende fácilmente porque cuenta con herramientas digitales de
acceso para el aprendizaje y atrae en múltiples sentidos, dado que es
considerada un instrumento para encontrar trabajo u obtener un mejor
estatus social, en un mundo globalizado y tecnológico, mientras que las
lenguas originarias aparentemente se aprecian de que carecen de estos
atractivos. Cabe señalar que el investigador también era estudiante del
curso de lengua extranjera, por lo que era considerado un integrante
más, por lo que la observación participante se facilitó. Los estudiantes
en general fueron grabados y sus discursos se han ido transcribiendo
para clasificar y determinar las ideologías lingüísticas que están
detrás de sus opiniones. De manera asincrónica, se pidió que completen
algunas preguntas relacionadas con la indagación, especialmente para
recabar sus opiniones sobre la proliferación de subtitulados que se dan
en inglés frente a cualquier otra lengua. La investigación viene
complementándose con entrevistas y observaciones de estudiantes de
traducción. Esto permite explorar aún más las ideologías lingüísticas
sobre la supuesta utilidad de las lenguas que tienen otro grupo de
estudiantes relacionados con la especialidad de idiomas. En términos
contrastivos, por el momento, se viene encontrando similitudes en la
forma en que se sobrevalora el inglés frente a lenguas originarias en la
traducción de series y películas. Este tipo de trabajos se inserta
dentro de las netnografías sociolingüísticas. En general, el
investigador se presenta y participa, garantiza la confiabilidad de los
informantes, identifica y estudia al grupo, ofrece conocimientos
generados por la comunidad. Recordemos que \textcite{kozinets2002} sugiere que
se sigan siempre los principios básicos en la netnografía.
\section{Conclusiones}\label{sec-conclusiones}

Se ha presentado una descripción de experiencias y necesidades de PeSD
en la búsqueda de su inserción laboral en dos comunas de Chile. A partir
de las entrevistas efectuadas fue posible trazar una ruta que identifica
experiencias de uso, situaciones de brecha digital, posibilidades de
alfabetización, de acceso al mundo laboral, así como también de
diagnosticar las políticas de accesibilidad.

Aunque todas las personas entrevistadas usan tecnología, esas
experiencias de uso generan limitaciones que se entrelazan con las
brechas digitales existentes. Las causas se vinculan con condiciones
estructurales y sociales, es decir, con las decisiones legales,
políticas, económicas que determinan el diseño y la implementación
social de tecnologías. Desde esta perspectiva hay que considerar el
diseño de dispositivos y aplicaciones, así como su implementación
social, es decir, la infraestructura por medio de la cual se ponen en
funcionamiento y que las orienta hacia un uso social que reduce
considerablemente las posibilidades de usar las tecnologías para el
desarrollo de capacidades laborales.

Las causas de las brechas estarían vinculadas, además, con una serie de
características socioculturales que hacen menos factible el
aprovechamiento de los beneficios tecnológicos. Entre las más relevantes
pueden identificarse la falta de educación formal, la situación de
pobreza, la discriminación por edad o género, la ausencia/presencia de
vínculos sociales, la situación de ruralidad y las condiciones de salud.
Es decir, una brecha digital con causas multidimensionales.

Pueden considerarse acá la falta de regulación respecto a la
implementación de políticas de accesibilidad, la orientación excesiva de
las tecnologías a públicos que pueden consumir productos de alto costo,
o las decisiones sobre la construcción de infraestructuras tecnológicas
(como la presencia o ausencia de conectividad en zonas rurales o en
poblaciones con mayor pobreza y desigualdad).

Respecto a las consecuencias, estas se vinculan fundamentalmente con las
restricciones a las posibilidades de usar la tecnología, y a la
generación de posibilidades de empleabilidad. Es decir, que la capacidad
de uso de tecnologías se ve afectada por la educación, la situación
socioeconómica, la edad, el género, la ausencia/presencia de vínculos
sociales, la ruralidad y la salud, cuestiones a las que se suma la
situación de discapacidad.

Pero el aspecto más relevante de las consecuencias es el para qué se usa
la tecnología, y ello, no depende exclusivamente de las personas
usuarias, sino del contexto social donde las tecnologías se implementan,
y las posibilidades de integración social que ahí se ponen de
manifiesto. En este sentido, hay un primer nivel de implementación
social de tecnologías donde se espera un uso en cuestiones funcionales
como prender o apagar un dispositivo, hacer una búsqueda de información,
o el uso básico de redes sociales (como, por ejemplo, enviar un audio).
Las brechas digitales emergen, por tanto, con relación a cuestiones más
complejas como el trabajo y el consumo: por ejemplo, el uso de
aplicaciones ofimáticas específicas o la realización de transacciones
como transferencias o una compra en línea. Podría considerarse todavía
un nivel mayor de complejidad que se refiere a los límites para la
creación de páginas web o aplicaciones digitales, esto es, la capacidad
de afectar los procesos de diseño e implementación.

Ello implica la detección y preocupación por la definición legal y de
política en relación con la accesibilidad digital; lo que, a su vez,
permite caracterizar las diversas brechas digitales existentes, y las
necesidades de alfabetización. Sin embargo, la superación de las brechas
digitales no se resuelve de forma exclusiva con la formación en
competencias digitales, sino con transformaciones mucho más profundas
que implican considerar tanto cambios de política como culturales, que
abran el camino hacia la inclusión digital en el mundo del trabajo.

En este sentido, la construcción de las condiciones de accesibilidad
digital no puede resolverse exclusivamente con un contenido legal, sino
con un proceso que involucra a los organismos públicos, al sector
productivo (tanto como creadores de tecnología, como en su rol de
empleadores), y a las propias PeSD.

En términos de política, los datos de las entrevistas plantean también
oportunidades de pensar nuevas políticas de accesibilidad específicas
que posibiliten la integración más efectiva de PeSD al mundo laboral.
Considerando una oferta de alfabetización para el trabajo más amplia,
que considere las experiencias de uso de tecnologías que ya poseen las
personas (ver videos de YouTube o de canales de Facebook) como un primer
peldaño en un proceso gradual de adquisición de competencias digitales.

Frente a una tendencia de uso de tecnologías individual, las políticas
de accesibilidad digital y de alfabetización pueden enfatizar la
dimensión comunitaria que ha sido valorada por las personas
entrevistadas. No se trata únicamente de usar tecnología, sino del
sentido o de los intereses que las tecnologías buscan potenciar, en este
caso, la construcción de capacidades de liderazgo y de identificación
con la comunidad.


\section{Agradecimientos}
El autor agradece a los pares ciego por sus alcances a la investigación.

\section{Financiamiento}
La presente investigación es un resultado del “Concurso de Proyectos de Investigación con recursos no monetarios 2023 para Grupos de Investigación” como parte del grupo de investigación Lenguas y Filosofías del Perú (LFP), de la Facultad de Letras y Ciencias Humanas de la Universidad Nacional Mayor de San Marcos – RD 001050-2023-D-FLCH/UNMSM.


\printbibliography\label{sec-bib}

\end{document}
