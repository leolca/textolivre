\documentclass[spanish]{textolivre}

% metadata
\journalname{Texto Livre}
\thevolume{18}
%\thenumber{1} % old template
\theyear{2025}
\receiveddate{\DTMdisplaydate{2024}{7}{15}{-1}}
\accepteddate{\DTMdisplaydate{2024}{8}{20}{-1}}
\publisheddate{\today}
\corrauthor{Marco Lovón}
\articledoi{10.1590/1983-3652.2025.46890}
%\articleid{NNNN} % if the article ID is not the last 5 numbers of its DOI, provide it using \articleid{} commmand 
% list of available sesscions in the journal: articles, dossier, reports, essays, reviews, interviews, editorial
\articlesessionname{articles}
\runningauthor{Marco Lovón}
%\editorname{Leonardo Araújo} % old template
\sectioneditorname{Daniervelin Pereira}
\layouteditorname{João Mesquita}

\title{La netnografía en la investigación lingüística}
\othertitle{A netnografia na investigação linguística}
\othertitle{Netnography in linguistic research}


\author[1]{Marco Lovón~\orcid{0000-0002-9182-6072}\thanks{Email: \href{mailto:mlovonc@unmsm.edu.pe}{mlovonc@unmsm.edu.pe}}}
\affil[1]{Universidad Nacional Mayor de San Marcos, Lima, Perú.}

\addbibresource{article.bib}

\begin{document}
\maketitle
\begin{polyabstract}

\begin{spanish}
\begin{abstract}
 La netnografía se emplea estratégicamente en diferentes investigaciones sociales, económicas y culturales. En el campo de la lingüística puede contribuir con el análisis de las identidades e interacciones sociales en las que el lenguaje tiene presencia en el ciberespacio. El propósito de este estudio es reflexionar sobre la utilidad práctica de la netnografía en la investigación lingüística. Para ello, se propone el rol que tiene el lingüista como netnógrafo y se sugieren un conjunto de áreas y líneas de investigación que le concierne en la gestación de una netnografía lingüística. El trabajo concluye que la netnografía es un método cualitativo que permite a los lingüistas interpretar la vida social de las comunidades virtuales, principalmente en relación con el lenguaje en uso y las estructuras sociales.  

\keywords{Etnografía \sep Netnografía \sep Lingüística \sep Trabajo de Campo \sep Transdisciplina}
\end{abstract}
\end{spanish}

\begin{portuguese}
\begin{abstract}
    A netnografia é usada estrategicamente em diferentes pesquisas sociais, econômicas e culturais. No campo da linguística, ela pode contribuir para a análise de identidades e interações sociais nas quais a linguagem está presente no ciberespaço. O objetivo deste estudo é refletir sobre a utilidade prática da netnografia na pesquisa linguística. Para isso, propõe o papel do linguista como netnógrafo e sugere um conjunto de áreas e linhas de pesquisa que lhe dizem respeito na gestação de uma netnografia linguística. O artigo conclui que a netnografia é um método qualitativo que permite aos linguistas interpretar a vida social das comunidades virtuais, principalmente em relação à linguagem em uso e às estruturas sociais.      
\keywords{Etnografia \sep Netnografia \sep Linguística \sep Trabalho de campo \sep Transdisciplina}
\end{abstract}
\end{portuguese}

\begin{english}
\begin{abstract}
    Netnography is used strategically in different social, economic and cultural research. In the field of linguistics it can contribute to the analysis of identities and social interactions in which language is present in cyberspace. The purpose of this study is to reflect on the practical usefulness of netnography in linguistic research. To this end, it proposes the role of the linguist as netnographer and suggests a set of areas and lines of research that concern him/her in the gestation of a linguistic netnography. The paper concludes that netnography is a qualitative method that allows linguists to interpret the social life of virtual communities, mainly in relation to language in use and social structures.

\keywords{Ethnography \sep Netnography \sep Linguistics \sep Field work \sep Transdiscipline}
\end{abstract}
\end{english}
\end{polyabstract}

\section{Introdução}\label{sec-intro}

O processo de escrita é uma das atividades mais complexas que o ser
humano é capaz de realizar, em razão de vários fatores, a exemplo das
exigências feitas à memória e ao raciocínio durante o momento de
produção \cite{garcez2020}. São inúmeros os conhecimentos e habilidades que
precisam ser articulados e harmonizados para que o texto tome forma.
Tendo em vista esse seu caráter complexo, ainda são recorrentes falsas
crenças sobre a produção textual, que levam pessoas a acreditarem que
podem dominá-la a partir de ``dicas'' desvinculadas de seu contexto de
produção.

A ideia de que fórmulas pré-fabricadas e ``dicas'' isoladas são métodos
cabíveis no ensino de produção textual, apenas negligencia as etapas
necessárias que caracterizam um texto adequado conforme seu contexto de
produção \cite{garcez2020}. O processo de escrita é uma atividade que
carece de idas e vindas, pois deve admitir três grandes momentos que se
intercalam e devem ser compreendidos de modo indissociável: o do
planejamento, o da escrita propriamente dita e o da revisão \cite{antunes2005}. Enquadrar esse processo em uma perspectiva prescrita e linear
pode resultar em estudantes frustrados pela construção de textos
truncados e artificiais.

Essa realidade se agrava quando observamos o cenário acadêmico, em que
as exigências com relação a produções textuais se intensificam. As
expectativas quanto a essas produções não se limitam à utilização
adequada da norma-padrão ou a vocabulários específicos; expandem-se para
aspectos implícitos de produção que precisam ser considerados, como o
que pode ser dito, por quem, de que forma, sob que ponto de vista e
fundamentado em qual autor \cite{oliveira2024}.

Pensando na comunidade discursiva acadêmica, uma das produções textuais
mais demandadas em cursos de graduação da área de humanas é o artigo
acadêmico \cite{motta-roth2010}. Por ser um dos principais
veículos de divulgação científica, a circulação desse gênero na academia
é incontornável, sendo bastante exigido o seu consumo e produção por
parte de professores, estudantes e pesquisadores. Embora seja uma
produção essencialmente ligada ao meio universitário, sua feitura é
quase sempre exigida sem antes ser ensinada. Esse fato pode levar os
alunos a somarem suas dificuldades com o processo de escrita à
dificuldade de produzir um texto do qual desconhecem seu contexto de
produção, estrutura composicional e outras ``dimensões escondidas''
\cite{street2010} que perpassam a construção de um artigo.

Ao exigir do autor capacidade de síntese, descrição, análise e
argumentação, utilizando-se das convenções próprias à determinada área,
o artigo contempla informações geradas em pesquisas a serem submetidas a
apreciações públicas \cite{motta-roth2010}. Sua relevância remonta
à popularização da ciência que, por sua vez, possui a potencialidade de
descrever fenômenos sociais e até mesmo gerar algum impacto benéfico ao
público em geral.

A partir dessas pontuações, torna-se clara a importância de produzir
artigos acadêmicos e a responsabilidade do seu produtor de popularizar
os conhecimentos produzidos na esfera acadêmica. Quando essa tarefa de
produção precisa ser desenvolvida por graduandos e estes normalmente não
recebem orientação para tal, muitas vezes, recorrem a materiais digitais
sobre esse assunto, pois lhes propiciam as mais variadas estratégias de
ensino de acordo com o ritmo e as preferências do estudante \cite{falkembach2005}. Um fator que pode justificar essa recorrência é a facilidade de
acesso a plataformas digitais, que disponibilizam, na maioria das vezes
de forma gratuita, conteúdos digitais educacionais. Antigamente, os
estudantes consultavam manuais impressos que ensinavam a como produzir
textos acadêmicos, hoje, frente aos recursos tecnológicos, os locais de
aprendizagem se ampliam para a cibercultura. Como cibercultura,
compreendem-se vários ambientes da esfera digital que abrigam
informações, até mesmo os que simulam uma sala de aula a partir de
vídeos \cite{martins2018,rocha2005}.

Tendo a cibercultura se tornado uma potencializadora de novas abordagens
educativas, deve-se averiguar sua eficiência enquanto ferramenta de
ensino, a forma como se ensina determinados conteúdos, a exemplo da
produção textual de artigo acadêmico e seus aspectos constitutivos, foco
do presente estudo. Nesse sentido, traçamos dois objetivos para este
trabalho: identificar e analisar objetos de ensino explorados em
videoaulas sobre artigo acadêmico publicadas na plataforma YouTube.

Para tanto, organizamos este artigo em 5 seções, a saber: esta
introdução, contendo uma contextualização inicial sobre o objeto de
investigação da pesquisa, a problemática que o envolve e os objetivos
delineados; o embasamento teórico, no qual apresentamos os pressupostos
que fundamentam o estudo --- as práticas de ensino de Língua Portuguesa
em contexto didático-digital \cite{laurentino2023}, o artigo
acadêmico \cite{motta-roth2010}, as etapas de produção textual
(Antunes, 2003) e os objetos de ensino \cite{linodearaujo2014}; a
metodologia, na qual explicitamos a abordagem e o tipo de pesquisa, bem
como os procedimentos de coleta e análise de dados; os resultados,
contendo a exploração dos objetos de ensino contemplados nas videoaulas
sobre ensino de produção de artigo acadêmico; as considerações finais,
nas quais sinalizamos algumas implicações advindas dos resultados
alcançados.
\section{Netnografía y lingüística
}\label{sec-netnografiaylinguisica}

La netnografía es un tipo de metodología de investigación cualitativa
que se ocupa de estudiar las comunidades y de las culturas emergentes en
el espacio \emph{online} o ciberespacio, donde la información se produce
de manera natural. Según \textcite{kozinets1998}, la netnografía es un enfoque
etnográfico que se interesa por las comunidades en línea. Con este
método, se puede comprender la vida social de los sujetos y las
comunidades virtuales en la sociedad contemporánea y en un periodo de
tiempo. La netnografía se ha ido empleando para estudiar las diferentes
percepciones y preferencias de consumo de los productos, así como
también los servicios que las comunidades en el entorno virtual demandan
en sus relaciones de interacción \cite{casas-romeo2014,moralesvasquez2020,xun2010netnography}. Y, más recientemente, como método de
investigación posibilita el abordaje de lo que acontece en las
comunidades virtuales \cite[p. 9]{turpo2008}. Dichas comunidades comparten
valores, creencias y comportamientos, y, además, cada vez aumentan en la
red con el fin de crear relaciones con otros usuarios, mantenerse
informadas o consumir algún tipo de información. Se tratan de
comunidades mediadas por un ordenador. \textcite[p. 147]{martinez2017etnografia} sostienen que la etnografía no convencional ha ido entendiendo que
el ciberespacio es una realidad donde se construyen significados, se
generan identidades y se establecen agrupaciones relativamente estables
y con intereses que se comparten. Este método permite, asimismo, pensar
en los efectos y la relación que tiene el lenguaje en la vida social de
los cibernautas, por lo que interesa pensar sobre la utilidad práctica
que tiene la netnografía como instrumento de la indagación en la
especialidad de la lingüística.

Desde los trabajos de \textcite{herring_2004}, con los trabajos hechos sobre la
Comunicación-Mediada-por-Computadoras (CMC), la lingüística ha prestado
más atención a los medios digitales. Desde entonces, se han desarrollado
investigaciones sobre los discursos que se generan en la web, así como
las reflexiones metalingüísticas que poseen los cibernautas sobre sus
lenguas \cite{tannen2013}. Los lingüistas se sumergen en el
análisis del uso del lenguaje en los videojuegos, sitios de video,
plataformas, foros en línea \cite{lovón-lóvon2022} y medios sociales u
otros de comunicación virtual, como Flickr, Facebook, Twitter, WhatsApp,
YouTube, Youku, Paideia o Blackboard Collaborate. Los investigadores de
netnografías se acercan a este método para conocer la vida social en
redes sociales, plataformas virtuales o páginas web que manejan los
cibernautas. Según \textcite{putrikusama2016}, ``Researchers can make
themselves familiarised with the netnographic approach to help them
study chats such as Skype, WhatsApp Messenger, and Line; blogs such as
Tumblr, Blogspot, and Wordpress; even social networking sites such as
Facebook, Twitter, and Path''. Con la netnografía, no solo se estudian
espacios virtuales recientes, sino también aquellos que ya no se emplean
o tienen pocos usuarios, tal es el caso de hi5. Así como los lingüistas
estudian lenguas en peligro de extinción o muertes lingüísticas
\cite{ramirez-cruz2021}, pueden también en la virtualidad
analizar comunidades de habla en espacios que existieron. El interés por
el mundo digital se centra en varias facetas: en los géneros discursivos
digitales, los estilos y la creatividad de los internautas, las
prácticas y los recursos multilingüísticos en la comunicación digital,
el desarrollo de literacidades, la redacción y el aprendizaje de
lenguas, la construcción de identidades y amistades en torno al
lenguaje, la comunicación virtual pública y sus consecuencias en la
publicación de textos, el rol de la lengua en la multimodalidad, el
desarrollo de habilidades lectoras\cite{georgakopoulou2016}.
Aunque, según \cite{kulavuz-onal2015}, la netnografía no es tan popular en
el campo de la educación ni en la lingüística aplicada o el aprendizaje
de idiomas asistido por ordenador.

Si bien distintas investigaciones se centran en el análisis de la
ortografía y la escritura de los mensajes de textos, la variación
lingüística o el sistema lingüístico, los nuevos estudios en el mundo
digital ponen peso al estudio del cibernauta como agente social. Dicho
de otro modo, la agenda de investigación ha cambiado, por lo que se
amerita conocer cómo piensa, siente, percibe, se transforma ---dado que
se performan identidades o despliegan nuevas constantemente--- y se
conduce en su entorno digital. \textcite{varis2019digital} señalan que la
etnografía digital se interesa por analizar las formas en que las
personas usan el lenguaje, interactúan entre sí, emplean discursos y
construyen comunidades, colectivos, conocimientos e identidades, a
través de las tecnologías digitales al mismo tiempo que se encuentran
bajo su influencia. Precisamente, la netnografía analiza los comentarios
y las conversaciones que se generan entre los participantes internautas,
así como sus formas de pensar y sus mitos a través del internet. \textcite[p. 41]{putrikusama2016} precisa que los lingüistas también pueden desarrollar
trabajos netnográficos: ``the development of cultures, communication,
and communities online can also encourage linguistic researchers to
conduct their research online'', y ello implica su acercamiento al mundo
virtual. Los lingüistas que se acercan al análisis del lenguaje en el
ciberespacio lo hacen, porque conocen este mundo o tienen interés por lo
que sucede en este espacio virtual. Entre múltiples comunidades
virtuales, los grupos indígenas y de lenguas originarias son importantes
de ser estudiados para explorar sus interacciones y demandas.

En relación con el campo de indagación, según \textcite{kozinets2002}, algunas
comunidades virtuales se presentan en los foros y los blogs de
discusión, las salas multiusuario, las listas de servidores, las páginas
Web independientes o los boletines electrónicos. En ese sentido, los
campos de investigación netnográficas para los lingüistas son las salas
de chat, los espacios de juego de videojuegos en red, los sitios de
campo en línea, como los tablones de anuncios o los foros, los servicios
de listas y las páginas web, los sitios de redes sociales, los wikis,
los blogs, los videoblogs, los microblogs. Teniendo en cuenta la
locación, la netnografía permite al investigador comprender el
funcionamiento de una comunidad que está distribuida en uno o múltiples
sitios y lugares de Internet, así como la interconexión entre sus
sitios, sus actividades y sus tecnologías. Debe repararse que para la
netnografía, los lugares de estudio surgen, además, en el proceso de
compromiso reflexivo del etnógrafo con los entornos digitales \cite{varis2019digital}. Por tanto, el lingüista entra en un compromiso con el
ciberespacio.

Asimismo, en términos de modalidades, a la netnografía le interesan los
espacios de comunicación sincrónica, como asincrónica \cite{kozinets2010}.
Con la netnografía incluso se puede obtener datos de los internautas
asincrónicamente cuando los obtiene de correos electrónicos o crea foros
de investigación específicos, o sincrónicamente por medio de plataformas
digitales como Zoom, Meet o Skype. A los lingüistas puede llamarle la
atención la comunicación por correos electrónicos entre personas e
instituciones, como de aquellas que se inician en la escritura virtual.
Cabe señalar que la desconectividad puede implicar una desventaja sobre
todo si se generan distancias entre la investigación y las personas
involucradas en el estudio.

Como se expone, la netnografía se involucra en la lingüística al
emplearse como método de investigación. Las metodologías en lingüística
son variadas y la netnografía contribuye con sus indagaciones en el
ciberespacio. La relación método y ciencia se unen para poder investigar
hechos, fenómenos y vidas en un espacio virtual llamativo.

Por último, debe señalarse que los resultados de las investigaciones
lingüísticas con netnografías deben ser valorados acorde con los
objetivos y la naturaleza planteadas por los investigadores. En tales
resultados se consideran las fortalezas, las limitaciones, las lecturas
e interpretaciones de los datos.
\section{El lingüista como netnógrafo
}\label{sec-ellinguistacomonetnografo}

Los investigadores de netnografías son llamados netnógrafos. Como
especialistas reconocen que las comunicaciones, las creencias y las
actuaciones de los participantes en línea cobran sentido. Su campo de
investigación se encuentra dentro de la pantalla. Los netnógrafos se
involucran en la observación participante en línea al conectarse a la
comunidad a través del ordenador. Como investigadores, toman notas de
apuntes sobre lo que observan y también preguntan en línea a la
comunidad, mientras recogen sus impresiones e interrogantes sobre la
vida social virtual. Sus notas sirven para ir reuniendo sus
interpretaciones. Los netnógrafos observan, por ejemplo, el discurso
textual \cite{kozinets2002}. Sin embargo, existen trabajos que se llevan a
cabo con una observación no participante, o sin entrevistas, o sin tomar
notas, aunque de preferencia en la netnografía hay que sumergirse en la
comunidad y participar en ella como un miembro. Asimismo, los
netnógrafos se inscriben en una comunidad en línea, participan en
actividades sincrónicas y asincrónicas seleccionadas, entrevistan a
miembros clave de la comunidad, archivan documentos relevantes para
apoyar sus entrevistas y notas de campo, capturan pantallazos, imágenes
o audios importantes que enriquecen su trabajo. Y pueden llegar en el
contexto en línea a más internautas a través de la técnica de la bola de
nieve \cite{Ardèvol_Lanzeni_2014}. Los analistas pueden basarse en los
datos textuales, imágenes, videos, sonidos.

El lingüista es netnógrafo cuando indaga aspectos del lenguaje y usa el
método de la netnografía para observar y entrevistar comunidades
digitales. Él usa notas o apuntes de campo sobre su trabajo realizado en
el ciberespacio. En su aplicación, es participante de las interacciones
e interpreta la vida social en que el lenguaje construye prácticas
sociales. En relación con ello, en la interpretación de información, aun
cuando los investigadores empleen \emph{software} de análisis de datos o
herramientas especializadas y se utilicen para establecer ideas lógicas
con sus interpretaciones, el análisis y la interpretación responde
finalmente a lo que concluye el investigador, pues es él quien maneja
los marcos conceptuales, conoce las visiones del mundo, abstrae
categorías, recolecta información directamente, a partir de trabajos de
campo con la comunidad e incluso previos, compara e integra información,
y reinterpreta los resultados acorde con sus progresos científicos y
humanísticos \cite[p. 10]{sanchez2017}. El lingüista es un
especialista preparado para las indagaciones lingüísticas \emph{face to
face} y virtual. Además, sabe que las comunidades virtuales que
investiga pueden estar en un espacio específico o en varios, pues se
tratan de comunidades desterritorializadas. Según \textcite[p. 4]{kulavuz-onalvásquez2013}, ``in netnography, field-sites can be diverse, because an
online community can exist in a single site or multiple sites''.

Como investigador se familiariza con los códigos lingüísticos y éticos
\cite{Tagg2017}. Así, al realizar netnografías debe
considerar el anonimato de los participantes, en el caso de que la
información que use la va a utilizar en alguna publicación o algunas
conferencias públicas o en cualquier otra forma de difusión. Lo más
recomendable es que consulte a sus interlocutores si puede usar sus
identificaciones o si prefieren ser ocultados en algún anonimato
\cite{boellstorff2012ethnography}. Sobre su responsabilidad en la
investigación, \textcite{kozinets2010} considera que se debe realizar siempre
netnografía éticas. Por eso, propone que los investigadores pidan
permiso; obtengan el consentimiento informado; citen, anonimicen o
acrediten a los participantes en la investigación; y tengan en cuenta
cualquier consideración legal relacionada con la investigación. Cabe
señalar que evita los prejuicios en su observación y análisis, si
encuentra por ejemplo faltas ortográficas en las redacciones o inclusive
agresiones verbales entre los usuarios, entonces toma el material de
estudio como lo halló, lo procesa en relación con las técnicas de
investigación correspondiente e interpreta con los lentes teóricos que
maneja. En relación con cuestionarios y preguntas de entrevistas, si se
aplican en el momento a los sujetos de estudio o se les entrega antes,
como podría ocurrir en algún proyecto, estos pasan por validaciones o
juicio de expertos.

Asimismo, como netnógrafo en comunidades virtuales en que participa se
revela como investigador en el sitio web que investiga, por lo que
informa sobre sus objetivos, los antecedentes y los detalles del
estudio. En su participación en línea, cuida no tener un papel
dominante, sino más bien participa de la misma manera que los
cibernautas reales lo realizan de forma rutinaria. El lingüista como
netnógrafo en sus entrevistas puede mantener una conversación
multimodal, no solo plantea preguntas oralmente, sino también, por
ejemplo, puede usar una presentación de diapositivas en una pizarra
virtual o presentar imágenes que fomenten la interacción en línea, como
puede ocurrir en una videollamada o una sala Zoom. De esta manera,
recoge datos para analizar. Como netnógrafo establece conexiones en la
comunidad y sigue los acontecimientos y las actividades. A través de las
narrativas digitales puede conocer modos de vida \textcite{londoñopalacio2012}. Como
el netnógrafo no está copresente en un espacio físico con la comunidad,
debe determinar su campo de estudio en función de las interacciones
discursivas que se generan entre los miembros. El netnógrafo lingüista
es quien decide qué interacción incluir y cuál no \cite{kulavuz-onal2015}.
Sin embargo, debe tener en cuenta que su netnografía depende de su
acceso a la tecnología que utiliza la comunidad en línea, como la
conexión fiable a internet. El desconectarse de una comunicación por un
tiempo corto o prolongado puede impactar en el estudio. Si bien algunas
investigaciones \emph{online} son sincrónicas, los investigadores por
diversos motivos ajenos a su indagación cesan de investigar los
fenómenos que buscaba comprender. El compromiso del investigador con la
comunidad en línea es variable: puede durar unos meses o unos años. Solo
su desvinculación con la comunidad determina la salida del campo.

En el caso de las entrevistas, como netnógrafo, puede grabar videos y
usar \emph{softwares} de captura de audio en sus ordenadores. Hasta el
momento, Facebook y Skype son medios útiles para estos casos. Aunque
debe considerar que una de las limitaciones es la poca posibilidad de
capturar el lenguaje no verbal, pues la mayoría de las netnografías
pueden correr el riesgo de no recoger la información tan igual que las
videograbaciones presenciales; además, la mayoría se basan más en la
data textual de los mensajes electrónicos, que incluso no requieren que
alguna cámara sea encendida, basta con escuchar al sujeto entrevistado
con un micrófono \cite{noveli2010}. Al respecto, como netnógrafo debe
considerar que la presencia de la cámara condiciona a los participantes
a sobreactuar o sentirse cohibidos \cite[p. 116]{boellstorff2012ethnography}, por lo que debe encontrar o construir un momento o ambiente de
familiarización que favorezca la interacción. Al respecto, el lingüista
requiere optar por protocolos de investigación adecuados.

Como se entiende, sus trabajos sobre el uso del lenguaje deben tener en
cuenta las normas sociolingüísticas, los valores lingüístico-culturales,
la manifestación de identidades intralingüísticas, las relaciones
personales interlingüísticas y las prácticas cotidianas e
institucionales lingüísticas de los internautas en el entorno
electrónico.
\section{Áreas y líneas de investigación}\label{sec-areasylineasdeinvestigacion}

Teniendo en cuenta la diversidad de posibilidades, las preocupaciones de
las netnografías lingüísticas pueden resumirse en netnopragmática,
netnosociolingüística y netnolingüística aplicada, las cuales mantienen
diálogo entre sí, pues lo social, lo cultural, lo cognitivo y lo
lingüístico están imbricados entre sí en el análisis de actores,
comunidades y acciones virtuales. Ha de precisarse que una netnografía
lingüística es una investigación sobre un tema lingüístico que emplea el
método netnográfico. En ese orden, la netnopragmática es la búsqueda de
fenómenos estudiados por la pragmática utilizando la netnografía; la
netnosociolingüística es la investigación acerca de aspectos en que la
lengua y la sociedad se unen manejando la metodología netnográfica; la
netnolingüística aplicada es la indagación en que interviene la
lingüística aplicada valiéndose de la netnografía. A continuación, se
muestran las tres áreas con líneas de investigación que pueden ayudar al
lingüista netnógrafo en la realización de sus pesquisas:

\subsection{Netnopragmática}\label{sub-sec-netnopragmática}

\textcite{putrikusama2016} sostiene que la netnopragmática es el estudio pragmático
sobre el uso del lenguaje por parte de los miembros de la comunidad
\emph{online} y propone la realización de estudios netnopragmáticos,
como el análisis de las implicaturas y las comunicaciones entre los
miembros del mismo grupo de Facebook. Para él, la netnopragmática
considera que los datos en línea son actos sociales, que el mundo en
línea funciona como un entorno social y que la búsqueda del significado
de los actos está en relación con el contexto que precede o sigue. Y
señala que la netnopragmática pueden estudiar datos visuales como el
diseño de páginas y mensajes, los emoticonos, los colores y tipo de
letra, las imágenes y fotografías. El mismo autor indica que en la
observación de los actos de habla de los miembros de una cultura en el
ciberespacio hay que tener en cuenta los cumplidos, las invitaciones,
las disculpas, las negativas y las ofertas, así como los lazos de
relación entre los miembros, como el grado de solidaridad, el poder
relativo y las imposiciones.

Al respecto, se puede investigar la cortesía en Facebook, por ejemplo,
en el mantenimiento de las relaciones laborales. Algunos estudios han
evaluado las estrategias lingüísticas que se usan para establecer
amistades, realizar posteos, replicar información, saludar por
cumpleaños, lo que finalmente se entiende como rituales en la web \cite{west2013}.

En esta línea, pueden realizarse trabajos sobre los usos de figuras
retóricas en la comunicación virtual, como la ironía, el humor o la
metáfora. \textcite{taiwo2016analyzing} han encontrado relaciones entre la
pragmática y la comunicación virtual. \textcite{coker2016} examinan la
retórica del humor visual en Facebook y descubren que este es
gubernativo, institucional, cultural y grotesco, y se emplea para
ridiculizar los problemas de la sociedad de forma abierta o encubierta.
Ellos recurren al trabajo netnográfico para interpretar que el humor
visual suele estar moldeado por las visiones del mundo, los prejuicios y
los intereses de sus usuarios. Ellos señalan que metodológicamente
fueron observadores y participantes muy activos en la publicación y
lectura de comentarios en línea por casi siete años. Con su trabajo los
autores concluyen que el humor no es un discurso neutral, sino más bien
legitima y forma una serie de valores.

Por otro lado, la netnopragmática puede servir para estudiar las
interacciones lingüísticas que se dan en torno al consumo de productos
en línea. \textcite{septianasari2021} utilizaron el método
netnográfico con un enfoque pragmático para recoger información sobre el
consumo de publicidad cosmética y analizar los actos de habla empleados
para persuadir por medio del marketing \emph{online} en Indonesia. Esto
nos recuerda que la netnografía se ha utilizado ampliamente en la
investigación de marketing. Para \textcite{casas-romeo2014}, la
netnografía es una herramienta que se aprovecha para recoger información
del mundo virtual para la toma de decisiones de marketing. Los estudios
netnográficos han cobrado importancia en el campo de la administración
en la búsqueda de los significados que pueden servir a las empresas para
lograr la comprensión, de lo que los clientes dicen por escrito en el
espacio virtual y fomentan el desarrollo de etnografías de la
comunicación \cite{freitas2012netnografia}.

\subsection{Netnosociolingüística}\label{sub-sec-netnosociolingüística}

La netnosociolingüística puede estudiar la construcción de identidades y
relaciones sociales en el ciberespacio. Le interesa las preocupaciones
de las identidades en curso. Se puede analizar la identidad digitalizada
de un grupo social que se negocia en línea a través de la actuación
textual y el metalenguaje de los participantes \cite{campbell2006}, pero
también las identidades que se cambian por seguridad en espacios
virtuales, especialmente en el campo de la política, la migración o la
religión, por ejemplo. Las actuaciones de identidad en los entornos en
línea no siempre están disociadas de las identidades fuera de línea,
pues tienden a ser continuas. Sin embargo, las identidades \emph{online}
no coinciden siempre con las \emph{offline}, pues en el mundo virtual se
performan otras identidades que el agente social configura, muchas veces
para protegerse, como sucede con las comunidades LGTBIQ+, que por
seguridad pueden recrear identidades \emph{online}, incluso en
aplicaciones de búsqueda de pareja. A los netnógrafos puede llamarle la
atención las aplicaciones como Grindr, Moovz, Growlr, Scissr, Scruff,
LGBTQutie, Chappy, Wapo y Wapa, Tinder. Los investigadores que pueden
integrar la relación entre las identidades \emph{online} y
\emph{offline} de los miembros lo logran al reunirse con los informantes
cara a cara.

Asimismo, desde la netnosociolingüística, se puede investigar los
lenguajes y discursos de ciberodio que se generan en la web, las
ideologías lingüísticas sobre ortografía de los internautas, las
ideologías que se encuentran tras la enseñanza de idiomas en el virtual,
las prácticas de vigilancia sobre la normativa lingüística y las
conductas sociales que mantienen diversos grupos con discursos
normativistas, así como el activismo lingüístico que busca deconstruir
prácticas de dominación social. Según \textcite[p. 57]{londoñopalacio2012}, ``Las redes
de comunicación tienen un papel definitivo en las narrativas digitales,
pues de manera explícita o implícita modifican entornos, hábitos y
prácticas sociales''. Dado que las literacidades muestran la relación
entre el discurso y el poder \cite{bloome2005discourse,jimenez2015}, a través del estudio netnográfico se puede analizar las
literacidades virtuales como prácticas sociales de relaciones de poder,
entre actores que se perciben con autoridad frente a los que no. La
netnosociolingüística debe ayudar a develar que los entornos
electrónicos no proporcionan neutralidad, sino más bien son espacios de
relaciones de poder en que se usa el lenguaje. Los análisis pueden estar
sesgados por las historias personales o por la ausencia de voces no
escuchadas \cite{noveli2010}. A este enfoque también le interesa las
literacidades no hegemónicas que muchas veces se ven afectadas por
prácticas de poder que evitan la aceptación y generación de
literacidades vernáculas. Adicionalmente, estudian las desigualdades
sociales y las posibilidades de transformación social asociadas con las
tecnologías \cite{Ardèvol_Lanzeni_2014}.

Además, el netnógrafo puede interesarse por los motivos en que los
internautas dan a sus interacciones sociales. \textcite{leppanen2009} estudian las particularidades de la acción lingüística, social y
cultural de un grupo de jóvenes finlandeses en los espacios translocales
de los nuevos medios de comunicación, y analizan la forma en que ellos
mismos dan sentido y razón a sus acciones. Los autores muestran que la
translocalidad se manifiesta, en particular, en la elección del idioma y
en la heteroglosia lingüística y estilística, la coexistencia, la mezcla
y la alternancia de diferentes lenguas, registros, géneros y estilos.
Los jóvenes seleccionan el inglés como lengua de comunicación en lugar
de su primera lengua, y emplean y combinan recursos de más de una
lengua, como el inglés con el japonés, e integran géneros y estilos de
una lengua en sus discursos.

También puede emplearse la netnografía para el análisis del discurso y
armado de corpus amplios como se ofrece con los datos en Twitter. Se
puede usar para analizar muestras de discusión pública generadas en
dicho espacio \cite{Bonilla_2022}. A propósito, debe señalarse que ``La
netnografía, como forma de investigación, hace un énfasis en la
recolección y el procesamiento de cantidades significativas de datos
empíricos para producir conocimiento al respecto'' \cite[p. 7]{Bonilla_2022}.

\subsection{Netnolingüística aplicada} \label{sub-sec-netnolingüísticaaplicada}

Desde la netnolingüística aplicada se pueden realizar trabajos sobre la
adquisición y enseñanza de lenguas nacionales y extranjeras, como
primeras y segundas lenguas, de canal oral o visual, como la lengua de
señas, de uno o grupo de usuarios en la red. \textcite{kulavuz-onal2015} señala
que con la netnografía se puede entender la cultura de las comunidades
de aprendizaje y enseñanza de idiomas en línea. El uso de las
tecnologías ha atraído la atención de muchos estudiosos en el campo del
aprendizaje de segundas lenguas, entonces a través de la observación
participante y las entrevistas en línea semiestructuradas con los
estudiantes se puede conocer sus experiencias, gustos, desventajas con
los sitios web de aprendizaje, y también las maneras en que integran la
tecnología a sus modos de estudiar. Investigar la enseñanza de idiomas
favorece a otros para que tomen decisiones de diseño de los aprendizajes
en el mundo digital. \textcite{kulavuz-onal2013} ha ido investigando los
aprendizajes de los profesores de inglés para poder enseñar con la
tecnología participativa en una comunidad de práctica en línea. Al
respecto, \textcite{kesller2021} sostienen que las netnografías se
pueden aplicar en el estudio de la adquisición de segundas lenguas; por
ejemplo, se puede analizar los hábitos formales e informales de
aprendizaje de idiomas que tienen los usuarios en chat o foros. Entre
diversas investigaciones, ellos señalan que el estudio de \textcite{isbell2018online}
ofrece una mirada hacia los hábitos informales de aprendizaje de
personas que aprenden coreano a través de chats, los que permiten
evidenciar comportamientos similares a las prácticas tradicionales del
aula al ver que los aprendices se centran en aprender la parte formal o
estructural de la lengua.

Por otro lado, \textcite{kulavuzonal2018} han sugerido analizar por
medio de netnografías las interacciones virtuales translingüísticas que
ocurren en la comunidad global de educadores de inglés como lengua
extranjera, por ejemplo. Así, observa que los participantes recurren a
sus repertorios multilingües dentro de Facebook, y que a través de
entrevistas etnográficas con los profesores y documentos en línea de su
telecolaboración, descubre que si bien el grupo se construyó
discursivamente como una zona exclusiva para emplear el inglés con el
fin de que los profesores orienten y fomenten el uso del inglés en sus
alumnos, todos los participantes rompen las instrucciones del espacio,
pues recurren a otras lenguas como el español y el árabe para diversos
propósitos, sobre todo los maestros, como en los casos de
establecimiento de solidaridad, selección de un destinatario, y el
modelado de la sensibilidad intercultural.

También, se puede estudiar las plataformas que facilitan procesos de
lectura a usuarios. Los estudios netnográficos pueden contribuir con el
análisis de los tipos de herramientas hipermedia disponibles que
permiten modificar prácticas previamente preferidas por otras formas de
leer y comprender los textos hipermedia \cite{azman2017hypermedia}. O
pueden estudiarse canales de YouTube donde se crean relatos y poemas
como casos de literacidad con subtitulados en lenguas extranjeras y
personas señantes \cite{broullón-lonzano2019}.

Incluso, se puede investigar netnográficamente el uso de software del
diccionario creado para la adquisición o el perfeccionamiento de una
lengua materna y una lengua extranjera, más aún si son solicitados por
profesores que trabajan en zonas con presencia de inmigración de otros
países y buscan que sus estudiantes logren el aprendizaje lingüístico
\cite{turrini2000}. El netnógrafo lingüista puede aprovechar
en interactuar con comunidades de consumo de diccionarios electrónicos
que cuentan con foro o retroalimentación virtual hacia la confección y
alimentación de los diccionarios.

Como se colige, a los lingüistas les puede interesar no solo la
comunicación humana mediada por el ordenador, sino también los trabajos
y aprendizajes colaborativos en el ordenador, que en última instancia
pueden generar interacciones entre los participantes del espacio
virtual.

En general, se ha visto que la netnografía se ajusta a las necesidades
de los lingüistas y los intereses que redundan en el progreso académico
científico y social.

\subsubsection{Procedimientos en una investigación netnográfica: hacia una
experiencia en trabajo}

En el período de pandemia y pospandemia, ha sido crucial las
netnografías en el análisis de varios temas. Una investigación en línea
emprendida ha sido de las ideologías lingüísticas que presentan
estudiantes universitarios sobre el inglés y las lenguas originarias del
país. Los estudiantes entrevistados y a quienes se ha observado por un
semestre académico son de la especialidad de Humanidades y llevan sus
cursos de forma virtual. Desde el tiempo en confinamiento por la
pandemia de la covid-19, la universidad impartió sus cursos en la
modalidad de educación a distancia, lo que dio paso a generar
comunidades digitales de estudiantes. En este caso, se decidió conocer
las formas en que piensan la enseñanza y el aprendizaje de una lengua
extranjera desde su profesión, incluso en entornos virtuales a los que
acceden también en tiempo de pandemia. Para ello, se procedió a
solicitar su consentimiento en esta investigación. Se observó por varias
ocasiones sus intervenciones \emph{online} y sus redacciones en los
chats, cuando referían a la lengua que aprenden. Conocían de la
presencia del investigador, con quien también pudieron interactuar en
ciertas oportunidades al ahondar en sus aprendizajes y opiniones,
especialmente sobre el tópico de indagación, en sincronía. El poder
entrevistarlos y ver sus valoraciones, además, en sesiones
individualizadas posteriores, a través de una plataforma como Zoom o
Google Meet, sobre los idiomas permite conocer las maneras en que
posicionan las lenguas. Algunos establecen que el inglés es una lengua
que se aprende fácilmente porque cuenta con herramientas digitales de
acceso para el aprendizaje y atrae en múltiples sentidos, dado que es
considerada un instrumento para encontrar trabajo u obtener un mejor
estatus social, en un mundo globalizado y tecnológico, mientras que las
lenguas originarias aparentemente se aprecian de que carecen de estos
atractivos. Cabe señalar que el investigador también era estudiante del
curso de lengua extranjera, por lo que era considerado un integrante
más, por lo que la observación participante se facilitó. Los estudiantes
en general fueron grabados y sus discursos se han ido transcribiendo
para clasificar y determinar las ideologías lingüísticas que están
detrás de sus opiniones. De manera asincrónica, se pidió que completen
algunas preguntas relacionadas con la indagación, especialmente para
recabar sus opiniones sobre la proliferación de subtitulados que se dan
en inglés frente a cualquier otra lengua. La investigación viene
complementándose con entrevistas y observaciones de estudiantes de
traducción. Esto permite explorar aún más las ideologías lingüísticas
sobre la supuesta utilidad de las lenguas que tienen otro grupo de
estudiantes relacionados con la especialidad de idiomas. En términos
contrastivos, por el momento, se viene encontrando similitudes en la
forma en que se sobrevalora el inglés frente a lenguas originarias en la
traducción de series y películas. Este tipo de trabajos se inserta
dentro de las netnografías sociolingüísticas. En general, el
investigador se presenta y participa, garantiza la confiabilidad de los
informantes, identifica y estudia al grupo, ofrece conocimientos
generados por la comunidad. Recordemos que \textcite{kozinets2002} sugiere que
se sigan siempre los principios básicos en la netnografía.
\section{Conclusiones y reflexiones finales}\label{sec-conclusionesyreflexionesfinales}

Como se expuso, la netnografía es un método que permite el desarrollo de
áreas y líneas de investigación, las cuales fomentan la constitución de
una netnografía lingüística. Si bien la etnografía lingüística se ha
conformado como un enfoque interpretativo que estudia las acciones
locales e inmediatas de los actores desde su punto de vista y que
considera que las interacciones se insertan en contextos y estructuras
sociales más amplios \cite{copland2015linguistic}, la netnografía
lingüística también es un enfoque que interpreta que tales acciones y
relaciones se dan en el ciberespacio, donde las instituciones y las
prácticas sociales en que el lenguaje se usa afectan la vida social
contemporánea, o donde hay prácticas digitales que no logran ser
prácticas sociales recurrentes porque las comunidades virtuales pueden
extinguirse, pero interesa ser estudiadas por los comportamientos e
interacciones sociales que se produjeron. A propósito, la lingüística
busca entender las prácticas sociales y comunicativas a través de
espacios \emph{online} mediante la netnografía, sin distanciarse del
medio \emph{offline}. En este sentido, debe entenderse que lo digital
está inmerso en la sociedad, la cual se ve transformada por las
tecnologías digitales, por lo que el análisis del mundo virtual no
termina en la investigación de la comunicación en línea, sino que a su
vez se nutre del análisis post-digital \cite{berry2015postdigital}. Así, es
importante también explorar las comunidades fuera de su espacio virtual,
tales como aquellas que convocan encuentros o reuniones presenciales
posteriores a la interacción digital. Dicho de otro modo, la
comunicación digital está entrelazada con los contextos \emph{offline},
pero también se encuentra anclada en las actividades fuera de línea de
los propios individuos. Es decir, con la netnografía lingüística se
puede realizar observaciones \emph{online} y \emph{offline}, como
entrevistas en persona con los usuarios. Además, métodos y técnicas
\emph{offline} junto con la netnografía lingüística pueden asegurar una
triangulación y rigurosidad en una investigación. Como se ha explicado,
la netnografía es un enfoque en la investigación social y cultural
establecido y su aplicación en el campo lingüístico es coherente y
razonable, puesto que la comunicación en línea involucra aspectos
lingüísticos esenciales.

Quien emplea la netnografía en el análisis del lenguaje es el netnógrafo
lingüístico. Como investigador, precisamente, el lingüista puede
entender que con la netnografía es posible obtener una comprensión
holística de las vidas experimentadas en el espacio virtual por una
comunidad o grupo de personas en particular acerca del uso del lenguaje,
sin aislar de su estudio las normas sociolingüísticas, los valores
lingüístico-culturales, la manifestación de identidades
intralingüísticas, las relaciones personales interlingüísticas y las
prácticas cotidianas e institucionales lingüísticas que los afectan. En
este sentido, los lingüistas pueden entender que la netnografía no es
una mera recopilación de conocimientos; por el contrario, se trata de un
trabajo de interpretación de fenómenos en que participa el lenguaje en
la vida social de los internautas. \textcite{Blommaert2010} sostuvieron
que en las etnografías el lenguaje se percibe como una herramienta
cargada y evaluada socialmente que posibilita que los seres humanos
actúen como seres sociales. Esto mismo sucede en el espacio virtual.
Dentro de este, se analizan los significados y las funciones de los
recursos lingüísticos dentro de sus contextos para conocer la vida, la
interacción, la identidad, el control, el poder, la vigilancia, el
ciberodio, las negociaciones, las representaciones, las aspiraciones y
los activismos de los interlocutores digitales.

En el artículo se ha visto que algunas áreas de investigación son la
netnopragmática, la netnosociolingüística y la netnolingüística
aplicada. En relación con ello, cabe indicar que al igual que etnografía
lingüística, la netnografía lingüística reclama la presencia de enfoques
interpretativos más amplios desde la sociología, la antropología, la
lingüística aplicada, más que concentrarse en las tradiciones
antropológicas para el estudio del lenguaje, como la etnografía de la
comunicación \cite{hymes1968ethnography} y la sociolingüística interaccional \cite{gumperz1972sociolinguistics}. Al respecto, \textcite{costello2017} consideran que la
netnografía es una oportunidad para las investigaciones, pues permite
revisar información y hasta participar en actividades en línea de forma
activa, que es de interés para la netnografía lingüística, la cual va
más allá de la aplicación de métodos etnográficos presenciales en el
estudio de las lenguas o actos de habla \cite{ramajo2011etnografico}. En términos de
temporalidad, las indagaciones lingüísticas pueden ser breves o implicar
años de investigación. En términos de cantidad, puede centrarse en una
comunidad o en múltiples comunidades. Y pueden combinar la netnografía
con otros métodos interdisciplinarios de investigación sobre comunidades
en línea.

Con este trabajo se intenta reflexionar sobre el método netnográfico que
los lingüistas pueden emplear en sus investigaciones. Para ello deben
desarrollar netnografías lingüísticas desde visiones amplias que
consideren que las interacciones lingüísticas en el ciberespacio
muestran los comportamientos de los internautas y que estas se insertan
en contextos y estructuras sociales que afectan la vida social
contemporánea. Con la netnografía tienen la posibilidad de interpretar y
documentar la interconexión y la complejidad de los valores y las
conductas sobre el lenguaje que tienen los cibernautas y las diversas
comunidades digitales. Para ello, la netnografía ha venido adaptando las
técnicas etnográficas tradicionales al estudio de la ``red'', no solo
para recoger, grabar y clasificar los datos, sino también para
analizarlos y representarlos \cite{addeo2020netnography}. Finalmente, los investigadores que usan el método
netnográfico tienen que considerar que las comunidades virtuales son
dinámicas y, por tanto, sus acercamientos con este deben en
consencuencia adaptarse, particularmente con los avances en inteligencia
artificial, que afectan caracterizaciones e interacciones en el mundo
digital.


\section{Agradecimientos}
El autor agradece a los pares ciego por sus alcances a la investigación.

\section{Financiamiento}
La presente investigación es un resultado del “Concurso de Proyectos de Investigación con recursos no monetarios 2023 para Grupos de Investigación” como parte del grupo de investigación Lenguas y Filosofías del Perú (LFP), de la Facultad de Letras y Ciencias Humanas de la Universidad Nacional Mayor de San Marcos – RD 001050-2023-D-FLCH/UNMSM.


\printbibliography\label{sec-bib}

\end{document}
