\section{Introducción}\label{sec-introducción}

Las nuevas formas de estar en el campo han reinventado las herramientas
y dispositivos de la investigación. \textcite{ortizocana2019} reclaman la
urgencia de que se decolonicen las ciencias sociales y creen nuevas
maneras de pensar el conocimiento, lo alterativo y lo configurativo. El
mismo llamamiento se da para las metodologías y la recolección de datos,
más aún si se busca explorar comunidades no conocidas u observadas y en
espacios virtuales. Entre las formas de indagar, las investigaciones
etnográficas se han interesado por grupos sociales y han dado
importancia al estudio de comunidades minoritarias \cite{urzua2019}, 
mayormente bajo enfoques cualitativos, que buscan escapar a las metodologías y
visiones positivistas de estudio, sobre todo generalizantes. Al mismo
tiempo, las diversas y nuevas investigaciones se enfrentan a los
desafíos que la virtualidad trae. La globalización ha afectado la vida
social de las personas y ha generado la creación de individuos y
comunidades virtuales y una comunicación sincrónica y asincrónica. Hoy,
la participación en comunidades en línea y la comunicación en entornos
de la Web 2.0 han aumentado. Los avances en las tecnologías repercuten
en muchas áreas. Esto ha provocado que diversas disciplinas opten por
generar métodos y teorías que permitan analizar el material del
ciberespacio, así como profundizar en las personas, sus discursos y las
relaciones que se encuentran detrás \cite{lovón2023}.

En este contexto, ha aparecido la netnografía como un método que permite
explorar las acciones o interacciones que realizan los internautas. La
internet se ha convertido en un medio no solo de comunicación, sino
también de construcción sociocultural. Se crean comunidades virtuales
con actitudes e identidades propias, muchas veces fluidas, que han
empezado a estudiarse para conocer, finalmente, el rol de muchos
individuos, así como el sentido que se da a su existencia e interacción
humana en un mundo digital tan variado. Es de interés, por ejemplo, para
los lingüistas el papel que tiene el lenguaje y su uso a través de los
espacios en línea en la construcción de filiaciones, representaciones y
relaciones sociales, incluso en su aplicación para el aprendizaje de
idiomas.

Este método coincide terminológicamente con otros que dan cuenta de lo
que ocurre en la web. La netnografía es también llamada etnografía en
línea \cite{markham2005methods}, etnografía digital \cite{murthy2008}, etnografía
virtual \cite{hine2000virtual}, \emph{online ethnography} y
\emph{cyber-ethnography}, pues el foco de la investigación es el mundo
virtual \cite{chanona2018}y la comprensión de las diversas
culturas digitales \cite{dominguiez2007}. Sin embargo, la literatura indica
también que es un método diferente a todas estas formas de acceso al
conocimiento, pues se sostiene que la netnografía se concentra en el
análisis de los comportamientos de los consumidores, por lo que el
marketing la emplea como técnica de investigación, mientras que las
otras se interesan por distintos estudios sociales y antropológicos. No
obstante, dada su aplicación amplia a otros contextos, se usa para
analizar percepciones, interacciones y comportamientos sociales más
diversos que el mercadeo, de modo que se diluye la distinción. Así, la
netnografía se entiende como un método que se interesa por las
comunidades virtuales. Por ello, diversos autores se decantan por esta
visión ancha de la netnografía, como también nos inclinamos, 
la cual permite estudiar los comportamientos y las interacciones
de los cibernautas. \textcite{ruiz2015etnografia} sugieren que todas estas
menciones al final coinciden en la investigación del mundo digital:

\begin{quote}
El interés también reside en poder mostrar la preocupación de los
metodólogos por nombrar el tipo de trabajo que realizan en los mundos
virtuales para el estudio de las prácticas socioculturales que allí se
observan: etnografía virtual, etnografía digital, ciberantropología,
etnografía mediada, netnografía, antropología de los medios, etnografía
del ciberespacio, etnografía de/en/a través de Internet. Incluso, en
esta diversidad hay quienes han hecho intentos de autoetnografía. Lo que
es un hecho, es que todas estas acepciones tienen un mismo objetivo,
estudiar las relaciones sociales, cognitivas, afectivas que se dan en el
ciberespacio, para lo cual han tomado como base los principios de la
etnografía para trasladarlo al ciberespacio \citeyear[p. 70]{ruiz2015etnografia}.
\end{quote}

A propósito, muchos de los investigadores vienen diferenciando la
etnografía de la netnografía. \textcite{hammersley1983ethnography} consideraron
que el método etnográfico se apoya en la observación participante y la
entrevista dirigida, de forma combinada y empleadas en el lugar donde se
producen los acontecimientos que se investigan. Con la etnografía se
observan las prácticas de los diferentes grupos humanos y el
investigador puede contrastar lo que la gente dice con lo que usual o
realmente hace en forma presencial. Por su lado, la netnografía también
emplea la observación participante y las entrevistas, además de los
datos de archivo, los datos obtenidos y otras formas de datos
disponibles para el investigador \cite{kulavuz-onal2015} en forma virtual.
Dicho de otro modo, la etnografía recoge los datos a través de
interacciones en persona, cara a cara, mientras que la netnografía a
través de interacciones en línea. En ese orden, la etnografía
lingüística se ocupa de comunidades fuera del espacio virtual, mientras
que la netnografía tiene interés por estas.

El objetivo de este artículo es reflexionar sobre la utilidad que tiene
la netnografía en la investigación lingüística. Dado el rol central del
lenguaje en los espacios digitales para construir discursos, amerita
explorar este método que favorece la indagación en el espacio virtual.
Al respecto, los investigadores del lenguaje tienen un llamamiento para
trabajar en el análisis del ciberespacio \cite{maine2021}, pero
no siempre encuentran maneras con las cuales pueden hacerlo, por lo que
pensar en la netnografía como alternativa resulta provechoso. Para la
elaboración de este manuscrito, se emplea la documentación de fuentes y
las referencias vinculadas con el tema. Se reúne la información
principal de trabajos netnográficos sobre el lenguaje.

El artículo se ordena de la siguiente manera: primero se define y se
presenta la netnografía; luego, se describe la función del lingüista
como netnógrafo; posteriormente, se propone un conjunto de áreas
investigativas de la netnografía lingüística; por último, se ofrecen
unas reflexiones a manera de conclusiones.
