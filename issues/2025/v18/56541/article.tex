\documentclass[portuguese]{textolivre}

% metadata
\journalname{Texto Livre}
\thevolume{18}
%\thenumber{1} % old template
\theyear{2025}
\receiveddate{\DTMdisplaydate{2024}{12}{16}{-1}}
\accepteddate{\DTMdisplaydate{2025}{3}{8}{-1}}
\publisheddate{\today}
\corrauthor{Juliana Cristina Faggion Bergmann}
\articledoi{10.1590/1983-3652.2025.56541}
%\articleid{NNNN} % if the article ID is not the last 5 numbers of its DOI, provide it using \articleid{} commmand 
% list of available sesscions in the journal: articles, dossier, reports, essays, reviews, interviews, editorial
\articlesessionname{dossier}
\runningauthor{Bergmann et al.}
%\editorname{Leonardo Araújo} % old template
\sectioneditorname{Hugo Heredia Ponce}
\layouteditorname{Saula Cecília}

\title{Traduções contrastivas na aprendizagem de línguas adicionais: análise de caminhos e limitações com uso de Inteligência Artificial}
\othertitle{Contrastive translations in additional language learning: analysis of paths and limitations using Artificial Intelligence}

\author[1]{Juliana Cristina Faggion Bergmann~\orcid{0000-0002-0535-5279}\thanks{Email: \href{mailto:juliana.bergmann@ufsc.br}{juliana.bergmann@ufsc.br}}}
\author[2]{Andréa Cesco~\orcid{0000-0002-4708-186X}\thanks{Email: \href{mailto:andrea.cesco@ufsc.br}{andrea.cesco@ufsc.br}}}
\author[3]{Luzia Antonelli Pivetta~\orcid{0000-0003-4283-5295}\thanks{Email: \href{mailto:lapivetta@gmail.com}{lapivetta@gmail.com}}}
\author[4]{Gabriela Marçal Nunes~\orcid{0000-0003-0075-4649}\thanks{Email: \href{mailto:mnunesgabriela@gmail.com}{mnunesgabriela@gmail.com}}}
\affil[1]{Universidade Federal de Santa Catarina, Programa de Pós-Graduação em Educação e Departamento de Metodologia de Ensino, Florianópolis, SC, Brasil.}
\affil[2]{Universidade Federal de Santa Catarina, Programa de Pós-Graduação em Estudos da Tradução, Departamento de Língua e Literatura Estrangeiras, Florianópolis, SC, Brasil.}
\affil[3]{Universidade Federal de Santa Catarina, Programa de Pós-Graduação em Estudos da Tradução, Florianópolis, SC, Brasil.}
\affil[4]{Universidade Federal de Santa Catarina, Programa de Pós-Graduação em Educação, Florianópolis, SC, Brasil.}

\addbibresource{article.bib}

\usepackage{array}
\usepackage{float}

\begin{document}
\maketitle
\begin{polyabstract}
\begin{abstract}
Este artigo tem como objetivo analisar os resultados de uma atividade tradutória contrastiva que compara a tradução realizada por aprendentes de espanhol como língua adicional, sem o uso de tecnologias, com as traduções realizadas, posteriormente, por Inteligências Artificiais (IAs). O objetivo foi observar e compreender as possibilidades e as limitações dos modelos de tradução automática, refletir criticamente sobre as traduções realizadas em termos de questões estilísticas, culturais, lexicais e de adequação linguística, dentre outros aspectos, bem como sobre a importância do(a) tradutor(a) humano(a). O artigo começa com uma apresentação sobre os conceitos a respeito do uso da tradução em sala de aula --- uma prática que, ao longo das décadas, vem ganhando novos olhares --- sobretudo no contexto de ensino e aprendizagem de línguas estrangeiras, com a tradução pedagógica. De maneira geral, discute-se sobre a perspectiva da IA generativa como uma ferramenta para alcançar objetivos educacionais e, por fim, faz-se uma análise crítico-reflexiva da sequência didática de tradução proposta aos aprendentes compreendendo as possibilidades, desafios e resultados alcançados.

\keywords{Tradução pedagógica. Inteligência artificial. Análise contrastiva}
\end{abstract}

\begin{english}
\begin{abstract}
This article analyzes the results of a contrastive translation activity comparing translations produced by learners of Spanish as an additional language, without the use of technologies, with the translations generated by Artificial Intelligences (AIs). The objective is to observe and understand the possibilities and limitations of machine translation models, critically reflecting on both human and AI-generated translations in terms of stylistic, cultural, lexical and linguistic adequacy issues, among other aspects, as well as on the importance of the human translator. The article begins with a presentation of concepts regarding the use of translation in the classroom --- a practice that has regained attention in recent decades --- particularly in the context of foreign language teaching and learning through pedagogical translation. We then explore the role of generative AI as a tool for achieving educational objectives and conclude with a critical-reflective analysis of the didactic translation sequence proposed to learners, assessing its possibilities, challenges, and outcomes.

\keywords{Pedagogical translation. Artificial intelligence. Contrastive analysis}
\end{abstract}
\end{english}
\end{polyabstract}

\section{Introdução}
Estudos recentes apontam que o uso da tradução com diferentes enfoques tem apresentado bons resultados quando aplicado ao ensino-aprendizagem de línguas adicionais \cite{carreres2019, pintado2019, bergmann2024}. Com um histórico de emprego intermitente da tradução em sala de aula ao longo do tempo, em que a valorização à presença da língua materna oscilava entre o incentivo total à rejeição cabal, nos últimos anos um equilíbrio parece favorecer sua utilização em contexto e\-ducativo \cite{bergmann2024}.

Pym, Malmkjaer e Gutierrez-Cólon Plana comentam em uma importante pesquisa publicada em 2013 sobre o papel da tradução no ensino de línguas na União Europeia que, embora ainda haja resistência por parte de muitos docentes em empregá-la em ambiente formal de aprendizagem, não há dados que inviabilizem seu uso. Os pesquisadores, mesmo considerando as críticas por parte de alguns estudiosos da tradução sobre o desenvolvimento de competências específicas durante o processo de aprendizagem, ainda assim a defendem dentro do contexto de ensino de línguas, considerando que a tradução “encoraja, em vez de agir contra, uma observação mais atenta da nova língua porque mostra as suas diferenças em relação à língua conhecida" \cite[p. 18, tradução nossa]{pym2013}.
O foco, portanto, não é formar tradutores, mas, ensinar línguas com a utilização da tradução, fomentando a reflexão crítica que esta pode proporcionar. E sobre isso, Carreres, Muñoz-Calvo e Noriega-Sánchez são categóricas:

\begin{quote}
aceitar que a tradução no aprendizado de idiomas pode fazer parte de uma metodologia comunicativa e que a tradução é uma atividade normal da língua (não apenas para especialistas), longe de ser uma ameaça à tradução profissional e à sua didática, abre novas e estimulantes possibilidades de diálogo e contribui para a criação de sinergias entre o aprendizado de segundas línguas e os estudos de tradução \cite[p. 102, tradução nossa]{carreres2017}.
\end{quote}

Outros documentos norteadores também apontam esse caminho de equilíbrio. O \textit{Marco común europeo de referencia para las lenguas – volumen complementario} \cite{consejo2022}, por exemplo, dispõe a tradução de um texto escrito como uma dentre quatorze atividades de mediação, organizadas no documento em três grandes áreas: mediar textos, mediar conceitos e mediar a comunicação. O texto, publicado pelo Conselho da Europa, é uma referência desde 2001 para professores de línguas do mundo todo, mesmo tendo sido elaborado para o público europeu.

Outras ações também se somaram à tradução no desenvolvimento de atividades interessantes para o contexto de aprendizagem em situação formal. Uma delas é o uso de tecnologias digitais associadas às novas práticas tradutórias \cite{bergmann2019}, ampliando a gama de possibilidades a partir dos objetivos e motivações dos aprendentes e de seus professores.

Assim, a (re)inserção de atividades tradutórias em contexto escolar, apoiada pelo uso de diferentes tecnologias digitais atualmente disponíveis, demonstram grande potencial para propiciar ao aprendente não apenas novas práticas críticas e reflexivas, como também o de proporcionar o aumento de suas competências linguísticas e culturais.

Tecnologias digitais já são usadas há tempos como colaboradoras em atividades de tradução. Desde o acesso de não falantes de uma língua a um conteúdo, através de ferramentas de tradução automática, por exemplo, até as desenvolvidas para uso profissional de tradutores, as quais facilitam o processo tradutório. Segundo Silva \textit{et al.}, estas tecnologias

\begin{quote}
têm servido como ‘promovedoras’ da tradução (e.g., tradução automática para não tradutores ou indivíduos monolíngues), ‘facilitadoras’ da tradução (e.g., sistemas de memória de tradução para tradutores) e o próprio ‘objeto’ da tradução (e.g., softwares e sítios eletrônicos que precisam ser traduzidos) \cite[p. 6]{silva2020}.
\end{quote}

Nos últimos anos, no entanto, tem-se um novo patamar de possibilidades, com a difusão de ferramentas de Inteligência Artificial Generativa (IAG), como o ChatGPT, o DALL-E ou o DeepArt, capazes de gerar novos conteúdos em formato de textos, imagens e áudios. Seus usos são múltiplos em diferentes áreas, assim como na educação, em que podem ser úteis para diferentes fins pedagógicos.

Mas talvez o papel mais importante que o ensino possa estimular com o uso das IAGs seja o desenvolvimento de um olhar analítico sobre essa tecnologia, mostrando aos aprendentes seus potenciais criativos, ao mesmo tempo que compreendem suas limitações éticas de forma crítica. “Uma das questões importantes relativas à IA, particularmente para educadores e aprendentes, é a perspectiva da IA generativa como uma ferramenta de aprendizagem” \cite[p. 28]{correia2024}, a partir da qual se possa refletir sobre seus usos e também sobre os resultados que ela manifesta.

Sendo assim, este artigo tem como propósito apresentar e analisar a prática do uso de tradução em sala de aula de línguas adicionais, a partir de uma sequência de atividades que une ensino de línguas, tradução, tecnologias digitais e inteligência artificial generativa, por meio da qual o aprendente de língua espanhola, em fase inicial, compara sua tradução com outras realizadas com o uso de recursos digitais para, a partir dos resultados, analisar e refletir sobre suas diferenças e semelhanças.


\section{Tradução pedagógica como caminho crítico-reflexivo de aprendizagem}\label{sec-2}

Embora a tradução por muito tempo tenha sido deixada de lado nas metodologias de ensino de línguas, dependendo da relação que cada metodologia esta\-belecia entre língua materna e língua adicional \cite{bergmann2024}, há atualmente uma nova visão a respeito de sua utilização em sala de aula, por meio da qual é possível considerá-la como “uma ferramenta a mais para o ensino e aprendizagem de uma língua” \cite[p. 9]{pintado2019}. Essa perspectiva pode ser ampliada, se levarmos em consideração o que dizem \textcite[p. 616, tradução nossa]{carreres2019}, quando afirmam que “a tradução transcende o caráter instrumental que vem associado à ideia de ferramenta ou método, e passa a constituir uma habilidade ou aptidão integral que deve desenvolver-se na aprendizagem de uma língua”.

A partir de estudos já elaborados desde os anos 90 e outros posteriormente fomentados pela publicação do \textit{Marco común europeo de referencia para las lenguas (MCER)} a partir de 2001, o qual insere a interação e a mediação como modos de comunicação juntamente com a compreensão e a expressão, bem como o reconhecimento do plurilinguismo, percebe-se um movimento que tenta demonstrar que há benefícios em realizar atividades contextualizadas de tradução, cujo intuito seja o ensino-aprendizagem de línguas adicionais sob a perspectiva de que o aprendente assume um papel de mediador quando em contato com falantes de diferentes línguas.

De acordo com os diversos modos de tradução que encontramos na aprendizagem formal de línguas adicionais -- a tradução de textos, a tradução interiorizada e a tradução explicativa \cite{HurtadoAlbir1987, ArribaGarcia1996, pintado2019} --, seria possível propor práticas coerentes para sua utilização. Diante disso, \textcite{carreres2019} também apontam que, para considerarmos a introdução de atividades de tradução para aprendizagem de línguas, é útil lembrar quais componentes ou habilidades elas podem ajudar a desenvolver. Nesta perspectiva, elas citam \cite[p. 619, tradução nossa]{carreres2019}:
\begin{quote}
\begin{itemize}
\item Consciência contrastiva
\item Competência lexical e morfossintática
\item Competência pragmática
\item Reconhecimento e correção de erros causados por interferências
\item Conhecimento sobre as convenções de tipos e gêneros textuais em ambas as línguas
\item Competência intercultural
\item Capacidade de leitura crítica
\item Desenvolvimento das outras quatro habilidades
\item Melhora da fluidez (oral ou escrita) e do estilo
\item Criatividade
\item Resolução de problemas
\item Autonomia e colaboração
\end{itemize}
\end{quote}

Como analisaremos de maneira mais detalhada adiante e considerando a proposta a ser apresentada neste trabalho, diversos desses componentes e habilidades são trabalhados a partir de atividades desenvolvidas em contexto de sala de aula de aprendizagem de língua espanhola: a consciência contrastiva, a competência lexical e morfossintática, a competência pragmática, o reconhecimento e correção de erros causados por interferências, a capacidade de leitura crítica e a competência intercultural. A respeito desta última, cabe destacar que

\begin{quote}
o aprendente terá de se interrogar sobre as referências culturais, o significado, a função etc., do texto a traduzir, e para isso terá de perceber que as respostas não estão sempre, ou apenas, no dicionário, mas na sua experiência de mundo e no seu conhecimento tanto da sua cultura de origem como da cultura que está estudando \cite[p. 64, tradução nossa]{mendo2009}.
\end{quote}

Por sua natureza contrastiva, a aplicação de atividades tradutórias na aprendizagem de línguas adicionais estimula o desenvolvimento de pensamento crítico, considerado aqui como um “processo reflexivo, em que constantemente temos que questionar nossos próprios preconceitos, interpretações e conclusões” \cite[p. 70]{buckingham2022}.

Dessa forma, percebe-se que, ademais das questões relacionadas diretamente com os aspectos linguísticos da língua adicional a ser aprendida, atividades pedagógicas de tradução podem contribuir na ampliação de conhecimentos sobre a própria língua materna, bem como estimular o reconhecimento e aceitação da cultura do outro.

\section{Novas perspectivas a partir da Inteligência Artificial}\label{sec-3}
A evolução da IA, desde seus primeiros passos em 1950 até as conquistas atuais em relação à aprendizagem profunda e ao processamento de linguagem natural, tem sido uma jornada marcada por avanços significativos e mudanças de paradigma na forma como as máquinas podem simular a inteligência humana.

Esses avanços desencadearam transformações em nossa sociedade, que vão desde áreas como a saúde e a indústria, até campos como a educação, entrando de forma substancial e profunda na vida cotidiana, exigindo da sociedade adequações e ajustes -- por vezes bastante complexos -- a novas necessidades e interesses. Nesta perspectiva:

\begin{quote}
a inteligência artificial aplicada ao cotidiano, a internet das coisas, a comunicação entre as máquinas que rodeiam nossas vidas, implica na automatização de tantos processos da vida cotidiana quanto possível, envolvendo não apenas rotinas simples, mas processos complexos de interação, compreensão e, inclusive, aprendizagem por meio da observação, acúmulo de enormes volumes de dados, que são analisados, relacionados e comparados para extrair padrões e modelos de compreensão e de ação \cite[p. 61]{perezgomez2021}.
\end{quote}

Com esse cenário, fica praticamente impossível ignorar a utilização de IAs na educação, incluindo o ensino-aprendizagem de línguas adicionais. A questão que se coloca, no entanto, é como compreender seu funcionamento, seus alcances e suas limitações, em um uso crítico e criativo destas ferramentas. Para isso, segundo \textcite[p. 357, tradução nossa]{munozbasols2024}, “é fundamental que o aprendente conheça as funcionalidades das ferramentas que utiliza (e de suas limitações) e que, na medida do possível, receba orientação sobre que tipo de tarefas e atividades enriquecem o uso dessa tecnologia”.

Como exemplo, podemos mencionar duas tecnologias digitais automáticas que podem ser utilizadas para traduzir textos: Google Tradutor\footnote{\url{https://translate.google.com/}} e ChatGPT\footnote{\url{https://chatgpt.com/}}. As duas ferramentas estão desenvolvidas com base em fundamentos tecnológicos diferentes. O Google Tradutor, mais antigo, utiliza-se de redes neurais e algoritmos avançados, os quais potencializaram seu desempenho, uma vez que não possuía originalmente a capacidade de traduzir além de frases simples, deixando a desejar na tradução de expressões idiomáticas e/ou metafóricas, palavras polissêmicas etc., gerando resultados literais e descontextualizados \cite{munozbasols2024}.

Já o ChatGPT, de desenvolvimento e acesso massivo mais recente, é um gerador de textos, a chamada Inteligência Artificial Generativa (IAG), cujo valor

\begin{quote}
    está na velocidade com que processa a linguagem natural e estabelece comunicação imediata; um ser humano não consegue gerar e processar informações nessa velocidade. Para gerar texto, a ferramenta pode analisar e entender os parâmetros linguísticos em profundidade, garantindo a coerência ou a ordem lógica das diferentes partes de um texto e a coesão ou a conexão interna da informação \cite[p. 346, tradução nossa]{munozbasols2024}.
\end{quote}

Seus usos perpassam o ensino-aprendizagem de línguas adicionais, à medida que já se observava a utilização constante, por parte dos aprendentes, do Google Tradutor e agora, mais do que nunca, do ChatGPT em diferentes situações fora do contexto educativo. Dessa forma, percebe-se a necessidade de explorar esses recursos com propostas de atividades que possam não só ampliar os conhecimentos a respeito de suas funcionalidades, como também refletir sobre seus caminhos e limitações, desenvolvendo o pensamento crítico e sua consequente ação \cite{buckingham2022}, também crítica e reflexiva.

Para que essa criticidade se desenvolva de maneira livre e aberta, é fundamental que os aprendentes percebam o uso da IA não como um “propósito em si, mas sim uma ferramenta para alcançar os objetivos educacionais” \cite[p. 33]{correia2024}. Isso quer dizer que o entendimento por parte de todos de que as traduções feitas pelas ferramentas são apenas mais uma das possibilidades tradutórias, dentre várias outras, é essencial a fim de que os objetivos pedagógicos da proposta possam ser atingidos.

A partir dessa perspectiva, apresenta-se a seguir uma sequência didática por meio da qual as atividades previstas desenvolvem e analisam contrastivamente traduções do excerto de uma narrativa, conforme descrito metodologicamente a seguir.


\section{Análise crítico-reflexiva em uma sequência didática}\label{sec-4}
Com base nas teorias apresentadas, analisamos neste trabalho uma atividade pedagógica de tradução, aplicada com aprendentes iniciantes de espanhol como língua adicional. Visando promover a reflexão sobre as diferentes traduções, as escolhas que fazemos quando traduzimos e a influência de ferramentas de IA nesse processo, a proposta foi elaborada para ser contrastiva e crítico-reflexiva.

Para isso, foi selecionado um excerto da história “La presunta abuelita” \cite[p. 156]{bartaburu2004}. A escolha se deu por ser este um texto repleto de heterossemânticos, comumente chamados de falsos cognatos ou falsos amigos, ou seja, com palavras em espanhol que, apesar da aparente semelhança gráfica ou fonética em português, possuem significados distintos e incompatíveis nas duas línguas, comprometendo a comunicação em um determinado contexto, o que acaba gerando armadilhas linguísticas para os aprendentes iniciantes.

\begin{quote}
    Precedentes de raiz etimológica idêntica, as línguas portuguesa e espanhola, como as línguas românicas que são, compartilham vários aspectos linguísticos em comum, tanto a nível semântico como estrutural, dada a grande semelhança tipológica entre elas. Os vários aspectos de semelhança, sejam parciais ou totais, manifestam-se com grande contundência no nível lexical da língua, o que se reflete na existência de palavras cognatas que ultrapassam 85\% do vocabulário comum entre português e espanhol. No entanto, nesse grande fluxo lexical nem tudo parece ser o que realmente é, já que grande parte desse léxico também é composto por falsos cognatos […] \cite[p. 3, tradução nossa]{sousapereira2011}.
\end{quote}

A escolha desse texto foi estratégica, pois permitiu a identificação e reflexão sobre essas armadilhas linguísticas com as quais frequentemente nos deparamos quando aprendemos uma língua estrangeira e ao se traduzir entre línguas próximas, como é o caso do par português-espanhol.

O texto-fonte foi organizado de maneira a que o excerto a ser traduzido tivesse coerência narrativa, ao mesmo tempo que chamava a atenção para algumas palavras sublinhadas, que foram previamente selecionadas pelas pesquisadoras por seu caráter desafiador a um lusoparlante aprendente de espanhol e que seriam utilizadas em etapas posteriores da atividade.

\begin{table}[H]
\centering
\caption{Texto-fonte - La presunta abuelita}
\label{quadro1}
\begin{tabular}{l p{13.6cm}}
\toprule 
\multicolumn{2}{c}{La presunta abuelita} \\
\midrule
A & Había una vez una niña que fue a pasear al bosque.\\ 
B & De repente \underline{se acordó} de que no le había comprado ningún regalo a su abuelita. \\  
C & Pasó por un parque y arrancó unos lindos \underline{pimpollos} rojos. \\
D & Cuando llegó al bosque vio una carpa entre los árboles y alrededor unos \underline{cachorros} de león comiendo carne. \\
E & El corazón le empezó a \underline{latir} muy fuerte.\\
F & En cuanto pasó, los leones se pararon y empezaron a caminar atrás de ella.\\
G & Buscó algún sitio para refugiarse y no lo encontró. \\
H & Eso le pareció \underline{espantoso}. \\
I & A lo lejos vio un \underline{bulto} que se movía y pensó que había alguien que la podría ayudar. \\
J & Cuando se acercó vio un \underline{oso de espalda}. \\
K & Se quedó en silencio \underline{un rato} hasta que el oso desapareció y luego, como la noche llegaba, \underline{se decidió a prender fuego para cocinar un pastel de berro que sacó del bolso}. \\
L & \underline{Empezó a preparar el estofado y lavó también unas ciruelas.} \\
\bottomrule
\end{tabular}
\source{Adaptado de “La Presunta Abuelita”, Español en Acción -- Tareas y Proyectos \cite[p. 156]{bartaburu2004}.}
\end{table}

A metodologia desenhada para aplicação da sequência didática proposta nesta atividade previu cinco grandes etapas principais, que apresentamos de maneira mais aprofundada a seguir: (1) tradução sem uso de recursos; (2) tradução com suporte de vocabulário; (3) tradução integral com uso de tecnologia; (4) comparação crítica das versões e (5) reescrita da tradução inicial.

Foram analisadas para este artigo as produções de cinco aprendentes lusofalantes em estágio inicial de estudo formal da língua espanhola. Disponibilizou-se uma cópia do texto-fonte a cada um deles, que, após leitura silenciosa, deveria realizar individualmente a atividade tradutória, que se seguiu etapa a etapa conforme descrevemos e comentamos a seguir.

\subsection{Tradução sem uso de recursos}\label{sec-4a}
De início, os aprendentes foram convidados a traduzir o texto individualmente, utilizando para isso apenas seus conhecimentos linguísticos prévios, sem recorrer a dicionários, ferramentas tecnológicas ou qualquer outro tipo de auxílio, como perguntas à professora ou aos colegas de turma. Essa etapa teve como principal objetivo diagnosticar níveis de familiaridade com o vocabulário e as estruturas gramaticais do espanhol exigidos para compreensão do texto, além de identificar o grau de confiança que tinham em suas habilidades de tradução. Também serviu como balizadora do conhecimento linguístico de partida de cada um dos aprendentes, que serviria de base contrastiva para os momentos seguintes. Suas traduções estão sistematizadas na Tabela \ref{quadro2}, em que são mantidas em destaque as palavras sublinhadas no texto-fonte.

%%%%%%% LUGAR DO QUADRO2 %%%%%%
\begin{small}
\begin{longtable}{l *{5}{>{\raggedright\arraybackslash}p{2.4cm}}}
\caption{Tradução dos aprendentes – La Presunta Abuelita – Etapa 1\label{quadro2}} \\
\arrayrulecolor{black}
\toprule 
 & Aprendente 1 & Aprendente 2 & Aprendente 3 & 
 Aprendente 4 & Aprendente 5 \\
\midrule
\endfirsthead

\arrayrulecolor{lightgray}
A & Era uma vez uma menina que foi passear no bosque. & Era uma vez uma menina que foi passear no bosque. & Era uma vez uma menina que foi passear no bosque. & Era uma vez uma menina que foi passear no bosque. & Era uma vez uma menina que foi passear no bosque. \\
\hline
B & De repente ela \underline{se lembrou} que não tinha comprado ne\-nhum presente para sua vovózinha. & De repente \underline{se deu conta} de que não havia comprado ne\-nhum presente para sua vó. & De repente \underline{se lembrou} que não havia comprado ne\-nhum presente para sua avó. & De repente \underline{se lembrou} de que não havia comprado ne\-nhum presente para sua avó. & De repente ela \underline{se lembrou} que não tinha comprado ne\-nhum presente pra sua avó. \\
\hline
C & Passou por um parque e arrancou umas lindas \underline{flores vermelhas}. & Passou por um parque e arrancou. & Passou por um parque e arrancou umas \underline{lindas flores}. & Passou por um parque e arrancou umas \underline{lindas flores}. & Passou por um parque e arrancou umas lindas \underline{flores vermelhas}. \\
\hline
D & Quando chegou no bosque viu uma $\phi$ entre as árvores e ao redor \underline{alguns leões} comendo carne. & Quando chegou no bosque viu \underline{alguns leões} entre as árvores comendo carne. & Quando chegou no bosque viu uma \underline{carpa} entre as árvores e \underline{alguns leões} comendo carne. & Quando chegou no bosque viu uma \underline{carpa} entre as árvores e \underline{alguns leões} comendo carne. & Quando chegou no bosque viu uma ∅ entre as árvores e ao redor \underline{alguns leões} comendo carne. \\
\hline
E & O coração começou a bater muito forte. & O coração começou a bater muito forte. & O coração começou a bater muito forte. & O coração começou a bater muito forte. & O coração começou a bater muito forte. \\
\hline
F & Enquanto passou, os leões pararam e começaram a caminhar atrás dela. & Enquanto passou, os leões pararam e começaram a caminhar atrás dela. & Enquanto passou, os leões pararam e começaram a caminhar atrás dela. & Enquanto passou, os leões pararam e começaram a caminhar atrás dela. & Enquanto passou, os leões pararam e começaram a caminhar atrás dela. \\
\hline
G & Buscou algum lugar pra \underline{se refugiar} e não encontrou. & Buscou algum lugar para \underline{se esconder} e não encontrou. & Buscou algum lugar para \underline{se refugiar} e não encontrou. & Buscou algum lugar para \underline{se refugiar} e não encontrou. & Buscou algum lugar para \underline{se refugiar} e não encontrou. \\
\hline
H & Isso pareceu \underline{espantoso}. & Isso pareceu \underline{assustador}. & Isso pareceu \underline{espantoso}. & Isso pareceu \underline{espantoso}. & Isso pareceu \underline{espantoso}. \\
\hline
I & \underline{Ao longe}, viu um \underline{vulto} que se mexia e pensou que havia alguém que poderia ajudar. & \underline{No longe}, viu um \underline{vulto} que se movia e pensou que havia alguém que poderia ajudá-la. & \underline{Ao longe}, viu um \underline{vulto} que se mexia e pensou que havia alguém que poderia ajudar. & \underline{Ao longe}, viu um \underline{vulto} que se mexia e pensou que havia alguém que poderia ajudar. & \underline{Ao longe}, viu um \underline{vulto} que se mexia e pensou que havia alguém que poderia ajudar. \\
\hline
J & Quando chegou perto, viu um \underline{urso}. & Quando se apro\-ximou, viu um \underline{urso}. & Quando se apro\-ximou, viu um \underline{urso}. & Quando chegou perto, viu um \underline{urso}. & Quando se apro\-ximou, viu um \underline{urso}. \\
\hline
K & Ficou em silêncio, até que o urso desapareceu e logo, como a noite chegava, decidiu \underline{acender} fogo para cozi\-nhar um \underline{bolo} que tirou do \underline{bolso}. & Ficou em silêncio um momento até que o urso desapareceu e logo, como a noite chegava, decidiu \underline{acender} fogo para cozinhar um \underline{bolo} de $\phi$ que tirou do \underline{bolso}. & Ficou em silêncio um momento até que o urso desapareceu e logo, como a noite chegava, decidiu \underline{acender} fogo para cozinhar um \underline{bolo}. & Ficou em silêncio até que o urso desapareceu e logo, como a noite chegava, decidiu \underline{acender} fogo para cozinhar um \underline{bolo}. & Ficou em silêncio, até que o urso desapareceu e logo, como a noite chegava, decidiu \underline{acender} fogo para cozinhar um \underline{bolo} de $\phi$ que tirou do \underline{bolso}. \\
\hline
L & Começou a preparar o… e lavou também umas… & - & Começou a preparar o… & - & - \\
\arrayrulecolor{black}
\bottomrule
\source{Elaborado pelas autoras.}
\end{longtable}
\end{small}

Após a tradução da primeira versão, notou-se que os aprendentes, embora iniciantes, conseguiram resolver diversas questões de vocabulário e encontraram no contexto a saída para alguns vocábulos, como é o caso do \textit{se acordó} 
 (B), que foi traduzido corretamente para “se lembrou”, ou uma opção que, embora não seja a tradução mais precisa, também consegue expressar a mesma ideia dentro do contexto: “se deu conta”.

Além da tradução usando a estratégia do contexto, outro procedimento utilizado por alguns dos aprendentes foi a omissão de palavras, como no caso de \textit{cachorros de león} (D), em que permanece apenas “leão” muito provavelmente pela semelhança com o português, e cachorros, que seriam “filhotes”, é suprimida. Percebe-se a mesma ocorrência em \textit{oso de espalda} (J), em que deixam de fora do texto-alvo a posição em que se encontrava o urso, desaparecendo a informação “de costas”, e no último parágrafo, em \textit{pastel de berro} (K), no qual eles utilizam o vocábulo “bolo”, mas não o especificam, excluindo a informação complementar “de agrião”.

Da mesma forma, observa-se uma substituição em \textit{pimpollos} (C), cujo correspondente em português é “botões de rosa”, que dá lugar ao termo mais genérico “flores”, motivado, provavelmente, pela associação à cor \textit{roja (rojos)} que viria na sequência, a qual também é um heterossemântico, mas que, por ser mais conhecida, não apresentou dificuldades de tradução.

Dentre os vocábulos que geraram dúvidas também está \textit{carpa} (D), que em espanhol significa “barraca”, um significado muito diferente da palavra “carpa” em português (tipo de peixe), que não seria resolvido nem mesmo pelo contexto. Nesse caso, alguns optaram por omiti-lo, enquanto outros, por repeti-lo tal qual se escreve em espanhol, mesmo não sendo equivalente na língua de chegada ou fazendo sentido no enredo.

Além deste, outro caso no qual ocorre essa manutenção da palavra do texto-fonte para o texto-alvo aparece no trecho (K), com a palavra \textit{bolso} (no português, bolsa), em que três deles deixaram como na língua de partida, mantendo um contexto sem sentido em português: como um bolo seria retirado de um bolso? E nas demais traduções o termo foi omitido, reforçando a falta de conhecimento em relação ao vocábulo em todas as escolhas.

Nessa etapa, conseguiu-se detectar o baixo nível de familiaridade dos aprendentes com as palavras destacadas e algumas das estruturas gramaticais do espanhol, dificultando, assim, a compreensão do texto. Além disso, contribuiu para indicar o conhecimento linguístico individual de partida, o qual serviu de base contrastiva para os momentos posteriores da atividade.

No entanto, pôde-se observar também, que mesmo sem a utilização de recursos, como o dicionário, por exemplo, os aprendentes, embora com dúvidas, valeram-se de estratégias pertinentes para tentar preencher lacunas em seu conhecimento linguístico e, de certa forma, conseguiram dar uma solução para a tarefa.

\subsection{Tradução com suporte de vocabulário}\label{sec-4b}
A segunda etapa constava da consulta em um dicionário de alguns dos vocábulos que foram considerados problemáticos, sendo a prática do uso apropriado do dicionário, segundo \textcite{alcarazo2014}, uma das aplicações da tradução pedagógica. O objetivo principal: os aprendentes deveriam realizar uma compreensão crítica do texto e das suas próprias traduções por meio da verificação, através de um suporte vocabular, na tentativa de interpretar melhor a narrativa evitando assim as armadilhas com os falsos amigos. Incentivados pela professora, eles buscaram as seguintes palavras: a) \textit{pimpollo}; b) \textit{carpa}; c) \textit{cachorro}; d) \textit{latir}; e) \textit{oso}; f) \textit{rato}; g) \textit{pastel}; h) \textit{berro}; i) \textit{estofado} e j) \textit{ciruela}.

As reações à descoberta dos vocábulos no dicionário foram as mais diversas. Alguns dos heterossemânticos realmente não eram de conhecimento dos aprendentes (como \textit{carpa}, por exemplo), porém outros, como \textit{cachorro} ou \textit{pastel}, foram relembrados, pois eles já os conheciam.

Esta é uma etapa importante do processo de aprendizagem, na qual se confrontam crenças iniciais, suposições linguísticas e contextuais e a eficácia das estratégias comunicativas. A partir do contraste entre o que achava que significava uma palavra e o seu significado existente no dicionário, novas regras internas sobre as línguas -- materna e adicional -- são criadas pelo indivíduo, que constrói, neste movimento, seu vocabulário pessoal.

Além disso, foi-lhes dada a oportunidade de reescreverem, ainda de forma individual, uma nova versão das suas traduções, agora baseadas nas descobertas que fizeram, imprimindo à história mais sentido.


\subsection{Tradução integral com uso de tecnologia}\label{sec-4c}
A terceira etapa introduziu o uso de ferramentas na tradução completa do texto. Os aprendentes as utilizaram para gerar duas traduções automatizadas da narrativa, uma com o Google Tradutor, outra com o ChatGPT gratuito, e foram incentivados a observar os pontos de divergência entre suas versões pessoais do texto e o resultado fornecido pelos tradutores automáticos.

As duas ferramentas foram sugeridas a partir de uma conversa inicial, na qual eles comentaram sobre aquelas que conheciam e já utilizavam ao traduzir textos, sentenças etc. O Google Tradutor foi, de maneira unânime, o recurso mais mencionado e, aparentemente, o mais usado pelos aprendentes. Das duas opções é o mais antigo e, por isso, o mais consolidado. Já o ChatGPT, embora tivessem ouvido falar sobre ele, nem todos sabiam da possibilidade de realizar traduções, o que ajudou na decisão de usá-lo na atividade, proporcionando um momento em que todos teriam a oportunidade de conhecê-lo melhor.

Um ponto importante a ser ressaltado aqui é a característica de atuação da IA generativa, caso do ChatGPT, que se caracteriza justamente pela capacidade que tem de adaptar seus textos às particularidades e estilo de escrita do usuário, fornecendo versões mais personalizadas a cada indivíduo. No caso da aplicação desta sequência em particular, isso seria um obstáculo, porque a intenção principal desta etapa é a análise crítica contrastiva, em grupo, das soluções tradutórias trazidas pelas IAs. Por isso, após as experiências individuais de traduções automáticas, as etapas seguintes foram desenvolvidas coletivamente, com a projeção de uma tabela (Tabela \ref{quadro3}), na qual constavam o texto-fonte e as duas versões traduzidas pelas ferramentas mencionadas, que serviu como base para uma discussão com todo o grupo sobre as escolhas verificadas em cada tradução: estrutura gramatical do texto, vocábulos, estilística etc.

Assim, a tabela que apresentamos a seguir traz as três versões trabalhadas em sala com os aprendentes: na primeira coluna, à esquerda, o texto-fonte, em espanhol; na segunda coluna a versão traduzida por um tradutor automático, neste caso o Google Tradutor, plataforma escolhida por seu caráter simples e gratuito; e na última coluna, à direita, a versão traduzida pela ferramenta ChatGPT, desenvolvida com tecnologia de Inteligência Artificial Generativa (IAG).

\begin{small}
\begin{longtable}{l *{3}{>{\raggedright\arraybackslash}p{4.2cm}}}
\caption{Comparativo entre traduções de La Presunta Abuelita com uso de tecnologia\label{quadro3}} \\
\arrayrulecolor{black}
\toprule
 & Texto-Fonte & Versão Google Tradutor & Versão ChatGPT \\
\midrule
\endfirsthead

\arrayrulecolor{lightgray}
A & La \underline{presunta} abuelita

Había una vez una niña que fue a pasear al bosque. & A \underline{suposta} avó

Era uma vez uma menina que foi passear na floresta. & A \underline{presunta} vovó

Era uma vez uma menina que foi passear na floresta. \\
\hline
B & De repente se acordó de que no le había comprado ningún regalo a su abuelita. & De repente, \underline{ele} se lembrou de que não havia comprado nenhum presente para sua avó. & De repente, \underline{ela} se lembrou de que não havia comprado nenhum presente para sua vovó. \\
\hline
C & Pasó por un parque y arrancó unos lindos pimpollos rojos. & \underline{Ele} passou por um parque e colheu alguns lindos \underline{botões vermelhos}. & Passou por um parque e arrancou alguns lindos \underline{brotinhos vermelhos}. \\
\hline
D & Cuando llegó al bosque vio una carpa entre los árboles y alrededor unos cachorros de león comiendo carne. & Quando chegou à floresta viu uma tenda entre as árvores e ao redor \underline{dela} havia alguns filhotes de leão comendo carne. & Quando chegou à floresta, viu uma tenda entre as árvores e ao redor, alguns filhotes de leão comendo carne. \\
\hline
E & El corazón le empezó a latir muy fuerte. & Seu coração começou a bater muito forte. & Seu coração começou a bater muito forte. \\
\hline
F & En cuanto pasó, los leones se pararon y empezaron a caminar atrás de ella. & Assim que ela passou, os leões pararam e começaram a \underline{andar} atrás dela. & Assim que passou, os leões pararam e começaram a \underline{caminhar} atrás \underline{dela}. \\
\hline
G & Buscó algún sitio para refugiarse y no lo encontró. & \underline{Ele} procurou um lugar para se refugiar \underline{e} não encontrou. & \underline{Ela} procurou um lugar para se refugiar, \underline{mas} não encontrou. \\
\hline
H & Eso le pareció espantoso. & Isso parecia assustador para \underline{ele}. & Isso \underline{lhe} pareceu aterrador. \\
\hline
I & A lo lejos vio un bulto que se movía y pensó que había alguien que la podría ayudar. & Ao longe \underline{ela} viu um \underline{caroço em movimento} e pensou que havia alguém que poderia ajudá-\underline{la}. & Ao longe, viu uma \underline{sombra se movendo} e pensou que havia alguém que poderia ajudá-la. \\
\hline
J & Cuando se acercó vio un oso de espalda. & Quando \underline{ele} se aproximou, viu um \underline{urso por trás}. & Quando se aproximou, viu um \underline{urso de costas}. \\
\hline
K & Se quedó en silencio un rato hasta que el oso desapareció y luego, como la noche llegaba, se decidió a prender fuego para cocinar un pastel de berro que sacó del bolso. & \underline{Ela} ficou em silêncio por um tempo até que o urso desapareceu e então, \underline{com a aproximação da noite}, \underline{ela} decidiu acender uma fogueira para fazer uma \underline{torta de agrião} que tirou da bolsa. & Ficou em silêncio por um tempo até que o urso desapareceu e então, \underline{como a noite estava chegando}, decidiu \underline{fazer} uma fogueira para cozinhar um \underline{bolo de agrião} que tirou da bolsa. \\
\hline
L & Empezó a preparar el estofado y lavó también unas ciruelas. & Começou a preparar o ensopado e também lavou algumas ameixas. & Começou a preparar o ensopado e também lavou algumas ameixas. \\
\bottomrule
\source{Elaborado pelas autoras.}
\end{longtable}
\end{small}

Esse é um momento que desperta muito interesse entre os aprendentes. Além da oportunidade de descobrirem novas formas de tecnologia, seus alcances e limitações, eles também relacionam diversos conhecimentos, têm o primeiro contato com o “resultado” da tradução feita pela máquina e se mostram curiosos quanto às suas potencialidades, o que motiva para a etapa que vem a seguir. A curiosidade pelas versões traduzidas incentivou a participação ao longo da proposta, com atenção aos termos trazidos pelas plataformas, às mudanças, sinônimos e alguns equívocos que as ferramentas cometeram, o que gerou profícuas discussões entre eles.

O primeiro ponto observado foi o título, “A suposta avó” (Google Tradutor) comparado com a “A presunta vovó” (ChatGPT), cuja percepção foi a de que o ChatGPT não fez a tradução mais adequada, pois, segundo eles, quebrou a expectativa que se tem a respeito do título indicar algo sobre o que será tratado na narrativa, dificultando a compreensão inicial do texto. Embora os termos “avó” e “vovó” sejam semelhantes, a diferença também gerou discussão, pois o título do texto-fonte traz o termo \textit{abuelita}”, que seria o diminutivo de avó, como “vovózinha” e, em acordo, todos expressaram que “vovó” seria o termo que mais se assemelha, trazendo a carga afetiva do diminutivo.

Outro ponto que merece destaque é o gênero dos pronomes. No Google Tradutor, embora o primeiro parágrafo inicie com “Era uma vez uma menina” (A), indicando que a personagem é feminina, já na sequência a ferramenta utiliza erroneamente o pronome “ele” (B), o que também ocorre no próximo parágrafo (C). Já em (F), o pronome volta a aparecer, porém feminino, “ela”, e, em seguida, inicia uma frase novamente no masculino (G). Essa divergência, que irá se repetir até o final do texto, fez com que os aprendentes, embora estivessem propensos a “gostar” mais da tradução feita pelo Google Tradutor, perdessem a confiança no recurso por sentirem a marca de uma tradução compartimentada, frase a frase, e não dentro de um contexto, como esperavam. Diferente do que ocorreu no ChatGPT que manteve a concordância de gênero até o fim.

As escolhas vocabulares também foram percebidas, como nos casos de (C) “botões vermelhos” (Google Tradutor) e “brotinhos vermelhos” (ChatGPT); (F) “andar atrás dela” (Google Tradutor) e “caminhar atrás dela” (ChatGPT); (I) “caroço em movimento” (Google Tradutor) e “sombra se movendo” (ChatGPT); (J) “um urso por trás” (Google Tradutor) e “um urso de costas” (ChatGPT); (K) “com a aproximação da noite” (Google Tradutor) e “como a noite estava chegando” (ChatGPT); “acender uma fogueira” (Google Tradutor) e “fazer uma fogueira” (ChatGPT); “uma torta de agrião” (Google Tradutor) e “um bolo de agrião” (ChatGPT). Elas geraram muitas discussões, já que para alguns eram preferíveis as soluções do Google Tradutor e para outros as do ChatGPT. Como eles já haviam pesquisado em dicionários o significado dos heterossemânticos que apareciam no texto, em casos que envolveram essas palavras, alguns aprendentes recorreram aos conhecimentos adquiridos para argumentar a favor da escolha que empregava o vocábulo que aprenderam na etapa anterior: \textit{pimpollos} como botões; \textit{bulto} como sombra; \textit{de espalda} como de costas e \textit{pastel} como bolo, demonstrando de forma significativa a ampliação do vocabulário pessoal.

Após as observações e discussões, eles puderam concluir que, embora ambas ferramentas possam auxiliar no processo de tradução, faz-se necessária uma análise crítica e atenta do texto por um humano.

\subsection{Comparação crítico-reflexiva das versões}\label{sec-4d}
Esta etapa foi desenvolvida em sala, em um grande grupo, no qual compararam juntos as quatro versões do texto: a tradução inicial feita sem recursos, a versão com auxílio de vocabulário e as duas versões geradas pelos tradutores automáticos. Ressalta-se aqui a importância dada ao processo individual de cada aprendente, sem a intenção de ser prescritivo e determinar uma versão “correta” para a tradução, o que iria no caminho oposto ao principal objetivo da sequência pedagógica. Assim, no que diz respeito às duas primeiras versões da tradução, feitas individualmente pelos aprendentes, sem e com o recurso do dicionário, cada um teve acesso apenas aos seus próprios textos. As fontes coletivas, iguais para todos, foram apenas aquelas criadas pelo Google Tradutor e o ChatGPT.

Esta etapa foi crucial para que houvesse uma reflexão sobre as escolhas tradutórias que fizeram em cada fase, as escolhas apresentadas pelas IAs, além do impacto do uso dessas ferramentas nas nuances linguísticas.


\subsection{Reescrita da versão inicial}\label{sec-4e}
Finalmente, a última etapa, que não estava prevista inicialmente e foi solicitada pelos próprios aprendentes, envolveu a revisão e adaptação da primeira versão da tradução, agora com base no conhecimento adquirido nas etapas anteriores da sequência didática e com uma análise mais crítica sobre os processos tradutórios. Esta fase permitiu que eles revissem suas próprias produções e escolhas, aplicando as lições aprendidas e aprimorando suas habilidades tradutórias. Dessa forma, deram início à reescrita de uma nova versão, única e desenvolvida coletivamente.

\begin{longtable}{l p{13.6cm} }
\caption{Versão final da tradução da adaptação de La Presunta Abuelita, elaborada coletivamente pelos aprendentes\label{quadro4}} \\
\arrayrulecolor{black}
\toprule
\multicolumn{2}{c}{{A \underline{suposta} vovózinha}} \\
\midrule

A & Era uma vez uma menina que foi passear pela floresta. \\

B & De repente, ela se lembrou que não havia comprado nenhum presente para sua \underline{vovózinha}. \\

C & Passou por um bosque e colheu \underline{lindos botões de rosas vermelhas}. \\

D & Quando chegou na floresta, viu uma \underline{barraca} entre as árvores e, ao redor, alguns filhotes de leão comendo carne. \\

E & \underline{Seu} coração começou a bater muito forte. \\

F & Enquanto passava, os leões pararam e começaram a caminhar atrás dela. \\

G & Ela buscou algum lugar para \underline{se esconder}, mas não encontrou. \\

H & Isso pareceu assustador. \\

I & Ao longe, viu \underline{algo} se movendo e pensou que era alguém que poderia ajudá-la. \\

J & Quando se aproximou, viu que era um urso de costas. \\

K & Ficou em silêncio por um tempo até que o urso desapareceu e logo, como a noite chegava, decidiu acender fogo para cozinhar uma \underline{torta} de agrião que tinha na bolsa. \\

L & Começou a preparar o ensopado e também lavou algumas ameixas. \\
\bottomrule
\source{Elaborado pelas autoras.} \\
\end{longtable}

Na análise da versão final, destacam-se algumas alterações significativas. Primeiramente, os aprendentes optaram por preservar o diminutivo presente no texto-fonte, utilizando o termo “vovózinha” tanto no título quanto no primeiro parágrafo (A), em referência a \textit{abuelita}. Em relação à descrição das flores, decidiram utilizar a expressão “botões de rosas vermelhas” (C) como tradução de \textit{pimpollos rojos}, baseado em suas buscas no dicionário, já que nenhuma das versões automáticas propõe dessa maneira. Por fim, uma das escolhas mais debatidas foi a tradução do termo \textit{pastel}. Enquanto alguns defenderam a manutenção do “bolo”, outros argumentaram que, devido ao sabor diferenciado – \textit{pastel de berro} –, seria mais adequado e menos estranho optar por “torta de agrião” (K).

Percebe-se algumas questões relacionadas aos aspectos linguísticos que chamam a atenção, como a pontuação na frase “mas não encontrou” (G), que em todas as propostas individuais aparecia como “e não encontrou”, não necessitando o uso de uma vírgula, e na versão coletiva, surge, muito provavelmente motivada pela solução do ChatGPT, com a utilização da conjunção adversativa, porém, sem a utilização da vírgula. Também se observou um problema no emprego da regência verbal, que já aparecia em algumas traduções individuais, mas que as traduções automáticas corrigiram por “ele/ela se lembrou de que” (B), cuja regra prevê que o verbo “lembrar”, quando pronominal, requer o uso da preposição “de”, esta que foi mais uma vez suprimida no texto final.

Essa sequência de atividades teve como objetivo não apenas proporcionar uma experiência prática de tradução, mas também fomentar uma discussão crítica sobre o uso de ferramentas de IA na atividade tradutória. Ao final, os aprendentes foram capazes de reconhecer os benefícios e limitações dessas tecnologias, bem como refletir sobre sua própria evolução no processo de tradução e compreensão da língua que estão conhecendo. A inclusão de uma etapa adaptativa, solicitada por eles, reforça a importância de práticas pedagógicas que permitam a autonomia e a reflexão crítica no processo de ensino-aprendizagem.

\section{Considerações finais}\label{sec-conclusao}
A presente análise sobre traduções contrastivas na aprendizagem de línguas adicionais reforça a relevância de práticas que integrem reflexão e estratégia no processo de aprendizagem. Reconhecer as escolhas “intuitivas” e as estratégias de comunicação utilizadas pelos aprendentes diante de lacunas lexicais é essencial para potencializar a autonomia e a competência comunicativa. Tais momentos revelam não apenas a capacidade de adaptação, mas também a criatividade envolvida no intuito de apresentar soluções para os desafios que se apresentaram. Corroborando, assim, com o que mencionam \textcite{alcarazo2014} sobre as atividades de tradução pedagógica serem um excelente exercício para introduzir ou praticar algumas estratégias de aquisição e análise de léxico. Tais como:

\begin{quote}
a generalização (a tradução de um termo por outro mais geral […]), a particularização (a tradução de um termo por outro mais preciso,[…]) ou a procura de um sinônimo conhecido pelo aprendente que tenha o mesmo significado ou um significado equivalente […] \cite[p. 5, tradução nossa]{alcarazo2014}.
\end{quote}

Além disso, o processo de tradução em sala de aula de línguas adicionais se mostrou desencadeador de um espaço privilegiado para um aprendizado mais profundo e significativo. Ao transitar entre língua materna e língua adicional, os aprendentes têm a oportunidade de ampliar sua percepção sobre ambas, promovendo reflexões que enriquecem tanto o domínio linguístico quanto a compreensão cultural. Esse movimento dialógico entre os idiomas evidencia a tradução como uma ponte entre teorias linguísticas e práticas pedagógicas. Afinal,

\begin{quote}
traduzir obriga a tomar decisões fundamentadas e razoavelmente rápidas para resolver problemas de comunicação. Obriga a observar cuidadosamente, a comparar e a analisar, a encontrar semelhanças, diferenças e correspondências entre línguas e culturas, estratégias que um falante pluricultural terá de manter durante uma aprendizagem de línguas e culturas ao longo da vida \cite[p. 68, tradução nossa]{mendo2009}.
\end{quote}

No entanto, a crescente disseminação de ferramentas de IA no campo da tradução exige uma abordagem crítico-reflexiva. Embora essas tecnologias ofereçam agilidade e precisão em muitos contextos, sua aplicação na educação demanda uma análise cuidadosa de seus limites e potencialidades. A dependência excessiva dessas ferramentas pode mascarar as lacunas na aprendizagem, ao passo que seu uso consciente pode complementar e fortalecer o desenvolvimento do aprendente. Na prática mencionada, por exemplo, dentre as limitações da IA com base no ensino de línguas, citadas por \textcite{desposito2024} foi possível observar:

\begin{quote}
Falta de compreensão contextual: a IA pode ajudar na correção gramatical e na tradução automática, mas muitas vezes ela falha em compreender o contexto e a intenção do usuário. Isso pode levar a traduções imprecisas ou a sugestões incorretas de correção \cite [p. 147]{desposito2024}.
\end{quote}

Dessa forma, integrar práticas que combinem tradução, reflexão e ferramentas de IA oferece caminhos promissores para a aprendizagem de línguas adicionais. Contudo, é fundamental manter o foco no papel ativo e estratégico do indivíduo, valorizando sua capacidade de análise, criação e adaptação ao longo do processo formativo. Além disso, destacar que, embora a tradução automática tenha melhorado consideravelmente, ela ainda não substitui completamente a habilidade de um tradutor humano em captar nuances contextuais e produzir traduções precisas e fluentes, daí a importância de reconhecer e refletir sobre suas contribuições e limitações.

\printbibliography\label{sec-bib}
%conceptualization,datacuration,formalanalysis,funding,investigation,methodology,projadm,resources,software,supervision,validation,visualization,writing,review

\subsection*{Agência de Fomento}
Conselho Nacional de Desenvolvimento Científico e Tecnológico (CNPq) -- Projeto em cooperação com comprovada articulação internacional (Projetos Int 2023) 441786/2023-5.

\begin{contributors}[sec-contributors]
\authorcontribution{Juliana Cristina Faggion Bergmann}[conceptualization, writing, review]

\authorcontribution{Andréa Cesco}[conceptualization, writing, review]

\authorcontribution{Luzia Antonelli Pivetta}[conceptualization, writing, review]

\authorcontribution{Gabriela Marçal Nunes}[conceptualization, writing, review]
\end{contributors}
\end{document}
