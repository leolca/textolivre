\section{Discussion}\label{sec-discussion}

In regard to the learning outcomes, it is not feasible to assess the
acquisition of the L2 content. However, we have identified recurring
errors in writing and pronunciation that illustrate potential areas for
guidance to facilitate pupil improvement in the language. Additionally,
the children committed mistales in pronunciation and intonation.
Although these issues require resolution, it is acknowledged that
mispronunciations of this nature are common among speakers of Basque and
Spanish, given the inherent challenges posed by the English vocalic
system for those with a limited number of vocalic phonemes.

Regarding the findings of the questionnaire, the outcomes are consistent
with our expectations. The pupils have demonstrated predisposition
towards this pedagogical approach, which integrates stimuli that align
with the fundamental principles of Montessori, namely the diverse
classroom environment to which they are accustomed. It is also
noteworthy that the method incorporates multimodality, which is a key
aspect of the world in which children live. The final products
demonstrate that pupils can successfully handle the L2. This is a
significant finding, as \textcite{alonso-perez2018} have
demonstrated that observing the final outcome and analysing the progress
that the students have achieved ``make everyone feel rewarded'' \cite[p. 21]{alonso-perez2018}, which implies an even stronger boost to motivation.

Conversely, children have demonstrated interest and engagement with the
DAT activities. A recurrent 5-2 parameter is evident when analysing
questions 3, 5 and 6. This signifies that five of the seven children
indicated a high level of enjoyment, with ratings of Bs and As, while
the remaining two children achieved a passing grade of C+. It is
accurate to conclude that the highest grade awarded was not an A+, which
can be attributed to the fact that this task was compulsory and required
the learning of internet tools, which served this purpose.

In examining the reasons for the implementation\textquotesingle s
success or failure, the indicators recur. It is presumed that the two
children who provided the lowest ratings were those who responded in the
negative. The issue is that the respondents did not provide reasons
related to their experiences with DAT activities. One respondent did not
provide any reasons, while the other provided a single negative response
regarding one of the teachers. This is a mishap, as their answers would
have been of great help for the research. Conversely, the remaining five
respondents provided positive feedback, citing enjoyment derived from
video recording, fun during implementation, and, most notably, two
respondents demonstrated appreciation for the activity\textquotesingle s
didactic potential, stating, "It would be good" and "I learn." This
latter response is particularly noteworthy, as it reflects a depth of
reflection that not all children possess. These statements are in
accordance with \textcite{fernandez-costeles2021} regarding the perceptions of
primary education pupils towards the didactic possibilities of DAT.

In light of the previous considerations, it can be posited that the
findings of this survey indicate a favourable outcome. In terms of
learning, this implementation has opened paths for the teacher to work
those aspects of language which have proven to be less developed in
learners. Conversely, with regard to motivation, the utilisation of DAT
has enhanced students motivation, as it is founded upon certain
customary activities among children, such as the recording of videos.
This results in a more enjoyable process of learning the L2, which
undoubtedly facilitates the work of both the teacher and the learners.
This research contributes to the existing body of literature on the
subject, building upon the findings of previous studies conducted by
scholars such as \textcite{neves2004language}, \textcite{talavan2009aplicaciones,talavan2010audiovisual}, \textcite{banos2015clipflair}, \textcite{talavan2015first}, \Textcite{BELTRAMELLO_2019}, \textcite{lertola2019audiovisual}, \textcite{talavan2022audiovisual}, \textcite{rodríguez-Arancón2023} and \textcite{talavan2024}.

\section{Conclusion}

This research has explored the relationship between the Montessori Method and DAT, demonstrating that both have a significant degree of overlap and that the latter can be integrated as an additional workshop within the Montessori classes. We encountered two main limitations. The first limitation is inherent to the methodology itself. Given that children do not take exams, it is not possible to adhere to the experimental scheme that would otherwise be appropriate for this type of research. Consequently, it has not been possible to measure the anticipated improvement in pupils' language acquisition.

A second limitation of the study is the low response rate to the satisfaction questionnaire, with only seven out of twenty-four students completing it. This is a potential issue, as the sample size is small and the responses are not fully representative. However, they do offer insights into a particular trend. Among the pupils who responded, we have been able to ascertain their motivations and levels of enjoyment regarding the implementation of the DAT. Five of them indicated that they found it motivating, while the other two expressed differing pedagogical opinions. The primary issue that emerged from the data was that learners lacked clarity regarding the specific applications of DAT.

In this particular instance, a class of twenty-four children at an educational institution that employs active methodologies and Basque as the L1 have demonstrated the efficacy of DAT from several perspectives. From the viewpoint of the teacher, the role differs from that of the traditional educator. DAT allows teachers to provide pupils with materials that align with their interests, thereby personalising their learning experience. Although pupils are permitted to select activities that exceed their current capabilities, the provision of effective scaffolding can facilitate their learning process. Furthermore, Montessori advocated the use of self-correcting materials, a concept that is exemplified by DAT. Audiovisual materials allow learners to listen to themselves and receive immediate feedback on their performance, which undoubtedly contributes to the improvement of their oral proficiency in the L2. DAT presents new opportunities for stimulating pupils, given its multimodal nature. It offers language in authentic contexts, which helps pupils comprehend the meaning of language in real situations. Thus, by providing a new lexical variety, children can expand their vocabulary to express themselves in different situations. Lastly, the integration of technology and multimodality has enabled learners to personalise their learning experience, allowing them to progress at their own pace and receive education that is specifically tailored to their needs. This is achieved through the creation of new teaching tools, which are based on interactive games, simulations or educational videos. In this regard, DAT has emerged as a highly valuable instrument in the context of this pedagogical approach, facilitating more engaging and efficacious learning, while fostering autonomous learning and self-regulation. 

Importantly, the implementation of DAT is in accordance with both the fundamental principles of the Montessori Method and current legislation. It has been demonstrated that DAT effectively implements the competencies and skills required by law with regard to the teaching of the L2. Furthermore, it facilitates the development of students' cognitive abilities, which is fundamental for lifelong learning.

Future research could encompass a longitudinal study to include an analysis of the role of multilingualism (learners work with three languages: Basque L1 of some pupils and of school, Spanish L1 of some children, and English as L2) in the general learning process and in the Montessori Method.