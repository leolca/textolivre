\documentclass[english]{textolivre}

% metadata
\journalname{Texto Livre}
\thevolume{18}
%\thenumber{1} % old template
\theyear{2025}
\receiveddate{\DTMdisplaydate{2024}{3}{21}{-1}}
\accepteddate{\DTMdisplaydate{2024}{6}{22}{-1}}
\publisheddate{\DTMdisplaydate{2025}{1}{22}{-1}}
\corrauthor{Edurne Goñi Alsúa}
\articledoi{10.1590/1983-3652.2025.51717}
%\articleid{NNNN} % if the article ID is not the last 5 numbers of its DOI, provide it using \articleid{} commmand 
% list of available sesscions in the journal: articles, dossier, reports, essays, reviews, interviews, editorial
\articlesessionname{articles}
\runningauthor{Goñi-Alsúa y Rejas Vicente}
%\editorname{Leonardo Araújo} % old template
\sectioneditorname{Daniervelin Pereira}
\layouteditorname{João Mesquita}

\title{Educational methodologies and didactic audiovisual translation: Results of an implementation of combined revoicing and subtitling in a class of primary education with the Montessori method}
\othertitle{Metodologias educacionais e tradução audiovisual didática: resultados de uma implementação de dublagem e legendagem combinadas em uma turma do Ensino Fundamental com o método Montessori}

\author[1]{Edurne Goñi-alsúa, ~\orcid{0000-0002-7488-2689}\thanks{Email: \href{Edurne.goni@unavarra.es}{Edurne.goni@u navarra.es}}}\affil[1]{Universidad Pública de Navarra, Pamplona, Spain.}
\author[1]{Galder Rejas Vicente~\orcid{0009-0007-4495-100X}\thanks{Email: \href{rejas.132134@e.unavarra.es}{rejas.132134@e.unavarra.es}}}


\addbibresource{article.bib}

\begin{document}
\maketitle
\begin{polyabstract}
\begin{english}
  \begin{abstract}
      Among the new methodologies developed to teach the L2, Didactic Audiovisual Translation (DAT) has proven to be highly successful as it deals with the four skills in a multimodal environment, which appears to increase motivation. This paper presents the results of a DAT implementation in an Elementary classroom (11-12 years old children) in a school which follows the Montessori Method. Twenty-four children, divided in groups of four, chose their favourite scenes of the TV series \emph{Goenkale} (broadcast in the Basque channel Euskal Televista 1) and translated, dubbed and subtitled them from their L1, Basque, into the L2, English, by means of the tool VideoPad. As Montessori Pedagogy does not include tests, the authors could not follow the experimental scheme of pre and post-test to observe any improvement in language acquisition. Data was therefore collected by the teacher, and subsequently analysed considering the mistakes made by the pupils during the process.  Pupils additionally completed a questionnaire, which indicated high levels of satisfaction towards this approach, thereby opening a new didactic path within Montessori Methodology. 

    \keywords{Didactic Audiovisual translation (DAT) \sep Montessori Method \sep Dubbing \sep Subtitling \sep Primary Education}
  \end{abstract}
\end{english}

\begin{portuguese}
\begin{abstract}
    Dentre as novas metodologias desenvolvidas para o ensino de L2, a Tradução Didática Audiovisual (TDAV) tem-se revelado de grande relevância, uma vez que faz convergir as quatro competências num ambiente multimodal, o que motiva os alunos. Neste artigo apresentaremos os resultados de uma implementação baseada nesta metodologia, realizada numa turma do 6º ano do ensino básico (12-13 anos de idade), numa escola do método Montessori. Os vinte e quatro alunos, divididos em grupos de quatro, escolheram as suas cenas preferidas da série televisiva \emph{Goenkale} (da \emph{Euskal Televista 1}) e desenvolveram uma unidade didática na qual traduziram, dobraram e legendaram indiretamente (basco L1 para inglês L2) utilizando o programa VideoPad. Como a pedagogia Montessori não inclui testes, não foi possível desenvolver o esquema experimental de pré e pós-teste, mas o professor pôde recolher os erros cometidos para os incluir em unidades didáticas posteriores. Outro aspeto positivo foi a resposta dos alunos nos questionários de satisfação, que mostraram resultados favoráveis, abrindo novas vias didáticas na pedagogia Montessori.  

\keywords{Tradução audiovisual educacional \sep Pedagogia Montessoriana \sep Dublagem \sep Legendagem \sep Educação primária}
\end{abstract}
\end{portuguese}

% \begin{spanish}
% \begin{abstract}
%   Entre las nuevas metodologías desarrolladas para la enseñanza de la L2, la Traducción Audiovisual Didáctica (TAD) ha demostrado que es de gran relevancia, ya que en ella convergen las cuatro habilidades en un entorno multimodal, lo que motiva a los estudiantes. En este artículo vamos a presentar los resultados de una implementación basada en esta metodología, llevada a cabo en una clase de 6º curso de educación primaria (12-13 años de edad) en un colegio del método Montessori. Los veinticuatro alumnos, divididos en grupos de cuatro, eligieron sus escenas favoritas de la serie de televisión Goenkale (de Euskal Televista 1) y desarrollaron una unidad didáctica en las que tradujeron, doblaron y subtitularon de manera indirecta (L1 vasco a L2 inglés) por medio del programa VideoPad. Dado que la pedagogía Montessori no contempla la realización de tests, no se pudo desarrollar el esquema experimental de pre y post test, no obstante, el profesor pudo recoger los errores cometidos para darles cabida en unidades didácticas posteriores. Otro aspecto positivo fue la respuesta de los alumnos en los cuestionarios de satisfacción, que mostraron resultados favorables, lo que abre nuevos caminos didácticos en la pedagogía Montessori.  
% \keywords{Traducción Audiovisual Didáctica (TAD) \sep Pedagogía Montessori \sep Doblaje \sep Subtitulación \sep Educación primaria}
% \end{abstract}
% \end{spanish}

\end{polyabstract}

\section{Introduction}\label{sec-introduction}

Over the past decades there has been a growing interest in the
development of new methodologies for second language (L2) learning and
teaching, to fulfil the requirements of the new legislation, as defined
by the Common European Framework of Reference for Languages (CEFR),
which emphasizes the communicative aspect of languages.

One of these new methodologies is Didactic Audiovisual Translation
(DAT), i.e., the application of different modes of audiovisual
translation (subtitling, revoicing, audio description and voice over) to
L2 teaching. This approach has been implemented and its results analysed
in different educational stages. Nevertheless, this approach has yet
been used in primary education within alternative methodologies, such as
the Montessori Method, in blingual contexts (in this case,
Basque-Spanish). The present study aims to address this research gap.

This paper shows the results of an intervention combining subtitling and
dubbing in a class of 11-12-year-old pupils following the Montessori
Method, at a Basque-immersion language primary school. While our study
could not assess language acquisition improvements due to the
school's methodology, our findings regarding children
satisfaction and the alignment of DAT principles with Montessori
principles, as demonstrated in the text, suggest promising avenues for
successful DAT use in this environment.



\section{Theoretical framework}\label{sec-theoreticalframework}
\subsection{Artificial Intelligence in Education}\label{sub-sec-artificialintelligenceineducation}

The integration of AI in education has garnered significant attention in
recent years, revolutionizing educational goals, practices, and the
learning environment. In this context, \textcite{roll2016} explore the
evolutionary and revolutionary aspects of AI in education, emphasizing
the transformative impact on educational objectives, classroom
practices, and the broader learning environment. Building upon this,
\textcite{chen2020} contribute by highlighting the positive
outcomes and potential of AI systems across administrative,
instructional, and learning domains, providing a comprehensive review of
AI applications in education. \posscite{roll2016} 
study underscores the shift in educational goals, moving away from rigid
knowledge preparation for the workforce towards equipping students as
adaptive experts and on-the-job learners. The ubiquity of smartphones
has transformed educational goals, emphasizing knowledge application,
collaboration, and self-regulated learning. This shift requires a
corresponding change in assessments from summative to ongoing formative
measures. In terms of practices, Roll and Wylie note the incorporation
of authentic elements in classrooms, resulting in increased complexity
and the challenge of personalization. The learning environment has
expanded beyond traditional classrooms to include informal and workplace
learning, transforming teachers from the "sage on the stage" to the
"guide on the side" \cite[p.~592]{roll2016}. Roll and Wylie also
acknowledge challenges posed to AI in education, raising questions about
effective technology support for teachers. \posscite{chen2020} study
complements this by providing a comprehensive review of AI applications
in education, emphasizing AI's positive impact on
administrative tasks, instructional quality, and learning experiences.
AI has the capability in assisting teachers in tasks such as exam
generation and grading. Chen et al. highlight the increasing integration
of AI into education, from early childhood to higher levels, showcasing
the adaptability of AI, including the use of robots (cobots) in teaching
routine tasks. It recognizes the growth in AI-related publications and
explores the diverse applications of AI, emphasizing its transformative
impact on various educational aspects. The positive outcomes of AI
deployment include improved learning quality, collaboration, global
access, enhanced academic integrity, and personalized learning plans.

On the other hand, the increasing integration of AI in education has
prompted extensive exploration into its challenges, implications, and
ethical considerations. \textcite[p.~4]{rodrigues2023} delve into
the philosophical, historical, and practical dimensions associated with
tools such as ChatGPT and AI in education, emphasizing the necessity for
a thoughtful and ethical approach to their integration within
educational frameworks. This discussion is crucial as AI technologies,
including natural language processing and datification, become more
pervasive in educational settings. At the same time, \textcite[p.~4222]{nguyen2023} explore the ethical principles governing AI in
education, focusing on K-12 settings, emphasizing the need for global
standards. The study critically evaluates existing ethical guidelines
and explores the applications of AI, such as personalized learning
systems, automated assessments, facial recognition, and predictive
analytics, in supporting both teachers and students. However, the study
acknowledges ethical challenges, ranging from systemic bias to privacy
concerns, which call for the urgent need to develop comprehensive
ethical guidelines in the field, emphasizing the need for global
standards and unified ethical principles for trustworthy AI. \textcite{rodrigues2023} emphasize the importance of an ethical and
considerate approach to AI integration, situating their study within the
broader context of AI technologies' increasing
influence, particularly in natural language processing. The authors
address concerns about AI's impact on professions,
particularly within education, questioning whether ChatGPT and AI pose a
threat or a challenge to the educational landscape. They highlight the
significance of datification in the Web 4.0 era, emphasizing ethical
considerations in the intertwining of human-machine interactions,
particularly in education. \textcite{nguyen2023} stress the urgent need
to educate teachers and students about ethical concerns and justify the
development of ethical guidelines in the field. Also, \textcite[p.~170]{regan2019} identify six privacy concerns for teachers and learners.
These are: information privacy, anonymity, surveillance, autonomy,
non-discrimination and ownership of information.

In summary, these studies highlight the transformative impact of AI
integration in education, emphasizing shifts towards adaptive learning
methodologies, personalized approaches, and the redefined role of
teachers. AI emerges as an evolving reality within the educational
realm, prompting considerations of ethical, social, and methodological
implications as discussed by the authors. Clear guidelines are deemed
necessary, particularly concerning assessment practices, privacy issues,
and technological infrastructure. Additionally, effective technology
support for teachers is identified as essential, especially regarding AI
tools such as cobots and chatbots.
\section{Proposal}\label{sec-proposal}

The research conducted was driven by two main questions, intended to
complement existing studies, and to gain insights into two key areas:
the first refers to the teaching-learning aspect, and the second to the
enjoyment of the pupils.

\begin{itemize}
    \item Does the implementation of a DAT didactic unit, based on the
    combination of interlingual indirect subtitling and dubbing (L1 Basque)
    enhance the results on acquisition of English language?

    \item Is DAT a motivational method for students of primary education
    following alternative schooling methodologies?
\end{itemize}

\subsection{Participants}\label{sub-sec-participants}

This implementation was conducted in a primary education school situated
in a middle-class neighbourhood of Pamplona, Spain, which follows the
Montessori Method, whose L1 is Basque. While not all pupils who
participated in this research speak this language at home, all classes
are taught in Basque. In the area, there is another school with two
lines of L1, Basque and Spanish, and a high school with the same L1
paradigm.

This school is committed to an active approach to learning, utilising
Montessori-inspired materials and a class structure that incorporates
learning environments and project-based learning as the primary
pedagogical methodology. The aforementioned pedagogical approach is
employed with the objective of fostering the development of
self-regulation and autonomy in students, as well as enhancing their
capacity for collaborative work through the utilisation of flexible
group structures and other active educational resources.

A total of 24 pupils, comprising the 6th course of primary education and
aged between 11 and 12 years old, participated in the project. The
pupils were divided into groups of four and worked together throughout
the implementation period. The participants demonstrated a level of
English proficiency corresponding to A1, as defined by the Common
European Framework of Reference for Languages (CEFR). Parents were
informed and gave their consent to the project, which has been approved
by the corresponding committee of UPNA.

\subsection{Materials}\label{sub-sec-materials}

During the implementation, learners were provided with the scripts of
the scenes and employed the class computers, dictionaries (both physical
and online), voice recorders and the editing app VideoPad
(\url{https://www.nchsoftware.com/videopad/es/index.html}). Finally, the
pupils completed a questionnaire (\Cref{annex-01}) to ascertain their level of
satisfaction with the implementation process, allowing for both
quantitative and qualitative results to be gathered.

\subsection{Design}\label{sub-sec-design}

One of the distinctive characteristics of the Montessori approach is the
absence of conventional testing and examination procedures.
Consequently, the implementation did not adhere to the customary
experimental pre-test--post-test design. Nevertheless, the instructor
could delineate certain guidelines to enhance the acquisition of the L2,
based on the errors that students had committed throughout the sessions.

A further aspect of this approach is that learners must be free to
select their own areas of interest and determine their own learning
strategies. This is reflected in the fact that they spent two sessions
searching for a TV series and a particular scene that they wished to
translate, dub and subtitle. As this was the first time that such a
project was implemented in the school, the instructor elected to develop
it in the Kitchen Corner, the designated workspace intended for the
duration of the internship.

The initial proposal was to implement a didactic unit on subtitling.
However, the instructor determined that pupils could undertake a
combined project encompassing both dubbing and subtitling, given the
absence of time constraints typically encountered in traditional
educational settings. This alteration has furnished the implementation
with an unanticipated degree of depth. As a result of this expansion,
the final products created by the students were the selected scenes
translated from the L1 to the L2, and both dubbed and subtitled. The
project was developed over four days, in sessions of 45 minutes each.
The subsequent schema was followed:

\begin{itemize}
    \item Choice of TV series and scene: two sessions
    
    \item Transcription and translation of the script: two sessions
    
    \item Audio recording: three sessions
    
    \item Scene editing: three sessions
\end{itemize}

\subsection{Procedure}\label{sub-sec-procedure}

\subsubsection{Day 1. The choice of series and scenes}

The session commenced with the activation of prior knowledge through the
posing of questions to the pupils, such as "What are your favourite TV
series?" This enabled the teacher to ascertain the
learners\textquotesingle{} interests, thereby facilitating the
personalisation of the learning process in accordance with their
preferences. A further crucial element of these two sessions was the
formation of flexible groups, comprising learners with diverse
abilities, so that they could learn from one another. The pupils elected
to work with their preferred scenes from the television series
\emph{Goenkale}, which features teenagers and is set in the Basque
Country. It is broadcast on \emph{Euskal Telebista} 1 (ETB1). This is a
highly popular programme with a particularly strong viewership among the
adolescent demographic. Prior to the DAT practice session, the children
were required to imitate both the intonation and rhythm of the speech of
the character they were to dub.

\subsubsection{Day 2. Transcriptions and translations}

The first session comprised an examination of the scene and an endeavour
to identify a suitable transcription. The pupils were initially required
to view the scene on multiple occasions, with the objective of
facilitating an in-depth comprehension of the nuances embedded within
the dialogue. This approach was designed to facilitate a more
straightforward engagement with the language.

Following the viewing, the children conducted online research to find
the transcriptions of the dialogues, utilising a range of digital
platforms. In the event that the desired material was not accessible,
the pupils utilized YouTube, as it offers subtitles for a number of
videos. They then proceeded to transcribe the scene by manually copying
the subtitles.

In the second session, the children completed the transcription and
translated it from L1 to L2 using both physical and online dictionaries.
The instructor directed the students\textquotesingle{} attention to the
fact that language can convey double meanings and transmit cultural
expressions that, when translated literally, are devoid of meaning. The
teacher\textquotesingle s main role was to provide assistance. She
clarified doubts and illustrated the lexicon, ensuring that all learners
were able to comprehend the message conveyed in the script.

\subsubsection{Day 3. Audio recording}

In the initial session, the students engaged in practice activities,
which entailed reading the transcriptions in both languages and
commencing the recording of the L2 audio. In the subsequent sessions,
the children proceeded with the audio recording.

During the audio recording sessions, the pupils concentrated on
developing their pronunciation and fluency. As the teacher consistently
reiterated the importance of aligning the spoken text with the
corresponding mouth movements, the pupils undertook a process of review
and editing of their recordings, with the objective of ensuring that
they met the desired quality standards. In order to achieve this, it was
necessary for the children to consider the previous work, in order to be
able to adjust the text in the L2 to the rhythm of the original scene
and to align the audio with the video. In addition, pupils were required
to become proficient in the use of the VideoPad application in order to
gain familiarity with its various functions.

The final stage of the process involved the deletion of the audio from
the original scene and the insertion of the new audio files, which were
then adjusted to the rhythm of the scenes. The children proceeded to
implement the final modifications to their product. The instructor
offered feedback and guidance to assist the pupils in enhancing their
productive oral skills.

\subsubsection{Day 4. Scene subtitling and editing}

In the initial session, the pupils were instructed in the techniques of
video editing and were given the opportunity to improve their writing
skill as they worked on the subtitles. In the remaining two sessions,
the children continued subtitling to create the final product, which was
a scene that was both dubbed and subtitled. The software utilized by the
pupils was once more VideoPad, which also facilitates the creation of
subtitles. By the conclusion of the sessions, children had produced
videos with high-quality audio and accurate subtitles.

\subsection{Assessment}\label{sub-sec-assesment}

Upon completion of the implementation phase, both the pupils and the
teacher conducted an assessment of the various projects using an
evaluation rubric (\Cref{annex-02}). Children were evaluated on five distinct
criteria: translation accuracy, coherence between the visual and
auditory content, adequate use of language and grammar, cultural
adaptation, and creativity and originality.

This assessment rubric was devised for use by both teachers and
learners. The teacher subsequently provided a more detailed assessment
in each category, including additional commentary on the areas that
require improvement and on those in which the pupils had demonstrated
proficiency.

\section{Results}\label{sec-results}

\subsection{First Expectations}\label{sub-sec-firstexpectations}

The authors hypothesised that pupils would be able to achieve a higher
linguistic level as a result of the unquestionable stimulation provided
by the audio-visual tools, which have been demonstrated to address the
learning needs of students. Additionally, in accordance with the
findings of previous research, it was anticipated that the
children\textquotesingle s level of motivation for the subject would
increase.

Nevertheless, it is reasonable to anticipate certain challenges
associated with the computer skills and linguistic proficiency of the
pupils. With regard to the latter, it is important to recognise that
children are simultaneously acquiring three languages without formal
instruction, which could potentially lead to some difficulties.

\subsection{Results on Acquisition of the Language}\label{sub-sec-resultsonacquisition}

As previously stated, the Montessori Method does not include exams,
therefore it was not feasible to propose a pre-test and post-test
framework to collect quantitative data. Consequently, it is not possible
to provide evidence of language improvement, if any. Nonetheless, it was
possible to anticipate certain improvements in their productions, given
that the children were engaged in activities such as translating,
dubbing and subtitling videos, practising listening comprehension,
pronunciation, writing and speaking.

A further defining feature of Montessori is that children select the
areas of study that they wish to pursue. Thus, this approach presented a
significant challenge for them, as their choices were based on the
scenes they enjoyed, irrespective of the level of L2 proficiency
required. This is the reason why the implementation resulted in
unforeseen language difficulties for the children, which obliged the
teacher to provide linguistic support by scaffolding the language.

Although it was not possible to analyse in detail the language
improvement resulting from the implementation, some common mistakes
committed by the pupils could be identified. These can be divided into
two main groups: those related to written productions and those related
to oral ones. In general, the children committed grammatical mistakes in
order to fit the sentences with the speech, such as the removal of the
subjects or the elision of auxiliary verbs. The errors associated with
oral production were primarily related to the rhythm of the
conversations and the pronunciation, particularly that of the vocalic
phonemes, as the pupils read the words in a literal manner. It is
important to highlight the intricate nature of these phonemes for
speakers of Basque and Spanish. In comparison to English, which has
twelve vocalic sounds, the two languages in question possess only five.
Furthermore, the pace of the original dialogues also affected the
pupils\textquotesingle{} oral productions, necessitating adjustments in
their speaking rate to align with the tempo of the original dialogues.

\subsection{Results of the Questionnaire}\label{sub-sec-resultsofthequestionnaire}

Following the completion of the implementation, pupils completed a
survey about AVT and DAT (see \Cref{annex-01}). Unfortunately, as the teachers
did not compel the pupils to fill it, and children were being prepared
for their incorporation to the formal instruction of the next course
(1st of Secondary Education), only seven children answered to the
questionnaire. The questions addressed their opinions regarding the
utilisation of DAT, the motivation it provides and its potential
integration into future English classes. As the pupils are taught in the
Basque language, and given the inherent complexity of the vocabulary and
the dearth of knowledge about the subject matter, the survey was
developed in Basque. However, for the sake of clarity and accessibility,
the sentences have been translated into English in the title of the
figures.

The initial question (see \Cref{fig-01}) sought to ascertain the
children\textquotesingle s level of enjoyment in the workshop. All of
the participants provided a rating that was above the minimum passing
grade. Four children, representing over half of the sample, rated the
workshop with an 8 or 9, while two children assigned a 6, and one child
a 7. These ratings indicate that the children felt at ease and content.
It is noteworthy that none of the children reached the 10-point mark,
given that they live in a multimodal world surrounded by video and
gaming platforms, which are likely to exert a strong influence on their
preferences. It may be posited that the introduction of this novel
activity, coupled with the necessity to master the utilisation of
internet-based tools, has instilled a sense of unease and lack of
confidence amongst them.
\begin{figure}[htbp]
    \centering
    \begin{minipage}{.5\textwidth}
    \includegraphics[width=\textwidth]{fig01.png}
    \caption{Question 1. From 1 to 10, how much have you enjoyed
    the ambiance of the workshop?}
    \label{fig-01}
    \source{Owm elaboration.}
    \end{minipage}
\end{figure}

In question 2 (see \Cref{fig-02}), the respondents were asked about their
previous knowledge of AVT. Four of the participants were unaware of its
existence, while the remaining three were cognizant of it. These figures
were unanticipated, given that the children have access to audiovisual
products in three different languages. This may have led them to assume
that there is a process of translation behind that range of options. It
seems reasonable to posit that the most probable reason for this lack of
awareness is that, due to their age, they are accustomed to consuming
audiovisual content in those languages and have not considered the
processes involved.

\begin{figure}[htbp]
    \centering
    \begin{minipage}{.5\textwidth}
    \includegraphics[width=\textwidth]{fig02.png}
    \caption{Question 2. Did you know what audiovisual translation
    was? (green-yes, purple-no).}
    \label{fig-02}
    \source{Owm elaboration.}
    \end{minipage}
\end{figure}

The answers to question 3, if they had enjoyed learning the L2 by means
of the editing of videos (see \Cref{fig-03}) were also positive.
Three-quarters of the children, five, demonstrated a positive attitude
towards the methodology employed, while two of them expressed a negative
opinion. Once more, we may cite the necessity of learning something new
compulsory as the primary reason for their refusal. These results align
with the responses to questions 1, 5 and 6, which will be analysed
subsequently.

\begin{figure}[htbp]
    \centering
    \begin{minipage}{.5\textwidth}
    \includegraphics[width=\textwidth]{fig03.png}
    \caption{Question 3. Have you enjoyed learning English by
    means of video editing?}
    \label{fig-03}
    \source{Owm elaboration.}
    \end{minipage}
\end{figure}

In response to question 4, which pertains to the two modes implemented
in class and translation (see \Cref{fig-04}), it can be observed that data do
not provide a clear indication of the pupils\textquotesingle{}
preferences towards either mode. The preference for dubbing is indicated
by a single pupil, while the other two modes were selected by two pupils
each. It is noteworthy that two children selected translation, a written
activity that is not directly relevant to their lived experience, rather
than the other two modes, which are inherently more visual.

\begin{figure}[htbp]
    \centering
    \begin{minipage}{.5\textwidth}
    \includegraphics[width=\textwidth]{fig04.png}
    \caption{Question 4. Which tool have you liked the most?
    (Green: interlingual indirect translation of texts; purple: dubbing;
    blue: subtitling).}
    \label{fig-04}
    \source{Owm elaboration.}
    \end{minipage}
\end{figure}


In question 5, the children were asked about the process of language
acquisition. As illustrated in \Cref{fig-05}, all pupils demonstrated an
understanding that they would benefit from these educational
initiatives. It is noteworthy that five of the pupils awarded the
project an 8, a high rating that reflects their trust in DAT. One pupil
considered the quality to be satisfactory, while one rated it slightly
above average. Overall, the marks are deemed satisfactory, as none of
the responses were below the passing mark of 5. These responses are
consistent with those provided in question 2, in which two children
indicated a lack of knowledge regarding AVT.

\begin{figure}[htbp]
    \centering
    \begin{minipage}{.5\textwidth}
    \includegraphics[width=\textwidth]{fig05.png}
    \caption{Question 5. From 1 to 10, how much do you think you
    would learn?}
    \label{fig-05}
    \source{Owm elaboration.}
    \end{minipage}
\end{figure}

The final question (number 6) asked the pupils whether they would
like to engage in DAT activities within the classroom setting. Five
pupils responded in the affirmative, while the remaining two expressed
opposition to the proposal. However, their responses lacked sufficient
clarity or referenced their disapproval of the instructor rather than
the DAT activities themselves. It can be inferred that these two
students are the same individuals who, in question 1, rated the workshop
atmosphere as 6, who provided negative responses to question 3 regarding
their enjoyment along the process, and who rated the development as 6
and 7. This is somewhat surprising, given that these ratings are not
particularly low.

Question 6 (original answers in \Cref{annex-03})

\begin{itemize}
    \item \textbf{Pupil 1}: Yes, because I learn.
    \item \textbf{Pupil 2}: Yes, because I like making videos a lot.
    \item \textbf{Pupil 3}: Yes, because I have never tried it and I think it would be good.
    \item \textbf{Pupil 4}: Yes, because I like recording videos a lot.
    \item \textbf{Pupil 5}: Yes, because it is fun.
    \item \textbf{Pupil 6}: No, because I do not want X (a teacher) to appear.
    \item \textbf{Pupil 7}: No.
\end{itemize}

\section{Discussion}\label{sec-discussion}

In regard to the learning outcomes, it is not feasible to assess the
acquisition of the L2 content. However, we have identified recurring
errors in writing and pronunciation that illustrate potential areas for
guidance to facilitate pupil improvement in the language. Additionally,
the children committed mistales in pronunciation and intonation.
Although these issues require resolution, it is acknowledged that
mispronunciations of this nature are common among speakers of Basque and
Spanish, given the inherent challenges posed by the English vocalic
system for those with a limited number of vocalic phonemes.

Regarding the findings of the questionnaire, the outcomes are consistent
with our expectations. The pupils have demonstrated predisposition
towards this pedagogical approach, which integrates stimuli that align
with the fundamental principles of Montessori, namely the diverse
classroom environment to which they are accustomed. It is also
noteworthy that the method incorporates multimodality, which is a key
aspect of the world in which children live. The final products
demonstrate that pupils can successfully handle the L2. This is a
significant finding, as \textcite{alonso-perez2018} have
demonstrated that observing the final outcome and analysing the progress
that the students have achieved ``make everyone feel rewarded'' \cite[p. 21]{alonso-perez2018}, which implies an even stronger boost to motivation.

Conversely, children have demonstrated interest and engagement with the
DAT activities. A recurrent 5-2 parameter is evident when analysing
questions 3, 5 and 6. This signifies that five of the seven children
indicated a high level of enjoyment, with ratings of Bs and As, while
the remaining two children achieved a passing grade of C+. It is
accurate to conclude that the highest grade awarded was not an A+, which
can be attributed to the fact that this task was compulsory and required
the learning of internet tools, which served this purpose.

In examining the reasons for the implementation\textquotesingle s
success or failure, the indicators recur. It is presumed that the two
children who provided the lowest ratings were those who responded in the
negative. The issue is that the respondents did not provide reasons
related to their experiences with DAT activities. One respondent did not
provide any reasons, while the other provided a single negative response
regarding one of the teachers. This is a mishap, as their answers would
have been of great help for the research. Conversely, the remaining five
respondents provided positive feedback, citing enjoyment derived from
video recording, fun during implementation, and, most notably, two
respondents demonstrated appreciation for the activity\textquotesingle s
didactic potential, stating, "It would be good" and "I learn." This
latter response is particularly noteworthy, as it reflects a depth of
reflection that not all children possess. These statements are in
accordance with \textcite{fernandez-costeles2021} regarding the perceptions of
primary education pupils towards the didactic possibilities of DAT.

In light of the previous considerations, it can be posited that the
findings of this survey indicate a favourable outcome. In terms of
learning, this implementation has opened paths for the teacher to work
those aspects of language which have proven to be less developed in
learners. Conversely, with regard to motivation, the utilisation of DAT
has enhanced students motivation, as it is founded upon certain
customary activities among children, such as the recording of videos.
This results in a more enjoyable process of learning the L2, which
undoubtedly facilitates the work of both the teacher and the learners.
This research contributes to the existing body of literature on the
subject, building upon the findings of previous studies conducted by
scholars such as \textcite{neves2004language}, \textcite{talavan2009aplicaciones,talavan2010audiovisual}, \textcite{banos2015clipflair}, \textcite{talavan2015first}, \Textcite{BELTRAMELLO_2019}, \textcite{lertola2019audiovisual}, \textcite{talavan2022audiovisual}, \textcite{rodríguez-Arancón2023} and \textcite{talavan2024}.

\section{Conclusion}

This research has explored the relationship between the Montessori Method and DAT, demonstrating that both have a significant degree of overlap and that the latter can be integrated as an additional workshop within the Montessori classes. We encountered two main limitations. The first limitation is inherent to the methodology itself. Given that children do not take exams, it is not possible to adhere to the experimental scheme that would otherwise be appropriate for this type of research. Consequently, it has not been possible to measure the anticipated improvement in pupils' language acquisition.

A second limitation of the study is the low response rate to the satisfaction questionnaire, with only seven out of twenty-four students completing it. This is a potential issue, as the sample size is small and the responses are not fully representative. However, they do offer insights into a particular trend. Among the pupils who responded, we have been able to ascertain their motivations and levels of enjoyment regarding the implementation of the DAT. Five of them indicated that they found it motivating, while the other two expressed differing pedagogical opinions. The primary issue that emerged from the data was that learners lacked clarity regarding the specific applications of DAT.

In this particular instance, a class of twenty-four children at an educational institution that employs active methodologies and Basque as the L1 have demonstrated the efficacy of DAT from several perspectives. From the viewpoint of the teacher, the role differs from that of the traditional educator. DAT allows teachers to provide pupils with materials that align with their interests, thereby personalising their learning experience. Although pupils are permitted to select activities that exceed their current capabilities, the provision of effective scaffolding can facilitate their learning process. Furthermore, Montessori advocated the use of self-correcting materials, a concept that is exemplified by DAT. Audiovisual materials allow learners to listen to themselves and receive immediate feedback on their performance, which undoubtedly contributes to the improvement of their oral proficiency in the L2. DAT presents new opportunities for stimulating pupils, given its multimodal nature. It offers language in authentic contexts, which helps pupils comprehend the meaning of language in real situations. Thus, by providing a new lexical variety, children can expand their vocabulary to express themselves in different situations. Lastly, the integration of technology and multimodality has enabled learners to personalise their learning experience, allowing them to progress at their own pace and receive education that is specifically tailored to their needs. This is achieved through the creation of new teaching tools, which are based on interactive games, simulations or educational videos. In this regard, DAT has emerged as a highly valuable instrument in the context of this pedagogical approach, facilitating more engaging and efficacious learning, while fostering autonomous learning and self-regulation. 

Importantly, the implementation of DAT is in accordance with both the fundamental principles of the Montessori Method and current legislation. It has been demonstrated that DAT effectively implements the competencies and skills required by law with regard to the teaching of the L2. Furthermore, it facilitates the development of students' cognitive abilities, which is fundamental for lifelong learning.

Future research could encompass a longitudinal study to include an analysis of the role of multilingualism (learners work with three languages: Basque L1 of some pupils and of school, Spanish L1 of some children, and English as L2) in the general learning process and in the Montessori Method.

\printbibliography\label{sec-bib}

\appendix

\section{Motivation Survey Questions}\label{annex-01}

\begin{enumerate}
  \item From 1 to 10, how much have you enjoyed the ambiance of the workshop?
  \item Did you know what audiovisual translation was?
  \item Have you enjoyed learning English by means of video editing?
  \item Which tool have you liked the most?
  \item From 1 to 10, how much do you think you would learn?
  \item Would you like to do this type of exercises in class? Why?
\end{enumerate}

\section{Assessment Rubrics}\label{annex-02}

See \Cref{fig-annex} on page \pageref{fig-annex}.

\begin{figure}[h!]
  \centering
  \includegraphics[width=\textwidth]{figure-annex.png}
  \caption{Assessment Rubrics.}
  \label{fig-annex}
  \source{Owm elaboration.}
\end{figure}

\section{Original Answers to the Questionnaire.}\label{annex-03}

Honelako ariketak egitea gustatuko litzaizuke? Zergatik?
\begin{enumerate}
  \item Bai, ikasiko nuelako.
  \item Bai, zeren asko gustatzen zait bideoak egitea.
  \item Bai, inoiz ez dudalako probatu eta ongi egongo zela uste dut.
  \item Bai, zeren asko gustatzen zait bideoak grabatzea.
  \item Bai, dibertigarria izango zelako.
  \item Ez, nahi dudalako Enara azaltzea.
  \item Ez.
\end{enumerate}


%conceptualization,datacuration,formalanalysis,funding,investigation,methodology,projadm,resources,software,supervision,validation,visualization,writing,review
\begin{contributors}[sec-contributors]
\authorcontribution{Edurne Goñi-alsúa}[conceptualization,methodology,review,resources,writing,visualization,supervision,projadm]
\authorcontribution{Galder Rejas Vicente}[formalanalysis,methodology,review,writing,investigation,datacuration,projadm]
\end{contributors}
\end{document}
