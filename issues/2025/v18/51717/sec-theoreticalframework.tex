\section{Theoretical Framework}\label{sec-theoreticalframework}

\subsection{Audiovisual Translation and Didactic Audiovisual Translation} \label{sub-sec-audiovisualtranslation}

As posited by \textcite{diaz-cintas2012subtitulos}, Audiovisual Translation (AVT) can be
defined as the process of transferring the cultural and linguistic
content of an audiovisual medium into another language. In doing so, it
is essential to consider the restrictions inherent to the medium in
question and to respect the original communicative purpose. Prior to
this definition, \textcite{diaz-cintas2009new} expounded that AVT had become a
crucial means of disseminating audiovisual content, not only on an
international scale, but also to ensure the accessibility of multimodal
content to diverse audiences.

The three principal modes of AVT are subtitling, dubbing and audio
description. In their study, \textcite{talavan2023didactic} define subtitling as the translation of a dialogue
into written form, text which is then placed at the base of the screen.
Moreover, dubbing entails the replacement of the original dialogue with
that in the target language, ensuring that the translated audio
syncronises with the movements of the actors' lips. As
a final point, audio description aims "to make the visual content of an
event accessible by conveying it into spoken words" \cite[p. 246]{ibanez2016audiodescription}.

Conversely, \textcite{talavan2019traduccion} defines Didactic Audiovisual Translation
(DAT) as the didactic application of AVT procedures to the teaching of
the L2. This researcher posits that DAT has its roots in the 1980s, when
researchers and scholars began to recognise the potential of this
approach to language teaching and started to incorporate subtitles as a
tool in language laboratories. Furthermore, \textcite{fernandez-costales2023tradilex} add that students must utilise technology (new tools
available on the internet) and employ the specific strategies associated
with each mode.

As early as 2009, \citefirstlastauthor{diaz-cintas2009new} emphasised the significance of DAT in L2
instruction, noting that it enables learners to engage with native
speakers' productions, thereby facilitating a more
genuine encounter with the L2 and its associated culture. In this way,
DAT has demonstrated to be an effective tool, as it engages learners by
providing them with authentic audiovisual materials, which allows to
immerse themselves not only in the language, but also in the target
culture. \textcite{fernandez-costeles2021} concurs with this assertion, having
conducted research into the effects of DAT on the learning of the L2 in
a primary education environment.

Additionally, \textcite{chaume2018audiovisual} emphasised that this approach facilitates a
more profound comprehension of the L2 through the combination of
auditory and visual input, thereby enabling learners to discern the
authentic, natural usage of language by native speakers. In this regard,
\textcite{gonzalez-vera2016audiovisual} illustrate that DAT facilitates more
effective learning by offering immediate feedback on pronunciation and
grammar, promoting more rapid language development.

\subsection{Didactic Dubbing and Subtitling}\label{sub-sec-didacticdubbing}

The combination of didactic dubbing and subtitling facilitates the
integration of the four linguistic skills. The primary focus of dubbing
is on oral skills, namely listening comprehension and speaking, whereas
subtitling is more closely aligned with written skills, namely reading
comprehension and writing. In addition to the aforementioned skills, we
can include the fifth skill, translation \cite{Carreres03072017}.

Regarding dubbing, \textcite{diaz-cintas2012subtitulos} mentions that students benefit
from exposure to a diverse range of dialects and accents. Furthermore,
it capacitates learners to enhance their oral comprehension and speaking
abilities concurrently, while fostering motivation through the
incorporation of a multimodal component. Moreover, it facilitates the
acquisition of the L2 in an authentic and natural manner
\cite{fernandez-costales2023tradilex}, as students engage with
genuine oral language in authentic contexts, diverging from the
structured materials typically provided in textbooks.

The benefits of subtitling have also been the subject of academic
investigation. \textcite{ÁlvarezSánchez_2017} asserts that the utilisation of
subtitles not only fosters the development of linguistic skills but also
upgrades the comprehension of paralinguistic and cultural elements. It
also encourages self- and cooperative learning, with learners at the
core of the learning process. Likewise, she explains that audiovisual
media reflects a multitude of communicative scenarios, thus facilitating
comprehension of an oral language text through the utilisation of
non-linguistic elements. Adding to this, \textcite{lertola2018translation} elucidates that
the advantages of employing subtitling are analogous to those of
translation. However, there is an additional benefit, as learners are
not merely translating a text from the source language to the target
language; they are also exposed to audiovisual material, which enables
them to observe and listen to authentic communication scenarios.
Additionally, \textcite{soler2020} states that subtitling enhances
vocabulary acquisition, provides motivation, facilitates productive
abilities (specifically, speaking and writing), ameliorates the L1, and
strengthens attention skills. As will be demonstrated, these
characteristics are consistent with the communicative approach that is
intended in Spanish L2 curricula.

Two previous studies have examined the combination of dubbing and
subtitling in L2 classes. \textcite{talavan2015first} put forth an
implementation of reverse subtitling and dubbing with the objective of
"improving oral and written production skills, along with general
translation competence" \cite[p. 169]{talavan2015first}. Their approach yielded highly
promising outcomes in terms of language acquisition. Similarly,
\textcite{BELTRAMELLO_2019} conducted an in-class implementation in accordance
with the aforementioned parameters. In her conclusions, she states that
the combination of the two modes is effective because, in addition to
facilitating language acquisition, students are exposed to "pragmatic
phenomena" \cite[p. 106]{BELTRAMELLO_2019}. In the process of dubbing, learners are required to
direct their attention towards the objectives of the speaker and the
utilisation of specific expressions, rather than others. Furthermore,
students have the opportunity to practise pronunciation, intonation, and
some paralinguistic aspects of language, thereby improving their
fluency. According to this scholar, ``it seems to reveal an untapped
potential in the combination of subtitling and revoicing as an aid to
language learning that offers abundant possibilities to practice and
develop different areas of language competence, such as pragmatic
competence'' \cite[p. 6]{BELTRAMELLO_2019}.

In light of the above, it can be posited that working with both DAT
modes concurrently represents an effective pedagogical approach. This
combination capacitates learners to simultaneously develop their
comprehension and production skills. Additionally, it improves
motivation, as students engage in a multimodal environment with
controlled exposure to authentic language usage in authentic contexts.

\subsection{The Montessori Method}\label{sub-sec-themontessorimethod}

The aim of developing innovative pedagogical approaches that align with
the needs of learners can be traced back to the 19th century. In
conjunction with the advent of new psychological theories, scholars
understood the necessity for a more individualised approach to teaching,
with a focus on the needs of the learner. Among the approaches that
emerged in this context were the Montessori Method or the Scientific
Pedagogy method, which was based on Maria Montessori\textquotesingle s
observations that children were capable of learning independently,
without the direct supervision of adults \cite{pla2007}.

\textcite{lillard2013playful} points that the Montessori Method is an educational
approach with the aim of promoting the holistic development of the
child. Its fundamental premise is the belief that children are innately
inclined towards learning. Consequently, educators must establish an
engaging and nurturing setting, provide individualized guidance through
experiential learning opportunities and encourage self-directed learning
by means of a blend of autonomy, suitable educational resources and
self-discipline \cite{marshall2017montessori}. Moreover, \textcite{pla2007}
assert that this method is founded upon four principal tenets: preparing
children for life, fostering a conducive learning environment,
refraining from undue interference in the learning process, and
providing sensorial materials to improve sensory development.

In addition, \textcite{marshall2017montessori} maintains that this method relies on the
development of two essential components: emotional and social skills.
The first seeks to ameliorate the emotional skills that help children
cope with the challenges of everyday life. The second focuses on
cooperation, empathy and mutual respect among students, and is performed
by activities which require cooperation. This way, learners build
relationships with their peers.

\textcite{montessori1937metodo} challenged the prevailing pedagogical trends of the
time, advocating for a teaching approach that empowers children to learn
through independent action. She emphasised the importance of educators
obtaining precise and logical observations of children, which serve as
the foundation for their instructional decisions. Accordingly, the
pedagogical approach is founded upon exploration and discovery, wherein
pupils progress within a tranquil, respectful, and structured milieu
with the objective of fostering autonomy. This signifies a shift in the
role of the teacher, moving from a director who oversees both
children\textquotesingle s learning and behaviour \cite{denervaud2019} to a supportive figure who establishes a
nurturing environment, in which pupils are able to flourish and reach
their full potential \cite{marshall2017montessori}. Pupils are encouraged to work
independently at their own pace in an environment replete with tangible
materials designed to promote discovery, exploration, and, most
crucially, creativity \cite{marshall2017montessori}. \textcite{denervaud2019} emphasise the necessity for these materials to be
self-correcting, thus enabling children to learn through trial and
error.

Additionally, \textcite{pla2007} claim that it is imperative
to respect the rhythms of pupil growth, requiring that teachers adapt
the content and devise individualised plans in accordance with the pace
of each child. Moreover, the aforementioned plans must be aligned with
the child\textquotesingle s interests, while also fostering
self-discipline. Furthermore, the learning proposals must develop five
key aspects: practice, imitation, repetition, classification and order.

It is therefore necessary to implement changes to the layout of the
classes and the grouping of the pupils. In the first case, \textcite{pla2007} explain that the classrooms must be reorganised, with the
elimination of desks, banks and class platforms for teachers, and the
adaptation of furniture to the height and strength of children. This
signifies that the classrooms are constituted as a structured framework,
which facilitates access to the materials and delineates specific spaces
to play, speak, rest and listen. Secondly, \textcite{lillard2013playful} points that
pupils should be distributed in mixed-age classrooms, separated by a
range of three years: infants to three-year-olds, three to
six-year-olds, six to nine-year-olds, and from nine to twelve-year-olds.

This approach is founded upon the premise that children learn through
active engagement \cite{pla2007}. Consequently, didactic
workshops represent an efficacious instrument for the organisation of
educational activities. Among the advantages is the fact that workshops
facilitate hands-on teaching, which helps to maintain
students' interest and attention. Moreover, workshops
can be tailored to the specific abilities and requirements of each
pupil, thereby optimising the learning process. Additionally, workshops
facilitate teamwork and creativity, as children can engage in
collaborative activities. Nowadays, these learning environments have
evolved to encompass both physical spaces, such as science laboratories
or kitchens, and virtual environments. Nevertheless, the underlying
rationale remains consistent: learners engage in experiential learning
activities to apply the concepts they have acquired, thereby developing
the practical skills they will require in their future endeavours.

Moreover, this approach places an emphasis on the cultivation of
critical thinking abilities throughout the learning process \cite{murray2011montessori}. Pupils are encouraged to engage in exploration, discovery and
questioning to gain a deeper understanding of their surrounding
environment, which provides numerous opportunities for mutual assistance
\cite{pla2007}. The efficacy of this approach has been the
subject of considerable investigation, with \cite{lillard2013playful} demonstrating
its advantages in domains such as cognitive development, academic
achievement and self-discipline.

Some of the objectives of this approach are aligned with those of DAT.
On the one hand, both are based on the development of
students' thinking skills. According to \textcite[p. 214]{ghaffari2017montessori}, "the process of independent
problem solving creates self-confidence and critical thinking skills".
Translation is categorized alongside the highest order thinking skills,
according to Bloom's Taxonomy. Conversely, both methods
encourage collaborative work, as the majority of DAT activities are
conducted in pairs or groups. Lastly, both methods facilitate
learners' autonomy in learning, allowing them to work
at their own pace while the teacher serves as a guide, providing access
to learning resources.

\subsection{Spanish Legislation on Education and DAT}\label{sub-sec-spanishlegislationoneducationanddat}

The latest Law on Education \cite{lomloe2020} and the Organic Law on the
Foral Decree 67/2022 of the Foral Community of Navarre establish the
compulsory curriculum for all schools in the region. With some additions
due to the particularities of the Community, the latter specifies what
the former decrees.

In regard to the teaching of the L2, it is asserted that the focus
should be on the development of oral skills, with educators providing a
range of tools to facilitate student engagement with the content.
Moreover, educators must promote instrumental learning, which enables
learners to develop other competencies and meaningful learning, by
integrating the different competences in the execution of projects, so
that students solve problems cooperatively, what strengthens autonomy,
reflection and responsibility. Additionally, the L1 will be employed
solely as a support tool in the acquisition of the L2. Finally, learners
must allocate time on a daily basis to audiovisual communication and
the promotion of creativity and scientific inquiry.

The legislation requires educators to provide multiple tools to
encourage learner engagement. The world in which students live is
multimodal. Video clips, online games, films and TV series constitute
the majority of their leisure activities. It could therefore be assumed
that introducing these new languages into the classroom will make
learners feel more at ease, which will in turn improve their motivation
and, consequently, the acquisition of the L2.

Undoubtedly, DAT promotes instrumental learning as students work with a
variety of inputs, including text, audio and the cultural contexts of
multimodal texts, and have to manipulate the language in order to align
it with the image. In addition, it strengthens autonomy and reflection,
as learners can work independently or in collaboration with others,
which encourages them to problem-solve collectively and improves their
sense of responsibility. Finally, it integrates a range of competencies,
including linguistic communication, plurilingualism, digital
proficiency, personal and social skills, the capacity to autonomous
learning, consciousness, and cultural expressions \cite{Bobadilla-Pérez_CarballodeSantiago_2022,rodríguez-Arancón2023}, thereby implying
life-long learning, which is a fundamental aspect of the current
teaching and learning process.

Although the current legislation does not contemplate the use of the L1
and of translation in the class, \textcite{Carreres03072017,Colina03072017} have suggested that
translation should be considered as the `fifth skill', which ``can be
used as a pedagogical tool to integrate the original four skills to
enhance second language study'' \cite[p.~2]{Colina03072017}.
