\section{Introduction}\label{sec-introduction}

Over the past decades there has been a growing interest in the
development of new methodologies for second language (L2) learning and
teaching, to fulfil the requirements of the new legislation, as defined
by the Common European Framework of Reference for Languages (CEFR),
which emphasizes the communicative aspect of languages.

One of these new methodologies is Didactic Audiovisual Translation
(DAT), i.e., the application of different modes of audiovisual
translation (subtitling, revoicing, audio description and voice over) to
L2 teaching. This approach has been implemented and its results analysed
in different educational stages. Nevertheless, this approach has yet
been used in primary education within alternative methodologies, such as
the Montessori Method, in blingual contexts (in this case,
Basque-Spanish). The present study aims to address this research gap.

This paper shows the results of an intervention combining subtitling and
dubbing in a class of 11-12-year-old pupils following the Montessori
Method, at a Basque-immersion language primary school. While our study
could not assess language acquisition improvements due to the
school's methodology, our findings regarding children
satisfaction and the alignment of DAT principles with Montessori
principles, as demonstrated in the text, suggest promising avenues for
successful DAT use in this environment.


