\section{Proposal}\label{sec-proposal}

The research conducted was driven by two main questions, intended to
complement existing studies, and to gain insights into two key areas:
the first refers to the teaching-learning aspect, and the second to the
enjoyment of the pupils.

\begin{itemize}
    \item Does the implementation of a DAT didactic unit, based on the
    combination of interlingual indirect subtitling and dubbing (L1 Basque)
    enhance the results on acquisition of English language?

    \item Is DAT a motivational method for students of primary education
    following alternative schooling methodologies?
\end{itemize}

\subsection{Participants}\label{sub-sec-participants}

This implementation was conducted in a primary education school situated
in a middle-class neighbourhood of Pamplona, Spain, which follows the
Montessori Method, whose L1 is Basque. While not all pupils who
participated in this research speak this language at home, all classes
are taught in Basque. In the area, there is another school with two
lines of L1, Basque and Spanish, and a high school with the same L1
paradigm.

This school is committed to an active approach to learning, utilising
Montessori-inspired materials and a class structure that incorporates
learning environments and project-based learning as the primary
pedagogical methodology. The aforementioned pedagogical approach is
employed with the objective of fostering the development of
self-regulation and autonomy in students, as well as enhancing their
capacity for collaborative work through the utilisation of flexible
group structures and other active educational resources.

A total of 24 pupils, comprising the 6th course of primary education and
aged between 11 and 12 years old, participated in the project. The
pupils were divided into groups of four and worked together throughout
the implementation period. The participants demonstrated a level of
English proficiency corresponding to A1, as defined by the Common
European Framework of Reference for Languages (CEFR). Parents were
informed and gave their consent to the project, which has been approved
by the corresponding committee of UPNA.

\subsection{Materials}\label{sub-sec-materials}

During the implementation, learners were provided with the scripts of
the scenes and employed the class computers, dictionaries (both physical
and online), voice recorders and the editing app VideoPad
(\url{https://www.nchsoftware.com/videopad/es/index.html}). Finally, the
pupils completed a questionnaire (\Cref{annex-01}) to ascertain their level of
satisfaction with the implementation process, allowing for both
quantitative and qualitative results to be gathered.

\subsection{Design}\label{sub-sec-design}

One of the distinctive characteristics of the Montessori approach is the
absence of conventional testing and examination procedures.
Consequently, the implementation did not adhere to the customary
experimental pre-test--post-test design. Nevertheless, the instructor
could delineate certain guidelines to enhance the acquisition of the L2,
based on the errors that students had committed throughout the sessions.

A further aspect of this approach is that learners must be free to
select their own areas of interest and determine their own learning
strategies. This is reflected in the fact that they spent two sessions
searching for a TV series and a particular scene that they wished to
translate, dub and subtitle. As this was the first time that such a
project was implemented in the school, the instructor elected to develop
it in the Kitchen Corner, the designated workspace intended for the
duration of the internship.

The initial proposal was to implement a didactic unit on subtitling.
However, the instructor determined that pupils could undertake a
combined project encompassing both dubbing and subtitling, given the
absence of time constraints typically encountered in traditional
educational settings. This alteration has furnished the implementation
with an unanticipated degree of depth. As a result of this expansion,
the final products created by the students were the selected scenes
translated from the L1 to the L2, and both dubbed and subtitled. The
project was developed over four days, in sessions of 45 minutes each.
The subsequent schema was followed:

\begin{itemize}
    \item Choice of TV series and scene: two sessions
    
    \item Transcription and translation of the script: two sessions
    
    \item Audio recording: three sessions
    
    \item Scene editing: three sessions
\end{itemize}

\subsection{Procedure}\label{sub-sec-procedure}

\subsubsection{Day 1. The choice of series and scenes}

The session commenced with the activation of prior knowledge through the
posing of questions to the pupils, such as "What are your favourite TV
series?" This enabled the teacher to ascertain the
learners\textquotesingle{} interests, thereby facilitating the
personalisation of the learning process in accordance with their
preferences. A further crucial element of these two sessions was the
formation of flexible groups, comprising learners with diverse
abilities, so that they could learn from one another. The pupils elected
to work with their preferred scenes from the television series
\emph{Goenkale}, which features teenagers and is set in the Basque
Country. It is broadcast on \emph{Euskal Telebista} 1 (ETB1). This is a
highly popular programme with a particularly strong viewership among the
adolescent demographic. Prior to the DAT practice session, the children
were required to imitate both the intonation and rhythm of the speech of
the character they were to dub.

\subsubsection{Day 2. Transcriptions and translations}

The first session comprised an examination of the scene and an endeavour
to identify a suitable transcription. The pupils were initially required
to view the scene on multiple occasions, with the objective of
facilitating an in-depth comprehension of the nuances embedded within
the dialogue. This approach was designed to facilitate a more
straightforward engagement with the language.

Following the viewing, the children conducted online research to find
the transcriptions of the dialogues, utilising a range of digital
platforms. In the event that the desired material was not accessible,
the pupils utilized YouTube, as it offers subtitles for a number of
videos. They then proceeded to transcribe the scene by manually copying
the subtitles.

In the second session, the children completed the transcription and
translated it from L1 to L2 using both physical and online dictionaries.
The instructor directed the students\textquotesingle{} attention to the
fact that language can convey double meanings and transmit cultural
expressions that, when translated literally, are devoid of meaning. The
teacher\textquotesingle s main role was to provide assistance. She
clarified doubts and illustrated the lexicon, ensuring that all learners
were able to comprehend the message conveyed in the script.

\subsubsection{Day 3. Audio recording}

In the initial session, the students engaged in practice activities,
which entailed reading the transcriptions in both languages and
commencing the recording of the L2 audio. In the subsequent sessions,
the children proceeded with the audio recording.

During the audio recording sessions, the pupils concentrated on
developing their pronunciation and fluency. As the teacher consistently
reiterated the importance of aligning the spoken text with the
corresponding mouth movements, the pupils undertook a process of review
and editing of their recordings, with the objective of ensuring that
they met the desired quality standards. In order to achieve this, it was
necessary for the children to consider the previous work, in order to be
able to adjust the text in the L2 to the rhythm of the original scene
and to align the audio with the video. In addition, pupils were required
to become proficient in the use of the VideoPad application in order to
gain familiarity with its various functions.

The final stage of the process involved the deletion of the audio from
the original scene and the insertion of the new audio files, which were
then adjusted to the rhythm of the scenes. The children proceeded to
implement the final modifications to their product. The instructor
offered feedback and guidance to assist the pupils in enhancing their
productive oral skills.

\subsubsection{Day 4. Scene subtitling and editing}

In the initial session, the pupils were instructed in the techniques of
video editing and were given the opportunity to improve their writing
skill as they worked on the subtitles. In the remaining two sessions,
the children continued subtitling to create the final product, which was
a scene that was both dubbed and subtitled. The software utilized by the
pupils was once more VideoPad, which also facilitates the creation of
subtitles. By the conclusion of the sessions, children had produced
videos with high-quality audio and accurate subtitles.

\subsection{Assessment}\label{sub-sec-assesment}

Upon completion of the implementation phase, both the pupils and the
teacher conducted an assessment of the various projects using an
evaluation rubric (\Cref{annex-02}). Children were evaluated on five distinct
criteria: translation accuracy, coherence between the visual and
auditory content, adequate use of language and grammar, cultural
adaptation, and creativity and originality.

This assessment rubric was devised for use by both teachers and
learners. The teacher subsequently provided a more detailed assessment
in each category, including additional commentary on the areas that
require improvement and on those in which the pupils had demonstrated
proficiency.
