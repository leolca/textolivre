\section{Processamento da ambiguidade linguística e o processamento da linguagem natural}\label{sec-referencial-teorico}

O processamento semântico é um domínio de investigação fundamental para os modelos linguísticos pré-treinados e dos grandes modelos de linguagem. No campo das tarefas de processamento semântico, a desambiguação do sentido das palavras demanda parâmetros definidos. No entanto, o estudo da ambiguidade é multidimensional na linguística, como veremos a seguir.

Esta pesquisa fundamenta-se em disciplinas que estão interligadas, demonstrando a natureza interdisciplinar das áreas envolvidas. Portanto, o estudo aproveita-se dos princípios teóricos relacionados à ambiguidade linguística (Seção \ref{sec-ambiguidade-linguistica}) e da aplicação dos modelos computacionais de grande escala, ChatGPT e Gemini (Seção \ref{sec-modelos-linguagem}).


É importante destacar que o arcabouço teórico sobre ambiguidade linguística, empregado nesta seção, servirá como alicerce para a construção do nosso conjunto de sentenças. Entre várias abordagens teóricas disponíveis, decidimos adotar a perspectiva de Cançado \cite{canccado2005manual} dada a sua compatibilidade com a metodologia da pesquisa e o objeto de estudo, os modelos de linguagem, que não se valem dos aspectos de interlocução para compreender ambiguidade, mas focam nos componentes linguísticos do enunciado. Dessa forma, a fundamentação dos princípios de ambiguidade dessa taxonomia permitiu que os dados fossem analisados de forma mais objetiva, evitando depender de fatores contextuais externos. Além disso, há trabalhos bem recentes de outros países que fazem estudos do ponto de vista computacional de análise de ambiguidade mantendo uma coesão similar com aspectos da teoria da Cançado em termos de ambiguidade lexical, semântica e sintática \cite{haber2021patterns,ortega2023linguistic,liu2023we,goel2023beyond}.



\subsection{Ambiguidade linguística}\label{sec-ambiguidade-linguistica}


A ambiguidade é um fenômeno semântico no qual uma palavra, expressão ou sentença pode ter mais de uma interpretação válida. A resolução desse tipo de fenômeno depende fortemente do contexto, que orienta a escolha do sentido adequado entre os que possíveis. No entanto, os modelos de linguagem, por apresentarem restrições na compreensão dos componentes multimodais presentes no contexto, não captam sinais extralinguísticos, fundamentais para processar ambiguidade ou desambiguar enunciados.

Historicamente, a ambiguidade é uma questão antiga e amplamente estudada em tarefas de PLN através de análises simbólicas de sentenças, como o parsing sintático e semântico \cite{church1982,koller2008} ou resolução de correferência \cite{poesio2005}. Entretanto, com o avanço recente no uso de redes neurais profundas e modelos de linguagem pré-treinados, o foco do campo tem se deslocado para problemas de compreensão em níveis mais altos, incluindo o raciocínio e a geração de texto. Nesse novo contexto, a ambiguidade continua sendo um desafio já que os modelos aprendem a lidar com ambiguidade de forma implícita, por meio de grandes quantidade de dados de treinamento.

Seguimos a proposta de \textcite{canccado2005manual} para o português brasileiro, que considera o nível de uso da língua e inclui a ambiguidade lexical, com casos de homonímia, polissemia, ambiguidade semântica e ambiguidade sintática.

A \textbf{ambiguidade lexical} compreende uma sentença com  dupla interpretação incidente em um item lexical, podendo ser gerada por homonímia ou polissemia. A homonímia se dá quando os sentidos do item lexical não são relacionados, como na oração ``Eu estou indo para o banco'' em que a palavra ``banco'' possui significados diferentes, pode corresponder à instituição financeira e ao assento. Já a polissemia ocorre quando os sentidos do termo identificado como ambíguo contém relação entre si, por exemplo na sentença ``O Frederico esqueceu a sua concha'', neste caso ``concha'' pode significar uma concha do mar ou uma concha de cozinha, em que ambos os objetos possuem o mesmo formato, e por isso, pode ocorrer uma associação polissêmica, o mesmo fenômeno ocorre com palavras como: rede (de internet, de deitar, de pescar) e pilha (de comida, de bateria).

Além disso, na taxonomia de \textcite{canccado2005manual}, a ambiguidade lexical também pode ser causada por meio de preposições, classificando-se assim como ambiguidade preposicional. Por exemplo, na sentença \enquote{O burro do Paulo anda doente} permite duas interpretações: \enquote{burro} pode se referir ao animal que Paulo possui, caracterizando uma ambiguidade literal, ou pode ser uma expressão figurativa, referindo-se a Paulo como sendo uma pessoa \enquote{burra}. Essa dupla interpretação ocorre por causa da homonímia, onde a palavra \enquote{burro} tem dois significados não relacionados que são possíveis de serem inferidos por conta da presença da preposição. Esse tipo de ambiguidade preposicional se soma às outras duas subclassificações de ambiguidade lexical identificadas por Cançado que são tratadas na pesquisa, homonímia e polissemia.

Embora a ambiguidade preposicional seja uma categoria relevante, o conjunto de dados criado para testar os modelos neste trabalho se restringiu a casos de polissemia e homonímia, excluindo assim a ambiguidade gerada por preposições, mas esses casos apareceram na Tarefa 4 de geração de frases por parte dos modelos.

Nessa mesma esteira teórica, Lyons \cite{lyons1977semantics} considera a homonímia e a polissemia como casos de ambiguidade lexical, assim como \cite{canccado2005manual}. A homonímia ocorre quando dois ou mais significados não possuem relação semântica entre si, sendo historicamente distintas no desenvolvimento da língua, o que reforça a independência dos sentidos em termos lexicais. Já a polissemia envolve uma relação intrínseca entre os diferentes sentidos de uma palavra, onde esses sentidos compartilham um núcleo comum de significado. Assim, no exemplo \enquote{banco},  os diferentes sentidos (assento e instituição financeira) são homônimos, uma vez que suas origens etimológicas e semânticas são divergentes. Por outro lado, no caso de \enquote{concha}, os diferentes usos podem ser entendidos como polissemia, já que ambos os sentidos remetem a uma semelhança de forma e função, refletindo a natureza interconectada das diversas acepções do termo.

É importante destacar que essa distinção de subcategorias é fundamental, visto que, para o português, a ambiguidade lexical é a que apresenta mais recursos descritivos para suporte computacional \cite{laporte2001resoluccao}, necessário aos LLMs.

 
Na taxonomia de Cançado, a \textbf{ambiguidade semântica} é abordada como uma questão de correferencialidade, em que os pronomes podem ter vários antecedentes. Consideremos, por exemplo, a seguinte frase: ``José falou com seu irmão?'' Esta sentença ilustra claramente esse tipo de ambiguidade, na qual não é possível determinar se o irmão mencionado é o irmão de José ou o irmão da pessoa para quem a pergunta é dirigida, ou seja, um terceiro elemento. Nesse contexto, as interpretações possíveis são atribuídas à natureza da ligação entre os pronomes presentes na sentença.

A gramática gerativa compartilha da mesma percepção sobre pronomes de referência como fator de ambiguidade. \cite{chomsky1981lectures} trata da correferencialidade por meio do conceito de \textit{binding}, que estabelece princípios para a associação entre pronomes e seus possíveis antecedentes dentro de uma estrutura sintática. A interpretação de um pronome é, portanto, condicionada por sua posição hierárquica na sentença e pelas relações gramaticais que mantém com os demais constituintes. No exemplo \enquote{José falou com seu irmão}, a ambiguidade semântica decorre da indefinição sobre a correferência do pronome possessivo \enquote{seu}, permitindo que ele se refira tanto a José quanto a um terceiro participante no discurso. Dessa forma, a teoria de Chomsky contribui para a análise da ambiguidade pronominal ao demonstrar como a estrutura sintática subjacente pode resultar em múltiplas possibilidades interpretativas.

A perspectiva assumida por \cite{canccado2005manual} e \cite{chomsky1981lectures} sobre ambiguidade semântica, ou anafórica, também é confirmada por outros estudos da linguística computacional, uma vez que esses também apresentam padrões regulares na retomada de antecedentes a partir de pronomes \cite{bruscato2021resoluccao,nogueira2014resoluccao,godoy2020efeitos,de2023interpretaccao}.


A \textbf{ambiguidade sintática} é um fenômeno de imprecisão de sentidos que não é ocasionado pela interpretação de uma palavra individual, mas se atribui às distintas estruturas sintáticas que originam diferentes interpretações: a frase concebe diferentes análises a partir dos seus possíveis sintagmas, que são divisões existentes dentro da frase em grupos de palavras. A sentença ``O magistrado julga as pessoas culpadas'' é um exemplo em que a organização da sentença pode ser: (a) O magistrado [julga] [as pessoas culpadas] ou (b) O magistrado [julga] [culpadas] [as pessoas]. A ambiguidade sintática envolve as diversas possibilidades de interpretação da sentença apenas reorganizando a posição das expressões envolvidas na oração, o que não ocorre nos tipos de ambiguidade tratados anteriormente. A ambiguidade sintática é um fenômeno amplamente estudado, com padrões obedecendo a princípios como o da aposição mínima e da aposição local \cite{maiadimensoes}, que atuam em diferentes línguas, incluindo o português brasileiro \cite{maia2003processamento,maia2004compreensao,brito2013processamento, machado1996sintaxe}. 

Os enunciados processados com esse tipo de ambiguidade corroboram com o que é descrito por \cite{jurafsky2000speech}, ao se basearem na decomposição dos constituintes por meio do \textit{parsing sintático}. A estrutura dos sintagmas permite identificar diferentes possibilidades de interpretação para o adjunto. Por exemplo, na frase \enquote{Ele saiu da loja de carro}, há a possibilidade de o adjunto modificar o verbo (1) ou de modificar o objeto indireto (2):
\begin{enumerate}
    \item {[Ele] [saiu [da loja] [de carro.]]}
    \item {[Ele] [saiu [da loja [de carro.]]]}
\end{enumerate}

Essa variação estrutural é o que dá origem à ambiguidade na sentença, dependendo da relação entre o adjunto e os outros elementos da frase.



\subsection{Modelos de linguagem}\label{sec-modelos-linguagem}

Os modelos de linguagem grandes (LLMs), ChatGPT 3.5 \cite{openai2023gpt4} e o Gemini \cite{ahmed2023chatgpt}, contam com aproximadamente 175 bilhões e 1,5 trilhão de parâmetros, respectivamente. Eles funcionam a partir de redes atencionais do tipo Transformer \cite{vaswani2017attention} que são pré-treinadas de forma auto-superivisionada em grandes conjuntos de dados. Posteriormente, eles são refinados através do aprendizado instrucional baseado em contexto e através do aprendizado por reforço baseado em feedback humano (RLHF) \cite{ouyang2022training}. Os mecanismos atencionais do tipo \textit{self-attention} \cite{vaswani2017attention} presentes nas arquiteturas permitem a captura de dependências de longas distâncias de forma computacionalmente eficaz, minimizando o esquecimento dos modelos em sequências longas. Por fim, a estratégia de indução de pensamento em cadeia (CoT) \cite{wei2023chainofthought} aplicada após os treinamentos permite que os modelos usem a sua última saída como entrada para gerar uma saída ainda mais refinada, melhorando a qualidade das respostas dadas.


No contexto da ambiguidade linguística, sabe-se que os mecanismos de \textit{self-attention} aprimoram a capacidade dos modelos em lidar com a ambiguidade semântica por meio do aprendizado da correferenciação, naturalmente presente na estrutura do mecanismo atencional \cite{ortega2023linguistic}. Entretanto, ainda não há evidências de que o RLHF, CoT e treinamentos instrucionais exclusivamente  presentes na ChatGPT e Gemini impactam no processamento da ambiguidade.

Assumimos a hipótese de que o aprendizado instrucional permite, de forma implícita, que esses modelos compreendam instruções pelo contexto e sigam direções específicas sobre a intenção do usuário, mitigando ou resolvendo certos casos de ambiguidade. Além disso, o RLHF pode ser particularmente valioso para lidar com situações ambíguas, já que ele provê o alinhamento correto do modelo com as intenções do usuário, por meio de recompensas dadas durante um treinamento de ajuste fino. Por fim, é possível que o CoT também tenha um papel crucial na resolução de ambiguidade, pois pode auxiliar os modelos a decompor o problema em etapas intermediárias mais gerenciáveis, permitindo abordar a resolução de ambiguidade de forma gradual, em vez de tentar resolvê-la de uma só vez. Diante dessas considerações, surge a questão: como esses elementos se adaptam durante o treinamento dos modelos para lidar com as sutilezas e complexidades do processamento de ambiguidades no contexto do português brasileiro? Explorar essa questão pode fornecer informações valiosas sobre como otimizar esses modelos para atender às nossas necessidades linguísticas e culturais.