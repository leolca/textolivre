\subsection{Qual dos modelos percebe melhor os fenômenos de homonímia e polissemia?}\label{resultados-q-4}


Neste estudo, foram examinadas as explicações fornecidas pelos modelos na tarefa 2, analisando apenas as respostas do conjunto de frases com ambiguidade lexical. Durante a análise, foi considerado que os modelos perceberam a homonímia e a polissemia através da explicação dada pelos modelos para justificar o tipo de ambiguidade identificada. Se a explicação dada pelos modelos foi referente a ambiguidade gerada devido aos diferentes significados que o item lexical pode assumir na frase, e se realmente os juízes-humanos enxergariam os diferentes significados do item lexical da mesma maneira, foi considerado acerto, caso contrário foi tido como um erro por parte dos modelos. 

Para demonstrar temos as frases \textbf{Isso não é legal!} e \textbf{A carteira foi danificada.} que foram classificadas como ambiguidade lexical de homonímia pelos juízes-humanos. Tais frases foram testadas no ChatGPT de modo que a primeira frase foi considerada correta, pois recebeu, em um dos testes, a seguinte resposta \textbf{A frase \enquote{Isso não é legal!} pode apresentar ambiguidade de sentido, pois a palavra \enquote{legal} possui múltiplas interpretações, dependendo do contexto em que é usada.} e a segunda frase foi classificada como incorreta, em um dos testes, por apresentar uma explicação incoerente \textbf{A frase \enquote{a carteira foi danificada} pode ser considerada ambígua devido à ambiguidade estrutural. Isso ocorre porque não está claro se a carteira sofreu dano físico ou se está se referindo a uma carteira de identidade ou pertencente a alguém. Portanto, a ambiguidade está relacionada à interpretação da frase em termos de sua estrutura sintática.}


Ambos os modelos demonstraram uma boa compreensão dos fenômenos de homonímia e polissemia. O ChatGPT obteve sucesso em 75\% das vinte frases testadas, enquanto o Gemini alcançou uma taxa de acerto de 80\%. Vale ressaltar a notável similaridade na forma como os modelos interpretaram esses fenômenos, uma vez que ambos obtiveram sucesso quase no mesmo subconjunto de frases.





