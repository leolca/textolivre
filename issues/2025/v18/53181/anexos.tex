\appendix
\section{Apêndice}
\label{sec-apendice}

\setlength\LTleft{-1in}
\setlength\LTright{-1in}
\begin{small}
\renewcommand{\arraystretch}{1.5}
\begin{longtable}{
    >{\raggedright\arraybackslash}p{0.5\textwidth}
    p{0.51\textwidth}
    p{0.08\textwidth}
    }
\caption{Conjunto de sentenças com ambiguidade por nós desenvolvido. As categorias foram atribuídas de acordo com o referencial teórico da seção \ref{sec-referencial-teorico}.}
\label{dataset_frases_ambiguas}
\\
\toprule
sentenças &  & categoria \\
\midrule
João foi à mangueira, assim como Maria. & João quer mangas, assim como Maria. & L \\
João fez uma rezinha. & Isso não é legal! & L \\
Essa dama é linda! & Que gato! & L \\
Ana vai ao banco. & Acende logo! & L \\
O papel foi bem feito. & Ele está com aquela matraca. & L \\
A carteira foi danificada. & Ela estava perto da mangueira. & L \\
O homem esperava no banco. & Gostamos de rosa. & L \\
Eu gosto de damas. & Era o ponto certo. & L \\
Pedi um prato ao garçom. & Ficamos sem rede. & L \\
Pegue a pilha, por favor. & A rede caiu. & L \\
Minha mãe e minha irmã ficaram chateadas depois que ela gritou com ela. & Ela não gosta da amiga dela. & SE \\
João falou comigo e com sua mãe. & Maria e sua mãe falaram comigo. & SE \\
Ele perguntou para ela se estava bem. & Ana me contou um segredo sobre ela. & SE \\
Ele não sabia que era o dia do seu aniversário. & O professor dele escreveu várias coisas em seu caderno. & SE \\ 
O Estado deve ajudar o povo para que ele prospere. & O padre bateu o carro dele. & SE \\
Paulo não entrou na sua casa. & Ela viu o ônibus passar na rua dela. & SE \\ 
O policial prendeu o bandido em sua casa. & Para ele ficar satisfeito, o chefe preparou um prato para o convidado. & SE \\
O homem matou seu tigre. & A professora proibiu que o aluno utilizasse seu dicionário. & SE \\
Ele a viu com a sua amiga. & A moça colocou sua mão na tinta. & SE \\
José comprou pão para Maria perto de sua casa. & Os empregados se revoltaram contra os superiores por causa dos seus salários. & SE \\
Ele saiu da loja de carro. &
Joana falou com a Maria brava. & SI \\
Carlos observou Tiago malhando. & Ele viu a moça com um binóculo. & SI \\
Abandonei meu irmão contrariado. & Soube do emprego novo de Janaína no restaurante. & SI \\
O menino viu o incêndio do prédio. & Pedro me mandou um cartão postal de Veneza. & SI \\
Idosa é presa por matar homem em crise na avenida. & A filha ligou para a mãe que tinha batido o carro. & SI \\
Policial prende criminoso com arma de brincadeira. & Eu li sobre a greve dos estudantes na universidade. & SI \\
Eu avisei à Júllia que estava atrasada. & Ricardo alimentou o pássaro cansado. & SI \\
O jogador o viu de tênis. & A mãe pegou o bebê chorando. & SI \\
A moça viu o rapaz andando com a amiga na rua. & Homens e mulheres inteligentes alcançam sucesso. & SI \\
Vendo carros e caminhões usados. & Ele foi atrás do táxi apressado. & SI \\

\bottomrule
\source{Fonte: Própria.}
\end{longtable}
\end{small}




\setlength\LTleft{-1in}
\setlength\LTright{-1in}
\begin{small}
\renewcommand{\arraystretch}{1.5}
\begin{longtable}{
    >{\raggedright\arraybackslash}p{0.55\textwidth}
    p{0.55\textwidth}
    }
\caption{Conjunto de sentenças sem ambiguidade por nós desenvolvido.}
\label{dataset_frases_nao_ambiguas}
\\
\toprule
sentenças & \\
\midrule
Michel Teló tem uma sorveteria. & Eu cavo com a pá. \\
O lápis ficou desesperado quando viu que o papel acabou. & O padre bateu o carro dando ré. \\
João foi à árvore da mangueira, assim como Maria. & João e Maria gostam da fruta que se chama manga. \\
João engatou a ré. & Viajem à praia vocês! \\
Viajem vocês! & Números não é uma boa temática para filmes. \\
O homem, que ama sertanejo, estava ouvindo arrocha. & A conduta dele é inadequada, porque ele vende drogas. \\
João ganhou a partida de Canastra com uma dama na mão. & O gato fez cocô fora da caixinha de areia. \\
Eu gosto de damas, são tão educadas. & Ana vai ao banco sacar dinheiro. \\
Acende logo! Eu quero cozinhar! & Ascenda e tenha prestígio. \\
Pedi o prato principal ao garçom, era filé! & Ontem ficamos sem rede, não dava mais pra navegar na internet. \\
A rede caiu, ninguém mais balançou! & A minha mãe, que estava brava, perguntou que horas eu cheguei. \\
Carlos, que estava malhando, observou Thiago. & Ele, por meio de um binóculo, viu a moça. \\
Proibido ultrapassar. & Janaína começará a trabalhar num restaurante. \\
O bandido foi preso em flagrante pelo policial. & Pegaram emprestado o livro muito culto dele. \\
Enquanto o menino estava no prédio, viu um incêndio. & Em Veneza, Pedro comprou um cartão postal pra mim. \\
A idosa estava em uma crise e matou, na avenida, o homem. & A filha, após sofrer um acidente de carro, ligou para a mãe. \\
Um policial, portando uma arma de brinquedo, prende criminoso. & Na universidade, os estudantes faziam uma greve, que eu vi. \\
Eu avisei atrasadamente à Júlia. & Como o pássaro estava cansado, Ricardo o alimentou. \\
O Estado deve ajudar o povo, que prosperará. & O tigre matou o homem. \\
Falei que estava passando mal à chefe. & O chefe preparou um prato para que o convidado ficasse satisfeito. \\
A filha gritou com a mãe e ambas ficaram chateadas. & Ela não gosta daquela mulher. \\
João falou comigo depois de falar com a mãe pelo telefone. & Conversei com José e a mãe dele, a qual o acompanhava. \\
Ele foi embora da loja dirigindo. & Ela estava doente e avisou o chefe. \\
Ele perguntou se ela estava bem. & Ana me contou um segredo. \\
Ele não sabia que já era o dia do próprio aniversário. & O professor fez várias correções no caderno do aluno. \\
Ele está conversando com aquela mulher. & Paulo não foi até a universidade. \\
Ela não viu o ônibus que passou. & A porta foi danificada. \\
Ela estava perto da árvore. & Pegue as pilhas de roupas, por favor. \\
Gostamos da cor rosa. & O homem esperava sentado. \\
Ela é profissional em tênis de mesa. & Era o ponto de ônibus certo. \\

\bottomrule
\source{Fonte: Própria.}
\end{longtable}
\end{small}




\setlength\LTleft{-1in}
\setlength\LTright{-1in}
\begin{small}
\renewcommand{\arraystretch}{1.5}
\begin{longtable}{
    >{\raggedright\arraybackslash}p{0.1\textwidth}
    p{0.1\textwidth}
    p{0.3\textwidth}
    p{0.1\textwidth}
    p{0.3\textwidth}
    p{0.1\textwidth}
    }
\caption{Conjunto de sentenças com ambiguidade lexical geradas pelos modelos de linguagem no experimento 4. A coluna de categoria real indica a classificação da sentença conforme a percepção humana da ambiguidade. Sentenças categorizadas como sem ambiguidade são aquelas em que seis seres humanos não detectaram qualquer ambiguidade. As demais categorias foram atribuídas com base no referencial teórico adotado neste trabalho.}
\label{longtbl-02}
\\
\toprule
modelo & categoria solicitada & sentença & categoria real & sentença & categoria real\\
\midrule

ChatGPT & lexical & O pássaro voou em direção à árvore mais alta. & sem ambiguidade & Ele viu o filme no cinema com os óculos escuros. & sem ambiguidade\\

ChatGPT & lexical & A fuga do preso foi rápida e espetacular. &  sem ambiguidade & Ela encontrou uma bela maçã na floresta. & sem ambiguidade\\

ChatGPT & lexical & O livro estava na mesa do professor. & sem ambiguidade & O jogador ferido foi substituído por um novato. & sem ambiguidade\\

ChatGPT & lexical & Ele comprou uma camisa azul e uma vermelha. & sem ambiguidade & O médico recomendou repouso e exercício. & sem ambiguidade\\

ChatGPT & lexical & A aula de biologia foi sobre células e tecidos. & sem ambiguidade & O rapaz correu para a casa da avó. &  sem ambiguidade\\

ChatGPT & lexical & Ela pintou a sala com as mãos. & sem ambiguidade & O barco ancorou na ilha deserta. & sem ambiguidade\\

ChatGPT & lexical & O cachorro late quando o carteiro chega. & sem ambiguidade & O guarda fechou a porta do parque. &  sem ambiguidade\\

ChatGPT & lexical & Ele comeu o bolo com gosto. & sintática & O homem viu o acidente com os próprios olhos. & sem ambiguidade\\

ChatGPT & lexical & O carro saiu da garagem com dificuldade. & sem ambiguidade & O escritor escreveu um romance emocionante. & sem ambiguidade\\

ChatGPT & lexical & O cantor entrou no palco com uma guitarra. & sintática & A mãe preparou a comida com amor. & sem ambiguidade\\

Gemini & lexical & O cachorro do vizinho mordeu a criança. & sintática e lexical & O homem comprou uma casa no campo. & sintática\\

Gemini & lexical & O advogado defendeu o criminoso. & sem ambiguidade & O soldado matou o inimigo. & sem ambiguidade\\

Gemini & lexical & O médico operou o paciente. & sem ambiguidade & O professor ensinou a classe. & lexical\\

Gemini & lexical & O juiz sentenciou o réu. & sem ambiguidade & O político falou para a multidão. & sem ambiguidade\\

Gemini & lexical & O artista pintou um quadro. & sem ambiguidade & O escritor escreveu um livro. & sem ambiguidade\\

Gemini & lexical & O carro preto passou por nós. & sem ambiguidade & A menina gorda está sentada ali. & sem ambiguidade\\

Gemini & lexical & A casa está vazia. & situacional\footnote{A ambiguidade situacional ocorre quando o contexto é crucial para determinar a interpretação de uma sentença, já que o enunciado não fornece pistas suficientes para uma única interpretação, e diferentes implicaturas podem ser geradas a partir das circunstâncias. No caso da frase "A casa está vazia", que pode significar que a casa está sem moradores ou está sem móveis e outros objetos dentro dela, a ambiguidade surge a partir da falta de informações contextuais claras. Segundo \cite{ilari1987}, essa é uma forma de ambiguidade que depende fortemente do contexto situacional e das inferências do ouvinte, já que ele precisará de outros indícios, como a situação em que a frase foi dita ou o tom de voz do falante, para escolher entre as possíveis implicaturas da sentença. } & A rua está molhada. & sem ambiguidade\\

Gemini & lexical & O livro está fechado. & sem ambiguidade & A porta está aberta. & situacional\footnote{É o mesmo caso de "A casa está vazia", pois a porta pode estar destrancada ou estar entreaberta, ou seja, visivelmente aberta.}\\

Gemini & lexical & O copo está cheio. & sem ambiguidade & A mesa está limpa. & situacional\footnote{É o mesmo caso de "A casa está vazia", pois a mesa pode estar sem nenhum objeto em cima ou não estar com sujeira.}\\

Gemini & lexical & O homem velho está dormindo na rua. & sem ambiguidade & A cama está arrumada. & sem ambiguidade\\

\bottomrule
\source{Fonte: Própria.}
\end{longtable}
\end{small}


\setlength\LTleft{-1in}
\setlength\LTright{-1in}
\begin{small}
\renewcommand{\arraystretch}{1.5}
\begin{longtable}{
    >{\raggedright\arraybackslash}p{0.1\textwidth}
    p{0.1\textwidth}
    p{0.3\textwidth}
    p{0.1\textwidth}
    p{0.3\textwidth}
    p{0.1\textwidth}
    }
\caption{Conjunto de sentenças com ambiguidade sintática que foram geradas pelos modelos de linguagem no experimento 4. A coluna de categoria real indica a classificação da sentença conforme a percepção humana da ambiguidade. Sentenças categorizadas como sem ambiguidade são aquelas em que seis seres humanos não detectaram qualquer ambiguidade. As demais categorias foram atribuídas com base no referencial teórico adotado neste trabalho.}
\label{longtbl-02}
\\
\toprule
modelo & categoria solicitada & sentença & categoria real & sentença & categoria real\\
\midrule

ChatGPT & sintática & Ela viu o homem com o telescópio. & sintática & Ele trouxe o livro para o amigo do professor. & sem ambiguidade\\

ChatGPT & sintática & O rapaz encontrou o carro dirigindo na estrada. & sintática & A professora elogiou o aluno com boas notas. & sintática\\

ChatGPT & sintática & A mãe deu um presente para a filha com uma fita bonita. & sintática & O cachorro mordeu o homem com a camisa rasgada. & sem ambiguidade\\

ChatGPT & sintática & O pintor viu o quadro com os olhos cansados. & sintática & O jogador marcou um gol com a camisa amarela. & sem ambiguidade\\

ChatGPT & sintática & O médico disse ao paciente que o tratamento seria difícil devido à sua saúde frágil. & sintática e semântica & O rapaz encontrou a carteira perdida no chão do parque. & lexical\\

ChatGPT & sintática & Vi o homem com o telescópio. & sintática & Ele viu o pássaro do vizinho com binóculos. & sem ambiguidade\\

ChatGPT & sintática & Eu gosto de estudar mais do que meu irmão. & sintática & Ela viu o homem com a luneta sentado na varanda. & sintática\\

ChatGPT & sintática & O cachorro do meu amigo late durante a noite. & sintática e lexical & Maria ama sua irmã mais do que eu. & sintática\\

ChatGPT & sintática & Comi a pizza com queijo. & sintática & Vi a fotografia da Maria com o cachorro. & sintática\\

ChatGPT & sintática & Ele quebrou a janela com a bola de beisebol jogando no quintal. & sem ambiguidade & O professor disse ao aluno que a matéria era fácil. & sem ambiguidade\\

Gemini & sintática & A menina viu o cachorro do vizinho correndo pela rua. & sintática e lexical & O homem levou a esposa ao médico. & sem ambiguidade\\

Gemini & sintática & O menino caiu da árvore e quebrou o braço. & semântica e lexical & A mulher comprou o vestido verde na loja. & sem ambiguidade\\

Gemini & sintática & O homem viu a mulher na janela. & sintática e lexical & O gato comeu o rato que estava no armário. & sem ambiguidade\\

Gemini & sintática & O menino jogou a bola para o cachorro. & sem ambiguidade & A menina estava brincando com a boneca quando a mãe chegou. & sem ambiguidade\\

Gemini & sintática & O homem comprou o livro que estava na prateleira. & sem ambiguidade & O menino tirou a roupa e foi tomar banho. & sem ambiguidade\\

Gemini & sintática & O cachorro do meu tio só dá trabalho. & sintática e lexical & Carla falou pro marido que tinha chegado. & sintática\\

Gemini & sintática & Eu fiquei te esperando no banco. & lexical & Ela puxou a cadeira e acabou quebrando o braço. &  semântica e lexical\\

Gemini & sintática & A galinha brincava com uma xícara até que quebrou a asa. & semântica e lexical & O guarda deteve o suspeito em sua casa. & semântica\\

Gemini & sintática & O cachorro comeu o osso e o menino. & sem ambiguidade & O homem comprou um carro novo e uma casa. & sem ambiguidade\\

Gemini & sintática & A mulher estava com medo do cachorro. & sem ambiguidade & O menino estava andando de bicicleta na rua. & sem ambiguidade\\

\bottomrule
\source{Fonte: Própria.}
\end{longtable}
\end{small}


\setlength\LTleft{-1in}
\setlength\LTright{-1in}
\begin{small}
\renewcommand{\arraystretch}{1.5}
\begin{longtable}{
    >{\raggedright\arraybackslash}p{0.1\textwidth}
    p{0.1\textwidth}
    p{0.3\textwidth}
    p{0.1\textwidth}
    p{0.3\textwidth}
    p{0.1\textwidth}
    }
\caption{Conjunto de sentenças com ambiguidade semântica que foram geradas pelos modelos de linguagem no experimento 4. A coluna de categoria real indica a classificação da sentença conforme a percepção humana da ambiguidade. Sentenças categorizadas como sem ambiguidade são aquelas em que seis seres humanos não detectaram qualquer ambiguidade. As demais categorias foram atribuídas com base no referencial teórico adotado neste trabalho.}
\label{longtbl-02}
\\
\toprule
modelo & categoria solicitada & sentença & categoria real & sentença & categoria real\\
\midrule

ChatGPT & semântica & Ela viu o homem com o telescópio. & sintática & O banco estava vazio, exceto por um único ocupante. & lexical\\

ChatGPT & semântica & Ele pegou a maçã verde do chão. & lexical & Vi o pássaro da janela enquanto dirigia. & sintática e lexical\\

ChatGPT & semântica & O jogador atirou a bola para o goleiro. & sem ambiguidade & Ele encontrou a carta de Maria na gaveta. & lexical\\

ChatGPT & semântica & A aula foi interessante para os alunos cansados. & sem ambiguidade & A luz do sol iluminava a sala escura. & sem ambiguidade\\

ChatGPT & semântica & A pele do caçador era macia. & lexical & O cachorro viu o gato no telhado com binóculos. & sintática\\

ChatGPT & semântica & Ela viu o homem com o telescópio. & sintática & Ele encontrou a mulher de chapéu na festa. & sintática\\

ChatGPT & semântica & O pato é bom para comer. & sintática & As crianças viram o filme no parque. & sem ambiguidade\\

ChatGPT & semântica & O ladrão roubou o banco com uma máscara. & sintática e lexical & Ele leu o livro com os óculos escuros. & sem ambiguidade\\

ChatGPT & semântica & O cachorro viu o gato na varanda. & sintática & Ela pintou o quadro com a mão. & sintática\\

ChatGPT & semântica & O jogador ferido saiu do campo mancando. & sem ambiguidade & A situação melhorou após a conversa com o terapeuta. & sem ambiguidade\\

Gemini & semântica &  A menina viu o homem com o binóculo. & sintática & O homem viu a mulher com o cachorro & sintática.\\

Gemini & semântica & O menino quebrou o braço da cadeira. & sem ambiguidade & A casa do João é azul. & sem ambiguidade\\

Gemini & semântica & O livro do menino está na mesa. & sem ambiguidade & O homem está morto. & lexical\\

Gemini & semântica & O menino está com fome. & sem ambiguidade & A mulher está cansada. & sem ambiguidade\\

Gemini & semântica &  O cachorro está feliz. & sem ambiguidade & O gato está dormindo. & sem ambiguidade\\

Gemini & semântica & O homem está procurando um cachorro. & sem ambiguidade & O menino foi ao banco. & lexical\\

Gemini & semântica & O guarda prendeu o suspeito em sua casa. & semântica e sintática & O professor deu uma bronca na aluna. & sem ambiguidade\\

Gemini & semântica & O carro está parado no sinal vermelho. & lexical & O homem está com fome. & sem ambiguidade\\

Gemini & semântica & A mulher está com frio. & sem ambiguidade & O menino está com medo. & sem ambiguidade\\

Gemini & semântica & A menina está com alegria. & sem ambiguidade & O homem está com raiva. & sem ambiguidade\\

\bottomrule
\source{Fonte: Própria.}
\end{longtable}
\end{small}


