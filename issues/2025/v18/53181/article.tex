\documentclass[portuguese]{textolivre}

% metadata
\journalname{Texto Livre}
\thevolume{18}
%\thenumber{1} % old template
\theyear{2025}
\receiveddate{\DTMdisplaydate{2024}{6}{28}{-1}}
\accepteddate{\DTMdisplaydate{2024}{9}{17}{-1}}
\publisheddate{\today}
\corrauthor{Lavínia Moraes}
\articledoi{10.1590/1983-3652.2025.53181}
%\articleid{NNNN} % if the article ID is not the last 5 numbers of its DOI, provide it using \articleid{} commmand 
% list of available sesscions in the journal: articles, dossier, reports, essays, reviews, interviews, editorial
\articlesessionname{articles}
\runningauthor{Moraes et al.}
%\editorname{Leonardo Araújo} % old template
\sectioneditorname{Daniervelin Pereira}
\layouteditorname{Leonardo Araújo}

\title{Análise de ambiguidade linguística em modelos de linguagem de grande escala (LLMs)}
\othertitle{Linguistic ambiguity analysis in large language models (LLMs)}

\author[1]{Lavínia de Carvalho Moraes~\orcid{0000-0002-4934-8505}\thanks{Email: \href{mailto:l237294@dac.unicamp.br}{l237294@dac.unicamp.br}}}
\author[1]{Irene Cristina Silvério~\orcid{0000-0002-7737-7469}\thanks{Email: \href{mailto:i169329@dac.unicamp.br}{i169329@dac.unicamp.br}}}
\author[1]{Rafael Alexandre Sousa Marques~\orcid{0000-0003-2807-037X}\thanks{Email: \href{mailto:rafaelalexandre001@gmail.com}{rafaelalexandre001@gmail.com}}}
\author[1]{Bianca de Castro Anaia~\orcid{0000-0001-9330-0288}\thanks{Email: \href{mailto:biancaanaia@gmail.com}{biancaanaia@gmail.com}}}
\author[1]{Dandara Freitas de Paula~\orcid{0000-0002-4995-1582}\thanks{Email: \href{mailto:d206236@dac.unicamp.br}{d206236@dac.unicamp.br}}}
\author[1]{Maria Carolina Schincariol de Faria~\orcid{0000-0003-0849-7237}\thanks{Email: \href{mailto:mariaschincariol@outlook.com}{mariaschincariol@outlook.com}}}
\author[1]{Iury Cleveston~\orcid{0000-0002-6010-4624}\thanks{Email: \href{mailto:iury@attenty.com.br}{iury@attenty.com.br}}}
\author[1]{Alana de Santana Correia~\orcid{0000-0002-7417-3727}\thanks{Email: \href{mailto:alana@attenty.com.br}{alana@attenty.com.br}}}
\author[2]{Raquel Meister Ko Freitag~\orcid{0000-0002-4972-4320}\thanks{Email: \href{mailto:rkofreitag@uol.com.br}{rkofreitag@uol.com.br}}}

\affil[1]{Attenty Sistemas de Software, Análise de dados, Campinas, SP, Brasil.}
\affil[2]{Universidade Federal de Sergipe, Departamento de Letras Vernáculas, Sergipe, SE, Brasil.}

\addbibresource{article.bib}

\usepackage{csquotes}
\usepackage{subcaption}
\usepackage{longtable}
\usepackage{array}

\begin{document}
\maketitle

\begin{polyabstract}
\begin{abstract}

A ambiguidade linguística ainda é um grande desafio para sistemas de processamento de linguagem natural (NLP) apesar dos avanços em arquiteturas como Transformers e BERT. Inspirado pelo êxito recente dos modelos instrucionais ChatGPT (versão 3.5) e Gemini (denominado Bard até 2023), este trabalho visa analisar e discutir a ambiguidade linguística nesses modelos a partir de três tipos de ambiguidade no Português Brasileiro: semântica, sintática e lexical. Para isso, foi desenvolvido um corpus com 120 frases ambíguas e não ambíguas, submetidas aos modelos para tipificação, explicação e desambiguação. Também foi explorada a capacidade de geração de frases ambíguas, solicitando a geração de conjuntos de frases para cada tipo de ambiguidade. Os resultados foram analisados qualitativamente, com base em referenciais linguísticos reconhecidos, e quantitativamente pela acurácia das respostas obtidas.  Evidenciamos que equívocos e deficiências nas respostas permanecem mesmo em modelos mais sofisticados, como ChatGPT e Gemini, com explicações frequentemente inconsistentes. A acurácia foi de no máximo 49,58\%, apontando a necessidade de estudos descritivos para o aprendizado supervisionado.



\keywords{Ambiguidade \sep Modelos de linguagem \sep ChatGPT \sep Gemini}
\end{abstract}

\begin{english}
\begin{abstract}
Linguistic ambiguity continues to represent a significant challenge for natural language processing (NLP) systems, notwithstanding the advancements in architectures such as Transformers and BERT. Inspired by the recent success of instructional models like ChatGPT and Gemini (Named Bard in 2023.), this study aims to analyze and discuss linguistic ambiguity within these models, focusing on three types prevalent in Brazilian Portuguese: semantic, syntactic, and lexical ambiguity. We create a corpus comprising 120 sentences, both ambiguous and unambiguous, for classification, explanation, and disambiguation. The models' capability to generate ambiguous sentences was also explored by soliciting sets of sentences for each type of ambiguity. The results underwent qualitative analysis, drawing on recognized linguistic references, and quantitative assessment based on the accuracy of the responses obtained. It was evidenced that even the most sophisticated models, such as ChatGPT and Gemini, exhibit errors and deficiencies in their responses, with explanations often providing inconsistent. Furthermore, the accuracy peaked at 49.58\%,indicating the need for descriptive studies for supervised learning. 

\keywords{Ambiguity \sep Language models \sep ChatGPT \sep Gemini}

\end{abstract}
\end{english}
% if there is another abstract, insert it here using the same scheme
\end{polyabstract}

\section{Introdução}\label{sec-intro}

A ambiguidade linguística, caracterizada pela possibilidade de uma palavra ou frase ter dois ou mais significados distintos em uma sentença, é um fenômeno complexo para os modelos de linguagem natural \cite{ortega2023linguistic}. Essa complexidade deriva da riqueza e sutilezas inerentes à estrutura e ao uso das línguas humanas, a partir da multiplicidade de significados que palavras e estruturas linguísticas podem assumir, dependendo do contexto em que são utilizadas. Mesmo os modelos de linguagem mais avançados, tais como Transformer \cite{vaswani2017attention} e BERT \cite{devlin2018bert}, enfrentam desafios ao lidar com diversos tipos de ambiguidade, devido à necessidade de considerar uma ampla variedade de contextos, conhecimentos prévios e nuances culturais que influenciam a interpretação das palavras e frases. Discernir o significado correto em um contexto específico demanda não apenas uma compreensão profunda da língua, mas também uma capacidade de inferência e abstração historicamente desafiadoras de replicar em sistemas computacionais \cite{ortega2023linguistic}. %Além disso, os modelos frequentemente apresentam dificuldades ao identificar e desambiguar um texto em situações mais complexas, nas quais múltiplas camadas de interpretação estão presentes.


A partir de 2010, a área de NLP testemunhou um grande avanço tecnológico com a evolução de técnicas de deep learning, principalmente em funções como sumarização de texto, classificação de tópicos, análise de sentimentos e sintetização de voz. Assim, foi possível experienciar tarefas com uma qualidade que ainda não era esperada até o momento. Desde o lançamento do ChatGPT e do Gemini\footnote[1]{O termo ChatPGT e Gemini serão tratados ao longo deste trabalho com pronomes masculinos por ser uma convenção adotada pelas demais literaturas brasileiras.}, modelos de linguagem instrucionais que têm revolucionado o mercado desde 2022, em consonância com o avanço tecnológico e o aumento da produtividade em diversas áreas do cotidiano, a inteligência artificial generativa tem impactado significativamente diversos setores, desde a educação até o mercado de trabalho. Essa influência se manifesta na distribuição de informações, na comunicação de ideias e na compreensão de dados discursivos, redefinindo a forma como interagimos e lidamos com o vasto espectro de informações disponíveis. 


Embora modelos avançados como o ChatGPT e o Gemini tenham apresentado progressos significativos na geração de linguagem natural, ainda subsistem incertezas quanto à sua capacidade de compreender fenômenos linguísticos complexos. Esses modelos, baseados em padrões estatísticos e processamento de grandes volumes de dados, não configuram uma verdadeira interlocução com o usuário, uma vez que anulam componentes fundamentais da comunicação humana, como gestos, entonação e expressões faciais. Além disso, eles não possuem uma compreensão do senso comum, o que frequentemente resulta na geração de dados que não refletem adequadamente a realidade compartilhada pelos seres humanos \cite{itankar2020ambiguity,singh2015role,chaplot2015unsupervised,agirre1997proposal}.

Por outro lado, a capacidade de processamento de um grande volume de dados empíricos provida pelos LLMs também tem oferecido oportunidades de reflexão sobre a relação entre teoria e dados, que é diferente a depender da abordagem assumida, seja indutivo, seja dedutivo, seja de orientação funcionalista, seja de orientação formalista. Mas há também abordagens que descrevem padrões fenômenos sem assumir uma teoria para língua, como é o caso de LLMs. \textcite{piantadosi2023modern} afirma que LLMs têm conseguido resultados de desvelamento da gramática sem usar nenhum dos métodos assumidos como necessários para a descrição linguística por humanos. Por outro lado, resultados de estudos que observam como LLMs lidam com tarefas linguísticas cotidianas, como a leitura de palavras que tiveram letras sistematicamente trocadas por números, como "l3tr4 tr0c4d4" (letra trocada) \cite{marcus2023sentence}, sugerem que, por não terem cognição fundamentada, LLMs não podem tirar proveito desses recursos e, em vez disso, dependem apenas de associações fixas entre palavras representadas e vetores de palavras. Os autores evidenciam um paradoxo: se os LLMs supostamente emulam a linguagem humana, e seu desempenho supostamente mostra habilidades que superaram as dos seres humanos, ao mesmo tempo que estes mesmos LLMs não conseguem fornecer respostas precisas em tarefas muito básicas de compreensão da linguagem, como é o caso da tarefa letra trocada, sendo necessários mais estudos em mais tarefas, como a de resolução de ambiguidades, que podem contribuir para a compreensão da base cognitiva da faculdade humana da linguagem. 

Do ponto de vista linguístico, é improvável que esses modelos de linguagem consigam reproduzir, em sua totalidade, as capacidades linguísticas humanas. A significação, por exemplo, é intrinsecamente multimodal, envolvendo não apenas a linguagem verbal, mas também elementos contextuais como expressões faciais, postura corporal e até detalhes sutis como o vestuário do interlocutor, no caso de interações orais \cite{freitag2022linguistic}. No contexto da comunicação escrita, aspectos como a cor da fonte, o tipo de papel e o suporte do texto podem influenciar a interpretação do significado. Além disso, fatores como variação linguística, tom de voz e o uso de linguagem figurativa são determinantes para a construção de sentido, o que torna a completa emulação da experiência linguística humana por modelos artificiais uma tarefa extremamente desafiadora. 

Essas dificuldades afetam a capacidade dos modelos de linguagem de processar adequadamente ambiguidades linguísticas e superar desafios já observados em modelos tradicionais, que também enfrentam limitações. Além disso, a maior parte dos estudos desenvolvidos até o momento têm como foco a língua inglesa \cite{ortega2023linguistic}. Análises mais aprofundadas no contexto do português brasileiro, uma língua de baixos recursos linguísticos \cite{finger2021inteligencia, freitag2022sociolinguistic}, ainda não foram conduzidas, revelando a importância de investigações neste campo.

Nesse contexto, este trabalho tem como objetivo responder as seguintes perguntas:

\begin{enumerate}
    \item Qual é a precisão dos modelos na detecção de ambiguidade linguística em frases do Português Brasileiro?    
    \item Os modelos conseguem desambiguar adequadamente as sentenças?    
    \item Qual dos modelos percebe melhor os fenômenos de homonímia e polissemia?    
    \item Quais padrões de ambiguidade os modelos ChatGPT e Gemini demonstram conhecer na
    geração de frases ambíguas?
\end{enumerate}


O estudo de ambiguidade é particularmente complexo, por envolver uma gama de variáveis que interferem, desde a natureza do item lexical, passando pela sintaxe da sentença, e envolvendo o conhecimento de mundo e experiência pessoal de cada falante. Além disso,  há um paradoxo a ser superado: um humano pode perceber ambiguidade e não saber explicá-la, enquanto a IA pode saber explicar o que é uma ambiguidade, mas não saber reconhecê-la.  
Assumindo que as IAs são capazes de imitar em grande medida o processamento da linguagem humana e que têm o potencial de fornecer informações sobre a forma como as pessoas aprendem e utilizam a linguagem \cite{cai2023does} conduzimos um estudo com a realização de quatro tarefas. 
Utilizando um conjunto de sentenças com e sem ambiguidade criado por nós, verificamos a consistência das respostas dos modelos ao fazer as mesmas perguntas duas vezes, em momentos distintos, e contrastamos algumas inconsistências nas respostas obtidas. 
\section{Processamento da ambiguidade linguística e o processamento da linguagem natural}\label{sec-referencial-teorico}

O processamento semântico é um domínio de investigação fundamental para os modelos linguísticos pré-treinados e dos grandes modelos de linguagem. No campo das tarefas de processamento semântico, a desambiguação do sentido das palavras demanda parâmetros definidos. No entanto, o estudo da ambiguidade é multidimensional na linguística, como veremos a seguir.

Esta pesquisa fundamenta-se em disciplinas que estão interligadas, demonstrando a natureza interdisciplinar das áreas envolvidas. Portanto, o estudo aproveita-se dos princípios teóricos relacionados à ambiguidade linguística (Seção \ref{sec-ambiguidade-linguistica}) e da aplicação dos modelos computacionais de grande escala, ChatGPT e Gemini (Seção \ref{sec-modelos-linguagem}).


É importante destacar que o arcabouço teórico sobre ambiguidade linguística, empregado nesta seção, servirá como alicerce para a construção do nosso conjunto de sentenças. Entre várias abordagens teóricas disponíveis, decidimos adotar a perspectiva de Cançado \cite{canccado2005manual} dada a sua compatibilidade com a metodologia da pesquisa e o objeto de estudo, os modelos de linguagem, que não se valem dos aspectos de interlocução para compreender ambiguidade, mas focam nos componentes linguísticos do enunciado. Dessa forma, a fundamentação dos princípios de ambiguidade dessa taxonomia permitiu que os dados fossem analisados de forma mais objetiva, evitando depender de fatores contextuais externos. Além disso, há trabalhos bem recentes de outros países que fazem estudos do ponto de vista computacional de análise de ambiguidade mantendo uma coesão similar com aspectos da teoria da Cançado em termos de ambiguidade lexical, semântica e sintática \cite{haber2021patterns,ortega2023linguistic,liu2023we,goel2023beyond}.



\subsection{Ambiguidade linguística}\label{sec-ambiguidade-linguistica}


A ambiguidade é um fenômeno semântico no qual uma palavra, expressão ou sentença pode ter mais de uma interpretação válida. A resolução desse tipo de fenômeno depende fortemente do contexto, que orienta a escolha do sentido adequado entre os que possíveis. No entanto, os modelos de linguagem, por apresentarem restrições na compreensão dos componentes multimodais presentes no contexto, não captam sinais extralinguísticos, fundamentais para processar ambiguidade ou desambiguar enunciados.

Historicamente, a ambiguidade é uma questão antiga e amplamente estudada em tarefas de PLN através de análises simbólicas de sentenças, como o parsing sintático e semântico \cite{church1982,koller2008} ou resolução de correferência \cite{poesio2005}. Entretanto, com o avanço recente no uso de redes neurais profundas e modelos de linguagem pré-treinados, o foco do campo tem se deslocado para problemas de compreensão em níveis mais altos, incluindo o raciocínio e a geração de texto. Nesse novo contexto, a ambiguidade continua sendo um desafio já que os modelos aprendem a lidar com ambiguidade de forma implícita, por meio de grandes quantidade de dados de treinamento.

Seguimos a proposta de \textcite{canccado2005manual} para o português brasileiro, que considera o nível de uso da língua e inclui a ambiguidade lexical, com casos de homonímia, polissemia, ambiguidade semântica e ambiguidade sintática.

A \textbf{ambiguidade lexical} compreende uma sentença com  dupla interpretação incidente em um item lexical, podendo ser gerada por homonímia ou polissemia. A homonímia se dá quando os sentidos do item lexical não são relacionados, como na oração ``Eu estou indo para o banco'' em que a palavra ``banco'' possui significados diferentes, pode corresponder à instituição financeira e ao assento. Já a polissemia ocorre quando os sentidos do termo identificado como ambíguo contém relação entre si, por exemplo na sentença ``O Frederico esqueceu a sua concha'', neste caso ``concha'' pode significar uma concha do mar ou uma concha de cozinha, em que ambos os objetos possuem o mesmo formato, e por isso, pode ocorrer uma associação polissêmica, o mesmo fenômeno ocorre com palavras como: rede (de internet, de deitar, de pescar) e pilha (de comida, de bateria).

Além disso, na taxonomia de \textcite{canccado2005manual}, a ambiguidade lexical também pode ser causada por meio de preposições, classificando-se assim como ambiguidade preposicional. Por exemplo, na sentença \enquote{O burro do Paulo anda doente} permite duas interpretações: \enquote{burro} pode se referir ao animal que Paulo possui, caracterizando uma ambiguidade literal, ou pode ser uma expressão figurativa, referindo-se a Paulo como sendo uma pessoa \enquote{burra}. Essa dupla interpretação ocorre por causa da homonímia, onde a palavra \enquote{burro} tem dois significados não relacionados que são possíveis de serem inferidos por conta da presença da preposição. Esse tipo de ambiguidade preposicional se soma às outras duas subclassificações de ambiguidade lexical identificadas por Cançado que são tratadas na pesquisa, homonímia e polissemia.

Embora a ambiguidade preposicional seja uma categoria relevante, o conjunto de dados criado para testar os modelos neste trabalho se restringiu a casos de polissemia e homonímia, excluindo assim a ambiguidade gerada por preposições, mas esses casos apareceram na Tarefa 4 de geração de frases por parte dos modelos.

Nessa mesma esteira teórica, Lyons \cite{lyons1977semantics} considera a homonímia e a polissemia como casos de ambiguidade lexical, assim como \cite{canccado2005manual}. A homonímia ocorre quando dois ou mais significados não possuem relação semântica entre si, sendo historicamente distintas no desenvolvimento da língua, o que reforça a independência dos sentidos em termos lexicais. Já a polissemia envolve uma relação intrínseca entre os diferentes sentidos de uma palavra, onde esses sentidos compartilham um núcleo comum de significado. Assim, no exemplo \enquote{banco},  os diferentes sentidos (assento e instituição financeira) são homônimos, uma vez que suas origens etimológicas e semânticas são divergentes. Por outro lado, no caso de \enquote{concha}, os diferentes usos podem ser entendidos como polissemia, já que ambos os sentidos remetem a uma semelhança de forma e função, refletindo a natureza interconectada das diversas acepções do termo.

É importante destacar que essa distinção de subcategorias é fundamental, visto que, para o português, a ambiguidade lexical é a que apresenta mais recursos descritivos para suporte computacional \cite{laporte2001resoluccao}, necessário aos LLMs.

 
Na taxonomia de Cançado, a \textbf{ambiguidade semântica} é abordada como uma questão de correferencialidade, em que os pronomes podem ter vários antecedentes. Consideremos, por exemplo, a seguinte frase: ``José falou com seu irmão?'' Esta sentença ilustra claramente esse tipo de ambiguidade, na qual não é possível determinar se o irmão mencionado é o irmão de José ou o irmão da pessoa para quem a pergunta é dirigida, ou seja, um terceiro elemento. Nesse contexto, as interpretações possíveis são atribuídas à natureza da ligação entre os pronomes presentes na sentença.

A gramática gerativa compartilha da mesma percepção sobre pronomes de referência como fator de ambiguidade. \cite{chomsky1981lectures} trata da correferencialidade por meio do conceito de \textit{binding}, que estabelece princípios para a associação entre pronomes e seus possíveis antecedentes dentro de uma estrutura sintática. A interpretação de um pronome é, portanto, condicionada por sua posição hierárquica na sentença e pelas relações gramaticais que mantém com os demais constituintes. No exemplo \enquote{José falou com seu irmão}, a ambiguidade semântica decorre da indefinição sobre a correferência do pronome possessivo \enquote{seu}, permitindo que ele se refira tanto a José quanto a um terceiro participante no discurso. Dessa forma, a teoria de Chomsky contribui para a análise da ambiguidade pronominal ao demonstrar como a estrutura sintática subjacente pode resultar em múltiplas possibilidades interpretativas.

A perspectiva assumida por \cite{canccado2005manual} e \cite{chomsky1981lectures} sobre ambiguidade semântica, ou anafórica, também é confirmada por outros estudos da linguística computacional, uma vez que esses também apresentam padrões regulares na retomada de antecedentes a partir de pronomes \cite{bruscato2021resoluccao,nogueira2014resoluccao,godoy2020efeitos,de2023interpretaccao}.


A \textbf{ambiguidade sintática} é um fenômeno de imprecisão de sentidos que não é ocasionado pela interpretação de uma palavra individual, mas se atribui às distintas estruturas sintáticas que originam diferentes interpretações: a frase concebe diferentes análises a partir dos seus possíveis sintagmas, que são divisões existentes dentro da frase em grupos de palavras. A sentença ``O magistrado julga as pessoas culpadas'' é um exemplo em que a organização da sentença pode ser: (a) O magistrado [julga] [as pessoas culpadas] ou (b) O magistrado [julga] [culpadas] [as pessoas]. A ambiguidade sintática envolve as diversas possibilidades de interpretação da sentença apenas reorganizando a posição das expressões envolvidas na oração, o que não ocorre nos tipos de ambiguidade tratados anteriormente. A ambiguidade sintática é um fenômeno amplamente estudado, com padrões obedecendo a princípios como o da aposição mínima e da aposição local \cite{maiadimensoes}, que atuam em diferentes línguas, incluindo o português brasileiro \cite{maia2003processamento,maia2004compreensao,brito2013processamento, machado1996sintaxe}. 

Os enunciados processados com esse tipo de ambiguidade corroboram com o que é descrito por \cite{jurafsky2000speech}, ao se basearem na decomposição dos constituintes por meio do \textit{parsing sintático}. A estrutura dos sintagmas permite identificar diferentes possibilidades de interpretação para o adjunto. Por exemplo, na frase \enquote{Ele saiu da loja de carro}, há a possibilidade de o adjunto modificar o verbo (1) ou de modificar o objeto indireto (2):
\begin{enumerate}
    \item {[Ele] [saiu [da loja] [de carro.]]}
    \item {[Ele] [saiu [da loja [de carro.]]]}
\end{enumerate}

Essa variação estrutural é o que dá origem à ambiguidade na sentença, dependendo da relação entre o adjunto e os outros elementos da frase.



\subsection{Modelos de linguagem}\label{sec-modelos-linguagem}

Os modelos de linguagem grandes (LLMs), ChatGPT 3.5 \cite{openai2023gpt4} e o Gemini \cite{ahmed2023chatgpt}, contam com aproximadamente 175 bilhões e 1,5 trilhão de parâmetros, respectivamente. Eles funcionam a partir de redes atencionais do tipo Transformer \cite{vaswani2017attention} que são pré-treinadas de forma auto-superivisionada em grandes conjuntos de dados. Posteriormente, eles são refinados através do aprendizado instrucional baseado em contexto e através do aprendizado por reforço baseado em feedback humano (RLHF) \cite{ouyang2022training}. Os mecanismos atencionais do tipo \textit{self-attention} \cite{vaswani2017attention} presentes nas arquiteturas permitem a captura de dependências de longas distâncias de forma computacionalmente eficaz, minimizando o esquecimento dos modelos em sequências longas. Por fim, a estratégia de indução de pensamento em cadeia (CoT) \cite{wei2023chainofthought} aplicada após os treinamentos permite que os modelos usem a sua última saída como entrada para gerar uma saída ainda mais refinada, melhorando a qualidade das respostas dadas.


No contexto da ambiguidade linguística, sabe-se que os mecanismos de \textit{self-attention} aprimoram a capacidade dos modelos em lidar com a ambiguidade semântica por meio do aprendizado da correferenciação, naturalmente presente na estrutura do mecanismo atencional \cite{ortega2023linguistic}. Entretanto, ainda não há evidências de que o RLHF, CoT e treinamentos instrucionais exclusivamente  presentes na ChatGPT e Gemini impactam no processamento da ambiguidade.

Assumimos a hipótese de que o aprendizado instrucional permite, de forma implícita, que esses modelos compreendam instruções pelo contexto e sigam direções específicas sobre a intenção do usuário, mitigando ou resolvendo certos casos de ambiguidade. Além disso, o RLHF pode ser particularmente valioso para lidar com situações ambíguas, já que ele provê o alinhamento correto do modelo com as intenções do usuário, por meio de recompensas dadas durante um treinamento de ajuste fino. Por fim, é possível que o CoT também tenha um papel crucial na resolução de ambiguidade, pois pode auxiliar os modelos a decompor o problema em etapas intermediárias mais gerenciáveis, permitindo abordar a resolução de ambiguidade de forma gradual, em vez de tentar resolvê-la de uma só vez. Diante dessas considerações, surge a questão: como esses elementos se adaptam durante o treinamento dos modelos para lidar com as sutilezas e complexidades do processamento de ambiguidades no contexto do português brasileiro? Explorar essa questão pode fornecer informações valiosas sobre como otimizar esses modelos para atender às nossas necessidades linguísticas e culturais.
\section{Metodologia}\label{sec-metodologia}

Para avaliar o desempenho dos modelos no processamento de ambiguidades quanto aos parâmetros da ambiguidade lexical, sintática e semântica, conduzimos tarefas utilizando um conjunto de dados que foi elaborado e analisado pelos autores do presente artigo, um grupo de seis estudantes de Letras e Linguística, cada um com um conhecimento maior em áreas distintas da linguística, como fonética, tradução, sintaxe e análise do discurso. Esses estudantes serão referidos no trabalho como juízes-humanos por se demonstrarem aptos a produzir e julgar de forma adequada os dados, além disso, vale salientar que os dados e resultados das tarefas foram avaliados, posteriormente, por uma coautora que é especialista na área de variação linguística.

Para garantir a consistência e a confiabilidade das frases geradas pelo grupo, foi adotado o procedimento de \textit{inter-annotator agreement}. As sentenças que não receberam consenso absoluto entre os juízes-humanos foram revisadas ou excluídas do corpus. Como resultado, não foi necessário calcular o coeficiente de concordância {\footnote{Ao criar o corpus, as frases foram revisadas por todos os autores do artigo, e apenas aquelas com 100\% de aprovação no procedimento de inter-annotator agreement foram disponibilizadas para os testes, com o objetivo de minimizar a presença de frases problemáticas. As frases geradas pelos modelos também passaram por uma revisão conjunta e foram discutidas entre os autores para garantir consenso absoluto a partir do mesmo procedimento. Por exemplo, na Tabela 4, que corresponde à geração de frases por parte do modelo, a sentença \enquote{O homem viu o acidente com os próprios olhos} contém um componente favorável à criação de ambiguidade sintática, o adjunto \enquote{com os próprios olhos}, no entanto, com base na teoria de Cançado, essa ambiguidade é válida quando acompanhada por uma ambiguidade semântica, que permite o duplo sentido na compreensão pragmática do contexto dentro do enunciado. Já na Tabela 3, a frase \enquote{Pedi o prato principal ao garçom, era filé!} é parte do corpus das sentenças distratoras, pois a estrutura do enunciado fornece informações suficientes para evitar outras interpretações, ou seja, entendemos que \enquote{prato} não se refere ao utensílio doméstico e \enquote{filé} não é um elogio metafórico. Sendo assim, essa frase foi considerada não ambígua pelos seis juízes-humanos e pela especialista em linguística que revisou o trabalho.}}, uma vez que todas as sentenças, tanto as ambíguas quanto as não ambíguas, só foram incluídas após alcançarem 100\% de aprovação. Esse critério rigoroso assegurou que o conjunto de dados utilizado nas tarefas fosse  confiável ao máximo para minimizar divergências.

O experimento foi composto por um grupo de 120 sentenças, distribuídas de forma balanceada entre os três tipos de ambiguidade. Dessas, 60 apresentam algum tipo de ambiguidade, seja semântica, lexical ou sintática. As frases ambíguas foram elaboradas com o objetivo de provocar especificamente um dos três tipos de ambiguidade (semântica, lexical ou sintática). No entanto, é possível que algumas sentenças apresentem mais de um tipo de ambiguidade, que não foi avaliado durante a criação das mesmas. No decorrer de nossa análise, buscamos isolar ao máximo cada frase, de modo que apenas um elemento causador de ambiguidade estivesse presente. Essa abordagem nos permite um controle mais rigoroso das variáveis observadas nos experimentos, uma vez que a ambiguidade é um fenômeno complexo e multifacetado.

É importante ressaltar que, dentro de uma única sentença, podem existir múltiplos fatores que contribuem para a ambiguidade, o que torna a sua identificação e análise ainda mais desafiadoras, então, para um estudo inicial, de caráter pioneiro com LLM's formamos dados linguísticos mais artificiais, mas planejamos trabalhar futuramente com dados provenientes de corpora ou textos reais, incorporando materiais autênticos que nos permitam uma compreensão mais abrangente das ambiguidades linguísticas em contextos variados.

Foram criadas 20 sentenças com ambiguidade lexical — que abrangem casos de homonímia e polissemia, sem distinção de categoria na análise das frases —, 20 sentenças com ambiguidade semântica, nas quais o referente dos pronomes não está claro, e, por fim, 20 sentenças com ambiguidade sintática, envolvendo adjuntos adnominais ou adverbiais ambíguos que provocam duplo sentido devido às diferentes organizações estruturais que a frase pode ter (Tabela \ref{tab:amostras_sentencas}).

As outras 60 sentenças distratoras tiveram sua ambiguidade barrada ao máximo, não sendo verificadas pelos juízes-humanos durante a elaboração dos dados linguísticos. As sentenças distratoras foram criadas com a intenção de evitar qualquer forma de ambiguidade. Da mesma forma que foi feito o julgamento das frases ambíguas, as frases não ambíguas foram avaliadas em um nível de significado mais isolado, sem considerar diversos contextos enunciativos figurativos, na maioria das vezes, nas quais ela poderia fazer sentido, logo assumimos um posicionamento de interpretação mais pragmático linguístico, que corrobora com Cançado. Essas mesmas considerações são válidas para as avaliações feitas em relação às frases que foram geradas pelos modelos de linguagem na Tarefa 4.

Um exemplo para ilustrar essa forma de análise é a frase \enquote{Pedi o prato principal ao garçom, era filé!} (Tabela 3) em que em um contexto muito específico, poderia significar que o cliente pediu o utensílio principal de servir comida e este utensílio era \enquote{filé}, um termo popular para se referir a algo bom, porém, a princípio, a frase foi escrita tendo em vista o significado mais óbvio, que se determinou a partir das pistas interpretativas deixadas dentro da frase na qual um sentido lexical confirmava o outro sem a necessidade de buscar condições exofóricas à sentença que justificassem uma polissemia.





\begin{table}[htpb]
\centering
\begin{threeparttable}
\caption{}
\label{tab:amostras_sentencas}
\begin{tabular}{llp{6cm}}
\toprule
Id & Sentença & Classe \\
\midrule
1 & A rede caiu. & Ambiguidade lexical \\
2 & Ela não gosta da amiga dela. & Ambiguidade semântica \\
3 & O menino viu o incêndio do prédio. & Ambiguidade sintática \\
\bottomrule
\end{tabular}
\source{\url{https:XXXXXXXXX}.}
%\notes{Se necessário, poderá ser adicionada uma nota ao final da tabela.}
\end{threeparttable}
\end{table}

%\source{\url{https://osf.io/u7wre/?view_only=572c74eb4c634d47a02ad25485ea8caa}.}

Para responder as nossas perguntas de pesquisa, foram conduzidas quatro tarefas {\footnote{As respostas dadas pelos modelos de linguagem nas tarefas foram coletadas entre julho de 2023 até janeiro de 2024. Coletas posteriores podem levar a resultados diferentes devido às atualizações dos modelos de linguagem. Para evitar ao máximo o enviesamento dos modelos, as perguntas foram formuladas da forma mais objetiva possível, evitando dar pistas sobre a resposta correta ou sinalizando qual era o resultado esperado. A preocupação em minimizar testes com viés tinha o objetivo de garantir uma avaliação mais precisa do desempenho dos modelos de linguagem, afastar casos de generalização nas respostas, favorecer a imparcialidade dos resultados e alcançar um conjunto de respostas o mais transparente possível.}} distintas com as sentenças criadas. Em todas, foram realizadas coletas duplicadas das interações para cada frase, reiniciando o \textit{console} entre cada coleta para evitar qualquer influência do contexto que pudesse gerar respostas tendenciosas. Essa abordagem permitiu avaliar a consistência dos modelos nas respostas fornecidas.


A tarefa 1 visava identificar se os modelos conseguem detectar a presença de ambiguidade em cada sentença por meio da seguinte instrução: \textbf{A sentença ``[sentença]'' é ambígua? Responda, sim, não ou não sei}. Foram apresentadas individualmente todas as sentenças e registradas as respostas dos modelos, comparando-as com a nossa classificação prévia. As respostas foram cuidadosamente avaliadas quanto à correção e abrangência das explicações fornecidas por seis juízes-humanos que as julgaram independentemente. A partir dos resultados, foi gerada uma matriz de confusão para computar a quantidade de verdadeiros positivos (sentenças que são ambíguas e que os modelos classificaram como ambíguas), falsos positivos (sentenças que não o são e que os modelos classificaram como ambíguas), verdadeiros negativos (sentenças que não são ambíguas e que os modelos assim classificaram como não ambíguas), e falsos negativos (sentenças que são ambíguas e que os modelos classificaram como não ambíguas).


Na tarefa 2, foi realizado um teste para avaliar a capacidade dos modelos em distinguir corretamente entre as três classes de ambiguidade estudadas neste trabalho, formulando a seguinte pergunta para cada modelo: \textbf{``Qual o tipo de ambiguidade?''}. A tarefa consistiu em perguntar qual o tipo de ambiguidade da sentença que foi classificada anteriormente como ambígua ou não ambígua. Na tarefa 3, foi verificada a capacidade dos modelos em desambiguar as sentenças que foram fornecidas a eles. Com esse propósito, foram apresentadas frases que incluem tanto sentenças ambíguas, quanto sentenças não ambíguas, e solicitado aos modelos a seguinte instrução: \textbf{Faça a desambiguação da frase: ``[sentença]''}. A tarefa busca testar a habilidade dos modelos em compreender e interpretar o contexto, escolhendo a interpretação mais apropriada quando a ambiguidade está presente. 

Na tarefa 4, foi avaliada a capacidade dos modelos em gerar frases ambíguas na categoria solicitada. Para isso, pedimos para cada modelo gerar frases da seguinte forma: \textbf{Gere 20 frases com ambiguidade ``[categoria]''}. Em seguida, as respostas obtidas foram avaliadas por juízes-humanos, buscando compreender quão preciso é o ChatGPT o Gemini ao criarem frases que apresentam múltiplas interpretações contextuais.

Para mensurar quantitativamente o desempenho dos modelos, foi utilizada a métrica de acurácia, a qual já é amplamente empregada na área de aprendizado de máquina  \cite{naser2021error, freitag2021funccao}. A acurácia, no contexto da classificação, representa a proporção de frases corretamente classificadas pelos modelos em relação ao total de frases apresentadas na tarefa, como apresentado na equação \ref{eq_1}.

\begin{equation}
    \text{acc} = \frac{\text{Número de previsões corretas}}{\text{Total de previsões}}
    \label{eq_1}
\end{equation}

Todas as sentenças criadas por nós e geradas pelos modelos durante as tarefas estão disponíveis no Apêndice \ref{sec-apendice}. As respostas dos modelos durante as tarefas estão disponíveis para download\footnote{\url{https://docs.google.com/spreadsheets/d/1AOff1GJmh3oWIuKdfBGGHWeox-7HtuHQW0LkaT5yFVI/edit?usp=sharing}} em nosso repositório.


%\footnote{\url{https://osf.io/u7wre/?view_only=572c74eb4c634d47a02ad25485ea8caa}}
\section{Resultados}\label{sec-resultados}

En la siguiente tabla se recogen los principales datos de los 10 vídeos
analizados para posteriormente proceder a un análisis multimodal
centrado principalmente en el vídeo, y en menor medida, en el uso del
texto. Se recogen solamente tres variables: \emph{Duración}, \emph{Me
	gusta} y \emph{Comentarios} para confirmar que son vídeos mucho más
breves y destacar que en nuestro análisis no se prioriza en la audiencia
y en la interacción.

\begin{table}[htbp]
\centering
\begin{threeparttable}
\caption{Análisis descriptivo del vídeo}
\label{tab-03}
\begin{tabular}{p{3cm} l l l l}
\toprule
Dinámica & Perfil & Duración & Me gusta & Comentarios \\
\midrule
1. What are you reading? & \seqsplit{@claudiacp\_books} & 05 seg. & 56 & 1 \\
2. Readers follow party & \seqsplit{@laslecturasdeloli} & 15 seg. & 3047 & 717 \\
3. Bibliotecas & \href{https://www.tiktok.com/@bibliotecaugena/video/6971466004585057542}{\seqsplit{@bibliotecaugena}} & 13 seg. & 2365 & 14 \\
4. Tipos de Booktoker & \href{https://www.tiktok.com/@pandi.book/video/7199333140748438790}{\seqsplit{@pandi.book}} & 1min. 54 seg. & 83.4K & 973 \\
5. Promoción lectora & \href{https://www.tiktok.com/@iriayselene/video/7211269180484750598}{\seqsplit{@iriayselene}} & 10 seg. & 21.4K & 194 \\
6. Editoras & \href{https://www.tiktok.com/@vreditorasya/video/7223140662022114565}{\seqsplit{@vreditorasya}} & 08 seg. & 243.8K & 487 \\
7. Bookinfluencer & \href{https://www.tiktok.com/@patriciafedz/video/7222221715819072774}{\seqsplit{@patriciafedz}} & 2 min, 15 seg. & 23K & 150 \\
8. Día del libro & \href{https://www.tiktok.com/@javierruescas/video/7224553867566910747}{\seqsplit{@javierruescas}} & 1 min. 35 seg. & 30.4K & 86 \\
9. Videopoemas & \seqsplit{@marinalcuadrado} & 1 min. 9 seg. & 12.5K & 719 \\
10. Así dijo\ldots & \href{https://www.tiktok.com/@solo.palabrqs/video/7190817595401030918}\seqsplit{@solo.palabrqs} & 20 seg. & 138.2K & 181 \\
\bottomrule
\end{tabular}
\source{Elaboración propia.}
\end{threeparttable}
\end{table}

Como se puede observar, muchos de estos vídeos tienen una enorme
audiencia, con miles de ``Me gusta''. Solamente la primera categoría
tiene muchos menos, pero se ha incluido este vídeo porque es el que más
claramente ejemplifica una nueva dinámica. El análisis de las diez
dinámicas analizadas en la plataforma TikTok permite una primera
diferenciación en la comunicación visual que se realiza entre aquellas
que adoptan un punto de vista mediado frente a las que no. Este punto de
vista mediado implica que las acciones que ven los espectadores son
vivenciadas como si fueran la persona que ha creado el vídeo (mediada:
inferida) o bien desde dentro de la imagen (mediada: inscrita) cuando se
observa alguna de las partes de su cuerpo (e inclusive desde sus
espaldas). Las dos dinámicas que emplean este tipo de estrategias de
comunicación visual son 1. \emph{¿Qué estás leyendo? (What are you
	reading?)} y 2. \emph{Fiesta de seguidores-lectores (Readers Follow
	Party)}, en la que se aprecian diferentes matices con algunos ejemplos
concretos. Estas dinámicas buscan la mayor intervención del espectador
al ubicarlo al mismo nivel experiencial que la persona que crea el
contenido. Y el protagonista de ambas son los libros o las estanterías
de los usuarios, siendo vídeos de corta duración acompañados de una
pieza de audio concreta que pone título a estas dinámicas en inglés.
Además, ambos son muy breves e indican nuevas dinámicas hiperbreves de
esta red social.

Siguiendo un ejemplo de la dinámica 1. \emph{¿Qué estás leyendo?} (de la
usuaria @claudiacp\_books) la perspectiva mediada e inscrita persigue
generar un efecto de suspense (inicialmente solo se muestra la parte
inversa al lomo del libro, donde solamente se ven unas páginas y no se
puede identificar la obra) y solo aparece la mano de la usuaria
sosteniéndolo unos segundos hasta revelarlo y generando una metonimia
visual \cite{moya-Guijarro}. Este tipo de dinámicas reducen la
distancia social entre el espectador y la autora del vídeo, para generar
una sensación de intimidad y adopta un plano cenital donde tanto la
persona creadora del contenido como el espectador se ubican en una
relación de superioridad frente al objeto-libro representado. En estos
vídeos, ambas personas, \emph{booktoker} y espectadores comparten la
misma perspectiva respecto al vídeo. En este caso, esta metonimia visual
que acompaña al giro para presentar la cubierta del libro va acompañada
de un audio específico, titulado: \emph{What this person reading right
	now?} Este audio dura solamente cinco segundos e inclusive se pueden
encontrar en la plataforma y usarlo en nuestras creaciones
\url{https://www.tiktok.com/music/Whats-this-person-reading-right-now-6904077601871170309}
y cuenta con más de cien mil vídeos que usan este misma dinámica y
sonido, que lo convierten en tendencia.

El ejemplo de la dinámica 2. \emph{Fiesta de seguidores-lectores} (de la
usuaria @laslecturasdelol) se basa en la creación de un solapamiento
entre la perspectiva del usuario y del espectador, dejando que el plano
tenga una mayor distancia para mostrar la panorámica de los espacios
lectores que recorre el usuario como sus estanterías. La implicación del
espectador también opta por una perspectiva mediada, pero en este caso
inferida, dado que ambos contemplan la escena en primera persona. Se
opta por una angulación donde prima la horizontalidad (y el aspecto
frontal) para que el espectador se sienta partícipe de la misma forma
que el usuario. La transcripción del vídeo de 15 segundos, que aparece
también en el vídeo es la siguiente:
\begin{quote}
		Fiesta de seguidores: Si sos \emph{booktoker} y tenés menos de cinco mil
		seguidores, seguime y así te puedo seguir. Deja un comentario con tu
		número de seguidores actual y a cada persona que deje un comentario
		seguila, No te olvides de compartir el link para que más booktokers lo
		vean. ¡Que lluevan seguidores!
		
		Fuente:
		\url{https://www.tiktok.com/@laslecturasdeloli/video/6972178596152511749}.
\end{quote}

Este vídeo es una adaptación al español de la dinámica \emph{Reader Follow Party}\footnote{\url{https://www.tiktok.com/tag/readerfollowparty}}
que se ha usado en más de diez mil ocasiones, aunque generalmente por
público anglófono. Por ejemplo, la usuaria @maddiesreadss ofrece un
vídeo con subtítulos automáticos de este audio, donde se observa que la
interacción entre cámara y elementos es idéntica al ejemplo seleccionado
en español.

\begin{quote}
	Reader Follow Party. If You're not already, follow me and I'll follow
	you back. Then, go to the comments and tell us what book you're
	currently reading. When someone likes your comment follow them and
	they're follow you back. Yay for more books friends!
	
	Fuente:
	\url{https://www.tiktok.com/@maddiesreadss/video/7236953611547184386}.
\end{quote}


El análisis se inicia con ambos vídeos porque son dos de las dinámicas
más novedosas de \#\emph{Booktok} y el protagonista central de ambas
prácticas son los libros. Aunque la dinámica con las que se identifican
estos vídeos son de usuarias hispanohablantes, las etiquetas y audios
originales son en inglés, con mucha mayor proyección, siendo claras
tendencias en el ámbito de TikTok.

Frente a estas dos dinámicas que adoptan una perspectiva subjetiva en su
grabación y montaje, se encuentran aquellas que optan por una
perspectiva no mediada en la que la implicación visual para el
espectador se compensa con otras estrategias como el contacto visual
directo o la creación de dinámicas con un componente humorístico. Entre
los casos analizados, se destaca un ejemplo por parte de una biblioteca
pública (@bibliotecaugena) 3. \emph{Bibliotecas} donde el efecto del
vídeo es ubicar en tercera persona al espectador y combinar las opciones
de focalización desde la alternancia de las miradas de oferta (aquella
que se dirigen por la persona protagonista, la bibliotecaria, hacia los
objetos y elementos que quiere que el espectador se fije) y las de
demanda donde prima el contacto visual (con las que interpela al
lector-espectador a seguir la dinámica). En este sentido, el plano se
aleja y se contempla a la persona con una mayor distancia respecto al
espectador (dado que prima la sucesión breve de acciones) y se ubica
desde el uso de ángulos frontales para que se facilite la sensación de
intervención del usuario. También tiene una duración breve, 13 segundos
para presentar varios libros. Este perfil tiene muchos vídeos similares,
con la bibliotecaria como protagonista y contenidos sobre la función e
importancia de las bibliotecas y es un claro ejemplo para la mediación
lectora en redes sociales.

El componente humorístico también forma parte de montajes como la
presentación de la dinámica 4. \emph{Tipos de Booktoker} (de
@pandi.book), que se basa de nuevo en la alternancia entre el contacto
visual de la joven protagonista del vídeo con el espectador (para
acrecentar la sensación de cercanía) y las miradas de demanda (ahora,
invitada) cuando se reproduce (simula, en este caso) un diálogo con otra
persona para darle réplica. En esos intercambios, cuando se dirige a
otra persona, una voz en \emph{off}, la posición corporal se angula
horizontalmente para crear en ese diálogo una mayor sensación de
desapego con el espectador (para compartir la reacción de la usuaria que
aparece en el vídeo). Cuando ese breve diálogo (en forma de reproche en
ocasiones) finaliza, se busca nuevamente la cercanía con la protagonista
al emplear preferentemente primeros planos (alternando con planos
americanos) para generar una situación de intimidad, como también ocurre
cuando se ubica en una relación neutral y frontal con el espectador.
Además, este vídeo muestra los tipos de nuevos lectores que se
identifican con estas dinámicas.

Otro ejemplo recurrente en la plataforma es la creación de un efecto
paródico a partir de la contradicción \cite{bateman2014} entre los modos
visuales y textuales. Por ejemplo, en esta presentación de su nuevo
libro por parte de sus autoras de LIJ en el vídeo 5. \emph{Promoción}
\emph{lectora} (de @iriayselene), se observa cómo la información textual
pasa a un segundo plano, sobreimpresionadas en la pantalla, mientras que
las protagonistas vocalizan la letra de la canción que suena de fondo
(\emph{Naughty} - Alisha Weir \& The Cast of Roald
Dahl\textquotesingle s Matilda The Musical). El texto dice y la
interacción con los libros es la siguiente:

\begin{quote}
	Cartela 1 Verde. Cuando nos criticaron un montón por hacer un rettelling
	de \emph{Ana de las Tejas Verdes} con representación LGTB+. (Muestran su
	libro \emph{Anne sin filtros})
	
	Cartela 2 Ocre. Así que hemos hecho lo mismo, pero con El Mago de Oz.
	(Presentan nuevo libro Seremos huracán)
\end{quote}


En este caso Iria y Selene ya eran famosas \emph{booktubers} y autoras
de literatura juvenil que han empezado una nueva aventura en
\emph{booktok}. Como se ha comentado anteriormente, en la plataforma se
opta con asiduidad al reemplazo de la voz de los usuarios por dejar que
sea la música la que complete el mensaje audiovisual, siendo este audio
del musical \emph{Matilda} muy habitual para presentar bromas o actos
traviesos.

Otro claro ejemplo de dinámicas paródicas es el vídeo de @vreditorasya
6. \emph{Editoras} (de @vreditorasya) donde se opta por no aparecer su
cabeza en la pantalla y se da respuesta a la pregunta sobreimpresionada
(``Cuando dice que me quiere regalar un libro, pero no sabe cuál'') con
la intención de generar un \emph{sketch} humorístico en la que poner de
manifiesto el interés desmedido por la lectura, porque luego presenta un
papel con una lista muy larga de libros deseados (\emph{Wish List}),
aunque en este caso, apenas aparecen libros y no se leen los títulos,
pero sí se destaca el gusto por la lectura. Además, este vídeo aprovecha
una tendencia humorística ``Ding dong'', donde después de oírse un ruido
sordo, una mujer imita el sonido de una campanilla para presentarse.
Muchas veces este sonido se usa de forma de incitación sexual, al usarse
para presentar a una persona semidesnuda se usa de forma sexista, pero
este \emph{booktok} lo aprovecha para presentar el interés por los
libros de la protagonista.
\url{https://www.tiktok.com/music/suono-originale-6830486689857784582}

Frente a las dinámicas anteriores (tanto de perspectiva mediada como no
mediada), en la plataforma también se reproducen dinámicas heredadas de
YouTube en el modo de presentar la información. En estos casos, las
personas creadoras de contenido dan relevancia a su figura como
``presentadoras de contenido''. En primer lugar, sobre contenidos como
``influencers'' \cite{establés2019}.
\emph{Bookinfluencer} (de @patriciafedz) en la que comparten las
anécdotas o vivencias de las personas usuarias, como es el caso de
@patriciafedz, donde se opta por primeros planos y el contacto visual
directo para compartir esa intimidad del relato con el espectador. Esta
tipología incluso explota la imagen de la protagonista, una mujer joven
y atractiva, como en otras redes sociales \cite{calvo2018,dezuanni2022}. Este vídeo tiene más de doscientas
mil visualizaciones y veinte mil me gusta, con varios comentarios que
incluso señalan algún error que realiza la protagonista al citar obras
literarias. Además, Patricia Fernández tiene más de trescientos mil
seguidores y una producción constante de vídeos, por lo que se puede
destacar como una ``\emph{bookinfluencer}''. En este tipo de vídeos, la
ubicación de la angulación vuelve a situarse levemente contrapicada,
dejando al espectador de nuevo en inferioridad frente a la persona que
narra dicha anécdota. Esta circunstancia también indica cómo el cambio
de grabación (principalmente el móvil) en una plataforma como TikTok
frente a Youtube donde la cámara suele ubicarse en una posición más
elevada en la pantalla del ordenador (siendo el espectador el que se
encuentra en superioridad, habitualmente, frente al creador del
contenido). Además, estos vídeos tienen mayor duración (más de dos
minutos) y edición, intercalando otros vídeos en el montaje. En este
caso, no se opta por efectos sonoros, y lo importante es la explicación
de la protagonista.

Además de vivencias relacionadas con experiencias alrededor de la
literatura, se han recogido vídeos sobre contenidos para compartir el
conocimiento sobre detalles relacionados con la lectura, como el de
@javirruescas 8. \emph{Día del libro} sobre el origen de esta efeméride.
Javier Ruescas es también un famoso bloguero, \emph{booktuber} y autor
de literatura juvenil \cite{ruescas2012} que ha sabido adaptarse al nuevo
medio, también con una gran audiencia. De nuevo, la importancia de estas
dinámicas es ubicarse frente al espectador ofreciéndole un plano cercano
que disminuya la distancia social, el contacto visual directo y una
angulación de neutralidad donde no ubicarse en superioridad con el
espectador. Otro efecto curioso es la sobreimpresión de los subtítulos
que permite simultáneamente leer la explicación. Como en el caso
anterior, no hay audio de acompañamiento, solamente la voz del
\emph{booktoker} y este vídeo como el precedente, también tiene mayor
duración, se ha grabado desde un ordenador sentado, y luego se ha subido
a la plataforma, pensada para las reproducciones en móviles.

En esta categoría también hay otros ejemplos que optan por crear
contenidos como recitados o lecturas en voz alta como la dinámica 9.
\emph{Videopoemas}. Pese a que pueden darse diferentes perspectivas, se
pone en relieve con el ejemplo de @marinalcuadrado cómo se diluye la
sensación de cercanía y la angulación vertical (contrapicada) juega para
ubicar a la persona que recita en superioridad respecto al espectador
que, además, encuentra la empatía de la mirada de la persona que recita
al emplear una focalización de contacto visual de demanda directo
(reclamando nuestra atención). Este formato con un protagonista
recitando también tiene una audiencia importante, con más ciento
cincuenta mil visualizaciones. Es un videopoema más cercano y con una
protagonista que hace atractiva la poesía. Las dinámicas tradicionales
de videopoemas en Youtube se basaban en montajes de diapositivas con el
recitado de fondo, como se verá a continuación.

Finalmente, con una estrategia más distante en la comunicación visual de
los protagonistas, también hay contenidos que optan por la desaparición
del creador para emplear una imagen que sirva de apoyo a un pasaje, cita
o fragmento de un autor, en este caso Mario Benedetti. La dinámica 10.
\emph{Así dijo\ldots{}} y se basa en reproducir citas y versos de
autores famosos. Se incluye esta categoría porque se considera que tiene
claras posibilidades didácticas para la enseñanza de la literatura y
confirma que grandes autores también tienen presencia en esta red.
Aunque también tienen mucha audiencia, estos vídeos muchas veces
atribuyen falsamente poemas o textos a autores, por lo que es necesario
usarlos con atención \cite{roviracollado-hernandez-ortega-2023}. Este tipo
de montajes intentan plasmar dinamismo mediante la sucesión de imágenes
acompañadas de un fragmento musical que amplifica de manera emotiva el
mensaje verbal escogido por el usuario @solo.palabrqs. Así, se trata de
un tipo de dinámica más impersonal y en la que se ofrece un videoclip o
sucesión de imágenes con música o recitado, más que la interacción con
los espectadores. En este caso se citan unas palabras concretas, pero
hay muchos videopoemas que aprovechan las voces originales de estos
autores.




\section{Conclusão}\label{sec-conclusao}

Nossos resultados indicaram melhorias significativas entre os diferentes modelos, assim como diversas vantagens e limitações. Nesse sentido, nosso trabalho apresenta as seguintes contribuições: 1) disponibiliza um conjunto de dados formatado para testar a ambiguidade linguística em modelos de linguagem natural no português brasileiro, que até o nosso conhecimento é o primeiro proposto na literatura; 2) é o primeiro trabalho a informar à comunidade científica sobre as limitações do ChatGPT e do Gemini em compreender fenômenos linguísticos complexos, como a ambiguidade na língua portuguesa; 3) apresenta uma metodologia para avaliar esses modelos quanto ao fenômeno da ambiguidade; e 4) demonstra, por meio de resultados qualitativos e quantitativos, qual dos dois modelos lida melhor com esses fenômenos linguísticos.

A análise do fenômeno de ambiguidade linguística nos modelos instrucionais ChatGPT e Gemini, cujas versões 3.5 e Bard, respectivamente, foram submetidos a quatro tarefas referentes à detecção, tipificação, desambiguação e geração de frases ambíguas. Os resultados obtidos mostraram que ambiguidade linguística ainda é um grande desafio para estas versões de modelos de processamento de linguagem natural, demandando ainda estudos e implementações para o aprimoramento.

Os modelos apresentaram baixa acurácia e baixo desempenho em praticamente todas as tarefas testadas. Na detecção de ambiguidade, o ChatGPT conseguiu uma acurácia de 28,75\% e o Gemini 49,58\%. Os modelos também apresentaram uma superinterpretação de sentenças não ambíguas, detectando e desambiguando frases que não tinham qualquer tipo de ambiguidade e que seres humanos facilmente conseguem interpretar apenas um sentido nas sentenças. Os melhores resultados obtidos ocorreram na tarefa de desambiguação e classificação onde a ambiguidade residia apenas no item lexical, demonstrando que é o tipo de ambiguidade em que os modelos tem mais facilidade para lidar. 

Merece destaque o fato de que, apesar da maior facilidade em lidar com ambiguidade lexical, os modelos tiveram o pior desempenho na geração de frases com ambiguidade desse tipo. Algo similar ocorreu com as frases de categoria semântica, em que os modelos confundiram com os padrões de ambiguidade estrutural ou geraram a maioria das frases sem ambiguidade. Por outro lado, ocorreu um melhor desempenho na geração de frases com ambiguidade sintática, porém ainda com uma interpretação errada sobre a origem da ambiguidade em várias frases, demonstrando que os modelos conseguiram gerar algumas frases corretamente mas ainda não conseguem explicar com clareza as causas da ambiguidade sintática. Outro ponto que chamou a atenção foi a tendência da ChatGPT em personificar alguns elementos inanimados para atribuir ambiguidade às frases, algo que um ser humano jamais faria. Os resultados relativos à identificação de ambiguidades se alinham ao que  \cite{marcus2023sentence} obtiveram com a tarefa de letra trocada: enquanto seres humanos são excelentes nessa tarefa, LLMs apresentam dificuldades. A resolução de ambiguidades é então uma das habilidades que ainda estão faltando no estado atual de desenvolvimento desses modelos.

Em suma, os resultados mostram que estas versões dos modelos instrucionais ainda estão distantes de emular plenamente a capacidade cognitiva dos seres humanos, não só envolvendo a relação entre linguagem e identidade social \cite{freitag2021preconceito}, mas também o uso na interação social, o que requer a compreensão de ambiguidades. No entanto, os resultados também sinalizam um progresso inicial na compreensão e aquisição do senso comum a respeito de como a linguagem humana funciona e reiteram a importância dos estudos descritivos em línguas ainda com poucos recursos, como é o caso do português \cite{finger2021inteligencia, freitag2022sociolinguistic}, para aprimoramento.

\printbibliography\label{sec-bib}
%conceptualization,datacuration,formalanalysis,funding,investigation,methodology,projadm,resources,software,supervision,validation,visualization,writing,review
\begin{contributors}[sec-contributors]
\authorcontribution{Lavínia de Carvalho Moraes}[conceptualization,datacuration,investigation,writing,review]
\authorcontribution{Irene Cristina Silvério}[datacuration,investigation,writing]
\authorcontribution{Rafael Alexandre Sousa Marques}[datacuration,investigation,writing]
\authorcontribution{Bianca de Castro Anaia}[datacuration,investigation,writing]
\authorcontribution{Dandara Freitas de Paula}[datacuration,investigation,writing]
\authorcontribution{Maria Carolina Schincariol de Faria}[datacuration,investigation,writing]
\authorcontribution{Iury Cleveston}[validation,resources,funding]
\authorcontribution{Alana de Santana Correia}[conceptualization,validation,resources,methodology,supervision,writing,review]
\authorcontribution{Raquel Meister Ko Freitag}[conceptualization,validation,methodology,review]
\end{contributors}

\appendix
\section{Apêndice}
\label{sec-apendice}

\setlength\LTleft{-1in}
\setlength\LTright{-1in}
\begin{small}
\renewcommand{\arraystretch}{1.5}
\begin{longtable}{
    >{\raggedright\arraybackslash}p{0.5\textwidth}
    p{0.51\textwidth}
    p{0.08\textwidth}
    }
\caption{Conjunto de sentenças com ambiguidade por nós desenvolvido. As categorias foram atribuídas de acordo com o referencial teórico da seção \ref{sec-referencial-teorico}.}
\label{dataset_frases_ambiguas}
\\
\toprule
sentenças &  & categoria \\
\midrule
João foi à mangueira, assim como Maria. & João quer mangas, assim como Maria. & L \\
João fez uma rezinha. & Isso não é legal! & L \\
Essa dama é linda! & Que gato! & L \\
Ana vai ao banco. & Acende logo! & L \\
O papel foi bem feito. & Ele está com aquela matraca. & L \\
A carteira foi danificada. & Ela estava perto da mangueira. & L \\
O homem esperava no banco. & Gostamos de rosa. & L \\
Eu gosto de damas. & Era o ponto certo. & L \\
Pedi um prato ao garçom. & Ficamos sem rede. & L \\
Pegue a pilha, por favor. & A rede caiu. & L \\
Minha mãe e minha irmã ficaram chateadas depois que ela gritou com ela. & Ela não gosta da amiga dela. & SE \\
João falou comigo e com sua mãe. & Maria e sua mãe falaram comigo. & SE \\
Ele perguntou para ela se estava bem. & Ana me contou um segredo sobre ela. & SE \\
Ele não sabia que era o dia do seu aniversário. & O professor dele escreveu várias coisas em seu caderno. & SE \\ 
O Estado deve ajudar o povo para que ele prospere. & O padre bateu o carro dele. & SE \\
Paulo não entrou na sua casa. & Ela viu o ônibus passar na rua dela. & SE \\ 
O policial prendeu o bandido em sua casa. & Para ele ficar satisfeito, o chefe preparou um prato para o convidado. & SE \\
O homem matou seu tigre. & A professora proibiu que o aluno utilizasse seu dicionário. & SE \\
Ele a viu com a sua amiga. & A moça colocou sua mão na tinta. & SE \\
José comprou pão para Maria perto de sua casa. & Os empregados se revoltaram contra os superiores por causa dos seus salários. & SE \\
Ele saiu da loja de carro. &
Joana falou com a Maria brava. & SI \\
Carlos observou Tiago malhando. & Ele viu a moça com um binóculo. & SI \\
Abandonei meu irmão contrariado. & Soube do emprego novo de Janaína no restaurante. & SI \\
O menino viu o incêndio do prédio. & Pedro me mandou um cartão postal de Veneza. & SI \\
Idosa é presa por matar homem em crise na avenida. & A filha ligou para a mãe que tinha batido o carro. & SI \\
Policial prende criminoso com arma de brincadeira. & Eu li sobre a greve dos estudantes na universidade. & SI \\
Eu avisei à Júllia que estava atrasada. & Ricardo alimentou o pássaro cansado. & SI \\
O jogador o viu de tênis. & A mãe pegou o bebê chorando. & SI \\
A moça viu o rapaz andando com a amiga na rua. & Homens e mulheres inteligentes alcançam sucesso. & SI \\
Vendo carros e caminhões usados. & Ele foi atrás do táxi apressado. & SI \\

\bottomrule
\source{Fonte: Própria.}
\end{longtable}
\end{small}




\setlength\LTleft{-1in}
\setlength\LTright{-1in}
\begin{small}
\renewcommand{\arraystretch}{1.5}
\begin{longtable}{
    >{\raggedright\arraybackslash}p{0.55\textwidth}
    p{0.55\textwidth}
    }
\caption{Conjunto de sentenças sem ambiguidade por nós desenvolvido.}
\label{dataset_frases_nao_ambiguas}
\\
\toprule
sentenças & \\
\midrule
Michel Teló tem uma sorveteria. & Eu cavo com a pá. \\
O lápis ficou desesperado quando viu que o papel acabou. & O padre bateu o carro dando ré. \\
João foi à árvore da mangueira, assim como Maria. & João e Maria gostam da fruta que se chama manga. \\
João engatou a ré. & Viajem à praia vocês! \\
Viajem vocês! & Números não é uma boa temática para filmes. \\
O homem, que ama sertanejo, estava ouvindo arrocha. & A conduta dele é inadequada, porque ele vende drogas. \\
João ganhou a partida de Canastra com uma dama na mão. & O gato fez cocô fora da caixinha de areia. \\
Eu gosto de damas, são tão educadas. & Ana vai ao banco sacar dinheiro. \\
Acende logo! Eu quero cozinhar! & Ascenda e tenha prestígio. \\
Pedi o prato principal ao garçom, era filé! & Ontem ficamos sem rede, não dava mais pra navegar na internet. \\
A rede caiu, ninguém mais balançou! & A minha mãe, que estava brava, perguntou que horas eu cheguei. \\
Carlos, que estava malhando, observou Thiago. & Ele, por meio de um binóculo, viu a moça. \\
Proibido ultrapassar. & Janaína começará a trabalhar num restaurante. \\
O bandido foi preso em flagrante pelo policial. & Pegaram emprestado o livro muito culto dele. \\
Enquanto o menino estava no prédio, viu um incêndio. & Em Veneza, Pedro comprou um cartão postal pra mim. \\
A idosa estava em uma crise e matou, na avenida, o homem. & A filha, após sofrer um acidente de carro, ligou para a mãe. \\
Um policial, portando uma arma de brinquedo, prende criminoso. & Na universidade, os estudantes faziam uma greve, que eu vi. \\
Eu avisei atrasadamente à Júlia. & Como o pássaro estava cansado, Ricardo o alimentou. \\
O Estado deve ajudar o povo, que prosperará. & O tigre matou o homem. \\
Falei que estava passando mal à chefe. & O chefe preparou um prato para que o convidado ficasse satisfeito. \\
A filha gritou com a mãe e ambas ficaram chateadas. & Ela não gosta daquela mulher. \\
João falou comigo depois de falar com a mãe pelo telefone. & Conversei com José e a mãe dele, a qual o acompanhava. \\
Ele foi embora da loja dirigindo. & Ela estava doente e avisou o chefe. \\
Ele perguntou se ela estava bem. & Ana me contou um segredo. \\
Ele não sabia que já era o dia do próprio aniversário. & O professor fez várias correções no caderno do aluno. \\
Ele está conversando com aquela mulher. & Paulo não foi até a universidade. \\
Ela não viu o ônibus que passou. & A porta foi danificada. \\
Ela estava perto da árvore. & Pegue as pilhas de roupas, por favor. \\
Gostamos da cor rosa. & O homem esperava sentado. \\
Ela é profissional em tênis de mesa. & Era o ponto de ônibus certo. \\

\bottomrule
\source{Fonte: Própria.}
\end{longtable}
\end{small}




\setlength\LTleft{-1in}
\setlength\LTright{-1in}
\begin{small}
\renewcommand{\arraystretch}{1.5}
\begin{longtable}{
    >{\raggedright\arraybackslash}p{0.1\textwidth}
    p{0.1\textwidth}
    p{0.3\textwidth}
    p{0.1\textwidth}
    p{0.3\textwidth}
    p{0.1\textwidth}
    }
\caption{Conjunto de sentenças com ambiguidade lexical geradas pelos modelos de linguagem no experimento 4. A coluna de categoria real indica a classificação da sentença conforme a percepção humana da ambiguidade. Sentenças categorizadas como sem ambiguidade são aquelas em que seis seres humanos não detectaram qualquer ambiguidade. As demais categorias foram atribuídas com base no referencial teórico adotado neste trabalho.}
\label{longtbl-02}
\\
\toprule
modelo & categoria solicitada & sentença & categoria real & sentença & categoria real\\
\midrule

ChatGPT & lexical & O pássaro voou em direção à árvore mais alta. & sem ambiguidade & Ele viu o filme no cinema com os óculos escuros. & sem ambiguidade\\

ChatGPT & lexical & A fuga do preso foi rápida e espetacular. &  sem ambiguidade & Ela encontrou uma bela maçã na floresta. & sem ambiguidade\\

ChatGPT & lexical & O livro estava na mesa do professor. & sem ambiguidade & O jogador ferido foi substituído por um novato. & sem ambiguidade\\

ChatGPT & lexical & Ele comprou uma camisa azul e uma vermelha. & sem ambiguidade & O médico recomendou repouso e exercício. & sem ambiguidade\\

ChatGPT & lexical & A aula de biologia foi sobre células e tecidos. & sem ambiguidade & O rapaz correu para a casa da avó. &  sem ambiguidade\\

ChatGPT & lexical & Ela pintou a sala com as mãos. & sem ambiguidade & O barco ancorou na ilha deserta. & sem ambiguidade\\

ChatGPT & lexical & O cachorro late quando o carteiro chega. & sem ambiguidade & O guarda fechou a porta do parque. &  sem ambiguidade\\

ChatGPT & lexical & Ele comeu o bolo com gosto. & sintática & O homem viu o acidente com os próprios olhos. & sem ambiguidade\\

ChatGPT & lexical & O carro saiu da garagem com dificuldade. & sem ambiguidade & O escritor escreveu um romance emocionante. & sem ambiguidade\\

ChatGPT & lexical & O cantor entrou no palco com uma guitarra. & sintática & A mãe preparou a comida com amor. & sem ambiguidade\\

Gemini & lexical & O cachorro do vizinho mordeu a criança. & sintática e lexical & O homem comprou uma casa no campo. & sintática\\

Gemini & lexical & O advogado defendeu o criminoso. & sem ambiguidade & O soldado matou o inimigo. & sem ambiguidade\\

Gemini & lexical & O médico operou o paciente. & sem ambiguidade & O professor ensinou a classe. & lexical\\

Gemini & lexical & O juiz sentenciou o réu. & sem ambiguidade & O político falou para a multidão. & sem ambiguidade\\

Gemini & lexical & O artista pintou um quadro. & sem ambiguidade & O escritor escreveu um livro. & sem ambiguidade\\

Gemini & lexical & O carro preto passou por nós. & sem ambiguidade & A menina gorda está sentada ali. & sem ambiguidade\\

Gemini & lexical & A casa está vazia. & situacional\footnote{A ambiguidade situacional ocorre quando o contexto é crucial para determinar a interpretação de uma sentença, já que o enunciado não fornece pistas suficientes para uma única interpretação, e diferentes implicaturas podem ser geradas a partir das circunstâncias. No caso da frase "A casa está vazia", que pode significar que a casa está sem moradores ou está sem móveis e outros objetos dentro dela, a ambiguidade surge a partir da falta de informações contextuais claras. Segundo \cite{ilari1987}, essa é uma forma de ambiguidade que depende fortemente do contexto situacional e das inferências do ouvinte, já que ele precisará de outros indícios, como a situação em que a frase foi dita ou o tom de voz do falante, para escolher entre as possíveis implicaturas da sentença. } & A rua está molhada. & sem ambiguidade\\

Gemini & lexical & O livro está fechado. & sem ambiguidade & A porta está aberta. & situacional\footnote{É o mesmo caso de "A casa está vazia", pois a porta pode estar destrancada ou estar entreaberta, ou seja, visivelmente aberta.}\\

Gemini & lexical & O copo está cheio. & sem ambiguidade & A mesa está limpa. & situacional\footnote{É o mesmo caso de "A casa está vazia", pois a mesa pode estar sem nenhum objeto em cima ou não estar com sujeira.}\\

Gemini & lexical & O homem velho está dormindo na rua. & sem ambiguidade & A cama está arrumada. & sem ambiguidade\\

\bottomrule
\source{Fonte: Própria.}
\end{longtable}
\end{small}


\setlength\LTleft{-1in}
\setlength\LTright{-1in}
\begin{small}
\renewcommand{\arraystretch}{1.5}
\begin{longtable}{
    >{\raggedright\arraybackslash}p{0.1\textwidth}
    p{0.1\textwidth}
    p{0.3\textwidth}
    p{0.1\textwidth}
    p{0.3\textwidth}
    p{0.1\textwidth}
    }
\caption{Conjunto de sentenças com ambiguidade sintática que foram geradas pelos modelos de linguagem no experimento 4. A coluna de categoria real indica a classificação da sentença conforme a percepção humana da ambiguidade. Sentenças categorizadas como sem ambiguidade são aquelas em que seis seres humanos não detectaram qualquer ambiguidade. As demais categorias foram atribuídas com base no referencial teórico adotado neste trabalho.}
\label{longtbl-02}
\\
\toprule
modelo & categoria solicitada & sentença & categoria real & sentença & categoria real\\
\midrule

ChatGPT & sintática & Ela viu o homem com o telescópio. & sintática & Ele trouxe o livro para o amigo do professor. & sem ambiguidade\\

ChatGPT & sintática & O rapaz encontrou o carro dirigindo na estrada. & sintática & A professora elogiou o aluno com boas notas. & sintática\\

ChatGPT & sintática & A mãe deu um presente para a filha com uma fita bonita. & sintática & O cachorro mordeu o homem com a camisa rasgada. & sem ambiguidade\\

ChatGPT & sintática & O pintor viu o quadro com os olhos cansados. & sintática & O jogador marcou um gol com a camisa amarela. & sem ambiguidade\\

ChatGPT & sintática & O médico disse ao paciente que o tratamento seria difícil devido à sua saúde frágil. & sintática e semântica & O rapaz encontrou a carteira perdida no chão do parque. & lexical\\

ChatGPT & sintática & Vi o homem com o telescópio. & sintática & Ele viu o pássaro do vizinho com binóculos. & sem ambiguidade\\

ChatGPT & sintática & Eu gosto de estudar mais do que meu irmão. & sintática & Ela viu o homem com a luneta sentado na varanda. & sintática\\

ChatGPT & sintática & O cachorro do meu amigo late durante a noite. & sintática e lexical & Maria ama sua irmã mais do que eu. & sintática\\

ChatGPT & sintática & Comi a pizza com queijo. & sintática & Vi a fotografia da Maria com o cachorro. & sintática\\

ChatGPT & sintática & Ele quebrou a janela com a bola de beisebol jogando no quintal. & sem ambiguidade & O professor disse ao aluno que a matéria era fácil. & sem ambiguidade\\

Gemini & sintática & A menina viu o cachorro do vizinho correndo pela rua. & sintática e lexical & O homem levou a esposa ao médico. & sem ambiguidade\\

Gemini & sintática & O menino caiu da árvore e quebrou o braço. & semântica e lexical & A mulher comprou o vestido verde na loja. & sem ambiguidade\\

Gemini & sintática & O homem viu a mulher na janela. & sintática e lexical & O gato comeu o rato que estava no armário. & sem ambiguidade\\

Gemini & sintática & O menino jogou a bola para o cachorro. & sem ambiguidade & A menina estava brincando com a boneca quando a mãe chegou. & sem ambiguidade\\

Gemini & sintática & O homem comprou o livro que estava na prateleira. & sem ambiguidade & O menino tirou a roupa e foi tomar banho. & sem ambiguidade\\

Gemini & sintática & O cachorro do meu tio só dá trabalho. & sintática e lexical & Carla falou pro marido que tinha chegado. & sintática\\

Gemini & sintática & Eu fiquei te esperando no banco. & lexical & Ela puxou a cadeira e acabou quebrando o braço. &  semântica e lexical\\

Gemini & sintática & A galinha brincava com uma xícara até que quebrou a asa. & semântica e lexical & O guarda deteve o suspeito em sua casa. & semântica\\

Gemini & sintática & O cachorro comeu o osso e o menino. & sem ambiguidade & O homem comprou um carro novo e uma casa. & sem ambiguidade\\

Gemini & sintática & A mulher estava com medo do cachorro. & sem ambiguidade & O menino estava andando de bicicleta na rua. & sem ambiguidade\\

\bottomrule
\source{Fonte: Própria.}
\end{longtable}
\end{small}


\setlength\LTleft{-1in}
\setlength\LTright{-1in}
\begin{small}
\renewcommand{\arraystretch}{1.5}
\begin{longtable}{
    >{\raggedright\arraybackslash}p{0.1\textwidth}
    p{0.1\textwidth}
    p{0.3\textwidth}
    p{0.1\textwidth}
    p{0.3\textwidth}
    p{0.1\textwidth}
    }
\caption{Conjunto de sentenças com ambiguidade semântica que foram geradas pelos modelos de linguagem no experimento 4. A coluna de categoria real indica a classificação da sentença conforme a percepção humana da ambiguidade. Sentenças categorizadas como sem ambiguidade são aquelas em que seis seres humanos não detectaram qualquer ambiguidade. As demais categorias foram atribuídas com base no referencial teórico adotado neste trabalho.}
\label{longtbl-02}
\\
\toprule
modelo & categoria solicitada & sentença & categoria real & sentença & categoria real\\
\midrule

ChatGPT & semântica & Ela viu o homem com o telescópio. & sintática & O banco estava vazio, exceto por um único ocupante. & lexical\\

ChatGPT & semântica & Ele pegou a maçã verde do chão. & lexical & Vi o pássaro da janela enquanto dirigia. & sintática e lexical\\

ChatGPT & semântica & O jogador atirou a bola para o goleiro. & sem ambiguidade & Ele encontrou a carta de Maria na gaveta. & lexical\\

ChatGPT & semântica & A aula foi interessante para os alunos cansados. & sem ambiguidade & A luz do sol iluminava a sala escura. & sem ambiguidade\\

ChatGPT & semântica & A pele do caçador era macia. & lexical & O cachorro viu o gato no telhado com binóculos. & sintática\\

ChatGPT & semântica & Ela viu o homem com o telescópio. & sintática & Ele encontrou a mulher de chapéu na festa. & sintática\\

ChatGPT & semântica & O pato é bom para comer. & sintática & As crianças viram o filme no parque. & sem ambiguidade\\

ChatGPT & semântica & O ladrão roubou o banco com uma máscara. & sintática e lexical & Ele leu o livro com os óculos escuros. & sem ambiguidade\\

ChatGPT & semântica & O cachorro viu o gato na varanda. & sintática & Ela pintou o quadro com a mão. & sintática\\

ChatGPT & semântica & O jogador ferido saiu do campo mancando. & sem ambiguidade & A situação melhorou após a conversa com o terapeuta. & sem ambiguidade\\

Gemini & semântica &  A menina viu o homem com o binóculo. & sintática & O homem viu a mulher com o cachorro & sintática.\\

Gemini & semântica & O menino quebrou o braço da cadeira. & sem ambiguidade & A casa do João é azul. & sem ambiguidade\\

Gemini & semântica & O livro do menino está na mesa. & sem ambiguidade & O homem está morto. & lexical\\

Gemini & semântica & O menino está com fome. & sem ambiguidade & A mulher está cansada. & sem ambiguidade\\

Gemini & semântica &  O cachorro está feliz. & sem ambiguidade & O gato está dormindo. & sem ambiguidade\\

Gemini & semântica & O homem está procurando um cachorro. & sem ambiguidade & O menino foi ao banco. & lexical\\

Gemini & semântica & O guarda prendeu o suspeito em sua casa. & semântica e sintática & O professor deu uma bronca na aluna. & sem ambiguidade\\

Gemini & semântica & O carro está parado no sinal vermelho. & lexical & O homem está com fome. & sem ambiguidade\\

Gemini & semântica & A mulher está com frio. & sem ambiguidade & O menino está com medo. & sem ambiguidade\\

Gemini & semântica & A menina está com alegria. & sem ambiguidade & O homem está com raiva. & sem ambiguidade\\

\bottomrule
\source{Fonte: Própria.}
\end{longtable}
\end{small}



\end{document}
