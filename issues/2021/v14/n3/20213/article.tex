% !TEX TS-program = XeLaTeX
% use the following command: 
% all document files must be coded in UTF-8
\documentclass{textolivre}
% for anonymous submission
%\documentclass[anonymous]{textolivre}
% to create HTML use 
%\documentclass{textolivre-html}
% See more information on the repository: https://github.com/leolca/textolivre

% Metadata
\begin{filecontents*}[overwrite]{article.xmpdata}
    \Title{Editorial}
    \Author{Bárbara Amaral da Silva}
    \Language{pt-BR}
    \Journaltitle{Texto Livre}
    \Journalnumber{1983-3652}
    \Volume{14}
    \Issue{3}
    \Firstpage{1}
    \Lastpage{2}
    \setRGBcolorprofile{sRGB_IEC61966-2-1_black_scaled.icc}
            {sRGB_IEC61966-2-1_black_scaled}
            {sRGB IEC61966 v2.1 with black scaling}
            {http://www.color.org}
\end{filecontents*}



% used in this example to provide source code environment
%\crefname{lstlisting}{lista}{listas}
%\Crefname{lstlisting}{Lista}{Listas}
%\usepackage{listings}
%\renewcommand\lstlistingname{Lista}
%\lstset{language=bash,
        breaklines=true,
        basicstyle=\linespread{1}\small\ttfamily,
        numbers=none,xleftmargin=0.5cm,
        frame=none,
        framexleftmargin=0.5em,
        framexrightmargin=0.5em,
        showstringspaces=false,
        upquote=true,
        commentstyle=\color{gray},
        literate=%
           {á}{{\'a}}1 {é}{{\'e}}1 {í}{{\'i}}1 {ó}{{\'o}}1 {ú}{{\'u}}1 
           {à}{{\`a}}1 {è}{{\`e}}1 {ì}{{\`i}}1 {ò}{{\`o}}1 {ù}{{\`u}}1
           {ã}{{\~a}}1 {ẽ}{{\~e}}1 {ĩ}{{\~i}}1 {õ}{{\~o}}1 {ũ}{{\~u}}1
           {â}{{\^a}}1 {ê}{{\^e}}1 {î}{{\^i}}1 {ô}{{\^o}}1 {û}{{\^u}}1
           {ä}{{\"a}}1 {ë}{{\"e}}1 {ï}{{\"i}}1 {ö}{{\"o}}1 {ü}{{\"u}}1
           {Á}{{\'A}}1 {É}{{\'E}}1 {Í}{{\'I}}1 {Ó}{{\'O}}1 {Ú}{{\'U}}1
           {À}{{\`A}}1 {È}{{\`E}}1 {Ì}{{\`I}}1 {Ò}{{\`O}}1 {Ù}{{\`U}}1
           {Ã}{{\~A}}1 {Ẽ}{{\~E}}1 {Ũ}{{\~u}}1 {Õ}{{\~O}}1 {Ũ}{{\~U}}1
           {Â}{{\^A}}1 {Ê}{{\^E}}1 {Î}{{\^I}}1 {Ô}{{\^O}}1 {Û}{{\^U}}1
           {Ä}{{\"A}}1 {Ë}{{\"E}}1 {Ï}{{\"I}}1 {Ö}{{\"O}}1 {Ü}{{\"U}}1
           {ç}{{\c{c}}}1 {Ç}{{\c{C}}}1
}


\journalname{Texto Livre: Linguagem e Tecnologia}
\thevolume{14}
\thenumber{3}
\theyear{2021}
\receiveddate{\DTMdisplaydate{2021}{11}{16}{-1}} % YYYY MM DD
\accepteddate{\DTMdisplaydate{2021}{11}{19}{-1}}
\publisheddate{\DTMdisplaydate{2021}{11}{19}{-1}}
% Corresponding author
\corrauthor{Bárbara Amaral da Silva}
\articledoi{}
\articleid{37090}
% Abbreviated author list for the running footer
\runningauthor{Silva}
\editorname{Daniervelin Pereira}

\title{Editorial}
% if there is a third language title, add here:
%\othertitle{Artikelvorlage zur Einreichung beim Texto Livre Journal}

\author[1]{Bárbara Amaral da Silva \orcid{0000-0002-1469-9575} \thanks{Email: \url{barbara.amaral87@gmail.com}}}

\affil[1]{University of Windsor, Windsor, Canadá; Universidade Federal de Minas Gerais, Belo Horizont-MG, Brasil.}

% set language of the article
\setdefaultlanguage[variant=brazilian]{portuguese}
\setotherlanguage{english}

%\usepackage{lineno}
%\linenumbers
\begin{document}
\maketitle

A revista Texto Livre conta com contribuições de autores nacionais e internacionais. Pesquisadores do Brasil, do Canadá, da Arábia Saudita, de Portugal, da Espanha, do Chile, do Equador e do México enriquecem a edição número 3 do 14° volume, de 2021, com diferentes perspectivas a respeito da linguagem e da tecnologia a partir de diferentes objetos de estudo. 

No artigo \href{https://doi.org/10.35699/1983-3652.2021.27047}{“Atividades experimentais em tempos de pandemia: o uso da plataforma online PCIBEX para experimentos psicolinguísticos”}, Aline Alves Fonseca, Júlia Greco Carvalho e Samara Cristina da Silva Zanella analisam o uso da plataforma Penn Controller for Ibex (PCIbex), que permite a realização de experimentos de forma online, comparando-a com outras semelhantes e mostrando seus pontos positivos. Em \href{https://doi.org/10.35699/1983-3652.2021.29627}{“Transletramentos: o ensino de língua portuguesa mediado pelas TDIC”}, Adriane Elisa Glasser e Maria Elena Pires Santos discutem a contribuição dos transletramentos em práticas pedagógicas em aulas de Língua Portuguesa da 3ª série do Ensino Médio de um Colégio Estadual, concluindo que a produção do conhecimento foi mais significativa para o aluno que participou ativamente de todo o processo. Em \href{https://doi.org/10.35699/1983-3652.2021.29714}{“Muito além do relvado: futebol, nacionalismo e redes sociais”}, Branco Di Fátima, Célia Gouveia e Sandra Miranda observam o envolvimento de fãs das seleções de Portugal, Espanha, Marrocos e Irã com suas respectivas páginas oficiais no Facebook e também como essas páginas se relacionam com os stakeholders, evidenciando que as interações sofrem influência dos contextos econômico, tecnológico e político. Em \href{https://doi.org/10.35699/1983-3652.2021.32563}{“Barreiras tecnológicas: um fator limitador na acessibilidade das pessoas com deficiência”}, Marcelo de Santana Porte e José Damião Trindade Rocha avaliam a literatura que aborda as barreiras tecnológicas existentes no Brasil direcionadas às Pessoas com Deficiência (PcD), concluindo que às barreiras tecnológicas se associam barreiras nas comunicações e informações. Em \href{https://doi.org/10.35699/1983-3652.2021.33162}{“O portal do projeto PROENEM (UNILAB) como plataforma pedagógica de ensino de argumentação e escrita”}, Adriely da Silva Clemente, Leonardo Chaves Ferreira e José Olavo da Silva Garantizado Júnior verificam como materiais didáticos de duas oficinas disponibilizadas no Portal PRO-ENEM (Unilab) colaboraram para a construção argumentativa em 10 textos, no estilo do ENEM, de estudantes pré-universitários, concluindo que utilizar recursos tecnológicos disponíveis nesse Portal pode ser eficaz para atingir tal objetivo. 

Em \href{https://doi.org/10.35699/1983-3652.2021.33890}{“An analysis of social communicative acts among MMORPG players”}, Ricardo Casañ-Pitarch, a partir de um questionário, analisa atos comunicativos utilizados por jogadores de videogame para se socializarem em MMORPG, apresentando variáveis que interferem nessa socialização. Em \href{https://doi.org/10.35699/1983-3652.2021.35333}{“Jogos digitais e acentuação gráfica conexões possíveis entre aprendizagem e ludicidade”}, Ana Luiza de Souza Couto, Letícia Pena Silveira e Marcelo de Castro exploram jogos digitais criados para o ensino e a aprendizagem de acentuação gráfica, mostrando que os jogos trabalharam a acentuação de forma descontextualizada e que não conseguiram garantir, nem mesmo, a ludicidade. Em \href{https://doi.org/10.35699/1983-3652.2021.29457}{“Tendencias en el estado del arte de las narrativas audiovisuales móviles en el siglo XXI: revisión sistemática de la literatura”}, Valeriano Durán Manso, María-Victoria Carrillo-Durán e Javier Trabadela-Robles retomam e analisam trabalhos que abordam narrativas audiovisuais móveis, ressaltando os diferentes enfoques de cada trabalho, assim como as variadas metodologias. Em \href{https://doi.org/10.35699/1983-3652.2021.29629}{“Matemática e Física em experiências de Robótica Livre explorando o sensor ultrassônico”}, Marcelo Pires da Silva e Fernando da Costa Barbosa abordam a aprendizagem de Matemática e Física a partir da criação de robôs com a utilização de materiais livres e tendo como base o Construcionismo e a Espiral de Aprendizagem Criativa. Em \href{https://doi.org/10.35699/1983-3652.2021.31351}{“La competencia digital del profesorado de literatura en Educación Secundaria en España”}, Francisco José Rodríguez Muñoz e María del Mar Ruiz-Domínguez, a partir de um questionário, quantificam e descrevem o grau de conhecimento e de utilização de TIC na aula de literatura, revelando a baixa porcentagem de professores com intenção de formação em TIC, assim como as variáveis que interferem na utilização.

Em \href{https://doi.org/10.35699/1983-3652.2021.32572}{“‘Eu vejo que eles estão engajados’: mediação, interação e investimento no desenvolvimento da compreensão leitora em Língua Inglesa em contexto de ensino remoto emergencial”}, Manuela da Silva Alencar de Souza e Christine Siqueira Nicolaides apresentam os dados de uma entrevista realizada com uma professora de línguas, a partir da qual observam que o investimento dos alunos no aprendizado de leitura em língua inglesa está relacionado a instrumentos físicos e simbólicos de mediação no contexto do ERE. Em \href{https://doi.org/10.35699/1983-3652.2021.33445}{“Personalized and adaptive learning: educational practice and technological impact”}, Rebeca Soler Costa, Qing Tan, Frédérique Pivot, Xiaokun Zhang e Harris Wang discorrem sobre como a aprendizagem personalizada e adaptativa pode ajudar o aluno a alcançar uma aprendizagem eficiente no contexto da educação tecnológica do século XXI, argumentando que é obrigação social e moral de educadores e instituições aplicá-la de forma sábia. Em \href{https://doi.org/10.35699/1983-3652.2021.34141}{“Percepções dos professores em formação sobre mensagens instantâneas e competência ortográfica”}, Francisco Núñez-Román, Alejandro Gómez-Camacho, María Constanza Errázuriz-Cruz e Juan Antonio Núñez-Cortés examinam a percepção de professores de espanhol em formação no Chile e na Argentina sobre a forma como adolescentes escrevem em app digitais em mensagens instantâneas, concluindo que esse uso impacta negativamente, comprometendo o desenvolvimento de competências ortográficas. Em \href{https://doi.org/10.35699/1983-3652.2021.24923}{“Aplicativos móveis como recursos didáticos digitais: um mapeamento na educação formal”}, Kadhiny Policarpo e Juliana Cristina Faggion Bergmann realizam uma revisão sistemática para compreenderem se e como as tecnologias móveis estão sendo usadas em salas de aula para fins pedagógicos. Em \href{https://doi.org/10.35699/1983-3652.2021.29459}{“Estratégias de pós-edição na tradução automática de provérbios por alunos da FLE e da tradução”}, Rana Kandeel estuda estratégias usadas na pós-edição de provérbios traduzidos automaticamente do francês para o árabe por alunos de francês como língua estrangeira. 

Em \href{https://doi.org/10.35699/1983-3652.2021.29687}{“El tema de la donación de órganos en Facebook: análisis de la fanpage del INDOT de Ecuador”}, Gabriel Francisco Cevallos, Jonathan Bladimir Zhiminaicela, María Fernanda Fernández e Sueny Paloma dos Santos examinam interações ocorridas na página “Instituto Nacional de Doação e Transplante de Órgãos, Tecidos e Células (INDOT) do Equador”, no Facebook, concluindo que, embora a fanpage tenha uma boa visibilidade, não há um diálogo significativo. Em \href{https://doi.org/10.35699/1983-3652.2021.32557}{“A intertextualidade em hipertextos: uma análise de tweets de cunho didático”}, Ana Claudia Oliveira Azevedo e Márcia Helena de Melo Pereira analisam diferentes tipos de intertextualidade em tweets que tratam de conteúdos didáticos de diversas áreas do conhecimento, evidenciando que o hipertexto viabiliza diversos tipos de intertextualidade e que esse fenômeno se manifesta a partir da multimodalidade. Em \href{https://doi.org/10.35699/1983-3652.2021.33027}{“Aplicaciones que emplean y recomendaciones que entregan las y los docentes universitarios para la autorregulación del aprendizaje en contexto de la pandemia por COVID-19”}, Valeria Aylín Infante-Villagrán, Bianca Maria Pia Dapelo Pellerano, Rubia Cobo-Rendon, Yaranay López-Ângulo, Bertha Escobar Alaniz e Christian Beyle apresentam um estudo exploratório em que são identificados aplicativos digitais usados e recomendados por professores chilenos para a autorregulação da aprendizagem em ensino virtual, no contexto da pandemia de COVID-19, sendo identificados 27 aplicativos relevantes. Em \href{https://doi.org/10.35699/1983-3652.2021.33750}{“Social innovation laboratories for the social construction of knowledge systematic review of literature”}, José-Antonio Yañez-Figueroa, María-Soledad Ramírez-Montoya e Francisco-José García-Peñalvo identificam os estudos mais relevantes sobre a construção social do conhecimento ligada a problemas ambientais e propõem soluções para a sustentabilidade. Em \href{https://doi.org/10.35699/1983-3652.2021.35052}{Fidelidad y praxeologías en aplicaciones didácticas desarrolladas para la resolución de expresiones matemáticas”}, Alberto Camacho Ríos, Bertha Ivonne Sánchez Luján e Marisela Caldera-Franco analisam um aplicativo matemático para Android e estabelecem a fidelidade existente entre a simbologia matemática apresentada pela interface do aplicativo em relação àquela descrita pelos alunos em seus cadernos. 

Desejamos a todos uma leitura produtiva desses textos e que eles inspirem novas pesquisas! 

\end{document}
