% !TEX TS-program = XeLaTeX
% use the following command: 
% all document files must be coded in UTF-8
\documentclass{textolivre}
% for anonymous submission
%\documentclass[anonymous]{textolivre}
% to create HTML use 
%\documentclass{textolivre-html}
% See more information on the repository: https://github.com/leolca/textolivre

% Metadata
\begin{filecontents*}[overwrite]{article.xmpdata}
    \Title{Tecnología educativa para la agenda 2030: Objetivos de Desarrollo Sostenible (ODS) ante la pandemia}
    \Author{Santiago Alonso-García \sep José María Romero-Rodríguez \sep José Antonio Marín-Marín \sep Fernando José Sadio-Ramos}
    \Language{es}
    \Keywords{Educación \sep Tecnología \sep Objetivos de Desarrollo Sostenible}
    \Journaltitle{Texto Livre}
    \Journalnumber{1983-3652}
    \Volume{14}
    \Issue{2}
    \Firstpage{1}
    \Lastpage{16}

    \setRGBcolorprofile{sRGB_IEC61966-2-1_black_scaled.icc}
            {sRGB_IEC61966-2-1_black_scaled}
            {sRGB IEC61966 v2.1 with black scaling}
            {http://www.color.org}
\end{filecontents*}

% used to create dummy text for the template file
\definecolor{dark-gray}{gray}{0.35} % color used to display dummy texts
\usepackage{lipsum}
\SetLipsumParListSurrounders{\colorlet{oldcolor}{.}\color{dark-gray}}{\color{oldcolor}}

% used here only to provide the XeLaTeX and BibTeX logos
\usepackage{hologo}

% used in this example to provide source code environment
%\crefname{lstlisting}{lista}{listas}
%\Crefname{lstlisting}{Lista}{Listas}
%\usepackage{listings}
%\renewcommand\lstlistingname{Lista}
%\lstset{language=bash,
        breaklines=true,
        basicstyle=\linespread{1}\small\ttfamily,
        numbers=none,xleftmargin=0.5cm,
        frame=none,
        framexleftmargin=0.5em,
        framexrightmargin=0.5em,
        showstringspaces=false,
        upquote=true,
        commentstyle=\color{gray},
        literate=%
           {á}{{\'a}}1 {é}{{\'e}}1 {í}{{\'i}}1 {ó}{{\'o}}1 {ú}{{\'u}}1 
           {à}{{\`a}}1 {è}{{\`e}}1 {ì}{{\`i}}1 {ò}{{\`o}}1 {ù}{{\`u}}1
           {ã}{{\~a}}1 {ẽ}{{\~e}}1 {ĩ}{{\~i}}1 {õ}{{\~o}}1 {ũ}{{\~u}}1
           {â}{{\^a}}1 {ê}{{\^e}}1 {î}{{\^i}}1 {ô}{{\^o}}1 {û}{{\^u}}1
           {ä}{{\"a}}1 {ë}{{\"e}}1 {ï}{{\"i}}1 {ö}{{\"o}}1 {ü}{{\"u}}1
           {Á}{{\'A}}1 {É}{{\'E}}1 {Í}{{\'I}}1 {Ó}{{\'O}}1 {Ú}{{\'U}}1
           {À}{{\`A}}1 {È}{{\`E}}1 {Ì}{{\`I}}1 {Ò}{{\`O}}1 {Ù}{{\`U}}1
           {Ã}{{\~A}}1 {Ẽ}{{\~E}}1 {Ũ}{{\~u}}1 {Õ}{{\~O}}1 {Ũ}{{\~U}}1
           {Â}{{\^A}}1 {Ê}{{\^E}}1 {Î}{{\^I}}1 {Ô}{{\^O}}1 {Û}{{\^U}}1
           {Ä}{{\"A}}1 {Ë}{{\"E}}1 {Ï}{{\"I}}1 {Ö}{{\"O}}1 {Ü}{{\"U}}1
           {ç}{{\c{c}}}1 {Ç}{{\c{C}}}1
}


\journalname{Texto Livre}
\thevolume{14}
\thenumber{2}
\theyear{2021}
\receiveddate{\DTMdisplaydate{2021}{6}{01}{-1}} % YYYY MM DD
\accepteddate{\DTMdisplaydate{2021}{6}{27}{-1}}
\publisheddate{\today}
% Corresponding author
\corrauthor{Santiago Alonso-García}
%\articledoi{{dito}rial}
\articledoi{}
%\articleid{34883}
% list of available sesscions in the journal: articles, dossier, reports, essays, reviews, interviews, editorial
\articlesessionname{editorial}
% Abbreviated author list for the running footer
\runningauthor{Alonso-García et al.}
\editorname{Daniervelin Pereira}

\title{Tecnología educativa para la agenda 2030: Objetivos de Desarrollo Sostenible (ODS) ante la pandemia}
\othertitle{Tecnologia educativa para a agenda 2030: Objetivos para Desenvolvimento Sustentável (ODS) diante da pandemia}
\othertitle{Educational technology for the 2030 agenda: Objectives of Sustainable Development (OSD) in the face of the pandemic}
% if there is a third language title, add here:
%\othertitle{Artikelvorlage zur Einreichung beim Texto Livre Journal}

\author[1]{Santiago Alonso-García~\orcid{0000-0002-9525-709X}~\thanks{Email: \url{salonsog@go.ugr.es}}}
\author[1]{José María Romero-Rodríguez~\orcid{0000-0002-9284-8919}~\thanks{Email: \url{romejo@ugr.es}}}
\author[1]{José Antonio Marín-Marín~\orcid{0000-0001-8623-4796}~\thanks{Email: \url{jmarin@ugr.es}}}
\author[2]{Fernando José Sadio-Ramos~\orcid{0000-0001-7654-5638}~\thanks{Email: \url{framos@esec.pt}}}

\affil[1]{Universidad de Granada, Grupo de Investigación AREA (HUM-672), Granada, España.}
\affil[2]{Instituto Politécnico de Coimbra, Grupo de Investigación AREA (HUM-672), Coimbra, Portugal.}
\renewcommand\Authand{ y }\renewcommand\Authands{, y }


\addbibresource{article.bib}
% use biber instead of bibtex
% $ biber tl-article-template

% set language of the article
\setdefaultlanguage{spanish}
\setotherlanguage{portuguese}
\setotherlanguage{english}

% for spanish, use:
%\setdefaultlanguage{spanish}
%\gappto\captionsspanish{\renewcommand{\tablename}{Tabla}} % use 'Tabla' instead of 'Cuadro'
%\AfterEndPreamble{\crefname{table}{tabla}{tablas}\Crefname{table}{Tabla}{Tablas}}

% for languages that use special fonts, you must provide the typeface that will be used
% \setotherlanguage{arabic}
% \newfontfamily\arabicfont[Script=Arabic]{Amiri}
% \newfontfamily\arabicfontsf[Script=Arabic]{Amiri}
% \newfontfamily\arabicfonttt[Script=Arabic]{Amiri}
%
% in the article, to add arabic text use: \textlang{arabic}{ ... }

% to use emoticons in your manuscript
% https://stackoverflow.com/questions/190145/how-to-insert-emoticons-in-latex/57076064
% using font Symbola, which has full support
% the font may be downloaded at:
% https://dn-works.com/ufas/
% add to preamble:
% \newfontfamily\Symbola{Symbola}
% in the text use:
% {\Symbola }

% reference itens in a descriptive list using their labels instead of numbers
% insert the code below in the preambule:
\makeatletter
\let\orgdescriptionlabel\descriptionlabel
\renewcommand*{\descriptionlabel}[1]{%
  \let\orglabel\label
  \let\label\@gobble
  \phantomsection
  \edef\@currentlabel{#1\unskip}%
  \let\label\orglabel
  \orgdescriptionlabel{#1}%
}
\makeatother
%
% in your document, use as illustraded here:
%\begin{description}
%  \item[first\label{itm1}] this is only an example;
%  % ...  add more items
%\end{description}
 

% custom epigraph - BEGIN 
%%% https://tex.stackexchange.com/questions/193178/specific-epigraph-style
\usepackage{epigraph}
\renewcommand\textflush{flushright}
\makeatletter
\newlength\epitextskip
\pretocmd{\@epitext}{\em}{}{}
\apptocmd{\@epitext}{\em}{}{}
\patchcmd{\epigraph}{\@epitext{#1}\\}{\@epitext{#1}\\[\epitextskip]}{}{}
\makeatother
\setlength\epigraphrule{0pt}
\setlength\epitextskip{0.5ex}
\setlength\epigraphwidth{.7\textwidth}
% custom epigraph - END


% if you use multirows in a table, include the multirow package
\usepackage{multirow}

% add line numbers for submission
%\usepackage{lineno}
%\linenumbers

\begin{document}
\maketitle

\section*{Editorial}\label{editorial}
La sociedad de la información ha provocado una significativa evolución en nuestra vida cotidiana en base al auge tecnológico que afecta al ámbito educativo, familiar, social, cultural y político, generándose esa relación que se da entre la sociedad de la información y sociedad del conocimiento con la finalidad de alcanzar mejores estándares de calidad para el bienestar y progreso de la educación \cite{perez_zuniga_sociedad_2018}.

Las tecnologías de aprendizaje son esenciales para la enseñanza y el aprendizaje de los estudiantes en el siglo XXI. El paradigma de la enseñanza ha cambiado, colocando al alumno como eje principal del proceso de enseñanza-aprendizaje, lo cual se ha acentuado en el contexto de pandemia sufrido en el último año, potenciándose diferentes metodologías de aprendizaje basadas en las TIC, como el e-learning , el blended learning, el aula invertida y el aprendizaje móvil \cite{rodriguez_tecnologias_2020}. En esta edición especial resultaron de interés aportaciones que incluían aspectos clave de estas metodologías, así como tecnologías emergentes como la realidad mixta, la inteligencia artificial, blockchain, telepresencia, Internet de las cosas (IoT), tecnologías analíticas, asistentes virtuales, entre otras tendencias destacadas en \textcite{alexander_educause_2019}.

El número incluye diferentes contribuciones agrupadas en diferentes secciones. Además, participaron docentes y profesionales del campo de la educación de instituciones ubicadas en diferentes localizaciones geográficas, como España, México, Chile, Brasil, Ecuador y Cuba. Esto denota el interés que despertó el número especial, que reúne una riqueza única, derivada de la variabilidad de los autores que solicitaron su participación.

La temática de los trabajos del número se presenta de manera ajustada al contexto pandémico y a la revulsión de las tecnologías como clave de la comunicación educativa en una sociedad aun escéptica sobre su valía real. Podemos ver trabajos como Uso de las TIC y atención a la diversidad en tiempos de COVID; Análisis de la productividad en torno a la alfabetización informacional a nivel de educación superior; TIC y ocio familiar durante el encierro, agentes involucrados; Efectividad del aprendizaje invertido y realidad aumentada en la nueva normalidad educativa de la era Covid-19; Análisis de la experiencia de los profesores de educación superior portugueses en el contexto del COVID-19; Obsolescencia del conocimiento y la formación para el desarrollo sostenible, voces de protagonistas dentro de COVID 19; Estilos de aprendizaje y metas de logro de estudiantes universitarios durante la pandemia COVID-19; Tecnología y abstracción, desarrollo de habilidades complejas a través de videojuegos; Modelo SEM de formación docente tecnológica, ecológica e inclusiva en tiempos de pandemia; Compromiso con la educación a distancia, experiencias durante la pandemia COVID-19; Educación de calidad y pandemia, desafíos, experiencias y propuestas de los estudiantes en la formación docente en Ecuador. 

Así, el número especial muestra una interesante línea de investigación, actual y de plena relevancia, como son las tecnologías de aprendizaje, especialmente en el contexto actual de una pandemia derivada del COVID-19, en el que las TIC dan la opción del éxito didáctico. No cabe duda de que los trabajos presentados se postulan como un punto de referencia em el ámbito de la tecnología educativa. Finalmente deseamos invitar a todos aquellos interesados en las tecnologías aplicadas a la educación a revisar y consultar los diferentes trabajos para reflexionar sobre ellos y para que sirvan de inspiración para futuras líneas de investigación y divulgación científica.

\printbibliography\label{sec-bib}
%\printbibliography[title={Contenido}]\label{sec-bib}

\end{document}
