% !TEX TS-program = XeLaTeX
% use the following command: 
% all document files must be coded in UTF-8
\documentclass{textolivre}
% for anonymous submission
%\documentclass[anonymous]{textolivre}
% to create HTML use 
%\documentclass{textolivre-html}
% See more information on the repository: https://github.com/leolca/textolivre

% Metadata
\begin{filecontents*}[overwrite]{article.xmpdata}
    \Title{Construcción y resignificación de prácticas tecnológicas en entornos escolares: el rizoma como camino metodológico}
    \Author{Leonardo Araújo \sep Daniervelin Pereira}
    \Language{es}
    \Keywords{Práctica \sep Tecnología \sep Cultura \sep Jóvenes}
    \Journaltitle{Texto Livre}
    \Journalnumber{1983-3652}
    \Volume{14}
    \Issue{1}
    \Firstpage{1}
    \Lastpage{16}
    \Doi{10.35699/1983-3652.2021.29523}

    \setRGBcolorprofile{sRGB_IEC61966-2-1_black_scaled.icc}
            {sRGB_IEC61966-2-1_black_scaled}
            {sRGB IEC61966 v2.1 with black scaling}
            {http://www.color.org}
\end{filecontents*}

% used to create dummy text for the template file
\definecolor{dark-gray}{gray}{0.35} % color used to display dummy texts
\usepackage{lipsum}
\SetLipsumParListSurrounders{\colorlet{oldcolor}{.}\color{dark-gray}}{\color{oldcolor}}

% used here only to provide the XeLaTeX and BibTeX logos
\usepackage{hologo}

% used in this example to provide source code environment
%\crefname{lstlisting}{lista}{listas}
%\Crefname{lstlisting}{Lista}{Listas}
%\usepackage{listings}
%\renewcommand\lstlistingname{Lista}
%\lstset{language=bash,
        breaklines=true,
        basicstyle=\linespread{1}\small\ttfamily,
        numbers=none,xleftmargin=0.5cm,
        frame=none,
        framexleftmargin=0.5em,
        framexrightmargin=0.5em,
        showstringspaces=false,
        upquote=true,
        commentstyle=\color{gray},
        literate=%
           {á}{{\'a}}1 {é}{{\'e}}1 {í}{{\'i}}1 {ó}{{\'o}}1 {ú}{{\'u}}1 
           {à}{{\`a}}1 {è}{{\`e}}1 {ì}{{\`i}}1 {ò}{{\`o}}1 {ù}{{\`u}}1
           {ã}{{\~a}}1 {ẽ}{{\~e}}1 {ĩ}{{\~i}}1 {õ}{{\~o}}1 {ũ}{{\~u}}1
           {â}{{\^a}}1 {ê}{{\^e}}1 {î}{{\^i}}1 {ô}{{\^o}}1 {û}{{\^u}}1
           {ä}{{\"a}}1 {ë}{{\"e}}1 {ï}{{\"i}}1 {ö}{{\"o}}1 {ü}{{\"u}}1
           {Á}{{\'A}}1 {É}{{\'E}}1 {Í}{{\'I}}1 {Ó}{{\'O}}1 {Ú}{{\'U}}1
           {À}{{\`A}}1 {È}{{\`E}}1 {Ì}{{\`I}}1 {Ò}{{\`O}}1 {Ù}{{\`U}}1
           {Ã}{{\~A}}1 {Ẽ}{{\~E}}1 {Ũ}{{\~u}}1 {Õ}{{\~O}}1 {Ũ}{{\~U}}1
           {Â}{{\^A}}1 {Ê}{{\^E}}1 {Î}{{\^I}}1 {Ô}{{\^O}}1 {Û}{{\^U}}1
           {Ä}{{\"A}}1 {Ë}{{\"E}}1 {Ï}{{\"I}}1 {Ö}{{\"O}}1 {Ü}{{\"U}}1
           {ç}{{\c{c}}}1 {Ç}{{\c{C}}}1
}


\journalname{Texto Livre: Linguagem e Tecnologia}
\thevolume{14}
\thenumber{1}
\theyear{2021}
\receiveddate{\DTMdisplaydate{2021}{2}{19}{-1}} % YYYY MM DD
\accepteddate{\DTMdisplaydate{2021}{3}{24}{-1}}
\publisheddate{\today}
% Corresponding author
\corrauthor{Juan Guillermo Diaz Bernal}
% DOI
\articledoi{10.35699/1983-3652.2021.29523}
% Abbreviated author list for the running footer
\runningauthor{Pérez Puentes y Diaz Bernal}
\editorname{Hugo Heredia Ponce}

\title{Construcción y resignificación de prácticas tecnológicas en entornos escolares: el rizoma como camino metodológico}
\othertitle{Construção e ressignificação das práticas tecnológicas em entornos escolares: rizoma como caminho metodológico}
\othertitle{Construction and reframing of technological practices in school environments: the rhizome as methodological path}
% if there is a third language title, add here:
%\othertitle{Artikelvorlage zur Einreichung beim Texto Livre Journal}

\author[1]{Ana Yamile Pérez Puentes \orcid{0000-0001-6175-6619} \thanks{Email: \url{anayamile.perez@uptc.edu.co}}}
\author[2]{Juan Guillermo Diaz Bernal \orcid{0000-0001-8910-820X} \thanks{Email: \url{juan.diaz@uptc.edu.co}}}

\affil[1]{Secretaria de Educación de Boyacá, Colômbia.}
\affil[2]{Universidad Pedagógica y Tecnológica de Colombia , Colômbia.}

\addbibresource{article.bib}
% use biber instead of bibtex
% $ biber tl-article-template

% set language of the article
\setdefaultlanguage{spanish}
\setotherlanguage{portuguese}
\setotherlanguage{english}
\gappto\captionsspanish{\renewcommand{\tablename}{Tabla}} % use 'Tabla' instead of 'Cuadro'
\AfterEndPreamble{\crefname{table}{tabla}{tablas}}

% for languages that use special fonts, you must provide the typeface that will be used
% \setotherlanguage{arabic}
% \newfontfamily\arabicfont[Script=Arabic]{Amiri}
% \newfontfamily\arabicfontsf[Script=Arabic]{Amiri}
% \newfontfamily\arabicfonttt[Script=Arabic]{Amiri}
%
% in the article, to add arabic text use: \textlang{arabic}{ ... }

% to use emoticons in your manuscript
% https://stackoverflow.com/questions/190145/how-to-insert-emoticons-in-latex/57076064
% using font Symbola, which has full support
% the font may be downloaded at:
% https://dn-works.com/ufas/
% add to preamble:
% \newfontfamily\Symbola{Symbola}
% in the text use:

% reference itens in a descriptive list using their labels instead of numbers
% insert the code below in the preambule:
%
\makeatletter
\let\orgdescriptionlabel\descriptionlabel
\renewcommand*{\descriptionlabel}[1]{%
  \let\orglabel\label
  \let\label\@gobble
  \phantomsection
  \edef\@currentlabel{#1\unskip}%
  \let\label\orglabel
  \orgdescriptionlabel{#1}%
}
\makeatother
%
% in your document, use as illustraded here:
%\begin{description}
%  \item[first\label{itm1}] this is only an example;
%  % ...  add more items
%\end{description}
 

\usepackage{multirow}

\begin{document}
\maketitle

\begin{polyabstract}
\begin{abstract}
Con el objetivo de situar construcciones y resignificaciones culturales de las prácticas tecnológicas de los jóvenes desde las dinámicas escolares, el presente artículo está orientado desde la problematización de estos campos para caracterizar y reconocer formas de construcciones culturales, resignificaciones y reinterpretación de acontecimientos y conceptos vinculados al cuestionar y reflexionar sobre ocho prácticas tecnológicas. El estado de conocimiento plantea cuatro líneas de indagación: (1) subjetividades, jóvenes y tecnologías digitales; (2) escenarios de comunicación digital y jóvenes; (3) cultura política y educativa en mediaciones tecnológicas; y (4) TIC y educación. Por su parte, el marco teórico se fundamenta en las relaciones entre tecnología, cultura, sociedad de la información y sociedad de control, las cuales convergen en el hecho educativo, comunicativo y tecnológico. La perspectiva metodológica gira en torno a una propuesta rizomática para construir un camino investigativo, lo que implica puntualizarla como una composición en pro de la reivindicación de la subjetividad. Esta atiende al \textit{encuentro} como posibilidad y al \textit{devenir} como caminos metodológicos, fundamentados teóricamente en las ideas de Deleuze y Guattari. Se plantean tres tipos de conclusiones: el primero tiene que ver con el investigador-escritor; el segundo, con las construcciones culturales diversificadas; y el tercero, con la aplicación de la perspectiva metodológica con sus variables.

\keywords{Práctica \sep Tecnología \sep Cultura \sep Jóvenes}
\end{abstract}

\begin{portuguese}
\begin{abstract}
Com o objetivo de situar construções e ressignificações culturais das práticas tecnológicas de jovens em diferentes dinâmicas escolares, o presente artigo é orientado pelas problematizações desses campos para caraterizar e reconhecer formas de construções culturais, ressignificação e reinterpretação de acontecimentos e conceitos dedicados a questionar e refletir sobre oito práticas tecnológicas. O estado de conhecimento levanta quatro linhas de pesquisa: (1) subjetividades, jovens e tecnologias digitais; (2) cenários de comunicação digital e jovens; (3) cultura política e educativa em mediações tenológicas; e (4) TIC e educação. A pesquisa teórica se fundamenta nas relações entre tecnologia, cultura, sociedade da informação e sociedade de controle, as quais convergem no fato educacional, comunicativo e tecnológico. A perspetiva metodológica gira em torno de uma proposta rizomática para construir um caminho de pesquisa, o que implica defini-la como uma composição a favor da reivindicação da subjetividade. Este atende ao \textit{encontro} como possibilidade e ao \textit{devir} como caminhos metodológicos, teoricamente a partir das ideias de Deleuze e Guattari. Propõem-se três tipos de conclusões: a primeira tem a ver com um pesquisador-escritor; a segunda, com as construções culturais diversificadas; e a terceira, com a aplicação da perspetiva metodológica e suas variáveis.

\keywords{Prática \sep Tecnologia \sep Cultura \sep Jovens}
\end{abstract}
\end{portuguese}

\begin{english}
\begin{abstract}
The point of view locating a cultural constructions and resignifications of the technological practices of young people from the school dynamics. The present article is oriented from the problematization of fields to characterize and recognize forms of cultural constructions, resignifications and reinterpretation of related events and concepts by questioning and reflecting on eight technological practices. The review raises four lines of inquiry: (1) subjectivities, youth, and digital technologies; (2) youth and digital communication scenarios; (3) political and educational culture in technological mediations; and (4) ICT and education. The theoretical framework is based on the relationships between technology, culture, information society and control society, which converge on the educational, communicative and technological fact. The methodological perspective revolves around a rhizomatic proposal to build an investigative path, which implies defining it as a composition in favor of the claim of subjectivity. This attends to the meeting as a possibility and to becoming as methodological, based on the ideas of Deleuze and Guattari. Three types of conclusions are raised: the first has to do with the researcher-writer; the second, with diversified cultural constructions; and the third, with the application of the methodological perspective with its variables.

\keywords{Practice \sep Technology \sep Culture \sep Young people}
\end{abstract}
\end{english}

% if there is another abstract, insert it here using the same scheme
\end{polyabstract}


\section{Introducción}\label{sec-intro}
A finales del siglo XX y principios del XXI, los estudios socio-técnico-culturales \cite{levy_cibercultura._2007} se entienden como la exploración de cierto consenso en la comprensión de la tecnología en las culturas ajustadas de las sociedades, implicando que las tecnologías configuren contundentemente las formas imperiosas tanto de comunicación, información y conocimiento como de investigación, elaboración y distribución. Esto abre varias líneas investigativas y, a la vez, evidencia la evolución de conceptos para entender lo tecnológico de otro modo, a partir de subjetividades, de nuevas formas de comunicación ―digital―, de la cultura política y de las influencias en las redes sociales. Puntualmente, esta investigación tiene como base las cuestiones educativas desde un enfoque de las prácticas tecnológicas, las cuales están en construcción y, por ende, en debates continuos desde diferentes perspectivas frente a las implicaciones mediadoras entre: tecnología y sociedad de la información; sujeto y tecnología; y TIC y educación. Esto permite cuestionar qué tanto, de qué forma o si realmente las tecnologías están involucradas ―o por lo menos cercanas― al contexto escolar.

Por lo anterior y con el objetivo de situar construcciones y resignificaciones culturales de las prácticas tecnológicas de los jóvenes desde las dinámicas escolares, se presenta un ejercicio investigativo orientado desde la problematización de estos campos, lo cual permite caracterizar y reconocer formas de construcciones culturales, resignificaciones y reinterpretación de acontecimientos y conceptos vinculados al cuestionar y reflexionar sobre ocho practicas tecnológicas:

\begin{itemize}
    \item prácticas comunicativas;
    \item prácticas performativas y escritura en la mirada;
    \item prácticas del “yo” y del “nosotros”;
    \item prácticas del control;
    \item prácticas del caos;
    \item prácticas de lo opuesto;
    \item prácticas del sonido y del movimiento;
    \item prácticas de la imagen y de la representación.
\end{itemize}

Estas están diferenciadas por atributos propios que permiten reconocer formas reestructuradas de sí mismas desde las percepciones e interpretaciones dependiendo de sus conexiones. Por ello, se atiende a la necesidad de interactuar con nuevas y transformadas prácticas, que a su vez favorecen nuevas responsabilidades y la posibilidad de otras acciones pedagógicas; en este caso, en el escenario escolar de la Fundación Pedagógica Rayuela de la ciudad de Tunja, Boyacá, Colombia.

Teniendo en cuenta lo anterior, se elabora un estado del conocimiento dedicado a la revisión bibliográfica sobre estudios ciberculturales. Se plantean cuatro líneas de indagación: 

\begin{itemize}
    \item subjetividades, jóvenes y tecnologías digitales;
    \item escenarios de comunicación digital y jóvenes;
    \item cultura política y educativa en mediaciones tecnológicas;
    \item TIC y educación, soportadas en la relación comunicación, información y educación. 
\end{itemize}
    
Por su parte, el marco teórico se fundamenta en las relaciones y argumentaciones entre tecnología, cultura, sociedad de la información y sociedad del control, las cuales convergen en el hecho educativo, comunicativo, tecnológico y en las prácticas tecnológicas. 

La perspectiva metodológica gira en torno a una propuesta rizomática como parte de la necesidad de construir un camino investigativo, lo que implica puntualizarla como una composición en pro de la reivindicación de subjetividad y no se propone como una metodología universal. Esta atiende al \textit{encuentro} como posibilidad y al \textit{devenir} como camino metodológico, fundamentados teóricamente en las ideas de Deleuze y Guattari. Por ello, se construye una caja de herramientas deleuziana para la educación, que consiste en cuatro moderaciones: a veces líneas de fuga, a veces creación del concepto, \textit{a veces acontecimiento y a veces acto de creación}; que, en conjunto con los principios del rizoma, permiten el análisis y la sistematización descriptiva.

La indagación permitió plantear tres líneas de conclusiones. La primera se orienta hacia el investigador-escritor desde el enfoque de la “propia vida” y en la urdimbre entre la alianza-devenir del pensamiento. La segunda refiere a las construcciones culturales diversificadas y movilizadas entre los conceptos de comunicación, performatividad, miradas, colectividad, diferencias y equiparaciones, lo cual repercute y se sistematiza desde las líneas de fuga, creación del concepto, acto creativo y acontecimientos. Y la última es la aplicación de la perspectiva metodológica con sus variables: pensamiento, escritura y lectura rizomática, entre otras; y las reflexiones frente a la posibilidad de otras construcciones metodológicas desde el sentir investigativo.

\section{Estado del conocimiento}\label{sec-estado}
La búsqueda de consensos sobre la comprensión de la tecnología en una cultura propia concede a los estudios socio-técnico-culturales \cite{levy_cibercultura._2007} una reconfiguración decisiva en las formas dominantes en la información, la comunicación, el conocimiento, la investigación, la producción y la organización. Esto permite que emerjan las cuatro tendencias investigativas acerca de las prácticas tecnológicas como componentes constitutivos de las dinámicas escolares que sustentan este trabajo.

\subsection{Subjetividades, jóvenes y tecnologías digitales}\label{sec-subjetividades}
Esta tendencia se sustenta desde las mutaciones de la subjetividad contemporánea \cite{amador_mutaciones_2010, erazo_caicedo_mediaciones_2006}, la comprensión del conjunto de tránsitos, continuidades e irrupciones en las formas de ser \cite{acosta_juventud_2012, mejia_tecnologi_2004, rueda_ortiz_cibercultura:_2008}, las expresiones de identidad y amor de los jóvenes en la era digital \cite{richardson_promposal:_2017}, la agencia de los actores sociales frente a la crisis de las instituciones y, en consecuencia, desde la potencia ―pero también el riesgo― que implica un doble proceso de tendencia tecnológica y cultural \cite{odgers_adolescent_2020, rueda_ortiz_convergencia_2009}, lo cual abre variados caminos donde se producen lenguajes, saberes y convergencias culturales \cite{munoz_culturas_2010a}.

\subsection{Escenarios de comunicación digital y jóvenes}\label{sec-escenarios}
Se plantean desde la discusión en cuanto al uso de pantalla y su dimensión más allá de ella \cite{munoz_2010c}. Se interpretan, la gestión de relaciones entre lo privado y lo público; concepciones frente a la comunicación, vida y experiencias en internet \cite{fonseca_redes_2015}; desde lo musical en MP3 y el iPod \cite{pelaez_coleccion_2010}; la mensajería de texto en el teléfono celular \cite{galindo_mensajerien_2010} y el computador; las redes sociales como Facebook y Twitter \cite{menzies_case_2017, munoz_redes_2010b}; y los videojuegos \cite{alba_pegados_2010}. Hay investigaciones sobre lo tecnocultural, la formación ciudadana y las representaciones sociales que ponen en el centro de la reflexión a los niños y jóvenes en calidad de agentes que producen otras formas comunicativas relacionadas con unas aproximaciones al aburrimiento, a la soledad y a la confianza \cite{soler_fonseca_confianza_2016}. En este sentido, asumen la postura de escuchar la “voz” y la representación en contextos educativos \cite{dahya_critical_2017} con unas preguntas de fondo: ¿Qué están haciendo los jóvenes en internet? y ¿qué ha hecho internet en la vida de los jóvenes?

\subsection{Cultura política y educativa en mediaciones tecnológicas}\label{sec-cultura}
En clave sociocultural, política y \textit{tecnopolítica educativa} \cite{rueda_ortiz_tecnocultura_2004b} se indagan fenómenos de la comunicación, tecnocultura/cibercultura y la educación, al explorar las relaciones posibles entre las transmediaciones, las subjetividades de las generaciones llamadas nativos digitales \cite{chaparro-hurtado_consumo_2013} y sus procesos de aprendizaje \cite{amador_transmediaciones_2014, munoz_culturas_2010a} en una noopolítica y experiencia de sí, donde el sujeto convierte la esfera íntima en una especie de escenario propicio para popularizar la vida, a través de la composición de recursos digitales \cite{amador_infancias_2013}. Esto produce nuevas modalidades de constitución de sujetos, tensiones por las emergencias de la sociedad contemporánea y, en particular, experiencias colectivas e individuales en interacción con los nuevos repertorios tecnológicos que configuran nuevos modos de ser, estar y actuar juntos \cite{barrios_tao_subjetividades_2015, fonseca_rueda_2012, ramirez_cabanzo_infancias_2013, rueda_tecnocultura_2004a, rueda_ortiz_tecnocultura_2004b, rueda_cultura_2007}.

\subsection{TIC y educación}\label{sec-modelo}
El uso y el consumo de la tecnología informática y cómo estos se ven reflejados en el desarrollo de los procesos de aprendizaje hacen parte de la estructura de la tendencia TIC y educación, con las que se articulan las prácticas pedagógicas \cite{carvajal_tecnocultura_2013} y algunos fenómenos en torno a la crisis de la cultura escrita en nuestro medio \cite{carvajal_barrios_jovenes_2005} desde las percepciones de los estudiantes acerca de su propio aprendizaje \cite{bagdasarov_influence_2017}. Esto revela el impacto de la tecnología de asistencia, tanto en el éxito académico de los niños con discapacidades como en los logros no académicos \cite{harper_assistive_2017}.

\section{Cibercultura y tecnología: aspectos teóricos}\label{sec-cibercultura}
Son variados los términos que se han acuñado a lo largo de la historia para referirse a los procesos y avances tecnológicos: \textit{virtualidad} \cite{levy_que_1998} \textit{cultura de la sociedad digital contemporánea} \cite{levy_educacion_1997}, \textit{cibercultura} o \textit{ciberespacio} \cite{levy_cibercultura._2007}, \textit{generación electrónica} \cite{buckingham_mas_2007}, \textit{comunicación digital interactiva} \cite{scolari_hipermediaciones._2008} y el neologismo \textit{technoculture} \cite{penley__1991}. Todos ellos se relacionan con \textit{producción cultural} \cite{baldessar_piel_2011} y con tratar de entender estas transformaciones como algo que trae cambios culturales positivos \cite{castells_sociedad_2006, eco_internet_2012, kerckhove_piel_1999, levy_cibercultura._2007}.

La tecnología como acción, instrumento, artefacto, producción y dispositivo electrónico ha provocado toda una serie de procesos de creación de objetos con sus propias capacidades para satisfacer necesidades, pero que, a la vez, generan toda una reflexión sobre el conocimiento, la organización social \cite{mejia_educaciones_2011a} y la ruptura natural, que en esencia re-crea una serie de caminos y devenires que desligan, aprecian, cambian y confortan esas otras formas de ser joven. 

Según lo anterior y para esta investigación, partimos de que la tecnología posee un carácter esencial de la realidad y del sujeto: “el carácter constitutivo de los medios [es] visto no bajo una perspectiva determinista tecnológica, sino como un espacio que exhorta a prácticas específicas, que a su vez sirven para reconfigurar nuevos tipos de sujeto” \cite[p. 4]{poster_whats_2001}. Esto implica unas prácticas tecnológicas y un cultivar de aptitudes humanas en entornos concretos, que integran un sistema cultural, diferenciados en sus materiales, símbolos y organización \cite{levy_cibercultura._2007} desde el lugar en donde buscan habitar.

La \textit{práctica tecnológica} se define, en esta indagación, desde un contenido de historicidad, un proceso de formación, una disciplina, una discursividad y una práctica en la sociedad \cite{zuluaga_garces_pedagogie_1999}, en relación con artefactos, instrumentos y dispositivos electrónicos en su mediación y en el cual involucra tanto al sujeto del saber como al saber mismo que supera la mirada instrumentalista \cite{mejia_tecnologi_2004, rios_beltran_practica_2018}.

La tecnología y el cultivo de sus prácticas requieren un lugar en donde habitar, lo que las hace estar atrapadas en diversas versiones que dependen del contexto. Están en variados procesos educativos, científicos, políticos o de producción, que las hace mantener poderes que subyacen en el sujeto, haciendo a la vez que estas prácticas se dispersen  “aceleradamente por toda la sociedad cambiando de funcionamiento los nuevos dispositivos de poder a una lógica total y constante que opera con velocidad” \cite[p. 18]{sibilia_hombre_2007}, pero también permiten explorar la política de su cotidianidad “sugiriendo adentrarse en sus formas de ‘participación estética’, en sus actos que expresan a la vez cierta alienación y cierta resistencia, cierta convivencia con el poder y cierto olfato para enfrentarlo” \cite[p. 7]{munoz_comunicacion_2007}.

En esa convivencia con el poder y con sus maneras de afrontarlo, se correlatan las tensiones entre sociedad de la información y el conocimiento y sociedad de control. La primera está conectada a los instrumentos de las TIC y, a su vez, es una apuesta por la construcción de una sociedad de conocimiento \cite{diaz_bernal_alisis_2012}, que, con el cuidado pertinente respecto a cantidades excesivas de información, supone un ejercicio de seleccionar, entender y asimilar el propio conocimiento \cite{barcelo_sociedad_1998}. La segunda está permeada entre el simulacro \cite{baudrillard_cultura_2013} y la modernidad líquida \cite{bauman_vigilancia_2013}, en un trasfondo que se caracteriza por el surgimiento de la simulación, la hiperrealidad, la confusión entre signo y sentido. Se comprende entre contradicciones de la sociedad, el escándalo moralizador y la fantasía de la construcción critica al capitalismo. Más aún, esa vigilancia se escurre entre una arquitectura social fijada, en una relación mutua entre medios y la fluidez de las relaciones personales que solo discurren en una fragilidad de vínculos \cite{bauman_vigilancia_2013}, los cuales permiten que se imponga el poder y a la vez se prueba la debilidad de unos en las \textit{relaciones de poder} \cite{foucault_redes_1982, garcia_universidad_2019}.

\subsection{Hecho educativo, hecho comunicativo, hecho tecnológico}\label{sec-hecho}
Algo es un \textit{hecho} cuando permite describir aquello que ocurre, pero que, a la vez, es un comprobante desde la percepción de los sentidos que profundiza en lo más acertado posible. Así, los hechos; educativo, tecnológico y comunicativo establecen una relación fundamental dentro de una problematización vista desde la práctica educativa. Aunque están divididos en tres, no se pueden ver desde lugares diferentes, ya que la reflexión debe tratarse desde una deconstrucción del saber educativo y desde una reconfiguración del saber desde el \textit{nosotros}. Comunicación, tecnología y educación \cite{mejia_comunicacion_2011b} convergen y producen nuevos lenguajes, nuevas formas de comunicación. En suma, construyen nuevas subjetividades y aceleran cambios sociales, políticos y económicos, en relación con la interacción de niños y jóvenes con las nuevas tecnologías. Además, configuran otras formas de experiencia y subjetivación \cite{munoz_culturas_2010a, ramirez_cabanzo_infancias_2013}. En ellas, el nivel del desarrollo humano corresponde a la tecnología. Cuando los procesos se dan desde la resignificación y se ha pasado del objeto, artefacto o instrumento tecnológico al uso y a la interpretación en la cotidianidad en maneras de lo vivido, se convierten en ejercicios de existencia y entendimiento de lo diverso y lo diferente.

\section{Perspectiva metodológica}\label{sec-perspectiva}
El investigador aprende a trasegar habitualmente con lo inesperado. Por eso, en cuanto a la forma, el sentir y la acción en este viaje, se encuentran remolinos y tormentas que dotan al investigador de la capacidad de riesgo y de la necesidad de sobrepasar lo conocido para llegar mucho más allá de las rutas trazadas. En este sentido, se plantea una propuesta rizomática como perspectiva metodológica. Comprendemos el concepto de \textit{rizoma} y sus principios como fases metodológicas. El \textit{encuentro} y el \textit{devenir} son el camino investigativo y conforman la caja de herramientas deleuziana, que permite sistematizar descriptivamente la deconstrucción-construcción de formas de comprender, de sentir y de pensar las prácticas tecnológicas. 

Hablar de \textit{rizoma}, en primera instancia, se aborda desde la esfera botánica. Tiene que ver con una planta de tallos entrelazados de tipo radicular. Entre sus características se halla una raíz que se hunde en la tierra y a la vez es una prolongación del tronco. Crece indefinidamente y crea nuevos y sanos brotes, sin obedecer a una estructura jerárquica, sino más bien a una comunicación y transformación horizontal. 

Desde una esfera filosófica, se aborda el movimiento rizomático basado en el cuestionamiento del pensamiento como aquella fuerza que en su generación se despliega como creación y se enuncia en modos concretos \cite{rangel_ecosofi:_2007}. Por ello, pensar en el rizoma es ampliar las posibilidades de la estructura en la que nos movemos. Es concebir las relaciones de diferentes formas y dar posibilidad a múltiples conceptos y aspectos \cite{fundacion_pedagogica_rayuela_proyecto_2019}. Sirve para adentrarse en todas aquellas posibilidades de las lógicas complejas y difusas que se hacen presentes con la naturalidad del devenir, los encuentros y el acontecimiento que convergen en la construcción de una autenticidad. Así, se busca que las prácticas tecnológicas sean componentes de las dinámicas escolares de la Fundación Pedagógica Rayuela, desde un ámbito educativo, que desemboca en la tercera y última instancia del significado de \textit{rizoma} en las ciencias humanas, el cual es el pretexto indudable de investigación.

En consecuencia, el rizoma convoca seis principios \cite{deleuze_mil_1980, perez_de_lama_avispa_2009}, los cuales sirven como umbrales metodológicos organizados en cuatro fases procedimentales para el análisis. La información se analiza y se encauza en una sistematización promovida desde la caja de herramientas deleuziana para la educación. 

\begin{itemize}
    \item Principios de conexión y de heterogeneidad: “cualquier punto del rizoma puede ser conectado con cualquier otro, y debe serlo” \cite[p. 13]{deleuze_mil_1980}.
    \item Principio de multiplicidad: una multiplicidad no tiene ni sujeto ni objeto, sino únicamente determinaciones, tamaños y dimensiones que no pueden aumentar sin cambiar de naturaleza. En un rizoma no hay puntos o posiciones como ocurre en una estructura, un árbol, una raíz. En un rizoma solo hay líneas.
    \item Principio de ruptura asignificante: un rizoma puede ser roto, interrumpido en cualquier parte, pero siempre recomienza según esta o aquella de sus líneas, y según otras. No hay imitación ni semejanza, sino surgimiento.
    \item Principios de cartografía y de calcomanía: El mapa no reproduce un inconsciente cerrado sobre sí mismo; lo construye. Por ende, no calca nada. Puede ser roto, conmovido, inquieto y adaptarse a distintos acoplamientos. Puede ser iniciado por un individuo, un grupo o una formación social.
\end{itemize}

Lo anterior se articula con una caja de herramientas deleuziana para la educación, que con sus atributos permite los diálogos entre las diferentes particularidades y la sistematización de ellos.

\subsection{Caja de herramientas deleuziana}\label{sec-caja}
El \textit{encuentro} y el \textit{devenir} se presentan desde el rizoma, en procesos de imaginar nuevas formas de comprender, sentir y pensar. Por ende, el primero ―además de ser la relación con el exterior, la desterritorialización, el esfuerzo para reterritorializarse y las relaciones que no dependen del uno sino del otro― es el esclarecimiento al conmoverse. Entonces, un encuentro se da cuando hay emociones que responden a una experiencia-alianza y, en efecto, conceden el privilegio de afectación frente al “hecho-acto” de experimentar con los sentidos mientras se evidencian sensibles fuerzas \cite{deleuze_dialogos._2004}. Junto al devenir ―en su camino de envolver el afuera en una relación con el exterior―, cada momento de la relación resulta en un envolver al otro ―reciprocidad―, sin imitar, sin reproducir; simplemente es devenir imperceptible, asignificante, donde se traza su ruptura, su propia línea de fuga, siendo sí mismo. 

\subsubsection{A veces líneas de fuga: relación diferente con el mundo}\label{sec-veces}
Las líneas de fuga permiten adquirir otro sentido y caminar en otra dirección, ya que hacen aparecer algo nuevo y modifican de todo en el todo; deshacen los equipamientos preestablecidos \cite{guattari_lineas_2013}. A la vez, son puntos de resistencia que requieren organización de toda una crítica incorporada a la lógica para dar otros sentidos; es decir, abrir paso al mundo de posibilidades que sencillamente orientan el problema general de las resistencias difusas. Las líneas de fuga podrían estar relacionadas con lugares, situaciones, hechos, experiencias, etc., por donde todo se escapa \cite{deleuze_que_2007}. En esta moderación dentro del camino metodológico se pretende encontrar esas líneas de fuga mutantes del devenir-joven, devenir-imperceptible, devenir-prácticas tecnológicas, entrelazadas rizomáticamente como puntos de resistencia que encuentran los jóvenes al pensar creativamente. Esto posibilita nuevas formas de resistencia. Los jóvenes, a la vez, están en un proceso de disposición crítica donde incorporan diferentes entendimientos desde sus propios sentidos y buscan un pasaje al mundo de las posibilidades.

\subsubsection{A veces creación del concepto}\label{sec-creacion}
Los conceptos tienen que ser creados. Para que esto llegue a suceder, según \textcite{deleuze_que_1993}, aparecen varios elementos para tener en cuenta: 

\begin{enumerate}
    \item Deben situarse desde un plano, un ámbito, un lugar de consistencia, es decir, fechados históricamente y localizados geográficamente. 
    \item La existencia de un problema hace que subsista desde el “acontecimiento”, en un verdadero acto de pensamiento que sobreviene desde y en corresponsabilidad a diferentes problemas: “el concepto se define por su consistencia” \cite[p. 27]{deleuze_que_1993}. 
    \item Se encuentra la subjetividad del tiempo. Somete al lenguaje a un trabajo de trasformación, a un devenir, es decir, a un compromiso de interiorización, por lo que se puede atender al tiempo como la subjetividad misma. 
    \item Movilidad e intensidad: es importante que el “concepto se sostenga a sí mismo” \cite[p. 65]{deleuze_que_1993} conllevando una búsqueda de impresiones que enuncian los afecto haciendo posibles los devenires con la naturaleza y lo no hombre.
\end{enumerate}

\subsubsection{A veces acontecimiento: “instantes”}\label{sec-acontecimento}
“El acontecimiento es un movimiento no-histórico, es un devenir no histórico, una línea de fuga que desterritorializa para reterritorializar nuevamente” \cite[p. 97]{deleuze_que_1993}. En otras palabras, se podría definir como un “instante” aquel con un vínculo esencial entre el sentido y el tiempo en el cual brota “algo” original, de algún modo una novedad que hace su irrupción en la realidad cuyo carácter primario es la contingencia \cite{esperon_acontecimiento_2014}. Esto le da al acontecimiento un sentido; mejor aún, es el acontecimiento mismo el que se dona como sentido. El acontecimiento es interiorizar en el encuentro la doble afirmación del mundo y del sentido, y es, a su vez, su radical diferenciación. Por eso, en el encuentro, “el tiempo del acontecimiento puro o del devenir enuncia velocidades y lentitudes relativas independientemente de los valores cronológicos o cronométricos que el tiempo adquiere en los otros modos” \cite[p. 267]{deleuze_mil_1980}. Así, para esta propuesta, el acontecimiento se toma como “un instante”, una suspensión del tiempo en el cual ocurre un evento que cobra, da sentido en la realidad y permite pensar en la cotidianidad y en lo que sucede con ella.

\subsubsection{A veces acto de creación: resistir es crear y crear es resistir}\label{sec-acto}
“Resistir es crear y crear es resistir” \cite{deleuze_que_2007}; este argumento al parecer se redescubre continuamente y es latente en el discurso desplegado por este filósofo. Hacia el final, lo vincula con un acto de resistencia. Sin embargo, ni la resistencia, ni la creación son actos comunes y corrientes relacionados con sujetos predeterminados, sino que están relacionados con un \textit{modus operandi} que abarca singularidades múltiples. La creación es algo solitario, pero es en nombre de la creación que se tiene algo que decir a alguien. Cuando esto sucede es porque también existe la más absoluta necesidad del creador de crear. Es la búsqueda al escaparse del control, en la medida que se convierte en un aspaviento irreductible conformado desde formas mutantes en el devenir de un proceso creativo. Por lo anterior y para este caso, el acto de creación es tomado como un “hecho” que se confronta a la homogeneidad y pretende expresar o captar alguna cosa, no idéntica: un gesto o una palabra, diferente en la repetición.

\section{Aplicación}\label{sec-aplicacion}
Se tuvieron en cuenta tres encuentros: música, arcoíris ―arte― y astronomía como parte del área de matemáticas. Tres diferentes espacios estuvieron dotados de guitarras eléctricas, bajo, batería, teclado, micrófonos, cables, televisor, computadores, bafles, cámaras fotográficas, telescopios y celulares, para el uso de los jóvenes pertenecientes a la Fundación Pedagógica Rayuela. Los encuentros oscilan entre treinta minutos y una hora.

\begin{itemize}
    \item Encuentro 1. Está dedicado a la música y se orienta hacia el ritmo, afinación, respiración, dicción, armonía entre los sonidos de los instrumentos y la vinculación con la voz, en pro de la preparación para una presentación de bandas. 
    \item Encuentro 2. Se da en dos espacios, en un aula de clase y fuera de ella. Parte de un trabajo previo dedicado a la fotografía y a la representación del cuerpo, para llevarlo a una técnica de la imagen llamada \textit{mapping} y luego esténcil. 
    \item Encuentro 3. Fue llevado a cabo en Sáchica, Boyacá. Es un campamento donde los jóvenes participan con el pretexto de ver la luna y las estrellas. Se realiza durante la noche y la madrugada. Se tienen en cuenta contenidos como ubicación de constelaciones y planetas, mitología griega y origen de la astronomía.
\end{itemize}

Para la observación de los tres encuentros, hay una matriz que tiene en cuenta los elementos planteados desde las moderaciones y los principios rizomáticos (\cref{tbl01}) para tenerlos en cuenta como puntos de análisis. Esta matriz se modifica a partir de la información recogida que alimenta los fotogramas de ocho prácticas tecnológicas. No obstante, para este caso, se van a ejemplificar dos de ellas: prácticas performativas y de escritura en la mirada y caracterización de las prácticas del “yo” y del “nosotros”.

Una vez obtenidas las características a partir de la observación, como se muestra en las \cref{tbl02} y \cref{tbl03}, se confrontan con los principios rizomáticos y se establecen las conexiones entre sus atributos, los mismos que se atribuyen, organizan y sistematizan en las moderaciones de la caja de herramientas deleuziana que muestra la \cref{tbl01}. Para que esto suceda, se debe evidenciar que los atributos compartan características entre ellos al confrontarlos entre prácticas ―es lo que se busca. Se deben encontrar todas las relaciones posibles entre ellas y generar conversaciones en torno a; a veces líneas de fuga, a veces creación del concepto, a veces acontecimiento y a veces acto de creación, mientras se hacen engranajes de indagación y de re-creación descriptiva y reflexiva de los datos recopilados. Así se obtienen textos con el propósito de darle paso a las nuevas formas de pensar, de escuchar, de ver y de sentir, en una producción de inconsciente: ser-acción. Con esto, emergen la representación de un mundo que está dado, la identificación de nuevos componentes, la creación de nuevas relaciones y territorios, la representación de lo que sí y no debe ser representado.



\renewcommand{\arraystretch}{.75}
\begin{small}
\begin{longtable}{llll}
\caption{Principios del rizoma y moderaciones de la caja de herramientas deleuziana para la educación. Elementos para el análisis.}
\label{tbl01}
\\
\toprule
\begin{tabular}[l]{@{}l@{}} Conexión y \\ Heterogeneidad \end{tabular} &
Multiplicidad &
\begin{tabular}[l]{@{}l@{}} Ruptura \\ asignificante \end{tabular} &
\begin{tabular}[l]{@{}l@{}} Cartografía y \\  calcomanía \end{tabular} \\
\midrule
Puntos de alianza & Determinaciones & Interrupción & Experimentación \\
Encuentros & Tamaños & Recomienzo & Realidades \\
Contenidos & Dimensiones & Comprensión & Dimensiones \\
 & Naturaleza & \begin{tabular}[l]{@{}l@{}} Líneas de \\ segmentariedad \end{tabular} & Mapa \\
  & Significados & Territorializaciones & Alteraciones \\
  & Correspondencia & Organización & Susceptibilidades \\
  & Procesos & Atribuciones & Modificaciones \\
  & Subjetivaciones & Significados & Dominancias \\
  & Línea abstracta & \begin{tabular}[l]{@{}l@{}} Líneas de \\ desterritorialización \end{tabular} & \\
  & Línea de fuga & \begin{tabular}[l]{@{}l@{}} Surgimiento de \\ heterogeneidades \end{tabular} &  \\
  & Desterritorialización &  \\
  & Cambios &  \\ 
  & Conexiones &  \\
 
\toprule
\begin{tabular}[l]{@{}l@{}} A veces líneas \\ de fuga \end{tabular} &
\begin{tabular}[l]{@{}l@{}} A veces creación \\ del concepto \end{tabular} &
\begin{tabular}[l]{@{}l@{}} A veces \\ acontecimiento \end{tabular} &
\begin{tabular}[l]{@{}l@{}} A veces acto \\ de creación \end{tabular} \\
\midrule
Resistencias & Ámbito & Instantes & Hecho \\
Agenciamiento & Lugar de consistencia & Suspensión del tiempo & Confrontación \\
Intensidades libres & \begin{tabular}[l]{@{}l@{}} Existencia de un \\ problema \end{tabular} & \begin{tabular}[l]{@{}l@{}} Originalidad al dar \\ sentido a “algo” \end{tabular} & Homogeneidad \\
Inconsistencias & Subjetividad del tiempo &  & Expresión \\
Roturas & Movilidad e intensidad & & No idéntico \\
Quiebres & & & \\
Explicación a la existencia & & & \\
Símbolos & & & \\
Mediaciones & & & \\
Reconocimientos & & & \\
Descripción del mundo & & & \\
\begin{tabular}[l]{@{}l@{}} Posibles encuentros \\ entre el “yo” y el otro \end{tabular} & & & \\
Visibilización & & & \\
Disfraces & & & \\
\bottomrule
\source{elaboración propia.}
\end{longtable}
\end{small}



\begin{table}[htpb]
\caption{Caracterización prácticas performativas y escritura en la mirada.}
\label{tbl02}
\begin{tabular}{p{4cm}llp{5.5cm}}
\toprule
Fotograma 1 & Encuentro & Fotograma 2 & Características \\
\midrule
\arrayrulecolor[gray]{.7}
\multirow{22}{=}{Prácticas performativas y escritura en la mirada} & \multirow{11}{*}{Música} & \multirow{3}{*}{Voz} & Voz-sonido-guía \\
 & & & Movimiento y melodía \\
 & & & Risa y armonía \\
 \cmidrule{3-4}
 & & \multirow{4}{*}{Miradas} & Miradas de complicidad \\
 & & & Miradas de permanencia \\
 & & & Miradas a la lejanía \\
 & & & Miradas alrededor \\
 \cmidrule{3-4}
 & & \multirow{4}{*}{\begin{tabular}[l]{@{}l@{}}Movimiento \\ y cuerpo\end{tabular}} & Instrumento y cuerpo \\
 & & & Movimiento y sonido \\
 & & & Señal y gesto \\
 & & & Cuerpo y Postura \\
 \cmidrule{2-4}
 & \multirow{4}{*}{Arcoíris} & \multirow{4}{*}{Imagen} & Imagen y movimiento \\
 & & & Captura de imagen \\
 & & & Imagen como emoción \\
 & & & Calcar imagen \\
 \cmidrule{2-4}
 & \multirow{6}{*}{Astronomía} & \multirow{6}{*}{Miradas} & Mirada enfocada \\
 & & & Miradas al lente --cámara fotográfica-- \\
 & & & Miradas ocultas \\
 & & & Miradas al cielo \\
 & & & Miradas de ubicación \\
 & & & Miradas al lente --telescopio-- \\
\arrayrulecolor{black}
\bottomrule
\end{tabular}
\source{Elaboración propia.}
\notes{Las características brindadas permitieron encontrar en este apartado la importancia del reconocimiento del otro en cuanto a las representaciones que superan el mundo táctil. Se identifican por lo menos doce tipos de miradas que se sustentan en la pedagogía de la mirada.}
\end{table}


\begin{table}[htpb]
\caption{Caracterización de prácticas del "yo" y del "nosotros".}
\label{tbl03}
\begin{tabular}{p{4cm}llp{5.5cm}}
\toprule
Fotograma 1 & Encuentro & Fotograma 2 & Características \\
\midrule
\arrayrulecolor[gray]{.7}
\multirow{36}{=}{Prácticas del “yo” y del “nosotros”} & \multirow{22}{*}{Música} & \multirow{2}{*}{\begin{tabular}[l]{@{}l@{}}Movimiento \\ y cuerpo\end{tabular}} & Movimiento y entendimiento \\
 & & & Apariencia-actitud-intención \\
 \cmidrule{3-4}
 & & \multirow{8}{*}{\begin{tabular}[l]{@{}l@{}}Percepción \\ y decisión\end{tabular}} & Apariencia-actitud-intención \\
 & & & Percepción de lo colectivo \\
 & & & Percepción del sujeto \\
 & & & Percepción del instrumento \\
 & & & Hasta que todos lo logren \\
 & & & Estar-espacio-tiempo \\
 & & & Técnica \\
 & & & Sentir \\
 \cmidrule{3-4}
 & & \multirow{5}{*}{Miradas} & Miradas de complicidad \\
 & & & Miradas de permanencia \\
 & & & Miradas a la lejanía \\
 & & & Mirada alrededor \\
 & & & Miradas de ubicación \\
 \cmidrule{3-4}
 & & \multirow{7}{*}{Voz y liderazgo} & Permanencia y función \\
 & & & Diálogo con instrumento \\
 & & & Paciencia y apoyo \\
 & & & Trabajo en equipo \\
 & & & Estar-espacio-tiempo \\
 & & & Diálogo con instrumento \\
 & & & Voz-sonido-guía \\
 \cmidrule{2-4}
 & \multirow{14}{*}{Arcoíris} & \multirow{5}{*}{\begin{tabular}[l]{@{}l@{}}Percepción y \\ representación\end{tabular}} & Representación de un sujeto \\
 & & & Percepción visual \\
 & & & Percepción social \\
 & & & Percepción del objeto \\
 & & & Apariencia-actitud-intención \\
 \cmidrule{3-4}
 & & \multirow{3}{*}{Imagen} & Perfección de la imagen\\
 & & & Captura de imagen \\
 & & & Técnica \\
 \cmidrule{3-4}
 & & \multirow{6}{*}{\begin{tabular}[l]{@{}l@{}}Realidades \\ y mundos\end{tabular}} & Identidades \\
 & & & Sentir \\
 & & & Apoyo-colectividad-comprensión-organización \\
 & & & Individualidad \\
 & & & Apariencia-Actitud-Intención \\
 & & & Representación de realidades \\
\arrayrulecolor{black}
\bottomrule
\end{tabular}
\source{Elaboración propia.}
\end{table}



\section{Resultados}\label{sec-resultados}
Ante la pregunta “¿Cómo los jóvenes construyen y resignifican culturalmente sus prácticas tecnológicas desde las dinámicas escolares?”, se tiene como resultado: 

La caracterización de ocho prácticas tecnológicas diferenciadas por atributos y elementos propios permitió reconocer nociones y formas reestructuradas desde las percepciones e interpretaciones de posibilidades en conexiones, teniendo en cuenta caminos que asumen nuevos procesos investigativos. Estos procesos se vinculan, primero, con \textit{las prácticas comunicativas} movilizadas en todas las rupturas asignificantes y conexiones, ya que se movilizan junto con las otras siete prácticas, con base en las relaciones entre cuerpo, movimiento, imagen y sonido y en las relaciones con lenguajes que evidencian las interacciones no solamente verbales de los jóvenes, sino también procesos de varias instancias entre el contexto, las intenciones, las realidades, los deseos y el conocimiento, traducidos en el escuchar, risas, miradas, sonidos, gestos y señales comunicativas entre instrumento, artefacto o dispositivo electrónico con el sujeto. 

En segundo lugar, aparecen \textit{las prácticas performativas y escritura en la mirada} con un objeto de estudio claro entre acciones y actos desarrollados en la vida cotidiana, las cuales se cuestionan en la acción misma y se defienden desde el movimiento corporal, en respuesta a la guía del sonido, las señales y los gestos, la emoción y las imágenes, para alcanzar formas de accionar frente a lo que se dice. Sin embargo, uno de los accionares más representativos en esta práctica tecnológica se encuentra en la mirada. Se evidencian por lo menos doce tipos de miradas, que se describen desde la escritura de la intención y la pedagogía de la mirada. Esto da como resultado, en el ámbito cultural, un reconocimiento del otro, en donde la corporeidad queda suspendida o pareciera que diera paso a representaciones que superan el mundo táctil. Queda, así, un \textit{devenir-mirada} que dota al sujeto de resistencia y de esperanza en vínculos tanto perceptibles como imperceptibles, colmados de responsabilidades y orden en algún tipo de caos.

Las \textit{prácticas del “yo” y “nosotros”} con énfasis en la diferencia, como resultado, crean espacios de identidad ―o por lo menos de semejanza―, ya que ocupan lugares de coexistencia y de prácticas compartidas que resuelven variaciones en la cultura al hacerse constituyentes de esta. En esta construcción se diversifican mensajes entre el accionar colectivo e individual, discursivo y protestante, enmarcados en limites alimentados por tres elementos fundamentales para esta práctica: \textit{estar, espacio y tiempo}. Los cuales corresponsabilizan conversiones y clasificaciones intencionadas en la representación y transformación de lo existente en cuanto a realidad-sentidos de la individualidad que produce colectividad.

Las \textit{prácticas del control}, en una apuesta por reconocer las relaciones entre autogestiones desde sí mismo, el otro y los otros, es el conocimiento de límites en diferentes ámbitos de la vida y de funciones para llegar a definir normas o controles que se convierten en una necesidad de representaciones, sobre todo desde los instrumentos, las herramientas y los dispositivos electrónicos. Las disposiciones de controlar cuerpo y acciones son secundadas por la coexistencia normalizada y la necesidad de equiparación de oportunidades, lenguaje, identidad; en sí, una concepción de conversaciones que configuran nuevas formas deliberadas y complejas de actuar, repeliendo los diferente \textit{controles} y \textit{controladores}.

En quinto lugar, están las \textit{prácticas del caos}, en respuesta a los hechos de construcción cultural desde un \textit{caos} en el lenguaje, las miradas, el sentir, la emoción, la concentración y la disposición, lo cual repercute en el aprendizaje y en la enseñanza de relaciones diferentes y enredadas. Esto resalta el valor del origen en la causalidad y, a la vez, la aclaración sobre la necesidad del arte para poder aprehender quiénes somos, de dónde venimos y hacia dónde vamos. En esta práctica, las variaciones y los efectos se determinan desde una producción cultural individual y desde la inmediatez, para salir de lo previsible y lo determinado. Lo anterior y ejemplificando, puede transitar entre la risa como mejora o complicación, palabras sutiles, ofensivas o en apariencia que no están claras ya que se relacionan con actitudes desalineadas, intenciones fortuitas, emociones que sufren y se alegran a la vez, sentires apesadumbrado y sistemas de fuerza que se trasladan entre el control de sí mismo, de otro y de lo otro de las \textit{prácticas del control}.

En un sexto lugar, las \textit{prácticas de lo opuesto} se identifican en una gama de contrastes, que desembocan culturalmente en una concreción cultural de díadas: palabra-acción, risa-equivocación, representación de lo existente, transformación de ello y la opción de convertirlo en otras realidades. Otras de las características son las acciones en cuanto a palabras y movimientos: cómo estas dos formas se pueden encontrar, pero, a su vez, se oponen bajo ciertas circunstancias e intenciones. Entonces, el sujeto se encuentra en continua oposición de sí mismo y de los otros; no como ofensa, sino como parte de su particularidad.

En séptima instancia, se reconocen las \textit{prácticas del sonido y el movimiento}, que recogen las prácticas ya mencionadas, pero requieren un espacio propio porque dan sentido al silencio, a lo no musical, a las aceleraciones y velocidades en relación con diversos temas que competen al ser humano en una riqueza respecto al territorio, los modos de transporte de comunicaciones y poder, en relación con el tiempo y la quietud. Esto da un contexto de inmediatez y velocidad, pero también de quietud corporal que moldea otras formas de estar en una cultura. 

Por último, en octavo lugar, se encuentran las \textit{prácticas de la imagen y representación}, las cuales se convierten en códigos del cuerpo y de las apariencias físicas arraigadas en la cultura dentro del retrato. Requieren análisis del lenguaje corporal dentro de diversas realidades y sirven como estudio detallado para mostrar en general el proceso cultural en pro de interpretaciones que muestran la experiencia que se escribe desde las ideologías. La representación es poder y el poder es representación en marcos que no se controlan, bajo la apariencia de estar aprehendiendo los conceptos o las ideas, en sí mismas, como si fueran esencias o sustancias que es necesario comprender desde diferentes perspectivas en presencia de la intención.	

Lo anterior se unifica en las formas de construcciones culturales en relación con los jóvenes, sus prácticas tecnológicas y las dinámicas escolares que trascienden entre las escrituras que se encuentran en las ya mencionadas ocho practicas tecnológicas y en las construcciones culturales diversificadas entre la naturaleza como tecnología y viceversa. Se convierten en líneas de fuga y se necesitan la una a la otra como hechos para explicar, definir y ser espacio de encuentro y diferencia y tiempo de reconocimiento entre sí.

\textit{A veces creación de conceptos} se presenta en cuatro situaciones en circunstancias propias; el susurro como determinación, rostro invisible como representación, la apariencia, actitud e interés crean un concepto e imagen proyectada en el tránsito de diversas figuras personales.

Se anuncian nueve acontecimientos en atribución a las prácticas tecnológicas de los jóvenes en la dinámica de los encuentros:

\begin{itemize}
    \item Acontecimiento “sentidos y conexión tecnológica”.
    \item Acontecimiento “risa”.
    \item Acontecimiento “tecnología mi propia guía”.
    \item Acontecimiento “extensión del cuerpo”
    \item Acontecimiento arte, música, matemáticas y física se fugan entre ellas.
    \item Acontecimiento “instante” gesto y señal.	
    \item Acontecimiento mirada al cielo y otras tantas en su propio instante.
    \item Acontecimiento “el avatar” de las emociones.
    \item Acontecimiento “hasta que todos lo logren”
\end{itemize}

\textit{A veces acto de creación} muestra acciones no convencionales y actos de resistencia que viajan entre la creación de espacios únicos que logran convertirse en experiencia desde la voz, el sonido y el movimiento para unirse a la identidad del sujeto en una creación propia de posibilidad. Estos convergen con la quietud y un individualismo enganchado con amigos y enemigos: felicidad, sexualidad, juventud, autenticidad, rechazo; normas que responden a revoluciones de los valores y subjetividades permeadas por democracias liberales que no se escapan a las dinámicas escolares. Allí está el reflejo de cada una de ellas. Se encuentran en movilidad permanente entre lo polémico, lo polisémico y lo globalizado que puede llegar a ser la identidad de un sujeto.

\section{En líneas de conclusión}\label{sec-conclusao}
El ejercicio investigativo permitió plantear tres tipos de conclusiones. El primero tiene que ver con el investigador-escritor, con quien se trasciende entre el ejercicio de la investigación y el ser, el cual precisa de requerimientos de organización en toda una crítica que se incorpora en la lógica desde los sentidos de los observados y el observador, con lo cual abre paso a posibilidades que se topan con líneas de fuga indeterminadas que con su irreverencia y comprensión perseveran en el argumento para salir del prototipo, donde no se escatima en recursos para el desorden y la confusión absoluta permitiendo pensar en lo originario y a su vez suponiendo una ordenación propia.

La segunda conclusión tiene que ver con las prácticas tecnológicas cotidianas de los jóvenes, las cuales se convierten en parte de la producción cultural de la sociedad, cuando se encuentran o construyen sus propias líneas de fuga y generan acontecimientos que proponen dinámicas como individuos en cada espacio. Generan cultura cuando no hay casualidad, sino que en las acciones encuentran una causa que reproduce un evento o hecho cultural que tiene un origen, un proyecto, un hilo conductor, un creador, pero, sobre todo, un tipo de compromiso con la vida. Si bien es cierto que los jóvenes se mueven en las dinámicas permanentes de la experiencia, en un constante construir y deconstruir sumergido en el compromiso con el mundo y la vida ―reflexionan sobre el movimiento, discurso, agenciamientos del tejido social, posturas críticas, funciones y sobre la construcción que se enmarca fundamentalmente en el vivir―, no se puede dejar de lado la incertidumbre e intranquilidad que sienten al estar expuestos y sin reconocimiento en muchas de sus formas, en un pensamiento alejado de la credulidad de hacer cultura. Son conscientes de diversas realidades en las que participan, ya sea activa o pasivamente.

La tercera línea de conclusiones se refiere a la aplicación de la perspectiva metodológica con sus variables, donde la posibilidad de hallarse en el concepto se sitúa en un enfoque de la “propia vida”. Por ende, la perspectiva metodológica rizomática permite una construcción de sí mismo en un continuo desplegarse desde la afectación mutua. Plantea reflexiones y caminos a otras construcciones desde el sentir investigativo para pasar a un recorrido teórico desde y en el rizoma, con sus principios y posibles formas de escribir y leer un rizoma en un devenir del pensamiento, ante la incidencia de plantear otros problemas en un llamado permanente de crear y recrear nuevos conceptos. Para lo anterior, la caja de herramientas deleuziana para la educación se convierte en un ejercicio práctico con sus particularidades de análisis que se adentra en la subjetividad y se convierte en algo no universal, que requiere una puesta en práctica en otros ejercicios investigativos para ir aclarando requerimiento y su uso. 

%Italicos

\printbibliography\label{sec-bib}

\end{document}
