% !TEX TS-program = XeLaTeX
% use the following command: 
% all document files must be coded in UTF-8
\documentclass{textolivre}
% for anonymous submission
%\documentclass[anonymous]{textolivre}
% to create HTML use 
%\documentclass{textolivre-html}
% See more information on the repository: https://github.com/leolca/textolivre

% Metadata
\begin{filecontents*}[overwrite]{article.xmpdata}
    \Title{Un acercamiento conceptual entre tres tipos de alfabetización: informática, tecnológica e informacional}
    \Author{Julián Rodríguez López \sep Maricela López Ornelas \sep Katiuska Fernández Morales \sep Javier Organista Sandoval}
    \Language{es}
    \Keywords{Alfabetización tecnológica \sep Alfabetización informática \sep Alfabetización informacional \sep Tecnologías de la información y la comunicación \sep Educación}
    \Journaltitle{Texto Livre}
    \Journalnumber{1983-3652}
    \Volume{14}
    \Issue{1}
    \Firstpage{1}
    \Lastpage{16}
    \Doi{10.35699/1983-3652.2021.29513}

    \setRGBcolorprofile{sRGB_IEC61966-2-1_black_scaled.icc}
            {sRGB_IEC61966-2-1_black_scaled}
            {sRGB IEC61966 v2.1 with black scaling}
            {http://www.color.org}
\end{filecontents*}

\journalname{Texto Livre: Linguagem e Tecnologia}
\thevolume{14}
\thenumber{1}
\theyear{2021}
\receiveddate{\DTMdisplaydate{2021}{2}{18}{-1}} % YYYY MM DD
\accepteddate{\DTMdisplaydate{2021}{4}{10}{-1}}
\publisheddate{\today}
% Corresponding author
\corrauthor{Julián Rodríguez López}
% DOI
\articledoi{10.35699/1983-3652.2021.29513}
% Abbreviated author list for the running footer
\runningauthor{Rodríguez López et al.}
\editorname{Hugo Heredia Ponce}

\title{Un acercamiento conceptual entre tres tipos de alfabetización: informática, tecnológica e informacional}
\othertitle{Uma abordagem conceitual de três tipos de letramento: informática, tecnológica e informacional}
\othertitle{A concept approach among three types of literacy: computer literacy, technological literacy and information literacy}
% if there is a third language title, add here:
%\othertitle{Artikelvorlage zur Einreichung beim Texto Livre Journal}

\author[1]{Julián Rodríguez López \orcid{0000-0001-5926-4519} \thanks{Email: \url{rodriguez.julian@uabc.edu.mx}}}
\author[1]{Maricela López Ornelas \orcid{0000-0002-4215-5591} \thanks{Email: \url{ornelas@uabc.edu.mx}}}
\author[1]{Katiuska Fernández Morales \orcid{0000-0002-6525-2298} \thanks{Email: \url{katiuska.fernandez@uabc.edu.mx}}}
\author[1]{Javier Organista Sandoval \orcid{0000-0001-8101-5084} \thanks{Email: \url{javor@uabc.edu.mx}}}
\affil[1]{Universidad Autónoma de Baja California, México.}

\addbibresource{article.bib}
% use biber instead of bibtex
% $ biber tl-article-template

% set language of the article
\setdefaultlanguage{spanish}
\gappto\captionsspanish{\renewcommand{\tablename}{Tabla}} % use 'Tabla' instead of 'Cuadro'
\AfterEndPreamble{\crefname{table}{tabla}{tablas}}
\setotherlanguage{portuguese}
\setotherlanguage{english}

\usepackage{multirow}
\usepackage{colortbl}

\begin{document}
%\crefname{table}{tabla}{tablas}
\maketitle

\begin{polyabstract}
\begin{abstract}
Ante el nacimiento de nuevas tendencias tecnológicas y el mejoramiento de las existentes, conceptos como el de alfabetización han evolucionado en función de las necesidades socioculturales que se derivan de las Tecnologías de la Información y la Comunicación (TIC). Precisamente estos cambios inevitables generan la ambigüedad y polisemia de conceptos para diferenciar y establecer dónde inicia y termina una alfabetización. El trabajo tiene como objetivo describir los límites conceptuales y características que existen entre la alfabetización tecnológica, informática e informacional. El proceso metodológico se auxilia del mapeo sistemático de la literatura, el cual identifica, describe y caracteriza las dimensiones de los tres tipos de alfabetización, con búsquedas en bases de datos de acceso abierto –Google Académico–, y de suscripción –ACM, EBSCO, Elsevier, Scopus y Web of Science. Los resultados muestran que, si bien existe una diferencia entre los tipos de alfabetización, estos requieren un orden en las habilidades y conocimientos para su aplicación, donde la alfabetización tecnológica representa conocimientos sobre hardware, la informática equivale al uso del software, mientras que la informacional resulta más compleja, ya que refiere a la apropiación y aplicación de las TIC, específicamente en la búsqueda, evaluación y utilización ética de la información. Se concluye que los límites entre las alfabetizaciones pueden clasificarse de acuerdo con el tipo de habilidades y conocimientos de cada individuo; sin embargo, también se identifica que un factor de relevancia está estrechamente relacionado con el contexto sociocultural educativo, en el cual se define la alfabetización.

\keywords{Alfabetización tecnológica \sep Alfabetización informática \sep Alfabetización informacional \sep Tecnologías de la información y la comunicación \sep Educación}
\end{abstract}

\begin{portuguese}
\begin{abstract}
Diante do surgimento de novas tendências tecnológicas e do aprimoramento das já existentes, conceitos como o letramento têm evoluído a partir das necessidades socioculturais derivadas das Tecnologias da Informação e Comunicação (TIC). Precisamente essas mudanças inevitáveis geram ambiguidade e polissemia de conceitos para diferenciar e estabelecer onde começa e termina o letramento. O trabalho tem como objetivo descrever os limites e características conceituais existentes entre o letramento tecnológico, informático e informacional. O processo metodológico é auxiliado pelo mapeamento sistemático da literatura, que identifica, descreve e caracteriza as dimensões dos três tipos de letramento, com buscas em bancos de dados de acesso aberto – Google Acadêmico –, e assinatura – ACM, EBSCO, Elsevier, Scopus e Web of Science. Os resultados mostram que, embora haja uma diferença entre os tipos de letramento, eles exigem uma ordenação de habilidades e conhecimentos para sua aplicação, em que o letramento tecnológico representa o conhecimento sobre hardware, a computação equivale ao uso de software, enquanto o letramento informacional é mais complexo, pois se refere à apropriação e aplicação das TIC, especificamente na busca, avaliação e uso ético da informação. Conclui-se que os limites entre letramentos podem ser classificados de acordo com o tipo de habilidades e conhecimentos de cada indivíduo; no entanto, também se identifica que um fator relevante está intimamente relacionado ao contexto sociocultural educacional, no qual se define o letramento. 

\keywords{Letramento tecnológico \sep Letramento informático \sep Letramento informacional \sep Tecnologias da Informação e Comunicação \sep Educação}
\end{abstract}
\end{portuguese}

\begin{english}
\begin{abstract}
In light of the birth of new technological tendencies and the improvement of the existing ones, concepts such as Literacy have evolved according to sociocultural needs that derive from the Information and Communication Technologies (ICT). In point of fact, these unavoidable changes generate ambiguity and polysemy of concepts to differentiate and establish where Literacy starts and where it ends. The objective of this work is to describe the concept limits and the common characteristics that exist among Technological Literacy, Computer Literacy and Information Literacy. The systematic mapping of scientific literature aids the methodological process, which identifies, describes and characterizes the dimensions of the three types of Literacy, with searches in Open Access databases —Google Scholar—, and subscription sources —ACM, EBSCO, Elsevier, Scopus and Web of Science. The results show that, while there is a difference among the types of Literacy, these types require an order in the skills and knowledge to their implementation: whereas Technological Literacy represents knowledge about hardware; Computer Literacy is equivalent to the use of software and Information Literacy is more complex, it refers to the adoption and appliance of ICT, particularly in searching for and assessing information and its ethical usage. We can conclude that the limits among the types of Literacy could be classified in consonance with the types of skills and knowledge of each individual; nonetheless, a factor of relevance that is closely related to the educational sociocultural context is identified, a factor in which Literacy is defined.

\keywords{Technological literacy \sep Computer literacy \sep Information literacy \sep Information and communication technologies \sep Education.}
\end{abstract}
\end{english}

% if there is another abstract, insert it here using the same scheme
\end{polyabstract}


\section{Introducción}\label{sec-intro}
Ante el nacimiento de nuevas tendencias tecnológicas y el mejoramiento de las existentes, conceptos como el de alfabetización han evolucionado en función de las necesidades contextuales y culturales que se derivan de las Tecnologías de la Información y la Comunicación (TIC). Precisamente estos cambios inevitables generan la ambigüedad y polisemia de conceptos para diferenciar y establecer dónde inicia y termina una alfabetización.

La creciente oleada de tecnología conlleva cambios significativos en el modo de operar las cosas; nuevas interfaces, software y aplicaciones obligan a las personas a desarrollar habilidades ex profeso para integrarlas en la vida académica. Debido a los recientes contextos tecnológicos en que nos desempeñamos y hacia los cuales está dirigida la capacitación actual, \textcite{avello__martinez_evolucion_2015} sugieren que no importa la alfabetización por muy básica que esta resulte, tiene que ser ‘digital y multimediática’. Es por eso que a las alfabetizaciones –donde su concepción original se define como la enseñanza de leer y escribir– se le agrega el componente tecnológico. 

La alfabetización es un proceso multimodal que se ajusta a cada sujeto y contexto. En ese sentido, \textcite[p. 151]{ruiz_requies_formar_2010}, ejemplificaron que “la inclusión en la educación superior (ES) de las alfabetizaciones no es lineal, acumulativa u homogénea”. Esto quiere decir que la alfabetización no es un proceso en cascada ni jerárquico, sino horizontal a cualquier nivel; tampoco es un procedimiento igualitario en cada una de las personas, ya que está estrechamente relacionado con el contexto sociocultural educativo. 

Al ser un tópico de trascendencia global en el ámbito educativo, el proceso de la alfabetización –como objeto de estudio y realidad social– ha sido investigado y promovido en diversos escenarios por instituciones internacionales, como la Organización de las Naciones Unidas para la Educación, la Ciencia y la Cultura (UNESCO), la Organización de Estados Iberoamericanos (OEI), la Organización de las Naciones Unidas (ONU) y el Centro de Cooperación Regional para la Educación de Adultos en América Latina y el Caribe (CREFAL). En la siguiente tabla, se presenta un resumen de las principales actividades que la UNESCO ha realizado en favor de la alfabetización en un lapso de cinco décadas.

\newpage
\begin{small}
\begin{longtable}{
    >{\raggedright\arraybackslash}
    p{0.36\textwidth}
    p{0.57\textwidth}
    }
\caption{Principales actividades de la UNESCO a favor de la alfabetización, 1965-2020.}
\label{tbl01}
\\
\toprule
Eventos a favor de la alfabetización & Principal acuerdo \\ 
\midrule
\arrayrulecolor[gray]{.7}
Congreso Mundial de Ministros de la Educación sobre la Erradicación del Analfabetismo (1965), Teherán, República Islámica de Irán. &
Emplear plenamente los medios de información disponibles para la difusión del nuevo principio de alfabetización de los adultos \cite{fernandez_1966}. \\
\midrule
XIV Conferencia General de la UNESCO (1966), París, Francia. &
Otorgar la correspondida prioridad a la  alfabetización funcional relacionada con la formación profesional y técnica, emplearla en los sectores esenciales para el desarrollo; integrarla en los planes nacionales de desarrollo; apoyar las disposiciones apropiadas para promover y apresurar la acción en favor de la  alfabetización \cite{unesco_actas_1969}. \\
\midrule
Simposio Internacional de  Alfabetización (1975), Persépolis, República Islámica de Irán. &
La alfabetización debe concernir al beneficio mundial y comprende la responsabilidad de superar las diferencias ideológicas, geográficas o económicas \cite{unesco_consejo_1987}. \\
\midrule
5ᵃ Conferencia Internacional de Educación de las Personas Adultas (Confintea V) (1977), Hamburgo, Alemania. &
Es necesario establecer un foro, así como estrategias de “consulta, que avalen la  aplicación y seguimiento de las recomendaciones y resultados sobre educación de adultos emitidas por la UNESCO” y sus estados miembros \cite[p. 44]{unesco_quinta_1997}. \\
\midrule
Conferencia Mundial sobre Educación para Todos (1990), Jomtien, Tailandia. &
“Se aprueba la Declaración Mundial sobre Educación para Todos, centralizada en las necesidades básicas de aprendizaje”. Se instituye que la educación es un derecho  humano ineludible a nivel internacional, por lo que se exhorta a incrementar los esfuerzos para mejorarla \cite[p. 97]{haggis_conferencia_1992} \\
\midrule
Decenio de las Naciones Unidas de la Alfabetización (2003 a 2012), diferentes países. &
Se establecen tres objetivos clave, expresados como desafíos: 1. Promover un compromiso más firme en favor de la alfabetización; 2. Reforzar la ejecución de programas eficaces de alfabetización; 3. Obtener nuevos recursos para la alfabetización \cite{unesco_decenio_2009}. \\
\midrule
6ᵃ Conferencia Internacional de la UNESCO sobre Educación de Adultos (Confintea VI) (2009--2010), Belén, Brasil. &
Se afirma que “El derecho a la alfabetización es inherente al derecho a la educación [\ldots] es un medio esencial de capacitación de las personas para afrontar los cambiantes problemas y complejidades de la vida, la cultura, la  economía y la sociedad” \cite[p. 28]{unesco_marco_2010}. \\
\midrule
Declaración de Incheon adoptada en el Foro Mundial sobre la Educación (2015), Incheon, República de Corea. & 
“Promover el uso de las TIC, en especial las tecnologías móviles, en los programas de alfabetización y aritmética” \cite[p. 22]{unesco_educacion_2016}. \newline
Conjuntamente, en el Marco de indicadores temáticos, cuyo objetivo “Garantizar una educación inclusiva y equitativa de calidad y promover oportunidades de aprendizaje permanente para todos, en la Meta 4.4., menciona: De aquí a 2030, aumentar considerablemente el número de jóvenes y adultos que tienen las competencias necesarias, en particular técnicas y profesionales, para acceder al empleo, el trabajo decente y el emprendimiento, se establece el término de alfabetización, en el indicador 16: \newline
\begin{enumerate}
\item Porcentaje de jóvenes/adultos que han alcanzado al menos un nivel mínimo de competencia en el ámbito de la alfabetización digital. 
\item Porcentaje de individuos que poseen competencias en materia de las TIC por tipo de competencia” \cite[p. V]{unesco_educacion_2016}
\end{enumerate}  \\
\midrule
Declaración de Seúl sobre la Alfabetización Mediática e Informacional Para Todos y Por Todos (2019). &
Se destacó que “la alfabetización mediática e informacional es una competencia básica para hacer frente a la desinfodemia y que también contribuye al acceso a la información, la libertad de expresión, la protección de la privacidad, la prevención del extremismo violento, la promoción de la seguridad digital y la lucha contra la incitación al odio y la desigualdad” \cite[párr.~4]{unesco_declaracion_2020}.
\\
\arrayrulecolor{black}
\bottomrule
\source{elaboración propia}
\end{longtable}
\end{small}

Lo descrito en la \cref{tbl01} sustenta dos puntos sobre el objeto de estudio: a) el fundamento y trascendencia de la alfabetización por parte de un organismo tan relevante en materia de educación mundial, como la UNESCO, y, b) la evolución del concepto de alfabetización funcional al de alfabetización digital, mediática e informacional.

\section{La alfabetización y el significado del entorno contextual}\label{sec-conceito}
Al no ser la alfabetización un proceso igualitario, es decir, que no se da en la misma medida en cada individuo, genera disparidad entre las personas adultas y jóvenes. Esta desproporción podría ser concebida por su aproximación a la tecnología dentro de cada ambiente, como lo establecieron \textcite[p. 451]{avello__martinez_evolucion_2015}:

\begin{quote}
    Los cambios acelerados en las TIC requieren una preparación urgente fundamentalmente para las personas con más edad que no nacieron ni crecieron con las nuevas TIC, conocidos como “inmigrantes digitales” y para los llamados “nativos digitales” que, si bien es cierto que estos últimos han integrado las TIC con rapidez y facilidad a su vida diaria, muchas veces carecen de un análisis crítico para hacer uso de ellas.
\end{quote}

En congruencia, el contexto cambiante donde se desarrollan las personas y su aproximación a la tecnología delimitan la alfabetización que ellas tengan hacia esta última. Dichas evoluciones han llevado a la sociedad de la información y el conocimiento a caracterizarse, a no tener un perfil en particular, ni homogéneo ni con la misma igualdad de condiciones y acceso a la tecnología, lo cual concibe la llamada brecha digital –disparidad en el acceso, conocimiento y uso de las TIC– \cite{uribe_tirado_acceso_2004}. Como señala \textcite[p. 57]{area_moreira_alfabetizacion_2014}, “las nuevas tecnologías son un nuevo factor de desigualdad social debido a que las mismas están empezando a provocar una mayor separación y distancia cultural entre aquellos sectores de la población que tienen acceso a las mismas y quienes no”. 

Ante el nacimiento de nuevas tendencias tecnológicas y el mejoramiento de las que ya existen, el concepto de alfabetización se ha modificado en función de los requerimientos que las TIC plantean. En este sentido, la gestión de las tecnologías supone reconocer la necesidad de nuevas competencias y prácticas en los sujetos o, dicho de otra manera, ampliar el concepto de alfabetización \cite{ruiz_requies_formar_2010}.

Por lo tanto, ante las demandas de adaptación que presupone la significación de la alfabetización según el contexto, \textcite[p.451]{avello__martinez_evolucion_2015} concuerdan en que “la definición de alfabetización va cambiando, con el propósito de abarcar el conjunto de competencias y habilidades que permite a las personas expresarse, explorar y cuestionar la información en el contexto tecnológico”. De esta manera, es natural que dicha significación se transforme con el paso del tiempo, con el fin de adaptarse a las exigencias del ambiente y las propias necesidades de la estructura contextual. Precisamente, derivado de los cambios que ha sufrido el término según su condición, en paralelo crece la ambigüedad para diferenciar y establecer dónde inicia una alfabetización y dónde acaba la otra, además de las propias enunciaciones donde se delimitan las competencias y habilidades que permite el uso de la tecnología.

En la \cref{tbl02}, se lista, bajo la perspectiva de \textcite{bawden_revision_2002, rodriguez_illera_alfabetizaciones_2004, uribe_tirado_interrelaciones_2009, epperson_computer_2010, ingerman_technological_2011, avello__martinez_evolucion_2015}, las primeras alfabetizaciones relacionadas con las TIC, así como los autores que aportaron una definición. Si bien las posturas de los autores mencionados en esta tabla no necesariamente coinciden al delimitar cada alfabetización, es importante reconocer que sus contribuciones han sido un referente teórico conceptual al momento de bosquejar trabajos de esta naturaleza. 

\begin{small}
\begin{longtable}{
    >{\raggedright\arraybackslash}
    p{0.20\textwidth}
    p{0.74\textwidth}
    }
\caption{Diversos tipos de alfabetizaciones en el ámbito de las tecnologías.}
\label{tbl02}
\\
\toprule
Tipo de alfabetización &
Autores \\ 
\midrule
\arrayrulecolor[gray]{.7}
Alfabetización informacional &
Johnson (1985); Kulthau (1987); American Library Association (ALA-ACRL, 1989); Coons (1989); Jackson (1989); Olsen y Coons (1989); Dess (1991); Ochs et al. (1991); Behrens (1994); Doyle (1994); McClure (1994); Bruce (1997, 1999); Carbo, (1997); Mutch (1997); Shapiro y Hughes Snavely y Cooper (1997);  OCDE (1997); Bundy (2000); Joint Information Services Committee (2002);  Olsen, (2002); Breivik (2003); UNESCO (2003); US National Commission on Libraries and Information Science and the National Forum on Information Literacy (2003); Chatered Institute of Library and Information Professionals (2004); Council of Australian University Librarians (2004); Gómez (2004); Pinto (2004). \\
\midrule
Alfabetización informática &
Departamento de Educación (1983); Hunter (1983); Baron (1984); Haigh (1985); Husen y Postlethwaite (1985); Johnson (1985); Sloan y Halaris (1985); Cohen (1987); Peterson (1987); Sellars (1988); Tuckett (1989); Arp (1990); Ovens (1991); Kanter (1992); Royal Society of Arts (1993); Behrens (1994); McClure (1994); Ueberroth (1994); Bhola (1997); Lowell (1997); Morgan (1998); Oxbrow (1998); Hoffman y Blake (2003); Bartholomew (2004); Hoffman et al. (2005); Vance (2005); Foster y Dannelly (2006); Gupta (2006); Stiller y LeBlanc (2006); Hoffman y Vance (2008); Lio y Pope (2008); Banerjee (2009); Huang y Briggs (2009).  \\
\midrule
Alfabetización tecnológica &
TfAAP e ITEA (1996); ITEA (2000, 2002, 2007); Pearson et al. (2002); Rose (2007); Collier-Reed (2008). \\
\midrule
Alfabetización bibliotecaria &
Lubans (1980); Johnson (1986); Fatzer (1987); Breivik (1989); Dusenbury (1989); Whitaker (1991); Rudolph, Smith y Argall (1996); Snavely and Cooper (1997). \\ 
\midrule
Alfabetización digital, alfabetización en información digital &
Barton (1994); Gilster (1997); Barton y Hamilton eds. (1999); Library and Information Science Abstracts (LISA) en Social Scisearch, entre 1980 y 1998; Cassany (2000); Gutiérrez (2003); Kress (2003); Azinian (2006); UNESCO (2006); Arrieta y Montes (2011); Area, Gutiérrez y Vidal (2012); Avello y Martín (2012). \\
\midrule
Alfabetización en redes, en Internet, alfabetización multimedia, multialfabetización &
McClure (1994); Barclay (1995); Martin (1997); Nicholas y Williams (1998);  Larsson (2000); The New London Group (2000); Kress y van Leeuwen (2001). \\
\arrayrulecolor{black}
\bottomrule
\source{elaboración propia.}
\end{longtable}
\end{small}


Por consiguiente, es necesario realizar una búsqueda a través del Mapeo Sistemático de la Literatura para responder a la siguiente pregunta general de investigación, ¿cuáles son las diferencias conceptuales, constitutivas y operacionales entre los diferentes tipos de alfabetización? a la par de la pregunta general, se desprenden las siguientes preguntas específicas (P) (\cref{tbl03}):

\begin{table}[htbp]
\caption{Preguntas de investigación.}
\label{tbl03}
\centering
\begin{tabular}{lp{12cm}}
\toprule
\textbf{P 1} & ¿Cuáles son los límites conceptuales entre los diferentes tipos de alfabetización? \\ 
\textbf{P 2} & ¿Cuál es la definición constitutiva y operacional de la alfabetización informática, tecnológica e informacional? \\ 
\textbf{P 3} & ¿Qué aportaciones teórico-metodológicas proponen los diferentes autores dentro del Mapeo Sistemático de la Literatura? \\ 
\textbf{P 4} & ¿Cuáles son los criterios e indicadores que definen según los autores sobre los tres tipos de alfabetizaciones? \\
\bottomrule
\end{tabular}
\source{elaboración propia.}
\end{table}

\section{Método}\label{sec-metodo}
De acuerdo con \textcite{garcia-penalvo_revisiones_2019}, entre las revisiones sistemáticas más utilizadas, se encuentran la revisión sistemática de la literatura (RSL) y el mapeo sistemático de la literatura (MSL). Si bien existen diversas características en el MSL, entre las principales destaca el nivel de profundidad, es decir, éste se aplica como una búsqueda a grandes rasgos de una temática en específico \cite{cascade_project_mapping_2012}; expresado esto, y tal como se mencionó al inicio, esta investigación se auxilió del MSL, que, de acuerdo con \textcite{garcia-holgado_tecnicas_2018}, dicho proceso provee diversos resultados, mismos que se determinan según el propósito de cada investigación.

Bajo esta perspectiva, el presente trabajo tiene como eje describir la alfabetización tecnológica, informática e informacional a través de su conceptualización, con el fin de establecer sus características y diferencias, por ende, el MSL, se describe por ser un método que provee técnicas para realizar “una amplia revisión de estudios primarios en un área específica que tiene como objetivo identificar qué evidencias están disponibles sobre el tema” \cite[s.p.]{garcia-penalvo_revisiones_2019}, por tanto, al ser flexible, su función principal es realizar una primera exploración de manera general hacia el objeto de estudio a través de una metodología establecida, así como mediante la revisión de cada artículo \cite{cruz-benito__systematic_2016}.
El proceso metodológico de la investigación empleó tres fases:
Fase 1. Residió en la aplicación de criterios para el protocolo de búsqueda.

\begin{enumerate}[label={\alph*}]
    \item Temporalidad en la búsqueda: 2010-2020 –valoración de documentos clásicos.
    \item Área de conocimiento: ciencias sociales.
    \item Exclusión: documentos de suscripción, documentos de embargo, artículos sin definiciones identificables en el resumen y definiciones derivadas de citas secundarias.
    \item Inclusión: cualquier tipo de comunicación científica formal o informal de acceso abierto que provea una cita directa al propio autor de la información. Autores clásicos sin importar su temporalidad.
    \item Idiomas: inglés y español.
    \item Definición de términos de búsqueda: alfabetización tecnológica, como la enseñanza de los conocimientos básicos de la computadora (particularmente hardware); alfabetización informática como la enseñanza del software y, por último, la alfabetización informacional como la búsqueda, evaluación y recuperación de información. 
    \item Bases de datos utilizadas en la búsqueda: de acceso abierto (Google Académico) y de suscripción (ACM, EBSCO, Elsevier, Scopus y Web of Science). 
\end{enumerate}

Fase 2. Refinamiento de la búsqueda.

En primer lugar, se establecieron los vocablos con su tentativa conceptualización, se buscaron los términos utilizando “ecuaciones” con los operadores booleanos. Los operadores booleanos que se utilizaron para la búsqueda fueron “AND” y “OR” como el término puesto en comillas para acotar el número de resultados y obtener mayor eficiencia. “AND” se aplicó para unir las palabras alfabetización con sus variantes; “OR” para poner alfabetización como palabra base y las variantes como posibles resultados, y las comillas con el fin de hacer una búsqueda exacta del término a localizar.

Fase 3. Consistió en el análisis en profundidad de los documentos identificados, es decir, se recuperaron los artículos en extenso, se revisaron, se examinaron las palabras claves, se registró el tipo de metodología (cuantitativa, cualitativa y mixta), los instrumentos descritos –en su caso–, los objetivos del estudio, la población, resultados y las referencias empleadas.

La \cref{tbl04} sintetiza un primer resultado derivado de la aplicación de la fase 1.

\begin{table}[htpb]
\caption{Resultados en idioma inglés y español de los términos identificados en las bases de datos.}
\label{tbl04}
\small
\begin{tabular}{lrrrrrr}
\toprule
\multicolumn{7}{c}{Términos} \\ 
\midrule
\begin{tabular}[c]{@{}l@{}}Bases de \\ datos\end{tabular} &
\begin{tabular}[c]{@{}l@{}}Alfabetización \\ informática\end{tabular} &
\begin{tabular}[c]{@{}l@{}}Computer \\ Literacy\end{tabular} &
\begin{tabular}[c]{@{}l@{}}Alfabetización \\ informacional\end{tabular} &
\begin{tabular}[c]{@{}l@{}}Information \\ Literacy\end{tabular} &
\begin{tabular}[c]{@{}l@{}}Alfabetización \\ tecnológica\end{tabular} &
\begin{tabular}[c]{@{}l@{}}Technological \\ Literacy\end{tabular} \\ 
\midrule
ACM & 28,864 & 2’315,586 & 58 & 1’574,167 & 1,732 & 1’078,181 \\ 
EBSCO & 59 & 25,559 & 346 & 58,011 & 198 & 6,931 \\ 
Elsevier & 41 & 24,277 & 30 & 61,119 & 80 & 14,859 \\ 
\begin{tabular}[t]{@{}l@{}}Google \\ académico\end{tabular} & 52,200 & 2’200,000 & 26,600 & 3’310,000 & 81,000 & 1’660,000 \\
Scopus & 231 & 94,758 & 503 & 148,292 & 391 & 26,642 \\ 
\begin{tabular}[t]{@{}l@{}}Web of \\ Science\end{tabular} &  0 & 3,726 & 6 & 15,436 & 0 & 1,456 \\ 
\rowcolor[HTML]{E9E9E9} Total & 81,395 & 4’663,906 & 27,543 & 5’167,025 & 83,401 & 2’788,069 \\
\bottomrule
\end{tabular}
\source{elaboración propia.}
\end{table}

En las columnas en color gris de la \cref{tbl04}, puede observarse mayor presencia numérica de términos en el idioma inglés, siendo Information Literacy –de los tres conceptos– el que muestra más concurrencias, con un total de 5’167,025 documentos, mientras Technological Literacy tuvo una menor representatividad. En cuanto a los resultados preliminares en el idioma español, la alfabetización tecnológica se destaca con 83,401 archivos, mientras que la alfabetización informacional asume la cifra más baja en ambos idiomas. Además, se observó que, en ambas lenguas, el vocablo de la alfabetización informática tiene un recuento medio.

La \cref{tbl05} representa lo identificado al aplicar la fase 2, es decir, el refinamiento de la búsqueda. Los parámetros establecidos en este filtro fueron fecha, palabras claves y relevancia en el campo.

\begin{table}[htpb]
\caption{Resultados con filtros.}
\label{tbl05}
\small
\centering
\begin{tabular}{lrrrrrr}
\toprule
 & \multicolumn{6}{c}{Base de datos} \\ \cline{2-7}  
  \multicolumn{1}{c}{Términos} &
  \multicolumn{1}{c}{ACM} &
  \multicolumn{1}{c}{EBSCO} &
  \multicolumn{1}{c}{Elsevier} &
  \multicolumn{1}{c}{\begin{tabular}[c]{@{}l@{}}Google \\ Académico\end{tabular}} &
  \multicolumn{1}{l}{Scopus} &
  \multicolumn{1}{l}{\begin{tabular}[c]{@{}l@{}}Web of \\ Science\end{tabular}} \\ 
\midrule
\begin{tabular}[c]{@{}l@{}}Alfabetización \\ tecnológica\end{tabular} & 1 & 27 & 0 & 76 & 0 & 0 \\ 
\begin{tabular}[c]{@{}l@{}}Technological \\ Literacy\end{tabular} & 11 & 66 & 321 & 344 & 371 & 31 \\ 
\rowcolor[HTML]{E9E9E9} Total & 12 & 93 & 321 & 420 & 371 & 31 \\
\begin{tabular}[c]{@{}l@{}}Alfabetización \\ informática\end{tabular} & 0 &  1 &  1 &  12 &  2 &  0 \\
\begin{tabular}[c]{@{}l@{}}Computer \\ Literacy\end{tabular} & 69 &   113 &  1,457 &  598 &   1,457 &  64 \\ 
\rowcolor[HTML]{E9E9E9} Total & 69 & 114 & 1,458 & 610 & 1,459 & 64 \\ 
\begin{tabular}[c]{@{}l@{}}Alfabetización \\ informacional\end{tabular} & 0 &   37 &  27 &  340 & 163 & 2 \\ 
\begin{tabular}[c]{@{}l@{}}Information \\ Literacy\end{tabular} & 93 & 2,089 & 1,517 & 9,940 & 1,917 & 1,052 \\ 
\rowcolor[HTML]{E9E9E9} Total & 93 & 2,126 & 1,554 & 10,280 & 2,080 & 1,054 \\
\bottomrule
\end{tabular}
\source{elaboración propia.}
\end{table}

Se observa la tendencia del hallazgo de mayor número de términos en el idioma inglés en las tres alfabetizaciones, aun con la aplicación de los filtros; mientras que en el idioma español la alfabetización informacional obtuvo más resultados. Al respecto, cabe resaltar que las bases de datos en inglés aportan la mayor cantidad de información y que en la suma de los casos (con excepción del vocablo Computer Literacy) Google Académico repuntó en la mayor concentración de artículos de investigación. Lo anterior se puede deber a que la mayoría de las bases de datos están especializadas en inglés (por lo tanto, las búsquedas como resultados están en esa lengua) y Google Académico y EBSCO son las únicas (de las consultadas) que arrojan información considerable completamente en español.

\section{Resultados}\label{sec-resultados}
De los 22, 209 artículos iniciales –identificados al principio de la investigación a través del MSL–, se obtuvo finalmente una muestra depurada de 40 documentos, de donde se derivan los resultados de las preguntas (RP) descritos a continuación, mismos que se presentan de manera general a través de los siguientes hallazgos (\cref{tbl06}): 

\begin{table}[htpb]
\caption{Resultados generales de las preguntas de investigación.}
\label{tbl06}
\begin{tabular}{lp{6.5cm}p{6.5cm}}
\toprule
 & Pregunta & Resultado \\
\midrule
\arrayrulecolor[gray]{.7}
\textbf{RP1} &
  ¿Cuáles son los límites conceptuales entre los diferentes tipos de alfabetización?  &
  Los límites conceptuales de los tres tipos de alfabetización residen en función del área de estudio: conocimiento del hardware, software y uso de la información.
  \\ \midrule
\textbf{RP2} &
  ¿Cuál es la definición constitutiva y operacional de la alfabetización informática, tecnológica e informacional? &
  Las diferencias constitutivas están orientadas al grado de conocimiento sobre la tecnología y las operacionales al nivel de ejecución.
  \\ \midrule
\textbf{RP3} &
  ¿Qué aportaciones teórico-metodológicas proponen los diferentes autores dentro del Mapeo Sistemático de la Literatura? &
  Se encontraron 13 aportaciones teórico-metodológicas encontradas en seis países y regiones. 
  \\ \midrule
\textbf{RP4} &
  ¿Cuáles son los criterios e indicadores que definen según los autores sobre los tres tipos de alfabetizaciones? &
  Se obtuvieron 13 criterios y 70 indicadores.
  \\ \arrayrulecolor{black}
\bottomrule
\end{tabular}
\source{elaboración propia.}
\end{table}

En seguida, se describe de forma más detallada las respuestas a cada una de las preguntas de investigación (\cref{tbl07}, \cref{tbl08}, \cref{tbl09} y \cref{tbl10}).

\begin{small}
\begin{longtable}{
    >{\raggedright\arraybackslash}
    p{0.03\textwidth}!{\color[gray]{.7}\vrule}
    p{0.93\textwidth}
}
\caption{RP1 Conceptualización de la alfabetización informática, tecnológica e informacional.}
\label{tbl07}
\\
\toprule
\multicolumn{1}{c}{ } & Definición \\ 
\midrule
\arrayrulecolor[gray]{.7}
\multirow{8}{*}{\rotatebox[]{90}{\centering Alfabetización tecnológica}}
 & “El propósito de esta alfabetización es desarrollar en los sujetos las habilidades para el uso de la informática en sus distintas variantes tecnológicas: computadoras personales” \cite[p. 3]{area_moreira_alfabetizacion_2014}. \\ 
 \cline{2-2} 
 &
  “La ‘alfabetización tecnológica’ capacita a las personas para comprender las aplicaciones de las tecnologías y las decisiones que implican su utilización” \cite[p. 194]{gallardo_marquez_alfabetizacion_2016}. \\ 
 \cline{2-2} 
 &
  “Facilita el conocimiento básico de la computadora para lograr su aplicación  funcional en la vida personal y laboral, con lo cual se busca sensibilizar a los educandos sobre la necesidad del uso de las Tecnologías de la Información y la Comunicación (TIC) como herramientas de conocimiento, información y aprendizaje” \cite[p. 37]{diario_oficial_de_la_federacion_reglas_2007}. \\ 
 \cline{2-2} 
 &
  “Promover el uso de las TIC, en especial las tecnologías móviles, en los  programas de alfabetización” \cite[p. 48]{unesco_educacion_2016}. \\ 
 \cline{2-2} 
 &
  “Comprende la adquisición de conocimientos básicos sobre los medios tecnológicos de comunicación más recientes e innovadores” \cite[p. 8]{unesco_estandares_2008}.  \\ 
 \cline{2-2} 
 &
  “Con lo que los ciudadanos y ciudadanas deben conocer para usar las herramientas producto del desarrollo tecnológico” \cite[p. 590]{garcia-vera_alfabetizacion_2007}. \\ 
 \cline{2-2} 
 &
  “Al uso y conocimiento del manejo de ordenadores, agendas electrónicas, sistemas GPS, entre otros” \cite[p. 112]{ortega_navas_dimension_2009}. \\ 
 \cline{2-2} 
 &
  “Consiste en la adquisición de conocimientos y habilidades cognitivas e instrumentales con relación al manejo de las nuevas tecnologías” \cite[p. 113]{ortega_navas_dimension_2009}. \\ 
\arrayrulecolor{black}
\midrule
\arrayrulecolor[gray]{.7}
\multirow{6}{*}{\rotatebox[]{90}{\centering Alfabetización informática \ \ }}  &
  “La alfabetización informática (saber usar un equipo de cómputo)” \cite[p. 31]{garcia_martinez_programa_2016}. \\ 
 \cline{2-2} 
 &
  “Alfabetización en informática es la que se ocupa de la formación en el uso de programas informáticos” \cite[p. 194]{gallardo_marquez_alfabetizacion_2016}. \\
 \cline{2-2} 
 &
  “Apropiarse de herramientas informáticas de interés general (como las que facilitan y/o permiten la organización semántica, la interpretación de información, la construcción de conocimiento, la conversación y colaboración)” \cite[p. 5]{azinian_multiples_2006}. \\ 
 \cline{2-2} 
 &
  “Utiliza el término en el contexto de la formación de estudiantes en el uso de  Internet como fuente de información” \cite[p. 3]{bawden_revision_2002}. \\
 \cline{2-2} 
 &
  “Tiene que ver con el hardware, las redes de datos y el software necesarios para tratar información de forma automática” \cite[p. 60]{pozo-jara_alfabetizacion_2017}. \\ 
 \cline{2-2} 
 &
 “Los conocimientos, habilidades y actitudes requeridos para la utilización de la  tecnología informática en la vida diaria” \cite[p. 141]{rodriguez_espinosa_alfabetizacion_2014}.  \\ 
\arrayrulecolor{black}
\midrule
\arrayrulecolor[gray]{.7}
\multirow{9}{*}{\rotatebox[]{90}{\centering Alfabetización informacional}} &
  “El origen de esta propuesta procede de los ambientes bibliotecarios. Surge como respuesta a la complejidad del acceso a las nuevas fuentes bibliográficas distribuidas en bases de datos digitales. Se pretende desarrollar las competencias y habilidades para saber buscar información en función de un propósito dado, localizarla, seleccionarla, analizarla y reconstruirla” \cite[p.  4]{area_moreira_alfabetizacion_2014}. \\ 
 \cline{2-2} 
 &
  “Conjunto de competencias cruciales que permite a los individuos beneficiarse de la gran cantidad de conocimiento disponible en formato oral, en papel y en  formato electrónico, lo esencial aquí es que la transformación de la información en conocimiento requiere competencias en Alfin” \cite[p. 15]{catts_hacia_2009}. \\ 
 \cline{2-2} 
 &
  “La alfabetización informacional está relacionada con el desarrollo de competencias de gestión de la información e implica la realización de las acciones adecuadas en relación con ella para identificar el medio o canal que responde a la necesidad, hacer una selección crítica de fuentes, seleccionar la manera apropiada de comunicarla, etcétera” \cite[p. 5]{azinian_multiples_2006}. \\ 
 \cline{2-2} 
 &
  “La AI es la capacidad de acceder, evaluar y utilizar la información a partir de una variedad de fuentes” \apud{doyle_information_1994}[p. 334]{castillo_perez_evolucion_2016}. \\ \cline{2-2} 
 &
 “Es saber cuándo y por qué necesitas información, dónde encontrarla, y cómo evaluarla, utilizarla y comunicarla de manera ética” \cite[p. 79]{abell_alfabetizacion_2004}. \\ 
 \cline{2-2} 
 &
  “De la alfabetización informacional como una metaliteracy que admite varios tipos de alfabetización, incluyendo la alfabetización digital, alfabetización mediática, la alfabetización visual y tecnológica” \cite[párr.~4]{alonso-arevalo_alfabetizacion_2014}.  \\ 
 \cline{2-2} 
 &
 “Se considera que tener AI es saber cuándo y por qué necesitas información, dónde encontrarla y cómo evaluarla, utilizarla y comunicarla de manera ética” \cite[p. 335]{castillo_perez_evolucion_2016}. \\ 
\cline{2-2} 
 &
  Se vincula con el cúmulo de conocimientos, habilidades, aptitudes, valores destrezas, capacidades, saberes, conductas; en otras palabras, las competencias  pertenecientes a la información \cite{seal_value-added_1988, tuckett_computer_1989, bruce_seven_1997, gomez_2000, gomez-hernandez_gestion_2002, angulo_marcialm_normas_2003, unesco_declaracion_2003, byrne_alfabetizacion_2005, ifla_2005}. \\ 
 \cline{2-2} 
 &
  “Habilidades para encontrar información (habilidades de localización y recuperación documental, y habilidades de manejo de equipos tecnológicos), usar información (habilidades de pensamiento, habilidades de estudio e investigación; habilidades de producción y de presentación), y para compartir y actuar éticamente respecto a la información” \cite[p. 196]{gomez_hernandez_problemas_2002}. \\ 
\arrayrulecolor{black}
\bottomrule
\source{elaboración propia.}
\end{longtable}
\end{small}

\begin{small}
\begin{longtable}{
    >{\raggedright\arraybackslash}
    p{0.3\textwidth}
    p{0.3\textwidth}
    p{0.3\textwidth}
}
\caption{RP2 Definición constitutiva y operacional de la alfabetización informática, tecnológica e informacional.}
\label{tbl08}
\\
\toprule
\multicolumn{3}{c}{Definición constitutiva} \\
%\arrayrulecolor[gray]{.7}
\midrule 
%\arrayrulecolor{black}
\underline{Tecnológica} & \underline{Informática} & \underline{Informacional} \\ 
%\midrule
Conocimiento básico del hardware de los equipos tecnológicos. “Conocimientos y habilidades tanto instrumentales como cognitivas en relación con la información vehiculada a través de nuevas tecnologías vinculadas con el hardware” \cite[párr. 12]{area_moreira_alfabetizacion_2014}. &
Conocimiento básico del software de los equipos tecnológicos. “Conocimiento y uso de las herramientas dentro de las tecnologías de la información, incluyendo el hardware, el software y los programas de multimedia” \cite[p. 4]{shapiro_information_1996}. &
La capacidad de identificar, evaluar, organizar, crear, localizar, utilizar y comunicar éticamente y con eficacia la información \cite{bawden_revision_2002}. \\
\\
\toprule
\multicolumn{3}{c}{Definición operacional} \\
\midrule 
\underline{Tecnológica} & \underline{Informática} & \underline{Informacional} \\
%\midrule
Identificar los componentes físicos de la computadora, encenderla y apagarla, conocer los dispositivos de entrada y salida, así como reconocer los principales proveedores de Internet. &
Conocer el sistema operativo Windows/MacOS, utilizar Internet y el paquete Microsoft Office. &
Acceso, evalúo y utilización de la información a partir de una variedad de fuentes. \\ 
\midrule 
\multicolumn{3}{p{0.9\textwidth}}{Las definiciones constitutiva y operacionales presentadas, refieren que los tipos de alfabetización son secuenciadas, entendido esto, como el dominio de habilidades y conocimientos requeridos para su conocimiento/aplicación, por tanto, van de la alfabetización tecnológica, a la alfabetización informática, hasta alcanzar la alfabetización informacional.} \\
\bottomrule
\source{elaboración propia.}
\end{longtable}
\end{small}





\begin{small}
\begin{longtable}{c!{\color[gray]{.7}\vline}cccp{5cm}}
\caption{RP3 Aportaciones teórico-metodológicas.}
\label{tbl09}
\\
\toprule
\multicolumn{1}{c}{Alfabetización} & Autores & Año & País & \begin{tabular}[c]{@{}c@{}}Principal aporte teórico/\\  metodológico\end{tabular} \\ 
\midrule
\arrayrulecolor[gray]{.7}
\multirow{6}{*}{Tecnológica} & Area & 2002 & España & Modelo educativo integral para cualificar y alfabetizar en las nuevas tecnologías. \\ 
 \cline{2-5} 
 &
  \begin{tabular}[c]{@{}l@{}}International \\ Technology \\ Education \\ Association\end{tabular} & 2007 & \begin{tabular}[c]{@{}l@{}}Estados \\ Unidos\end{tabular} &
  Aplicación de 20 estándares en dos dimensiones: lo que los estudiantes deberían conocer acerca de la tecnología y lo que ellos deberían de ser capaces de hacer. \\ \cline{2-5} 
 &
  \begin{tabular}[c]{@{}l@{}}European \\ Computer \\ Driver Licence\end{tabular} & 2011 & Europa &
  Elaboración de certificación sobre habilidades computacionales a nivel básico. \\ \cline{2-5} 
  &
  \begin{tabular}[c]{@{}l@{}}Online \\ Computer \\ Library Center\end{tabular} & 2011 & \begin{tabular}[c]{@{}l@{}}Estados \\ Unidos\end{tabular} &
  Seis nuevas habilidades para ser un alfabeto tecnológico. \\ \cline{2-5} 
  &
  \begin{tabular}[c]{@{}l@{}}Wilson \\ Castaño\end{tabular} & 2014 & \begin{tabular}[c]{@{}l@{}}Costa \\ Rica\end{tabular} &
  Propuesta de cinco dimensiones para la utilización de la tecnología. \\ \cline{2-5} 
  & Avello et al. & 2015 & Cuba & Tres dimensiones y criterios sobre la tecnología. \\
\arrayrulecolor{black}
\midrule
\arrayrulecolor[gray]{.7}
\multirow{2}{*}{Informática} & \begin{tabular}[c]{@{}l@{}}Shapiro y \\ Hughes\end{tabular} &  1996 & \begin{tabular}[c]{@{}l@{}}Estados \\ Unidos\end{tabular} &
  Programa de alfabetización informática basado en siete dimensiones. \\ \cline{2-5} 
  &
  \begin{tabular}[c]{@{}l@{}}European \\ Computer \\ Driver Licence\end{tabular} & 2011 & Europa & Elaboración de certificación sobre habilidades informáticas a nivel básico. \\ 
\arrayrulecolor{black}
\midrule
\arrayrulecolor[gray]{.7}
\multirow{7}{*}{Informacional} & \begin{tabular}[c]{@{}l@{}}Eisenberg y \\ Berkowitz\end{tabular} &  1990 & \begin{tabular}[c]{@{}l@{}}Estados \\ Unidos\end{tabular} &
  Modelo que utiliza el pensamiento crítico para la solución de problemas de información. \\ \cline{2-5} 
 & \begin{tabular}[c]{@{}l@{}}American \\ Association \\ of School \\ Librarians\end{tabular} & 1998 & \begin{tabular}[c]{@{}l@{}}Estados \\ Unidos\end{tabular} &
  Elaboración de nueve estándares de alfabetización informacional para el aprendizaje estudiantil. \\ \cline{2-5} 
 &
  \begin{tabular}[c]{@{}l@{}}American \\ Library \\ Association\end{tabular} & 2000 & \begin{tabular}[c]{@{}l@{}}Estados \\ Unidos\end{tabular} &
  Diseño de un marco de referencia que se integra al currículo universitario, este contiene cinco estándares, 22 indicadores de rendimiento y resultados esperables. \\ \cline{2-5} 
 & Félix Benito & 2000 & España & Propuesta de 11 habilidades de información. \\ \cline{2-5} 
 & Bruce & 2003 & \begin{tabular}[c]{@{}l@{}}Estados \\ Unidos\end{tabular} & Modelo constituido por siete fases (categorías) para el uso de información. \\ \cline{2-5} 
 & Abell et al. & 2004 & Inglaterra & Desarrollo de ocho habilidades para que una persona sea considerada alfabetizada en información. \\ \cline{2-5} 
 & \begin{tabular}[c]{@{}l@{}}Wilson \\ Castaño\end{tabular} & 2014 & \begin{tabular}[c]{@{}l@{}}Costa \\ Rica\end{tabular} & Propuesta de cinco dimensiones para la búsqueda de información.\\
\arrayrulecolor{black}
\bottomrule
\source{elaboración propia.}
\end{longtable}
\end{small}


\begin{small}
\begin{longtable}{c!{\color[gray]{.7}\vline}c!{\color[gray]{.7}\vline}cp{7.6cm}}
\caption{RP4 Criterios e indicadores de las aportaciones teórico-metodológicas.}
\label{tbl10}
\\
\toprule
\multicolumn{1}{c}{Constructo} & \multicolumn{1}{c}{\begin{tabular}[c]{@{}c@{}}Dimensión\\ dominada\end{tabular}} & Criterios & Indicadores \\ 
\midrule
\arrayrulecolor[gray]{.7}
\begin{tabular}[t]{@{}c@{}}Alfabetización \\ tecnológica\end{tabular} & 
\begin{tabular}[t]{@{}c@{}}Dimensión \\ instrumental\end{tabular} & 
\begin{tabular}[t]{@{}c@{}}Conceptos \\ básicos \\ de la \\ computadora\end{tabular} &
\vspace{-\baselineskip}
\begin{itemize}[label={--},noitemsep,leftmargin=*,topsep=0pt,partopsep=0pt]
\item Enumerar y comparar diferentes tipos de computadora: computadora personal, computadora portátil, todo en uno, reproductor multimedia, teléfono inteligente, tableta. 
\item Identificar y comprender diferentes componentes: unidad del sistema, monitor, ratón, teclado.
\item Iniciar sesión en la computadora de forma segura con un nombre de usuario y contraseña.
\item Comprender las funciones de un ratón: seleccionar elementos, mover elementos, emitir comandos a la computadora.
\item Comprender cómo usar el ratón: haga clic, haga doble clic, mueva con clic y arrastrar.
\item Comprender las formas del puntero del ratón: haga clic para ingresar texto, punto, computadora ocupada, hipervínculo aquí.
\item Usar el hacer clic y arrastrar para mover los elementos seleccionados en la pantalla.
\item Comprender que el teclado es un modo de entrada de datos y un método de dar comandos a la computadora.
\item Comprender y usar las teclas del teclado como letras y números, enter, retroceso, desplazamiento, barra espaciadora, bloqueo de mayúsculas, eliminar.
\end{itemize}
\\ 
\arrayrulecolor{black}
\midrule
\arrayrulecolor[gray]{.7}
\multirow{6}{*}{\begin{tabular}[c]{@{}c@{}}Alfabetización \\ informática\end{tabular}} &
\multirow{6}{*}{\begin{tabular}[c]{@{}c@{}}Dimensión \\ digital\end{tabular}} &
Escritorio & 
\vspace{-\baselineskip}
\begin{itemize}[label={--},noitemsep,leftmargin=*,topsep=0pt,partopsep=0pt]
\item Comprender qué es un escritorio de computadora.
\item Comprender el término ícono. Reconocer y comprender los íconos de escritorio comunes como documentos, computadora, redes, elementos eliminados, navegador.
\item Comprender qué es la barra de tareas y algunas de sus características: botón de inicio para iniciar una aplicación, reloj, capacidad para cambiar entre ventanas abiertas, indicador de idioma.
\item Seleccionar, activar íconos de escritorio comunes.
\item Apagar la computadora correctamente.
\end{itemize} \\
\cline{3-4}
 & & Windows &
\vspace{-\baselineskip}
\begin{itemize}[label={--},noitemsep,leftmargin=*,topsep=0pt,partopsep=0pt]
\item Identificar partes de una ventana como barra de título, barras de desplazamiento, marcadores de barra de desplazamiento, barra de estado, barra de menú, cinta de opciones, barra de herramientas.
\item Contraer, expandir, cambiar el tamaño, mover, cerrar una ventana.
\item Desplazarse hacia arriba y hacia abajo en una ventana.
\item Cambiar entre ventanas abiertas.
\item Conocer los principales tipos de medios de almacenamiento como disco duro interno, unidad \textit{flash} USB, DVD, almacenamiento de archivos en línea.
\item Comprender la función de diferentes tipos de aplicaciones como procesamiento de textos, hoja de cálculo, base de datos, presentación.
\end{itemize} \\
\cline{3-4}
 & & \begin{tabular}[c]{@{}l@{}}Creación de \\ documentos\end{tabular} &
\vspace{-\baselineskip}
\begin{itemize}[label={--},noitemsep,leftmargin=*,topsep=0pt,partopsep=0pt]
\item Abrir una aplicación de procesamiento de texto.
\item Cambiar el formato del texto: tipos de fuente, tamaño de fuente.
\item Aplicar formato de texto: negrita, cursiva, subrayado.
\item Copiar, cortar, mover texto dentro de un documento.
\item Imprimir un documento desde una impresora instalada utilizando opciones de salida como documento completo, páginas específicas, número de copias.
\item Guardar y nombrar un documento.	
\end{itemize} \\
\cline{3-4}
 & & \begin{tabular}[c]{@{}l@{}}Manejo de \\ archivos\end{tabular} &
\vspace{-\baselineskip}
\begin{itemize}[label={--},noitemsep,leftmargin=*,topsep=0pt,partopsep=0pt]
\item Comprender qué es un archivo, carpeta.
\item Saber dónde se almacenan normalmente los archivos, programas.
\item Reconocer tipos de archivos comunes e íconos asociados como doc, .xls, .mdb, .jpg, .mp3.
\item Hacer doble clic para abrir archivos, carpetas.
\item Cerrar un archivo.
\end{itemize} \\
\cline{3-4}
 & & Internet &
\vspace{-\baselineskip}
\begin{itemize}[label={--},noitemsep,leftmargin=*,topsep=0pt,partopsep=0pt]
\item Comprender que Internet es la red física global de redes y se utiliza para soportar servicios como World Wide Web (WWW) y correo electrónico.
\item Comprender qué es la World Wide Web (WWW).
\item Identificar los tipos de recursos disponibles en la World Wide Web (WWW).
\item Comprender la importancia de evaluar la información en la World Wide Web (WWW).
\item Comprender que un proveedor de servicios de Internet (ISP) proporciona acceso a Internet.
\end{itemize} \\
\cline{3-4}
 & & \begin{tabular}[c]{@{}l@{}}Buscando en \\ Internet\end{tabular} &
\vspace{-\baselineskip}
\begin{itemize}[label={--},noitemsep,leftmargin=*,topsep=0pt,partopsep=0pt]
\item Comprender qué es un navegador web.
\item Reconocer que la página predeterminada de un navegador web se llama Página de inicio/inicio.
\item Comprender los términos localizador uniforme de recursos (URL), hipervínculo.
\item Comprender los términos favoritos/marcadores.
\item Comprender el término historial del navegador.
\item Ir a una URL.
\item Iniciar sesión en un sitio web con un nombre de usuario y contraseña.
\item Activar un hipervínculo/enlace de imagen.
\item Navegar en un sitio web: atrás, adelante, inicio.
\item Imprimir una página web.
\item Completar y enviar un formulario basado en la web.
\item Comprender el término motor de búsqueda.
\item Buscar información mediante palabras clave.
\item Descargar un archivo de una página web.
\item Comprender el concepto de una comunidad en línea (virtual). Reconocer ejemplos como sitios web de redes sociales, foros de Internet, salas de chat, juegos de computadora en línea, blogs.
\item Comprender el término \textit{phishing}. Reconocer intentos de \textit{phishing}.
\end{itemize} \\
\arrayrulecolor{black}
\midrule
\arrayrulecolor[gray]{.7}
\multirow{6}{*}{\begin{tabular}[c]{@{}c@{}}Alfabetización \\ informacional \end{tabular}} &
\multirow{6}{*}{\begin{tabular}[c]{@{}c@{}}Dimensión \\ informacional \end{tabular}} & 
\begin{tabular}[c]{@{}l@{}}Definición \\ de la tarea a \\ realizar\end{tabular} &
\vspace{-\baselineskip}
\begin{itemize}[label={--},noitemsep,leftmargin=*,topsep=0pt,partopsep=0pt]
\item Comunicarse con un profesor, académico o bibliotecario, expresando su necesidad con claridad, así como lo que necesita.
\item Definir problemas, facilitar actividades cooperativas en grupos.
\item Definir o refinar el problema de información, desarrollar un lenguaje de búsqueda.
\end{itemize} \\
\cline{3-4}
 & & \begin{tabular}[c]{@{}l@{}}Estrategias \\ para buscar \\ información\end{tabular} &
\vspace{-\baselineskip}
\begin{itemize}[label={--},noitemsep,leftmargin=*,topsep=0pt,partopsep=0pt]
\item Evaluar el valor de varios tipos de recursos electrónicos.
\item Evaluar la necesidad y el valor de los recursos primarios.
\item Identificar y aplicar un criterio específico para evaluar recursos.
\end{itemize} \\
\cline{3-4}
 & & \begin{tabular}[c]{@{}l@{}}Localización \\ y acceso\end{tabular} &
\vspace{-\baselineskip}
\begin{itemize}[label={--},noitemsep,leftmargin=*,topsep=0pt,partopsep=0pt]
\item Localizar y usar apropiadamente los recursos.
\item Reconocer los roles y experiencia de la gente que provee asistencia o información.
\item Usar material de referencia electrónica disponible dentro de sus posibilidades.
\end{itemize} \\
\cline{3-4}
 & & \begin{tabular}[c]{@{}l@{}}Uso de la \\ información\end{tabular} &
\begin{itemize}[label={--},noitemsep,leftmargin=*,topsep=0pt,partopsep=0pt]
\item Conectar y operar la tecnología para acceder a la información, conocer las guías y manuales asociados con esta tarea.
\item Conocer y ser capaz de utilizar de manera óptima los recursos.
\item Citar apropiadamente los trabajos utilizados para satisfacer la necesidad de información.
\item Analizar y filtrar la información en función de la tarea que está realizando, rechazando la información que no es relevante.
\end{itemize} \\
\cline{3-4}
 & & Síntesis &
\begin{itemize}[label={--},noitemsep,leftmargin=*,topsep=0pt,partopsep=0pt]
\item Clasificar y agrupar información.
\item Transformar la información en datos cuantificables.
\item Citar apropiadamente y dar crédito a los recursos.
\end{itemize} \\
\cline{3-4}
 & & Evaluación &
\begin{itemize}[label={--},noitemsep,leftmargin=*,topsep=0pt,partopsep=0pt]
\item Evaluar las presentaciones en términos de contenido, formato y diseño.
\item Aplicar principios legales y ética a la información relacionada con copyright y plagio.
\end{itemize} \\
\arrayrulecolor{black}
\bottomrule
\source{Elaboración propia con base a la aportación teórica-metodológica de \textcite{eisenberg_modelo_nodate, shapiro_information_1996, american_association_of_school_librarians_information_1998, american_library_association_information_2000, benito_morales_nuevas_2000, area_moreira_igualdad_2002, bruce_siete_2003, abell_alfabetizacion_2004, international_technology_education_association_standards_2000, european_computer_driver_licence_ecdl_ecdl_2011, oclc_libraries_2011, castano_munoz_alfabetizaciones_2014, avello__martinez_evolucion_2015}.}
\end{longtable}
\end{small}

\section{Consideraciones finales}\label{sec-consideracoes}
Si bien se reconoce que ningún resultado de investigación es del todo contundente, cabe expresar que, bajo una perspectiva teórica dentro del contexto de la tecnología educativa, el aporte principal de este artículo puede ser visualizado como sustento conceptual para el desarrollo de trabajos, cuyo objetivo se centre en el análisis de la alfabetización informática, tecnológica e informacional. En congruencia, este estudio disemina conceptos de las alfabetizaciones en el área tecnológica para lograr una adhesión importante entre las diferentes partes que componen el contexto de los individuos: la tecnología, la sociedad y la cultura. 

Con referencia a lo anterior, el contexto sociocultural legitima la aproximación que tienen los individuos a la tecnología, lo cual hace que su apropiación y conceptualización sea totalmente distinta entre ellos \cite{rueda_ramos_adultos_2009}. Además, el proceso de adaptación y apropiación a las TIC puede ser relativamente fácil para algunas personas y extremadamente complejo para otras, por lo tanto, es necesario conocer en qué consiste cada nivel de alfabetización, con el fin de diseñar programas de actualización y capacitación, además de aprovechar el uso de la tecnología a favor del desarrollo pleno de la sociedad. 

Lo anterior hace sentido si se toma en cuenta que, en el contexto actual, la tecnología ha cambiado la forma en que se realizan las cosas. En esta línea, \textcite{castano_munoz_alfabetizaciones_2014} menciona que las TIC no solo modificaron la manera en que se accede a la información, sino también el modo de trabajar en diferentes ambientes y, por otro lado, la adquisición de equipo tecnológico ha transformado las estructuras sociales, así como la interacción social que, a final de cuentas, lleva al surgimiento de nuevas perspectivas y acercamientos para ver y vivir en el mundo globalizado \cite{carrera_cruce_2019}.

Por ello, los diferentes tipos de alfabetización son cruciales para el desarrollo pleno de las personas en una sociedad activa y ligada enteramente con la tecnología. No se puede perder de vista que “Las tecnologías digitales están presentes en todas las esferas de la vida, y configuran de manera sustancial el modo en que vivimos, trabajamos, aprendemos y socializamos” \cite[párr. 4]{unesco_alfabetizacion_2017}, como lo expresó Irina Bokova, directora general de la UNESCO, el Día Internacional de la Alfabetización.
En nuestra sociedad, los medios tecnológicos son clave para la comunicación, comercio, educación, inclusión, igualdad y salud, entre otros; además, la sociedad depende cada vez más de los avances científicos y tecnológicos, por lo tanto, no basta con saber leer y escribir, también es necesario obtener competencias que aseguren la participación ciudadana dentro de la sociedad del conocimiento y la economía digital.

\printbibliography\label{sec-bib}

\end{document}
