%\setlength\LTleft{-0.3in}
%\setlength\LTright{-0.3in}
\begin{small}
\begin{longtable}{
    p{0.31\textwidth}
    p{0.63\textwidth}
    }
\caption{Éléments d’analyse contrastive.}
\label{tbl-tabela-01}
\\
\toprule 
L’original et sa traduction en français & Source de la traduction dans Reverso  et observations  \\
\midrule
\textlang{arabic}{المؤمن كالنحلة تأكل طيبا وتضع طيبا} \newline 
Le croyant comme une abeille bien manger et bien mettre & 
Remarque 1 :  verbes qui ne sont pas conjugués \newline
Remarque 2 :  le mot «\textlang{arabic}{ طيبا  }» ne signifie pas l’adverbe bien mais plutôt c’est un nom qui signifie « les grains de pollen » et produit « le miel ». Cette expression signifie que le croyant musulman est une bonne personne au service de sa communauté. La rime contenue dans l’adage donne une musicalité qui véhicule l’idée de la bonté.  \\
\midrule
\textlang{arabic}{أن تقرا كتابا ألف مرة خير من أن تقرا ألف كتاب} \newline
Lire un livre mille fois vaut mieux que lire mille livres &
Aucune erreur détectée \\
\midrule
\textlang{arabic}{علم ابنك الصيد خير من أن تعطيه سمكة} \newline
Ton fils a enseigné la chasse est mieux que de lui donner un poisson & 
Remarque 1 : non-respect de la structure française de l’impératif :  Enseigne à ton fils puisque la traduction automatique a été proposée sur le modèle de la phrase française qui commence souvent par sujet+ verbe+ complément. \newline
Remarque 2 : le choix du verbe être qui n’est pas adéquat au contexte. Or, ce choix n’affecte aucunement la sémantique de la phrase.
\\
\midrule
\textlang{arabic}{الجاهل عدو نفسه فكيف يكون صديق غيره} \newline 
L’ignorant est l’ennemi de lui-même, comment peut-il être l’ami des autres
&
Aucune erreur détectée 
\\
\midrule
\textlang{arabic}{سترى حين ينجلي الغبار أفرس تحتك أم حمار} \newline
Vous verrez quand la poussière se dessèche, montez-vous ou âne &
Remarque 1 : erreur de traduction de l’interrogation contenue dans une lettre à savoir “Alif”, la première lettre arabe. Ce à quoi n’est pas habitué le système. \newline
Remarque 2 : traduction du nom \textlang{arabic}{فرس } qui veut dire cheval par le verbe « montez ». C’est une erreur à la fois sémantique, lexicale et syntaxique : traduction d’un nom par un verbe, ce qui a faussé la phrase.
Remarque 3 : Omission de la préposition «\textlang{arabic}{ تحتك }» en raison du choix du verbe français « montez » \newline
Remarque 4 : Traduction de la deuxième personne du singulier par la deuxième personne du pluriel. Il s’agit d’une erreur d’énonciation. 
\\
\midrule
\textlang{arabic}{الحِلْم أجَلُّ من العقل} \newline 
Rêve pour raison &
Remarque 1 : traduction erronée du mot \textlang{arabic}{الحِلم } qui veut dire sagesse et gentillesse confondu avec son homonyme \textlang{arabic}{ الحُلم } qui veut dire rêve. A défaut de reconnaitre les flexions arabes et les distinguer, une erreur de sens s’est produite dans la traduction. \newline
Remarque 2 :  le logiciel a confondu entre «\textlang{arabic}{ من أجل  }» qui veut dire « pour » et «\textlang{arabic}{ أجل من }» qui veut dire « mieux que ». Cela peut s’expliquer par les modèles proposés au système qui ne reproduit que ce qui existe dans ses archives et ses bases de données.
\\
\midrule
\textlang{arabic}{خير الكلام ما قل ودل} \newline
\begin{enumerate}
\item Les meilleurs mots sont ce que j’ai dit et ce que j’ai dit
\item Moins on en dit, mieux c'est, non ?
\item Concis et direct
\end{enumerate} 
&
Remarque : la répétition  en français “ce que j’ai dit” qui ne rend pas le sens de la deuxième partie de l’adage ni d’ailleurs le sens global.
Les deux autres sont des traductions par adaptation qui donne l’essentiel du message.
\\
\midrule
\textlang{arabic}{الصِّيتُ ولا الغنى} \newline
Pas de traduction proposée mais traduction des deux mots: \newline
\textlang{arabic}{الصِّيتُ} : bien connue \newline
\textlang{arabic}{الغنى} : Riche, richesse abondance.
&
Remarque : le logiciel a détecté le sens des deux mots constituants la phrase, mais il n’a pas restitué la sémantique de la périphrase pour deux raisons : la première,  la tournure phrastique relève d’un dialecte arabe, à savoir l’arabe égyptien et qu’il ne reconnait pas dans sa base de données. La deuxième raison ; la négation accompagnée de la conjonction de coordination \textlang{arabic}{و} qui relève d’une façon de parler dialectale.
\\
\midrule
\textlang{arabic}{العلم يُؤْتَى ولا يَأْتِي} \newline
La science vient et ne vient pas
&
Remarque : le logiciel ne distingue pas entre la forme passive et la forme active du verbe contenu dans la phrase par les flexions \textlang{arabic}{ ُ ِ }. \newline
Remarque 2 : la traduction du verbe arabe par « venir » associé au nom science ne produit pas le sens voulu par la périphrase. La personnification n’a pas été reproduite avec l’effet souhaité par la langue arabe. 
\\
\midrule
\textlang{arabic}{الناس معادن} \newline
C'est dans de tels moments que les gens se révèlent sous leur vrai jour.
\textlang{arabic}{ وفي مثل هذه الأوقات تظهر معادن الناس}
&
Remarque : la traduction dégagée du corpus parallèle proposé par Reverso Context véhicule le sens mais ne donne pas la traduction de l’adage sous sa forme préconstruite, c’est à dire en tant qu’expression figée.  
\\
\midrule
\textlang{arabic}{أشد الجهاد مجاهدة الغيظ} \newline 
Pas de traduction contextualisée
&
Remarque : le mot Djihad en arabe a été repris tel quel dans les propositions de traduction (procédé de l’emprunt dans le sens islamiste du terme et synonyme de terrorisme ), alors que le sens voulu dans l’adage est celui de la lutte que mène toute personne contre ses défauts et ses vices à l’instar de la colère.
\\
\midrule
\textlang{arabic}{أهل مكة أدرى بشعابها
فالعراقيون أدرى بشعاب أحيائهم ولغتهم وثقافتهم وهم الأفضل تأهيلا لتقديم حل أمني طويل الأجل.} \newline
Connaissant la vie des quartiers, la langue et la culture, les Iraquiens sont les mieux placés pour assurer la sécurité à long terme.
&
Remarque : la traduction dégagée du corpus parallèle proposé par Reverso Context véhicule le sens mais ne donne pas la traduction de l’adage sous sa forme d’expression figée.
\\
\midrule
\textlang{arabic}{درهم وقاية خير من قنطار علاج} \newline 
Mieux vaut prévenir que guérir
&
Remarque : cet adage semble circuler sur les bases de données des sites web dans les versions arabes et françaises, c’est pourquoi il a été correctement traduit vers la langue française. \newline
\textlang{arabic}{في الختام، اسمحوا لي أن أشدد على أنه بينما يتصف المبدأ القائل بأن درهم وقاية خير من قنطار علاج بالحقيقة} \newline
Pour terminer, permettez-moi de souligner que si l'adage selon lequel \href{https://context.reverso.net/traduction/francais-arabe/mieux+vaut+prévenir+que+guérir}{mieux vaut prévenir que guérir} est sans doute vrai
\\
\midrule
\textlang{arabic}{عند الامتحان يكرم المرء أو يهان} \newline
À l’examen, on honore ou on insulte 
&
Remarque 1 : Absence de la forme passive contenu dans la phrase arabe \newline
Remarque 2 : suppression du sujet et son remplacement par le pronom personnel « On », ce qui n’est pas faux, mais devait être suivi par un temps verbal adéquat qui rend la forme passive \newline
Remarque 3 : usage du temps présent au lieu du passé composé \newline
Remarque 4 : erreur de la traduction du verbe : \textlang{arabic}{يُهان } qui veut dire  être humilié dans ce contexte et non insulter.
\\
\midrule
\textlang{arabic}{فاقد الشيء لا يعطيه} \newline
\href{https://context.reverso.net/traduction/francais-arabe/nemo+dat+quod+non+habet}{nemo dat quod non habet}
&
Remarque :  cette expression qui veut dire “Personne ne peut donner ce qu’il n ‘a pas” a été traduite par une expression latine qui relève du domaine du droit. 
\\
\midrule
\textlang{arabic}{من سَلِمَتْ سريرته سلمت علانيته} \newline 
Qui a livré son lit a livré son public
&
Remarque 1 : Usage du pronom relatif sans ajout du pronom démonstratif pour assurer la cohérence énonciative. \newline 
Remarque 2 :  traduction lexicale erronée du verbe “\textlang{arabic}{سَلم}” qui veut dire avoir une bonne intention quand il est associé au mot “\textlang{arabic}{سريرته}” qui a été traduit par “lit”. L’expression adaptée à l’arabe sera : celui qui est bon, l’est en privé ou en public 
\\
\midrule
\textlang{arabic}{من لا يَرْحم لا يُرْحَمْ} \newline
Qui ne graisse pas n’est pas graissé
&
Remarque 1 : Usage du pronom relatif sans ajout du pronom démonstratif pour assurer la cohérence énonciative.\newline 
Remarque 2 : erreur de transposition de la forme passive sans distinguer entre cette forme et la forme active \newline
Remarque 3 : Erreur de traduction du sens du verbe qui veut dire avoir de la pitié. Quand il est à la forme négative : il peut signifier : sans pitié, impitoyable et cruel.
\\
\midrule
\textlang{arabic}{هي الدنيا تُحَبُّ ولا تُحابِي} \newline
Elle est le monde qui aime et ne favorise pas
&
Remarque 1 : traduction du pronom personnel par “ elle”,  au lieu de commencer par le sujet “ la vie “; ce qui a alourdi la phrase française. \newline
Remarque 2 :  présence du pronom relatif qui n’existe pas en arabe. \newline
Remarque 3 : les flexions de la forme passive n’ont pas été prises en compte lors de la traduction. \newline
L’adage veut véhiculer l’idée suivante : C’est la vie ; aimez là mais avec précaution (il ne faut pas s’y attacher trop)
\\
\midrule
\textlang{arabic}{وشر البلية ما يضحك} \newline 
\textlang{arabic}{البلية} :  ce fléau
\textlang{arabic}{ما يضحك} :  rien de drôle ce qui est drôle
&
Remarque : le logiciel a détecté le sens des deux mots constituants la phrase, mais il n’a pas restitué la traduction de la périphrase. La combinaison des deux mots qui forme l’expression figée de l’adage signifie qu’il y a des malheurs qui font rire. Par conséquent \textlang{arabic}{البلية } ne veut pas dire fléau, mais plutôt malheur. \newline
Les traductions tirées du corpus parallèle ne correspondent pas au contexte de l’adage.
\\
\midrule
\textlang{arabic}{وهل يصلح العطار ما أفسد الدهر} \newline
Et il réparera ce qui a gâté la dot
&
Remarque 1 : omission de la forme interrogative qui peut être reformulée en phrase déclarative puisqu’il s’agit d’informer et ne de poser une question à laquelle on attend une réponse. \newline
Remarque 2 : omission du sujet et son remplacement par le pronom personnel “il” qui rend la phrase incompressible. \newline
Remarque 3 : faux sens  (\textlang{arabic}{أفسد الدهر}) : \textlang{arabic}{الدهر} veut dire le temps dans ce contexte puisque l’expression a été rendue par gâté la dot. 
\textlang{arabic}{الدهر} a été confondu avec \textlang{arabic}{المهر} \newline
L’adage adapté veut dire : on ne peut réparer ce qui a été endommagé par le temps
\\
\bottomrule
\source{d'après les auteurs.}
\end{longtable}
\end{small}
