% !TEX TS-program = XeLaTeX
% use the following command: 
% all document files must be coded in UTF-8
\documentclass{textolivre}
% for anonymous submission
%\documentclass[anonymous]{textolivre}
% to create HTML use 
%\documentclass{textolivre-html}
% See more information on the repository: https://github.com/leolca/textolivre

% Metadata
\begin{filecontents*}[overwrite]{article.xmpdata}
    \Title{Desenvolvimento de competências por meio das TIC e formação de professores de Música: uma experiência biográfica}
    \Author{Fernando José Sadio-Ramos \sep María Angustias Ortiz-Molina  \sep María del Mar Bernabé-Villodre}
    \Language{pt-BR}
    \Keywords{Competências \sep Criatividade \sep Educação Musical \sep Investigação biográfico-narrativa \sep Professores \sep TIC} 
    \Journaltitle{Texto Livre}
    \Journalnumber{1983-3652}
    \Volume{14}
    \Issue{1}
    \Firstpage{1}
    \Lastpage{16}
    \Doi{10.35699/1983-3652.2021.25419}

    \setRGBcolorprofile{sRGB_IEC61966-2-1_black_scaled.icc}
            {sRGB_IEC61966-2-1_black_scaled}
            {sRGB IEC61966 v2.1 with black scaling}
            {http://www.color.org}
\end{filecontents*}

\journalname{Texto Livre: Linguagem e Tecnologia}
\thevolume{14}
\thenumber{1}
\theyear{2020}
\receiveddate{\DTMdisplaydate{2020}{9}{22}{-1}} % YYYY MM DD
\accepteddate{\DTMdisplaydate{2020}{12}{29}{-1}}
\publisheddate{\DTMdisplaydate{2021}{2}{25}{-1}}
% Corresponding author
\corrauthor{Fernando Sadio-Ramos}
% DOI
\articledoi{10.35699/1983-3652.2021.25419}
% list of available sesscions in the journal: articles, dossier, reports, essays, reviews, interviews, editorial
\articlesessionname{Educação e Tecnologia}
% Abbreviated author list for the running footer
\runningauthor{Sadio-Ramos et al.}
\editorname{Leonardo Araújo}

\title{Desenvolvimento de competências por meio das TIC e formação de professores de Música: uma experiência biográfica}
\othertitle{Developing creativity through ICT and Music Teachers' Training: a biographical experience}
% if there is a third language title, add here:
%\othertitle{Artikelvorlage zur Einreichung beim Texto Livre Journal}

\author[1]{Fernando José Sadio-Ramos\orcid{0000-0001-7654-5638}\thanks{Email: \url{framos@esec.pt}}}
\author[2]{María Angustias Ortiz-Molina\orcid{0000-0003-2857-5992}\thanks{Email: \url{maortiz@ugr.es}}}
\author[3]{María del Mar Bernabé-Villodre\orcid{0000-0001-8983-6602}\thanks{Email: \url{maria.mar.bernabe@uv.es}}}

\affil[1]{Escola Superior de Educação, Politécnico de Coimbra, Coimbra, Portugal.}
\affil[2]{Universidad de Granada, Granada, España.}
\affil[3]{Universidad de Valencia, Valencia, España.}

\addbibresource{article.bib}
% use biber instead of bibtex
% $ biber tl-article-template

% set language of the article
\setdefaultlanguage[variant=brazilian]{portuguese}
\setotherlanguage{english}

% for spanish, use:
%\setdefaultlanguage{spanish}
%\gappto\captionsspanish{\renewcommand{\tablename}{Tabla}} % use 'Tabla' instead of 'Cuadro'

\begin{document}
\maketitle

\begin{polyabstract}
\begin{abstract}
O texto faz parte de uma investigação qualitativa biográfico-narrativa que
temos em curso, e que se processa mediante a obtenção de testemunhos de
professores e investigadores que formam professores com o auxílio das TIC de
modo a promover as competências de criatividade, colaboração, cooperação e
comunicação, entendidas como essenciais para o desenvolvimento de uma educação
integral, promotora da sustentabilidade curricular e da cidadania
participativa. Neste artigo, damos conta da experiência de uma professora e
investigadora, Cristina Arriaga Sanz, que se dedica à formação de professores
no âmbito da Educação Musical no ensino superior espanhol, de que salientaremos
a relação estreita com as aportações teóricas que apresentamos e com resultados
anteriores obtidos por meio da nossa investigação.

\keywords{Competências \sep Criatividade \sep Educação Musical \sep Investigação biográfico-narrativa \sep Professores \sep TIC}
\end{abstract}

\begin{english}
\begin{abstract}
The text is part of a qualitative biographical-narrative investigation in
process, which proceeds by obtaining testimonies from teachers and researchers
who train teachers with the help of ICT to promote the skills of creativity,
collaboration, cooperation, and communication, understood as essential for the
development of comprehensive education, promoting curricular sustainability and
participatory citizenship. In this article, we report on the experience of a
teacher and researcher, Cristina Arriaga Sanz, who dedicates her labor to
teacher training in the field of Musical education in Spanish Higher Education,
of which we will emphasize the close relationship of its contents with the
theoretical contributions we present and with previous results obtained through
our investigation.

\keywords{Biographical-narrative investigation \sep Competencies \sep Creativity \sep ICT \sep Musical education \sep Teacher}
\end{abstract}
\end{english}

% if there is another abstract, insert it here using the same scheme
\end{polyabstract}


\section{Introdução}\label{sec-intro}
A situação que vivemos, em virtude da crise sanitária mundial provocada pelo
vírus SARS-CoV-2, trouxe à luz do dia uma crua realidade educacional: as
dificuldades no uso das Tecnologias da Informação e da Comunicação (TIC, de
aqui adiante) por parte da generalidade dos professores. É certo que poderíamos
falar de muitos outros problemas derivados desta situação, não somente de
natureza educativa, mas também de integração da diversidade \cite{rodriguezdelrincon2020},
por exemplo, tema que se estenderia todavia para lá dos interesses visados
neste documento; não obstante, alguns investigadores já estão a alertar para o
que poderá implicar o prolongamento desta digitalização da educação: brechas
sociais pelo acesso desigual aos recursos tecnológicos \cite{almazan2020,jimenez2020,rogerogarcia2020}
e, inclusivamente, o fim da escola tal como a
conhecemos face ao sistema em-linha \cite{feitoAlonso2020}.

Os problemas da gestão não-presencial do processo formativo do futuro graduado
universitário e os problemas no controle virtual do processo de
ensino-aprendizagem dos alunos das etapas educativas obrigatórias derivam dos
apuros em que se veem os professores ao terem que passar para um ensino virtual
como única possibilidade de garantir a continuidade da educação dos seus alunos
\cite{gomezgerdel2020} e, portanto, fazê-lo mediante a utilização de diversas
ferramentas tecnológicas. Os professores veem como um desafio a adaptação da
sua prática docente \cite{colasbravo2017,sanchezmendila2020}, pois têm que aprender a
utilizar materiais digitais que supõem a adoção de diferentes tipos de
apropriação dos conteúdos disciplinares \cite{CaramsBeltrn2019}.
Como consequência, os professores tornaram-se mais conscientes da sua melhor ou
pior competência digital, a qual se tornou fundamental para o desenvolvimento
do seu trabalho docente \cite{trujillo2020,resolucion2020} no
que é visto como um processo forçado \cite{villafuerte2020},
mas necessário.

Desde o surgimento e a ampla difusão das TIC, têm sido muitas as investigações
relacionadas com a sua aplicação nas distintas etapas e matérias educativas
\cite{gallego2009,espana2013,romeroMartn2017,garridomiranda2018,CasalOtero2018,rodriguez2019,padillahernndez2020}.
As mesmas têm patenteado
a falta de coerência entre o que o Espaço Europeu de Educação Superior exige e
o que sucede na Espanha: o “saber fazer” e o “aprender a aprender”. Os alunos
deveriam desenvolver essas competências, o que exigiria que os próprios
professores adquirissem uma série de competências prévias que lhes
possibilitassem que eles próprios “saibam fazer” e sejam capazes de “aprender a
aprender”, mediante o uso das TIC como novas formas de proporcionar uma valiosa
mudança do processo de ensino-aprendizagem \cite{lopes2018}. E é de referir
que o uso dessas ferramentas tecnológicas não deveria ser marcado ou
determinado pela situação de saúde originada pela COVID-19; a presença ativa
das TIC na vida dos alunos (e dos próprios professores) tornou necessário o seu
domínio e controle por parte dos professores \cite{roldan2016} antes da pandemia.
Falaríamos do desenvolvimento de competências básicas \cite{galera2011}
relacionadas com o Espaço Europeu de Educação Superior: a comunicação oral, o
trabalho em equipe \cite{alonso2018}, a criatividade e as
competências digitais para introduzir as tecnologias na prática diária
\cite{colasbravo2017}. Isto é, torna-se necessário que a
formação dos futuros professores nas TIC, compreendidas como motoras de muitos
setores sociais \cite{rodrguezcorrea2017}, os leve a transformar a prática
educativa e integrar as TIC mais além do mero nível básico \cite{CasalOtero2018}. 
\textcite{ColsBravoBravo2019} consideram que, se os professores são competentes digitalmente, podem
desenvolver a competência digital nos seus estudantes.

Autores como \textcite{alonso2018} insistiram na importância das
novas formas que as TIC trouxeram ao processo de ensino-aprendizagem, ao
possibilitarem o desenvolvimento de uma perspectiva construtivista do mesmo
\cite{casanova2016}, vista como fundamental para o “aprender a aprender”,
mas também para o “saber fazer”. A competência digital não deveria ser,
todavia, conformada unicamente pela aquisição das destrezas tecnológicas
\cite{palaumartn2019}, mas teria de se relacionar com as restantes
competências curriculares \cite{escudero2018}.
Poder-se-ia dizer, então, que as tecnologias mudaram a forma de entender o
processo educativo \cite{colasbravo2017}, provocando uma crise
do paradigma tradicional \cite{casanova2016} e dando lugar a um câmbio de
paradigma educativo \cite{rodriguez2019}. Neste, as TIC
converteram-se num eixo transversal para a formação nos distintos níveis
educativos \cite{marques2004}, gerando oportunidades de recriação e solução de
problemas para os alunos atuais \cite{rodriguez2012}, de quem se
chegou a considerar as TIC como um signo de identidade \cite{cozargutirrez2015}.

Tudo isso leva à necessidade de promover o uso educativo das TIC nas/ a partir
das matérias curriculares universitárias \cite{castaneda2019}. Essa promoção
assume uma grande relevância pois as TIC podem considerar-se como
imprescindíveis no âmbito educativo ao plasmarem as mudanças sociais atuais
\cite{dezlatorre2018}, sobretudo na actual situação pandémica em que as TIC são um
recurso fundamental para manter o funcionamento da sociedade \cite{beltran2020}. 
Chegamos, assim, à concretização da área de estudo desta investigação:
que utilização das TIC se faz na formação inicial de professores de Música e
como isso se pode repercutir no desenvolvimento de competências dos seus
futuros alunos.



\section{Estado da questão}\label{sec-estado}
\subsection{Formação musical e formação tecnológica nos cursos de professores}\label{sec-formacao}
Trabalhar com as TIC nas aulas de Música da Educação Básica e Secundária é uma
realidade amplamente demonstrada \cite{martin1992,tejadaGimnez2004,castanon2012,vargas,padula2016,espigares2018,perezdiaz2019};
no entanto, a pandemia veio revelar importantes
carências tecnológicas dos professores universitários, que se defrontaram
inclusive, com dificuldades na adaptação às exigências derivadas da não
presencialidade nas aulas. Realmente, se se tiverem em conta investigações
prévias, já é possível encontrar demonstrada a falta de adaptação à exigência
tecnológica derivada da deficitária formação inicial nas TIC \cite{vaquer2012,dezlatorre2018};
todavia, ainda eram muitos os especialistas que
vinham insistindo na possibilidade de as integrar na docência musical
\cite{tourinan2018} tendo em consideração o seu enorme potencial para o ensino
musical \cite{colasbravo2014}. De fato, já havia sido
demonstrado como a difusão e o consumo musical estão fortemente relacionados
com o desenvolvimento das TIC \cite{cozargutirrez2015}, o
que tornaria recomendável a sua incorporação nas aulas.

Nas salas de aula universitárias em que se formam os futuros professores de
Educação Básica já haviam sido realizadas diversas experiências, de que se
destacam as de: \textcite{galera2011}, que mostraram como a inclusão das TIC
facilita a aprendizagem musical e a criatividade; \textcite{dezlatorre2018}, que mostrou como
as carências formativas iniciais dos professores de Música os podem levar a uma
utilização das TIC desigual e escassa nesse nível educativo; \textcite{cozargutirrez2015}, 
que mostraram como se torna necessária uma
competência musical relativa ao uso das TIC; \textcite{palazonherrera2016}, que
considerou que, na educação musical, as TIC deveriam ser um recurso de apoio; e
\textcite{loresgomez2019}, que mostraram um défice na competência digital
dos professores de Música originado numa formação inicial afastada das
necessidades reais dos professores.

\textcite{hepp2002} assinalou que o aumento do interesse pela formação tecnológica dos
professores de Música está determinado pela facilidade de acesso às partituras
e aos programas de composição em conjunto com a motivação que isto supõe para
os alunos. Devem, também, ter-se em conta as considerações de \textcite{tourinan2018},
que mostrou como com a inclusão das TIC na aula de Música se poderá desenvolver
o pensamento crítico, por exemplo.

A mudança em direção a uma maior predisposição e utilização das TIC em Música
\cite{dezlatorre2018} seria determinada pelas crenças e autopercepções sobre as
próprias capacidades dos professores \cite{colasbravo2014}. Os
professores mostram-se cada vez mais interessados na sua formação contínua para
trabalharem com os seus alunos com este tipo de ferramentas \cite{vaquer2012}, 
sobretudo naqueles casos em que não tenham podido aceder
inicialmente a essa formação, pelo que uma formação permanente poderia terminar
com a problemática da resistência às TIC \cite{colasbravo2017}.
Seria necessário que as TIC começassem a ser vistas como ferramentas que
dão resposta às exigências culturais dos alunos \cite{romero2004}, que permitem
gerar conhecimentos, enriquecer a aprendizagem e a criação \cite{CaberoAlmenara2019}.



\subsection{A criatividade “tecnológica” na formação dos professores de Música}\label{sec-criatividade}

A criatividade é uma capacidade inata \cite{lago2012}, uma
qualidade ou característica que todos devem possuir \cite{torres2017} e que pode
desenvolver-se em todos os alunos se, desde os primórdios da sua formação, se
facilitam ferramentas para o seu treino \cite{vecina2006}. Se se tiverem em conta
estas considerações, a formação inicial dos professores deverá levá-los a
desenvolver as suas próprias capacidades criativas para oferecer aos alunos um
processo de ensino-aprendizagem através do qual eles possam, por sua vez,
chegar a ser criativos. Em tudo isto, as TIC podem ter muito que dizer, podendo
chegar a falar-se numa mediatização do vínculo criativo graças ao
desenvolvimento tecnológico \cite{padula2016}; a motivação dos alunos será sempre,
não obstante, um fator determinante para a aparição da criatividade \cite{gomezcantero2005}, 
acima de qualquer outro.

A criatividade deve desenvolver-se mediante a motivação e a estimulação
\cite{torres2017}, pontos fundamentais sobre os quais os professores têm muito que
dizer. \textcite{galera2011} mostraram como a orientação do processo
formativo do futuro docente de Música a partir das TIC pode levar ao fomento de
atitudes criativas para com os seus futuros estudantes do Ensino Básico. Estes
autores partiram de propostas colaborativas, numa linha similar à de Vargas,
\textcite{vargas}, que utilizaram experiências compositivas
colaborativas com as TIC para mostrar a retroalimentação que se produziria
entre os alunos, em claro contraste com a composição tradicional. Nessa mesma
direção, \textcite{acosta2019} mostraram como as TIC
possibilitam aprendizagens colaborativas e, por isso, é necessário estimular a
utilização das TIC durante a formação inicial de professores.

Essa formação inicial dos professores de Música carece das TIC para mostrar a
importância da articulação de matérias artísticas com competências comuns
\cite{martinpinol2016}. Assim, se poderá
chegar a tomar consciência de como se podem definir novos recursos didáticos
para as aulas, a partir da consideração de que a formação digital e a
criatividade estão ligadas.


\section{Metodologia}\label{sec-metodologia}

A investigação, que temos em curso e da qual damos conta neste artigo de mais
um dos seus produtos, é de natureza qualitativa e recorre a entrevistas em
profundidade como procedimentos metodológicos e instrumentos
biográfico-narrativos \cite{goodson2016,huhn2009,landinmiranda2019,edwards2013}.

O principal objetivo é o recolhimento e análise de experiências de formação
partilhadas por professores e investigadores do ensino superior que utilizem as
TIC como um instrumento importante na formação que desenvolvem, tendo em vista
o desenvolvimento nos seus alunos das competências de criatividade,
colaboração, cooperação e comunicação, consideradas como essenciais para a
realização de uma educação integral \cite{ramos2018,comenius1966,fabre2006,reboul1982,freire1998}
fundamentada na sustentabilidade curricular e
dirigida para o desenvolvimento de uma cidadania responsável e participativa
\cite{crue2005,cer1994,copernicus2011,copernicus2020b,fernandes2018}.
A noção de educação integral que funciona como referencial
teórico mais amplo assenta-se numa concepção personalista do ser humano e do
professor, visto como ser prático, dialógico e histórico, bem como em uma visão
relacional e dialógica da educação \cite{ramos2018}. Essa concepção personalista
do ser humano e seu caráter prático estão na base na nossa opção pelo uso de
metodologias qualitativas, pois estas permitem a busca e apreensão do sentido
que os agentes sociais dão às suas ações \cite{ramos2018}. De acordo com o caráter
descritivo e interpretativo da pesquisa qualitativa, não pretendemos
desenvolver perspectivas e procedimentos de ensino suscetíveis de serem
generalizados ou universalizados, mas sim sublinhar o estatuto pessoal,
individual e exemplar dos professores/investigadores entrevistados, e logo a
possibilidade de analisar e entender outras experiências biográficas.

O problema da investigação é o de determinar que importância e estatuto assume
o uso das TIC na compreensão e prática profissional dos participantes do
estudo, tal como é visível nas suas narrativas biográficas. Com base nas
opiniões e percepções recolhidas, reconstruímos as biografias profissionais dos
professores e procuramos responder ao problema colocado.

O objeto de estudo da investigação é o corpus constituído pelos discursos que
as entrevistas permitem obter e reconstituídos posteriormente. No caso deste
artigo, podemos aceder aos resultados possibilitados pela entrevista em
profundidade realizada com a professora e investigadora Cristina Arriaga Sanz,
da Universidade do País Basco, que partilha a sua experiência de formação e
investigação no campo da formação de professores.

Como ocorre com outros participantes desta investigação, a entrevistada
autorizou a divulgação da sua identidade. O carácter público do seu trabalho
poderia permitir a sua identificação.

A entrevista foi realizada com recurso ao \textit{software} Skype, no dia 8 de maio de
2020. Da gravação efetuada, procedeu-se à transcrição da entrevista, cujo texto
foi validado por Cristina.

O protocolo do instrumento utilizado coloca em pauta a questão ética do
anonimato e da confidencialidade das informações e possibilita a obtenção de
considerações biográficas e curriculares da entrevistada. Assentou-se, por sua
vontade, na divulgação do nome, reservando a confidencialidade da gravação em
bruto (garantida finalmente através da sua destruição) e do texto transcrito,
extraindo deste as informações necessárias à reconstituição da sua experiência
biográfica. A assinatura da declaração de consentimento informado encerrou o
capítulo do cuidado ético da investigação. No que concerne às perguntas, é
importante referir que são colocadas de forma temática e aberta, abrangendo as
áreas pertinentes para o tratamento dos aspectos biográficos relevantes para a
temática geral da investigação.

Finalmente, o tratamento do discurso faz-se por meio da análise textual e
indutiva do discurso da entrevista, pondo em relevo as percepções e opiniões
presentes no mesmo.


\section{Resultados}\label{sec-resultados}
\subsection{Sinopse biográfica e curricular de Cristina}\label{sec-sinopse}
A primeira seção de resultados que passamos a apresentar diz respeito à visão
sinóptica biográfica e curricular de Cristina. A título introdutório,
destaque-se a sua disponibilidade para participar deste projeto e partilhar sua
experiência pessoal, contribuindo para que se conheçam as histórias de
professores e, através delas, se aceda aos sentidos da profissão.

De nacionalidade espanhola, com 56 anos de idade, é Doutora em Filosofia e
Letras com o doutoramento sobre Educação Musical. Possui o Diploma Superior de
Pedagogia Musical de Conservatório Superior, assim como de outras habilitações
conexas, a saber, o Curso de Solfejo, o Curso Profissional de Piano e o Curso
de Jornalista no ramo de Publicidade.

Trabalha atualmente na Universidade do País Basco, na Faculdade de Educação de
Bilbau. Tem 36 anos de tempo de serviço, com os últimos 22 dedicados ao Ensino
Superior. No seu percurso anterior a este nível de ensino, foi professora de
Conservatório e de Música no Ensino Básico Obrigatório, com um amplo currículo.

Cristina começa por falar da sua atividade docente. Na Universidade, têm-lhe
sido atribuídas inúmeras disciplinas, “tantas que nem me recordo”, no
curso de Bacharelado em Educação Musical, dirigido a alunos “que vão trabalhar
como Professores de Educação Básica e Educadores de Infância”, assim como em
diversos cursos de \textit{Máster}, de que destaca o de \textit{Máster em
Psicodidática}, curso que “não é específico de música”; neste, leciona
disciplinas relacionadas com a música e a criatividade. Assinala, também, a
disciplina da Menção de Música da formação de professores e educadores sobre
\textit{A influência dos meios audiovisuais na perspectiva de género}.

Destaque especial merece-lhe o seu trabalho em \textit{Formação Instrumental}
“que é a disciplina com que mais tenho trabalhado” de forma contínua até à
atualidade; “\ldots as outras – como disse – dei uns anos uma, outros anos
outra\ldots”. Essas mudanças frequentes devem-se sobretudo a “situações da
universidade”. A par da docência na Universidade, leciona também em cursos de
formação de professores e educadores organizados pelo Governo Basco.

Quanto ao seu trabalho como investigadora, indica que o seu “primeiro trabalho
importante de investigação foi a tese (Tese Doutoral), que comecei a fazer
quando entrei na universidade” e após terminar os cursos de doutoramento, tese
intitulada “\textit{La motivación para estudiar música en Educación Primaria}”,
cuja linha de trabalho sempre a interessou.

Tem integrado equipas de investigação, de que destaca a ligação passada a “um
grupo com companheiros de outras universidades, UniTICArte”, e a ligação a
outro grupo da Faculdade de Belas Artes da sua Universidade, dedicado à
“investigação sonora”.

No âmbito do grupo UniTICArte, integrador de quatro docentes – “Alberto Cabedo
(Universidade de Castellón), Noemy Berbel (Universidade das Ilhas Baleares),
María Elena Riaño (Universidade da Cantábria) – realizaram-se “bastantes
coisas” \cite{cabedo2017}. Destaca “o Projecto
I+D que tivemos sobre Património”, assim como o seu trabalho conjunto
interuniversitário com a utilização das TIC numa perspectiva multidisciplinar.
Dessa colaboração, resultou um amplo conjunto de produtos, apresentado em
eventos científicos, e que originou “alguns artigos interessantes” (uns, já
publicados, outros aguardando a sua publicação), assim como a criação de
“alguns materiais muito enriquecedores que eu utilizo muito e que deram lugar a
algumas publicações” bem como um prêmio SIMO.

Atualmente, integra um grupo da sua Faculdade, o “KideOn, centrado na inclusão”
e colabora com outro grupo da Universidade Jaime I de Castellón, o “EDARSO,
Comunicação, Arte e Sociedade”, com os quais “estou atualmente a trabalhar num
projeto que estuda ‘A influência do conjunto instrumental na convivência
positiva e na resolução de conflitos’”. Trata-se de um projeto I+D financiado
pelo MINECO – Ministério da Economia, Indústria e Competitividade, do Governo
do Reino de Espanha, promovido pela Universidade Jaime I de Castellón, e que
“se intitula ¡Musiquemos! Fazendo música comunitária nas escolas”. A ideia
fundamental do projeto é a de trabalhar “com a música e a participação em
conjuntos instrumentais para a melhoria da convivência positiva e das relações,
pois pensamos que a música e, sobretudo, o conjunto instrumental têm muitos
benefícios” para tal.

Termina esta seção declarando “continuo a trabalhar sobre a motivação, e também
tenho alguns artigos aceitos para publicação, a maioria relacionados com o
projeto da Universidade Jaime I de Castellón” e conclui: “Bom, bastante
trabalho!”


\subsection{As TIC na investigação e docência de Cristina}\label{sec-tic}
Cristina põe em relevo, para começar, que considera as TIC como “uma das pedras
angulares das aulas” de qualquer disciplina no que concerne aos produtos e
explorações que os alunos fazem e ao desenvolvimento da criatividade e de
competências, em geral. A formação para o ensino da educação musical beneficia
particularmente do contributo das TIC a aprendizagem individual e grupal;
propicia a exploração pois “permite-te gravar coisas, permite-te escutá-las,
permite-te trabalhar logo com elas, moldá-las, modificá-las, intercambiá-las,
criar efeitos, arranjos, criar obras originais, reúne um montão de
possibilidades”; é também um poderoso instrumento no que diz respeito à
“comunicação com outros grupos de trabalho”.

Um segundo aspecto abordado relaciona-se com a sua investigação sobre a sua
atividade e experiência docente.

Cristina refere-se inicialmente aos trabalhos desenvolvidos com o grupo
UniTICArte, para os quais “as TIC foram fundamentais” e a “pedra angular”, já
que o seu “eixo era a criatividade e as TIC (as duas coisas) e o trabalho
colaborativo”. Esta associação permitiu investigar de “forma multidisciplinar”
assuntos relacionados à música e à cultura, ao mesmo tempo que se entabulava
relação com a comunidade e se intercambiavam culturas e experiências entre
diferentes universidades e os próprios alunos.

Atualmente, desenvolve trabalhos de ordem autoscópica para o desenvolvimento
das competências docentes dos futuros professores, para os quais considera
essencial o contributo das TIC. A troca de informação e o registro de
atividades que permitem depois o “observar-te, observar o que se passou na aula
é muito importante nos cursos que realizamos e no trabalho que fazemos com o
grupo de Castellón”; que “os alunos possam gravar-se, ver-se, observar-se,
registar as suas opiniões, que eles mesmos possam manipular se necessitam de
espremer alguma informação recolhida”, tal deve muito à presença das TIC.

A participação e o envolvimento dos alunos no trabalho é destacada em seguida.
Assinala que os alunos se sentem muito motivados e implicados, crescendo com
estas atividades. “A aceitação das TIC é tremenda, creio que é um elemento
motivador importante”. A abertura à sua utilização leva Cristina a considerar
que, “em geral, o processo e o produto são muito variados, mas há alguns
trabalhos que são muito bons. No projeto UniTICArte há alguns trabalhos muito
interessantes”.

Há, todavia, que referir que as aulas apresentam diversos ritmos; os alunos são
jovens muito diferentes, apesar da proximidade das suas idades (“20, 21 e 22
anos”). Relativamente aos programas musicais, para a maioria, são realidades
novas e “aprendem connosco; eu creio que aprenderam nas minhas aulas muitos dos
programas que temos utilizado ligados com a música, não ligados com fazer uma
montagem ou com outras coisas; então, pois, é logicamente diferente o nível de
progresso”.

Os bons resultados obtidos não ficam “só no projeto UniTICArte, por exemplo,
nessa disciplina de música que estamos a fazer agora, (\ldots) sobre a perspectiva
de gênero nos meios audiovisuais, pois estão a fazer também uns trabalhos com
instalações, com gravações que recolhem coisas muito inovadoras além das TIC,
através das TIC fazem o produto final, mas com ideias muito inovadoras;
mediante as TIC as montagens que fazem são muito interessantes”.

Outro aspecto poderoso da utilização das TIC diz respeito à disseminação dos
trabalhos realizados. Cristina considera que “o trabalho é muito positivo,
normalmente em quase todas as disciplinas há que elaborar o produto final e
logo temos a oportunidade na aula de vê-lo e comentá-lo”. Isso é alvo de um
grato acolhimento por parte dos alunos, muitas vezes surpreendidos com os seus
trabalhos, o que “é um elemento que faz com que os grupos se envolvam e a
aceitação é elevada”.

Os programas com que trabalha são gratuitos, “mas [os alunos] continuam a
procurar” e, embora alguns programas que encontram não sejam gratuitos,
adquirem-nos “porque querem ir mais além\ldots”. Incluem esses programas nos
concertos “– porque nas minhas aulas costumamos fazer sempre um concerto –
(vamos a uma escola ou vêm eles e fazemos um concerto)”.

As experiências que teve são de diversa ordem, mas revelam competências de
resolução de problemas de assinalar e de produção de trabalhos muito atrativos.
Já “me aconteceu em alguns anos – nem sempre porque isto já é muito avançado –
que no próprio [concerto em] direto os alunos utilizam as TIC, claro, logo
noutras vezes temos que chegar à escola e adaptar-nos ao que há porque em
muitas escolas há carências e no melhor podemos demorar uma hora a preparar-nos
e fazer a montagem, mas, muitas vezes, é um aspecto que me surpreende
gratamente [sic]”.

Abordando o tema das TIC que utiliza, Cristina tece uma consideração prévia
para situar a sua resposta. “Bom, na universidade estamos a adaptar-nos, porque
a verdade é que não temos demasiados meios já que temos muitos alunos e chegar
a todos eles com as tecnologias não é simples; há aulas de 50 pessoas que ainda
por cima trabalham num espaço reduzido embora normalmente da minha aula sempre
saem, procuramos outros espaços no próprio edifício ou às vezes também fora [do
edifício]”.

Uma vez terminada esta consideração, refere que “normalmente, propomos
programas que são gratuitos, sejam de edição de som como de edição de vídeo e
áudio”. A aprendizagem da utilização deles faz-se sobretudo com tutoriais, que
se dão aos alunos; “tão pouco explicamos muito na aula, entendemos que hoje em
dia com um tutorial e com a aprendizagem entre iguais – que é algo que
praticamos muito – se pode entrar em ação mais rapidamente do que estar a
explicar os detalhes de programas, é mais dar umas noções que eles ponham em
prática”.

Enumerando os produtos utilizados, indica em primeiro lugar que recomendam
programas gratuitos como \textit{Audacity} e \textit{Movie Maker}, “porque me
parece que não temos que propor-lhes um programa pago, aí tenho as minhas
dúvidas (\ldots) porque sempre sai alguém\ldots ou gente que tem a tecnologia, logo
os \textit{Apple} que também servem e produzem o \textit{iMovie}, depende, mas
são eles que costumam obter o \textit{Adobe Premiere Pro}, o \textit{Filmora},
que também usam muito”. Utilizam também o \textit{Makey Makey}.

No âmbito dos dispositivos tecnológicos, refere também o uso de uma ampla gama
de artefctos, “de microfones de contacto, de multi-conectores, para que aquilo
que gravam possam eles escutá-lo na aula sem incomodar o colega do lado; temos
gravadores, temos alguns microfones, temos pedais\ldots”. Refere a falta de
computadores, que leva a que se solicite normalmente aos alunos “que tragam
eles os computadores para a aula, que se revezem, etc. Para não mudarmos de
sala para sala por causa da tecnologia\ldots, por isso lhes dizemos que tragam
eles os computadores e normalmente trazem-nos”, assim como outros equipamentos
que sejam necessários caso a caso. “Damos, também, muita liberdade” para que
proponham alternativas tecnológicas para potenciar os seus trabalhos.

As competências desenvolvidas/mobilizadas por estas práticas revelaram-se de
particular utilidade na situação de confinamento geral provocado pela COVID-19.
Cristina refere a situação vivida na sua Universidade. Nos meses de março a
julho de 2020, a sua experiência formou-se a partir dos alunos que estavam em
processo de estágio nas escolas e que ela orientava no seu trabalho final de
licenciatura. A sua intervenção consistiu em animá-los a que continuassem o seu
trabalho com a utilização das TIC para superar as grandes dificuldades que se
apresentaram às famílias e escolas. A falta de acesso das famílias às escolas
tornou o processo de ensino-aprendizagem bastante complicado; havia “que
preparar – para qualquer eventualidade – as coisas também em papel e deixá-las
na escola para que os país as fossem buscar\ldots ou em gravações”. A brecha
socioeconômica fez-se sentir: “Nas escolas em que o poder aquisitivo das
famílias é mais elevado, pois bem, mas nas outras escolas, tiveram que fazer
rendas de bilros, dar às crianças alguns desses computadores pequenos que se
distribuem\ldots bom, problemas. Vamos andando para a frente com desafios, as
respostas não são uniformes, os pais – entendo – que tão-pouco podem estar
todos em cima\ldots. tenho um que se ligou a um programa de televisão!!!, e então à
quinta-feira, às 11:30h, vêm ali dois professores dar uma aula para quem
quiser, de música e não sei quê!”.

Constata, assim, que as pessoas, que sempre tiveram muita imaginação, revelam
“agora ainda mais. Porque eu creio que aos professores de música nos pedem
muita imaginação e agora temos que ter ainda mais\ldots Imaginação!!!”

Cristina considera em seguida se os resultados obtidos com estas tecnologias
têm valor acrescido relativamente ao que se obtinha antes da sua
disponibilidade mais geral, com instrumentos tecnológicos mais rudimentares e
primitivos.

A primeira referência surge de forma admirativa: “Homem, eu, a mim parece-me,
por exemplo, que agora se me tornaria muito estranho dar uma aula em que não se
usassem as TIC, para o que quer que seja, porque o uso das TIC é muito amplo”.
Dando exemplos da sua prática conta que “nas minhas aulas, por exemplo, utilizo
muito a linguagem \textit{Soundpainting}, que é uma linguagem de signos com que
tu comunicas com os alunos através de signos, de gestos, e se cria música no
momento; é uma linguagem de tu a tu e não utiliza as TIC, mas eu a utilizo em
projetos que temos feito noutros anos para que os meus alunos os vejam,
peço-lhes que vão às escolas a pô-los em prática e que se gravem e na aula
vemos o resultado, logo passamos as coisas que fizemos, ou estamos a fazer, aos
alunos das escolas, nisto utilizamos as TIC”.

Deste modo, as TIC constituem-se “um complemento que enriquece as aulas”, mas
devemos referir que “a mim, parece-me que o contacto face-a-face e a
afetividade, o diálogo, o poder estar com a pessoa e vê-los e senti-los (sic) é
muito importante”. As TIC podem até auxiliar “no ensino tradicional, que tem
sido mais baseado na repetição, a qual pode estar bem para determinados
momentos da aprendizagem musical”; Cristina refere que através das TIC “podes
pegar num programa que te ajude a aprender\ldots, bom até tens tutoriais para
aprender a tocar um instrumento\ldots”. Conclui esta seção sublinhando que “as
TIC bem utilizadas, sempre que – bom, no meu ponto de vista – a autonomia dos
alunos exista, o diálogo entre alunos e professores exista, são enriquecedoras
para o ensino”.

Cristina passa a referir em que medida se mantém o contato com os alunos, uma
vez formados, para obter deles informações acerca das suas experiências
laborais e a importância da formação recebida para essas atividades. A reflexão
acerca da sua aprendizagem ocorre logo quando se formam, ao elaborarem “em
portfólio individual de reflexão (\ldots), no qual têm que explicar um
pouco o que aprenderam”. Mas há que falar no contato que mantém, “ou porque fez
algum curso, (\ldots) ou porque muitas vezes fui às suas aulas falar das coisas
que faço e também os convidei a vir às minhas aulas para contarem o que estão a
fazer. Sim, sim tenho contacto; homem, claro, são tantos, tantos anos\ldots”
Não é fácil e, se pudesse, “faria mais, mas creio que a vida do professor de
música é muito dura nas escolas porque tem tantos alunos e tantos anos que
muitas vezes marcar algo com eles é complicado; mas sim, sim, eu creio que
costumo acercar-me na medida em que posso, procuro contar-lhes coisas que
descobri, e como te disse convido-os a vir às minhas aulas a contar coisas”.

Tema obrigatório, para encerrar a narração, foi a situação covídica. Cristina
confessa o efeito de surpresa que constituíram as classes em-linha, destacando
os “problemas de conexão, a falta de recursos entre os alunos, a
‘frescura’\footnote{No sentido de oportunismo, aproveitamento fácil.}\ldots”, a
que se aliou algum oportunismo em falsas queixas públicas acerca de algumas
instituições. No que concerne à falta de recursos, sublinha o fato de “alguns
estudantes não terem computador a toda a hora já que há momentos em que precisa
ser partilhado com toda a família”.

Na sua experiência, não se deparou com questões de falta de dinheiro, pelo que
supõe que o fato de esses alunos não terem um computador pessoal se deverá a
“prioridades nas suas famílias”. Mas, aqui, as TIC constituem o meio que
possibilitou superar as severas limitações trazidas pelo confinamento geral
imposto à população. Considera, nas suas palavras, “que eu agora tenho sorte
porque os grupos que tenho não são muito numerosos, então arranjamo-nos de tu a
tu e a docência faço-a através de uma plataforma que se chama
\textit{Blackboard Collaborate}, mas eventualmente para as aulas vimo-nos pela
\textit{Zoom}, por \textit{Skype} o por outra que se chama \textit{Jitsy Meet},
com que crias a reunião num momento”.

Recorre a diversas plataformas porque com a Blackboard Collaborate, que “é a
que recomenda a Universidade”, não nos vemos todos a cara e, bom, para mim em
algumas atividades – porque temos tentado ver como interpretar juntos – costuma
haver muitos problemas de atraso\ldots Creio que para a música é uma perda; claro, é
que não vi aqueles a quem estou a dar a aula!”

Desenvolve este ponto, frisando que “Os do ‘um a um’, esses não me importam
porque com eles, bem, maravilhoso, não há praticamente nenhuma mudança, o único
é que não os tive na Universidade, no gabinete, (\ldots) mas claro, para as
aulas\ldots”. A falta de conhecimento pessoal, por exemplo, nas “aulas do
mestrado que estou agora a dar\ldots, a essas pessoas é que não as conheço
fisicamente, e para mim isso é um \textit{handicap}; e não lhes ver a cara na
minha aula, isso para mim é outro \textit{handicap}”.

O último aspecto que partilha da sua experiência diz respeito ao fato de estar,
atualmente, “a tentar fazer música em direto”, experiência que não está sendo
muito positiva dadas as dificuldades técnicas e logísticas que surgem: “Temos
problemas; temos problemas de atraso, a qualidade do som que não é boa, alguns
dias ligam-se todos perfeitamente, logo de repente há dias em que um não se
pode ligar, que outro não sei quê\ldots, a uma não lhe funciona o auto-falante,
outra comunica por \textit{chat}, eu, no outro dia a uma tive que lhe ligar por
telefone e comunicava então pelo auto-falante do telefone”.


\section{Discussão e conclusão}\label{sec-conclusao}
Chegados a este ponto, para concluir a apresentação das perspectivas educativas
de Cristina, é de assinalar a convergência verificada entre o conteúdo da sua
percepção, dos fundamentos pedagógicos e os dados aportados na fundamentação
teórica e empírica do estudo aqui apresentado, assim como em outros trabalhos a
que procedemos, publicados e/ou em via de publicação.

Assim, assinale-se o reconhecimento e assumpção da notável capacidade das TIC
para o desenvolvimento de competências dos alunos em formação superior em
educação musical (criatividade, colaboração, cooperação, participação e
comunicação). Essa capacidade é posta em relevo de forma afirmativa e
substancial, realizando, assim, o desejado aquando da criação e desenvolvimento
do EEES.

No entanto, apesar desse reconhecimento e assumpção, não se assiste à
fetichização da tecnologia, seus produtos e derivados, pois o valor do
professor e da relação pedagógica, assim como da reflexão dos alunos, numa
perspectiva integral da educação, são insistentemente destacados. Alguns
instrumentos e atividades pedagógicos utilizados por Cristina são
particularmente adaptados a este último tema, como sejam a prática da
autoscopia e a elaboração de portfólios reflexivos; não basta a utilização das
TIC, importa questionar o sentido das atividades educativas, de forma
consciente e reflexiva. Por outro lado, põe-se em relevo por diversas vezes a
necessária referência relacional que a educação, por muito apoiada que esteja
na pedra angular das TIC, deve comportar e mediante a qual cobra o seu pleno
sentido. Sublinhe-se, assim, o valor relativo das TIC pois, se são
importantíssimas e poderosas para o desenvolvimento da formação, têm que
assentar e exercitar-se na relação dialógica entre os intervenientes no
processo educativo, alunos e professores, em primeiro lugar.

As competências adquiridas e mobilizadas pelos alunos são também assinaladas de
forma notável por Cristina. A participação e envolvimento profundo dos alunos
na aprendizagem e formação, a sua autonomia, criatividade e imaginação têm um
amplo reconhecimento no testemunho biográfico que nos proporciona a nossa
biografada. O potencial das TIC para apoiar a formação e disseminar os seus
produtos é destacado, tornando-as um complemento/ suporte relevantíssimo para o
processo de ensino-aprendizagem. A ampla panóplia de recursos tecnológicos
mobilizados é de assinalar. Mas estes aspectos, de que Cristina dá amplo
testemunho, não deixam de sofrer o efeito de dificuldades de diferentes ordens
técnicas e/ou financeiras, além da situação gerada pela pandemia covídica.
Dificuldades essas que são, todavia, superadas exatamente pelas competências
adquiridas/ mobilizadas pelos alunos e em interação com a docente. Merece
especial referência a constatação da vontade de superar as dificuldades que se
apresentam e o sério e profundo envolvimento nesse processo. Mais uma vez, é de
assinalar a importância da autonomia, criatividade e relação dos alunos para a
resposta aos desafios levantados em situações pedagógicas habituais e/ou
extraordinárias e sua boa resposta.

A dimensão relacional que Cristina foi pondo em relevo traduz-se, também, no
retorno do ensino-aprendizagem efetuado pelos seus alunos mediante o contato
que ela mantém com eles uma vez terminada a formação e estando a desenvolver a
sua atividade profissional. Isso ocorre, não obstante severas e múltiplas
limitações e dificuldades práticas que o exercício concreto da profissão lhes
levanta.

O último ponto desta seção final visa acentuar a estreita e forte imbricação da
atividade docente e investigadora de Cristina, que conduz a produtos de
assinalável importância de que o seu relato dá testemunho e que são
verificáveis nos seus produtos.

Este artigo constitui a segunda publicação que efetuamos no âmbito deste
projeto. Foi antecedido por outro, cuja referência se nos perdoará
\cite{sadioramos2020}, mas que consideramos ser
imprescindível para cruzamento de perspectivas expressas por ambas as docentes
e investigadoras entrevistadas. De aí, também, que para finalizar este trabalho
se assinale a convergência verificada entre ambos os testemunhos biográficos
recolhidos na investigação. Um ponto merece um destaque final, que consiste na
referência de que, em ambos, se procede ao reconhecimento essencial da relação
pedagógica e do papel dialógico dos seus intervenientes, associado ao
contributo fortíssimo que as TIC dão à formação de professores. Nestes dois
trabalhos, falamos no âmbito da Educação Musical. Em próximos trabalhos,
traremos testemunhos biográficos já obtidos, oriundos da Educação Matemática
(Ensino Superior) e da Educação Básica e Secundária.




\printbibliography\label{sec-bib}
% if the text is not in Portuguese, it might be necessary to use the code below instead to print the correct ABNT abbreviations [s.n.], [s.l.] 
%\begin{portuguese}
%\printbibliography[title={Bibliography}]
%\end{portuguese}

\end{document}
