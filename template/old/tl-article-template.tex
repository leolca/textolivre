\documentclass[red]{textolivre}
\usepackage[utf8]{inputenc}
\usepackage[portuguese,english]{babel}

\usepackage{amsmath}
\usepackage{amsfonts}
\usepackage{amssymb}
\usepackage{amsthm}

\usepackage{csquotes}

\usepackage{graphicx} 
\usepackage{booktabs}

% used to create dummy text for the template file
\usepackage{lipsum} 

%\usepackage{setspace} % for \onehalfspacing and \singlespacing macros
%\singlespacing  

\def \tituloportugues {Modelo de artigo para a revista Texto Livre}
\def \tituloingles {Article template for Free Text Journal}
\title{%
    \vspace{-15mm}%
    \selectfont\textbf{\MakeUppercase{\tituloportugues}} \\
    \vspace{2.1mm}
    \selectfont\textbf{\textit{{\MakeUppercase{\tituloingles}}}}%
} 

\date{\today}

\tlauthor{Nome do Autor 1}{Universidade do Autor1, País}{autor1@universidade.edu}
\tlauthor{Nome do Autor 2}{Universidade do Autor2, País}{autor2@universidade.edu}


\begin{otherlanguage}{english}
\begin{abstract}
% write your abstract here
This article is a template article which aims to guide authors who will submit their papers to the Texto Livre journal. This template should facilitate their endeavor, viewing and configuring the text in the format already configured accordingly. Just write or paste your content in the desired places, replacing texts and figures, checking if they keep up the format. The abstract should have between 150 and 200 words.
\keywords{word1; word2; many-words; word3}
\end{abstract}
\end{otherlanguage}

\begin{resumo}
% escreva o resumo aqui
Este artigo é um modelo que visa orientar os autores que submeterão seus textos à revista Texto Livre a fazê-lo no padrão determinado pela revista a fim de facilitar sua empreitada, visualizando e configurando o texto no formato já configurado nesse modelo. Para fazê-lo, basta escrever ou colar seus textos no lugar aqui designado, substituindo os textos e figuras aqui existentes, verificando se eles mantêm a formatação aqui descrita. O resumo deverá ter entre 150 e 200 palavras.
\palavraschave{palavra1; palavra2; muitas-palavras; palavra3}
\end{resumo}

\usepackage[backend=biber,style=abnt, ittitles]{biblatex}
\DeclareLanguageMapping{brazil}{brazil-apa}
\addbibresource{tl-article-template.bib}     
% use biber instead of bibtex
% $ biber tl-article-template
% $ pdflatex tl-article-template.tex


\begin{document}
\selectlanguage{portuguese}
\maketitle

\section{Introdução}
Este artigo é o modelo de documento que visa orientar os autores que submeterão seus textos à Revista Texto Livre. Este documento fornece o padrão determinado pela revista, a fim de facilitar sua empreitada, visualizando e configurando o texto no formato já configurado no modelo para a submissão. Para redigir seu texto neste modelo, basta escrever ou colar seus textos no lugar aqui designado, substituindo os textos e figuras aqui existentes, verificando se eles mantêm a formatação aqui descrita.

O corpo do texto deverá ter um recuo de 1,25 cm, utilizar a fonte “Liberation Sans” em tamanho 12, justificado, com entrelinhado simples formatar. O espaço abaixo de cada paragrafo deverá ser de 0,1cm.

\section{Título do artigo}
O título do artigo deverá ser em caixa alta, negrito e utilizando a fonte “Liberation Sans” com tamanho 12. Abaixo do título em português, virá o título em inglês, com espaço de 0,4cm entre eles. O título em inglês utilizará a mesma formatação, porém deverá estar em itálico. Entre o título em inglês e o resumo do artigo, deverá haver um espaçamento de 1,10cm.

\section{Título de seções}
O título de cada seção deverá utilizar a fonte “Liberation Sans” com tamanho 12, em negrito, numerados com algarismos arábicos. Após o título de cada seção, será deixado um espaço em branco de 0,5cm, separando o título do texto da referida seção. O texto será justificado, com entrelinhamento simples e espaço de 0,10cm abaixo de cada parágrafo.

\subsection{Organização em seções, subseções e subsubseções}
O artigo poderá ser organizado de forma a conter seções, subseções e subsubseções. Em todos os casos, a formação dos títulos e parágrafos destas seguirão as mesmas regras dispostas anteriormente. Ou seja, deverá ser utilizada da fonte “Liberation Sans” em tamanho 12. Os títulos de seções serão numerados com algarismos arábicos e destacados em negrito. Haverá um espaço em branco de 1,00cm acima e 0,5cm abaixo do título de cada seção.

\section{Litas numeradas e não numeradas}
Os artigos poderão conter listas numeradas ou não. Quando não houver numeração, deverão ser marcadas com ponto, conforme o exemplo abaixo: 
\begin{itemize}
\item exemplo 1
\item exemplo 2
\item exemplo 3
\end{itemize}

Quando for necessário uma lista ordenada, deverá, preferencialmente seguir a numeração arábica:
\begin{enumerate}
\item exemplo 1
\item exemplo 2
\item exemplo 3
\end{enumerate}

\section{Inserindo figuras e tabelas}
Nesta seção iremos descrever como inserir figuras no documento em conformidade com o padrão da revista. Sempre que possível, utilize figuras vetoriais para manter uma melhor qualidade e propiciar uma boa apresentação do texto. Use os formatos PDF ou EPS para as imagens vetoriais e os formatos PNG ou JPEG para as imagens rasterizadas. As figuras deverão ficar centralizadas, com legenda abaixo e informação de fonte, caso necessário.

\begin{figure}[htbp]
 \centering
 \includegraphics[width=0.5\textwidth]{example-image-a}
 \caption{Imagem de teste.}
 \label{fig-img-a}
 \source{Fonte da imagem.}
\end{figure}

\lipsum[12]


\begin{table}[htpb]
\caption{Legenda centralizada utilizando fonte Liberation Sans corpo 10. O texto "tabela" deverá ser mantido em itálico.}
\label{tbl-tabela-01}
\begin{tabular}{llp{5.5cm}}
\toprule 
Tabela                      & Coluna 1                       & Coluna 2                                                               \\ 
\midrule
Lorem ipsum                 & Non consectetur                & Leo vel fringilla                                                      \\ 
\midrule
Lorem ipsum dolor sit amet. & Non consectetur a erat nam at. & Leo vel fringilla est ullamcorper eget nulla facilisi etiam dignissim. \\ 
\bottomrule
\end{tabular}
\source{Fonte da tabela.}
\end{table}

\lipsum[3]

\section{Citações e notas de rodapé}
Para inserir citações no texto utilize a formatação descrita nesta seção. 
Citações com mais de três linhas devem possuir um recuo de 4cm, utilizar a fonte 
Liberation Sans maiúscula\slash{}minúscula, corpo 10, justificado, entrelinhado simples, sem aspas. 
Entre parênteses fornecer a referência para a citação.  Ponto final apenas após o fechamento de parênteses. 

\begin{quoting}
Book printing differs significantly from ordinary typing with respect to dashes, hyphens, and minus signs.
In good math books, these symbols are all different; in fact there usually are at least four different symbols (...)
Hyphens are used for compound words like `daughter-in-law' and `X-rated'. En-dashes are used for number ranges like 
`pages 13--34', and also in contexts like `exercise 1.2.6--52'. 
Em-dashes are used for punctuation in sentences---they are what we often call simply dashes. 
And minus signs are used in formulas. A conscientious user of \TeX{} will be careful to distinguish these four usages 
\cite[p. 4]{donaldknuth1984}.
\end{quoting}

Elementum pulvinar etiam non quam lacus. Posuere urna nec tincidunt praesent semper feugiat nibh. 
Nunc sed id semper risus in hendrerit. Fermentum odio eu feugiat pretium. 
Ac tortor dignissim convallis aenean et tortor. 
Elementum integer enim neque volutpat ac tincidunt vitae\footnote{
Notas de rodapé devem aparecer no final da página onde a nota foi utilizada. A fonte utilizada é  Liberation Sans, corpo 10, alinhamento justificado.
}.


\section{Equações matemáticas}
\lipsum[13-14]
\begin{equation}\label{eq-bin}
\binom{n}{k} = \frac{n!}{k!(n-k)!}
\end{equation}
\lipsum[15] (ver Equação \ref{eq-bin} e Tabela \ref{tbl-tabela-01}).

\lipsum[3-5]\footnote{\lipsum[30]}

\begin{equation}\label{eq-plancherel}
\int_{-\infty}^\infty \left| f(x) \right|^2\,dx = \int_{-\infty}^\infty \left| \hat{f}(\xi) \right|^2\,d\xi.
\end{equation}

\lipsum[20-21]
Equação \ref{eq-plancherel} e Tabela \ref{tbl-tabela-01}.

\section{Conclusão}
Para elaborar um texto com melhor qualidade \cite{donaldknuth1984,leslielamport1994}, busque utilizar o modelo para \LaTeX{} também
disponível no sítio da revista.

\lipsum[17-19]

% Referências
\printbibliography[title=Refer\^{e}ncias]

\appendix
\section{Primeiro anexo}
Se houver anexo, este deverá vir após as referências.
\lipsum[1]

\section{Segundo anexo}
\lipsum[2]

\vspace{2cm}
\begin{flushright}
Recebido no dia XX de mmmm de 20YY.\\
Aprovado no dia XX de mmmm de 20YY.
\end{flushright}

\end{document}

