% !TEX TS-program = XeLaTeX
% use the following command:
% all document files must be coded in UTF-8
\documentclass[portuguese]{textolivre}
% build HTML with: make4ht -e build.lua -c textolivre.cfg -x -u article "fn-in,svg,pic-align"

\journalname{Texto Livre}
\thevolume{14}
%\thenumber{1} % old template
\theyear{2020}
\receiveddate{\DTMdisplaydate{2020}{10}{22}{-1}} % YYYY MM DD
\accepteddate{\DTMdisplaydate{2020}{9}{3}{-1}}
\publisheddate{\today}
\corrauthor{Leonardo Araújo}
\articledoi{10.35699/1983-3652.yyyy.nnnnn}
%\articleid{NNNN} % if the article ID is not the last 5 numbers of its DOI, provide it using \articleid{} commmand
\articlesessionname{Artigos}
\runningauthor{Araújo e Pereira} 
%\editorname{Leonardo Araújo} % old template
\sectioneditorname{Daniervelin Pereira}
\layouteditorname{Leonado Araújo}

\title{Modelo de artigo para submissão à revista Texto Livre}
\othertitle{Article template for submitting to the Texto Livre Journal}
% if there is a third language title, add here:
%\othertitle{Artikelvorlage zur Einreichung beim Texto Livre Journal}

\author[1]{Leonardo Araújo \orcid{0000-0003-3884-2177} \thanks{Email: \url{leolca@ufsj.edu.br}}}
\author[2]{Daniervelin Pereira \orcid{0000-0003-1861-3609} \thanks{Email: \url{revista@textolivre.org}}}
\affil[1]{Universidade Federal de São João del Rei, DTECH, Ouro Branco, MG, Brasil.}
\affil[2]{Universidade Federal de Minas Gerais, Faculdade de Letras, Belo Horizonte, MG, Brasil.}

\addbibresource{article.bib}
% use biber instead of bibtex
% $ biber article

% used to create dummy text for the template file
\definecolor{dark-gray}{gray}{0.35} % color used to display dummy texts
\usepackage{lipsum}
\SetLipsumParListSurrounders{\colorlet{oldcolor}{.}\color{dark-gray}}{\color{oldcolor}}

% used here only to provide the XeLaTeX and BibTeX logos
\usepackage{hologo}

% if you use multirows in a table, include the multirow package
\usepackage{multirow}

% provides sidewaysfigure environment
\usepackage{rotating}

% CUSTOM EPIGRAPH - BEGIN 
%%% https://tex.stackexchange.com/questions/193178/specific-epigraph-style
\usepackage{epigraph}
\renewcommand\textflush{flushright}
\makeatletter
\newlength\epitextskip
\pretocmd{\@epitext}{\em}{}{}
\apptocmd{\@epitext}{\em}{}{}
\patchcmd{\epigraph}{\@epitext{#1}\\}{\@epitext{#1}\\[\epitextskip]}{}{}
\makeatother
\setlength\epigraphrule{0pt}
\setlength\epitextskip{0.5ex}
\setlength\epigraphwidth{.7\textwidth}
% CUSTOM EPIGRAPH - END

% LANGUAGE - BEGIN
% ARABIC
% for languages that use special fonts, you must provide the typeface that will be used
% \setotherlanguage{arabic}
% \newfontfamily\arabicfont[Script=Arabic]{Amiri}
% \newfontfamily\arabicfontsf[Script=Arabic]{Amiri}
% \newfontfamily\arabicfonttt[Script=Arabic]{Amiri}
%
% in the article, to add arabic text use: \textlang{arabic}{ ... }
%
% RUSSIAN
% for russian text we also need to define fonts with support for Cyrillic script
% \usepackage{fontspec}
% \setotherlanguage{russian}
% \newfontfamily\cyrillicfont{Times New Roman}
% \newfontfamily\cyrillicfontsf{Times New Roman}[Script=Cyrillic]
% \newfontfamily\cyrillicfonttt{Times New Roman}[Script=Cyrillic]
%
% in the text use \begin{russian} ... \end{russian}
% LANGUAGE - END

% EMOJIS - BEGIN
% to use emoticons in your manuscript
% https://stackoverflow.com/questions/190145/how-to-insert-emoticons-in-latex/57076064
% using font Symbola, which has full support
% the font may be downloaded at:
% https://dn-works.com/ufas/
% add to preamble:
% \newfontfamily\Symbola{Symbola}
% in the text use:
% {\Symbola 😥}
% EMOJIS - END

% LABEL REFERENCE TO DESCRIPTIVE LIST - BEGIN
% reference itens in a descriptive list using their labels instead of numbers
% insert the code below in the preambule:
%\makeatletter
%\let\orgdescriptionlabel\descriptionlabel
%\renewcommand*{\descriptionlabel}[1]{%
%  \let\orglabel\label
%  \let\label\@gobble
%  \phantomsection
%  \edef\@currentlabel{#1\unskip}%
%  \let\label\orglabel
%  \orgdescriptionlabel{#1}%
%}
%\makeatother
%
% in your document, use as illustraded here:
%\begin{description}
%  \item[first\label{itm1}] this is only an example;
%  % ...  add more items
%\end{description}
% LABEL REFERENCE TO DESCRIPTIVE LIST - END


% add line numbers for submission
%\usepackage{lineno}
%\linenumbers

\begin{document}
\maketitle

\begin{polyabstract}
\begin{abstract}
Este artigo é um modelo que visa orientar os autores que submeterão seus textos
à revista \emph{Texto Livre} a fazê-lo no padrão determinado pela revista a fim
de facilitar sua empreitada, visualizando e configurando o texto no formato já
configurado nesse modelo. Para fazê-lo, basta escrever ou colar seus textos no
lugar aqui designado, substituindo os textos e figuras aqui existentes,
verificando se eles mantêm a formatação aqui descrita. O resumo deverá ter
entre 150 e 200 palavras.

\keywords{palavra1 \sep palavra dois \sep palavra-três \sep palavra4}
\end{abstract}

\begin{english}
\begin{abstract}
This article is a template article which aims to guide authors who will submit
their papers to the \emph{Texto Livre} journal. This template should facilitate
their endeavor, viewing and configuring the text in the format already
configured accordingly. Just write or paste your content in the desired places,
replacing texts and figures, checking if they keep up the format. The abstract
should have between 150 and 200 words.

\keywords{word1 \sep word two \sep word-three \sep word4}
\end{abstract}
\end{english}
% if there is another abstract, insert it here using the same scheme
\end{polyabstract}

\section{Introdução}\label{sec-intro}
Este artigo é o modelo de documento que visa orientar os autores que submeterão seus textos à revista \emph{Texto Livre}.
Este documento fornece o padrão determinado pela revista, a fim de facilitar sua empreitada, visualizando e configurando
o texto no formato já configurado no modelo para a submissão. Para redigir seu texto neste modelo, basta escrever ou colar
seus textos no lugar aqui designado, substituindo os textos e figuras aqui existentes, verificando se eles mantêm a formatação aqui descrita.

A \Cref{sec-organizacao} apresenta as normas para a submissão. Informações sobre a utilização do modelo
estão presentes na \Cref{sec-modelo}. A \Cref{sec-organizacao} apresenta as considerações gerais sobre a organização do artigo,
apresenta a utilização de listas numeradas e não-numeradas (\Cref{sec-listas}),
ilustra como inserir figuras e tabelas no documento (\Cref{sec-figuras-tabelas}), assim como citações
e notas de rodapé (\Cref{sec-quotesandfootnotes}). A \Cref{sec-equacao} apresenta exemplos de equações
e a \Cref{sec-codigos} apresenta como inserir algoritmos e códigos.
Na página \pageref{sec-bib} encontram-se as referências e, em seguida, o \Cref{apx-longtable} que
apresenta a inclusão de uma tabela longa, que se entende por várias páginas.

\lipsum[1-5]

\section{Normas para submissão}\label{sec-normas}
A revista \emph{Texto Livre} tem como foco publicações originais em área interdisciplinar entre
Estudos da Linguagem e Tecnologias digitais, com foco na produção textual e na produção de
documentação para software livre e aceita submissões de textos inéditos (artigos, resenhas, ensaios e traduções).
As resenhas devem tratar de livros publicados nos últimos 24 meses no Brasil ou no Exterior,
em primeira edição ou tradução, abrangendo áreas contempladas neste periódico.

O texto deverá vir devidamente revisado pelo autor. A comissão editorial reserva-se o direito de
fazer nova revisão e de fazer as alterações necessárias. Textos que apresentem problemas de forma,
estilo e/ou adequação aos padrões da revista serão rejeitados.
Todos os artigos na revista \emph{Texto Livre} são publicados sob
a \href{https://creativecommons.org/}{Licença Creative Commons CC-BY}.
A reprodução de textos ou trechos devem seguir atribuições dada pela licença.
Para qualquer finalidade, solicitamos a comunicação prévia aos editores da revista.

Para textos com mais de um autor, todos os autores devem ser indicados na página de registro da submissão.
Os autores devem ter registro no \href{https://orcid.org/}{ORCID}. No que se refere à autoria,
\begin{quote}
``Não se considera co-autor quem simplesmente auxiliou o autor na produção da obra literária, artística ou científica,
revendo-a, atualizando-a, bem como fiscalizando ou dirigindo sua edição ou apresentação por qualquer meio'' (\citefield[Art.~15, §1º]{lei9610}{title}).
\end{quote}
Em caso de artigos em coautoria, todos os autores devem ser indicados na página de registro da submissão pelo autor
responsável, na seção de metadados da submissão e em documento suplementar. Além disso, no documento suplementar,
o responsável pela submissão deve declarar qual foi a contribuição específica de cada autor para a produção do artigo.

O artigo submetido não poderá ter sido publicado anteriormente, nem estar sob avaliação em outro periódico.
A originalidade e o ineditismo são cruciais para o trâmite de textos no periódico
em questão (um programa anti-plágio será utilizado pelo editor para verificação da originalidade do artigo).
A revista \emph{Texto Livre}, entretanto, encoraja os autores a publicarem \emph{preprints}
como forma de acelera a disseminação de seus resultados e obter comentários sobre seu trabalho,
antes do processo de revisão por pares.

\subsection{Código de Conduta e Boas Práticas}\label{sec-conduta}
A revista \emph{Texto Livre} segue as diretrizes do Código de Conduta e Boas Práticas do
\href{http://publicationethics.org/}{COPE} (Committee on Publication Ethics)
e as submissões devem atender a essas diretrizes: para conhecimento do Código,
consulte o texto original em \href{http://publicationethics.org/files/Code_of_conduct_for_journal_editors_1.pdf}{inglês}
ou sua tradução para o \href{http://www.periodicos.letras.ufmg.br/CCBP-COPE.pdf}{português}.

Todas as submissões que contiverem relatos de pesquisas feitas com seres humanos deverão apresentar análise do COEP (apresentação do número CAAE) ou de outros órgãos de ética em pequisa.


\subsection{Formato do manuscrito}\label{sec-fmt-manuscrito}
O formato de página deverá ser A4 (já definido no modelo para \LaTeX{}). O limite de palavras é de 5 a 10 mil palavras (do título às referências)
para artigos e de 800 a 1.000 palavras para resenhas.
Serão aceitos textos em português, inglês, francês ou espanhol.
O título, resumo e palavras-chave deverão ser fornecidos no idioma em que for escrito o manuscrito, em português e em inglês.
O resumo deve seguir as normas da \href{https://www.abntcatalogo.com.br/norma.aspx?ID=2003}{ABNT NBR 6028:2003}: conter objetivo, método, resultados e conclusões do artigo;
compor-se de sequência de frases concisas em parágrafo único;  usar verbo na voz ativa e na terceira pessoa do singular;
e ter entre 100 e 250 palavras. O resumo não deve conter citações, nem abreviações (se possível).
As palavras-chave devem vir logo abaixo do resumo.


\subsection{Formato de arquivo}\label{sec-formato}
O manuscrito deverá ser enviado em formato \TeX{}, utilizando o modelo da revista (veja a \Cref{sec-modelo}),
ou em formato ODT (\emph{Open Document Format}). Não serão aceitos formatos proprietários, como .doc, .docx ou .rtf
(conforme Comunicado de 13 de setembro de 2011 \url{http://www.periodicos.letras.ufmg.br/index.php/textolivre/about/editorialPolicies#custom-0}).
As imagens deverão ser enviadas separadamente em arquivos (veja as orientações na \Cref{sec-figuras-tabelas}).
Deverá ser enviado também o arquivo \texttt{.bib} contendo a bibliografia utilizada, no formato \Hologo{BibTeX}.
O arquivo \texttt{.bib} deverá ser enviado mesmo que os autores optem por enviar o manuscrito no formato ODT.
Veja mais sobre o formato \Hologo{BibTeX} no tutorial do \href{https://www.overleaf.com/learn/latex/Bibliography_management_in_LaTeX#The_bibliography_file}{Overleaf},
no \href{https://en.wikibooks.org/wiki/LaTeX/Bibliography_Management}{Wikibooks} ou em \textcite{araujo2020}.
Ao gerar o \texttt{.pdf} para submissão, enumere as linhas utilizando o pacote \texttt{lineno} (ver comentários
no cabeçalho deste documento). 


\section{Utilização do modelo}\label{sec-modelo}
Recomendamos a utilização do \Hologo{XeLaTeX}\footnote{
Para ter suporte a várias línguas, decidiu-se adotar o \Hologo{XeLaTeX} ao invés do \LaTeX{}.
O \Hologo{XeLaTeX} processa documentos \TeX{} codificados em URF-8, isto possibilita
a utilização de qualquer caractere do Unicode diretamente, o que não é possível com o \Hologo{pdfTeX}.
} para a confecção do manuscrito.
Toda a formatação do texto é definida no arquivo de classe \texttt{textolivre.cls}.
Este arquivo não deverá ser editado.
Utilize o arquivo \texttt{article.tex} como base para redigir o manuscrito.
Caso haja necessidade de incluir novos pacotes e criar novas definições específicas para um artigo,
estas deverão ser feitas no preâmbulo do arquivo \texttt{.tex}.
Em caso de dúvida sobre a utilização do estilo, entre em contato com a equipe editorial da revista.

A bibliografia deverá ser elaborara separadamente, em um arquivo \texttt{.bib}, utilizando as
convenções do \Hologo{BibTeX}. Para gerar a bibliografia deverá ser utilizada a ferramenta \texttt{Biber}.
A \Cref{lst-compiledocument} apresenta a sequência de comandos para compilar o documento final.
\begin{lstlisting}[language=bash, label=lst-compiledocument, caption={Sequência para gerar o documento final.}]
xelatex article.tex
biber article
xelatex article.tex # pode ser necessário novamente, caso o número de páginas altere após inserir a bibliografia
\end{lstlisting} %stopzone

\section{Organização do Texto}\label{sec-organizacao}
O artigo deverá ter título na língua da manuscrito, título em português e título em inglês, nesta ordem.
Se o manuscrito for redigido em português, teremos apenas dois títulos: em português e em inglês.
Se for redigido em inglês, teremos a ordem inversa: inglês e português.
A mesma regra vale para o resumo e palavras-chave que deverão ser apresentados nestas três ou duas línguas.

O artigo poderá ser organizado de forma a conter seções, subseções e subsubseções.
Não há limites para o tamanho de cada uma. Entretanto, deve-se evitar secções vazias, que não possuam texto e contenham apenas subseções.
Ainda nesta secção, iremos apresentar a utilização de listas (\Cref{sec-listas}),
tabelas e figuras (\Cref{sec-figuras-tabelas}), citações e notas de rodapé (\Cref{sec-quotesandfootnotes}).
Na \Cref{sec-equacao} iremos apresentar a utilização de equações e na \Cref{sec-codigos}
apresentaremos a forma de utilização de códigos fonte.

\lipsum[5]

\subsection{Definição do título, resumo, autores, seções e idioma no modelo para \LaTeX{}}\label{sec-organizacao-latex}
Nesta seção iremos apresentar algumas definições básicas para a elaboração do manuscrito utilizando
o modelo da revista.

\subsubsection{Título}\label{sec-titulo}
O título, no idioma do manuscrito, é definido utilizando o comando \verb|\title{...}|. Em seguida,
utiliza o comando \verb|\othertitle{...}| para definir o título em outra língua. Repita o comando
caso seja necessário definir o título em uma terceira língua.

\subsubsection{Autores}\label{sec-autores}
Os autores do manuscrito são definidos utilizando os comandos do pacote \verb|authblk|.
Veja a documentação no \href{CTAN}{https://ctan.org/pkg/authblk}. Abaixo reproduzimos um exemplo típico de utilização
em conformidade com o modelo da revista:

\begin{lstlisting}[language=tex, label=lst-authors, caption={Definição dos autores do manuscrito.},postbreak=\mbox{$\hookrightarrow$\space}]
\author[1]{author1 \orcid{0000-0000-0000-0000} \thanks{Email: \url{author1@dept.univ.edu}}}
\author[1]{author2 \thanks{Email: \url{author2@dept.univ.edu}}}
\author[2]{author3 \orcid{0000-0000-0000-0000} \thanks{Email: \url{author3@inst.gov}}}
\author[2]{author4 \orcid{0000-0000-0000-0000} \thanks{Email: \url{author4@inst.gov}}}
\affil[1]{affil1}
\affil[2]{affil2}
\end{lstlisting} %stopzone

O nome do autor correspondente será definido através do comando \verb|\corrauthor{...}|. 
Este nome aparecerá na primeira página, na lateral esquerda, junto de outras informações sobre o artigo,
tais como DOI, data de recebimento, aceite e publicação.
Para definir a forma como o nome dos autores aparecerá no rodapé do artigo, utilize o comando \verb|\runningauthor{...}|.

\subsubsection{Idioma}\label{sec-idioma}
Os artigos para a revista \textit{Texto Livre} podem ser submetidos em português, inglês,
espanhol ou francês. Definição a língua principal do artigo na evocação da classe, conforme exemplos na
\Cref{lst-language}. O comportamento \textit{default} da classe é utilizar o português como língua principal do artigo,
desta forma, o inglês será a língua secundária. Caso o inglês seja a língua principal, o português passará a ser a língua secundária.
Nos demais casos, o português será a língua secundária e o inglês a língua terciária.

\begin{lstlisting}[language=tex, label=lst-language, caption={Definição da língua principal do artigo.},postbreak=\mbox{$\hookrightarrow$\space}]
\documentclass{textolivre} % utilização da língua padrão: português
\documentclass[portuguese]{textolivre}
\documentclass[english]{textolivre}
\documentclass[spanish]{textolivre}
\documentclass[french]{textolivre}
\end{lstlisting}

Eventualmente, trechos em outras línguas poderão ser utilizados no manuscrito. Para prover o suporte a diferentes línguas,
utilizamos o pacote \verb|polyglossia|. Para utilizar outras línguas, especifique no preâmbulo utilizando o 
comando \verb|\setotherlanguage{...}|. Algumas línguas requerem ainda utilização de fontes que tenham suporte
aos caracteres desta língua. Na \Cref{lst-setlang-arabic} apresentamos o que deve ser utilizado no preâmbulo para 
prove o suporte a árabe e na \Cref{lst-setlang-russian} o que deve ser utilizado para prover o suporte a russo.

\begin{lstlisting}[language=tex, label=lst-setlang-arabic, caption={Código para acrescentar ao preâmbulo para prover o suporte a árabe no texto.}]
\setotherlanguage{arabic}
\newfontfamily\arabicfont[Script=Arabic]{Amiri}
\newfontfamily\arabicfontsf[Script=Arabic]{Amiri}
\newfontfamily\arabicfonttt[Script=Arabic]{Amiri}
\end{lstlisting} %stopzone

\begin{lstlisting}[language=tex, label=lst-setlang-russian, caption={Código para acrescentar ao preâmbulo para prover o suporte a russo no texto.}]
\usepackage{fontspec}
\setotherlanguage{russian}
\newfontfamily\cyrillicfont{Times New Roman}
\newfontfamily\cyrillicfontsf{Times New Roman}[Script=Cyrillic]
\newfontfamily\cyrillicfonttt{Times New Roman}[Script=Cyrillic]
\end{lstlisting} %stopzone

Para a utilização de emojis no manuscrito, utilizamos a fonte Symbola, conforme explicação
apresentada no \href{https://stackoverflow.com/questions/190145/how-to-insert-emoticons-in-latex/57076064}{StackOverflow}.
A fonte pode ser baixada na internet, por exemplo em \url{https://dn-works.com/ufas/}.
Após instalar a fonte, basta acrescentar ao preâmbulo \verb|\newfontfamily\Symbola{Symbola}|
e utilizar o comando \verb|\Symbola| seguido do emoji desejado.



\subsubsection{Resumo}\label{sec-resumo}
O texto do resumo, \emph{abstract} e, eventualmente, o resumo em um terceira língua serão inseridos, cada um,
dentro de seu próprio ambiente \verb|abstract|. O conjunto dos resumos serão inseridos dentro do ambiente \verb|polyabstract|.
A lista de palavras-chave será definida pelo comando \verb|\keywords{...}|, sendo as palavras separadas pelo
comando \verb|\sep|. Veja o exemplo na \Cref{lst-abstract}.
\begin{lstlisting}[language=tex, label=lst-abstract, caption={Como definir os resumos e palavras chaves em várias línguas.}]
\begin{polyabstract}
\begin{abstract}
...

\keywords{palavra1 \sep palavra2}
\end{abstract}

\begin{english}
\begin{abstract}
...

\keywords{word1 \sep word2}
\end{abstract}
\end{english}
\end{polyabstract}
\end{lstlisting} %stopzone

\subsubsection{Seções}\label{sec-secoes}
As seções e subseções do artigo são criadas utilizando o comando \verb|section|
e \verb|subsection|, respectivamente. Para referenciá-las ao longo do texto,
defina um rótulo utilizando o comando \verb|label|. Este rótulo será utilizado
pelo comando \verb|ref| ou \verb|Cref| para estabelecer a referência no ponto desejado.
Veja um exemplo na \Cref{lst-seclbl}.
\begin{lstlisting}[language=tex, label=lst-seclbl, caption={Seções, subseções e referências.}, source={Elaboracao propria.}]
\section{Introdução}\label{intro}
...

\section{Conclusão}
Vimos na \Cref{intro} que ...
\end{lstlisting} %stopzone

\subsection{Formatação simples}\label{sec-format-simple}
Algumas formações simples de texto são utilizadas para evidenciar trechos e conceitos.
Nesta secção veremos como utilizar itálico, negrito e sublinhar palavras.
Para utilizar essas três formatações simples, devemos utilizar, respectivamente, os seguintes
comandos: \verb|\emph{...}| ou \verb|\textit{...}|, \verb|\textbf{...}| e \verb|\underline{...}|.
Veja a seguir uma frase utilizando estes comandos:
`Muitas \textbf{grandes} descobertas das \textit{ciências} ocorrem por \underline{acidente}'.

Os comandos \verb|\emph{...}| e \verb|\textit{...}| muitas vezes podem levar a um mesmo resultado,
mas são essencialmente diferentes. O comando \verb|\textit{...}| coloca um texto em itálico enquanto
o comando \verb|\emph{...}| dá ênfase a um texto. A forma como a ênfase será expressa, dependerá do contexto.
Veja um exemplo em que utilizamos um comando dentro do contexto do outro comando.
O comando \verb|\emph{isto é \emph{um} teste}| terá como resultado: \emph{isto é \emph{um} teste}.
Já o comando \verb|\textit{isto é \textit{um} teste}| terá como resultado: \textit{isto é \textit{um} teste}.
E utilizando \verb|\textit{isto é \emph{um} teste}| terá como resultado: \textit{isto é \emph{um} teste}.

\subsection{Links}\label{sec-links}
Para incluir links para sítios na internet, podemos utilizar os comandos \verb|\url{...}|
ou \verb|\href{url}{texto}|. O primeiro irá imprimir o endereço do sítio,
enquanto o segundo cria um link para um endereço para um determinado texto.
Por exemplo, o síto da revista \href{https://periodicos.ufmg.br/index.php/textolivre/}{Texto Livre} é o seguinte:
\url{https://periodicos.ufmg.br/index.php/textolivre/},
onde utilizamos ambos comandos: \verb|href| para criar o link em `Texto Livre' e \verb|url| para incluir
a URL da revista no texto.

\subsection{Outras estruturas}\label{sec-outras-estr}
Em um artigo, usualmente utilizamos outros recursos além do puro texto, como por exemplo:
figuras, listas, citações e tabelas. Nesta seção iremos ver como inserir cada um destes
seguindo o modelo da revista.


\subsubsection{Listas numeradas e não numeradas}\label{sec-listas}
Os artigos poderão conter listas numeradas ou não numeradas.
Quando não houver numeração, deverão preferencialmente serem marcadas com ponto, conforme o exemplo a seguir:
\begin{itemize}
\item exemplo 1
\item exemplo 2
\item exemplo 3
\end{itemize}
As listas não-numeradas são definidas utilizando o ambiente \verb|itemize| e cada item da lista
é definido por um \verb|item|, conforme exemplificado na \Cref{lst-itemize}.
\begin{lstlisting}[language=tex, label=lst-itemize, caption={Listas não-numeradas.}]
\begin{itemize}
\item exemplo 1
\item exemplo 2
\item exemplo 3
\end{itemize}
\end{lstlisting} %stopzone

\lipsum[2]

Quando for necessária uma lista ordenada, dever-se-á, preferencialmente, seguir a numeração arábica:
\begin{enumerate}
\item exemplo 1
\item exemplo 2
\item exemplo 3
\end{enumerate}
Por sua vez, as listas numeradas são definidas no ambiente \verb|enumerate|, conforme vemos na \Cref{lst-enum}.
\begin{lstlisting}[language=tex, label=lst-enum, caption={Listas numeradas.},float]
\begin{enumerate}
\item exemplo 1
\item exemplo 2
\item exemplo 3
\end{enumerate}
\end{lstlisting} %stopzone

\lipsum[3]

Em \LaTeX{} existem 3 tipos de ambientes para criar estruturas de listas: \ref{list-itemize}, \ref{list-enumerate} e \ref{list-description}.
Os dois primeiros foram ilustrados anteriormente nesta seção, já o último tipo é
utilizo a seguir para descrever os três tipos.
\begin{description}[topsep=1ex,partopsep=1ex]
  \item[itemize\label{list-itemize}] para listas do tipo \emph{ponto lista};
  \item[enumerate\label{list-enumerate}] para listas \emph{numeradas}; e
  \item[description\label{list-description}] para listas descritivas.
\end{description}
Observe no exemplo que os itens de uma lista podem ser referenciados. Para criar uma referência para lista descritiva,
conforme ilustrado aqui, é necessário adicionar um código ao preâmbulo do documento. Veja o código \texttt{.tex} deste documento.

Para referenciar itens de uma lista descritiva utilizando seus rótulos, ao invés dos números dos itens, 
veja a \Cref{lst-refdescitem}, onde há o código que deverá ser acrescentado ao preâmbulo e a forma de utilização no documento.

\begin{lstlisting}[language=tex, label=lst-refdescitem, caption={Código para ser inserido para realizar referências atravé do rótulo de itens de uma lista descritiva.}]
% inserir o código abaixo no preâmbulo do documento
% preâmbulo - BEGIN
\makeatletter
\let\orgdescriptionlabel\descriptionlabel
\renewcommand*{\descriptionlabel}[1]{%
  \let\orglabel\label
  \let\label\@gobble
  \phantomsection
  \edef\@currentlabel{#1\unskip}%
  \let\label\orglabel
  \orgdescriptionlabel{#1}%
}
\makeatother
% preâmbulo - END

% código para utilização no documento
\begin{description}
  \item[first\label{itm1}] this is only an example;
  % ...  add more items
\end{description}
Exemplo de referência a um item no texto: \ref{itm1}.

\end{lstlisting} %stopzone



\subsubsection{Figuras e tabelas}\label{sec-figuras-tabelas}
Nesta seção iremos descrever como inserir figuras no documento em conformidade com o padrão da revista.
As figuras serão numeradas na ordem em que são mencionadas no texto. Os arquivos de imagem deverão
ser enviados separadamente com nomenclatura que siga a numeração das figuras (não é necessário para envios em \LaTeX{}).
As figuras deverão ficar centralizadas, com legenda abaixo e informação de fonte, caso necessário.
O título, fonte e legenda de figuras devem ser parte do manuscrito, e não parte do arquivo de figura.
As figuras devem ser apropriadamente recortadas para exibir apenas o que é de interesse dos autores,
reduzindo assim os espaços em branco que circundam as figuras.
Os autores devem enviar a figura no formato adequado.
Sempre que possível, utilize figuras vetoriais para manter uma melhor qualidade e propiciar uma boa apresentação do artigo.
Use os formatos PDF ou EPS para as imagens vetoriais e os formatos PNG ou JPEG para as imagens rasterizadas.

Não faça referência às figuras ou tabelas dizendo `a figura abaixo/acima'.
Figuras e tabelas são elementos flutuantes. Muitas vezes não há espaço suficiente para
eles no final de uma página ou, quando ocuparem muito espaço na página, podemos preferir
ter uma página apenas com estes elementos, para não prejudicar o texto.
A \LaTeX{} busca decidir o melhor local para inserir estes elementos, seguindo uma ordem
de preferências dada pelo usuário, não garantindo que estes elementos apareçam exatamente
no local em que se encontram no arquivo \verb|.tex|. Além disso, no processo de diagramação
da revista, poderá ser necessário realizar alguns ajustes e, assim, a figura/tabela poderá
não ficar exatamente logo abaixo/acima do texto que a referencia.
Desta forma, prefira sempre referenciar as figuras/tabelas pelo seu nome,
por exemplo, veja a \Cref{fig-img-a}. O mesmo é válido para qualquer outro elemento flutuante\footnote{
Elementos flutuantes são aqueles que não podem ser quebrados na transição de páginas.
Os elementos flutuantes não são parte do fluxo do texto, são tratados como entidades separadas
e podendo serem posicionados em locais na página destinados a eles. Os elementos flutuantes devem então
ser numerados, para serem referenciados no texto.
}
que seja inserido no texto. A princípio, figuras e tabelas, porém outros poderão ser definidos
como elementos flutuantes.

\begin{figure}[htbp]
 \centering
 \includegraphics[width=0.5\textwidth]{example-image-a}
 \caption{Imagem de teste.}
 \label{fig-img-a}
 \source{Forneça aqui a fonte da imagem.}
\end{figure}

As figuras deverão ser inseridas no manuscrito utilizando o ambiente \verb|figure|.
A \Cref{lst-figure} apresenta um exemplo de utilização.
\begin{lstlisting}[language=tex, label=lst-figure, caption={Inserindo uma figura.}]
\begin{figure}[htbp]
 \centering
 \includegraphics[width=0.5\textwidth]{example-image-a}
 \caption{Imagem de teste.}
 \label{fig-01}
 \source{Forneça aqui a fonte da imagem.}
\end{figure}
\end{lstlisting} %stopzone

\lipsum[12]

De maneira similar, as tabelas serão inseridas utilizando o ambiente \verb|table|.
A \Cref{lst-table} apresenta um exemplo de utilização.
\begin{lstlisting}[language=tex, label=lst-table, caption={Inserindo uma tabela.}]
\begin{table}[htpb]
\caption{Legenda da tabela.}
\label{tbl-tabela-01}
\begin{tabular}{llp{3cm}}
\toprule 
A & B & C \\ 
\midrule
1 & 2 & 3 \\ 
4 & 5 & 6 \\ 
\bottomrule
\end{tabular}
\source{Fonte da tabela.}
\notes{Se necessário, poderá ser adicionada uma nota ao final da tabela.}
\end{table}
\end{lstlisting} %stopzone

Podemos adicionar ao final das tabelas e figuras a fonte e notas. Para tanto
utilize os comandos \verb|\source{...}| e \verb|\notes{...}| respectivamente.
Estes comandos devem vir logo antes do fim da tabela (\verb|\end{table}|).

\begin{table}[htpb]
\caption{Legenda da tabela do documento modelo da revista \emph{Texto Livre}.}
\label{tbl-tabela-01}
\begin{tabular}{llp{4.3cm}}
\toprule 
Tabela                      & Coluna 1                       & Coluna 2                                                               \\ 
\midrule
Lorem ipsum                 & Non consectetur                & Leo vel fringilla                                                      \\ 
Lorem ipsum dolor sit amet. & Non consectetur a erat nam at. & Leo vel fringilla est ullamcorper eget nulla facilisi etiam dignissim. \\ 
\bottomrule
\end{tabular}
\source{Fonte da tabela.}
\notes{Esta é uma nota exemplo que poderá, opcionalmente, ser adicionada a uma tabela ou figura.}
\end{table}

Uma tabela é um arranjo de células organizadas em colunas verticais e linhas horizontais.
As células são elementos mínimos indivisíveis. Dependendo da necessidade, as células podem
ser mescadas, expandindo assim ao longo de linhas (\verb|\multirow|) ou colunas (\verb|\multicolumn|). 
Assim como ocorre com as figuras, as tabelas são elementos que se destacam do texto e ambos
são elementos que convêm informação em forma visual.

Utilize tabelas apenas quando necessário. Muitas vezes as informações podem ser expostas no 
corpo do texto ou através de listas, deixando o texto mais fluído. Se criar uma determinada tabela torna-se uma tarefa extremamente
complicada, talvez esta não seja a forma ideal de expor tais informações. Pode ser necessário utilizar uma lista,
uma lista aninhada, ou talvez dividir a tabela em duas. Uma tabela deve ser tão simples quanto possível
e facilmente compreendida de forma isolada, sem a necessidade de impor ao leitor a busca por outras informações
no texto. O texto deve faze menção à tabela, destacando seus principais aspectos e resumindo as
informações lá apresentadas, nunca criando redundâncias desnecessárias.

As tabelas são muito úteis para apresentar dados numéricos, facilitando a visualização e comparação entre eles.
Se a quantidade de dados a ser apresentada for muito grande e não havendo necessidade de expôr todos os detalhes,
talvez um gráfico seja mais apropriado. Evite criar tabelas muito longas e tabelas de textos.

Para criar uma unidade ao longo do texto e facilitar a comparação de informações entre diferentes tabelas,
mantenha sempre um padrão de estilo para representação dos dados. Mantenha a mesma escolha de fontes,
espaçamentos, linhas, formato de cabeçalho e abreviações em todas as tabelas do texto. 

Quando optar por utilizar uma tabela, busque deixá-la visualmente leve. Lembre-se que os dados são os elementos
importantes em uma tabela. As linhas devem ser utilizadas com parcimônia. 
Evite utilizar linhas verticais. Busque utilizar apenas as linhas horizontais 
necessárias: início de tabela, utilizando \verb|\toprule|; fim de cabeçalho, utilizando \verb|\midrule|; e
fim da tabela, utilizando \verb|\bottomrule|. Em geral, as demais linhas horizontais apenas deixam a tabela
pesada e difícil de ler. Se realmente for necessário, opte por linhas em cinza claro, para não criar competição 
com os dados da tabela. Outra alternativa é alterar o espaçamentos entre linhas. Em caso de dúvida quando ao alinhamento 
dos textos na tabela, utilize o alinhamento à esquerda. Utilize cores apenas quando necessário destacar
alguma informação ou quando estas irão facilitar a leitura da tabela. As tabelas no meio de um texto devem ser curtas.
Se necessário, faça uma página apenas com a tabela. Quando a tabela for muito longa, opte por inseri-la como apêndice.

O \emph{template} da revista utiliza o pacote \verb|booktabs| para melhor a qualidade das tabelas. 
Leia a documentação do pacote disponível no \href{https://www.ctan.org/pkg/booktabs}{CTAN}. 
As \href{https://people.inf.ethz.ch/markusp/teaching/guides/guide-tables.pdf}{notas de aula do Markus Püschel}
podem ser utilizadas como um bom guia sobre como criar uma tabela de forma elegante.

\begin{table}[htpb]
\caption{Exemplo de tabela utilizando linhas horizontais.}
\label{tbl02}
\begin{tabular}{llp{11cm}}
\toprule
Rótulo 1 & Rótulo 2 & Texto \\
\midrule
%\arrayrulecolor[gray]{.7}
\multirow{6}{*}{AAAA} & \multirow{3}{*}{AABB} & Nam sed ex in massa tincidunt elementum nec dignissim leo.  \\
 \cmidrule{3-3}
 & & Sed nec placerat felis. In urna leo, convallis a nunc non, sollicitudin auctor nisi. \\
 \cmidrule{3-3}
 & & Proin eget rutrum elit. \\
 \cmidrule{2-3}
 & \multirow{3}{*}{BBCC} & Mauris vulputate magna id ante placerat molestie. \\
 \cmidrule{3-3}
 & & Praesent nisi magna, aliquam vel libero sit amet, sagittis volutpat tortor. \\
 \cmidrule{3-3}
 & & Pellentesque habitant morbi tristique senectus et netus et malesuada fames ac turpis egestas. \\
%\arrayrulecolor{black}
\bottomrule
\end{tabular}
\source{Template da revista Texto Livre.}
\notes{Evite fazer tabelas como esta. Prefira utilizar uma lista neste tipo de situação. Se realmente for necessário, utilize poucas linhas e use cinza claro.}
\end{table}

\lipsum[3]

Ainda outros exemplos de utilização de figuras e tabelas são apresentados nas \Cref{fig:example,fig:twosubs,fig:landscape,tab:example}. Em especial, observe os exemplos para criar subfiguras (\Cref{fig:twosubs}) e inserir uma figura grande com orientação paisagem (\Cref{fig:landscape}).

\begin{figure}[htbp]
\centering
\includegraphics[width=0.6\textwidth]{example-image}
\caption{Esta é a legenda da figura. Esta pode ser breve ou longa e conter referências se necessário.}
\label{fig:example}
\source{Referência ao autor e à publicação original da figura. Se a figura foi de autoria própria, apenas indicar ``autoria própria'' como fonte.}
\end{figure}

\lipsum[30-35]

\begin{table}[htbp]
\caption{Recorde de velocidade de automóveis (GR 5-10).}
\label{tab:example}
\centering
\begin{tabular}{l l l l l}
\headrow \thead{Speed (mph)} & \thead{Driver} & \thead{Car} & \thead{Engine} & \thead{Date} \\
407.447     & Craig Breedlove & Spirit of America          & GE J47    & 8/5/63   \\
413.199     & Tom Green       & Wingfoot Express           & WE J46    & 10/2/64  \\
434.22      & Art Arfons      & Green Monster              & GE J79    & 10/5/64  \\
468.719     & Craig Breedlove & Spirit of America          & GE J79    & 10/13/64 \\
526.277     & Craig Breedlove & Spirit of America          & GE J79    & 10/15/65 \\
536.712     & Art Arfons      & Green Monster              & GE J79    & 10/27/65 \\
555.127     & Craig Breedlove & Spirit of America, Sonic 1 & GE J79    & 11/2/65  \\
576.553     & Art Arfons      & Green Monster              & GE J79    & 11/7/65  \\
600.601     & Craig Breedlove & Spirit of America, Sonic 1 & GE J79    & 11/15/65 \\
622.407     & Gary Gabelich   & Blue Flame                 & Rocket    & 10/23/70 \\
633.468     & Richard Noble   & Thrust 2                   & RR RG 146 & 10/4/83  \\
763.035     & Andy Green      & Thrust SSC                 & RR Spey   & 10/15/97\\
\end{tabular}
\source{\url{https://www.sedl.org/afterschool/toolkits/science/pdf/ast_sci_data_tables_sample.pdf}}
\end{table}

\lipsum[2-4]

\begin{figure}[htbp]
\begin{minipage}[t]{0.47\textwidth}
\includegraphics[width=\linewidth]{example-image}
\subcaption{Esta é uma subfigura.}
\source{Autoria própria.}
\end{minipage}
\hfill
\begin{minipage}[t]{0.47\textwidth}
\includegraphics[width=\linewidth]{example-image}
\subcaption{Esta é outra subfigura.}
\end{minipage}

\caption{Esta é a legenda geral aplicada a ambas figuras.}
\label{fig:twosubs}
\source{Caso ambas figuras tenham a mesma autoria, basta especificar a fonte uma única vez.}
\end{figure}

\begin{sidewaysfigure}
\centering
\includegraphics[width=0.85\textwidth]{example-image}
\caption{Esta é a legenda para uma figura grande que ocupa toda a página. Para melhor apresentação desta figura, ela é rotacionada utilizando o ambiente \texttt{sidewaysfigure}, sendo então exibida no formato paisagem.}
\label{fig:landscape}
\source{Figura de autoria própria.}
\end{sidewaysfigure}

Se possível, evite tabelas muito longas. Se elas foram necessários, utilize-as preferencialmente na sessão de apêndice. 
Veja como exemplo a \Cref{longtbl-01} no \Cref{apx-longtable}.


\subsubsection{Citações e notas de rodapé}\label{sec-quotesandfootnotes}
Para inserir citações no texto utilize a formatação descrita nesta seção.

\begin{quote}
Book printing differs significantly from ordinary typing with respect to dashes, hyphens, and minus signs.
In good math books, these symbols are all different; in fact there usually are at least four different symbols (...)
Hyphens are used for compound words like `daughter-in-law' and `X-rated'. En-dashes are used for number ranges like
`pages 13--34', and also in contexts like `exercise 1.2.6--52'.
Em-dashes are used for punctuation in sentences---they are what we often call simply dashes.
And minus signs are used in formulas. A conscientious user of \TeX{} will be careful to distinguish these four usages
\cite[p. 4]{donaldknuth1984}.
\end{quote}

As notas de rodapé serão numeradas e devem aparecer no final da página onde foi utilizada\footnote{
Observe que esta nota de rodapé está presente na página em que foi citada.}. Para os textos
redigidos com \LaTeX{}, basta utilizar o comando \verb|\footnote{...}| para inserir uma nota de rodapé
no local desejado.

Para realizar citações devemos utilizar os comandos do \texttt{biblatex}. Por exemplo, para realizar uma
citação textual, utilize comando \verb|\textcite{...}| (equivalente ao comando \texttt{citet}) para gerar o
resultado como aqui exemplificado: \textcite{donaldknuth1984}.
Para criar uma citação entre parênteses, utilize o comando \verb|\parencite{...}| (equivalente ao comando \texttt{citep})
para gerar um resultado como o do exemplo a seguir: \parencite{donaldknuth1984}.
Outra opção é utiliza o comando \verb|\cite{...}| para obter o mesmo resultado: \cite{donaldknuth1984}.

Muitas vezes, em uma citação, queremos referenciar um página específica de uma obra ou passar
alguma outra informação adicional.
Neste caso, devemos fazer como ilustrado na citação acima, onde utilizamos o seguinte comando: \verb|\cite[p. 4]{donaldknuth1984}|.
Em outros casos, queremos realizar uma citação indireta. Isto pode ser feito utilizando os
comandos do \href{https://github.com/abntex/abntex2}{abnTeX2}: \verb|\apud{autor_indireto}{autor_direto}| ou \verb|\textapud{autor_indireto}{autor_direto}|.
Veja o seguinte exemplo:
\begin{quote}
Some bookes are to bee tasted,
others to bee swallowed,
and some few to bee chewed and disgested \apud{bacon}{donaldknuth1984}.
\end{quote}

\begin{english}
English texts might need a possessive citation for a given author. For example,
suppose we want to talk about \posscite{donaldknuth1984} book. To accomplish such citation, we have used the command \verb|\posscite{...}|
provided in the template.
\end{english}

Se desejar referenciar o nome completo do primeiro autor, podemos utilizar o seguinte comando disponível no \emph{template} da revista: \verb|\citefirstlastauthor{...}|.
Para qualquer referência, o item referenciado deverá estar contido no arquivo \texttt{.bib}. A \Cref{lst-bib} ilustra a sintaxe básica para
as entradas no arquivo de bibliografia. É importante fornecer o nome completo do autor, sem abreviações.
Muitas publicações aceitam abreviações do primeiro nome do autor. Esta prática, entretanto, acarreta maior
dificuldade na identificação do autor.

\begin{lstlisting}[language=tex, label=lst-bib, caption={Estrutura básica de uma entrada no arquivo de bibliografia.}, source={\LaTeX{} Wikibook (\protect\url{https://en.wikibooks.org/wiki/LaTeX/Bibliography_Management}).}]
@article{greenwade93,
    author  = "George D. Greenwade",
    title   = "The {C}omprehensive {T}ex {A}rchive {N}etwork ({CTAN})",
    year    = "1993",
    journal = "TUGBoat",
    volume  = "14",
    number  = "3",
    pages   = "342--351"
}
\end{lstlisting} %stopzone

No exemplo a seguir, utilizamos uma epígrafe, seguindo o modelo do \emph{template} (veja o código fonte deste documento), 
onde citamos o nome completo do autor e o título da obra:

\epigraph{Computers are good at following instructions, but not at reading your mind.}{\citefirstlastauthor{donaldknuth1984}, \emph{\citetitle{donaldknuth1984}}}

Para inserir uma epígrafe seguindo o modelo, demove-se acrescentar ao preâmbulo do documento o código apresentado na \Cref{lst-epigrafe}.
\begin{lstlisting}[language=tex, label=lst-epigrafe, caption={Utilização de epígrafe seguindo o modelo da revista.}]
% código para ser inserido no preâmbulo
% custom epigraph - BEGIN 
% https://tex.stackexchange.com/questions/193178/specific-epigraph-style
\usepackage{epigraph}
\renewcommand\textflush{flushright}
\makeatletter
\newlength\epitextskip
\pretocmd{\@epitext}{\em}{}{}
\apptocmd{\@epitext}{\em}{}{}
\patchcmd{\epigraph}{\@epitext{#1}\\}{\@epitext{#1}\\[\epitextskip]}{}{}
\makeatother
\setlength\epigraphrule{0pt}
\setlength\epitextskip{0.5ex}
\setlength\epigraphwidth{.7\textwidth}
% custom epigraph - END

% exemplo de utilização no documento
\epigraph{Computers are good at following instructions, but not at reading your mind.}{\citefirstlastauthor{donaldknuth1984}, \emph{\citetitle{donaldknuth1984}}}
\end{lstlisting} %stopzone



\subsubsection{Equações}\label{sec-equacao}
Nesta secção iremos apresentar a forma de utilização de equações. A \Cref{eq-poisson} apresenta um
exemplo de equação no modelo da revista \emph{Texto Livre}.
\begin{equation}
l(\Lambda)=\sum_{i=1}^{n} \sum_{w=1}^{q} (z_{i w} \ln (\lambda_{i w}) - \lambda_{i w} - \ln (z_{i w}!))
\label{eq-poisson}
\end{equation}
As equações deverão ser numeradas, para que seja possível realizar referência a elas ao longo do texto.
Outro exemplo é apresentado na \Cref{eq-frac}.
\begin{equation}
  x = a_0 + \frac{1}{\displaystyle a_1 
          + \frac{1}{\displaystyle a_2 
          + \frac{1}{\displaystyle a_3 + a_4}}}
\label{eq-frac}
\end{equation}

Apresentamos nas \Cref{eq-align-ex1,eq-align-ex2,eq-align-ex3,eq-align-ex4,eq-align-ex5,eq-align-ex6} um exemplo de utilização de equações com mais de uma linha. 
Para este tipo de equação devemos utilizar o ambiente \texttt{aling}. 
As equações que compõem a sequência poderão ser enumeradas ou não. Case sejam enumeradas, opte por inseri-las dentro do contexto do ambiente 
\texttt{subsequations} para que a numeração seja conforme o exemplo apresentado.
Se esta sequência de equações for longa, recomenda-se permitir a quebra de linha ao longo desta. 
Para tanto, utilize o comando \verb|\allowdisplaybreaks|. Isto evitará o aparecimento de espaços vazios 
no manuscrito.

\begin{subequations}
\allowdisplaybreaks
\begin{align}
H(Z|X) &= \sum_x p(x) H(Z|X=x) \label{eq-align-ex1} \\
       &= - \sum_x p(x) \sum_z p(Z=z|X=x) \log p(Z=z|X=x) \label{eq-align-ex2}\\
       &= - \sum_x p(x) \sum_y p(Y=z-x|X=x) \log p(Y=z-x|X=x) \label{eq-align-ex3} \\
       &= - \sum_x p(x) \sum_y p(Y=y|X=x) \log p(Y=y|X=x) \label{eq-align-ex4} \\
       &= \sum_x p(x) H(Y|X=x) \label{eq-align-ex5} \\
       &= H(Y|X) \label{eq-align-ex6}
\end{align}
\end{subequations}

A \Cref{fig:example} apresenta uma figura normal, enquanto 
a \Cref{fig:twosubs} ilustra a divisão de uma figura em duas sub-figuras. 
A \Cref{fig:landscape} é um exemplos de utilização de figura na orientação paisagem.
Utilize o comando \verb|\subcaption{...}| do pacote \texttt{subcaption} para adicionar legendas 
às sub-figuras e sub-tableas. Não utilize o pacote \texttt{subfigure} para evirar incompatibilidades com este modelo.

\lipsum[10-14]

\subsubsection{Códigos}\label{sec-codigos}
Para inserir códigos no texto utilize o ambiente \texttt{lstlisting}. 
O modelo da revista permite a inserção da fonte utilizada como referência, confome 
pode ser visto no exemplo apresentado na \Cref{lst-code}.
 
\begin{lstlisting}[language=python, label=lst-code, caption={\emph{Bubble sort}, ou ordenação por flutuação.}, source={Rosetta Code (\url{https://rosettacode.org/wiki/Sorting_algorithms/Bubble_sort}).}]
def bubble_sort(seq):
    """Inefficiently sort the mutable sequence (list) in place.
       seq MUST BE A MUTABLE SEQUENCE.
 
       As with list.sort() and random.shuffle this does NOT return 
    """
    changed = True
    while changed:
        changed = False
        for i in xrange(len(seq) - 1):
            if seq[i] > seq[i+1]:
                seq[i], seq[i+1] = seq[i+1], seq[i]
                changed = True
    return seq
\end{lstlisting} %stopzone

\lipsum[20-21]


\section{Contribuição dos autores}\label{sec-contributors-expl}
Espera-se que cada autor tenha realizado contribuição substanciais para o desenvolvimento do trabalho.
Em publicações com mais de um autor, deverá haver ao final do artigo, após as referências, uma
secção com declaração de contribuição de cada autor. Para criar esta secção basta utilizar o
ambiente \verb|contributors| e elencar os autores na lista de taxonomia com as 14 funções, conforme o
\href{http://credit.niso.org/}{CRediT (\emph{Contributor Roles Taxonomy})}, utilizadas para representar as contribuições do autores:
Conceituação; Curadoria de dados; Análise Formal; Aquisição de financiamento; Investigação;
Metodologia; Administração de projetos; Recursos; Programas; Supervisão; Validação;
Visualização; Escrita - rascunho original; Escrita - revisão e edição.
Neste exemplo de utilização do \emph{template}, a informação das contribuições do autores encontra-se na página \pageref{sec-contributors}.

A contribuição dos autores deverá ser inserida dentro do ambiente \verb|contributors|. Para cada autor do manuscrito,
deve-se chamar o comando \verb|\authorcontribution{nome}[contr1,contr2]|, passando como parâmetro obrigatório o nome do autor 
e parâmetros opcionais os códigos das contribuições. Veja o exemplo de utilização apresentado na \Cref{lst-contributions}.
A lista completa dos códigos e suas descrições é apresentada a seguir:
\begin{description}
\item[conceptualization] Conceituação;
\item[datacuration] Curadoria de dados;
\item[formalanalysis] Análise Formal;
\item[funding] Aquisição de financiamento;
\item[investigation] Investigação;
\item[methodology] Metodologia;
\item[projadm] Administração de projetos; 
\item[resources] Recursos;
\item[software] Programas;
\item[supervision] Supervisão;
\item[validation] Validação;
\item[visualization] Visualização;
\item[writing] Escrita - rascunho original;
\item[review] Escrita - revisão e edição.
\end{description}

\begin{lstlisting}[language=tex, label=lst-contributions, caption={Contribuição dos autores.}]
\begin{contributors}[sec-contributors]
\authorcontribution{Leonardo Araújo}[conceptualization,datacuration,formalanalysis,investigation,methodology,software,validation,visualization,writing,review]
\authorcontribution{Daniervelin Pereira}[methodology,projadm,resources,validation,writing,review]
\end{contributors}
\end{lstlisting} %stopzone



\section{Conclusão}\label{sec-conclusao}
Para elaborar um texto com melhor qualidade \cite{donaldknuth1984,leslielamport1994,araujo2020}, busque utilizar o modelo para \LaTeX{} 
disponível no sítio da revista. Leia as regras e orientações sobre a utilização de cada tipo de estrutura em um texto.


\lipsum[17-19]


\printbibliography\label{sec-bib}
% if the text is not in Portuguese, it might be necessary to use the code below instead to print the correct ABNT abbreviations [s.n.], [s.l.]
%\begin{portuguese}
%\printbibliography[title={Bibliography}]
%\end{portuguese}


%full list: conceptualization,datacuration,formalanalysis,funding,investigation,methodology,projadm,resources,software,supervision,validation,visualization,writing,review
\begin{contributors}[sec-contributors]
\authorcontribution{Leonardo Araújo}[conceptualization,datacuration,formalanalysis,investigation,methodology,software,validation,visualization,writing,review]
\authorcontribution{Daniervelin Pereira}[methodology,projadm,resources,validation,writing,review]
\end{contributors}


\appendix 
\section{Tabela longa}\label{apx-longtable}
Apresentamos aqui um exemplo de uma tabela longa. Para este tipo de tabela utilzie o ambiente \texttt{longtable}.

\setlength\LTleft{-1in}
\setlength\LTright{-1in}
\begin{small}
\renewcommand{\arraystretch}{1.5}
\begin{longtable}{
    >{\raggedright\arraybackslash}p{0.3\textwidth}
    p{0.4\textwidth}
    p{0.4\textwidth}
    }
\caption{Exemplo de tabela longa.}
\label{longtbl-01}
\\
\toprule
coluna 1 & coluna 2 & coluna 3 \\
\midrule
\lipsum[2] & \lipsum[3] & \lipsum[4] \\
\lipsum[5] & \lipsum[6] & \lipsum[7] \\
\lipsum[8] & \lipsum[9] & \lipsum[10] \\
\lipsum[11] & \lipsum[12] & \lipsum[13] \\
\bottomrule
\source{Texto de preenchimento gerado pelo pacote lipsum.}
\end{longtable}
\end{small}


\end{document}

